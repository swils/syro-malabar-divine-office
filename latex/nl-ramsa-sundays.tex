\documentclass[12pt,twoside,a5paper]{article}

\usepackage{multicol}

\usepackage[main=dutch]{babel}
\usepackage{divine-office}

% % % % % % % % % % % % % % % % % % % % % % % % % % % % % % % % % % % % % % % %

% Version: 2024-07-13
\begin{document}

\title{Ramsa~--- Zondagen, feesten en gedachtenissen}
\author{}
\date{}
\maketitle

% The following prevents footnotes and paracol from interacting in bad ways.
% Not really an idea why...
% See: https://stackoverflow.com/questions/61779911/paracol-and-footnote-placing-in-latex
\footnotelayout{\ }

% % % % % % % % % % % % % % % % % % % % % % % % % % % % % % % % % % % % % % % %

\begin{halfparskip}
  \cc~Eer aan God in de hoge \liturgicalhint{(3x)}. En op aarde vrede en goede hoop aan de mensen, altijd en in eeuwigheid.

  \rr~Amen. Zegen Heer. \liturgicalhint{(Vredekus)}

  \cc~Onze Vader die in de hemelen zijt,

  \rr~Geheiligd zij Uw Naam. Uw rijk kome, heilig, heilig, heilig zijt Gij. Onze Vader die in de hemelen zijt, de hemel en de aarde zijn gevuld met Uw onmetelijke glorie; de engelen en de mensen roepen U toe: heilig, heilig, heilig zijt Gij.~--- Onze Vader die in de hemelen zijt, geheiligd zij Uw Naam. Uw rijk kome, Uw wil geschiede op aarde zoals in de hemel. Geef ons heden het brood dat we nodig hebben en vergeef ons onze schulden en zonden zoals wij ook vergeven hebben aan onze schuldenaren. En leid ons niet in bekoring, maar verlos ons van de Kwade. Want van U is het koninkrijk en de kracht en de heerlijkheid in eeuwigheid, amen.

  \cc~Eer aan de Vader, de Zoon, en de Heilige Geest.

  \rr~Vanaf het begin en in alle eeuwigheid, amen en amen. Onze Vader die in de hemelen zijt, geheiligd zij Uw naam, Uw rijk kome, heilig, heilig, heilig zijt Gij. Onze Vader die in de hemelen zijt, de hemel en de aarde zijn gevuld met Uw onmetelijke glorie; de engelen en de mensen roepen U toe: heilig, heilig, heilig zijt Gij.

  \dd~Laat ons bidden, vrede zij met ons.

  \liturgicaloption{Zondagen en feesten:} \cc~Laat ons, mijn Heer, Uw Godheid met geestelijke lofzangen belijden, Uw Majesteit aanbidden met aardse aanbiddingen, en Uw geheime en verborgen Natuur verheerlijken met reine en heilige gedachten, Heer van alles, Vader, Zoon en Heilige Geest in alle eeuwigheid.~--- \rr~Amen.

  \liturgicaloption{Gedachtenissen:} \cc~We willen, Heer, Uw Godheid prijzen (herhaal) en Uw Majesteit aanbidden en aan Uw glorierijke Drievuldigheid onophoudelijk lof brengen voor altijd, Heer van alles, Vader...~--- \rr~Amen.
\end{halfparskip}

% % % % % % % % % % % % % % % % % % % % % % % % % % % % % % % % % % % % % % % %

\markedsection{Marmita}

\begin{halfparskip}
  \liturgicalhint{Begin de marmita van de dag als aangegeven in de Hudra of Gaza. Elke psalm heeft een eigen antifoon op feesten en gedachtenissen; de antifoon op zondagen en feries zijn 3 alleluias. Herhaal het eerste vers.}
\end{halfparskip}

\begin{halfparskip}
  \markedday{a) Zondagen en feesten van Advent tot Epifanie, \Pss{87--88}.}
\end{halfparskip}

\begin{halfparskip}
  \psalm{\Ps{87}} Zijn stichting op de heilige bergen bemint de Heer:~\sep\ de poorten van Sion~--- \liturgicalhint{alleluia, alleluia, alleluia}.

  Zijn stichting op de heilige bergen bemint de Heer:~\sep\ de poorten van Sion boven alle tenten van Jacob.

  Roemrijke dingen verhaalt men van u,~\sep\ O stad van God!

  Rahab en Babel zal Ik tot Mijn vereerders rekenen:~\sep\ zie, Filistea en Tyrus en het volk der Ethiopiërs: daar zijn ze geboren!

  Over Sion zal men zeggen: ``Allen, man voor man, zijn in haar geboren,~\sep\ en de Allerhoogste zelf heeft haar bevestigd.''

  De Heer zal schrijven in het boek der volken:~\sep\ ``Daar zijn ze geboren.''

  En in reidans zullen zij zingen:~\sep\ ``Al mijn bronnen zijn in u.''
\end{halfparskip}

\begin{halfparskip}
  \psalm{\Ps{88}} Heer, mijn God, ik roep overdag,~\sep\ en ik jammer 's nachts voor Uw aanschijn.

  Dringe mijn bede toch door tot U,~\sep\ neig Uw oor naar mijn klagen!

  Want mijn ziel is verzadigd met rampen,~\sep\ mijn leven is het dodenrijk nabij.

  Ik word gerekend onder hen, die ten grave dalen,~\sep\ ik ben als een man zonder kracht.

  Onder de doden is mijn legerstede,~\sep\ als van verslagenen, die liggen in het graf,

  Aan wie Gij niet meer denkt,~\sep\ die aan Uw zorgen zijn onttrokken.

  In een diepe groeve hebt Gij mij neergelegd,~\sep\ in duisternis, in een diep ravijn.

  Uw verontwaardiging drukt zwaar op mij,~\sep\ met al Uw golven slaat Gij mij neer.

  Gij hebt mijn vrienden van mij vervreemd, mij tot afschuw voor hen gemaakt;~\sep\ ik zit gevangen, en kan niet ontkomen.

  Van ellende verkwijnen mijn ogen; iedere dag roep ik tot U, O Heer,~\sep\ naar U strek ik mijn handen uit.

  Of doet Gij voor doden nog wonderen,~\sep\ of zullen gestorvenen, herrijzend, U loven?

  Of wordt Uw goedheid in het graf verkondigd,~\sep\ Uw trouw in het dodenrijk?

  Openbaart men in het duister Uw wonderen,~\sep\ in het land der vergetelheid Uw genade?

  Ik echter roep tot U, O Heer,~\sep\ mijn bede stijgt tot U op in de morgen.

  Waarom toch, O Heer, verstoot Gij mij,~\sep\ verbergt Gij voor mij Uw gelaat?

  Van jongs af ben ik ellendig en stervend,~\sep\ ik torste Uw verschrikkingen en kwijnde.

  Uw toorn is over mij heengegaan,~\sep\ Uw verschrikkingen sloegen mij neer.

  Zij omgeven mij immer als water,~\sep\ omringen mij alle tezamen.

  Vriend en makker hebt Gij van mij vervreemd,~\sep\ mijn vertrouweling is de duisternis.

  \liturgicalhint{Eer...}~--- \liturgicalhint{3x alleluia.}~--- Zijn stichting op de heilige bergen bemint de Heer:~\sep\ de poorten van Sion.
\end{halfparskip}

\begin{halfparskip}
  \markedday{b) Andere feesten en zondagen, \Pss{65--67}.}
\end{halfparskip}

\begin{halfparskip}
  \psalm{\Ps{65}} Aan U, O God, komt een lofzang toe in Sion;~\sep\ men volbrenge zijn gelofte aan U, die de bede verhoort.

  \liturgicalhint{3x alleluia.}~--- \liturgicalhint{Eerste vers.}

  Tot U komt alle vlees,~\sep\ omwille der ongerechtigheden.

  Onze misdaden drukken ons neer:~\sep\ Gij scheldt ze kwijt.

  Gelukkig die Gij uitkiest en tot U neemt:~\sep\ hij woont in Uw voorhoven.

  Dat wij verzadigd worden met de goederen van Uw huis,~\sep\ met de heiligheid van Uw tempel.

  Met gerechtigheid verhoort Gij ons door wondere tekenen,~\sep\ God, onze Redder,

  Gij zijt de Hoop van alle grenzen der aarde,~\sep\ en van de verre zeeën;

  Die de bergen vastlegt door Uw kracht,~\sep\ die met macht zijt omgord,

  Die het bulderen der zee bedwingt,~\sep\ het bulderen van haar golven en het woelen der naties.

  En die de grenzen der aarde bewonen, huiveren om Uw tekenen:~\sep\ met vreugde vervult Gij het uiterste oosten en westen.

  Gij hebt de aarde bezocht en haar besproeid,~\sep\ met rijkdommen haar overstelpt.

  De stroom van God is met water gevuld: Gij hebt hun graan bereid;~\sep\ zo hebt Gij haar gereed gemaakt:

  Haar voren hebt Gij besproeid,~\sep\ geëffend haar kluiten,

  Door stortregens hebt Gij haar geweekt,~\sep\ en haar gewas gezegend.

  Met Uw mildheid hebt Gij het jaar gekroond,~\sep\ en Uw wegen druipen van vet.

  De weiden der woestijn druipen ervan,~\sep\ en de heuvelen omgorden zich met jubel.

  De weiden zijn met kudden bekleed en de dalen met koren bedekt:~\sep\ zij juichen U toe en zingen!
\end{halfparskip}

\begin{halfparskip}
  \psalm{\Ps{66}} Juicht God toe, alle landen, bezingt de glorie van Zijn Naam,~\sep\ heft voor Hem een heerlijk loflied aan.

  Zegt tot God: Hoe ontzagwekkend zijn Uw werken!~\sep\ Om Uw geweldige kracht brengen Uw vijanden U vleiend hulde.

  Dat heel de aarde U aanbidde en voor U zinge,~\sep\ dat zij bezinge Uw Naam.

  Komt en ziet de werken van God:~\sep\ wondere daden volbracht Hij onder de kinderen der mensen!

  De zee veranderde Hij in land, en zij trokken te voet door de stroom;~\sep\ laten wij daarom over Hem ons verheugen.

  Eeuwig heerst Hij door Zijn macht; Zijn ogen slaan de volken gade:~\sep\ dat de weerspannigen zich niet verheffen.

  Zegent, gij volken, onze God,~\sep\ en verkondigt Zijn wijd verbreide lof.

  Hij behield ons in leven,~\sep\ en liet onze voeten niet wankelen.

  Want Gij hebt ons beproefd, O God,~\sep\ met vuur ons gelouterd, zoals men zilver loutert.

  Gij liet ons de strik inlopen,~\sep\ een zware last hebt Gij op onze heupen gelegd.

  Mensen liet Gij over onze hoofden treden; wij zijn door vuur en water gegaan,~\sep\ maar Gij hebt ons uitkomst gebracht.

  Met brandoffers wil ik Uw huis betreden,~\sep\ U mijn geloften inlossen,

  Die mijn lippen hebben uitgesproken,~\sep\ en mijn mond heeft beloofd in mijn kwelling.

  Brandoffers wil ik U brengen van vette schapen met het vet van rammen,~\sep\ runderen en bokken zal ik offeren.

  Komt en hoort, gij allen, die God vreest,~\sep\ ik wil U verhalen hoe grote dingen Hij aan mij gedaan heeft!

  Ik riep Hem aan met mijn mond,~\sep\ en prees Hem met mijn tong.

  Had ik in mijn hart op boosheid gezonnen,~\sep\ dan had de Heer mij niet verhoord.

  Maar God heeft mij verhoord,~\sep\ heeft gelet op de stem van mijn smeken.

  Gezegend zij God, die mijn bede niet heeft versmaad,~\sep\ mij Zijn ontferming niet heeft onthouden.
\end{halfparskip}

\begin{halfparskip}
  \psalm{\Ps{67}} God zij ons genadig en zegene ons;~\sep\ Hij tone ons Zijn vredig gelaat.

  Opdat men op aarde Zijn weg lere kennen,~\sep\ onder alle volken Zijn heil.

  Dat de volken U prijzen, O God,~\sep\ dat alle volken U prijzen!

  Laat juichen en jubelen de naties, omdat Gij met rechtvaardigheid de volken regeert,~\sep\ en de naties op aarde bestuurt.

  Dat de volken U prijzen, O God,~\sep\ dat alle volken U prijzen!

  De aarde heeft haar vrucht gegeven;~\sep\ God, onze God, heeft ons gezegend.

  Dat God ons zegene,~\sep\ en dat alle grenzen der aarde Hem vrezen!

  \liturgicalhint{Eer...}~--- \liturgicalhint{3x alleluia.}~--- Aan U, O God, komt een lofzang toe in Sion;~\sep\ men volbrenge zijn gelofte aan U, die de bede verhoort.
\end{halfparskip}

\begin{halfparskip}
  \markedday{c) Gedachtenissen op Vrijdagen, \Pss{85--86}.}
\end{halfparskip}

\begin{halfparskip}
  \psalm{\Ps{85}} Gij zijt Uw land genadig geweest, O Heer,~\sep\ hebt het lot van Jacob ten goede gekeerd.

  \liturgicalhint{3x alleluia.}~--- Gij zijt Uw land genadig geweest, O Heer,~\sep\ hebt het lot van Jacob ten goede gekeerd.

  Vergeven hebt Gij de schuld van Uw volk,~\sep\ en al zijn zonden bedekt.

  Uw gramschap hebt Gij geheel bedwongen,~\sep\ de gloed van Uw toorn gestild.

  Herstel ons, o God, onze Redder,~\sep\ en leg Uw wrevel tegen ons af.

  Zult Gij dan eeuwig tegen ons toornen,~\sep\ of verbolgen blijven van geslacht tot geslacht?

  Zult Gij ons dan niet opnieuw doen leven,~\sep\ opdat Uw volk zich verblijde in U?

  Toon ons, Heer, Uw barmhartigheid,~\sep\ en schenk ons Uw heil!

  Ik wil horen naar wat de Heer God spreekt:~\sep\ vrede voorzeker kondigt Hij aan.

  Voor Zijn volk en Zijn heiligen,~\sep\ en voor hen, die zich van harte keren tot Hem.

  Ja waarlijk, Zijn heil is nabij voor wie Hem vrezen,~\sep\ en zo zal er glorie wonen in ons land:

  Barmhartigheid en trouw zullen elkander ontmoeten,~\sep\ gerechtigheid en vrede elkander de kus geven.

  Trouw zal aan de aarde ontspruiten,~\sep\ en gerechtigheid neerzien vanuit de hemel.

  De Heer zelf zal zegen schenken,~\sep\ en ons land zijn vruchten geven.

  Gerechtigheid zal vóór Hem uitgaan,~\sep\ en heil zijn schreden volgen.
\end{halfparskip}

\begin{halfparskip}
  \psalm{\Ps{86}} Neig Uw oor, O Heer; verhoor mij,~\sep\ want ik ben ellendig en arm.

  Bescherm mij, want ik ben U toegewijd;~\sep\ red Uw dienaar, die op U hoopt.

  Mijn God zijt Gij; wees mij genadig, O Heer,~\sep\ want almaar door roep ik tot U.

  Verblijd de ziel van Uw dienaar,~\sep\ want tot U, O Heer, verhef ik mijn ziel.

  Want Gij, O Heer, zijt goed en genadig,~\sep\ vol erbarming voor al wie U aanroept.

  Luister, Heer, naar mijn bede,~\sep\ en geef acht op de stem van mijn smeken.

  Op de dag van mijn kwelling riep ik tot U,~\sep\ omdat Gij mij verhoren zult.

  Onder de goden, O Heer, is er geen als Gij,~\sep\ en geen werk is gelijk aan het Uwe.

  Alle volken, door U geschapen, zullen komen, en U aanbidden, O Heer,~\sep\ en verheerlijken Uw Naam.

  Want Gij zijt groot en Gij doet wonderwerken:~\sep\ Gij zijt God, en Gij alleen.

  Toon mij Uw weg, O Heer, opdat ik wandele in Uw waarheid,~\sep\ richt mijn hart op de vrees voor Uw Naam.

  Ik zal U prijzen, Heer, mijn God, uit heel mijn hart:~\sep\ en eeuwig Uw Naam verheerlijken.

  Want Uw erbarming voor mij was groot,~\sep\ en uit de diepten van het dodenrijk hebt Gij mij opgehaald.

  Trotsen, O God, zijn tegen mij opgestaan, een bende geweldenaars staat mij naar het leven,~\sep\ zij houden U niet voor ogen.

  Maar Gij, O Heer, zijt een barmhartige en liefdevolle God,~\sep\ lankmoedig, rijk aan ontferming en trouw.

  Blik op mij neer en wees mij genadig;~\sep\ schenk aan Uw dienaar Uw kracht, en red de zoon van Uw dienstmaagd.

  Geef mij een teken van Uw gunst, opdat die mij haten, Heer, vol schaamte zien,~\sep\ dat Gij, O Heer, mij hulp en troost hebt geschonken.

  \liturgicalhint{Eer...}~--- \liturgicalhint{3x alleluia.}~--- Gij zijt Uw land genadig geweest, O Heer,~\sep\ hebt het lot van Jacob ten goede gekeerd.
\end{halfparskip}

\begin{halfparskip}
  \markedday{d) Gedachtenissen op andere dagen, \Pss{15--17}}
\end{halfparskip}

\begin{halfparskip}
  \liturgicalhint{Zie Psalterium.}

  \liturgicalhint{Eer...}~--- \liturgicalhint{Alleluia, alleluia, alleluia.}~--- \liturgicalhint{Eerste vers.}
\end{halfparskip}

\begin{halfparskip}
  \dd~Laat ons bidden; vrede zij met ons.

  \liturgicalOption{Zondagen en feesten.} \cc~Wij moeten altijd de grote, vreeswekkende, heilige, gezegende, uitmuntende en onbegrijpelijke Naam van Uw roemrijke Drieëenheid en Uw goedheid jegens ons geslacht belijden, aanbidden
  en verheerlijken, Heer van alles, Vader, Zoon en H.~Geest in eeuwigheid.~--- \rr~Amen.

  \liturgicalOption{Gedachtenissen.} \cc~U, goede, vriendelijke, medelevende, vol van barmhartigheden, grote Koning van glorie, Wezen dat van eeuwigheid is, belijden, aanbidden en verheerlijken wij te allen tijde, Heer van alles...
\end{halfparskip}

% % % % % % % % % % % % % % % % % % % % % % % % % % % % % % % % % % % % % % % %

\markedsection{Wierookhymne (Aik etra)}

\vspace{0.5em}

\begin{halfparskip}
  \liturgicalOption{1) Zon- en feestdagen.}

  1.~\dd~Hoe geliefd zijn Uw woningen, U Heer der heerscharen\footnote{\liturgicalhint{Wierookzegen:} Wij willen te allen tijde Uw glorieuze Drie-eenheid loven, amen. \liturgicalhint{Of:} Christus, die het bloed der martelaren aanvaardde op de dag dat ze werden gedood, aanvaard in de goedheid van Uw mededogen deze wierook voor altijd uit mijn zwakke handen, amen.}.
\end{halfparskip}

\CLEARPAGEAV

\liturgicalhint{Tune: Bahar Lemba~/ Misiha Kartave}

\begin{doublecols}
  \englishl \rr~Our Lord Jesus Christ, mankind's Saviour~/ accept in mercy our supplications~/ like sweet smelling frankincense~/ which we offer You.

  \dutchc{1} \rr~Als de geur van zoete wierook en de geur van een aangenaam wierookvat, ontvang, Christus onze Verlosser, het verzoek en gebed van Uw dienaren.
\end{doublecols}

\begin{halfparskip}
  2.~\dd~Mijn ziel verlangt en smacht naar de voorhoven van de Heer.~--- \liturgicalhint{Our Lord...}~/ \liturgicalhint{Als de geur...}
\end{halfparskip}

\vspace{0.5em}

\begin{halfparskip}
  \liturgicalOption{2) Gedachtenissen.}

  1.~\dd~Ik zal altijd de Heer zegenen.

  \liturgicalhint{Our Lord Jesus...} \liturgicaloption{of:} \liturgicalhint{Als de geur...}

  2.~\dd~En Zijn lof zal altijd in mijn mond zijn.

  \liturgicalhint{Our Lord Jesus...} \liturgicaloption{of:} \liturgicalhint{Als de geur...}
\end{halfparskip}

\vspace{0.5em}

\begin{halfparskip}
  \liturgicalOption{3) Feesten van de Heer.}

  3.~\dd~Mijn hart en mijn lichaam juichen voor de levende God.

  4.~\dd~Mijn Koning en mijn God, gezegend zijn zij die in Uw huis verblijven.
\end{halfparskip}

\vspace{0.5em}

\begin{halfparskip}
  \liturgicalOption{*~Op alle dagen:}

  \liturgicalhint{Eer...}~--- Vanaf het begin en in alle eeuwigheid, amen en amen.~--- \liturgicalhint{Our Lord...}~/ \liturgicalhint{Als de geur...}
\end{halfparskip}

\vspace{0.5em}

\begin{halfparskip}
  \dd~Vrede zij met ons.

  \cc~Voor al Uw hulp en genaden aan ons, die niet terugbetaald kunnen worden, zullen wij U onophoudelijk belijden en verheerlijken in Uw gekroonde Kerk, die vol is van alle hulp en alle zegeningen, want U bent de Heer en Schepper van alles, Vader...~--- \rr~Amen.
\end{halfparskip}

% % % % % % % % % % % % % % % % % % % % % % % % % % % % % % % % % % % % % % % %

\CLEARPAGEAV

\markedsection{Laku Mara}

\vspace{0.5em}

\vspace{0.4em}
\begin{doublecols}
  \englishl 1. \rr~You, Lord of all, we worship You~/ Jesus Christ, we exalt You~/ You give life to our bodies~/ and salvation to our souls.

  \dutchc{1} 1. U, Heer van alles, prijzen wij; U, Jezus Christus loven wij; U bent de Levendmaker van onze lichamen; U bent de Verlosser van onze zielen.
\end{doublecols}

\begin{halfparskip}
  2. Ik was blij toen men me zei, Wij gaan op naar het huis van de Heer.~--- \liturgicalhint{You, Lord...}~/ \liturgicalhint{U, Heer van alles...}

  \liturgicallbracket\liturgicaloption{In een privaat huis (Assyrische Hudra):} In elke plaats zijt Gij, God, aanvaard onze bede.\liturgicalrbracket

  3. \liturgicalhint{Eer...}~--- Vanaf het begin en in alle eeuwigheid, amen en amen.~--- \liturgicalhint{You, Lord...}~/ \liturgicalhint{U, Heer van alles...}

  \liturgicalOption{Op feesten van de Heer.} 2. Ik was blij toen men me zei.~--- \liturgicalhint{You, Lord...}~/ \liturgicalhint{U, Heer van alles...}

  3. Wij gaan op naar het huis van de Heer.~--- \liturgicalhint{You, Lord...}~/ \liturgicalhint{U, Heer van alles...}

  4. Eer aan de Vader, de Zoon en de Heilige Geest.~--- \liturgicalhint{You, Lord...}~/ \liturgicalhint{U, Heer van alles...}

  5. Vanaf het begin en in alle eeuwigheid, amen en amen.~--- \liturgicalhint{You, Lord...}~/ \liturgicalhint{U, Heer van alles...}

  \dd~Laat ons bidden; vrede zij met ons.

  \cc~U bent waarlijk de Levendmaker van onze lichamen, de goede Verlosser van onze zielen, en de trouwe Bewaker van onze levens. U moeten wij altijd loven, aanbidden en verheerlijken, Heer van alles in alle eeuwigheid.~--- \rr~Amen.
\end{halfparskip}

% % % % % % % % % % % % % % % % % % % % % % % % % % % % % % % % % % % % % % % %

\markedsection{Suraya Da'Qdam}

\liturgicalhint{(Op zondagen, niet op feesten en gedachtenissen. Met 3x alleluia in het begin en op het einde.)}


\markedsubsectionrubricwithhint{Eerste Zondag van elk seizoen}{\Ps{47,1--5}. Van Hemelvaart tot Aankondiging voeg er vv. 5-9 aan toe.}

\begin{halfparskip}
  Volken, gij alle, klapt in de handen,~\sep\ juicht God toe met jubelzang!

  \liturgicalhint{3x Alleluia.}~--- \liturgicalhint{Herhaal het eerste vers.}

  Want hoogverheven, ontzagwekkend is de Heer,~\sep\ de grote Koning van heel de aarde.

  Hij onderwerpt ons de volken,~\sep\ en legt de naties onder onze voeten.

  Ons erfdeel kiest Hij voor ons uit,~\sep\ de roem van Jacob, die Hij liefheeft.

  Hemelvaart tot Aankondiging: God stijgt op onder gejubel,~\sep\ de Heer onder bazuingeschal.

  Zingt voor God, zingt Hem toe,~\sep\ zingt voor onze Koning, zingt Hem toe!

  Want God is Koning over heel de aarde:~\sep\ zingt een lofzang.

  God heerst over de volkeren,~\sep\ God zetelt op Zijn heilige troon.
\end{halfparskip}

\markedsubsectionrubricwithhint{Tweede Zondag}{\Ps{65,1--5}.}

\begin{halfparskip}
  Aan U, o God, komt een lofzang toe in Sion;~\sep\ men volbrenge zijn gelofte aan U, die de bede verhoort.

  \liturgicalhint{3x Alleluia.}~--- \liturgicalhint{Herhaal het eerste vers.}

  Tot U komt alle vlees,~\sep\ omwille der ongerechtigheden.

  Onze misdaden drukken ons neer:~\sep\ Gij scheldt ze kwijt.

  Gelukkig die Gij uitkiest en tot U neemt:~\sep\ hij woont in Uw voorhoven.

  Dat wij verzadigd worden met de goederen van Uw huis,~\sep\ met de heiligheid van Uw tempel.

  Met gerechtigheid verhoort Gij ons door wondere tekenen,~\sep\ God, onze Redder,

  Gij zijt de Hoop van alle grenzen der aarde,~\sep\ en van de verre zeeën;
\end{halfparskip}

\markedsubsectionrubricwithhint{Derde Zondag}{\Ps{89,1--5}.}

\begin{halfparskip}
  De gunsten van de Heer wil ik eeuwig bezingen,~\sep\ door alle geslachten heen zal mijn mond Uw trouw verkondigen,

  \liturgicalhint{3x Alleluia.}~--- \liturgicalhint{Herhaal het eerste vers.}

  Want Gij hebt gezegd; ``De genade staat eeuwig vast'';~\sep\ in de hemel hebt Gij Uw trouw gegrondvest.

  ``Een verbond ging Ik aan met Mijn uitverkorene;~\sep\ aan David, Mijn dienaar, zwoer Ik een eed:

  Ik zal uw nazaat voor eeuwig bevestigen,~\sep\ en uw troon in stand houden door alle geslachten.''
\end{halfparskip}

\markedsubsectionrubricwithhint{Vierde Zondag}{\Ps{93}.}

\begin{halfparskip}
  De Heer is Koning, met majesteit bekleed;~\sep\ bekleed is de Heer met macht, Hij heeft Zich omgord;

  \liturgicalhint{3x Alleluia.}~--- \liturgicalhint{Herhaal het eerste vers.}

  Hij heeft het aardrijk bevestigd,~\sep\ dat niet zal wankelen.

  Hecht staat Uw troon van ouds;~\sep\ Gij zijt van eeuwigheid.

  De stromen verheffen, o Heer, de stromen verheffen hun stem,~\sep\ de stromen verheffen hun bruisen.

  Maar boven de stem der wijde wateren, boven de branding der zee~\sep\ is machtig de Heer in de hoge.

  Betrouwbaar bovenmate zijn Uw getuigenissen;~\sep\ Uw huis, Heer, past heiligheid in lengte van dagen.
\end{halfparskip}

\markedsubsectionrubricwithhint{Vijfde Zondag}{\Ps{125,1--7}.}

\begin{halfparskip}
  Toen de Heer de gevangenen van Sion deed wederkeren,~\sep\ was het of wij droomden.

  \liturgicalhint{3x Alleluia.}~--- \liturgicalhint{Herhaal het eerste vers.}

  Toen werd onze mond met lachen gevuld,~\sep\ en onze tong met gejubel.

  Toen zei men onder de volken:~\sep\ ``De Heer heeft grote dingen aan hen gedaan.''

  Ja, grote dingen heeft de Heer aan ons gedaan;~\sep\ wij zijn nu van vreugde vervuld.

  Wend ons lot ten beste, o Heer,~\sep\ als de bergstromen in het Zuiderland.

  Wie met tranen zaaien,~\sep\ zullen met gejubel maaien.
\end{halfparskip}

\markedsubsectionrubricwithhint{Zesde Zondag}{\Ps{49,1--5}.}

\begin{halfparskip}
  Aanhoort het, alle volkeren;~\sep\ luistert, alle bewoners der aarde:

  \liturgicalhint{3x Alleluia.}~--- \liturgicalhint{Herhaal het eerste vers.}

  Zowel geringen als edelen,~\sep\ rijken als armen op eenzelfde wijze.

  Mijn mond gaat wijsheid verkondigen,~\sep\ de overweging van mijn hart brengt inzicht.

  Mijn oor wil ik neigen naar een leer van wijsheid,~\sep\ bij het spel van de citer mijn raadsel onthullen.
\end{halfparskip}

\markedsubsectionrubricwithhint{Zevende Zondag}{\Ps{136,1--4}.}

\begin{halfparskip}
  Aan de stromen van Babylon, daar zaten wij en weenden,~\sep\ als wij aan Sion dachten.

  \liturgicalhint{3x Alleluia.}~--- \liturgicalhint{Herhaal het eerste vers.}

  Aan de wilgen van dat land,~\sep\ hingen wij onze citers op.

  Want die ons hadden weggevoerd, vroegen ons daarginds om liederen, en die ons verdrukten, om een jubelzang:~\sep\ ``Zingt ons uit Sions liederen!''
\end{halfparskip}

\liturgicalOption{*~Alle zondagen.} \liturgicalhint{Eer.~--- 3x alleluia}

% % % % % % % % % % % % % % % % % % % % % % % % % % % % % % % % % % % % % % % %

\markedsection{Onita D'Qdam \markedsectionhint{(Alle Zondagen.)}}

\begin{halfparskip}
  \dd~Verblijd de ziel van Uw dienaar.

  \rr~Met al Uw heiligen, laat, Christus de Koning, de geest van Uw dienaren in vrede rusten, waar lijden niet regeert, noch leed, noch verdriet, maar het beloofde leven zonder einde.

  \dd~Want hoogverheven, ontzagwekkend is de Heer.

  \rr~Ons vertrouwen is in God, de Maker van onze vader Adam, de Hoop van onze dood en ons leven. De wereld is niets, en niets zijn haar genoegens; maar Hij doet ons verrijzen en geeft ons leven in Zijn goedheid.

  \cc~Eer aan de Vader...~--- \rr~Christus de Koning, onze Verlosser, doe ons opstaan op de dag van Uw komst, en laat ons staan aan Uw rechterhand, met de rechtvaardigen die U hebben behaagd, en in Uw Kruis geloofden en Het beleden, dat we met hen het eeuwige leven mogen beërven.

  \liturgicalhint{*} \cc~Wij moeten altijd Uw barmhartigheid en de zorg van Uw goede wil jegens ons, onze Heer en onze God, erkennen, aanbidden en eren, Heer van alles, Vader, Zoon en Heilige Geest in eeuwigheid.~--- \rr~Amen.
\end{halfparskip}

% % % % % % % % % % % % % % % % % % % % % % % % % % % % % % % % % % % % % % % %

\markedsection{Maria Qretak}

\liturgicalhint{(Ps 140--141;118:105--112;116 onder één Eer aan de Vader.)}

\begin{halfparskip}
  \psalm{\Ps{140}}\footnote{\liturgicalhint{Op feesten van de Heer na elk vers van Ps. 140:} Eer aan U; glorious is Uw geboorte~/ verschijning~/ intrede~/ verrijzenis~/ hemelvaart~/ nederdaling~/ openbaring (\emph{Transfiguratie})~/ Kruis.} Ik roep tot U, Heer; snel mij te hulp.~--- \qanona{Luister naar mijn smeken, wanneer ik tot U roep.}

  Ik roep tot U, Heer; snel mij te hulp;~\sep\ luister naar mijn smeken, wanneer ik tot U roep.

  Laat mijn bede als een reukoffer opgaan tot U,~\sep\ het heffen van mijn handen als een avondoffer zijn.

  Heer, zet een wacht voor mijn mond,~\sep\ een post voor de deur van mijn lippen.

  Neig mijn hart niet tot kwaad,~\sep\ om boze daden te stellen;

  En geef dat ik nooit met boosdoeners,~\sep\ hun uitgezochte spijzen eet.

  Laat de rechtvaardige mij slaan: dat is liefde;~\sep\ mij berispen: dat is olie op mijn hoofd,

  Die mijn hoofd niet zal weigeren;~\sep\ maar immer zal ik bidden onder hun kastijding.

  Hun vorsten werden neergelaten langs de rots,~\sep\ en zij hoorden hoe zachtzinnig mijn woorden waren.

  Zoals wanneer men de grond doorploegt en scheurt,~\sep\ zo liggen hun beenderen verstrooid bij de poort van het dodenrijk.

  Maar op U, Heer God, zijn mijn ogen gericht,~\sep\ naar U vlucht ik heen: laat mij niet vergaan.

  Behoed mij voor het net, dat ze mij spanden;~\sep\ en voor de valstrikken van hen, die het kwade bedrijven.

  Laat de bozen tezamen in hun eigen netten vallen,~\sep\ terwijl ik behouden ontkom.
\end{halfparskip}

\begin{halfparskip}
  \psalm{\Ps{141}}\footnote{\liturgicalhint{Feesten: Voor Ps 141:} Glorious bent U, en glorious is Uw \emph{N}...} Luid roep ik tot de Heer,~\sep\ luid smeek ik de Heer.

  Voor Hem stort ik mijn zorgen uit,~\sep\ voor Hem leg ik mijn kommer bloot.

  Als mijn geest in mij is beangst,~\sep\ kent Gij mijn weg.

  Op het pad waarlangs ik ga,~\sep\ heeft men mij heimelijk een strik gelegd.

  Ik wend mij naar rechts en zie uit,~\sep\ maar niet één die om mij zich bekommert,

  Er is voor mij geen uitweg meer,~\sep\ en niemand draagt zorg voor mijn leven.

  Ik roep tot U, o Heer, ik zeg: Gij zijt mijn toevlucht,~\sep\ mijn aandeel in het land der levenden.

  Geef acht op mijn geroep,~\sep\ want diep ellendig ben ik geworden.

  Ontruk mij aan die mij vervolgen,~\sep\ want sterker zijn ze dan ik.

  Leid mij uit de kerker,~\sep\ opdat ik Uw Naam moge danken.

  De rechtvaardigen zullen mij omringen,~\sep\ wanneer Gij mij hebt welgedaan.
\end{halfparskip}

\begin{halfparskip}
  \psalm{\Ps{108:105vv}}\footnote{\liturgicalhint{Feesten: Voor Ps 119:} Glorious is Uw \emph{N}... die ons allen verblijdt.} Uw woord is een lamp voor mijn voeten,~\sep\ en een licht op mijn pad.

  Ik zweer en neem mij voor,~\sep\ Uw rechtvaardige besluiten na te leven.

  Ik ben in de diepste ellende, o Heer;~\sep\ spaar mijn leven naar Uw woord.

  Aanvaard, o Heer, de offers van mijn mond,~\sep\ en leer mij Uw besluiten.

  In voortdurend gevaar is mijn leven,~\sep\ maar Uw Wet vergeet ik niet.

  De bozen hebben mij een strik gelegd,~\sep\ maar van Uw bevelen week ik niet af.

  Uw voorschriften zijn mijn erfdeel voor eeuwig,~\sep\ want ze zijn de vreugde van mijn hart.

  Ik heb er mijn hart op gezet Uw verordeningen na te komen,~\sep\ voortdurend en stipt.
\end{halfparskip}

\begin{halfparskip}
  \psalm{\Ps{116}}\footnote{\liturgicalhint{Feesten: Voor Ps 116:} Laat het Volk en de volken U verheerlijken.} Looft de Heer, alle volkeren,~\sep\ alle naties, verheerlijkt Hem,

  Want zijn erbarming blijft ons verzekerd,~\sep\ en de trouw van de Heer duurt eeuwig.

  Eer...~--- Ik roep tot U, o Heer; snel mij te hulp;~\sep\ luister naar mijn smeken, wanneer ik tot U roep.
\end{halfparskip}

\begin{halfparskip}
  \dd~Laat ons bidden; vrede zij met ons.

  \cc~Hoor, onze Heer en onze God, het gebed van Uw dienaars in uw mededogen, ontvang de petitie van Uw aanbidders in Uw barmhartigheid, en heb medelijden met onze zondigheid in Uw goedheid en barmhartigheid, O Geneesheer van ons lichaam en goede Hoop van onze ziel, Heer van alles, Vader...~--- \rr~Amen.
\end{halfparskip}

% % % % % % % % % % % % % % % % % % % % % % % % % % % % % % % % % % % % % % % %

\markedsection{Suraya D'Batar}

\liturgicalhint{(Op zondagen, niet op feesten en gedachtenissen; met 3x alleluia in het begin en op het einde.)}

\markedsubsectionrubricwithhint{Eerste Zondag}{\Ps{48,1--3}.}

\begin{halfparskip}
  Groot is de Heer en hoogst lofwaardig,~\sep\ in de stad van onze God.

  \liturgicalhint{Alleluia.}~--- \liturgicalhint{Herhaal het eerste vers.}

  Zijn heilige, zijn roemvolle heuvel,~\sep\ is de vreugde van heel het aardrijk.

  De berg Sion, het uiterste noorden,~\sep\ is de stad van de grote Koning.

  God in haar burchten,~\sep\ toonde zich een veilige schutse.
\end{halfparskip}

\markedsubsectionrubricwithhint{Tweede Zondag}{\Ps{66,1--4}.}

\begin{halfparskip}
  Juicht God toe, alle landen, bezingt de glorie van Zijn Naam,~\sep\ heft voor Hem een heerlijk loflied aan.

  \liturgicalhint{Alleluia.}~--- \liturgicalhint{Herhaal het eerste vers.}

  Zegt tot God: hoe ontzagwekkend zijn Uw werken!~\sep\ Om Uw geweldige kracht brengen Uw vijanden U vleiend hulde.

  Dat heel de aarde U aanbidde en voor U zinge,~\sep\ dat zij bezinge Uw Naam.
\end{halfparskip}

\markedsubsectionrubricwithhint{Derde Zondag}{\Ps{89,5--9}.}

\begin{halfparskip}
  De hemelen loven Uw wonderen, o Heer,~\sep\ en Uw trouw in de kring der heiligen.

  \liturgicalhint{Alleluia.}~--- \liturgicalhint{Herhaal het eerste vers.}

  Want wie in de wolken zal de Heer evenaren,~\sep\ wie onder Gods zonen is gelijk aan de Heer?

  Ontzagwekkend is God in de gemeenschap der heiligen,~\sep\ groot en vreeswekkend boven allen om Hem heen.

  O Heer, God der legerscharen,~\sep\ wie is U gelijk?
\end{halfparskip}

\markedsubsectionrubricwithhint{Vierde Zondag}{\Ps{148,1--7}.}

\begin{halfparskip}
  Looft de Heer in de hemel,~\sep\ looft Hem in de hoge.

  \liturgicalhint{Alleluia.}~--- \liturgicalhint{Herhaal het eerste vers.}

  Looft Hem al Zijn engelen,~\sep\ looft Hem, al Zijn legerscharen!

  Looft Hem zon en maan,~\sep\ looft Hem alle fonkelende sterren!

  Looft Hem hemelen der hemelen,~\sep\ en gij wateren boven de hemel!

  Dat zij de Naam van de Heer loven,~\sep\ want Hij gebood, en ze waren geschapen;

  En Hij heeft ze gevestigd voor immer en eeuwig:~\sep\ Hij gaf een wet, die niet zal vergaan.~\sep\  Looft de Heer op aarde.
\end{halfparskip}

\markedsubsectionrubricwithhint{Vijfde Zondag}{\Ps{126,1--5}.}

\begin{halfparskip}
  Als de Heer het huis niet bouwt,~\sep\ werken vergeefs die er aan bouwen.

  \liturgicalhint{Alleluia.}~--- \liturgicalhint{Herhaal het eerste vers.}

  Als de Heer de stad niet behoedt,~\sep\ waakt de wachter tevergeefs.

  Het heeft voor u geen zin vóór het daglicht op te staan,~\sep\ en op te blijven tot diep in de nacht,

  Voor u, die het brood van harde arbeid eet,~\sep\ want Zijn geliefden schenkt Hij gaven in hun slaap.

  Waarlijk, kinderen zijn een gave van de Heer.
\end{halfparskip}

\markedsubsectionrubricwithhint{Zesde Zondag}{\Ps{129}.}

\begin{halfparskip}
  Uit de diepten roep ik tot U, o Heer,~\sep\ Heer, luister naar mijn klagen!

  \liturgicalhint{Alleluia.}~--- \liturgicalhint{Herhaal het eerste vers.}

  Laat Uw oren zich neigen,~\sep\ naar de stem van mijn smeken.

  Als Gij de zonden blijft gedenken, Heer,~\sep\ Heer, wie zal dan staande blijven?

  Maar bij U is vergeving van zonden,~\sep\ opdat men vol eerbied U diene.

  Ik stel mijn hoop op de Heer,~\sep\ mijn ziel hoopt op Zijn woord;

  Verlangend zie ik uit naar de Heer,~\sep\ meer dan wachters naar de dageraad.

  Meer dan wachters naar de dageraad,~\sep\ ziet Israël verlangend uit naar de Heer.

  Want bij de Heer is barmhartigheid,~\sep\ en bij Hem overvloedige verlossing.

  Hij zal Israël verlossen,~\sep\ van al Zijn ongerechtigheden.
\end{halfparskip}

\markedsubsectionrubricwithhint{Zevende Zondag}{\Ps{137,1--4}.}

\begin{halfparskip}
  Ik wil U prijzen, Heer, uit heel mijn hart,~\sep\ daar Gij hebt geluisterd naar de woorden van mijn mond.

  \liturgicalhint{Alleluia.}~--- \liturgicalhint{Herhaal het eerste vers.}

  Ik wil U bezingen voor het aanschijn der engelen,~\sep\ mij neerwerpen, naar Uw heilige tempel gericht,

  En Uw Naam zal ik prijzen,~\sep\ om Uw goedheid en trouw,

  Daar Gij boven alles verheerlijkt hebt,~\sep\ Uw Naam en Uw belofte.

  Wanneer ik U aanriep, hebt Gij mij verhoord,~\sep\ en mijn zielskracht vermeerderd.

  Alle koningen der aarde zullen U prijzen, o Heer.
\end{halfparskip}

\liturgicalOption{*~Alle zondagen.} \liturgicalhint{Eer.~--- 3x alleluia}

% % % % % % % % % % % % % % % % % % % % % % % % % % % % % % % % % % % % % % % %

\CLEARPAGEAV

\markedsection{Onita D'Batar}

\vspace{0.5em}

\begin{halfparskip}
  \liturgicalOption{Weken ``voor'':} Neig Uw oor, O Heer; verhoor mij.~--- Christus de Zoon, die voor onze verlossing kwam om het verdorven beeld van Adam te vernieuwen, en ons lichaam aannam en daarin ons ras redde, en de troost schonk van de opstanding der doden, vergeef in Uw goedheid Uw dienaren op de dag van Uw komst.

  In Uw goedheid, God.~--- In Uw goedheid heeft U ons ras geschapen; en bekleedde het met uitmuntende glorie in het paradijs. En omdat het [ons ras] aanmatigend was, zondigde en uit zijn glorie viel, hebt U Uw geliefde Zoon naar ons gezonden, en in Zijn mededogen heeft Hij ons het beloofde leven gegeven dat geen einde kent.

  Eer...~--- Glorie aan U, Jezus, onze zegevierende Koning, die door Uw Kruis ons ras van de zonden heeft gered. Moge Uw grote kracht ons lichaam vernieuwen; moge er een einde komen aan de dood en mag de opstanding heersen, en mogen wij door Uw wil waardig gemaakt worden om genade te ontvangen, O Koning en Levendmaker.
\end{halfparskip}

\begin{halfparskip}
  \liturgicalOption{Weken ``na'':} Hij zal met rechtvaardigheid de wereld regeren.~--- \rr~De Koning, de Levendmaker, verschijnt in heerlijkheid vanuit de hoge, en geeft leven aan de doden, en wekt zij die werden begraven. En de doden staan samen op uit de graven, en loven Hem die leven geeft aan de doden.

  \dd~Gij allen, die God vreest.~--- \rr~Verwijdert het verdriet uit uw hart, stervelingen, want de dag van onze Heer komt en geeft ons vreugde, en wekt ons uit de slaap, wanneer de wachters (= engelen) Hem lof zingen. En de engelen verheugen zich op de dag van de verrijzenis.

  \cc~Eer...~--- \rr~Maak onze doden levend, Heer, en laat hun lichamen verrijzen, zoals de profeet, de zoon van Amos [Isaïas] voorzegd heeft: ``Zij die slapen zullen ontwaken, en zij die in het stof liggen zullen glorie geven'', want Uw dauw is een dauw van licht en van waarheid.
\end{halfparskip}

% % % % % % % % % % % % % % % % % % % % % % % % % % % % % % % % % % % % % % % %

\markedsection{Karozuta I}

\begin{halfparskip}
  \liturgicaloption{I.} \dd~Laat ons allen ordelijk staan met vreugde en vrolijkheid; laat ons bidden en zeggen: Heer, ontferm U over ons.~--- \rr~Heer, ontferm U over ons. \liturgicalhint{(wordt herhaald na elke aanroeping)}

  \dd~Vader van barmhartigheid en God van alle troost, wij bidden U,

  \dd~Onze Verlosser, die voor ons zorgt en in alle dingen voorziet, wij bidden U,

  \dd~Voor vrede, eendracht en bestendigheid in de hele wereld en in alle kerken, wij bidden U,

  \dd~Voor ons land, voor alle landen en de gelovigen die er wonen, wij bidden U,

  \dd~Voor de temperatuur van de lucht, een voorspoedig (\translationoptionNl{vruchtbaar}) jaar, een goede oogst (\translationoptionNl{opbrengst}) van fruit en het behoud (\translationoptionNl{stabiliteit}) van de hele wereld, bidden wij U,

  \dd~Voor de gezondheid van onze heilige vaders, Paus \NN , hoofd van de hele Kerk van Christus, van Patriarch \NN , van onze Catholicos \NN , van onze Metropoliet \NN , van onze Bisschop \NN , en van al hun helpers, wij bidden U,

  \liturgicallbracket\footnote{De invoegingen tussen \liturgicaloption{[\ ]} kunnen worden weggelaten, behalve op feesten van Christus en vastenzondagen (Hudra, 414, voetnoot). Breviarium Chaldaicum somt alle aanroepingen zonder enig onderscheid op.}\dd~Voor de heersers die de macht hebben in deze wereld, wij bidden U,\liturgicalrbracket

  \dd~Barmhartige God, die met Uw liefde alles bestuurt, wij bidden U,

  \liturgicallbracket\dd~Voor rechtgelovige priesters en diakens, en al onze broeders in Christus, wij bidden U,\liturgicalrbracket

  \dd~Gij, rijk in barmhartigheid, en overvloedig in goedheid, wij bidden U,

  \liturgicallbracket\dd~Gij, die van vóór alle tijden zijt, en wiens macht (\translationoptionNl{rijk}) eeuwig blijft, wij bidden U,\liturgicalrbracket

  \dd~Gij, goed van natuur en Gever van alle goed, wij bidden U,

  \liturgicallbracket\dd~Gij, die geen behagen schept in de dood van de zondaar, maar wilt dat hij berouw heeft over zijn boosheid en leeft, wij bidden U,\liturgicalrbracket

  \dd~Gij die in de hemel wordt geprezen en op aarde wordt aanbeden, wij bidden U,

  \liturgicallbracket\dd~Gij, die door Uw heilige~--- geboorte~/ verschijning~/ vasten~/ intrede~/ verrijzenis~/ hemelvaart~/ nederdaling~/ Kruis~--- de aarde deed verheugen en de hemelen blij zijn, wij bidden U,\liturgicalrbracket

  \dd~Gij, die onsterfelijk zijt van natuur, en die in stralend licht woont, wij bidden U,

  \liturgicallbracket\dd~Redder van alle mensen en in het bijzonder van hen die in U geloven, wij bidden U,\liturgicalrbracket

  \dd~Verlos ons allemaal, Christus onze Heer, in Uw genade, doe in ons Uw vrede en rust toenemen en ontferm U over ons.
\end{halfparskip}

% % % % % % % % % % % % % % % % % % % % % % % % % % % % % % % % % % % % % % % %

\markedsection{Karozuta II}

\begin{halfparskip}
  \dd~Laat ons bidden, vrede zij met ons, laat ons God, de Heer van alles bidden en vragen.~--- \rr~Amen\footnote{Het volk antwoordt ``Amen'' na de woorden ``Heer van alles'' volgens Breviarium en Hudra \emph{of} op het einde van elke aanroeping. Beiden gewoonten bestaan.}.

  \dd~Dat Hij de stem van ons gebed moge horen, onze bede ontvangen en Zich over ons ontfermen.

  \dd~Voor de heilige katholieke Kerk, hier en in elke plaats, laat ons bidden en vragen God, de Heer van alles, dat Zijn heil en vrede mag vertoeven tot aan het einde van de wereld.

  \dd~Voor onze Vaders, de bisschoppen, laat ons bidden en vragen God, de Heer van alles, dat zij aan het hoofd van hun bisdommen mogen staan zonder blaam of vlek alle dagen van hun leven.

  \dd~Bijzonder voor het welzijn van onze heilige vaders, de Paus, de Patriarch, de Catholicos, de Metropoliet; de Bisschop, laat ons bidden en vragen God, de Heer van alles, dat Hij hen mag bewaren en behouden aan het hoofd van al hun bisdommen, dat zij mogen weiden, dienen en gereed maken voor de Heer, een volmaakt volk, ijverig in goede en mooie werken:

  \dd~Voor de priesters en diakens die in deze dienst van de waarheid zijn, laat ons bidden en vragen God, de Heer van alles, dat zij met een goed hart en zuivere gedachten vóór Hem mogen dienen.

  \dd~Voor alle kuise en heilige leden van het verbond, de kinderen van de heilige katholieke Kerk, laat ons bidden en vragen God, de Heer van alles, dat zij hun goede en heilige levensloop mogen voleinden, en van de Heer de hoop en de belofte in het land van het Leven ontvangen.

  \dd~Voor de gedachtenis van de gezegende mart Maria, de heilige maagd, de moeder van Christus, onze Redder en Levengever, laat ons bidden en vragen God de Heer van alles, dat de H. Geest die in haar verbleef ons moge heiligen in zijn goedheid, in ons Zijn wil vervullen, en ons zegelen in de waarheid alle dagen van ons leven.

  \dd~Voor de herdenking der profeten, apostelen, martelaren en belijders, laat ons bidden en vragen God de Heer van alles, dat door hun gebeden en lijden Hij ons geve met hen: goede hoop en redding, en ons waardig maken van hun gezegende gedachtenis en hun levende en ware belofte in het rijk der hemelen.

  \dd~Voor de gedachtenis van onze heilige vaderen, mar Gregorius [Nazianze], mar Basilius, mar Johannes [Chrysostomus], bisschoppen en leraren van de waarheid, mar Efrem, mar Narsai en mar Abraham, en alle heilige, oude en ware leraren, laat ons bidden en vragen God, de Heer van alles, dat door hun gebeden en smeekbeden de zuivere waarheid van de leer van hun godsdienst en van hun geloof bewaard kunnen blijven in heel de heilige katholieke Kerk tot aan het einde van de wereld.

  \dd~Voor de gedachtenis van onze vaders en broers, ware gelovigen, die zijn vertrokken en gegaan uit deze wereld in dit ware geloof en de orthodoxe godsdienst, laat ons bidden en vragen God, de Heer van alles, dat Hij hun overtredingen en fouten moge vergeven en wegnemen en ze waardig maken vreugde te hebben met de heiligen en rechtvaardigen die door Zijn wil zijn goedgekeurd.

  \dd~Voor dit land en haar inwoners; voor deze stad/dorp en degenen die erin wonen, voor dit huis en zij die er zorg voor dragen, en vooral voor deze gemeenschap, laat ons bidden en vragen God, de Heer van alles, dat Hij in Zijn goedheid van ons moge verwijderen het zwaard, gevangenschap, diefstal, aardbevingen, hongersnood, hagel, pest en alle boze plagen die vijandig zijn aan het lichaam.

  \dd~Voor degenen die dwalen van dit ware geloof en worden gehouden in de strikken van Satan, laat ons bidden en vragen God, de Heer van alles, dat Hij de hardheid van hun harten mogen doen omkeren, en hen doen weten dat God is één, de Vader van de waarheid, en Zijn Zoon Jezus Christus onze Heer.

  \dd~Voor hen die zwaar ziek zijn en bekoord door boze geesten, laat ons bidden en vragen God, de Heer van alles, dat Hij hen, in de overvloed van Zijn goedheid en barmhartigheid, Zijn engel van barmhartigheid en genezing moge sturen, om hen te bezoeken, te genezen, te sterken, te helpen en te troosten.

  \dd~Voor de armen, verdrukten, wezen, weduwen, gekwelden, verwarden en bedroefden in geest in deze wereld, laat ons bidden en vragen God de Heer van alles, dat Hij hen in Zijn goedheid moge geven wat ze nodig hebben, in Zijn genade hen verzorgen, in Zijn mededogen hen troosten, en hen redden van hem die hen schandelijk misbruikt.

  \dd~Bidt en vraagt God de Heer van alles dat gij voor Hem een koninkrijk van heilige priesters en volk moge zijn. Roept tot de almachtige Heer God met heel uw hart en gans uw ziel. Want Hij is God de Vader van mededogen, barmhartig en genadig, die niet wil dat zij die Hij heeft gevormd verloren zouden gaan, maar bekeren en leven vóór Hem. Vooral moeten we bidden, belijden, aanbidden, verheerlijken, eren en verheffen de ene God, de aanbiddelijke Vader, Heer van alles, die door zijn Christus goede hoop en verlossing gaf aan onze zielen, dat Hij moge vervullen in ons Zijn goedheid, barmhartigheid en mededogen tot het einde.~--- \rr~Amen.
\end{halfparskip}

% % % % % % % % % % % % % % % % % % % % % % % % % % % % % % % % % % % % % % % %

\markedsection{Karozuta III}

\begin{halfparskip}
  \dd~Laat ons vragen door gebed en smeking voor de engel van vrede en barmhartigheid.

  \rr~Van U, Heer. \liturgicalhint{(na elke aanroeping)}

  \dd~Dag en nacht, alle dagen van ons leven vragen wij het behoud van de vrede voor de Kerk en een leven zonder zonde.

  \dd~De eenheid in de liefde, de band van volmaaktheid, vragen wij door de werking van de Heilige Geest,

  \dd~Vergeving der zonden en alles wat nodig is voor ons leven en dat behaagt aan Uw Godheid vragen wij,

  \dd~De barmhartigheid van de Heer en Zijn goedheid vragen wij altijd,

  \dd~Wij bevelen onszelf en ieder van ons aan, aan de Vader, de Zoon en de H.~Geest.~--- \rr~Aan U, Heer.

  \cc~Aan U, Heer, almachtige God, vertrouwen wij onze lichamen en zielen toe; en van U, onze Heer en onze God, vragen we vergeving van onze overtredingen en zonden; geef ons dit in Uw goedheid en barmhartigheid, zoals Gij gewoon zijt, ten allen tijde, Heer van alles, Vader....~--- \rr~Amen.
\end{halfparskip}

% % % % % % % % % % % % % % % % % % % % % % % % % % % % % % % % % % % % % % % %

\markedsection{Trisagion \& conclusie van Ramsa}

\dd~Verheft uw stem, geheel het volk, en prijst de levende God.

\vspace{\parskip}
\begin{doublecols}
  \fontsize{11}{12}\selectfont

  \dutchl \rr~Qadisha Alaha~/ qadisha Hailthana~/ qadisha Lamayotha, ethrahaim alein.

  \dutchc{1} \rr~Heilige God, heilige Machtige, heilige Onsterfelijke, ontferm U over ons.
\end{doublecols}

\begin{halfparskip}
  Eer aan...~--- Heilige God...~--- Vanaf het begin en in alle eeuwigheid, amen.~--- Heilige God...

  \dd~Laat ons bidden; vrede zij met ons.
\end{halfparskip}

\begin{halfparskip}
  \liturgicalOption{Zondagen en Feesten.} \cc~Wij belijden, aanbidden en verheerlijken te allen tijde U, Heilige, die van nature heilig bent, glorieus in Uw Wezen, en hoog en verheven boven alles in Uw Godheid; U, heilige en gezegende Natuur, die van eeuwigheid bent, Heer van alles...
\end{halfparskip}

\begin{halfparskip}
  \liturgicalOption{Gedachtenissen.} \cc~Heilige, Glorieuze, Machtige en Onsterfelijke, die in de heiligen woont en wiens wil over hen tevreden is. Wij smeken U: luister, Heer, heb medelijden met ons en wees ons altijd genadig, zoals U gewend bent. Heer van alles...
\end{halfparskip}

\begin{halfparskip}
  \liturgicalOption{Alle dagen.} \dd~Zegen, mijn Heer. Buigt uw hoofden voor de handoplegging en ontvang de zegen.

  \cc~Moge Christus uw dienst heerlijk maken in het koninkrijk der hemelen.

  \liturgicalhint{(De voorhang van het heiligdom wordt nu geopend.)}
\end{halfparskip}

\begin{halfparskip}\begin{sfpar}
  % No frame here!
  \fullline

  \liturgicalhint{[Op grote feesten en gedachtenissen: Suyake: Pss 93--95 en 96--98. Volgende gebeden komen van de Hudra:]}

  \liturgicaloption{Gebed v\'o\'or suyaka I:} Versterk, Heer, onze zwakheid, en help en ondersteun onze broosheid, zodat we met heel ons hart en ziel het grote en eerbiedwaardige feest van \NN~mogen vieren, door de kracht en sterkte van Uw machtige arm, Heer van alles...

  \liturgicaloption{Gebed v\'o\'or suyaka II:} Wij belijden, aanbidden en verheerlijken altijd U, die verheven bent in Uw Wezen en verheerlijkt in Uw Godheid, die de hemel heeft doen neerdalen voor hen die hier beneden zijn, en hen hebt geheiligd door de heilige Eersteling te nemen, die U hebt verenigd met Uzelf, en die door Uw verschijning in het vlees engelen en mensen hebt doen verheugen, Heer van alles...

  \fullline
\end{sfpar}\end{halfparskip}

\begin{halfparskip}
  \liturgicaloption{Feesten van Christus:} \cc~Moge Uw goedertierenheid, Heer, neerdalen om Uw aanbidders te helpen, en laat Uw  genade overvloeien om hen te helpen die Uw Naam aanroepen. Openbaar Uzelf aan ons om Uw volk te redden, en om alle schapen van Uw weide te redden van alle kwaad, verborgen en open, Heer van alles...

  \liturgicaloption{Zondagen:} \cc~En terwijl onze zielen vervolmaakt zijn in het ene volledige geloof in Uw glorieuze Drie-eenheid, mogen wij allen in één eenheid van liefde waardig zijn om te allen tijde U te verheerlijken, te eren, te belijden en te aanbidden, Heer van alles...
\end{halfparskip}

% % % % % % % % % % % % % % % % % % % % % % % % % % % % % % % % % % % % % % % %

\markedsection{Onita D-Basaliqe \markedsectionhint{(tekst eigen aan de dag)}}

\begin{halfparskip}
  \liturgicalOption{Van Aankondiging tot Denha:} Vanaf het begin en in alle eeuwigheid.~--- Vanouds beloofde en bevestigde God aan Abraham: ``In uw zaad zullen worden gezegend alle zondige naties, die dood zijn in hun zonden en vernietigd door dwaling''. Want Hij is het die hen reinigt en hun kwalen geneest, zoals de profeet in de oudheid voorzag dat Hij onze pijnen zou opnemen en onze ziekten dragen. Daarom roepen en zeggen we: ``Glorie aan U, Zoon, de Heer van alles''.
\end{halfparskip}

\begin{halfparskip}
  \liturgicalhint{Het vers, Laat al het volk zeggen, etc., wordt niet gereciteerd van Aankondiging tot Maria's feest.}
\end{halfparskip}

\begin{halfparskip}
  \liturgicalOption{Seizoen van Denha:} Vanaf het begin en in alle eeuwigheid.~--- In de latere dagen verschenen en werden geopenbaard in de schepping de drie wezenlijke Personen aan de Jordaan: De Vader die riep en Zijn stem van boven liet horen: ``Dit is Mijn Zoon, Mijn Welbeminde'', en de Geest die ons het ware geloof leerde.
\end{halfparskip}

\begin{halfparskip}
  \liturgicalOption{Van de Grote Vasten tot Pinksteren} \liturgicalhint{worden deze slotverzen niet gezegd.}
\end{halfparskip}

\begin{halfparskip}
  \liturgicalOption{Seizoen der Apostelen:} Vanaf het begin en in alle eeuwigheid.~--- Christus, die Uw apostelen heeft uitgekozen en hen bekleed heeft met de kracht van de Geest, zodat zij predikers in de hele wereld konden zijn en Uw heerlijkheid en Godheid in de schepping konden openbaren door de machtige daden (gewrocht) door hun handen. Verblijd Uw Kerken met vrede en eendracht, en verhef het hoofd van hen die Uw Naam prediken en Uw geboden onderhouden. Moge, Heer, onze redding door Uw rechterhand worden bewaard; en mogen zij in de vrede, die van U komt, Uw overwinning prediken.
\end{halfparskip}

\begin{halfparskip}
  \liturgicalOption{Seizoen van de Zomer tot het feest van het H. Kruis:} Vanaf het begin en in alle eeuwigheid.~--- Verhoor Uw aanbidders, Christus, en zend ons vanuit Uw schatkamer mededogen, genade, redding, en vergeving van overtredingen. En zoals U Daniël verhoorde vanuit de leeuwenkuil en het gezelschap van Ananias in de vuuroven, red zo ook ons, Heer, van koningen, heersers, slechte mensen en wrede demonen, die als leeuwen dreigen ons te vernietigen. Geef door Uw machtige macht vrede in onze verwarring, verlos ons van hun boosheid, doe hun macht teniet en laat ons ons verheugen in Uw redding, o glorieuze Koning.
\end{halfparskip}

\begin{halfparskip}
  \liturgicalOption{Van het feest van het H. Kruis tot de Kerkwijding:} Vanaf het begin en in alle eeuwigheid.~--- Door de grote kracht van het Kruis heeft de Kerk vertrouwen gekregen over de dood en over Satan, en verheugt zij zich in [haar] verlossing, en viert het Kruis met lof in het zicht van de vijanden van de waarheid, want Het [Kruis] heeft haar kleinheid verheven. Door de machtige daden die Het [Kruis] in haar teweegbracht, en op de dag dat het werd gevonden, prijzen haar kinderen Het met lofliederen. O Grote Macht die de overwinnaar was in haar [de Kerk] strijd, houd Uw belofte aan haar, de roem van haar kinderen en de glorie van Uw openbaring.
\end{halfparskip}

\begin{halfparskip}
  \liturgicalOption{Zondagen van de Kerkwijding:} Vanaf het begin en in alle eeuwigheid.~--- U, Heer, hebt dit heilig huis gemaakt tot fundament van Uw troon. Heer, bevestig het door Uw handen. Mogen in haar de gebeden en tranen der noodlijdenden aanvaard worden en mogen alle hulp en geschenken vloeien van haar over Uw volk en de schapen van Uw weide, die gered zijn door Uw Kruis en hun toevlucht zoeken in de Naam van Uw majesteit.
\end{halfparskip}

\begin{halfparskip}
  \liturgicalOption{Alle seizoenen} \liturgicalhint{(behalve van de Vasten tot Pinksteren):} Laat al het volk zeggen ``Amen, amen''.~--- Heilige maagd Maria, moeder van Jezus onze Verlosser, bid en smeek om genade van het Kind dat uit uw boezem verscheen, dat Hij, in Zijn goedheid, tijden vol beproevingen van ons doet verdwijnen; moge Hij vrede en rust onder ons vestigen. Mogen de Kerk en haar kinderen door uw gebeden beschermd worden tegen de boze. En mogen wij op de glorieuze dag waarop Zijn Majesteit wordt geopenbaard, samen met u waardig zijn om vreugde te genieten in de bruidskamer van licht.
\end{halfparskip}

\begin{halfparskip}
  \liturgicalOption{}
\end{halfparskip}

% % % % % % % % % % % % % % % % % % % % % % % % % % % % % % % % % % % % % % % %

\markedsection{Zumara (d’Marya Qretak)\footnote{De ``zumara'' hier is niet die van de qurbana, maar de \emph{d'Marya Qretak}, de eerst eigen tekst in Hudra \& Breviarium.} \markedsectionhint{(eigen aan de dag)}}

\markedsection{Evangelion \markedsectionhint{(eigen, zie Hudra of Gaza)}}

\begin{halfparskip}
  \liturgicaloptionsc{Slota:} \liturgicaloption{Seizoenen van Aankondiging, Denha, Verrijzenis en alle feesten van de Heer.}

  \cc~Wij loven, eren, belijden en aanbidden Uw wonderbare en onuitsprekelijke heilsplan, Heer, dat in barmhartigheid en mededogen was vervolmaakt, voltooid en vervuld voor de vernieuwing en redding van onze natuur, in de Eersteling die van ons was, te allen tijde, Heer van alles...

  \liturgicalOption{Grote Vasten, Zomer, Elijas tot H. Kruis:} Heb medelijden met ons, Medelevende en Genadevolle. In Uw genade wend U tot ons, aarzel niet om naar ons te kijken en voor ons te zorgen, Heer, want in U is onze hoop en ons vertrouwen in alle seizoenen en tijden, Heer van alles...

  \liturgicalOption{Apostelen:} Heer, moge het gebed der heilige apostelen, de bede der ware predikers, het bidden en smeken der beroemde atleten, de verkondigers van gerechtigheid en zaaiers van vrede in de schepping, altijd bij ons zijn, in alle seizoenen en tijden, Heer van alles...

  \liturgicalOption{Heilig Kruis:} Laat Uw vrede wonen in alle gebieden, verhef Uw Kerk door Uw Kruis, bewaar haar kinderen in Uw goedheid, zodat zij in haar te allen tijde lof, eer, belijdenis en aanbidding aan U kunnen brengen, Heer van alles...

  \liturgicalOption{Kerkwijding:} Heer, verstevig in Uw mededogen de fundamenten van Uw Kerk; in Uw goedheid versterk haar grenzen, en laat Uw heerlijkheid wonen in de tempel die apart is gezet voor Uw dienst, alle dagen van de wereld, Heer van alles...

  \liturgicalOption{Mart Maria:} Moge het gebed, Heer, van de heilige maagd, het verzoek van de gezegende moeder, het verzoeken en smeken van haar die vol genade is, de gezegende mart Maria, bij ons zijn, te allen tijde en in alle seizoenen, Heer van alles...

  \liturgicalOption{Patroonheilige:} Moge het gebed, verzoek, smeken en vragen van onze beroemde (\translationoptionNl{zegevierende}) en heilige vader, de glorieuze mar N, en van al zijn metgezellen, voortdurend bij ons zijn, te allen tijde en in alle seizoenen, Heer van alles…

  \liturgicalOption{Martelaren en belijders:} Moge het gebed, Heer, van Uw martelaren, het verzoek van Uw belijders, het verzoeken en smeken van de atleten die Uw wil vervulden, voor ons bidden tot Uw Godheid, dat U ons Uw vrede en voorspoed mag schenken alle dagen van de wereld, Heer van alles...

  \liturgicalOption{Kerkleraren:} Moge het gebed, Heer, der heilige priesters, het verzoek der illustere leraren, het smeken en bidden der atleten, de vervullers van Uw wil, voortdurend bij ons zijn, altijd en in eeuwigheid, Heer van alles...

  \liturgicalOption{Johannes de Doper:} Moge het gebed van de beproefde en geteste Doper, het verzoek van de goede heraut, het verzoeken en smeken van de ware prediker, de glorieuze, heilige en beroemde martelaar mar Johannes, voortdurend bij ons zijn, in alle tijden en seizoenen, Heer van alles...
\end{halfparskip}

% % % % % % % % % % % % % % % % % % % % % % % % % % % % % % % % % % % % % % % %

\markedsection{Derde Suraya \markedsectionhint{(eigen)\footnote{Op feesten, wanneer er suyaka is, gaat de suraya vooraf aan de Onita d'Basaliqe, behalve op recentelijk toegevoegde feesten.}}}

\liturgicalhint{(Gevolgd door 3x alleluia en het eerste vers herhaald.)}

\begin{halfparskip}
  Zegen, Heer.~--- \liturgicalOption{Onze Vader} \liturgicalhint{(met qanona).}

  \cc~Onze heilige Heer en God, moge Uw Naam worden verheerlijkt, Uw Godheid aanbeden, Uw Majesteit geëerd, Uw grootheid gevierd en Uw Wezen verheven, en moge de eeuwige genade van Uw glorieuze Drie-eenheid Uw volk en de schapen van Uw weide altijd beschermen, Heer van alles...

  \cc~In de hemel en op aarde is Uw Godheid gezegend, Heer, en Uw Majesteit aanbeden. Heilig, glorieus, verheerlijkt, hoog en verheven is te allen tijde de aanbiddelijke en glorieuze Naam van Uw glorieuze Drie-eenheid, Heer van alles...
\end{halfparskip}

\fullline
\begin{halfparskip}\begin{sfpar}
  \liturgicalOption{SUBA'A} \liturgicalhint{(op gedachtenissen, in de Vasten, Rogatie der Nineviten, niet op Zondagen en feesten van de Heer)}

  \cc~Maak ons waardig, onze Heer en onze God, te genieten van een vredige avond, een rustgevende nacht, een ochtend waarin goede dingen worden verkondigd en een dag van goede daden van gerechtigheid; opdat wij daardoor Uw Godheid gunstig kunnen stemmen, alle dagen van ons leven, Heer van alles...

  \liturgicalhint{Zeg nu de Suba'a (\& qanona \& tesbohta).}

  \dd~Laat ons allen ordelijk staan met vreugde en vrolijkheid (\liturgicalhint{vasten:} \emph{met berouw en zorg}); laat ons bidden en zeggen: Heer, ontferm U over ons.~--- \rr~Heer, ontferm U over ons. \liturgicalhint{(wordt herhaald na elke aanroeping)}

  \dd~Machtige Heer, almachtige God van onze vaderen, wij bidden U.

  \dd~Heilige en glorieuze, die onder de heiligen woont en wiens wil (door hen) gunstig wordt gestemd, wij bidden wij U,

  \dd~Koning der koningen en Heer der heren, die in het voortreffelijke licht woont, wij bidden U,

  \dd~U die niemand heeft gezien, noch kan zien, wij bidden U,

  \dd~U die wilt dat alle mensen leven en zich tot de kennis van de waarheid wenden, wij bidden U,

  \dd~Voor de gezondheid van onze heilige vaders, Paus \NN , hoofd van de hele Kerk van Christus, van Patriarch \NN , van onze Catholicos \NN , van onze Metropoliet \NN , van onze Bisschop \NN , en van al hun helpers, wij bidden U,

  \dd~Barmhartige God, die met Uw liefde alles bestuurt, wij bidden U,

  \dd~Gij die in de hemel wordt geprezen en op aarde wordt aanbeden, wij bidden U,

  \dd~Laat Uw vrede en rust wonen in de vergadering van Uw aanbidders, Christus onze Verlosser, en ontferm U over ons.

  \liturgicalhint{\textbf{Trisagion} (Heilige God...) en \textbf{Onze Vader}...}

  \cc~U die Uw deur opent voor hen die erop kloppen en de smeekbeden van hen die U vragen beantwoordt, open, onze Heer en onze God, de deur van barmhartigheid voor ons gebed; ontvang ons verzoek, en antwoord in Uw barmhartigheid op onze smeekbeden uit Uw rijke en overvloedige schatkist, U die goed bent: en laat Uw barmhartigheid en gaven aan de behoeftigen en verdrukten niet achterwege, Uw dienaren die U aanroepen en U smeken, in alle seizoenen en tijden, Heer van alles...

  \cc~U die de stem hoort der gerechtigen en rechtvaardigen die U voortdurend gunstig stemmen, en die de wens vervult van hen die U vrezen, hoor, mijn Heer, in Uw mededogen het gebed van Uw dienaren, en ontvang in Uw mededogen het verzoek van Uw aanbidders. en heb medelijden met de getroffenen en gekwelden, Uw dienaren die U aanroepen en U smeken, in alle seizoenen en tijden, Heer van alles...
\end{sfpar}\end{halfparskip}

\fullline

% % % % % % % % % % % % % % % % % % % % % % % % % % % % % % % % % % % % % % % %

\markedsection{Gebeden voor hulp}

\liturgicalhint{In aanwezigheid van meerdere priesters in volgorde van prioriteit. Elk gebed wordt afgesloten met ``Amen, zegen, mijn Heer''. Het eerste wordt altijd gezegd indien het officie plaats heeft in een kerk.}

\begin{halfparskip}
  [\cc~Moge, Heer, Uw barmhartige hulp, de grote bijstand van Uw goedheid, de verborgen en glorieuze kracht van Uw glorieuze Drie-eenheid, en Uw rechterhand vol erbarmen en genade, de zwakheid van Uw aanbidders beschutten en begeleid worden vanuit Uw heilig huis dat vol is van alle hulp en alle zegeningen, door het gebed van de zalige Maria en van alle heiligen die U gunstig stemmen, Heer van alles, Vader...

  \liturgicalhint{2.} Onze Heer en God, mogen Uw dienaren gezegend worden door Uw zegen, en Uw aanbidders worden beschermd door de zorg van Uw wil; Heer, moge de voortdurende vrede van Uw goddelijkheid en de blijvende vrede van Uw Goddelijkheid heersen over Uw volk en in Uw Kerk, alle dagen van de wereld, Heer van alles...

  \liturgicalhint{3.} Moge de zegen van Hem die alles zegent, de vrede van Hem die alles tot bedaren brengt, de genade van Hem die iedereen genadig is, de bescherming van onze aanbiddelijke God, met ons, onder ons en rondom ons zijn en ons beschermen van de boze en van zijn krachten altijd en in eeuwigheid, Heer van alles...

  \liturgicalhint{4.} Onze Heer en onze God, moge wij gezegend zijn door Uw zegen, mogen wij beschermd worden door Uw voorzienigheid, moge Uw kracht ons ondersteunen, moge Uw hulp ons vergezellen, moge Uw rechterhand ons overschaduwen, moge Uw vrede heersen onder ons, moge Uw Kruis een hoge vesting (\translationoptionNl{muur}) en toevlucht voor ons zijn en mogen wij onder zijn vleugels worden verdedigd tegen de boze en zijn legers, altijd en in eeuwigheid, Heer van alles...

  \liturgicalhint{5.} Heer, gezegend is de genade van Uw goedheid, aanbiddelijk zijn de beloften van Uw Heerlijkheid, die ons leren om altijd naar U te kijken en in U te roemen. Laat onze hoop niet van U gescheiden worden alle dagen van ons leven, Heer van alles...

  \liturgicalhint{6.} Onze Heer en God, moge Uw zegen rusten op Uw volk, en moge Uw genade voortdurend op ons, zwakke zondaars zijn; onze goede Hoop, onze barmhartige Toevlucht en Vergever van schulden en zonden, Heer van alles...

  \liturgicalhint{7.} Moge de vrede van de Vader met ons zijn, en de liefde van de Zoon onder ons, en moge de Heilige Geest ons leiden volgens Zijn wil, en mogen Zijn barmhartigheid en medeleven altijd en in eeuwigheid op ons zijn, Heer van alles...

  \liturgicalhint{8.} Heer, moge Uw vrede in ons wonen en Uw rust in ons heersen, en moge Uw liefde onder ons toenemen alle dagen van ons leven, Heer van alles...

  \liturgicalhint{9.} Heer, bescherm ons door Uw rechterhand, verdedig ons onder Uw vleugels en laat Uw hulp ons alle dagen van ons leven vergezellen, Heer van alles...

  \liturgicalhint{10.} Heer, geef ons Uw voortdurende vrede, liefde, liefde voor kennis, leven, geluk en vreugde, en laat Uw zorg over ons geen enkele dag van ons leven ontbreken, Heer van alles...

  \liturgicalhint{11.} Wees een slapeloze Bewaker van het bolwerk waarin Uw schapen wonen, opdat ze niet worden gekwetst door de wolven die dorsten naar het bloed van Uw kudde, want U bent de zee die niet zal afnemen, Heer van alles...

  \liturgicalhint{12.} Heer, zegen ons met Uw zegeningen, omring ons met de vesting (\translationoptionNl{muur}) van Uw zorg, beroof ons niet van het goede en laat ons aan tafel liggen in Uw stralend bruidsfeest, Heer van alles...

  \liturgicalhint{13.} Heer, kom ons te hulp in Uw barmhartigheid, openbaar onze verlossing in Uw mededogen, en leid onze stappen op de paden van gerechtigheid alle dagen van ons leven, Heer van alles...

  \liturgicalhint{14.} Heer, laat Uw goedheid tot ons doordringen wanneer Uw gerechtigheid ons oordeelt, en laat Uw genade ons te hulp komen op de dag waarop Uw Majesteit zal verschijnen, Heer van alles...

  \liturgicalhint{15.} Heer van alles, laat Uw zegen en Uw genade, Uw rechterhand vol barmhartigheid en mededogen, de gemeenschappen van Uw aanbidders die U altijd aanroepen en smeken, overschaduwen en begeleiden, Heer van alles...]
\end{halfparskip}

% % % % % % % % % % % % % % % % % % % % % % % % % % % % % % % % % % % % % % % %

\fullline

\markedsection{Finale gebeden}

\begin{halfparskip}
  \liturgicalhint{(1) Maria.} Moge het gebed, Heer, van de heilige maagd, het verzoek van de gezegende moeder, het verzoeken en smeken van haar die vol genade is, de gezegende mart Maria, de grote kracht van het zegevierende Kruis, de goddelijke hulp, en de bede van mar Johannes de Doper altijd met ons zijn in alle eeuwigheid, Heer van alles...

  \liturgicalhint{Apostelen.} Moge het gebed, Heer, van de heilige apostelen, de bede der ware predikers, het verzoeken en smeken der beroemde atleten, de verkondigers van gerechtigheid, de predikers van heiligheid, de zaaiers van vrede in de schepping, altijd met ons zijn in alle eeuwigheid, Heer van alles, Vader...

  \liturgicalhint{Heiligen.} Moge het gebed, verzoek, smeken en vragen van onze beroemde en heilige vader, mar Thomas apostel, van mar Adai en mar Mari, leraren van het Oosten, mar Stefanus, de eerstgeborene der martelaren, mar Simon bar Sabbae, mar Jacob, mar Efrem, van de krachtige reus, mar Joris, de beroemde martelaar, van mar Cyriacus, mar Pethion, mar Hormizd, van de gezegende mar Eugenius en geheel zijn geestelijk gezelschap, van sint Barbara en van Shmuni en haar zonen, van Meskenta en haar twee zonen, van alle martelaren en heiligen van onze Heer, altijd voor ons een hoge muur en een stevig huis van toevlucht zijn, om onze lichamen en  zielen te verlossen, te bevrijden, te redden en te bewaren van de Boze en zijn legers, in alle eeuwigheid, Heer van alles.
\end{halfparskip}

\fullline

\begin{halfparskip}
  \liturgicalhint{(2) Of korte versie:} Moge het gebed, Heer, van de heilige maagd, het verzoek van de gezegende moeder, het smeken en bidden van haar die vol genade is, de gezegende mart Maria, de grote kracht van het zegevierende Kruis, de goddelijke hulp, de bede van mar Johannes de Doper, het gebed der heilige apostelen, de bede der ware predikers, het verzoeken en smeken der beroemde atleten, de verkondigers van gerechtigheid, de predikers van heiligheid, de zaaiers van vrede in de schepping; het gebed van onze beroemde en heilige vader mar Thomas apostel, van mar Adai en mar Mari, leraren van het Oosten, mar Stefanus, de eerstgeborene der martelaren, mar Simon bar Sabbae, mar Jacob, mar Efrem, de krachtige reus, mar Joris, de beroemde martelaar, van mar Cyriacus, mar Pethion, mar Hormizd, de gezegende mar Eugenius en geheel zijn geestelijk gezelschap, sint Barbara en Shmuni en haar zonen, Meskenta en haar twee zonen, van alle martelaren en heiligen van onze Heer, altijd voor ons een hoge muur en een stevig huis van toevlucht zijn, om onze lichamen en zielen te verlossen, te bevrijden, te redden en te bewaren van de Boze en zijn legers, in alle eeuwigheid, Heer van alles...
\end{halfparskip}

\begin{halfparskip}
  \liturgicalhint{Hutama.} \liturgicalhint{1.} Glorie aan U, Jezus, onze zegevierende Koning, de Glans van de eeuwige Vader, verwekt zonder begin, vóór alle tijden en (geschapen) dingen; we hebben geen hoop en verwachting tenzij U, de Schepper. Door het gebed der rechtvaardigen en uitverkorenen die vanaf het begin U hebben behaagd (\translationoptionNl{door U goedgekeurd waren}), vergeef onze zonden, scheld kwijt onze overtredingen, verlos ons van onze ellende, verhoor onze verzoeken, breng ons naar het glanzende licht, en bewaar ons door Uw levend Kruis van alle kwaad, verborgen en open, Christus de Hoop van onze natuur, nu en altijd en in eeuwigheid. \hspace{1em} \liturgicalhint{of:}

  \liturgicalhint{2.} Moge het gebed van Uw zwakke dienstknechten, onze Heer en onze God, aangenomen worden voor de troon van Uw Godheid; en moge deze, onze samenkomst, behagen aan de Wil van Uw majesteit; dat wij van U de gift van een goede gezondheid voor het lichaam en bescherming voor de ziel mogen ontvangen; toename van voedsel; vergeving van schulden en kwijtschelding van zonden, en eeuwige vrede, o Heer, en langdurige rust; eenheid in liefde die niet voorbijgaat en niet uit ons midden verdwijnt, in elk tijdperk van deze wereld, nu en altijd en in eeuwigheid.

  \liturgicalhint{3.} Gezegend zij God voor altijd, en verheerlijkt zij Zijn heilige Naam tot in alle eeuwigheid. Aan Hem doen wij een verzoek, en wij smeken de overstromende zee van Zijn barmhartigheid, dat Hij ons waardig zou maken van de verheven heerlijkheid van Zijn rijk, van de zaligheid met Zijn heilige engelen en de onthulling van gelaat voor Hem [= vertrouwen], en het staan aan Zijn rechterhand in het hemelse Jeruzalem, in Zijn goedheid en barmhartigheid, nu en altijd en in eeuwigheid.

  \liturgicalhint{4.} Moge God, de Heer van alles, in wiens huis we zijn samengekomen en voor wiens Majesteit we gebeden hebben, in de grote hoop op Zijn genade, ons gebed in Zijn mededogen horen, en ons verzoek in Zijn medelijden aanvaarden; en moge Hij het vuil van onze overtredingen en zonden wassen en reinigen met de hysop van Zijn overvloedige medelijden, en rust geven aan de zielen der overledenen in de glorieuze woningen van Zijn koninkrijk. Moge Hij ons allen besprenkelen met de dauw van Zijn zoetheid. En moge de rechterhand van Zijn zorg ons en alle schepselen overschaduwen in Zijn liefdevolle goedertierenheid en barmhartigheid, nu en altijd en in eeuwigheid.

  \liturgicalhint{5.} Moge God, de Heer van alles, die Zijn lofprijzingen aan onze mond heeft toevertrouwd, Zijn liederen aan onze tong, Zijn lofzangen aan onze kelen, Zijn belijdenis aan onze lippen, Zijn geloof aan ons harten, onze gebeden verhoren, onze beden aannemen, verzoend worden door (\translationoptionNl{Zijn behagen vinden in}) ons smeken, onze schulden kwijtschelden, onze verzoeken inwilligen met weldaden en niet met terechtwijzing; en moge Hij uit de grote schatkamer van Zijn barmhartigheid Zijn erbarmen en mededogen over ons en over de hele wereld storten, nu en altijd en in eeuwigheid.

  \liturgicalhint{6.} Moge de Naam van God, de Heer van alles, die tijden en seizoenen ordent, onder ons verheerlijkt worden; en moge de rechterhand van de zorg van Zijn genade ons overschaduwen, die zwak en zondig zijn, en de hele wereld, de heilige Kerk en haar kinderen, onze vaders, broeders, oversten en leraren, onze overledenen die van ons gescheiden zijn en uit ons midden zijn genomen, en heel onze broederschap in Christus, nu en altijd en in eeuwigheid.

  \liturgicalhint{7.} Aan God zij glorie, aan de engelen eer, aan Satan beschaming, aan het Kruis verering, aan de Kerk verheerlijking, aan de overledenen verkwikking, aan de boetvaardigen opname, aan de gevangenen vrijlating, aan de zieken en zwakken herstel en genezing, en aan de vier uiteinden van de wereld grote vrede en rust. Ook over ons, die zwak en zondig zijn, mogen de barmhartigheid en genade van onze aanbiddelijke God komen, mogen zij ons overschaduwen, over ons stromen, en standvastig blijven en voortdurend regeren, nu en altijd en in eeuwigheid.

  \liturgicalhint{8.} Bij de rechterhand van Uw Majesteit, onze Vader die in de hemel is, zegen ons allen, o mijn Heer; behoud ons allen; help ons allen; steun en bescherm ons allen; verkwik de overledenen; laat Uw rechterhand ons allen overschaduwen; mogen Uw genade en barmhartigheid over ons allen worden uitgestort; en moge voortdurende lof, eer, belijdenis, aanbidding en dankzegging tot U opstijgen uit de mond van ons allen, nu en altijd en in eeuwigheid.

  \liturgicalhint{9. (in een klooster of huis:)} Moge God, de Heer van alles, in Zijn goedheid onze gemeenschap zegenen, en in de overvloedige menigte van Zijn barmhartigheid ons behoeden van te vallen; moge Hij onze verzoeken vanuit Zijn schatkamer beantwoorden; en moge over de hele wereld, over de heilige Kerk en haar kinderen, over dit land en haar inwoners, over deze woning en zij die erin wonen, en over ons allemaal en ieder van ons samen, de barmhartigheid en genade van ons goede God komen en voortdurend worden uitgestort, nu en altijd en in eeuwigheid.

  \rr~Amen. \liturgicalhint{Kiss of peace.}
\end{halfparskip}

% % % % % % % % % % % % % % % % % % % % % % % % % % % % % % % % % % % % % % % %

\end{document}