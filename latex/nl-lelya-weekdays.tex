\documentclass[12pt,twoside,a5paper]{article}

\usepackage{multicol}

\usepackage[main=dutch]{babel}
\usepackage{divine-office}

% % % % % % % % % % % % % % % % % % % % % % % % % % % % % % % % % % % % % % % %

% Version: 2024-07-13
\begin{document}

\title{Lelya~--- weekdagen}
\author{}
\date{}
\maketitle

% The following prevents footnotes and paracol from interacting in bad ways.
% Not really an idea why...
% See: https://stackoverflow.com/questions/61779911/paracol-and-footnote-placing-in-latex
\footnotelayout{\ }

% % % % % % % % % % % % % % % % % % % % % % % % % % % % % % % % % % % % % % % %

\begin{halfparskip}
  \cc~Eer aan God in den hoge \liturgicalhint{(3x)}. En op aarde vrede en goede hoop aan de mensen, altijd en in eeuwigheid.

  [Amen].~--- \rr~Zegen, Heer.~--- \liturgicalhint{[vredekus].}

  \cc~Onze Vader die in de hemelen zijt,

  \rr~Geheiligd zij Uw Naam. Uw rijk kome, heilig, heilig, heilig zijt Gij. Onze Vader die in de hemelen zijt, de hemel en de aarde zijn gevuld met Uw onmetelijke glorie; de engelen en de mensen roepen U toe: heilig, heilig, heilig zijt Gij.~--- Onze Vader die in de hemelen zijt, geheiligd zij Uw Naam. Uw rijk kome, Uw wil geschiede op aarde zoals in de hemel. Geef ons heden het brood dat we nodig hebben en vergeef ons onze schulden en zonden zoals wij ook vergeven hebben aan onze schuldenaren. En leid ons niet in bekoring, maar verlos ons van de Kwade. Want van U is het koninkrijk en de kracht en de heerlijkheid in eeuwigheid, amen.

  \cc~Eer aan de Vader, de Zoon, en de Heilige Geest.

  \rr~Vanaf het begin en in alle eeuwigheid, amen en amen. Onze Vader die in de hemelen zijt, geheiligd zij Uw naam, Uw rijk kome, heilig, heilig, heilig zijt Gij. Onze Vader die in de hemelen zijt, de hemel en de aarde zijn gevuld met Uw onmetelijke glorie; de engelen en de mensen roepen U toe: heilig, heilig, heilig zijt Gij.

  \dd~Laat ons opstaan om te bidden, vrede zij met ons.\footnote{In deze verkorte editie hebben we elke hulala door een marmita vervangen.}

  \cc~Laat ons opstaan, o Heer, in Uw kracht, en bevestigd worden in Uw hoop, mogen we opgetild en gesterkt worden door de hoge arm van Uw macht; en mogen we waardig zijn, met de hulp van Uw goedertierenheid, om te allen tijde lof, eer, belijdenis en aanbidding tot U te verheffen, Heer van alles, Vader...
\end{halfparskip}

% % % % % % % % % % % % % % % % % % % % % % % % % % % % % % % % % % % % % % % %

\markedsection{PSALMEN}

% % % % % % % % % % % % % % % % % % % % % % % % % % % % % % % % % % % % % % % %

\markedsection{Eerste marmita\footnote{Breviarium: alle Maandagen: hulala 1--3; Dinsdagen: 4--6; Woensdagen: 7--9; Donderdagen: 10--11 en 15; Vrijdagen: 16--18; Zaterdagen: 19--21; Hulale 12--14 zijn gereserveerd voor feesten en gedachtenissen.}\markedsectionhint{(origineel: 3 hullale)}}

\liturgicalhint{Eer aan... na elke marmita.}

\begin{halfparskip}
  \markedsubsectionrubricwithhint{Maandagen ``voor'': Marmita 1*}{(origineel: hulala 1)}

  \psalm{\Ps{1}} Zalig de man die de raad van goddelozen niet volgt,~\sep\ die de weg van zondaars niet inslaat,

  \liturgicalhint{Alleluia, Alleluia, Alleluia.~--- Eerste vers.}

  noch neerzit in de kring van spotters.

  Maar die zijn vreugde vindt in de Wet van de Heer,~\sep\ en Zijn Wet overweegt bij dag en bij nacht.

  Hij is als een boom,~\sep\ geplant aan waterstromen.

  Die vrucht geeft op zijn tijd, en wiens lover niet verwelkt;~\sep\ ja, al wat hij doet, gedijt.

  Niet zo de goddelozen, niet zo;~\sep\ ze zijn als kaf, dat de wind verstrooit.

  Daarom zullen de goddelozen niet standhouden in het oordeel,~\sep\ noch de zondaars in de kring der
  rechtvaardigen.

  Want de Heer draagt zorg voor de weg der rechtvaardigen,~\sep\ maar de weg der goddelozen loopt uit op verderf.

  \psalm{\Ps{2}} Waarom woelen de heidenen,~\sep\ en smeden de naties ijdele plannen?

  De koningen der aarde rijzen samen op,~\sep\ en de vorsten spannen tezamen tegen de Heer en Zijn Gezalfde:

  ``Laten wij Hun boeien verbreken,~\sep\ en werpen wij Hun kluisters van ons af''.

  Die in de hemelen woont, Hij lacht hen uit,~\sep\ de Heer drijft de spot met hen.

  Dan spreekt Hij hen toe in Zijn toorn,~\sep\ in Zijn gramschap doet Hij hen sidderen:

  ``Maar Ik, Ik heb Mijn Koning aangesteld,~\sep\ op de Sion, Mijn heilige berg!''

  Ik wil het besluit van de Heer bekendmaken: de Heer sprak tot mij:~\sep\ ``Mijn Zoon zijt Gij, Ik heb U heden voortgebracht.

  Vraag Mij, en Ik geef U de volken tot erfdeel,~\sep\ en tot Uw bezit de grenzen der aarde.

  Gij zult hen regeren met ijzeren scepter,~\sep\ hen in stukken slaan als het vat van een pottenbakker''.

  Nu dan, koningen, komt tot inzicht,~\sep\ laat u gezeggen, die de wereld bestuurt.

  Dient de Heer in vreze en juicht Hem toe;~\sep\ met huivering Hem uw hulde gebracht!

  Opdat Hij niet toorne en gij van de weg af vergaat als weldra Zijn toorn zal zijn ontbrand;~\sep\ zalig allen die vluchten tot Hem.

  \psalm{\Ps{3}} Heer, hoe talrijk zijn zij, die mij kwellen!~\sep\ Velen staan tegen mij op!

  Velen zijn er, die van mij zeggen:~\sep\ ``Voor hem is er geen redding bij God!''

  Maar Gij, Heer, zijt mijn schild,~\sep\ Gij mijn roem, die mijn hoofd opbeurt.

  Ik riep tot de Heer met luide stem,~\sep\ en Hij verhoorde mij van Zijn heilige berg.

  Ik legde mij neer en ik sliep in;~\sep\ dan stond ik op, want de Heer is mijn steun.

  Neen, nu vrees ik de drommen van duizenden niet,~\sep\ die zich opstellen rondom mij heen.

  Verhef U, Heer!~\sep\ red mij, mijn God!

  Want al mijn weerstrevers hebt Gij op de kaken geslagen,~\sep\ de tanden der bozen hebt Gij verbrijzeld.

  Bij de Heer is redding:~\sep\ Uw zegen zij over Uw volk!

  \psalm{\Ps{4}} Verhoor mij, als ik U aanroep, mijn rechtvaardige God,

  die mij in kwelling verlichting bracht;~\sep\ wees mij genadig en verhoor mijn gebed.

  Mannen, hoelang nog blijft gij verstokt van hart;~\sep\ waarom ijdelheid bemind en leugen gezocht?

  Weet dat de Heer jegens Zijn heiligen wonderbaar handelt;~\sep\ de Heer zal mij verhoren, als ik Hem aanroep.

  Siddert en wilt niet zondigen,~\sep\ denkt na bij u zelf, op uw sponde, en zwijgt!

  Brengt gerechte offers,~\sep\ en hoopt op de Heer.

  Velen zeggen: ``Wie zal ons voorspoed doen zien?''~\sep\ Doe opgaan over ons, Heer, het licht van Uw gelaat!

  Gij hebt mij een vreugde in het hart gestort,~\sep\ groter dan bij overvloed van tarwe en wijn.

  Zodra ik mij neerleg, slaap ik in vrede,~\sep\ want Gij alleen, Heer, stelt mij in veiligheid.
\end{halfparskip}

\begin{halfparskip}
  \markedsubsectionrubricwithhint{Dinsdagen ``voor'': Marmita 11*}{(origineel: hulala 4)}

  \psalm{\Ps{31}} Ik vlucht tot U, O Heer: dat ik nimmer te schande worde;~\sep\ bevrijd mij toch in Uw gerechtigheid!

  \liturgicalhint{Alleluia, Alleluia, Alleluia.~--- Eerste vers.}

  Neig Uw oor naar mij;~\sep\ haast U mij te redden!

  Wees mij een rots, waar ik vluchten kan,~\sep\ een versterkte burcht tot mijn behoud.

  Want Gij zijt mijn Rots en mijn burcht,~\sep\ en omwille van Uw Naam zult Gij mijn Leider zijn en Gids.

  Gij zult mij trekken uit het net, dat zij heimelijk mij spanden,~\sep\ want Gij zijt mijn toevlucht.

  In Uw handen beveel ik mijn geest:~\sep\ Gij zult mij bevrijden, O Heer, getrouwe God.

  Gij haat, die nietige afgoden dienen,~\sep\ maar ik vertrouw op de Heer.

  Vol vreugde zal ik juichen om Uw ontferming, omdat Gij op mijn ellende hebt neergezien:~\sep\ Gij waart mijn hulp in tijden van nood.

  Gij gaaft mij niet prijs aan de macht van een vijand,~\sep\ maar plaatste mijn voeten op ruime baan.

  O Heer, wees mij genadig, want ik ben in nood;~\sep\ van droefheid kwijnt mijn oog, mijn ziel en mijn lichaam.

  Ja, mijn leven teert weg in smart,~\sep\ en mijn jaren in geween.

  Van verdriet is mijn kracht gebroken,~\sep\ en mijn gebeente verdord.

  Voor al mijn vijanden ben ik tot smaad geworden, voor mijn buren tot spot, en tot afschrik voor mijn bekenden;~\sep\ die mij buiten zien, vluchten van mij heen.

  Ik ben als een dode door vergetelheid uit het hart gewist,~\sep\ en ik werd als een vat in scherven.

  Ja, ik hoorde het fluiten van velen~--- verschrikking van alle zijden!~\sep\ Zij schoolden tegen mij samen en zonnen op moord.

  Maar ik, O Heer, vertrouw op U,~\sep\ en zeg: Gij zijt mijn God.

  Mijn levenslot ligt in Uw hand,~\sep\ ontruk mij aan de hand van mijn vijanden en vervolgers.

  Toon aan Uw dienstknecht Uw vredig gelaat,~\sep\ red mij in Uw barmhartigheid.

  Heer, laat mij niet beschaamd worden, want U riep ik aan;~\sep\ maar dat de bozen zich schamen en zwijgen, voortgedreven naar het dodenrijk.

  Dat de leugenlippen verstommen,~\sep\ die vermetel tegen de rechtvaardige spreken, in trots en verachting.

  Hoe groot, Heer, is Uw goedheid,~\sep\ die Gij hebt weggelegd voor hen, die U vrezen,

  Die Gij bewijst aan hen, die vluchten tot U,~\sep\ ten aanschouwen der mensen.

  Gij beschermt hen onder de schutse van Uw aanschijn,~\sep\ tegen het samenzweren der mannen,

  Gij verbergt hen in Uw tent,~\sep\ tegen het schelden der tongen.

  Gezegend de Heer, want Hij bewees mij~\sep\ Zijn wondere barmhartigheid in de versterkte stad.

  Wel sprak ik in mijn onrust:~\sep\ ``Ik ben van Uw aanschijn verstoten.''

  Maar Gij hebt de stem van mijn smeken gehoord,~\sep\ daar ik tot U riep.

  Bemint de Heer, gij, al Zijn heiligen:~\sep\ de getrouwen behoedt de Heer,

  Maar overvloedig vergeldt Hij,~\sep\ die handelen in trots.

  Houdt moed, en weest sterk van hart,~\sep\ gij allen, die hoopt op de Heer.

  \psalm{\Ps{32}} Gelukkig hij, wiens misdaad vergeven,~\sep\ wiens zonde bedekt is.

  Gelukkig de mens, wie de Heer zijn schuld niet toerekent,~\sep\ en in wiens geest geen bedrog is.

  Zolang ik bleef zwijgen, werd mijn gebeente verteerd,~\sep\ onder mijn aanhoudend gezucht.

  Want dag en nacht drukte Uw hand op mij;~\sep\ mijn kracht teerde weg als bij zomerse hitte.

  Mijn zonde heb ik beleden voor U,~\sep\ en mijn schuld hield ik niet verborgen;

  Ik sprak: ``Ik belijd mijn boosheid voor de Heer,''~\sep\ en Gij hebt de schuld van mijn zonde vergeven.

  Daarom moet iedere vrome bidden tot U,~\sep\ in tijden van nood.

  En al breekt dan de watervloed los,~\sep\ hem zal hij niet genaken.

  Gij zijt mijn Toeverlaat, Gij zult voor nood mij behoeden,~\sep\ en mij omgeven met vreugde over mijn redding.

  Ik zal u leren, en de weg wijzen, die gij moet gaan,~\sep\ u onderrichten en vast Mijn ogen richten op u.

  Weest niet als paard en muilezel zonder verstand, wier onstuimigheid men bedwingt met toom en gebit,~\sep\ anders komen ze niet naar u toe.

  Veel smart valt de boze ten deel,~\sep\ maar erbarming omgeeft wie vertrouwt op de Heer.

  Verheugt u in de Heer, weest blijde, gij, rechtvaardigen,~\sep\ en jubelt, gij allen, die oprecht zijt van harte.
\end{halfparskip}

\begin{halfparskip}
  \markedsubsectionrubricwithhint{Woensdagen ``voor'': Marmita 17*}{(origineel: hulala 7)}

  \psalm{\Ps{44}} Met eigen oren, O God, hebben wij het gehoord,~\sep\ onze vaderen hebben het ons verhaald:

  \liturgicalhint{Alleluia, Alleluia, Alleluia.~--- Eerste vers.}

  Het werk, dat Gij gewrocht hebt in hun dagen,~\sep\ in overoude tijden.

  Met eigen hand hebt Gij de heidenen verdreven, maar hen geplant;~\sep\ om hen te doen bloeien hebt Gij volken verslagen.

  Neen, hun zwaard was het niet, dat het land heeft veroverd,~\sep\ noch hun arm, die hun redding bracht.

  Maar Uw rechter en Uw arm,~\sep\ en het licht van Uw gelaat, want Gij hadt hen lief.

  Gij zijt mijn Koning, mijn God,~\sep\ die de zege verleende aan Jacob.

  Door U hebben wij onze tegenstanders verdreven,~\sep\ en in Uw Naam vertrapten wij die tegen ons waren opgestaan.

  Neen, ik heb niet vertrouwd op mijn boog,~\sep\ noch was het mijn zwaard, dat mij redding bracht.

  Maar Gij hebt ons verlost van onze weerstrevers,~\sep\ en die ons haten, hebt Gij beschaamd.

  In God roemden wij ten allen tijde,~\sep\ en Uw Naam prezen wij immer.

  Maar nu hebt Gij ons verstoten en te schande gemaakt,~\sep\ en Gij trekt niet meer op, God, met onze legerscharen.

  Gij hebt ons doen wijken voor onze tegenstanders;~\sep\ die ons haten, hebben ons uitgeplunderd.

  Als slachtschapen hebt Gij ons overgeleverd,~\sep\ en onder de volken verstrooid.

  Voor een spotprijs hebt Gij Uw volk verkocht,~\sep\ en weinig winst heeft U de verkoop gebracht.

  Gij hebt ons te schande gemaakt voor onze buren,~\sep\ tot spot en hoon voor hen, die ons omringen.

  Gij hebt ons tot spreekwoord gemaakt onder de heidenen;~\sep\ de volken schudden het hoofd over ons.

  Mijn schande staat mij immer voor ogen,~\sep\ en schaamte bedekt mijn gelaat,

  Om de praatjes van schimper en spotter,~\sep\ om vijand en tegenstander.

  Dit alles kwam over ons; en toch: wij zijn U niet vergeten;~\sep\ en hebben Uw verbond niet geschonden;

  Ook viel ons hart niet van U af,~\sep\ en onze schreden weken niet af van Uw wegen,

  Toen Gij ons gebroken hebt in het oord der kwelling,~\sep\ en ons met duisternis hebt omhuld.

  Hadden wij de Naam van onze God vergeten,~\sep\ en naar een vreemde god onze handen uitgestrekt,

  Zou God dat niet hebben ontdekt?~\sep\ Hij toch doorschouwt de geheimen van het hart.

  Ja, om Uwentwil blijft men ons doden,~\sep\ worden wij als slachtschapen beschouwd.

  Ontwaak dan, Heer, wat slaapt Gij?~\sep\ Waak op, blijf ons niet eeuwig verstoten!

  Waarom verbergt Gij Uw aanschijn,~\sep\ vergeet Gij onze ellende en onze verdrukking?

  Want onze ziel ligt neer in het stof,~\sep\ ons lichaam kleeft aan de aarde.

  Sta op om ons te helpen,~\sep\ en bevrijd ons om Uw barmhartigheid.

  \psalm{\Ps{45}} Een heerlijk lied welt op uit mijn hart: de Koning wijd ik mijn zang;~\sep\ mijn tong is de stift van een vaardige schrijver.

  Gij zijt de schoonste onder de kinderen der mensen; bevalligheid ligt op uw lippen:~\sep\ daarom heeft God u voor eeuwig gezegend.

  Gord uw zwaard om de heup, gij, machtige held,~\sep\ uw sieraad en luister.

  Ruk zegerijk uit voor waarheid en recht;~\sep\ uw rechterhand lere u roemrijke daden.

  Uw pijlen zijn scherp; volkeren worden aan u onderworpen;~\sep\ aan de vijanden van de Koning ontzinkt de moed.

  In de eeuwen der eeuwen staat Uw troon, O God,~\sep\ een scepter van recht is de scepter van Uw rijk.

  Gij hebt de gerechtigheid lief en haat de boosheid; daarom heeft God, uw God, u gezalfd,~\sep\ met de olie der vreugde boven uw genoten.

  Van mirre en aloë en cassia geuren uw gewaden;~\sep\ uit ivoren paleizen klinkt u blij het harpgeluid tegen.

  Koningsdochters treden u tegemoet,~\sep\ de Koningin staat aan uw rechterhand, met goud uit Ofir getooid.

  Hoor, dochter, en zie, en neig uw oor,~\sep\ en vergeet uw volk en het huis van uw vader.

  Dan zal aan de Koning uw schoonheid behagen;~\sep\ Hij is uw Heer, breng Hem uw hulde.

  Dan komt met geschenken het volk van Tyrus,~\sep\ de voornamen onder het volk dingen om uw gunst.

  In volle luister treedt de dochter van de Koning binnen;~\sep\ met goud doorweven is haar gewaad.

  In een kleurige mantel wordt zij voor de Koning geleid,~\sep\ in haar gevolg worden maagden, haar
  gezellinnen, tot u gevoerd;

  Zij worden voorgeleid in blijde jubel,~\sep\ en treden het paleis van de Koning binnen.

  In de plaats van uw vaderen komen uw zonen:~\sep\ gij zult hen aanstellen tot vorsten over heel de wereld.

  Ik zal uw naam doen gedenken bij alle geslachten;~\sep\ daarom zullen de volken u prijzen in de eeuwen der eeuwen.

  \psalm{\Ps{46}} God is ons een Toevlucht en Kracht;~\sep\ een machtige Helper toonde Hij zich in de nood.

  Daarom vrezen wij niet, al wordt ook de aarde geschokt,~\sep\ al storten de bergen midden in zee.

  Laat bruisen en koken haar wateren,~\sep\ laat schudden de bergen door haar geweld:

  De Heer der heerscharen is met ons,~\sep\ de God van Jacob is onze bescherming.

  De armen van de stroom verblijden de stad van God,~\sep\ de heilige woontent van de Allerhoogste.

  God woont daarbinnen, zij zal niet wankelen;~\sep\ God staat haar bij van de vroege dageraad af.

  Volkeren woedden, en koninkrijken werden geschokt:~\sep\ daar galmde Zijn donderstem, en weg vloeide de aarde.

  De Heer der heerscharen is met ons,~\sep\ de God van Jacob is onze bescherming.

  Komt en aanschouwt de werken van de Heer,~\sep\ de wonderen, die Hij op aarde gewrocht heeft.

  Hij bedwingt de oorlogen tot aan het einde der aarde,~\sep\ Hij verbrijzelt de bogen, breekt lansen stuk, in het vuur verbrandt Hij de schilden.

  Houdt op, en erkent Mij als God,~\sep\ verheven onder de volken, verheven op aarde.

  De Heer der heerscharen is met ons,~\sep\ de God van Jacob is onze bescherming.
\end{halfparskip}

\begin{halfparskip}
  \markedsubsectionrubricwithhint{Donderdagen ``voor'': Marmita 25*}{(origineel: hulala 10)}

  \psalm{\Ps{68}} God rijst op: Zijn vijanden stuiven uiteen,~\sep\ en die Hem haten, vluchten weg voor Zijn aanschijn.

  \liturgicalhint{Alleluia, Alleluia, Alleluia.~--- Eerste vers.}

  Zij verdwijnen, zoals rook verdwijnt;~\sep\ zoals was wegsmelt bij vuur, zo vergaan de zondaars voor Gods
  aanschijn.

  Maar de rechtvaardigen juichen, springen op voor het aanschijn van God,~\sep\ zijn opgetogen van vreugde.

  Zingt God toe, tokkelt het psalter voor Zijn Naam;~\sep\ baant een weg voor Hem, die voorttrekt door de woestijn,

  Wiens naam is: de Heer,~\sep\ en juicht voor Zijn aanschijn!

  Vader der wezen en Beschermer der weduwen,~\sep\ is God in Zijn heilige woonstede.

  Voor verlatenen bereidt God een woning, gevangenen brengt Hij tot voorspoed;~\sep\ slechts de weerspannigen blijven achter in het verdorde land.

  Bij Uw uittocht, O God, aan het hoofd van Uw volk,~\sep\ bij Uw optrekken door de woestijn,

  Beefde de aarde en dropen de hemelen voor het aanschijn van God;~\sep\ de Sinaï sidderde voor God, de God van Israël.

  Milde regen hebt Gij, O God, over Uw erfdeel uitgestort;~\sep\ en als het was uitgeput, hebt Gij het verkwikt.

  Uw kudde heeft er gewoond;~\sep\ in Uw goedheid, O God, hebt Gij het de arme bereid.

  De Heer doet een uitspraak:~\sep\ en groot is de menigte, die blijde dingen meldt:

  ``De vorsten der legerscharen vluchten, vluchten:~\sep\ en de huisgenoten verdelen de buit.

  Terwijl gij rustte in de stallen der kudden, schitterden van zilver de vleugels der duif,~\sep\ en haar pennen in goudgele glans.

  Terwijl er de Almachtige de vorsten verstrooide,~\sep\ viel er sneeuw op de Salmon.''

  Hoge bergen zijn de bergen van Basan,~\sep\ steile bergen zijn de bergen van Basan.

  Waarom ziet gij afgunstig, gij, steile bergen, naar de berg, waar het God behaagd heeft te wonen,~\sep\ ja, waar de Heer zelfs voor immer zal wonen?

  De strijdwagens van God zijn myriaden in aantal, duizendmaal duizend;~\sep\ van de Sinaï af trekt de Heer naar Zijn heiligdom.

  Gij hebt de hoogte bestegen, gevangenen meegevoerd, mensen als gaven ontvangen,~\sep\ zelfs hen, die weigeren bij God de Heer te wonen.

  Geprezen zij de Heer, dag aan dag;~\sep\ onze lasten draagt God, ons heil!

  Onze God is een God, die redding brengt,~\sep\ en de Heer God schenkt uitkomst in doodsgevaar.

  Waarlijk, God verbrijzelt de hoofden van Zijn vijanden,~\sep\ de ruige schedel van hem, die de weg der misdaad bewandelt.

  De Heer heeft gesproken: ``Ik haal ze terug uit Basan,~\sep\ ik haal ze op uit de diepte der zee,

  Opdat ge uw voet moogt dopen in bloed,~\sep\ en de tong van uw honden haar deel aan de vijanden heeft!''

  Zij zien Uw intrede, O God,~\sep\ de intrede van mijn God, mijn Koning, in het heiligdom:

  Voorop gaan de zangers, de harpspelers aan het einde,~\sep\ in het midden slaan de meisjes de pauken.

  ``Looft God in uw feestelijke bijeenkomsten,~\sep\ de Heer, gij, die uit Israël geboren zijt!''

  Daar treedt Benjamin aan, de jongste; hij gaat voor hen uit; dan volgen de vorsten van Juda met hun scharen,~\sep\ de vorsten van Zabulon, de vorsten van Neftali.

  Ontplooi Uw macht, O God,~\sep\ Uw macht, O God, Gij, die werkt voor ons!

  Omwille van Uw tempel, die in Jeruzalem staat,~\sep\ mogen de vorsten U gaven brengen!

  Bedwing het wilde beest in het riet,~\sep\ de troep stieren met de runderen der volken;

  Dat zij zich ter aarde werpen met platen van zilver:~\sep\ verstrooi de volken, die zich verlustigen in oorlog.

  Laat de machtigen uit Egypte zich aandienen,~\sep\ dat Ethiopië de handen uitstrekke naar God.

  Koninkrijken der aarde, zingt voor God, bespeelt het psalter voor de Heer,~\sep\ die door de hemelen, de aloude hemelen rijdt!

  Zie, Zijn stem laat Hij horen, Zijn machtige stem:~\sep\ ``Erkent de macht van God'';

  Zijn Majesteit strekt zich uit over Israël,~\sep\ en in de wolken straalt Zijn macht.

  Ontzagwekkend is God in Zijn heiligdom, Hij, de God van Israël. Hij zelf schenkt macht en kracht aan Zijn volk.~\sep\ Geprezen zij God!
\end{halfparskip}

\begin{halfparskip}
  \markedsubsectionrubricwithhint{Vrijdagen ``voor'': Marmita 41*}{(origineel: hulala 16)}

  \psalm{\Ps{104}} Loof, mijn ziel, de Heer!~\sep\ Heer, mijn God, Gij zijt ontzaglijk groot;

  \liturgicalhint{Alleluia, Alleluia, Alleluia.~--- Eerste vers.}

  Met majesteit en luister zijt Gij omkleed,~\sep\ met licht omhuld als met een mantel.

  Als een tentdoek hebt Gij de hemelen uitgespreid,~\sep\ op de wateren Uw opperzalen gebouwd.

  Gij maakt de wolken tot Uw wagen,~\sep\ en zweeft op de wieken der winden.

  Tot Uw boden maakt Gij de winden,~\sep\ tot Uw knechten het verzengende vuur.

  Gij hebt de aarde op haar pijlers gegrondvest;~\sep\ in de eeuwen der eeuwen zal zij niet wankelen.

  Met de oceaan als kleed hebt Gij haar bedekt;~\sep\ boven de bergen stonden de wateren.

  Bij Uw dreigen vluchtten zij heen,~\sep\ ze sidderden voor de stem van Uw donder.

  Bergen rezen en dalen zonken,~\sep\ op de plaats, die Gij hun hadt bestemd.

  Gij hebt een grens gesteld, die zij niet overschrijden mogen,~\sep\ opdat ze niet opnieuw de aarde bedekken.

  Gij beveelt de bronnen in de beken te vloeien,~\sep\ die tussen de bergen stromen;

  Zij laven alle dieren van het veld;~\sep\ de woudezels lessen hun dorst.

  Daar nestelen de vogels van de hemel,~\sep\ en doen er hun lied in de takken weerklinken.

  Vanuit Uw zalen besproeit Gij de bergen,~\sep\ met de vrucht van Uw werken wordt de aarde verzadigd.

  Gij laat voor het vee het gras ontspruiten,~\sep\ en het kruid tot nut van de mens,

  Opdat hij uit de aarde brood zal verwekken,~\sep\ en wijn, die het hart van de mens verblijdt;

  Opdat hij met olie het gelaat zou doen glanzen,~\sep\ en met brood zou verkwikken het hart van de mens.

  De bomen van de Heer worden verzadigd,~\sep\ de ceders van de Libanon, die door Hem zijn geplant.

  Daar bouwen de vogels hun nest;~\sep\ tot verblijf van de ooievaar dienen de dennen.

  De hoogste bergen zijn voor de gemzen,~\sep\ de rotsen bieden de egels een schuilplaats.

  Gij hebt de maan geschapen om de tijd te bepalen,~\sep\ de zon kent het uur van haar ondergang.

  \emph{Wij aanbidden U,~\sep\ Vader, Zoon en Heilige Geest.}

  Maakt Gij het donker en wordt het nacht,~\sep\ dan zwerven erin alle dieren van het woud.

  De leeuwenwelpen brullen om prooi,~\sep\ en vragen aan God om voedsel.

  En rijst dan de zon, dan sluipen ze weg,~\sep\ en leggen zich neer in hun holen.

  De mens gaat uit naar zijn werk,~\sep\ naar zijn arbeid tot de avond toe.
  Hoe talrijk zijn Uw werken, O Heer! Gij hebt alles met wijsheid geschapen,~\sep\ van Uw schepselen is de aarde vervuld.

  Zie de zee, groot en wijd naar elke einder:~\sep\ daar wemelt het van talloze vissen, van dieren, klein en groot.

  Daar trekken de schepen doorheen,~\sep\ Leviathan, door U geschapen, om er te spelen.

  Allen zien naar U uit,~\sep\ dat Gij op tijd hun voedsel schenkt.

  Geeft Gij het hun: ze verzamelen,~\sep\ opent Gij Uw hand: ze worden met het goede verzadigd.

  Verbergt Gij Uw aanschijn: ze worden ontsteld;~\sep\ ontneemt Gij hun de adem: ze sterven en keren terug tot hun stof.

  Zendt Gij Uw geest uit: zij worden geschapen,~\sep\ en Gij vernieuwt het aanschijn der aarde.

  Eeuwig dure de glorie van de Heer;~\sep\ de Heer verheuge zich over Zijn werken.

  Hij blikt naar de aarde: zij beeft;~\sep\ Hij raakt de bergen aan: zij roken.

  Ik wil zingen voor de Heer, mijn leven lang,~\sep\ voor mijn God de snaren tokkelen, zolang ik besta.

  Moge mijn zang Hem behagen:~\sep\ ik zal in de Heer mij verblijden.

  Dat de zondaars van de aarde verdwijnen, en de goddelozen niet langer bestaan!~\sep\ Zegen, mijn ziel, de Heer!
\end{halfparskip}

\begin{halfparskip}
  \markedsubsectionrubricwithhint{Zaterdagen ``voor'': Marmita 47*}{(origineel: hulala 19)}

  \psalm{\Ps{115}} De Heer heb ik lief, omdat Hij hoorde,~\sep\ naar de stem van mijn smeken,

  Omdat Hij Zijn oor neigde tot mij,~\sep\ op welke dag ik ook tot Hem riep.

  \liturgicalhint{Alleluia, Alleluia, Alleluia.~--- Eerste vers.}

  Mij omstrengelden strikken van de dood en boeien van het dodenrijk omknelden mij;~\sep\ kommer en smart werden mijn deel.

  Maar de Naam van de Heer riep ik aan:~\sep\ ``Ach, Heer, red toch mijn leven !''

  De Heer is goed en rechtvaardig,~\sep\ ja, meedogend is onze God.

  De Heer behoedt de eenvoudigen:~\sep\ ik was ellendig en Hij heeft mij gered.

  Keer terug, mijn ziel, tot Uw rust,~\sep\ want de Heer heeft u welgedaan.

  Want Hij heeft mij ontrukt aan de dood,~\sep\ mijn ogen behoed voor tranen, mijn voeten voor de val.

  Ik zal wandelen voor de Heer,~\sep\ in het land der levenden.

  Ik bleef vertrouwen, ook toen ik sprak:~\sep\ ``Ik ben diep bedroefd.''

  In mijn verslagenheid heb ik gezegd:~\sep\ ``Er is geen mens te vertrouwen.''

  Wat zal ik de Heer weergeven,~\sep\ voor alles, wat Hij mij schonk?

  De kelk van het heil zal ik heffen,~\sep\ en aanroepen de Naam van de Heer.

  Ik zal de Heer mijn geloften volbrengen,~\sep\ in het bijzijn van geheel Zijn volk.

  Kostbaar in de ogen van de Heer,~\sep\ is de dood van Zijn heiligen.

  Ik ben Uw dienstknecht, Heer, Uw dienstknecht, een zoon van Uw dienstmaagd,~\sep\ Gij hebt mijn boeien verbroken.

  Een dankoffer zal ik U brengen,~\sep\ en aanroepen de Naam van de Heer.

  Ik zal de Heer mijn geloften volbrengen,~\sep\ in het bijzijn van geheel Zijn volk,

  In de voorhoven van het huis van de Heer,~\sep\ in uw midden, Jeruzalem.

  \psalm{\Ps{116}} Looft de Heer, alle volkeren,~\sep\ alle naties, verheerlijkt Hem.

  Want Zijn erbarming blijft ons verzekerd,~\sep\ en de trouw van de Heer duurt eeuwig.

  \psalm{\Ps{117}} Dankt de Heer, want Hij is goed,~\sep\ want eeuwig duurt Zijn barmhartigheid.

  Het huis van Israël zegge:~\sep\ ``Eeuwig duurt Zijn barmhartigheid.''

  Het huis van Aäron zegge:~\sep\ ``Eeuwig duurt Zijn barmhartigheid?''

  Die de Heer vrezen, zeggen:~\sep\ ``Eeuwig duurt Zijn barmhartigheid.''

  In kwelling riep ik tot de Heer;~\sep\ de Heer heeft mij verhoord en bevrijd.

  De Heer is met mij: ik ben niet bevreesd;~\sep\ wat kan een mens mij aandoen?

  De Heer is met mij, Hij is mijn Helper;~\sep\ ik zal mijn vijanden beschaamd zien staan.

  Het is beter tot de Heer te vluchten,~\sep\ dan op een mens te vertrouwen;

  Het is beter tot de Heer te vluchten,~\sep\ dan te vertrouwen op vorsten.

  Alle volken omsingelden mij;~\sep\ in de Naam van de Heer sloeg ik ze neer.

  Van alle zijden omringden zij mij;~\sep\ in de Naam van de Heer sloeg ik ze neer.

  Ze zwermden als bijen om mij heen, zij brandden mij weg als het vuur de doornen;~\sep\ in de Naam van de Heer sloeg ik ze neer.

  Zij stieten mij hevig, opdat ik zou vallen,~\sep\ maar de Heer kwam mij te hulp.

  Mijn kracht en mijn sterkte is de Heer,~\sep\ Hij is mij tot Redder geworden.

  Er klinkt een juich- en zegekreet~\sep\ in de tenten der rechtvaardigen.

  De rechterhand van de Heer ontplooide haar kracht, de rechterhand van de Heer hief mij op,~\sep\ de rechterhand van de Heer ontplooide haar kracht.

  Niet sterven zal ik, maar leven,~\sep\ en de werken van de Heer verkondigen.

  Wel heeft de Heer mij streng gekastijd,~\sep\ maar Hij gaf mij niet prijs aan de dood.

  Ontsluit mij de poorten der gerechtigheid:~\sep\ door hen wil ik ingaan om de Heer te gaan danken.

  Dit is de poort van de Heer,~\sep\ de rechtvaardigen zullen er door binnentreden.

  Ik zal U danken, daar Gij mij verhoord hebt,~\sep\ en mij tot Redder geworden zijt.

  De steen, die de bouwlieden hebben verworpen,~\sep\ is tot een hoeksteen geworden.

  Dit heeft de Heer gedaan,~\sep\ het is wonderbaar in onze ogen.

  Dit is de dag, die de Heer heeft gemaakt:~\sep\ laat ons daarover juichen en blijde zijn.

  Schenk redding, O Heer;~\sep\ O Heer, geef voorspoed!

  Gezegend die komt in de Naam van de Heer, wij zegenen u uit het huis van de Heer.~\sep\ God is de Heer, Hij heeft ons verlicht.

  Spreidt praal ten toon met weelderig lover,~\sep\ tot bij de hoornen van het altaar.

  Mijn God zijt Gij, U breng ik dank;~\sep\ mijn God, hoog wil ik U prijzen.

  Dankt de Heer, want Hij is goed,~\sep\ eeuwig duurt Zijn barmhartigheid.
\end{halfparskip}

\begin{halfparskip}
  \markedsubsectionrubricwithhint{Maandagen ``na'': Marmita 6*}{(origineel: hulala 1)}

  \psalm{\Ps{18}} Ik heb U lief, o Heer, mijn Sterkte,~\sep\ Heer, mijn Rots, mijn Burcht, mijn Bevrijder;

  \liturgicalhint{Alleluia, Alleluia, Alleluia.~--- Eerste vers.}

  Mijn God, mijn Rotswand, waarheen ik vlucht,~\sep\ mijn Schild, de Hoorn van mijn heil, mijn Toeverlaat.

  Aanroepen zal ik de Heer, de Lofwaardige,~\sep\ en van mijn vijanden worden verlost.

  Mij omspoelden de golven van de dood,~\sep\ en vernietigende stromen ontstelden mij.

  De strikken van het dodenrijk omknelden mij,~\sep\ de boeien van de dood vielen op mij neer.

  In mijn nood riep ik tot de Heer,~\sep\ en mijn geschrei steeg op tot mijn God;

  En Hij hoorde mijn stem vanuit Zijn tempel,~\sep\ en mijn hulpgeroep drong door tot Zijn oren.

  Daar schudde de aarde en beefde; de grondvesten der bergen werden geschokt,~\sep\ en zij dreunden want Hij brandde van toorn.

  Rook steeg uit Zijn neusgaten op, verslindend vuur uit Zijn mond,~\sep\ gloeiende kolen sprongen van Hem uit.

  Hij haalde de wolkenhemel neer en daalde af,~\sep\ en zwarte wolken hingen onder Zijn voeten.

  Hij voer op de cherub en vloog,~\sep\ op de wieken van de wind werd Hij gedragen.

  Hij omhulde zich met duisternis als met een kleed,~\sep\ met donkere nevels en dichte wolken als met een mantel.

  Door de gloed vóór Hem uit,~\sep\ ontbrandden gloeiende kolen.

  En de Heer deed de donder rollen uit de hemel,~\sep\ en weergalmen deed de Allerhoogste Zijn stem.

  En Hij schoot Zijn pijlen af en dreef hen uiteen,~\sep\ talloze flitsen, en Hij velde hen neer.

  En de bodem der zee kwam te voorschijn,~\sep\ en het fundament der aarde lag bloot.

  Door het dreigen van de Heer,~\sep\ door de ademtocht van Zijn toorn.

  Hij strekte Zijn hand uit de hoge, Hij greep mij aan,~\sep\ en trok mij op uit de watervloed.

  Hij bevrijdde mij van mijn geweldige vijand,~\sep\ en van hen, die mij haatten, die machtiger waren dan ik.

  Zij overvielen mij op de dag van mijn rampspoed,~\sep\ maar tot bescherming was mij de Heer.

  En Hij leidde mij uit in het vrije veld;~\sep\ Hij heeft mij gered, omdat Hij mij liefheeft.

  Zo loonde mij de Heer naar mijn gerechtigheid;~\sep\ naar de reinheid van mijn handen vergold Hij mij.

  Want de wegen van de Heer heb ik gevolgd,~\sep\ door geen zonde ben ik afgeweken van mijn God.

  Ja, al Zijn geboden hield ik voor ogen,~\sep\ en Zijn wetten wierp ik niet van mij af.

  Maar voor Zijn aanschijn was ik rein,~\sep\ en ik heb mij behoed voor de zonde.

  Zo vergold mij de Heer naar mijn gerechtigheid,~\sep\ naar de reinheid van mijn handen voor Zijn ogen.

  Met de vrome handelt Gij liefdevol,~\sep\ met de rechtschapene rechtschapen;

  Voor de reine toont Gij U rein,~\sep\ met de sluwe handelt Gij slim.

  Want Gij redt het nederige volk,~\sep\ maar trotse blikken slaat Gij neer.

  Ja, Gij doet mijn lamp schijnen, O Heer;~\sep\ mijn God, mijn duisternis maakt Gij tot licht.

  Ja, met U storm ik los op de drommen der vijanden,~\sep\ en met mijn God bespring ik de wallen.

  Gods wegen zijn volmaakt, het woord van de Heer is door het vuur gelouterd;~\sep\ Hij is een schild voor allen, die vluchten tot Hem.

  Wie is God buiten de Heer,~\sep\ of wie een rots buiten onze God?

  God, die mij met kracht heeft omgord,~\sep\ en mij een veilige weg heeft gebaand;

  Die aan mijn voeten de snelheid der hinden gaf,~\sep\ en mij plaatste op de hoogten,

  Die mijn handen oefende tot de strijd,~\sep\ en tot het spannen van de koperen boog mijn armen.

  Gij schonkt mij Uw schild, dat redding brengt, en Uw rechterhand heeft mij staande gehouden,~\sep\ en Uw zorgzame liefde maakte mij groot.

  Gij hebt de weg voor mijn schreden verbreed,~\sep\ en mijn voeten wankelden niet.

  Ik zette mijn vijanden na, en greep ze aan,~\sep\ en ik keerde niet terug, eer ik ze had vernietigd.

  Ik heb ze verpletterd en opstaan konden ze niet,~\sep\ ze bleven liggen onder mijn voeten.

  Ja, Gij hebt mij met kracht omgord tot de strijd;~\sep\ en die mij weerstaan, hebt Gij voor mij doen bukken.

  Gij hebt mijn vijanden op de vlucht gedreven,~\sep\ en die mij haten, hebt Gij verdelgd.

  Zij schreeuwden het uit~--- maar niemand schonk redding -~\sep\ tot de Heer, maar Hij verhoorde hen niet.

  En ik heb ze vergruisd als stof voor de wind,~\sep\ vertrapt als slijk in de straten.

  Gij hebt mij ontrukt aan het muitende volk,~\sep\ mij gesteld aan het hoofd van de naties.

  Een volk, dat mij vreemd was, werd mij dienstbaar;~\sep\ nauwelijks hoorde het van mij, of het was mij onderdanig.

  Vreemden brachten mij vleiend hulde,~\sep\ vreemden, geslagen met schrik, kropen sidderend uit hun burchten.

  Leve de Heer, mijn Rots zij gezegend;~\sep\ hooggeprezen zij God, mijn Redder!

  God, die mij de wraak in handen gaf,~\sep\ en mij de volkeren onderwierp,

  Gij, die mij van mijn vijanden hebt bevrijd, en mij verheven hebt boven mijn weerstrevers,~\sep\ mij hebt ontrukt aan de geweldenaar.

  Daarom zal ik U prijzen onder de volken, O Heer,~\sep\ en verheerlijken Uw Naam.

  Gij hebt Uw koning een schitterende zege verleend,~\sep\ en barmhartigheid bewezen aan Uw gezalfde, aan David en zijn geslacht voor eeuwig.
\end{halfparskip}

\begin{halfparskip}
  \markedsubsectionrubricwithhint{Dinsdagen ``na'': Marmita 14*}{(origineel: hulala 4)}

  \psalm{\Ps{37}} Ontbrand niet in toorn vanwege de zondaars,~\sep\ en benijd de boosdoeners niet.

  \liturgicalhint{Alleluia, Alleluia, Alleluia.~--- Eerste vers.}

  Spoedig toch vallen zij neer als hooi,~\sep\ en verwelken als het groene gras.

  Vertrouw op de Heer, en doe het goede,~\sep\ om het Land te bewonen en een veilig bestaan te genieten.

  Stel uw vreugde in de Heer,~\sep\ en Hij zal u schenken wat uw hart maar begeert.

  Vertrouw de Heer uw levensweg toe,~\sep\ en hoop op Hem, Hij zal wel zorgen.

  Hij zal als het licht uw gerechtigheid doen opgaan,~\sep\ en als de middagzon uw recht.

  Verlaat u op de Heer,~\sep\ en stel uw hoop op Hem.

  Vertoorn u niet op hem, die het wel gaat in het leven,~\sep\ op de mens, die het kwade beraamt.

  Leg uw verbolgenheid af en laat varen uw gramschap,~\sep\ ontbrand niet in toorn om geen kwaad te bedrijven.

  Immers: de boosdoeners worden te gronde gericht,~\sep\ maar die hopen op de Heer, zullen het land bezitten.

  Nog een weinig tijds, en weg is de boze;~\sep\ en zoekt ge zijn plaats, hij is er niet meer.

  Maar de zachtmoedigen zullen het Land bezitten,~\sep\ en een overvloedige vrede genieten.

  De goddeloze zint op onheil tegen de rechtvaardige,~\sep\ en knarst tegen hem met de tanden;

  De Heer spot met hem,~\sep\ omdat Hij zijn dag ziet naderen.

  De bozen trekken het zwaard en spannen de boog, om de ellendige en arme neer te vellen,~\sep\ te doden die gaan langs de rechte weg.

  Maar hun zwaard zal hun eigen hart doorboren,~\sep\ en hun bogen zullen worden gebroken.

  Beter het schamele, dat de gerechte bezit,~\sep\ dan de grote rijkdom der bozen.

  Want de armen der bozen zullen worden gebroken,~\sep\ maar de Heer is een steun voor de rechtvaardigen.

  De Heer draagt zorg voor het leven der vromen,~\sep\ en hun erfdeel blijft eeuwig bestaan.

  Zij zullen bij rampspoed niet worden beschaamd,~\sep\ maar verzadigd worden bij hongersnood.

  Maar de goddelozen zullen vergaan, en de vijanden van de Heer als de tooi der weiden verwelken:~\sep\ ze zullen vervliegen als rook.

  De goddeloze leent en geeft niet terug,~\sep\ maar de gerechte is genadig en geeft.

  Want die Hij zegent, zullen het Land bezitten,~\sep\ maar die Hij vloekt, zullen vergaan.

  Door de Heer worden de schreden van de mens ondersteund;~\sep\ en in zijn wandel schept Hij behagen.

  Mocht hij vallen, hij valt niet languit,~\sep\ want de Heer houdt hem vast bij de hand,

  Eens was ik een kind en nu ben ik een grijsaard, maar nooit heb ik een rechtvaardige verlaten gezien,~\sep\ noch zag ik zijn kinderen bedelen om brood.

  Steeds is hij meedogend en geeft hij te leen,~\sep\ en zijn nakroost zal gezegend worden.

  Houd u af van het kwaad en doe het goede,~\sep\ opdat gij voor eeuwig moogt leven.

  Want de Heer heeft de gerechtigheid lief,~\sep\ en Hij verlaat Zijn heiligen niet.

  De bozen worden vernietigd,~\sep\ het geslacht der goddelozen verdelgd.

  De rechtvaardigen zullen het Land bezitten,~\sep\ en zullen daar wonen voor immer.

  De mond van de rechtvaardige spreekt wijsheid,~\sep\ en wat recht is, verkondigt zijn tong.

  Hij draagt de Wet van zijn God in zijn hart,~\sep\ en zijn schreden wankelen niet.

  De goddeloze bespiedt de rechtvaardige,~\sep\ en zoekt hem te doden.

  De Heer laat hem niet in zijn macht,~\sep\ en veroordeelt hem niet in het gericht.

  Vertrouw op de Heer,~\sep\ en bewandel Zijn weg;

  Dan helpt Hij u voort om het Land te bezitten,~\sep\ en gij zult vol vreugde de verdelging der bozen aanschouwen.

  Ik heb de goddeloze gezien in zijn trots:~\sep\ hij breidde zich uit als een bladerrijke ceder.

  En ik ging voorbij, doch zie, hij was er niet meer,~\sep\ ik zocht naar hem, maar hij was niet te vinden.

  Let op de vrome en beschouw de rechtvaardige,~\sep\ want een vreedzaam man heeft een nageslacht.

  Maar de zondaars gaan allen te gronde,~\sep\ het nakroost der goddelozen zal worden verdelgd.

  Het heil der rechtvaardigen komt van de Heer;~\sep\ hun Toevlucht is Hij ten tijde van rampspoed.

  De Heer is hun Helper, Hij schenkt hun bevrijding,~\sep\ Hij bevrijdt hen van bozen en brengt hun de redding, daar zij hun toevlucht nemen tot Hem.
\end{halfparskip}

\begin{halfparskip}
  \markedsubsectionrubricwithhint{Woensdagen ``na'': Marmita 20*}{(origineel: hulala 7)}

  \psalm{\Ps{53}} De dwaas zegt bij zich zelf:~\sep\ ``Er is geen God.''~\sep\ Ze zijn bedorven, gruwelen hebben ze bedreven;~\sep

  \liturgicalhint{Alleluia, Alleluia, Alleluia.~--- Eerste vers.}

  daar is er niet één, die deugdzaam handelt.

  God blikt uit de hemel neer op de kinderen der mensen,~\sep\ om te zien of er wel één is met verstand, wel één, die God zoekt.

  Maar allen zonder uitzondering zijn ze afgedwaald, allen diep bedorven;~\sep\ er is er niet één die deugdzaam handelt, niet één.

  Zullen die bozen dan nimmer tot inzicht komen, zij die Mijn volk verslinden als aten ze brood;~\sep\ roepen zij dan God niet aan?

  Zij sidderden van angst,~\sep\ waar niets viel te vrezen;

  Want God heeft de beenderen verstrooid van die u bestookten;~\sep\ ze werden ontsteld, want God heeft hen verworpen.

  O, mocht er uit Sion toch heil voor Israël dagen! Als God het lot van Zijn volk ten goede keert,~\sep\ zal er gejubel zijn in Jacob en vreugde in Israël!

  \psalm{\Ps{54}} Red mij, o God, door Uw Naam,~\sep\ en treed in Uw kracht voor mijn rechtszaak op!

  Luister naar mijn bede, O God,~\sep\ hoor naar de woorden van mijn mond!

  Want trotsaards zijn tegen mij opgestaan, en geweldenaars stonden mij naar het leven;~\sep\ zij hielden God niet voor ogen.

  Zie, God komt mij te hulp,~\sep\ de Heer behoudt mijn leven.

  Wend op mijn vijanden de rampen af,~\sep\ vernietig hen omwille van Uw trouw.

  Van harte wil ik U offers brengen;~\sep\ Uw Naam zal ik prijzen, O Heer, want Hij is goed.

  Want Hij heeft mij verlost uit alle nood,~\sep\ en mijn oog zag mijn vijanden beschaamd staan

  \psalm{\Ps{55}} Luister, O God, naar mijn bede, en wend U niet af van mijn smeken;~\sep\ geef acht op mij en schenk mij verhoring.

  Ik sidder van angst en word ontsteld~\sep\ bij het razen van de vijand, en het geschreeuw van de zondaar.

  Want zij storten rampen over mij uit,~\sep\ en vallen mij grimmig aan.

  Mijn hart is ontsteld in mijn binnenste,~\sep\ en de angst van de dood overvalt mij.

  Vrees en siddering storten op mij neer,~\sep\ en ontzetting grijpt mij aan.

  En ik zeg: Had ik maar vleugelen als de duif,~\sep\ dan vloog ik heen en vond ik rust.

  Ja, ver, ver zou ik heenvluchten,~\sep\ verwijlen in de woestijn.

  Spoedig zou Ik mij een schuilplaats zoeken,~\sep\ tegen stormwind en orkaan.

  Verstrooi ze, Heer, verwar hun spraak,~\sep\ want in de stad zie ik geweld en twist.

  Dag en nacht doen zij de ronde op haar wallen,~\sep\ en daarbinnen heerst boosheid en verdrukking.

  Hinderlagen legt men er,~\sep\ en onrecht en bedrog wijken niet van haar straten.

  Och, was het slechts mijn vijand, die mij had gehoond,~\sep\ voorzeker, ik had het verdragen;

  Of was, die mij haatte, tegen mij opgestaan,~\sep\ ik had mij voor hem verborgen.

  Maar gij waart het, mijn disgenoot,~\sep\ mijn vriend en vertrouweling,

  Met wie ik zo gemoedelijk omging;~\sep\ in het huis van God trokken we samen met de feeststoet op.

  Dat de dood op hen aanstorme, dat zij levend neerdalen in het dodenrijk,~\sep\ want boosheid woont in hun huizen en in hun binnenste!

  Ik echter zal roepen tot God,~\sep\ en de Heer zal mij redden.

  `s~Avonds en `s~morgens en `s~middags zal ik weeklagen en jammeren,~\sep\ en mijn stem zal Hij horen.

  Mijn leven zal Hij in veiligheid brengen tegen hen, die mij bestoken:~\sep\ want velen staan tegen mij op.

  God zal het horen, en Die van eeuwigheid regeert, hen neerdrukken;~\sep\ want onverbeterlijk zijn ze en ze vrezen God niet.

  Ieder heft zijn hand op tegen zijn vrienden,~\sep\ en schendt zijn verbond.

  Gladder dan boter is hun gelaat,~\sep\ maar hun hart wil strijd;

  Zachter dan olie zijn hun woorden,~\sep\ maar het zijn getrokken zwaarden.

  Werp Uw zorg op de Heer, en Hij zal Uw steun zijn:~\sep\ nooit zal Hij dulden dat de rechtvaardige wankelt.

  Maar hen, O God, zult Gij neerstorten,~\sep\ in de afgrond van verderf.

  Die bloed vergieten en bedriegen, zullen de helft van hun dagen niet halen;~\sep\ ik echter hoop op U, O Heer.
\end{halfparskip}

\begin{halfparskip}
  \markedsubsectionrubricwithhint{Donderdagen ``na'': Marmita 28*}{(origineel: hulala 10)}

  \psalm{\Ps{73}} Hoe goed is Israëls God voor de rechtvaardigen,~\sep\ de Heer voor die rein zijn van hart!~\sep\ Toch wankelden bijna mijn voeten,~\sep

  \liturgicalhint{Alleluia, Alleluia, Alleluia.~--- Eerste vers.}

  haast gleden mijn schreden uit.

  Daar ik de goddelozen benijdde,~\sep\ toen ik de voorspoed der zondaren zag.

  Want kwellingen kennen zij niet,~\sep\ gezond en gezet is hun lichaam.

  De zorgen der stervelingen delen zij niet,~\sep\ en zij ontkomen de gesels der mensen.

  Daarom omsluit hen de trots als een halssnoer,~\sep\ en bedekt hen geweld als een kleed.

  De misdaad puilt uit hun zinnelijk hart,~\sep\ de verzinsels van hun geest dringen door naar buiten.

  Zij spotten en lasteren,~\sep\ zij dreigen op hoge toon met geweld,

  Zij zetten een mond op tegen de hemel,~\sep\ en hun tongen striemen de aarde.

  Daarom loopt mijn volk achter hen aan,~\sep\ en slurpen zij water in overvloed.

  En zij zeggen ``Hoe zou God het weten,~\sep\ en zou de Allerhoogste er kennis van dragen?''

  Zie, zo zijn de zondaars,~\sep\ en, steeds ongestoord, vermeerderen zij hun macht.

  Heb ik dan vergeefs mijn hart in reinheid bewaard,~\sep\ en mijn handen in onschuld gewassen?

  Want almaar door word ik gegeseld,~\sep\ en iedere dag gekastijd.

  Had ik gedacht: Laat mij spreken als zij,~\sep\ dan had ik de aard van Uw kinderen verloochend.

  Ik dacht dus na om het te vatten,~\sep\ maar het leek mij een moeilijke zaak,

  Totdat ik binnentrad in Gods heiligdom,~\sep\ en op hun einde ging letten.

  Waarlijk, Gij zet hen op een glibberig pad,~\sep\ stort hen neer in het verderf.

  Hoe zijn ze in een oogwenk ineengestort,~\sep\ verdwenen, in schrikkelijke angst vergaan!

  Als een droombeeld, O Heer, bij hem, die ontwaakt,~\sep\ zo zult Gij hun beeld, als Gij oprijst, versmaden.

  Toen mijn geest verbitterd was,~\sep\ en mijn hart werd geprikkeld,

  Was ik een dwaas zonder enig begrip,~\sep\ als een stuk vee voor Uw aanschijn.

  Maar ik zal immer bij U zijn:~\sep\ Gij houdt mij vast aan mijn rechterhand.

  Met Uw raad zult Gij mij leiden,~\sep\ en mij opnemen, eens, in de glorie.

  Wie bezit ik in de hemel buiten U;~\sep\ en ben ik bij U, dan geeft mij de aarde geen vreugde.

  Mijn lichaam bezwijkt en mijn hart,~\sep\ de Rots van mijn hart en mijn aandeel voor eeuwig is God.

  Want zie, die U verlaten, zullen vergaan;~\sep\ die U afvallig worden, verdelgt Gij allen.

  Maar mij is het goed bij God te zijn,~\sep\ mijn toevlucht te nemen bij God, de Heer.

  Al Uw werken zal ik verhalen,~\sep\ in de poorten der dochter Sion.

  \psalm{\Ps{74}} Waarom, God, hebt Gij ons voor eeuwig verstoten,~\sep\ ontbrandt Uw toorn tegen de schapen van Uw weide?

  Gedenk Uw volksgemeenschap, die Gij in oude tijden gesticht hebt, de stam, die Gij tot Uw bezit hebt vrijgekocht,~\sep\ de Sionsberg, waar Gij Uw zetel hebt gevestigd.

  Richt Uw schreden naar de eeuwige puinen:~\sep\ alles heeft de vijand in het heiligdom verwoest.

  Uw tegenstanders raasden op de plaats van Uw vergadering,~\sep\ en richtten er hun banieren als zegetekens op.

  Ze zijn als zij die met de bijl in het kreupelhout zwaaien;~\sep\ en zie, met houweel en hamer verbrijzelen zij tezamen zijn deuren.

  Aan het vuur hebben zij Uw heiligdom prijsgegeven,~\sep\ de woontent van Uw Naam tot de grond toe ontwijd.

  Zij spraken bij zichzelf: ``Laten wij hen allen tezamen verdelgen;~\sep\ verbrandt alle heiligdommen Gods in het land.''

  Onze tekenen zien wij reeds niet meer, er is geen profeet;~\sep\ en niemand onder ons weet hoe lang nog.

  Hoe lang nog, O God, zal de vijand smaden,~\sep\ de tegenstander Uw Naam maar immer lasteren?

  Waarom wendt Gij Uw hand van ons af,~\sep\ en houdt Gij Uw rechter terug in Uw schoot?

  God toch is van oudsher mijn Koning,~\sep\ die midden op de aarde redding brengt.

  Gij hebt door Uw macht de zee gescheiden,~\sep\ in de wateren de koppen der draken verpletterd.

  Gij hebt de koppen van Leviathan verbrijzeld,~\sep\ hem tot voedsel gegeven aan de monsters der zee.

  Gij liet bronnen en beken ontspringen,~\sep\ Gij hebt waterrijke stromen drooggelegd.

  Van U is de dag en van U is de nacht;~\sep\ maan en zon hebt Gij hun vaste plaats gegeven.

  Alle grenzen der aarde hebt Gij bepaald;~\sep\ Gij hebt zomer en winter geschapen.

  Herinner U dit: de vijand heeft U gehoond, O Heer,~\sep\ en een waanzinnig volk heeft Uw Naam gelasterd.

  Geef het leven van Uw tortel niet prijs aan de gier,~\sep\ vergeet het leven van Uw armen niet voor immer.

  Denk aan Uw verbond,~\sep\ want geweld heerst in de schuilhoeken van land en veld.

  Dat geen verdrukte vol schaamte heenga:~\sep\ dat de arme en behoeftige prijzen Uw Naam.

  Rijs op, O God, verdedig Uw zaak,~\sep\ gedenk de smaad, die de dwaze U aandoet dag aan dag.

  Vergeet het geraas van Uw vijanden niet;~\sep\ het geschreeuw van die opstaan tegen U stijgt immer omhoog.
\end{halfparskip}

\begin{halfparskip}
  \markedsubsectionrubricwithhint{Vrijdagen ``na'': Marmita 44*}{(origineel: hulala 16)}

  \psalm{\Ps{107}} Looft de Heer, want Hij is goed, want eeuwig duurt Zijn barmhartigheid.~\sep

  Zo moeten nu spreken, die de Heer heeft verlost,~\sep

  \liturgicalhint{Alleluia, Alleluia, Alleluia.~--- Eerste vers.}

  die Hij redde uit de hand van de vijand,

  Die Hij bijeenbracht uit de landen,~\sep\ van oost en west, van noord en zuid.

  Daar doolden er in woestijn en wildernis rond,~\sep\ en vonden geen weg naar een bewoonbare stad.

  Zij werden gekweld door honger en dorst,~\sep\ hun leven verkwijnde in hen.

  En zij riepen tot de Heer in hun nood,~\sep\ en uit hun ellende verloste Hij hen.

  Hij leidde hen langs een rechte weg,~\sep\ om in een bewoonbare stad te komen.

  Dat zij de Heer om Zijn barmhartigheid danken,~\sep\ om Zijn wonderwerken voor de kinderen der mensen,

  Want wie was uitgehongerd, heeft Hij verzadigd,~\sep\ en de hongerige met het goede vervuld.

  Daar zaten er in donker en duister,~\sep\ in ellende, en in boeien geslagen,

  Want zij hadden Gods woorden weerstreefd,~\sep\ en het raadsbesluit van de Allerhoogste veracht.

  Toen heeft Hij hun hart door rampspoed gebroken;~\sep\ zij wankelden, maar niemand die hielp.

  En zij riepen tot de Heer in hun nood,~\sep\ en uit hun ellende bevrijdde Hij hen.

  Hij voerde hen weg uit donker en duister,~\sep\ en sloeg hun boeien aan stukken.

  Dat zij de Heer om Zijn barmhartigheid danken,~\sep\ om Zijn wonderwerken voor de kinderen der mensen.

  Daar Hij bronzen poorten stuk heeft geslagen,~\sep\ en ijzeren grendels verbrijzeld.

  Daar waren er ziek om hun zonden,~\sep\ in lijden om hun wangedrag;

  Alle voedsel was hun een walg,~\sep\ en zij naderden de poorten van de dood.

  En zij riepen tot de Heer in hun nood,~\sep\ en uit hun ellende bevrijdde Hij hen.

  Zijn woord zond Hij uit om hen te genezen,~\sep\ hen aan de dood te ontrukken.

  Dat zij de Heer om Zijn barmhartigheid danken,~\sep\ om Zijn wonderwerken voor de kinderen der mensen.

  Dat zij Hem dankoffers brengen,~\sep\ en jubelend Zijn werken verkondigen.

  Daar staken er op schepen in zee,~\sep\ om handel te drijven op de wijde wateren;

  Dezen hebben de werken van de Heer aanschouwd,~\sep\ Zijn wonderen op de hoge zee.

  Hij sprak: en Hij joeg de stormwind op,~\sep\ die zweepte haar golven omhoog.

  Tot de hemel sloegen ze op, in de diepten ploften ze neer;~\sep\ zij vergingen van angst in die rampen.

  Zij waggelden en knikten als waren zij dronken;~\sep\ en al hun kunde verdween.

  En zij riepen tot de Heer in hun nood,~\sep\ en uit hun ellende verloste Hij hen.

  Hij bedaarde de stormwind tot een bries,~\sep\ en de golven der zee legden zich neer.

  Zij waren blij dat het stil was geworden,~\sep\ en Hij hen voerde naar de verlangde haven.

  Dat zij de Heer om Zijn barmhartigheid danken,~\sep\ om Zijn wonderwerken voor de kinderen der mensen.

  Dat zij Hem roemen in de vergadering van het volk,~\sep\ in de raad van de oudsten Hem prijzen.

  Hij maakte rivieren tot een woestijn,~\sep\ en waterbronnen tot dorstige aarde,

  Tot zilte grond het vruchtbare land,~\sep\ om de boosheid van die er wonen.

  Hij herschiep de woestijn in een watervlakte,~\sep\ tot waterbronnen het dorre land.

  Daar gaf Hij een plaats aan de hongerigen,~\sep\ en zij stichtten een bewoonbare stad.

  Zij bezaaiden de akkers, legden wijngaarden aan,~\sep\ en verkregen een opbrengst aan vruchten.

  Hij zegende hen en zij groeiden sterk aan,~\sep\ en Hij schonk hun een talrijk vee.

  Toen slonk hun getal en ze werden verachtelijk,~\sep\ onder druk van rampen en kwelling.

  Maar Hij, die vorsten met smaad overstelpt,~\sep\ hen laat dwalen door ongebaande woestijnen,

  Hij richtte uit de ellende de behoeftige op,~\sep\ en maakte de gezinnen zo talrijk als kudden.

  De goeden zien het vol blijdschap,~\sep\ en al wat boos is, sluit zijn mond.

  Wie is er zo wijs en neemt dit ter harte,~\sep\ en overweegt naar behoren de barmhartigheden van de Heer?

  \psalm{\Ps{108}} Gesterkt is mijn hart, O God, gesterkt is mijn hart;~\sep\ ik zal zingen en het psalter bespelen.

  Ontwaak toch, mijn ziel, gij, psalter en citer, ontwaakt;~\sep\ ik wil de dageraad wekken.

  Onder de volken wil ik U prijzen, O Heer,~\sep\ en U bezingen onder de naties.

  Want hoog tot de hemel reikt Uw barmhartigheid,~\sep\ en tot de wolken Uw trouw.

  Verschijn hoog boven de hemelen, O God;~\sep\ dat Uw glorie strale over heel de aarde.

  Opdat Uw geliefden worden bevrijd,~\sep\ help ons door Uw rechterhand, en verhoor ons.

  God heeft gesproken in Zijn heiligdom:~\sep\ ``Ik zal juichen en Sichem verdelen, en het dal van Succoth meten.

  Van Mij is het land Galaäd, van Mij het land Manasse,~\sep\ Efraïm is de helm van Mijn hoofd, Juda Mijn scepter.

  Moab is Mijn wasbekken, op Edom werp Ik Mijn schoeisel,~\sep\ over Filistea zal Ik zegevieren.''

  Wie zal mij binnenvoeren in de versterkte stad,~\sep\ wie mij naar Edom geleiden?

  Zijt Gij het niet, O God, die ons hebt verstoten,~\sep\ en niet meer uittrekt, O God, met onze legerscharen!

  Schenk ons Uw hulp tegen de vijand,~\sep\ want ijdel is de hulp van mensen.

  Met God zullen wij dapper strijden,~\sep\ en Hij zelf zal onze vijanden vertreden.
\end{halfparskip}

\begin{halfparskip}
  \markedsubsectionrubricwithhint{Zaterdagen ``na'': Marmita 53*}{(origineel: hulala 19)}

  \psalm{\Ps{153}} Looft de Heer, want Hij is goed:~\sep\ want eeuwig duurt Zijn barmhartigheid!

  \liturgicalhint{Alleluia, Alleluia, Alleluia.~--- Eerste vers.}

  Looft de Heer der heren:~\sep\ want eeuwig duurt Zijn barmhartigheid!

  Die grote wonderen deed, Hij alleen:~\sep\ want eeuwig duurt Zijn barmhartigheid!

  Die de hemelen met wijsheid schiep:~\sep\ want eeuwig duurt Zijn barmhartigheid!

  Die op de wateren de aarde spreidde:~\sep\ want eeuwig duurt Zijn barmhartigheid!

  Die de grote lichten schiep:~\sep\ want eeuwig duurt Zijn barmhartigheid!

  De zon, om over de dag te heersen:~\sep\ want eeuwig duurt Zijn barmhartigheid!

  Maan en sterren, om over de nacht te heersen,~\sep\ want eeuwig duurt Zijn barmhartigheid!

  Die Egypte sloeg in zijn eerstgeborenen:~\sep\ want eeuwig duurt Zijn barmhartigheid!

  En Israël uit hun midden voerde:~\sep\ want eeuwig duurt Zijn barmhartigheid!

  Met sterke hand en uitgestrekte arm,~\sep\ want eeuwig duurt Zijn barmhartigheid!

  Die de Rode Zee in tweeën deelde:~\sep\ want eeuwig duurt Zijn barmhartigheid!

  En Israël er middendoor deed gaan:~\sep\ want eeuwig duurt Zijn barmhartigheid.

  En Farao met zijn legermacht neerwierp in de Rode Zee,~\sep\ want eeuwig duurt Zijn barmhartigheid.

  Die Zijn volk door de woestijn heeft geleid:~\sep\ want eeuwig duurt Zijn barmhartigheid!

  Die machtige vorsten versloeg:~\sep\ want eeuwig duurt Zijn barmhartigheid!

  En sterke koningen doodde:~\sep\ want eeuwig duurt Zijn barmhartigheid!

  Sehon, de vorst der Amorieten,~\sep\ want eeuwig duurt Zijn barmhartigheid!

  En Og, de vorst van Basan:~\sep\ want eeuwig duurt Zijn barmhartigheid.

  En hun land tot eigendom gaf:~\sep\ want eeuwig duurt Zijn barmhartigheid!

  Tot eigendom aan Israël, Zijn dienstknecht:~\sep\ want eeuwig duurt Zijn barmhartigheid!

  Die in onze verdrukking aan ons dacht:~\sep\ want eeuwig duurt Zijn barmhartigheid!

  En ons verloste van onze vijanden,~\sep\ want eeuwig duurt Zijn barmhartigheid!

  Die voedsel geeft aan alle vlees:~\sep\ want eeuwig duurt Zijn barmhartigheid!

  Looft de God van de hemel:~\sep\ want eeuwig duurt Zijn barmhartigheid!

  \psalm{\Ps{136}} Aan de stromen van Babylon, daar zaten wij en weenden,~\sep\ als wij aan Sion dachten.

  Aan de wilgen van dat land,~\sep\ hingen wij onze citers op.

  Want die ons hadden weggevoerd, vroegen ons daarginds om liederen, en die ons verdrukten, om een jubelzang:~\sep\ ``Zingt ons uit Sions liederen!''

  Hoe zouden wij een lied van de Heer zingen,~\sep\ op vreemde bodem!

  Jeruzalem, indien ik u zou vergeten,~\sep\ dan worde mijn rechterhand aan de vergetelheid prijsgegeven.

  Dat mijn tong aan mijn gehemelte kleve,~\sep\ zo ik u niet gedenk,

  Zo ik Jeruzalem niet stel,~\sep\ boven al mijn geneugten.

  Heer reken de zonen van Edom,~\sep\ de dag van Jeruzalem aan.

  Zij riepen: ``Verdelgt haar, verdelgt haar,~\sep\ tot in haar grondvesten toe!''

  O gij, dochter van Babylon, verdelgster;~\sep\ gelukkig hij, die u vergeldt de rampen, die gij bracht over ons!

  Gelukkig hij, die uw kleinen grijpt,~\sep\ en te pletter slaat tegen de rots!

  \psalm{\Ps{137}} Ik wil U prijzen, Heer, uit heel mijn hart,~\sep\ daar Gij hebt geluisterd naar de woorden van mijn mond.

  Ik wil U bezingen voor het aanschijn der engelen,~\sep\ mij neerwerpen, naar Uw heilige tempel gericht,

  En Uw Naam zal ik prijzen,~\sep\ om Uw goedheid en trouw,

  Daar Gij boven alles verheerlijkt hebt,~\sep\ Uw Naam en Uw belofte.

  Wanneer ik U aanriep, hebt Gij mij verhoord,~\sep\ en mijn zielskracht vermeerderd.

  Alle koningen der aarde zullen U prijzen, O Heer,~\sep\ als zij horen de woorden van Uw mond.

  En zij zullen de wegen van de Heer bezingen:~\sep\ ``Waarlijk, groot is de glorie van de Heer!''

  Waarlijk, verheven is de Heer, op de geringe rust Zijn oog,~\sep\ maar op de trotse ziet Hij neer van verre.

  Al wandel ik ook te midden van kwelling, Gij houdt mij in leven, en strekt Uw hand naar mijn woedende vijanden uit;~\sep\ Uw rechterhand is mijn redding.

  Wat de Heer begon, voltooit Hij voor mij; Heer, Uw goedheid blijft eeuwig;~\sep\ trek u niet terug van het werk van Uw handen.
\end{halfparskip}

\begin{halfparskip}
  \liturgicalOption{Alle dagen:}~\sep\ \dd~Alleluia, alleluia; Eer aan U, God, alleluia; Eer aan U, God, alleluia; Heer, ontferm U over ons. Laat ons bidden; vrede zij met ons.

  \cc~Versterk, onze Heer en onze God, onze zwakheid in Uw mededogen; in Uw goedertierenheid bemoedig
  (\translationoptionNl{troost}) en help de broosheid van onze ziel; maak de slaperigheid van onze gedachten wakker, verlicht (\translationoptionNl{neem weg}) de last van onze ledematen; was en reinig het vuil van onze schulden en zonden; verlicht de duisternis van ons intellect; strek uit en leg de machtige hand van bescherming op ons, zodat wij daardoor kunnen opstaan en U onophoudelijk belijden en verheerlijken alle dagen van ons leven, Heer van alles...
\end{halfparskip}

% % % % % % % % % % % % % % % % % % % % % % % % % % % % % % % % % % % % % % % %

\markedsection{Tweede marmita}

\begin{halfparskip}
  \markedsubsectionrubricwithhint{Maandagen ``voor'': Marmita 2*}{(origineel: hulala 2)}

  \psalm{\Ps{5}} Luister, Heer, naar mijn woorden, geef acht op mijn zuchten.~\sep\ Let op mijn bede, mijn Koning en God!

  \liturgicalhint{Alleluia, Alleluia, Alleluia.~--- Eerste vers.}

  Tot U toch richt ik mijn bede, Heer; in de morgen hoort Gij mijn stem,~\sep\ in de morgen leg ik mijn bede voor U neer en wacht.

  Neen, geen God zijt Gij, aan wie de boosheid behaagt; geen boze mag blijven bij U,~\sep\ noch houden goddelozen voor Uw aanschijn stand.

  Gij haat allen, die ongerechtigheid plegen,~\sep\ en stort alle leugenaars in het verderf;

  De bloeddorstige en de bedrieger~\sep\ zijn een afschuw voor de Heer.

  Maar dank aan Uw vele genaden,~\sep\ zal ik binnentreden in Uw huis,

  En mij neerwerpen voor Uw heilige tempel,~\sep\ in ontzag voor U, Heer.

  Geleid mij in Uw gerechtigheid omwille van mijn vijanden;~\sep\ baan Uw weg voor mij uit.

  Want geen oprechtheid is in hun mond,~\sep\ hun hart zint op bedrog;

  Een open graf is hun keel,~\sep\ vleitaal spreekt hun tong.

  Kastijd hen, o God;~\sep\ dat zij falen in hun plannen;

  Verdrijf hen om hun vele misdaden,~\sep\ want ze zijn weerspannig tegen U.

  Mogen allen zich verblijden, die vluchten tot U,~\sep\ en juichen voor immer!

  Wil hen beschermen, en dat in U zich verblijden,~\sep\ die Uw Naam beminnen!

  Want Gij, o Heer, zult de gerechtige zegenen;~\sep\ met welwillendheid hem omgeven als met een schild.

  \psalm{\Ps{6}} Heer, straf mij niet in Uw toorn,~\sep\ en in Uw gramschap kastijd mij niet.

  Wees mij genadig, o Heer, omdat ik zwak ben;~\sep\ genees mij, Heer, want ontwricht is mijn gebeente.

  En mijn ziel is diep geschokt;~\sep\ maar Gij, Heer, hoelang nog?

  Keer terug, Heer, en bevrijd mij;~\sep\ red mij om Uw barmhartigheid.

  Want in de dood denkt niemand aan U;~\sep\ wie looft U in het dodenrijk?

  Door mijn zuchten ben ik afgetobd, ik besproei mijn sponde iedere nacht door mijn geween,~\sep\ en drenk met mijn tranen mijn rustbed.

  Mijn ogen staan dof van verdriet,~\sep\ en flets om allen, die mij haten.

  Gaat weg van mij, gij allen, die onrecht pleegt,~\sep\ want de Heer heeft mijn schreien gehoord.

  De Heer heeft naar mijn smeken geluisterd,~\sep\ de Heer heeft mijn bede aanvaard.

  Dat al mijn vijanden zich schamen en hevig ontstellen,~\sep\ haastig vluchten, met schande overdekt!

  \psalm{\Ps{7}} Heer, mijn God, naar U vlucht ik heen;~\sep\ verlos en bevrijd mij van al mijn vervolgers,

  Opdat er geen als een leeuw mij het leven ontrove,~\sep\ mij verscheure, en niemand mij redt.

  Heer, mijn God, als ik dat heb gedaan,~\sep\ als er onrecht kleeft aan mijn handen,

  Als ik mijn vriend soms kwaad heb berokkend,~\sep\ terwijl ik toch spaarde die mij onrechtmatig bestreden:

  Dan mag de vijand mij achtervolgen en grijpen, op de grond mij vertreden,~\sep\ en mijn eer vergooien in het stof.

  Rijs op, Heer, in Uw toorn, verhef U tegen de woede van mijn verdrukkers,~\sep\ en treed voor mij op in het gericht, door U bepaald.

  De vergaderde volken mogen U omringen;~\sep\ zetel boven hen uit in de hoge.

  De Heer is de Rechter der volken: doe mij recht, O Heer, naar mijn gerechtigheid,~\sep\ en naar de onschuld van mijn hart.

  De snoodheid der bozen neme een einde, maar geef de rechtvaardige kracht,~\sep\ Gij, rechtvaardige God, die harten en nieren doorgrondt.

  God is mij tot schild;~\sep\ Hij redt de oprechten van hart.

  God is een rechtvaardige Rechter,~\sep\ een God, die voortdurend bedreigt.

  Bekeert men zich niet, dan scherpt Hij Zijn zwaard,~\sep\ dan spant en richt Hij Zijn boog,

  Moordende schichten bereidt Hij voor hen,~\sep\ en gloeiend maakt Hij Zijn pijlen.

  Zie, door ongerechtigheid werd hij bevrucht, hij gaat zwanger van boosheid;~\sep\ en wat hij baart, is bedrog.

  Hij groef een kuil, en diepte hem uit,~\sep\ maar viel zelf in de groeve, die hij had gedolven.

  Op eigen hoofd zal zijn boosheid wederkeren,~\sep\ en neerdalen op eigen schedel zijn wreedheid.

  Maar ik zal de Heer om Zijn gerechtigheid prijzen,~\sep\ en de Naam van de allerhoogste Heer bezingen onder citerspel.
\end{halfparskip}

\begin{halfparskip}
  \markedsubsectionrubricwithhint{Dinsdagen ``voor'': Marmita 12*}{(origineel: hulala 5)}

  \psalm{\Ps{33}} Jubelt, rechtvaardigen, in de Heer:~\sep\ de rechtschapenen past een lofzang.

  \liturgicalhint{Alleluia, Alleluia, Alleluia.~--- Eerste vers.}

  Looft de Heer met de citer,~\sep\ tokkelt voor Hem de tiensnarige harp.

  Zingt voor Hem een nieuw lied,~\sep\ zingt voor Hem in welluidende klanken.

  Want het woord van de Heer is oprecht,~\sep\ en al wat Hij doet, is betrouwbaar.

  Gerechtigheid en recht heeft Hij lief,~\sep\ van de goedheid van de Heer is de aarde vervuld.

  Door het woord van de Heer zijn de hemelen gemaakt,~\sep\ door de adem van Zijn mond heel hun legerschaar.

  Als in een lederen zak verzamelt Hij de wateren der zee,~\sep\ de vloeden bergt Hij op in schuren.

  Heel de aarde vreze de Heer,~\sep\ dat alle bewoners der wereld Hem duchten.

  Want Hij sprak: en het was;~\sep\ Hij beval: en het bestond.

  De Heer verijdelt de raadslagen der naties,~\sep\ doet de plannen der volken te niet.

  Het besluit van de Heer staat voor eeuwig vast,~\sep\ de gedachten van Zijn hart van geslacht tot geslacht.

  Gelukkig het volk, wiens God de Heer is,~\sep\ de natie, die Hij tot erfdeel koos.

  De Heer blikt neer uit de hemel,~\sep\ en ziet alle kinderen der mensen.

  Uit Zijn woonplaats ziet Hij van verre neer,~\sep\ op allen, die de aarde bewonen:

  Hij, die aller hart geschapen heeft,~\sep\ die let op al hun werken.

  Geen koning verwint door een machtig leger,~\sep\ geen strijder vindt heil in geweldige kracht.

  Het ros is onmachtig de zege te schenken,~\sep\ en brengt geen redding bij al zijn kracht.

  Maar de ogen van de Heer rusten op hen, die Hem vrezen,~\sep\ op hen die op Zijn goedheid hopen.

  Om hen te ontrukken aan de dood,~\sep\ en bij hongersnood te voeden.

  Onze ziel stelt haar hoop op de Heer,~\sep\ Hij is onze Helper en ons Schild.

  Daarom verblijdt zich ons hart in Hem,~\sep\ en vertrouwen wij op Zijn heilige Naam.

  Uw barmhartigheid, Heer, kome over ons,~\sep\ naarmate wij hopen op U.

  \psalm{\Ps{34}} Prijzen wil ik de Heer te allen tijde,~\sep\ steeds zal mijn mond Hem loven.

  Moge mijn ziel op de Heer zich beroemen;~\sep\ dat de geringen het horen en juichen!

  Verheerlijkt de Heer met mij,~\sep\ en prijzen wij samen Zijn Naam!

  Ik heb de Heer gezocht, en Hij heeft mij verhoord,~\sep\ mij bevrijd van al mijn angsten.

  Ziet naar Hem op, opdat ge van blijdschap moogt stralen,~\sep\ en geen schaamrood Uw aanschijn bedekke.

  Zie, een bedrukte riep luid, en de Heer heeft geluisterd,~\sep\ en hem van al zijn kommer bevrijd.

  De engel van de Heer slaat een legerplaats op,~\sep\ rond hen, die Hem vrezen: en hij bevrijdt hen.

  Smaakt en ziet hoe goed de Heer is;~\sep\ gelukkig de man, die heenvlucht naar Hem.

  Vreest de Heer, gij, Zijn heiligen,~\sep\ want die Hem vrezen, lijden geen nood.

  Machtigen werden arm en moesten honger lijden,~\sep\ maar die de Heer zoeken, zal niets goeds ontbreken.

  Komt, kinderen, luistert naar mij,~\sep\ ik zal u leren de vreze van de Heer.

  Wie is de mens, die van het leven houdt,~\sep\ naar dagen verlangt om het goede te smaken?

  Behoed uw tong voor het kwaad,~\sep\ en uw lippen voor bedrieglijke woorden.

  Wijk terug van het kwaad en doe het goede;~\sep\ zoek de vrede en jaag hem na!

  De ogen van de Heer zien op de rechtvaardigen neer,~\sep\ en Zijn oren luisteren naar hun geroep.

  Het aanschijn van de Heer is tegen de bozen gericht,~\sep\ om de herinnering aan hen van de aarde te verdelgen.

  De rechtvaardigen riepen luid, en de Heer heeft hen verhoord,~\sep\ en hen van al hun angsten bevrijd.

  De vermorzelden van harte is de Heer nabij,~\sep\ de terneergeslagenen van geest schenkt Hij redding.

  Vele rampen treffen de rechtvaardige,~\sep\ maar uit alle bevrijdt hem de Heer.

  Al zijn beenderen behoedt Hij,~\sep\ niet één ervan zal worden gebroken.

  De boosheid drijft de goddeloze in de dood,~\sep\ die de rechtvaardige haten, worden gestraft.

  De Heer spaart Zijn dienaars het leven,~\sep\ en wie tot Hem vlucht, blijft vrij van straf.
\end{halfparskip}

\markedsubsectionrubric{Woensdagen ``voor'': Marmita 18*}

\begin{halfparskip}
  \psalm{\Ps{47}} Volken, gij alle, klapt in de handen,~\sep\ juicht God toe met jubelzang!

  \liturgicalhint{Alleluia, Alleluia, Alleluia.~--- Eerste vers.}

  Want hoogverheven, ontzagwekkend is de Heer,~\sep\ de grote Koning van heel de aarde.

  Hij onderwerpt ons de volken,~\sep\ en legt de naties onder onze voeten.

  Ons erfdeel kiest Hij voor ons uit,~\sep\ de roem van Jacob, die Hij liefheeft.

  God stijgt op onder gejubel,~\sep\ de Heer onder bazuingeschal.

  Zingt voor God, zingt Hem toe,~\sep\ zingt voor onze Koning, zingt Hem toe!

  Want God is Koning over heel de aarde:~\sep\ zingt een lofzang.

  God heerst over de volkeren,~\sep\ God zetelt op Zijn heilige troon.

  De vorsten der volken sloten zich aan,~\sep\ bij het volk van Abrahams God.

  Want aan God behoren de machtigen der aarde:~\sep\ Hij is hoogverheven.

  \psalm{\Ps{48}} Groot is de Heer en hoogst lofwaardig,~\sep\ in de stad van onze God.

  Zijn heilige, zijn roemvolle heuvel,~\sep\ is de vreugde van heel het aardrijk.

  De berg Sion, het uiterste noorden,~\sep\ is de stad van de grote Koning.

  God in haar burchten,~\sep\ toonde zich een veilige schutse.

  Want ziet, de koningen sloten een verbond,~\sep\ en rukten gezamenlijk op.

  Eén blik! Ze staan verbijsterd,~\sep\ ze sidderen en stuiven uiteen.

  Daar grijpt ontzetting hen aan,~\sep\ smart als van een barende,

  Zoals wanneer de oostenwind,~\sep\ de schepen van Tharsis verbrijzelt.

  Gelijk wij het hoorden, zo hebben wij het nu gezien,~\sep\ in de stad van de Heer der heerscharen,

  In de stad van onze God:~\sep\ God houdt haar eeuwig in stand.

  Wij gedenken Uw barmhartigheid, O God,~\sep\ binnen Uw tempel.

  Zoals Uw Naam, O God, zo ook Uw lof:~\sep\ hij reikt tot de grenzen der aarde.

  Vol gerechtigheid is Uw rechterhand,~\sep\ dat de berg Sion zich verblijde,

  Dat Juda's steden jubelen,~\sep\ om Uw gerichten!

  Laat Uw blikken gaan over Sion, en wandelt er omheen,~\sep\ telt zijn torens.

  Beschouwt zijn bolwerken,~\sep\ doorloopt zijn burchten,

  Om het te verhalen aan het nageslacht:~\sep\ ``Zo groot is God,

  Onze God voor eeuwig en immer:~\sep\ Hij zelf zal ons geleiden.''

  \psalm{\Ps{49}} Aanhoort het, alle volkeren;~\sep\ luistert, alle bewoners der aarde:

  Zowel geringen als edelen,~\sep\ rijken als armen op eenzelfde wijze.

  Mijn mond gaat wijsheid verkondigen,~\sep\ de overweging van mijn hart brengt inzicht.

  Mijn oor wil ik neigen naar een leer van wijsheid,~\sep\ bij het spel van de citer mijn raadsel onthullen.

  Waarom zou ik vrezen op dagen van rampspoed,~\sep\ als de boosheid van belagers mij omringt,

  Die op hun rijkdom vertrouwen,~\sep\ en pochen op hun groot bezit?

  Want niemand kan zichzelf bevrijden,~\sep\ noch zijn losgeld betalen aan God.

  Te hoog is de losprijs voor zijn leven, en nimmer toereikend,~\sep\ om eeuwig te leven en de dood te ontgaan.

  Want de wijzen ziet hij sterven, ook de dwaas en de domme ziet hij vergaan,~\sep\ en hun rijkdommen aan vreemden achterlaten.

  Het graf is voor eeuwig hun woning, hun verblijf van geslacht tot geslacht,~\sep\ al hebben zij ook landgoederen naar hun naam genoemd.

  Neen, de mens in weelde blijft niet voortbestaan,~\sep\ hij is als het vee, dat vergaat.

  Dit is de weg van wie als dwazen vertrouwen,~\sep\ en dit is het eind van wie in hun lot zich verlustigen.

  Zij worden als schapen in het dodenrijk geborgen;~\sep\ de dood is hun herder, de rechtvaardigen zijn hun meesters.

  Spoedig gaat hun gedaante voorbij,~\sep\ het dodenrijk zal hun woonplaats zijn.

  Doch God zal mijn ziel uit het dodenrijk redden,~\sep\ doordat Hij mij tot Zich neemt.

  Maak u geen zorg, als iemand rijk is geworden,~\sep\ als het vermogen van zijn huis is toegenomen;

  Want bij zijn sterven neemt hij niets met zich mee,~\sep\ zijn schatten dalen niet met hem af.

  Al heeft hij zich bij zijn leven gelukkig geprezen:~\sep\ ``Men zal u roemen, omdat gij u te goed deedt.''

  Afdalen zal hij in de kring van zijn vaderen,~\sep\ die in eeuwigheid het licht niet zullen aanschouwen.

  De mens, die onbezonnen en in weelde leeft,~\sep\ is gelijk aan het vee, dat vergaat.
\end{halfparskip}

\begin{halfparskip}
  \markedsubsectionrubricwithhint{Donderdagen ``voor'': Marmita 26*}{(origineel: hulala 11)}

  \psalm{\Ps{69}} Red mij, O God,~\sep\ want de wateren zijn tot aan mijn hals gestegen.

  Ik ben weggezonken in een diepe modderpoel,~\sep\ en er is geen plaats, waar ik mijn voet kan zetten.

  \liturgicalhint{Alleluia, Alleluia, Alleluia.~--- Eerste vers.}

  Ik zonk in diepe wateren,~\sep\ en de golven bedelven mij.

  Ik ben uitgeput door het roepen,~\sep\ en schor is mijn keel,

  Mijn ogen zijn vermoeid,~\sep\ van het hoopvol opzien naar mijn God.

  Talrijker dan mijn hoofdharen zijn zij,~\sep\ die zonder reden mij haten;

  Krachtiger dan mijn beenderen, die mij ten onrechte vervolgen;~\sep\ of moet ik soms teruggeven wat ik niet heb geroofd?

  God, Gij kent mijn dwaasheid,~\sep\ en mijn zonden zijn U niet verborgen.

  Laat hen, die op U hopen, niet te schande worden om mij,~\sep\ O Heer, Heer der legerscharen.

  Laat niet beschaamd worden om mij,~\sep\ die U zoeken, God van Israël.

  Want om U leed ik versmading,~\sep\ en bedekte schaamrood mijn gelaat.

  Ik ben een vreemdeling geworden voor mijn broeders,~\sep\ en een onbekende voor de zonen van mijn moeder.

  Want de ijver voor Uw huis heeft mij verteerd,~\sep\ en de smaad van die U smaden viel op mij neer.

  Door vasten heb ik mij gekastijd,~\sep\ en het werd mij tot smaad.

  Als gewaad trok ik een boetekleed aan,~\sep\ en ik werd hun tot spot.

  Die in de poort zitten, bepraten mij,~\sep\ en de wijndrinkers voegen mij schimpwoorden toe.

  Maar tot U, Heer, richt ik mijn bede,~\sep\ in de tijd der genade, O God:

  Verhoor mij volgens Uw grote goedheid,~\sep\ volgens Uw trouwe hulp.

  Trek mij op uit slijk, opdat ik niet verzinke; red mij uit de handen van die mij haten,~\sep\ en uit de diepten der wateren.

  Laat de watervloeden mij niet bedelven, en de diepe zee mij niet verzwelgen,~\sep\ noch de kuil zijn mond boven mij sluiten.

  Verhoor mij, Heer, want mild is Uw genade,~\sep\ zie op mij neer volgens Uw grote barmhartigheid.

  Verberg Uw aanschijn niet voor Uw dienstknecht;~\sep\ verhoor mij spoedig, want ik word gekweld.

  Nader tot mij om mij te redden;~\sep\ wil mij bevrijden vanwege mijn vijanden.

  Gij kent mijn smaad en mijn schaamte en mijn schande;~\sep\ allen, die mij kwellen, staan U voor ogen.

  De smaad brak mijn hart en ik kwijnde weg; ik zag uit naar een vol medelijden, maar er was er geen;~\sep\ naar troosters, maar ik vond ze niet.

  Zij mengden mijn spijzen met gal,~\sep\ en in mijn dorst gaven zij mij azijn te drinken.

  Moge hun tafel hun worden tot val,~\sep\ en voor hun vrienden tot strik.

  Dat hun ogen worden verduisterd, zodat zij niet zien;~\sep\ en doe hun schreden immer wankelen!

  Stort Uw verontwaardiging over hen uit,~\sep\ en laat de gloed van Uw toorn hen aangrijpen!

  Hun woonstede worde verwoest,~\sep\ en niemand wone nog in hun tenten.

  Want zij hebben vervolgd, die Gij hebt geslagen,~\sep\ en de smart verhoogd van die Gij hebt gewond.

  Vermeerder hun schuld,~\sep\ en dat zij niet meer gerechtvaardigd worden bij U.

  Men wisse hen weg uit het boek der levenden,~\sep\ schrijve hen niet op met de rechtvaardigen.

  Maar ik ben ellendig en treur;~\sep\ Uw hulp, O God, moge mij beschermen!

  Prijzen zal ik de Naam van God met een jubelzang,~\sep\ Hem met een danklied verheerlijken.

  Dit zal God welgevalliger zijn dan een offerstier,~\sep\ dan een rund met horens en hoeven.

  Ziet het, bedrukten, en weest blijde,~\sep\ dan leeft Uw hart weer op, O gij, die God zoekt.

  Want de Heer hoort de armen aan,~\sep\ en versmaadt Zijn gevangenen niet.

  Dat hemel en aarde Hem loven,~\sep\ de zeeën en al wat zich daarin beweegt.

  Want God zal Sion redden, en de steden van Juda herbouwen,~\sep\ daar zal men wonen en het bezitten.

  En het geslacht van Zijn dienaren zal het beërven,~\sep\ en die Zijn Naam liefhebben, zullen er wonen.

  \psalm{\Ps{70}} Gewaardig u, O God, mij te verlossen;~\sep\ Heer, snel mij te hulp.

  Laat schande en beschaming hen treffen,~\sep\ die mij naar het leven staan.

  Dat zij vol schaamte terugdeinzen,~\sep\ die zich over mijn rampen verheugen,

  Terugwijken, met schaamte overladen,~\sep\ die mij toeroepen: Ha, ha!

  Over U mogen jubelen en zich verblijden,~\sep\ zij allen, die U zoeken.

  En steeds herhalen: ``Hooggeprezen zij God'',~\sep\ die uitzien naar Uw hulp.

  Ik echter ben ellendig en arm,~\sep\ God, sta mij bij!

  Gij zijt mijn Helper en mijn Redder;~\sep\ Heer, wil toch niet toeven!
\end{halfparskip}

\begin{halfparskip}
  \markedsubsectionrubricwithhint{Vrijdagen ``voor'': Marmita 42*}{(origineel: hulala 17)}

  \psalm{\Ps{105}} Looft de Heer, bejubelt Zijn Naam,~\sep\ maakt Zijn daden bekend aan de volken.

  \liturgicalhint{Alleluia, Alleluia, Alleluia.~--- Eerste vers.}

  Prijst Hem met stem en snaren,~\sep\ verhaalt van al Zijn wonderwerken.

  Roemt op Zijn heilige Naam;~\sep\ verheugd zij het hart van wie de Heer zoeken.

  Beschouwt de Heer en Zijn macht,~\sep\ zoekt immer Zijn aanschijn.

  Gedenkt de wonderen, die Hij wrochtte,~\sep\ Zijn tekenen en de oordelen van Zijn mond,

  Gij, nageslacht van Abraham, Zijn dienaar,~\sep\ gij, zonen van Jacob, Zijn uitverkorene.

  Hij, de Heer, is onze God;~\sep\ Zijn gerichten gelden over heel de aarde.

  Zijn verbond is Hij voor eeuwig indachtig,~\sep\ tot in duizend geslachten de belofte, die Hij deed,

  Het verbond, dat Hij sloot met Abraham,~\sep\ de eed, die Hij Isaac zwoer;

  Wat Hij voor Jacob maakte tot een vaste wet,~\sep\ voor Israël tot een eeuwig verbond,

  Terwijl Hij sprak: ``Het land Chanaan zal Ik u schenken,~\sep\ als het erfelijk bezit, u bestemd.''

  Toen zij nog klein in aantal waren,~\sep\ weinigen en als vreemdelingen in dat land,

  Toen zij nog zwierven van volk tot volk,~\sep\ en van het ene rijk naar het andere,

  Duldde Hij niet, dat iemand hen verdrukte,~\sep\ en tuchtigde koningen omwille van hen:

  ``Wilt niet raken aan Mijn gezalfden,~\sep\ en doet Mijn profeten geen leed.''

  Hij riep hongersnood over het land,~\sep\ en onttrok alle brood, dat versterkt.

  Hij zond een man voor hen uit:~\sep\ tot slavendienst werd Jozef verkocht.

  Zij hadden zijn voeten in boeien geslagen,~\sep\ zijn hals was in ijzer gekluisterd,

  Totdat zijn voorspelling in vervulling ging,~\sep\ hem de uitspraak van de Heer rechtvaardigde.

  De koning beval zijn boeien te slaken,~\sep\ de heerser der volken liet hem weer vrij.

  Hij stelde hem aan tot heer van zijn huis,~\sep\ en tot beheerder van heel zijn bezit,

  Om zijn rijksgroten naar goedvinden te leiden,~\sep\ zijn ouderlingen wijsheid te leren.

  Toen trok Israël Egypte binnen,~\sep\ Jacob werd gast in het land van Cham.

  \emph{Want in Egypte verrichtte Hij wonderen~\sep\ en zij geloofden niet in Hem.}

  En Zijn volk deed Hij krachtig in aantal wassen,~\sep\ en maakte het boven zijn vijanden sterk.

  Hun gezindheid veranderde Hij, opdat zij Zijn volk zouden haten,~\sep\ en arglistig Zijn dienaars bejegenen.

  Toen zond Hij Moses, Zijn dienaar,~\sep\ Aäron, die Hij had uitverkoren.

  Zij wrochtten Zijn tekenen onder hen,~\sep\ en wonderwerken in het land van Cham.

  Hij zond duisternis af, en het werd donker,~\sep\ maar zij weerstonden Zijn bevel.

  Hij veranderde hun wateren in bloed,~\sep\ en deed hun vissen sterven.

  Hun land krioelde van kikkers,~\sep\ tot in de zalen van hun vorsten.

  Hij sprak: en er kwam een menigte muggen,~\sep\ muskieten over heel hun gebied.

  Voor regen zond Hij hun hagel,~\sep\ een vlammend vuur over hun land.

  Hun wijnstok en vijgenboom sloeg Hij neer,~\sep\ en knakte de bomen in hun gebied.

  Hij sprak: en sprinkhanen kwamen,~\sep\ en kevers zonder tal;

  Zij verslonden al het gewas op hun bodem,~\sep\ zij verslonden de vruchten van hun veld.

  Hij sloeg alle eerstgeborenen in hun land,~\sep\ de eerstelingen van al hun mannenkracht.

  Toen leidde Hij ze uit met zilver en goud,~\sep\ geen zieke was er onder hun stammen.

  \emph{Weg uit Egypte verrichtte Hij wonderen,~\sep\ en redde Hij Zijn volk.}

  De Egyptenaren waren blij dat ze gingen,~\sep\ want angst voor hen had hen bevangen.

  Hij spreidde een wolk tot dekking uit,~\sep\ een vuur tot lichtbaken in de nacht.

  Zij smeekten: en Hij voerde kwakkels aan,~\sep\ en verzadigde hen met brood uit de hemel.

  Hij spleet de rots: en het water vloeide,~\sep\ het liep in de woestijn als een stroom.

  Want Hij dacht aan Zijn heilig woord,~\sep\ dat Hij gaf aan Abraham, Zijn dienaar.

  En Hij voerde Zijn volk in vreugde weg,~\sep\ Zijn uitverkorenen onder gejubel.

  Hij schonk hun de landen der heidenen,~\sep\ zij verwierven de schatten der volken,

  Opdat ze Zijn geboden zouden volbrengen,~\sep\ en Zijn wetten onderhouden.
\end{halfparskip}

\begin{halfparskip}
  \markedsubsectionrubricwithhint{Zaterdagen ``voor'': Marmita 48*}{(origineel: hulala 20)}

  \psalm{\Ps{118}}

  \acrosticletter{Aleph} Gelukkig zij, die smetteloos zijn op hun levensweg,~\sep\ die wandelen naar de Wet van de Heer.

  \liturgicalhint{Alleluia, Alleluia, Alleluia.~--- Eerste vers.}

  Gelukkig zij, die Zijn voorschriften volbrengen,~\sep\ Hem zoeken met heel hun hart,

  Die geen ongerechtigheid plegen,~\sep\ maar op Zijn wegen wandelen.

  Gij hebt Uw bevelen gegeven,~\sep\ opdat men ze stipt zou volbrengen.

  Mochten mijn wegen standvastig zijn,~\sep\ om Uw verordeningen na te leven.

  Dan zal ik niet te schande worden,~\sep\ als ik acht geef op al Uw geboden.

  In oprechtheid des harten zal ik U prijzen,~\sep\ als ik de besluiten van Uw gerechtigheid ken.

  Uw verordeningen zal Ik volbrengen;~\sep\ wil mij in het geheel niet verlaten!

  \acrosticletter{Beth} Hoe houdt een jongeling zijn levenspad rein?~\sep\ Door het onderhouden van Uw woord.

  Met heel mijn hart dan zoek ik U;~\sep\ laat mij van Uw geboden niet wijken.

  Ik houd in mijn hart Uw uitspraak verborgen,~\sep\ om niet te zondigen tegen U.

  Gezegend zijt Gij, O Heer;~\sep\ leer mij toch Uw verordeningen.

  Met mijn lippen verkondig ik~\sep\ al de besluiten van Uw mond.

  Over de weg van Uw voorschriften verheug ik mij,~\sep\ als over alle rijkdom.

  Uw bevelen zal ik overpeinzen,~\sep\ en acht geven op Uw wegen.

  Ik zal mij verheugen over Uw verordeningen;~\sep\ en Uw woorden niet vergeten!

  \acrosticletter{Ghimel} Doe wel aan Uw dienaar, opdat ik leve,~\sep\ en Ik zal Uw woorden onderhouden.

  Open mijn ogen,~\sep\ om de wonderen van Uw Wet te beschouwen.

  Ik ben een vreemdeling op aarde,~\sep\ verberg mij Uw geboden niet.

  Mijn ziel kwijnt weg~\sep\ van aanhoudend verlangen naar Uw besluiten.

  Gij hebt de trotsen berispt;~\sep\ vervloekt, die Uw geboden verlaten.

  Neem smaad en verachting van mij weg;~\sep\ want Uw voorschriften volg ik op.

  Zelfs als vorsten vergaderen en tegen mij plannen beramen,~\sep\ dan nog overweegt Uw dienaar Uw verordeningen.

  Want Uw voorschriften zijn mijn geneugte,~\sep\ Uw verordeningen mijn raadgevers.

  \acrosticletter{Daleth} Mijn ziel ligt neer in het stof;~\sep\ geef mij nieuw leven, volgens Uw belofte.

  Ik heb mijn wegen opengelegd, en Gij hebt mij verhoord;~\sep\ leer mij Uw verordeningen!

  Onderricht mij in de weg van Uw geboden,~\sep\ en ik zal Uw wonderwerken overwegen.

  Ik stort tranen van droefheid;~\sep\ richt mij op naar Uw woord.

  Houd mij van dwaalwegen af,~\sep\ maar schenk mij vrijgevig Uw Wet.

  Ik heb de weg der waarheid gekozen,~\sep\ Uw besluiten mij voor ogen opengesteld.

  Ik houd mij vast aan Uw voorschriften;~\sep\ maak mij niet te schande, O Heer.

  Ik zal de weg van Uw geboden bewandelen,~\sep\ als Gij mijn hart hebt verruimd.

  \acrosticletter{He} Toon mij, O Heer, de weg van Uw verordeningen,~\sep\ en ik zal er mij stipt aan houden.

  Onderricht mij, opdat ik Uw Wet volbrenge;~\sep\ en ik zal ze naleven uit geheel mijn hart.

  Leid mij langs de weg van Uw geboden,~\sep\ want daarin vind ik mijn vreugde.

  Wil mijn hart naar Uw voorschriften neigen,~\sep\ en niet in hebzucht naar gewin.

  Wend mijn ogen af, opdat ze geen ijdelheid zien~\sep\ ; schenk mij het leven langs Uw weg.

  Kom voor Uw dienaar Uw belofte na,~\sep\ gedaan aan hen, die U vrezen.

  Wend de smaad, die Ik ducht, van mij af,~\sep\ want aangenaam zijn Uw besluiten.

  Ja waarlijk, ik zie uit naar Uw bevelen;~\sep\ geef mij het leven naar Uw gerechtigheid.

  \acrosticletter{Vau} Dat Uw ontferming, Heer, op mij neerdale,~\sep\ en Uw hulp, zoals Gij beloofd hebt.

  En Ik zal, die mij honen, te woord staan,~\sep\ omdat ik op Uw woorden hoop.

  Neem het woord der waarheid niet uit mijn mond,~\sep\ daar ik mijn hoop stel op Uw besluiten.

  Uw Wet zal ik steeds onderhouden,~\sep\ voor eeuwig en immer.

  Op een brede weg zal ik wandelen,~\sep\ omdat ik Uw bevelen doorzoek.

  Ik zal voor koningen over Uw voorschriften spreken,~\sep\ en er mij niet over schamen,

  Ik zal mij verheugen in Uw geboden,~\sep\ die ik bemin.

  Opheffen zal ik mijn handen naar Uw geboden,~\sep\ en Uw verordeningen overwegen.

  \acrosticletter{Zaïn} Gedenk Uw woord tot Uw dienstknecht,~\sep\ waardoor Gij mij hoop hebt gegeven.

  Dit is mijn troost in mijn droefheid,~\sep\ dat Uw uitspraak mij het leven schenkt.

  Trotsen honen mij zeer,~\sep\ maar van Uw Wet wijk ik niet af.

  Ik gedenk Uw aloude uitspraken, O Heer,~\sep\ en ik voel mij getroost.

  Verontwaardiging grijpt mij aan vanwege de zondaars,~\sep\ die Uw Wet verlaten.

  Uw verordeningen zijn mij gezangen geworden,~\sep\ in het oord van mijn ballingschap.

  Des nachts, O Heer, gedenk ik Uw Naam,~\sep\ en ik wil Uw Wet onderhouden.

  Dit is mij ten deel gevallen,~\sep\ omdat ik Uw bevelen heb nageleefd.

  \acrosticletter{Heth} Mijn deel, O Heer, zo sprak ik,~\sep\ is het onderhouden van Uw woorden.

  Van ganser harte bid ik tot Uw aanschijn:~\sep\ wees mij genadig naar Uw belofte.

  Ik dacht over mijn levensweg na,~\sep\ en richtte naar Uw voorschriften mijn schreden.

  Ik heb mij gehaast, en gedraald heb ik niet,~\sep\ Uw geboden te onderhouden.

  De strikken der bozen hielden mij omstrengeld;~\sep\ Uw Wet vergat ik niet.

  Te middernacht sta ik op om U te prijzen,~\sep\ voor Uw rechtvaardige uitspraken.

  Ik ben de vriend van allen, die U vrezen,~\sep\ en Uw bevelen onderhouden.

  De aarde is vol van Uw goedheid, O Heer;~\sep\ leer mij Uw verordeningen.

  \acrosticletter{Teth} Gij hebt Uw dienstknecht welgedaan,~\sep\ O Heer, zoals Gij beloofd hebt.

  Geef mij oordeel en inzicht,~\sep\ want ik vertrouw op Uw geboden.

  Eer ik getuchtigd werd, dwaalde ik af;~\sep\ maar nu houd ik mij aan Uw uitspraak.

  Gij zijt goed en weldadig;~\sep\ leer mij Uw verordeningen.

  De trotsen beramen listige plannen tegen mij,~\sep\ toch leef ik van ganser harte Uw bevelen na.

  Gestold als vet is hun hart;~\sep\ ik vind in Uw Wet mijn vreugde.

  Het was goed voor mij, dat ik werd gekastijd,~\sep\ opdat ik Uw verordeningen zou leren.

  De Wet van Uw mond is mij meer,~\sep\ dan duizenden in goud en zilver.

  \acrosticletter{Jod} Uw handen hebben mij gemaakt en gevormd;~\sep\ onderricht mij, opdat ik Uw geboden leer kennen.

  Die U vrezen, zullen mij zien en zich verheugen,~\sep\ omdat ik op Uw woord heb vertrouwd.

  Ik weet, O Heer, dat Uw besluiten rechtvaardig zijn,~\sep\ en terecht hebt Gij mij gekastijd.

  Uw barmhartigheid sta mij bij om mij te troosten,~\sep\ volgens de belofte, aan Uw dienaar gedaan.

  Dat Uw ontferming op mij neerdale,~\sep\ opdat ik leve, daar Uw Wet mijn vreugde is.

  Schande aan de trotsen, want ten onrechte kwellen zij mij;~\sep\ ik zal Uw bevelen overwegen.

  Mogen zich tot mij wenden, die U vrezen,~\sep\ en op Uw voorschriften bedacht zijn.

  Mijn hart zij volmaakt in Uw verordeningen,~\sep\ opdat ik niet te schande worde.

  \acrosticletter{Caph} Mijn ziel kwijnt van verlangen naar Uw hulp;~\sep\ ik stel mijn hoop op Uw woord.

  Mijn ogen kwijnen van verlangen naar Uw uitspraak;~\sep\ wanneer zult Gij mij troosten?

  Want ik ben als een lederen zak in de rook,~\sep\ toch vergat ik Uw verordeningen niet.

  Hoeveel dagen zijn Uw dienaar beschoren?~\sep\ Wanneer voltrekt Gij Uw oordeel aan mijn vervolgers?

  De trotsen hebben mij kuilen gegraven,~\sep\ zij, die niet handelen naar Uw Wet.

  Waarachtig zijn al Uw geboden;~\sep\ men vervolgt mij ten onrechte; kom mij te hulp!

  Men heeft mij haast van de aarde verdelgd;~\sep\ maar Uw bevelen verwaarloosde ik niet.

  Spaar mijn leven naar Uw erbarming,~\sep\ en de voorschriften van Uw mond zal ik volgen.
\end{halfparskip}

\begin{halfparskip}
  \markedsubsectionrubricwithhint{Maandagen ``na'': Marmita 7*}{(origineel: hulala 2)}

  \psalm{\Ps{19}} De hemelen verhalen de glorie van God,~\sep\ en het uitspansel roemt het werk van Zijn handen.

  \liturgicalhint{Alleluia, Alleluia, Alleluia.~--- Eerste vers.}

  De dag galmt het uit aan de dag,~\sep\ en de nacht geeft het door aan de nacht.

  Dat is geen taal, dat zijn geen woorden,~\sep\ waarvan de klank niet wordt vernomen.

  Over heel de wereld golft hun sein,~\sep\ en tot de grenzen der aarde hun uitspraak.

  Daar sloeg Hij Zijn tent op voor de zon, die als een bruidegom uit zijn bruidskamer treedt,~\sep\ en als een juichende reus zijn baan doorloopt.

  Aan het einde van de hemel is zijn opgang, en zijn kringloop reikt tot het einde van de hemel,~\sep\ aan zijn gloed kan niets zich onttrekken.

  De Wet van de Heer is volmaakt: zij schenkt aan de ziel nieuw leven;~\sep\ het gebod van de Heer staat vast, het onderricht de eenvoudige.

  De voorschriften van de Heer zijn rechtmatig: een vreugde voor het hart;~\sep\ het bevel van de Heer is rein: een licht voor de ogen.

  De vrees van de Heer is zuiver: zij blijft eeuwig bestaan;~\sep\ de oordelen van de Heer zijn waarachtig: alle even rechtvaardig,

  Te verkiezen boven goud en schatten van het edelst metaal,~\sep\ en zoeter dan honing en druipend honingzeem.

  Al wijdt er Uw dienaar zijn aandacht aan,~\sep\ en onderhoudt hij ze vol ijver,

  Wie kent er zijn fouten?~\sep\ Reinig mij van die mij verborgen zijn!

  Behoed ook Uw dienaar voor hoogmoed;~\sep\ dat hij mij niet beheerse!

  Dan zal ik zuiver zijn,~\sep\ en rein van zware misdaad.

  Mogen de woorden van mijn mond en de overweging van mijn hart welgevallig zijn,~\sep\ voor Uw aanschijn, O Heer, mijn Rots en mijn Redder!

  \psalm{\Ps{20}} Dat de Heer u verhore in tijden van nood,~\sep\ u bescherme de Naam van Jacobs God!

  Hij zende u hulp uit het heiligdom,~\sep\ en uit Sion steune Hij u.

  Hij gedenke al uw offergaven,~\sep\ en uw brandoffer behage Hem.

  Hij schenke u wat uw hart begeert,~\sep\ en doe al uw plannen slagen.

  Mogen we over uw zegepraal juichen, in de Naam van onze God de banieren verheffen;~\sep\ dat de Heer al uw beden vervulle!

  Reeds weet ik dat de Heer Zijn gezalfde de zegepraal schonk,~\sep\ hem heeft verhoord vanuit Zijn heilige hemel door de kracht van Zijn verwinnende rechterhand.

  Dat anderen vertrouwen op strijdwagens, anderen op rossen,~\sep\ wij echter zijn sterk door de Naam van de Heer, onze God.

  Zij zijn gevallen en liggen terneer,~\sep\ maar wij houden onwankelbaar stand.

  O Heer, schenk de koning de zege,~\sep\ en verhoor ons op de dag van ons smeken.

  \psalm{\Ps{21}} Over Uw macht, O Heer, verheugt zich de koning,~\sep\ hoe uitbundig jubelt hij over Uw hulp!

  Zijn hartewens hebt Gij verhoord,~\sep\ en de bede van zijn lippen niet afgewezen.

  Ja, Gij hebt hem voorkomen met rijke zegen,~\sep\ een kroon van zuiver goud hem op het hoofd gedrukt.

  Leven vroeg hij U; Gij hebt hem gegeven,~\sep\ lengte van dagen voor immer.

  Groot is zijn roem, dank aan Uw hulp;~\sep\ met majesteit en luister hebt Gij hem getooid,

  Ja, Gij hebt hem voor eeuwig overladen met zegening,~\sep\ hem met vreugde overstelpt voor Uw aanschijn.

  Want de koning vertrouwt op de Heer,~\sep\ en door de gunst van de Allerhoogste zal hij niet wankelen.

  Moge Uw hand al Uw vijanden treffen,~\sep\ Uw rechter aangrijpen al die U haten.

  Maak ze als tot een gloeiende oven,~\sep\ wanneer Gij Uw aanschijn zult tonen.

  Dat de Heer hen in Zijn toorn vertere,~\sep\ en het vuur hen verslinde.

  Verdelg hun kroost op aarde,~\sep\ en hun zaad onder de kinderen der mensen.

  Als zij U kwaad willen doen, listige plannen beramen,~\sep\ zullen zij niets vermogen;

  Want Gij zult ze doen vluchten,~\sep\ Uw boog op hun aangezicht richten.

  Rijs op, o Heer, in Uw kracht!~\sep\ Wij zullen Uw macht bezingen en prijzen.
\end{halfparskip}

\begin{halfparskip}
  \markedsubsectionrubricwithhint{Dinsdagen ``na'': Marmita 15*}{(origineel: hulala 5)}

  \psalm{\Ps{38}} Heer, straf mij niet in Uw toorn,~\sep\ en in Uw gramschap kastijd mij niet.

  \liturgicalhint{Alleluia, Alleluia, Alleluia.~--- Eerste vers.}

  Want Uw pijlen zijn in mij doorgedrongen,~\sep\ en Uw hand is op mij neergedaald.

  Door Uw toorn is er geen plek meer gezond aan mijn vlees,~\sep\ in mijn gebeente niets gaafs vanwege mijn zonde.

  Want mijn zonden zijn mij boven het hoofd gestegen:~\sep\ al te zeer drukken ze mij als een zware last.

  Kwalijk rieken mijn builen in hun ontbinding,~\sep\ vanwege mijn dwaasheid.

  Ik ben gebogen en diep gekromd;~\sep\ ik sleep mij treurend heel de dag voort.

  Want mijn lendenen zijn geheel ontstoken,~\sep\ en geen gezonde plek is er nog aan mijn vlees.

  Ik kwijn weg en ben geheel gebroken;~\sep\ van hartzeer snik ik het uit.

  Heer, voor Uw aanschijn ligt heel mijn verlangen open,~\sep\ en mijn zuchten is niet verborgen voor U.

  Mijn hart klopt hevig, mijn kracht heeft mij begeven,~\sep\ en het licht van mijn ogen, ook dat moet ik derven.

  Mijn vrienden en gezellen houden zich ver van mijn wonde,~\sep\ mijn verwanten blijven op afstand staan.

  Die mijn leven belagen, spannen strikken, die mij kwaad willen dreigen met ondergang;~\sep\ voortdurend zijn ze uit op bedrog.

  Ik echter ben als een dove en hoor niet,~\sep\ als een stomme, die zijn mond niet opent.

  Ik ben geworden als een mens, die niet hoort,~\sep\ en wiens mond het antwoord schuldig blijft.

  Op U toch, Heer, vertrouw ik;~\sep\ Gij, O Heer, mijn God, zult mij verhoren.

  Ik zeg immers: ``Laten zij zich niet vrolijk over mij maken,~\sep\ niet schamper tegen mij uitvallen, als mijn voet soms wankelt.''

  Want ja, ik ben de val nabij,~\sep\ en mijn smart staat mij steeds voor ogen.

  En inderdaad, ik belijd mijn schuld,~\sep\ en ben vol kommer vanwege mijn zonde.

  Maar machtig zijn zij, die mij zonder reden bestrijden,~\sep\ en talrijk, die mij ten onrechte haten.

  En die goed met kwaad vergelden,~\sep\ vallen mij aan, omdat ik het goede volg.

  Heer, verlaat mij toch niet,~\sep\ mijn God, sta niet ver van mij af!

  \psalm{\Ps{39}} Ik sprak: Over mijn wandel zal ik waken,~\sep\ om niet te zondigen met mijn tong;~\sep

  Mijn mond zal ik beteugelen,~\sep\ zolang de boze voor mij staat.

  Zwijgend bleef ik en stom, verstoken van geluk,~\sep\ maar des te feller werd mijn smart.

  In mijn binnenste gloeide mijn hart; als ik nadacht, laaide het vuur in mij op;~\sep\ mijn tong begon te spreken.

  Heer, doe mij mijn einde kennen, en het getal van mijn dagen,~\sep\ opdat ik wete hoe vergankelijk ik ben.

  Zie, enkele handpalmen lang hebt Gij mijn dagen gemaakt, en als een niet is mijn leven voor U;~\sep\ als een ademtocht slechts leeft iedere mens.

  Als een schaduw slechts gaat de mens voorbij; hij is in verwarring om niets;~\sep\ schatten stapelt hij op, maar weet niet wie ze zal krijgen.

  En nu, Heer, wat kan ik nog verwachten?~\sep\ Op U is mijn vertrouwen gesteld.

  Verlos mij van al mijn zonden,~\sep\ lever mij niet over aan de spot van de dwaas.

  Ik zwijg en doe mijn mond niet open,~\sep\ want Gij deedt het mij aan.

  Wend toch Uw slagen van mij af;~\sep\ ik bezwijk onder de druk van Uw hand.

  Met straf voor schuld kastijdt Gij de mens; Gij verteert als de motten zijn kostbaarheden:~\sep\ een ademtocht slechts is iedere mens.

  Luister, Heer, naar mijn gebed, en hoor naar mijn smeken;~\sep\ blijf niet doof voor mijn snikken.

  Immers, ik ben een gast bij U,~\sep\ een pelgrim gelijk al mijn vaderen.

  Wend Uw ogen van mij af, opdat ik nog ademe,~\sep\ eer ik heenga en niet meer ben.

  \psalm{\Ps{40}} Ik heb gehoopt, gehoopt op de Heer, en Hij boog zich naar mij,~\sep\ en verhoorde mijn smeken.

  Hij trok mij op uit de kuil van de dood, uit modder en slijk; op de rots heeft Hij mijn voeten geplaatst,~\sep\ mijn schreden heeft Hij gesteund.

  Hij legde een nieuw lied in mijn mond,~\sep\ een lofzang voor onze God.

  Velen zullen het zien, vervuld van ontzag,~\sep\ en zullen op de Heer vertrouwen.

  Gelukkig de man, die op de Heer zijn hoop heeft gesteld,~\sep\ geen dienaars van afgoden volgt, noch hen, die tot verzinsels zich wenden.

  Talrijk, O Heer, mijn God, hebt Gij Uw wonderwerken gemaakt,~\sep\ en in Uw raadsbesluiten over ons is niemand U gelijk.

  Wilde ik ze verhalen en verkondigen:~\sep\ ze zijn te talrijk om te worden geteld.

  Slacht-- noch spijsoffer hebt Gij gewild,~\sep\ maar Gij hebt mij de oren geopend.

  Brand-- noch zoenoffer hebt Gij voor de zonde geëist:~\sep\ toen heb ik gezegd: ``Zie, ik kom; in de boekrol staat over mij geschreven:

  Het is mijn geneugte, mijn God, Uw wil te volbrengen,~\sep\ en diep in mijn hart staat Uw Wet gegrift.''

  De gerechtigheid heb ik verkondigd in de volle vergadering;~\sep\ neen, Heer, Gij weet het: mijn lippen hield ik niet gesloten.

  Uw gerechtigheid verborg ik niet in mijn hart;~\sep\ Uw trouw en Uw hulp heb ik verkondigd;

  Uw goedheid hield ik niet geheim,~\sep\ noch in de volle vergadering Uw trouw.

  Gij dan, O Heer, onthoud mij Uw erbarming niet;~\sep\ laat Uw genade en trouw mij immer behoeden.

  Want talloze rampen hebben mij omgeven, mijn zonden hebben mij aangegrepen,~\sep\ zodat ik niet kan zien;

  Talrijker zijn ze dan de haren op mijn hoofd:~\sep\ en de moed is mij ontzonken.

  Het behage U, O Heer, mij te verlossen;~\sep\ Heer, snel mij te hulp!

  Laten allen te schande en schaamrood worden,~\sep\ die zoeken mij van het leven te beroven.

  Terugwijken vol schaamte mogen zij,~\sep\ die zich over mijn rampen verheugen,

  Dat zij verstommen, met schande bedekt,~\sep\ die mij toeroepen: Ha, ha!

  In U mogen jubelen vol blijdschap allen die U zoeken;~\sep\ dat zij, die uitzien naar Uw hulp, immer herhalen: ``Hooggeprezen zij de Heer.''

  Ach, ik ben ellendig en arm,~\sep\ maar de Heer is vol zorg voor mij.

  Mijn Helper en Redder zijt Gij;~\sep\ mijn God, wil niet toeven!
\end{halfparskip}

\begin{halfparskip}
  \markedsubsectionrubricwithhint{Woensdagen ``na'': Marmita 21*}{(origineel: hulala 8)}

  \psalm{\Ps{56}} Wees mij genadig, God, want de mensen vertrappen mij;~\sep\ zij bestrijden en verdrukken mij al maar door.

  \liturgicalhint{Alleluia, Alleluia, Alleluia.~--- Eerste vers.}

  Voortdurend vertreden mij mijn vijanden,~\sep\ want velen zijn het, die tegen mij strijden.

  Allerhoogste, als vrees mij bevangt,~\sep\ stel ik toch mijn vertrouwen op U.

  Op God, wiens belofte ik verheerlijk, op God vertrouw ik, ik zal niet vrezen:~\sep\ wat kan een mens mij aandoen?

  De hele dag kleineren ze mij,~\sep\ tegen mij zijn al hun gedachten ten kwade gericht.

  Zij komen tezamen, belagen mij,~\sep\ gaan mijn gangen na en staan mij naar het leven.

  Zet ze hun boosheid betaald,~\sep\ werp, O God, de volkeren neer in Uw toorn.

  De wegen van mijn ballingschap zijn U bekend; geborgen zijn mijn tranen in Uw lederen zak;~\sep\ zijn ze in Uw boek niet opgetekend?

  Dan zullen mijn vijanden terug moeten wijken, telkens wanneer ik U aanroep;~\sep\ dit weet ik goed: God is met mij.

  Op God, wiens belofte ik verheerlijk, op God vertrouw ik, ik zal niet vrezen:~\sep\ wat kan een mens mij aandoen?

  Ik ben gehouden, O God, door geloften, die ik U heb gedaan:~\sep\ inlossen zal ik U de offers van lof,

  Want Gij hebt mijn leven ontrukt aan de dood, en mijn voet bewaard voor de val,~\sep\ opdat ik wandele voor God in het licht der levenden.

  \psalm{\Ps{57}} Wees mij genadig, O God, wees mij genadig,~\sep\ want mijn ziel neemt haar toevlucht tot U,

  En ik schuil in de schaduw van Uw vleugels,~\sep\ totdat het onheil voorbijgaat.

  Ik roep tot God, de Allerhoogste,~\sep\ tot God, die mij weldoet.

  Hij zende uit de hemel mij redding, en overlade mijn vervolgers met smaad;~\sep\ God zende Zijn liefde en Zijn trouw.

  Ik lig neer te midden van leeuwen,~\sep\ die gulzig de kinderen der mensen verslinden.

  Hun tanden zijn lansen en pijlen,~\sep\ een vlijmend zwaard is hun tong.

  Verschijn hoog boven de hemelen, O God,~\sep\ dat Uw glorie strale over heel de aarde.

  Zij spanden een net voor mijn voeten,~\sep\ en drukten mij neer;

  Zij groeven een kuil voor mij;~\sep\ dat zij zelf er in vallen!

  Gesterkt is mijn hart, O God, gesterkt is mijn hart;~\sep\ ik zal zingen en het psalter bespelen.

  Ontwaak toch, mijn ziel, gij, psalter en citer, ontwaakt;~\sep\ ik wil de dageraad wekken.

  Onder de volken wil ik U prijzen, O Heer,~\sep\ U bezingen onder de naties met snarenspel.

  Want hoog tot de hemel reikt Uw erbarming,~\sep\ en tot de wolken Uw trouw.

  Verschijn hoog boven de hemelen, O God,~\sep\ en dat Uw glorie strale over heel de aarde.

  \psalm{\Ps{58}} Spreekt gij nu waarlijk recht, gij machtigen,~\sep\ oordeelt gij rechtvaardig, gij mensenkinderen?

  Veeleer pleegt gij ongerechtigheid in uw hart;~\sep\ uw handen zaaien onrecht in het land.

  Reeds van de moederschoot weken de goddelozen af van het pad,~\sep\ reeds van hun geboorte dwaalden de leugenaars.

  Hun venijn is als slangengif,~\sep\ als het gif van een dove adder, die haar oren stopt,

  Om de stem niet te horen van de bezweerders,~\sep\ noch de toverban van de bedreven belezer.

  O God, verbrijzel de tanden in hun mond,~\sep\ sla de leeuwenkiezen stuk, O Heer.

  Laat ze verdwijnen als water, dat wegvloeit;~\sep\ mogen hun pijlen afstompen, wanneer zij richten.

  Dat ze vergaan als de slak, die wegsmelt,~\sep\ als een misdracht, die het zonlicht niet heeft gezien.

  Eer Uw ketels de doornstruik voelen,~\sep\ terwijl deze nog groen is, vage de gloeiende wervelwind hem weg.

  De rechtvaardige zal zich verblijden bij het zien van de wraak;~\sep\ zijn voeten zal hij wassen in het bloed van de boze.

  En de mensen zullen zeggen: Ja waarlijk, de rechtschapene plukt zijn vruchten;~\sep\ waarlijk, er is nog een God, die rechtspreekt op aarde.
\end{halfparskip}

\begin{halfparskip}
  \markedsubsectionrubricwithhint{Donderdagen ``na'': Marmita 32*}{(origineel: hulala 11)}

  \psalm{\Ps{82}} God rijst op in de goddelijke raad, Hij houdt gericht te midden der goden.

  \liturgicalhint{Alleluia, Alleluia, Alleluia.~--- Eerste vers.}

  ``Hoe lang nog zult gij onrechtvaardig oordelen,~\sep\ en de zaak der bozen begunstigen?

  Verdedigt verdrukten en wezen,~\sep\ geeft aan ellendigen en armen hun recht,

  Bevrijdt verdrukten en behoeftigen:~\sep\ ontrukt ze aan de hand van de bozen.''

  Zij hebben inzicht noch verstand, ze wandelen in duisternis;~\sep\ alle grondslagen der aarde worden  geschokt.

  Ik heb gezegd: ``Goden zijt gij,~\sep\ en zonen van de Allerhoogste, gij allen.

  Toch zult gij sterven als mensen,~\sep\ neervallen als welke machthebber ook''.

  Rijs op, O God, en richt de aarde,~\sep\ want rechtens horen alle volken U toe.

  \psalm{\Ps{83}} Wil niet zwijgen, O Heer,~\sep\ wil niet zwijgen, O God, noch rusten!

  Want zie, Uw vijanden woelen,~\sep\ en die U haten, steken het hoofd omhoog.

  Tegen Uw volk smeden zij plannen,~\sep\ en tegen Uw beschermelingen spannen zij samen.

  ``Komt'', zo spreken zij, ``vagen wij hen weg uit de rij van de volken,~\sep\ en dat men de naam van Israël niet meer gedenke.''

  Waarlijk, zij overleggen eensgezind,~\sep\ en sluiten een verbond tegen U:

  De tenten van Edom en de Ismaëlieten,~\sep\ Moab en de zonen van Hagar,

  Gebal, Ammon en Amalec,~\sep\ de Filistijnen en de bewoners van Tyrus;

  Ook de Assyriërs verbonden zich met hen,~\sep\ en leenden hun arm aan de zonen van Lot.

  Behandel hen als Madian,~\sep\ als Sisara en Jabin bij de beek Cison,

  Die bij Endor werden verdelgd,~\sep\ en tot mest op het veld zijn gemaakt.

  Maak hun vorsten als Oreb en Zeb,~\sep\ als Zebëe en Salmana, al hun leiders,

  Die zeiden:~\sep\ ``Laten wij het gebied van God gaan bezetten.''

  Maak ze, mijn God, als bladeren die dwarrelen in de storm,~\sep\ als een strohalm, heen en weer gezwiept door de wind,

  Als vuur, dat bossen verteert,~\sep\ en als een vlam, die bergen verschroeit,

  Zo moogt Gij hen vervolgen met Uw stormwind,~\sep\ door Uw orkaan hen verwarren.

  Overdek hun gelaat met schande,~\sep\ opdat zij Uw Naam zoeken, O Heer;

  Laat ze beschaamd en verbijsterd staan voor eeuwig,~\sep\ laat ze te schande worden en vergaan.

  Dat zij erkennen dat Gij, wiens naam de Heer is,~\sep\ de enig Verhevene zijt over heel de aarde.

  \psalm{\Ps{84}} Hoe liefelijk is Uw woonstede, Heer der legerscharen;~\sep\ mijn ziel verlangt, ziet smachtend uit naar de voorhoven van de Heer.

  Mijn hart en mijn lichaam,~\sep\ juichen voor de levende God.

  Ook de mus vindt een woning,~\sep\ en de zwaluw een nest, waar ze haar jongen in neerlegt.

  Uw altaren, O Heer der legerscharen,~\sep\ mijn Koning en mijn God!

  Gelukkig zij, die wonen in Uw huis, O Heer;~\sep\ eeuwig loven zij U.

  Gelukkig de man, die hulp krijgt van U,~\sep\ als hij het plan heeft op bedevaart te gaan:

  Trekken zij door een dorre vallei, dan maken zij haar tot bron,~\sep\ en de vroege regen bekleedt haar met zegeningen.

  Al gaande zal hun kracht vermeerderen:~\sep\ de God der goden zullen zij in Sion zien.

  Heer der legerscharen, hoor naar mijn bede,~\sep\ ach, luister toch, O God van Jacob.

  Zie toe, O God, ons schild,~\sep\ en zie op het gelaat van Uw gezalfde.

  Waarlijk, één dag in Uw voorhoven is beter,~\sep\ dan duizend andere.

  Liever blijf ik staan op de drempel van het huis van mijn God,~\sep\ dan te toeven in de tenten der bozen.

  Want een zon en een schild is God de Heer;~\sep\ de Heer schenkt genade en glorie.

  Hij weigert het goede niet,~\sep\ aan die in onschuld wandelen.

  Heer der legerscharen,~\sep\ gelukkig de mens, die op U vertrouwt.
\end{halfparskip}

\begin{halfparskip}
  \markedsubsectionrubricwithhint{Vrijdagen ``na'': Marmita 45*}{(origineel: hulala 17)}

  \psalm{\Ps{109}} God, mijn roem, wil toch niet zwijgen,~\sep\ want een mond vol boosheid en bedrog zetten zij tegen mij op.

  \liturgicalhint{Alleluia, Alleluia, Alleluia.~--- Eerste vers.}

  Zij spraken mij toe met leugentaal, en met woorden van haat omringden zij mij,~\sep\ en bestreden mij zonder reden.

  In ruil voor mijn liefde klaagden zij mij aan,~\sep\ maar ik bleef bidden.

  Zij vergolden mij goed met kwaad,~\sep\ en met haat mijn liefde.

  Wek een booswicht tegen hem op,~\sep\ en een aanklager sta aan zijn rechterhand.

  Hij kome veroordeeld uit het gerecht,~\sep\ en zijn bede om genade zij vruchteloos.

  Kort zij het aantal van zijn dagen,~\sep\ en zijn ambt vervalle aan een ander.

  Dat zijn kinderen wezen worden,~\sep\ en een weduwe zijn vrouw.

  Mogen zijn kinderen als bedelaars zwerven,~\sep\ uit hun vernielde huizen worden verjaagd.

  Een woekeraar loere op al zijn bezit,~\sep\ en dat vreemden plunderen de vrucht van zijn arbeid.

  Dat niemand hem barmhartigheid bewijze,~\sep\ noch zijn wezen genadig zij.

  Zijn nageslacht zij ten ondergang gedoemd,~\sep\ in het volgend geslacht verdwijne reeds hun naam.

  Dat de Heer de misdaad van zijn vaderen gedenke,~\sep\ en de zonde van zijn moeder niet worde uitgewist.

  De Heer houde ze immer voor ogen,~\sep\ en verdelge hun aandenken van de aarde.

  Want hij dacht er niet aan barmhartig te zijn,~\sep\ maar vervolgde de ellendige en behoeftige mens, en de bedroefde van harte, ten dode toe.

  Vervloeking had hij lief: die kome op hem neer;~\sep\ zegen versmaadde hij: laat die van hem wijken!

  Hij trekke de vloek aan als een gewaad,~\sep\ die moge als water in zijn binnenste dringen, als olie tot in zijn gebeente;

  Hij zij hem tot kleed, dat hem omhult,~\sep\ als een gordel, die hem immer omsluit.

  Dit zij van de Heer het loon voor wie mij beschuldigen,~\sep\ en kwaadspreken tegen mijn ziel.

  Maar Gij, Heer God, handel met mij omwille van Uw Naam,~\sep\ red mij, daar Uw barmhartigheid welwillend is.

  Want ik ben ellendig en arm,~\sep\ en mijn hart is gewond in mijn boezem.

  Ik verdwijn als een schaduw, die lengt,~\sep\ en als een sprinkhaan schudt men mij af.

  Van het vasten knikken mijn knieën,~\sep\ mijn vermagerd lichaam teert uit.

  Ik ben hun tot smaad geworden,~\sep\ zij schudden het hoofd, als zij mij zien.

  Wil mij toch helpen, O Heer, mijn God;~\sep\ red mij naar Uw barmhartigheid.

  Dat zij erkennen dat dit Uw hand is,~\sep\ dat Gij, O Heer, dit hebt gedaan.

  Zij mogen vloeken maar Gij: schenk Uw zegen;~\sep\ wie tegen mij opstaat worde te schande maar dat Uw dienaar vol vreugde zij.

  Laat mijn aanklagers met smaad worden omkleed,~\sep\ en als met een mantel in hun schande gehuld.

  De Heer zal ik loven met luide stem,~\sep\ temidden van velen Hem prijzen,

  Want Hij stond aan de rechterhand van de arme,~\sep\ om hem van de rechters te redden.

  \psalm{\Ps{110}} De Heer sprak tot mijn Heer: ``Zit aan mijn rechterhand,~\sep\ totdat Ik Uw vijanden leg als een rustbank voor Uw voeten!''

  De Heer zal van Sion uit Uw machtige scepter uitstrekken:~\sep\ ``Voer heerschappij te midden van Uw vijanden!

  Bij U is het oppergezag in heilige luister op de dag van Uw oorsprong;~\sep\ als de dauw heb Ik U voortgebracht vóór de morgenster.''

  De Heer heeft gezworen en het zal Hem niet berouwen: ``\sep\ Gij zijt priester voor eeuwig naar de wijze van Melchisedech.''

  De Heer staat aan Uw rechterhand;~\sep\ Hij zal vorsten verpletteren op de dag van Zijn toorn.

  De volkeren zal Hij richten, lijken ophogen,~\sep\ wijd over de aarde de hoofden verbrijzelen.

  Uit een beek langs de weg zal Hij drinken,~\sep\ en daarom zal Hij Zijn hoofd verheffen.

  \psalm{\Ps{111}} De Heer wil ik prijzen uit heel mijn hart,~\sep\ in de kring der rechtvaardigen en in de vergadering.

  Groot zijn de werken van de Heer,~\sep\ waard te worden doorvorst door allen, die ze beminnen.

  Majesteit en luister is Zijn werk,~\sep\ en Zijn gerechtigheid duurt eeuwig.

  Zijn wonderen maakte Hij vermeldenswaard;~\sep\ barmhartig en mild is de Heer.

  Spijs gaf Hij aan die Hem vrezen;~\sep\ Zijn verbond blijft Hij eeuwig indachtig.

  Hij toonde Zijn volk Zijn machtige daden,~\sep\ door hun het erfdeel der heidenen te schenken.

  De werken van Zijn handen zijn trouw en gerechtig,~\sep\ al Zijn bevelen staan vast.

  Bevestigd voor eeuwig en immer,~\sep\ gemaakt met kracht en billijkheid.

  Hij heeft Zijn volk verlossing gebracht, Zijn verbond gesloten voor eeuwig;~\sep\ Zijn Naam is heilig en eerbiedwaardig.

  Het begin van de wijsheid is de vreze van de Heer: verstandig handelt al wie ze betracht;~\sep\ in eeuwigheid duurt Zijn roem.
\end{halfparskip}

\begin{halfparskip}
  \markedsubsectionrubricwithhint{Zaterdagen ``na'': Marmita 54*}{(origineel: hulala 20)}

  \psalm{\Ps{138}} Heer, Gij doorvorst en Gij kent mij;~\sep\ of ik neerzit of oprijs, Gij kent mij,

  \liturgicalhint{Alleluia, Alleluia, Alleluia.~--- Eerste vers.}

  Gij doorschouwt mijn gedachten van verre, of ik wandel of te ruste lig, Gij ziet het~\sep\ en let op al mijn wegen.

  Nog ligt het woord niet op mijn tong,~\sep\ of zie, het is U reeds ten volle bekend, O Heer.

  Achter en voor omsluit Gij mij,~\sep\ en laat Uw hand op mij rusten.

  Te wonderbaar is deze kennis voor mij,~\sep\ te verheven: ze gaat mijn begrip te boven.

  Waarheen kan ik gaan, ver van Uw geest,~\sep\ waarheen voor Uw aanschijn vluchten?

  Stijg ik op ten hemel: Gij zijt er;~\sep\ leg ik mij neer in het dodenrijk: Gij zijt er.

  Neem ik de vleugels van de dageraad,~\sep\ ga ik wonen aan de grenzen der zee,

  Ook daar geleidt mij Uw hand,~\sep\ en houdt Uw rechter mij vast.

  Zeg ik: ``Het duister zal tenminste mij dekken,~\sep\ en de nacht mij omhullen als licht,''

  Maar zelfs het duister zal voor U niet duister zijn, en de nacht zal stralen als de dag:~\sep\ voor U is het duister als licht.

  Want Gij hebt mijn nieren geschapen,~\sep\ mij geweven in de schoot van mijn moeder.

  Ik prijs U, daar ik zo wondervol ben gevormd,~\sep\ daar Uw werken wonderbaar zijn.

  Ook mijn ziel kent Gij ten volle,~\sep\ mijn lichaam was voor U niet verborgen,

  Toen ik in het verborgene gevormd werd,~\sep\ geweven in de diepten der aarde.

  Uw ogen aanschouwden mijn daden, en in Uw boek staan ze alle vermeld;~\sep\ bepaald waren mijn dagen voor er één van bestond.

  Hoe diep zijn voor mij Uw gedachten, O God,~\sep\ hoe overweldigend is haar getal!

  Zou ik ze tellen: talrijker zijn ze dan zandkorrels;~\sep\ kwam ik er mee ten einde: nog ben ik bij U.

  Ontneem toch de bozen het leven, O God,~\sep\ en laat de bloeddorstigen wijken van mij.

  Want vol bedrog staan tegen U op,~\sep\ trouweloos verheffen zich Uw vijanden.

  Moet ik niet haten, die U haten, O Heer,~\sep\ niet walgen van hen, die tegen U opstaan?

  Ja, ik haat ze met felle haat,~\sep\ mijn eigen vijanden zijn ze geworden.

  Doorvors mij, O God, en doorgrond mijn hart;~\sep\ toets mij en doorschouw mijn gedachten.

  En zie of ik de verkeerde weg bewandel,~\sep\ en leid mij langs het aloude pad.

  \psalm{\Ps{139}} Verlos mij, Heer, van de boze,~\sep\ behoed mij voor de geweldenaar,

  Voor wie kwaad beramen in hun hart,~\sep\ twist verwekken dag aan dag,

  Die als slangen hun tongen scherpen:~\sep\ addergif schuilt er onder hun lippen.

  Red mij, Heer, uit de hand van de boze,~\sep\ behoed mij voor de geweldenaar;

  Die mij de voet willen lichten, de trotsaards,~\sep\ zij spannen mij heimelijk een strik,

  Zij spannen als een net hun koorden,~\sep\ leggen mij valstrikken langs de weg.

  Ik zeg tot de Heer: Mijn God zijt Gij;~\sep\ luister, O Heer, naar de stem van mijn smeken.

  Heer, God, mijn krachtige hulp,~\sep\ Gij beschut mijn hoofd op de dag van de strijd.

  O Heer, vervul toch de wensen van de boze niet,~\sep\ laat toch zijn plannen niet slagen!

  Zij verheffen het hoofd, die rondom mij staan;~\sep\ de boosheid van hun lippen overstelpe hen!

  Laat gloeiende kolen op hen regenen,~\sep\ Hij werpe hen in de kuil, zodat ze niet meer opstaan.

  Geen kwaadspreker zal standhouden op aarde;~\sep\ de geweldenaar zullen plotseling rampen treffen.

  Ik weet dat de Heer recht verschaft aan de behoeftige,~\sep\ gerechtigheid aan de armen.

  Ja, de gerechtigen zullen Uw Naam verheerlijken,~\sep\ de deugdzamen voor Uw aanschijn wonen.

  \psalm{\Ps{140}} Ik roep tot U, O Heer; snel mij te hulp;~\sep\ luister naar mijn smeken, wanneer ik tot U roep.

  Laat mijn bede als een reukoffer opgaan tot U,~\sep\ het heffen van mijn handen als een avondoffer zijn.

  Heer, zet een wacht voor mijn mond,~\sep\ een post voor de deur van mijn lippen.

  Neig mijn hart niet tot kwaad,~\sep\ om boze daden te stellen;

  En geef dat ik nooit met boosdoeners,~\sep\ hun uitgezochte spijzen eet.

  Laat de rechtvaardige mij slaan: dat is liefde;~\sep\ mij berispen: dat is olie op mijn hoofd,

  Die mijn hoofd niet zal weigeren;~\sep\ maar immer zal ik bidden onder hun kastijding.

  Hun vorsten werden neergelaten langs de rots,~\sep\ en zij hoorden hoe zachtzinnig mijn woorden waren.

  Zoals wanneer men de grond doorploegt en scheurt,~\sep\ zo liggen hun beenderen verstrooid bij de poort van het dodenrijk.

  Maar op U, Heer God, zijn mijn ogen gericht,~\sep\ naar U vlucht ik heen: laat mij niet vergaan.

  Behoed mij voor het net, dat ze mij spanden,~\sep\ en voor de valstrikken van hen, die het kwade bedrijven.

  Laat de bozen tezamen in hun eigen netten vallen,~\sep\ terwijl ik behouden ontkom.
\end{halfparskip}

\begin{halfparskip}
  \liturgicalOption{Alle dagen:}~\sep\ \dd~Alleluia, alleluia; Eer aan U, God, alleluia; Eer aan U, God, alleluia; Heer, ontferm U over ons. Laat ons bidden; vrede zij met ons.

  \cc~Moge de geheime kracht, o Heer, van Uw Godheid, de wonderbaarlijke hulp van Uw Majesteit, en de grote hulp van Uw genade, de zwakheid van onze broze natuur versterken om te allen tijde U te loven, te eren, te belijden en te aanbidden, Heer van alles...
\end{halfparskip}

% % % % % % % % % % % % % % % % % % % % % % % % % % % % % % % % % % % % % % % %

\markedsection{Derde marmita}

\begin{halfparskip}
  \markedsubsectionrubricwithhint{Maandagen ``voor'': Marmita 3*}{(origineel: hulala 3)}

  \psalm{\Ps{8}} Heer, onze Heer, hoe wonderbaar is Uw Naam over heel de aarde;~\sep\ boven de hemelen hebt Gij Uw Majesteit verheven.

  \liturgicalhint{Alleluia, Alleluia, Alleluia.~--- Eerste vers.}

  Ten spijt van Uw weerstrevers hebt Gij U lof bereid uit de mond van kind en zuigeling,~\sep\ om te beteugelen Uw vijand en hater.

  Als ik Uw hemelen zie, het werk van Uw vingeren,~\sep\ maan en sterren, die Gij hebt gegrondvest;

  Wat is dan de mens, dat Gij hem gedenkt,~\sep\ of een mensenkind, dat Gij zorg voor hem draagt?

  Toch hebt Gij hem weinig minder dan de engelen gemaakt,~\sep\ met glorie en eer hem gekroond.

  Gij schonkt hem macht over de werken van Uw handen,~\sep\ alles hebt Gij onder zijn voeten gelegd:

  Alle schapen en runderen,~\sep\ ook de dieren in het wild,

  De vogels in de lucht en de vissen in de zee:~\sep\ al wat de paden der zeeën doorwandelt.

  Heer, onze Heer,~\sep\ hoe wonderbaar is Uw Naam over heel de aarde!

  \psalm{\Ps{9}} Prijzen wil ik U, Heer, uit heel mijn hart,~\sep\ verhalen al Uw wonderwerken.

  Om U wil ik juichen en jubelen,~\sep\ Uw Naam, Allerhoogste, bezingen.

  Want teruggeweken zijn mijn vijanden,~\sep\ ze zijn gevallen en kwamen om voor Uw aanschijn.

  Want Gij hebt mijn recht en mijn rechtszaak behartigd,~\sep\ Gij waart gezeten op Uw troon, als rechtvaardige Rechter.

  Gij hebt de heidenen getuchtigd, de goddeloze doen omkomen,~\sep\ hun naam uitgewist voor eeuwig.

  De vijanden kwamen om, voor immer ten onder gebracht;~\sep\ de steden hebt Gij verwoest: de gedachtenis aan hen is vergaan.

  Maar de Heer troont in eeuwigheid,~\sep\ en heeft Zijn rechterstoel gevestigd ten oordeel.

  En Hij zelf zal de wereld oordelen volgens recht,~\sep\ de volken richten volgens billijkheid.

  De Heer zal voor de verdrukte een toevlucht zijn,~\sep\ een veilige toevlucht in bange tijden.

  En die Uw Naam kennen, zullen hopen op U,~\sep\ want die U zoeken, Heer, verlaat Gij niet.

  Bezingt de Heer, die woont in Sion,~\sep\ verkondigt aan de volken Zijn daden.

  Want Hij, die bloedschuld wreekt, was hun indachtig,~\sep\ de kreten der armen vergeet Hij niet.

  Wees mij genadig, Heer, zie de ellende die ik lijd vanwege mijn vijanden,~\sep\ Gij die mij terugvoert van de poorten van de dood;

  Opdat ik al Uw lof verkondige bij de poorten der dochter van Sion,~\sep\ en juiche over Uw hulp.

  Bedolven zijn de volken in de kuil, die zij groeven;~\sep\ hun voet zit verward in de strik, die zij heimelijk hebben gelegd.

  De Heer heeft Zich geopenbaard, Hij heeft recht gesproken;~\sep\ in de werken van zijn handen ligt de zondaar verstrikt.

  Dat de bozen neerdalen in het dodenrijk,~\sep\ alle volken, die God zijn vergeten.

  Neen, niet voor immer zal de arme aan de vergetelheid worden prijsgegeven,~\sep\ niet voor immer de hoop der ellendigen worden beschaamd.

  Heer, sta op, laat de mens niet overmachtig worden;~\sep\ laten de heidenen voor Uw aanschijn worden gericht.

  Sla hen met ontzetting, Heer;~\sep\ dat de heidenen inzien, dat zij maar mensen zijn.

  \psalm{\Ps{10}} Waarom, Heer, blijft Gij veraf,~\sep\ verbergt Gij U in bange tijden.

  Terwijl de goddeloze zich trots verheft, de arme verdrukt wordt~\sep\ en verstrikt ligt in de listen, die hij heeft uitgedacht.

  Ziet, de boze gaat groot op zijn driften,~\sep\ en de woekeraar lastert en minacht de Heer.

  De goddeloze spreekt in trots gemoed: ``Hij zal niet straffen;~\sep\ er bestaat geen God'': dit is heel zijn
  gedachtengang.

  Zijn wandel is immer voorspoedig,~\sep\ hij bekommert zich niet om Uw gerichten, al zijn weerstrevers veracht hij.

  Hij zegt bij zichzelf: ``Ik zal niet wankelen,~\sep\ van geslacht tot geslacht zal geen onheil mij treffen.''

  Zijn mond is vol verwensingen, vol list en bedrog;~\sep\ smart en kwelling kleeft aan zijn tong.

  Hij ligt bij de dorpen in hinderlaag, doodt de onschuldige in het geheim;~\sep\ zijn ogen bespieden de ongelukkige.

  Hij ligt op de loer in zijn schuilplaats als een leeuw in zijn hol; hij bespiedt de ongelukkige om hem te grijpen:~\sep\ hij sleept de rampzalige weg en trekt hem in zijn net.

  Hij bukt, werpt zich op de grond,~\sep\ en onder zijn geweld bezwijken de armen.

  Hij zegt bij zichzelf: ``God denkt er niet aan,~\sep\ Hij wendt Zijn aangezicht af, ziet nooit naar hem om.''

  Rijs op, Heer, God, hef Uw hand omhoog;~\sep\ vergeet de armen niet!

  Waarom blijft de boze God tergen,~\sep\ en spreekt hij bij zichzelf: ``Neen, Hij zal toch niet straffen?''

  Maar Gij ziet toe: lijden en smart staan U voor ogen,~\sep\ om het in Uw handen te nemen.

  Op U verlaat zich de ongelukkige,~\sep\ Gij zijt de Helper der wezen.

  Verbrijzel de arm van zondaar en boze;~\sep\ zijn boosheid zult Gij straffen, en geen spoor blijve er van over.

  De Heer is Koning in eeuwigheid,~\sep\ in Zijn land zijn de heidenen vernietigd.

  Het verlangen der ellendigen hebt Gij gehoord, Heer;~\sep\ Gij hebt hun hart versterkt, Uw oor naar hen geneigd,

  Om het recht te beschermen van wees en verdrukte,~\sep\ zodat geen mens ter wereld hen nog vrees aanjaagt.
\end{halfparskip}

\begin{halfparskip}
  \markedsubsectionrubricwithhint{Dsindagen ``voor'': Marmita 13*}{(origineel: hulala 6)}

  \psalm{\Ps{35}} Bestrijd, Heer, die mij bestrijden, bekamp die mij bekampen.~\sep\ Grijp schild en beukelaar, en rijs op om mij te helpen.

  \liturgicalhint{Alleluia, Alleluia, Alleluia.~--- Eerste vers.}

  Slinger de lans en bedwing mijn vervolgers,~\sep\ zeg tot mijn ziel: ``Ik ben Uw redding.''

  Laat smaad en schande hen treffen, die mijn leven belagen,~\sep\ en vol schaamte terugdeinzen die kwaad tegen mij beramen.

  Dat zij worden als kaf voor de wind,~\sep\ wanneer de engel van de Heer hen voortdrijft.

  Duister en glibberig worde hun pad,~\sep\ wanneer de engel van de Heer hen nazet.

  Want zonder reden hebben zij mij hun net gespannen,~\sep\ zonder reden een kuil voor mij gegraven.

  Moge onverhoeds de ondergang hen treffen, en het net, dat zij mij spanden, hen zelf vangen;~\sep\ mogen zij zelf neerstorten in de kuil, die zij groeven.

  Dan zal ik juichen in de Heer,~\sep\ mij verblijden over Zijn hulp.

  Met geheel mijn wezen zal ik zeggen:~\sep\ ``Heer, wie is U gelijk,

  Die de zwakke bevrijdt van de overmachtige,~\sep\ de zwakke en arme van zijn berover.''

  Brutale getuigen stonden tegen mij op;~\sep\ wat ik mij niet was bewust, legden zij mij ten laste.

  Zij vergolden mij goed met kwaad:~\sep\ lieten mij alleen en verlaten.

  En toch toen zij ziek lagen, trok ik het boetekleed aan,~\sep\ putte mij uit door vasten, en bad, diep voorovergebogen.

  Als gold het een vriend of mijn broeder, zo schreed ik droevig voort;~\sep\ als een, die rouwt over zijn moeder, zo ging ik van droefheid gebukt.

  Maar toen ik wankelde, waren zij blij en liepen te hoop,~\sep\ zij schoolden tegen mij samen en sloegen mij, die geen kwaad vermoedde;

  Zonder ophouden verscheurden zij mij, zij vielen mij aan, bespotten mij,~\sep\ terwijl ze tegen mij knarsetandden.

  Heer, hoe lang nog zult Gij het aanzien?~\sep\ Bevrijd mijn ziel van die brullende dieren en van die leeuwen mijn leven.

  Dan zal ik U danken in de volle vergadering,~\sep\ voor de talloze scharen U prijzen.

  Gun geen leedvermaak aan hen, die zonder reden mijn vijanden zijn,~\sep\ laat hen elkander met de ogen niet toewenken, die mij onverdiend haten.

  Want zij spreken geen woorden van vrede,~\sep\ en tegen de rustige bewoners van het land zinnen zij op bedrog.

  En hun mond sperren zij tegen mij open,~\sep\ zeggende: ``Ha, ha, met eigen ogen hebben wij het gezien!''

  Heer, Gij hebt het gezien, zwijg niet langer!~\sep\ Neen, Heer, houd U niet verre van mij!

  Ontwaak en blijf wakker om mij te verdedigen,~\sep\ ten gunste van mijn rechtszaak, mijn God en mijn Heer!

  Oordeel mij naar Uw gerechtigheid, Heer;~\sep\ mijn God, dat zij niet over mij juichen!

  Laat ze niet denken bij zichzelf: ``Ha, wat gaat het ons naar wens!''~\sep\ Laat ze niet zeggen: ``Wij hebben hem verslonden.''

  Dat ze te schande worden en zich schamen allen tezamen,~\sep\ die zich verheugen over mijn ongeluk;

  Met schaamte en schande worden bedekt,~\sep\ die zich tegen mij verheffen.

  Laat juichen en zich verblijden hen, die mijn zaak gunstig gezind zijn,~\sep\ en laat hen immerdoor spreken:

  ``Verheerlijkt zij de Heer,~\sep\ die het heil van Zijn dienaar bevordert.''

  Dan zal mijn tong Uw gerechtigheid verkondigen,~\sep\ en ten allen tijde Uw roem.

  \psalm{\Ps{36}} De zonde spreekt tot het hart van de goddeloze;~\sep\ geen vreze van God staat hem voor ogen.

  Want bedrieglijk stelt hij zich voor,~\sep\ dat zijn zonde niet wordt bemerkt, noch wordt verafschuwd.

  De woorden van zijn mond zijn boosheid en bedrog;~\sep\ niet langer gedraagt hij zich wijs en behoorlijk.

  Zelfs op zijn sponde zint hij op boosheid;~\sep\ op het slechte pad houdt hij zich op, wendt zich niet af van het kwaad.

  Uw erbarming, o Heer, reikt tot de hemel,~\sep\ en tot de wolken zelf Uw trouw.

  Uw gerechtigheid is als de bergen van God, Uw oordelen zijn diep als de zee;~\sep\ van mens en dier, O Heer, zijt Gij het behoud.

  Wat is Uw genade kostbaar, O God!~\sep\ in de schaduw van Uw vleugels bergen zich de kinderen der mensen.

  Zij worden verzadigd door de weelde van Uw huis,~\sep\ en met de stroom van Uw geneugten laaft Gij hen.

  Want bij U is de bron van het leven,~\sep\ en in Uw licht zien wij het licht.

  Blijf Uw genade schenken aan hen, die U dienen,~\sep\ en Uw gerechtigheid aan de oprechten van hart.

  Laat de voet van de trotse mij niet vertreden,~\sep\ en mij niet schokken de hand van de zondaar.

  Ziet reeds zijn de booswichten gevallen;~\sep\ neergeworpen zijn ze en kunnen niet meer opstaan.
\end{halfparskip}

\begin{halfparskip}
  \markedsubsectionrubricwithhint{Woensdagen ``voor'': Marmita 19*}{(origineel: hulala 9)}

  \psalm{\Ps{50}} God de Heer heeft gesproken en de wereld gedagvaard,~\sep\ van het rijzen der zon tot haar dalen.

  \liturgicalhint{Alleluia, Alleluia, Alleluia.~--- Eerste vers.}

  Uit Sion, de volschone, straalde God in heerlijkheid.~\sep\ Onze God verscheen en zwijgt niet.

  Verslindend vuur gaat voor Hem uit,~\sep\ en om Hem heen woedt de stormwind.

  Hij dagvaardt de hemel daarboven en ook de aarde,~\sep\ want oordelen gaat Hij Zijn volk.

  ``Verzamelt vóór Mij Mijn heiligen,~\sep\ die door offers het Verbond met Mij sloten.''

  En de hemelen kondigen Zijn gerechtigheid aan,~\sep\ want God zelf is de Rechter.

  ``Luister, Mijn volk, Ik ga spreken, en tegen u getuigen, o Israël,~\sep\ God, uw God ben Ik.

  Niet om uw offers berisp Ik u,~\sep\ want uw brandoffers stijgen steeds voor Mij op.

  Ik zal geen kalf uit uw stal aanvaarden,~\sep\ en uit uw kudden geen bokken.

  Want Mij behoren alle beesten der bossen,~\sep\ de duizenden dieren op Mijn bergen.

  Ik ken alle vogels in de lucht,~\sep\ en wat zich beweegt op het veld, is Mij bekend.

  Had Ik honger, Ik zou het u niet zeggen;~\sep\ want Mij behoort de aarde met al wat zij bevat.

  Of zou Ik het vlees van stieren eten,~\sep\ of drinken het bloed van bokken?

  Breng aan God een offer van lof,~\sep\ en voldoe uw geloften aan de Allerhoogste.

  En roep Mij dan aan op de dag van kwelling:~\sep\ Ik zal u redden, en gij zult Mij vereren.''

  Maar tot de zondaar spreekt God:~\sep\ ``Wat praat gij over Mijn wetten, en hebt de mond vol van Mijn Verbond?

  Gij, die de tucht veracht,~\sep\ en u aan Mijn woorden niet hebt gestoord?

  Zaagt gij een dief, gij liept met hem mee,~\sep\ en met echtbrekers gingt gij vertrouwelijk om.

  Uw mond stond open voor boosheid,~\sep\ en uw tong spon enkel bedrog.

  Gij zat neer, en getuigde tegen uw broeder;~\sep\ de zoon van uw moeder bracht gij in schande:

  Dat hebt gij gedaan, en zou Ik dan zwijgen; meent gij, dat Ik ben zoals gij?~\sep\ Hier is Mijn aanklacht en Ik ga ze voor uw ogen ontvouwen.

  Beseft dit wel, gij, die God zijt vergeten,~\sep\ anders roof Ik u weg en kan geen u nog redden.

  Wie een offer brengt van lof, verheerlijkt Mij;~\sep\ en wie in oprechtheid wandelt, hem zal Ik tonen Gods heil.''

  \psalm{\Ps{51}} Ontferm u over mij, O God, naar Uw barmhartigheid;~\sep\ delg toch mijn misdaad uit volgens Uw grote ontferming.

  Zuiver mij geheel van mijn schuld,~\sep\ en reinig mij van mijn zonde.

  Want mijn boosheid erken ik,~\sep\ en mijn zonde staat mij steeds voor ogen.

  Tegen U alleen heb ik gezondigd,~\sep\ en wat kwaad is in Uw ogen, heb ik gedaan:

  Zo zult Gij rechtvaardig in Uw uitspraak blijken,~\sep\ onberispelijk in Uw oordeel.

  Zie, in schuld ben ik geboren,~\sep\ en in zonde ontving mij mijn moeder.

  Zie, in de oprechtheid van het hart schept Gij behagen;~\sep\ en Gij leert mij de wijsheid in het diepst van mijn hart.

  Besprenkel mij met hysop, en ik zal gereinigd worden;~\sep\ was mij, en ik zal blanker zijn dan sneeuw.

  Laat mij een blijde tijding vernemen,~\sep\ laat juichen mijn gebeente, dat Gij hebt verbrijzeld.

  Wend van mijn zonden Uw aanschijn af,~\sep\ en delg al mijn schulden uit.

  Schep in mij een zuiver hart, O God,~\sep\ en vernieuw in mij een standvastige geest.

  Verwerp mij niet van Uw aanschijn,~\sep\ en neem Uw heilige geest niet van mij weg.

  Schenk mij terug de vreugde van Uw heil,~\sep\ en sterk mij door een edelmoedige geest.

  Dan zal ik de bozen Uw wegen doen kennen,~\sep\ en de zondaars zullen zich bekeren tot U.

  Verlos mij van bloedschuld, O God, God, mijn Redder,~\sep\ en dat mijn tong over Uw gerechtigheid juiche. Heer, open mijn lippen,~\sep\ en mijn mond zal Uw lof verkondigen.

  Neen, in een slachtoffer schept Gij geen vreugde,~\sep\ en zo ik een brandoffer bracht, Gij zoudt het niet aanvaarden.

  Mijn offer, O God, is een rouwmoedige geest;~\sep\ een berouwvol en vernederd hart zult Gij niet versmaden, O God.

  Volgens Uw goedheid, O Heer, handel genadig met Sion,~\sep\ en bouw weer de muren van Jeruzalem op.

  Dan zult Gij weer wettige offers, dank- en brandoffers aanvaarden;~\sep\ dan zal men op Uw altaar weer varren opdragen.

  \psalm{\Ps{52}} Wat roemt gij op boosheid,~\sep\ gij, overmachtige eerloze?~\sep\ Al maar door zint gij op bedrog; scherp als een scheermes is uw tong, O leugenmeester.

  Gij stelt het kwade boven het goede,~\sep\ spreekt liever leugens dan de waarheid.

  Slechts woorden, die verderf aanbrengen, zijn u lief,~\sep\ verraderlijke tong!

  Daarom zal God u doen omkomen,~\sep\ u uit de weg ruimen voor immer.

  U wegsleuren uit uw tent,~\sep\ en u met wortel en al uitrukken uit het land der levenden.

  De rechtvaardigen zullen het zien en huiveren,~\sep\ en zij zullen lachen met hem:

  ``Dat is nu de man, die niet erkende,~\sep\ dat God zijn bescherming is,

  Maar die op zijn grote rijkdom vertrouwde,~\sep\ door zijn misdaden overmachtig werd.''

  Ik echter ben als een bloeiende olijf in Gods huis;~\sep\ op Gods barmhartigheid vertrouw ik voor immer.

  Eeuwig zal ik U loven, omdat Gij dit hebt gedaan; en verheerlijken zal ik Uw Naam,~\sep\ want Hij is goed, voor het aanschijn van Uw heiligen.
\end{halfparskip}

\begin{halfparskip}
  \markedsubsectionrubricwithhint{Donderdagen ``voor'': Marmita 27*}{(origineel: hulala 15)}

  \psalm{\Ps{71}} Tot U neem ik mijn toevlucht, O Heer, moge ik niet beschaamd worden in eeuwigheid.~\sep\ Bevrijd en verlos mij volgens Uw gerechtigheid;~\sep

  \liturgicalhint{Alleluia, Alleluia, Alleluia.~--- Eerste vers.}

  neig tot mij Uw oor en red mij.

  Wees mij een rots tot toevluchtsoord, een versterkte burcht om mij te redden:~\sep\ want mijn Rots en mijn Burcht zijt Gij.

  Ontruk mij, mijn God, aan de hand van de boze,~\sep\ uit de vuist van de goddeloze en van de verdrukker;

  Want Gij, mijn God, zijt mijn verwachting,~\sep\ mijn hoop, O Heer, vanaf mijn jeugd.

  Gij waart mijn Steun reeds vóór mijn geboorte, van de moederschoot af reeds mijn Beschermer,~\sep\ op U heb ik immer gehoopt.

  Als een wonder geleek ik voor velen,~\sep\ want Gij waart mijn krachtige Helper.

  Mijn mond was vol van Uw lof,~\sep\ de hele dag vol van Uw glorie.

  Verwerp mij toch niet in mijn ouderdom,~\sep\ verlaat mij niet, nu mijn kracht mij begeeft.

  Want mijn vijanden bepraten mij,~\sep\ zij gaan mijn gangen na en overleggen tezamen.

  Ze zeggen: ``God heeft hem verlaten; zet hem na en grijpt hem aan,~\sep\ want er is niemand, die hem kan
  redden.''

  O God, blijf niet ver van mij af;~\sep\ mijn God, snel mij te hulp!

  Dat zij beschaamd worden en vergaan, die mijn leven belagen;~\sep\ dat smaad en schande bedekken, die mijn ongeluk zoeken.

  Ik echter zal immer vertrouwen,~\sep\ en bij al Uw lof nog dagelijks nieuwe voegen.

  Mijn mond zal Uw gerechtigheid verkondigen, en Uw bijstand heel de dag,~\sep\ want de maat ervan ken ik niet.

  Gods macht zal ik verhalen,~\sep\ Uw gerechtigheid roemen, O Heer, de Uwe alleen.

  O God, vanaf mijn jeugd hebt Gij mij onderwezen,~\sep\ en tot heden verhaal ik Uw wonderen.

  Nu ik oud ben en grijs,~\sep\ verlaat mij toch niet, O God.

  Nu ik Uw kracht ga verkondigen aan dit geslacht,~\sep\ Uw macht aan allen, die nog komen,

  En Uw gerechtigheid, O God, die tot de hemel reikt,~\sep\ waardoor Gij zoveel groots hebt volbracht: God, wie is aan U gelijk?

  Vele en zware beproevingen hebt Gij gebracht over mij:~\sep\ weer zult Gij mij doen leven, en weer mij optrekken uit de diepten der aarde.

  Verhoog, mijn waardigheid,~\sep\ en troost mij opnieuw.

  Ook ik, O God, zal dan Uw trouw bij psalterspel roemen,~\sep\ U, Heilige van Israël, op de citer bezingen.

  Mijn lippen zullen juichen, wanneer ik voor U zing,~\sep\ zo ook mijn ziel, die Gij hebt verlost.

  Ook mijn tong zal heel de dag Uw gerechtigheid verkondigen,~\sep\ omdat die mijn ongeluk zoeken, met schande en schaamrood bedekt zijn.

  \psalm{\Ps{72}} Geef aan de Koning Uw rechtsmacht, O God,~\sep\ en Uw rechtvaardigheidszin aan de Zoon van de Koning.

  Hij moge Uw volk met rechtvaardigheid besturen,~\sep\ en Uw geringen naar billijkheid.

  De bergen zullen vrede brengen aan het volk,~\sep\ en de heuvelen gerechtigheid.

  De geringen uit het volk zal Hij beschermen, redding brengen aan de zonen der armen~\sep\ en de verdrukker met voeten treden.

  Hij zal leven zolang als de zon bestaat,~\sep\ en als de maan door alle geslachten.

  Hij zal neerdalen als regen op het gras,~\sep\ als buien, die de aarde besproeien.

  In Zijn dagen zal de gerechtigheid bloeien,~\sep\ en overvloedige vrede, totdat de maan niet meer schijnt.

  Hij zal heersen van zee tot zee,~\sep\ van de Stroom tot aan de grenzen der aarde.

  Zijn vijanden zullen voor Hem neervallen,~\sep\ en Zijn weerstrevers de aarde kussen.

  Tharsis' koningen en die der eilanden zullen geschenken offeren,~\sep\ de koningen van Arabië en Saba gaven aanbrengen:

  Alle koningen zullen Hem aanbidden,~\sep\ alle volken Hem dienen.

  Ja, bevrijden zal Hij de arme, die tot Hem roept,~\sep\ en de verdrukte, die geen helper heeft.

  Hij zal de behoeftige en arme genadig zijn,~\sep\ en het leven der armen redden,

  Hen van onrecht en verdrukking bevrijden;~\sep\ en hun bloed zal kostbaar zijn in Zijn oog.

  Daarom zal Hij leven en schenkt men Hem goud van Arabië,~\sep\ voortdurend zal men voor Hem bidden, Hem zegenen immerdoor.

  Overvloed van koren zal er zijn in het land; op de toppen der bergen zullen Zijn aren als de Libanon ruisen,~\sep\ en de bewoners der steden zullen bloeien als gras op het veld.

  Zijn Naam zal eeuwig gezegend zijn;~\sep\ zolang de zon zal schijnen, zal ook Zijn Naam bestaan.

  Alle stammen der aarde zullen gezegend worden in Hem,~\sep\ en alle volken Hem gelukkig prijzen.

  Gezegend zij de Heer, de God van Israël,~\sep\ die wonderen doet, Hij alleen.

  En Zijn roemrijke Naam zij gezegend voor eeuwig;~\sep\ en heel de aarde zij van Zijn glorie vervuld. Het zij zo, het zij zo.
\end{halfparskip}

\begin{halfparskip}
  \markedsubsectionrubricwithhint{Vrijdagen ``voor'': Marmita 43*}{(origineel: hulala 18)}

  \psalm{\Ps{106}} Looft de Heer, want Hij is goed,~\sep\ want eeuwig duurt Zijn barmhartigheid.

  \liturgicalhint{Alleluia, Alleluia, Alleluia.~--- Eerste vers.}

  Wie zal de machtige werken van de Heer verhalen,~\sep\ verkondigen al Zijn roem?

  Gelukkig zij, die de Wet onderhouden,~\sep\ het goede doen te allen tijde!

  Gedenk mij, Heer, naar Uw welwillendheid voor Uw volk,~\sep\ kom tot mij met Uw hulp,
  Opdat ik geniete het geluk van Uw uitverkorenen, mij verblijde in de vreugde van Uw volk,~\sep\ mij beroeme met Uw erfdeel.

  Wij hebben gezondigd als onze vaderen,~\sep\ wij hebben misdreven, wij hebben misdaan.

  Onze vaderen in Egypte,~\sep\ hebben Uw wonderen niet overwogen;

  Aan Uw talrijke gunsten dachten zij niet,~\sep\ maar weerstonden de Allerhoogste bij de Rode Zee.

  Toch heeft Hij hen om Zijn Naam gered,~\sep\ opdat Hij Zijn macht zou tonen.

  De Rode Zee bedreigde Hij: en ze lag droog,~\sep\ Hij leidde hen door de golven als door een woestijn.

  Hij redde hen uit de hand van de hater,~\sep\ uit de macht van de vijand verloste Hij hen.

  De wateren spoelden over hun vijanden heen:~\sep\ niet één van hen ontkwam.

  Toen sloegen zij geloof aan Zijn woorden,~\sep\ en zongen Zijn lof.

  Ze waren spoedig Zijn werken vergeten:~\sep\ zij verlieten zich niet op Zijn raadsbesluit.

  Zij gaven zich in de woestijn aan de begeerlijkheid over,~\sep\ en stelden in de eenzaamheid God op de proef.

  Hij voldeed aan hun verzoek,~\sep\ maar zond hun de tering over.

  Ze werden in het kamp afgunstig op Moses,~\sep\ op Aäron, de gewijde van de Heer.

  De aarde scheurde open en zij zwolg Dathan in,~\sep\ en bedolf de bende van Abiron.

  Een vuur ontbrandde in hun bende,~\sep\ een vlam heeft de bozen verteerd.

  Ze maakten een kalf bij de Horeb,~\sep\ en aanbaden het gouden beeld.

  Zij kozen in plaats van hun Glorie,~\sep\ het beeld van een stier, zich voedend met gras.

  Zij vergaten de God, die hen redde,~\sep\ die wonderen wrochtte in Egypte,

  Wonderwerken in het land van Cham,~\sep\ ontzagwekkende daden bij de Rode Zee.

  Hij dacht er aan hen te verdelgen,~\sep\ was niet Moses, Zijn uitverkorene,

  Bij Hem tussenbeide getreden,~\sep\ om af te wenden Zijn toorn, opdat Hij hen niet zou verdelgen.

  Zij versmaadden het begeerlijke land;~\sep\ zij sloegen geen geloof aan Zijn woord.

  Zij morden in hun tenten,~\sep\ luisterden niet naar de Heer.

  Toen stak Hij Zijn hand op en zwoer hun de eed:~\sep\ Hij zou hen neerslaan in de woestijn,

  Onder de volken hun nakroost verstrooien,~\sep\ hen over de landen verspreiden.

  Ze hingen Beëlfegor aan,~\sep\ en aten offers van levenloze goden.

  Zij tergden Hem door hun wangedrag,~\sep\ zodat er een slachting onder hen losbrak.

  Maar Fineës stond op om te richten,~\sep\ en de slachting nam een einde.

  Het werd hem tot verdienste gerekend,~\sep\ door alle geslachten voor immer.

  Bij de wateren van Meriba tergden zij Hem,~\sep\ en slecht verging het Mozes omwille van hen.

  Want zij hadden zijn geest verbitterd,~\sep\ en ondoordacht kwamen de woorden over zijn lippen.

  Zij roeiden de volken niet uit,~\sep\ gelijk de Heer het hun had bevolen,

  Maar zij vermengden zich met de heidenen,~\sep\ en leerden hun werken aan:

  Zij vereerden hun gesneden beelden,~\sep\ die hun tot valstrik werden;

  Zij offerden hun zonen,~\sep\ en dochters aan de duivels.

  Zij vergoten onschuldig bloed, het bloed van hun zonen en dochters,~\sep\ die zij offerden aan de beelden van Chanaän:

  Het land werd door bloed bevlekt; zij bezoedelden zich door hun werken,~\sep\ en door hun daden bedreven zij overspel.

  Toen ontvlamde de toorn van de Heer tegen Zijn volk,~\sep\ en Zijn erfdeel werd Hem een gruwel.

  Hij leverde hen over aan de macht van de heidenen,~\sep\ en die hen haatten, overheersten hen.

  Hun vijanden kwelden hen,~\sep\ ze werden neergedrukt onder hun hand,

  Zo menigmaal bevrijdde Hij hen, maar zij bleven door hun plannen Hem tergen,~\sep\ en werden om hun zonden te gronde gericht.

  Toch zag Hij op hun ellende neer,~\sep\ zodra Hij hun smeken vernam.

  Hun ten zegen was Hij Zijn verbond indachtig,~\sep\ en had met hen deernis naar Zijn grote ontferming.

  En Hij liet hen genade vinden,~\sep\ bij allen, die hen gevangen hadden weggevoerd.

  Schenk ons toch redding, O Heer, onze God,~\sep\ en breng ons uit de naties tezamen,

  Opdat wij prijzen Uw heilige Naam,~\sep\ en over Uw lof ons beroemen,

  Gezegend de Heer, de God van Israël, van eeuwigheid tot eeuwigheid!~\sep\ En heel het volk moge nu zeggen: amen!
\end{halfparskip}

\begin{halfparskip}
  \markedsubsectionrubricwithhint{Zaterdagen ``voor'': Marmita 52*}{(origineel: hulala 21)}

  \psalm{\Ps{131}} Gedenk, Heer ten gunste van David,~\sep\ al zijn moeizame zorgen:

  Hoe hij de Heer een eed heeft gezworen,~\sep\ een gelofte gedaan aan de Sterke van Jacob:

  \liturgicalhint{Alleluia, Alleluia, Alleluia.~--- Eerste vers.}

  ``Ik zal mijn woontent niet binnentreden,~\sep\ noch mijn legerstede bestijgen,

  Mijn ogen de slaap niet gunnen,~\sep\ de sluimer niet aan mijn wimpers,

  Eer ik een rustplaats voor de Heer heb gevonden,~\sep\ een woonstede voor de Sterke van Jacob.''

  Ziet, wij hoorden van haar in Efrata,~\sep\ wij vonden haar in de velden van Jaär.

  Laten wij Zijn woonplaats binnentreden,~\sep\ ons vóór Zijn voetbank neerwerpen.

  Trek op, o Heer, naar Uw rustplaats,~\sep\ Gij en Uw glorievolle ark.

  Laat Uw priesters zich bekleden met heiligheid,~\sep\ en Uw heiligen juichen vol vreugde.

  Omwille van David, Uw dienstknecht,~\sep\ verstoot het aanschijn van Uw gezalfde niet.

  De Heer heeft David een eed gezworen,~\sep\ een vaste belofte, waarvan Hij niet wijkt:

  ``Een telg van Uw geslacht~\sep\ zal Ik plaatsen op uw troon.

  Zo uw zonen Mijn verbond onderhouden,~\sep\ en de wetten, die Ik hun leren zal,

  Zullen ook hun zonen voor eeuwig,~\sep\ zetelen op uw troon.''

  Want de Heer heeft Sion verkoren,~\sep\ het als Zijn woning begeerd:

  ``Dit is Mijn rustplaats voor eeuwig,~\sep\ hier zal Ik wonen, omdat Ik haar heb begeerd.

  Mild zal Ik zijn voedsel zegenen,~\sep\ zijn armen verzadigen met brood.

  Zijn priesters zal Ik met heil bekleden,~\sep\ en zijn heiligen zullen juichen vol vreugde.

  Hier zal Ik David een hoorn verwekken,~\sep\ een lamp bereiden voor Mijn Gezalfde.

  Zijn vijanden zal Ik overdekken met schaamte,~\sep\ maar op hem zal stralen Mijn kroon.''

  \psalm{\Ps{132}} Ziet, hoe goed en heerlijk het is,~\sep\ als broeders samenwonen:

  Het is als kostelijke olie op het hoofd die afdruipt op de baard, de baard van Aäron,~\sep\ die afdruipt op de zoom van zijn kleed.

  Het is als de dauw van de Hermon,~\sep\ die neerslaat op de Sionsberg.

  Want daar schenkt de Heer Zijn zegen,~\sep\ leven tot in eeuwigheid.

  \psalm{\Ps{133}} Welaan dan, zegent de Heer,~\sep\ gij allen, dienaars van de Heer;

  Die dienst doet in het huis van de Heer,~\sep\ in de nachtelijke uren.

  Heft Uw handen naar het heiligdom op,~\sep\ en zegent de Heer.

  Moge uit Sion de Heer u zegenen,~\sep\ die hemel en aarde gemaakt heeft.

  \psalm{\Ps{134}} Looft de Naam van de Heer,~\sep\ looft Hem, gij dienaars van de Heer!

  Gij, die dienst doet in het huis van de Heer!~\sep\ in de voorhoven van het huis van onze God.

  Looft de Heer, want de Heer is goed;~\sep\ bezingt Zijn Naam, want die is liefelijk.

  Want de Heer heeft Jacob verkoren,~\sep\ Israël tot Zijn eigendom.

  Ja, dit weet ik: groot is de Heer,~\sep\ onze Heerser boven alle goden.

  Al wat Hij wil, volbrengt de Heer in hemel en op aarde,~\sep\ in de zee en in alle diepten der wateren.

  Van de kimmen der aarde voert Hij de wolken aan,~\sep\ stort regen uit door de bliksems, haalt de wind uit Zijn schuren.

  Hij sloeg de eerstgeborenen van Egypte,~\sep\ mensen zowel als dieren.

  In u, Egypte, verrichtte Hij tekenen en wonderwerken,~\sep\ voor Farao en al zijn dienaars.

  Hij sloeg vele volkeren,~\sep\ en machtige vorsten verdelgde Hij:

  Sehon, de vorst der Amorieten, en Og, de vorst van Basan,~\sep\ en alle koningen van Chanaan.

  Hij gaf hun land in eigendom,~\sep\ in eigendom aan Israël, Zijn volk.

  Uw Naam, Heer, blijft eeuwig bestaan,~\sep\ Uw aandenken, Heer, van geslacht tot geslacht.

  Want de Heer beschermt Zijn volk,~\sep\ en is Zijn dienaars genadig.

  De goden der heidenen zijn zilver en goud,~\sep\ maaksels van mensenhanden.

  Ze hebben een mond, en spreken niet;~\sep\ ze hebben ogen, en kunnen niet zien;

  Ze hebben oren, en horen niet;~\sep\ en geen ademtocht is in hun mond.

  Aan hen worden gelijk die ze maken,~\sep\ en allen, die vertrouwen op hen.

  Huis van Israël, zegent de Heer,~\sep\ huis van Aaron, zegent de Heer!

  Huis van Levi, zegent de Heer;~\sep\ gij, die de Heer dient, zegent de Heer!

  Gezegend zij de Heer uit Sion,~\sep\ Hij, die in Jeruzalem woont!
\end{halfparskip}

\begin{halfparskip}
  \markedsubsectionrubricwithhint{Maandagen ``na'': Marmita 8*}{(origineel: hulala 3)}

  \psalm{\Ps{22}} Mijn God, mijn God, waarom hebt Gij mij verlaten?~\sep\ Ver houdt Gij U af van mijn bede, van mijn noodgeschrei.

  \liturgicalhint{Alleluia, Alleluia, Alleluia.~--- Eerste vers.}

  Bij dag roep ik U aan, mijn God, en Gij verhoort mij niet;~\sep\ bij nacht, en Gij slaat geen acht op mij.

  Toch woont Gij in het heiligdom,~\sep\ Gij, de roem van Israël.

  Onze vaderen hoopten op U,~\sep\ zij hoopten op U, en Gij hebt hen bevrijd;

  Zij riepen U aan, en werden gered;~\sep\ zij hoopten op U, en zijn niet beschaamd.

  Maar ik ben een worm en geen mens,~\sep\ de smaad der mensen en de verachting van het volk.

  Allen, die mij zien, spotten met mij,~\sep\ vertrekken de lippen en schudden het hoofd:

  ``Hij vertrouwt op de Heer; laat Die hem bevrijden,~\sep\ laat Die hem verlossen, zo Hij hem bemint.''

  Ja, Gij hebt mij geleid van de moederschoot af,~\sep\ mij veilig gelegd aan de borst van mijn moeder.

  U werd ik toevertrouwd vanaf mijn geboorte,~\sep\ vanaf de schoot van mijn moeder zijt Gij mijn God.

  Blijf toch niet ver van mij, want ik word gekweld;~\sep\ wees mij nabij, want er is geen helper.

  Jonge stieren stuwen in menigte om mij heen,~\sep\ stieren van Basan omsingelen mij.

  Zij sperren hun muil tegen mij open,~\sep\ als een roofzuchtige en brullende leeuw.

  Als water ben ik uitgestort,~\sep\ en al mijn beenderen zijn ontwricht.

  Mijn hart is geworden als was,~\sep\ het smelt in mijn binnenste weg.

  Mijn keel is droog als een potscherf, en mijn tong kleeft vast aan mijn gehemelte;~\sep\ Gij hebt mij gebracht tot het stof van de dood.

  Want vele honden staan om mij heen,~\sep\ een bende boosdoeners houdt mij omsingeld.

  Zij hebben mijn handen en voeten doorboord,~\sep\ ik kan al mijn beenderen tellen.

  Zij slaan mij gade, en bij die aanblik verheugen zij zich; zij verdelen mijn klederen onder elkander,~\sep\ en werpen het lot over mijn gewaad.

  Gij nu, Heer, blijf niet van verre staan;~\sep\ mijn Bijstand, snel mij te hulp!

  Ontruk mijn ziel aan het zwaard,~\sep\ aan de greep van de hond mijn leven.

  Red mij uit de muil van de leeuw,~\sep\ mij, ongelukkige, van de hoornen der buffels.

  Ik zal mijn broeders Uw Naam verkondigen,~\sep\ in volle vergadering U prijzen:

  ``Looft de Heer, gij, die Hem vreest, heel Jacobs geslacht, verheerlijk Hem:~\sep\ vreest Hem, alle kinderen van Israël!

  Want Hij heeft niet versmaad, noch geminacht de ellende van de verdrukte, en Hij hield zijn aanschij voor hem niet verborgen;~\sep\ Hij heeft hem aanhoord, toen hij riep tot Hem.''

  Van U komt mijn lof in de volle vergadering;~\sep\ ten aanschouwen van Zijn vereerders zal ik mijn geloften volbrengen.

  De armen zullen eten en zich verzadigen; die de Heer zoeken, zullen Hem loven:~\sep\ ``dat uw harten leven in eeuwigheid!''

  Dit indachtig, zullen tot de Heer zich bekeren,~\sep\ alle grenzen der aarde;

  En voor Zijn aanschijn zullen neervallen,~\sep\ alle stammen der heidenen,

  Want aan de Heer behoort het koningschap,~\sep\ Hij is het, die over de volkeren heerst;

  Hem alleen zullen allen aanbidden die onder de aarde rusten,~\sep\ voor Hem zullen allen zich buigen, die neerdalen in het stof.

  En mijn ziel zal leven voor Hem,~\sep\ mijn nageslacht Hem dienen;

  Het zal van de Heer verhalen aan het geslacht, dat komen zal,~\sep\ en Zijn gerechtigheid zal men vermelden aan het volk, dat wordt geboren: ``Dit heeft de Heer gedaan.''

  \psalm{\Ps{23}} De Heer is mijn Herder: het ontbreekt mij aan niets;~\sep\ in groenende beemden laat Hij mij sluimeren.

  Hij voert mij naar wateren, waar ik kan rusten;~\sep\ Hij verkwikt mijn ziel.

  Hij leidt mij langs rechte wegen,~\sep\ omwille van Zijn Naam.

  Al schrijd ik dan voort in een donker dal,~\sep\ geen kwaad zal ik vrezen, omdat Gij met mij zijt.

  Uw roede en Uw herdersstaf,~\sep\ zijn mij tot troost.

  Gij richt voor mij een maaltijd aan,~\sep\ ten aanschouwen van mijn weerstrevers.

  Met olie zalft Gij mijn hoofd;~\sep\ mijn beker is overvol.

  Goedertierenheid en genade zullen mij volgen,~\sep\ al de dagen van mijn leven;

  En wonen zal ik in het huis van de Heer,~\sep\ tot in de verre toekomst.

  \psalm{\Ps{24}}  De Heer behoort de aarde met al wat zij bevat,~\sep\ de wereld en die er op wonen.

  Want Hij heeft haar op de zeeën gegrondvest,~\sep\ en legde haar vast op de stromen.

  Wie mag de berg van de Heer bestijgen,~\sep\ of wie verwijlen in Zijn heilige plaats?

  Die rein is van handen en zuiver van hart, zijn geest niet richt op ijdele dingen,~\sep\ en tegen zijn naaste geen meineed zweert.

  Die zal zegen ontvangen van de Heer,~\sep\ en loon van God, zijn Redder.

  Dit is het geslacht van die naar Hem zoeken,~\sep\ van die zoeken het aanschijn van Jacobs God.

  Poorten, uw bogen omhoog, omhoog, gij, aloude poorten,~\sep\ opdat de Koning der glorie Zijn intrede doe!

  ``Wie is die Koning der glorie?''~\sep\ ``De Heer, de Sterke en de Machtige, de Heer, de Held in de strijd.''

  Poorten, uw bogen omhoog, omhoog, gij, aloude poorten,~\sep\ opdat de Koning der glorie Zijn intrede doe!

  ``Wie is die Koning der glorie?''~\sep\ ``De Heer der heerscharen, Hij is de Koning der glorie.''
\end{halfparskip}

\begin{halfparskip}
  \markedsubsectionrubricwithhint{Dinsdagen ``na'': Marmita 16*}{(origineel: hulala 6)}

  \psalm{\Ps{41}} Gelukkig die denkt aan de behoeftige en arme:~\sep\ de Heer zal hem redden op de dag van het onheil.

  \liturgicalhint{Alleluia, Alleluia, Alleluia.~--- Eerste vers.}

  De Heer zal hem behoeden en in het leven bewaren, hem gelukkig maken op aarde,~\sep\ en niet prijsgeven aan de moedwil van zijn vijanden.

  De Heer zal hem bijstaan op zijn lijdenssponde:~\sep\ alle zwakheid van hem wegnemen in zijn ziekte.

  Ik bid dan: Heer, wees mij genadig;~\sep\ genees mij, omdat ik tegen U heb gezondigd.

  Kwade dingen zeggen mijn vijanden over mij:~\sep\ ``Wanneer zal hij sterven en zal zijn naam verdwijnen?''

  En wie op bezoek komt, spreekt huichelachtig;~\sep\ stof tot laster verzamelt zijn hart, en hij gaat het buiten vertellen.

  Allen, die mij haten, fluisteren samen over mij,~\sep\ en denken aan de rampspoed, die over mij kwam.

  ``Een boosaardige pest heeft hem besmet,''~\sep\ en: ``Die daar neerligt, zal niet meer opstaan.''

  Zelfs mijn vriend, op wie ik vertrouwde,~\sep\ die mijn brood at, heeft tegen mij zijn hiel geheven.

  Maar Gij, O Heer, wees mij genadig en richt mij op,~\sep\ opdat ik het hun vergelde.

  Hieraan zal ik erkennen, dat Gij mij gunstig gezind zijt,~\sep\ dat mijn vijand niet over mij juicht.

  Na mijn herstel nu zult Gij mij steunen,~\sep\ mij voor eeuwig voor Uw aangezicht plaatsen.

  Gezegend zij de Heer, de God van Israël,~\sep\ van eeuwigheid tot eeuwigheid. Het zij zo, het zij zo.

  \psalm{\Ps{42}} Gelijk de hinde smacht naar waterstromen,~\sep\ zo smacht mijn ziel naar U, O God.

  Naar God dorst mijn ziel, naar de levende God;~\sep\ wanneer mag ik komen en Gods aanschijn aanschouwen?

  Mijn tranen zijn mij tot spijs geworden dag en nacht,~\sep\ terwijl men steeds tot mij zegt: ``Waar is uw God?''

  Ik denk er met diepe weemoed aan terug, hoe ik eenmaal voortschreed met de schare,~\sep\ ja, hen voorging naar Gods huis,

  Onder gejubel en lofzang,~\sep\ in feestelijke stoet.

  Waarom zijt gij bedrukt, mijn ziel,~\sep\ en waarom vol onrust in mij?

  Stel Uw hoop op God, want opnieuw zal ik Hem prijzen,~\sep\ het Heil van mijn aanschijn en mijn God.

  Mijn ziel is bedrukt in mij:~\sep\ daarom denk ik aan U vanuit het land van de Jordaan en de Hermon, en vanaf de berg Mizar.

  De ene kolk roept de andere op bij het bruisen van Uw watervallen:~\sep\ al Uw golven en baren gingen over mij heen.

  Bij dag schenke de Heer vrijgevig Zijn genade,~\sep\ en bij nacht wil ik voor Hem zingen, en prijzen de God van mijn leven.

  Ik zeg tot God: ``Mijn Rots, waarom vergeet Gij mij;~\sep\ waarom ga ik droevig voort, door de vijand verdrukt?''

  Mijn beenderen worden verbrijzeld terwijl mijn weerstrevers mij honen,~\sep\ en mij dagelijks zeggen: ``Waar blijft toch uw God?''

  Waarom zijt gij bedrukt, mijn ziel,~\sep\ en waarom vol onrust in mij?

  Stel uw hoop op God, want opnieuw zal ik Hem prijzen,~\sep\ het Heil van mijn aanschijn en mijn God!

  \psalm{\Ps{43}} Schaf mij recht, God, en verdedig mijn zaak tegen een onheilig volk;~\sep

  verlos mij van de bedrieger en de boze,

  Want gij, God, zijt mijn Kracht; waarom hebt Ge me verstoten,~\sep\ waarom ga ik droevig voort terwijl de vijand me verdrukt?

  Zend Uw licht en Uw trouw: laat deze mij leiden,~\sep\ mij voeren naar Uw heilige berg en in Uw woontenten.

  Dan zal ik opgaan naar het altaar van God,~\sep\ naar God, mijn Vreugde en mijn Jubel;

  En ik zal U loven op de citer,~\sep\ God, mijn God.

  Waarom zijt gij bedrukt, mijn ziel,~\sep\ en waarom opstandig in mij?

  Vertrouw op God, want opnieuw zal ik Hem prijzen,~\sep\ het Heil van mijn aanschijn en mijn God.
\end{halfparskip}

\begin{halfparskip}
  \markedsubsectionrubricwithhint{Woensdagen ``na'': Marmita 22*}{(origineel: hulala 9)}

  \psalm{\Ps{59}} Ontruk mij, mijn God, aan mijn vijanden,~\sep\ en behoed mij voor hen, die tegen mij opstaan.

  \liturgicalhint{Alleluia, Alleluia, Alleluia.~--- Eerste vers.}

  Bevrijd mij van hen, die onrecht plegen,~\sep\ en red mij van mannen, die bloed vergieten.

  Want zie, ze staan mij naar het leven,~\sep\ de machtigen spannen tegen mij samen.

  Geen misdrijf, Heer, geen zonde is er in mij;~\sep\ buiten mijn schuld rukken zij op en vallen zij aan.

  Ontwaak, snel mij te hulp, en zie toe,~\sep\ want, O Heer der legerscharen, Gij zijt Israëls God.

  Ontwaak en tuchtig al de heidenvolkeren;~\sep\ heb geen medelijden met al die trouwelozen.

  `s~Avonds keren zij terug, ze blaffen als honden,~\sep\ en zwerven rond in de stad.

  Zie, grootspraak is in hun mond en smaad op hun lippen:~\sep\ ``Wie immers hoort het?''

  Maar Gij, Heer, lacht hen uit,~\sep\ Gij drijft de spot met al de heidenen.

  Mijn kracht, U schenk ik mijn aandacht, want Gij, O God, zijt mijn Bescherming,~\sep\ mijn God en mijn Barmhartigheid.

  Dat God mij helpe,~\sep\ mij doe juichen over mijn vijanden.

  Dood ze, O God, opdat ze mijn volk geen aanstoot geven;~\sep\ breng ze in verwarring door Uw kracht en vel ze neer, Gij, ons Schild, O Heer.

  Een zonde van hun mond zijn de woorden van hun lippen;~\sep\ dat ze in hun trots worden verstrikt, in de lastertaal en leugens, die zij spreken.

  Verdelg hen in Uw toorn, verdelg hen zo, dat zij niet meer bestaan,~\sep\ opdat men wete, dat God heerst in Jacob en tot aan de grenzen der aarde.

  `s~Avonds keren zij terug, ze blaffen als honden,~\sep\ en zwerven rond in de stad;

  Ze dolen rond om voedsel te zoeken,~\sep\ en huilen, als ze niet zijn verzadigd.

  Maar ik, ik zal Uw macht bezingen,~\sep\ en `s~morgens jubelen over Uw barmhartigheid,

  Want Gij zijt mij tot bescherming geworden,~\sep\ en tot toevlucht op de dag van mijn kwelling.

  Mijn kracht, voor U zal ik het psalter bespelen, want Gij, O God, zijt mijn Bescherming,~\sep\ mijn God en mijn Barmhartigheid.

  \psalm{\Ps{60}} O God, Gij hebt ons verstoten, onze gelederen verbroken:~\sep\ Gij zijt vertoornd: herstel ons weer!~\sep\ Het land hebt Gij geschokt, Gij hebt het gescheurd:~\sep

  herstel zijn kloven, want het wankelt.

  Uw volk hebt Gij harde dingen opgelegd,~\sep\ ons bedwelmende wijn doen drinken.

  Voor die U vrezen hebt Gij een banier geheven,~\sep\ om er te vluchten voor de boog;

  Opdat Uw geliefden zouden worden bevrijd,~\sep\ help ons door Uw rechterhand, en verhoor ons.

  God heeft gesproken in Zijn heiligdom:~\sep\ ``Ik zal juichen en Sichem verdelen, en het dal van Succoth meten.

  Van Mij is het land Galaäd, van Mij het land Manasse,~\sep\ Efraïm is de helm van Mijn hoofd en Juda Mijn scepter;

  Moab is Mijn wasbekken, op Edom werp Ik Mijn schoeisel,~\sep\ over Filistea zal Ik zegevieren.''

  Wie zal mij binnenvoeren in de versterkte stad,~\sep\ wie mij naar Edom geleiden?

  Zijt Gij het niet, O God, die ons hebt verstoten,~\sep\ die niet meer uittrekt, O God, met onze legerscharen?

  Schenk ons Uw hulp tegen de vijand,~\sep\ want ijdel is de hulp van mensen.

  Met God zullen wij dapper strijden,~\sep\ en Hij zelf zal onze vijanden vertreden.

  \psalm{\Ps{61}} Luister, O God, naar mijn smeken,~\sep\ geef acht op mijn bede.

  Van het uiteinde der aarde roep ik tot U,~\sep\ wanneer mijn hart het begeeft.

  Gij zult mij verheffen op de rots, mij rust verschaffen,~\sep\ want Gij zijt mij tot bescherming, een sterke Toren tegen de vijand.

  Mocht ik toch immer wonen in Uw tent,~\sep\ en schuilen onder de dekking van Uw vleugels!

  Gij toch, O God, hebt mijn geloften aanhoord,~\sep\ mij het erfdeel gegeven van die Uw Naam vrezen.

  Vermenigvuldig de dagen van de koning,~\sep\ mogen zijn jaren gelijk zijn aan vele geslachten.

  Eeuwig moge hij tronen voor Gods aanschijn;~\sep\ zend tot zijn behoud Uw genade en trouw.

  Zo zal ik Uw Naam voor immer bezingen,~\sep\ en te allen tijde mijn geloften inlossen.
\end{halfparskip}

\begin{halfparskip}
  \markedsubsectionrubricwithhint{Donderdagen ``na'': Marmita 40*}{(origineel: hulala 15)}

  \psalm{\Ps{102}} Heer, luister naar mijn bede,~\sep\ en mijn klagen dringe tot U door.

  \liturgicalhint{Alleluia, Alleluia, Alleluia.~--- Eerste vers.}

  Verberg voor mij Uw aanschijn niet,~\sep\ op de dag van mijn kwelling.

  Neig Uw oor tot mij;~\sep\ wanneer ik U aanroep, verhoor mij dan spoedig!

  Want mijn dagen vervliegen als rook,~\sep\ en mijn gebeente gloeit als vuur.

  Verschroeid als gras, verdort mijn hart,~\sep\ mijn brood vergeet ik te eten.

  Door het geweld van mijn zuchten,~\sep\ hecht mijn gebeente zich vast aan mijn huid.

  Ik ben als een pelikaan in de woestijn,~\sep\ als een nachtuil te midden der puinen.

  Ik ben slapeloos en lig maar te zuchten,~\sep\ als een eenzame vogel op het dak.

  Mijn vijanden honen mij almaar door;~\sep\ die tegen mij razen, gebruiken mijn naam als verwensing.

  As immers eet ik als brood,~\sep\ en met tranen vermeng ik mijn drank,

  Om wille van Uw gramschap en toorn,~\sep\ want Gij naamt mij op en wierpt mij neer.

  Mijn dagen zijn als de schaduw die lengt,~\sep\ en ik verdor als het gras.

  Maar Gij, O Heer, blijft in eeuwigheid,~\sep\ en Uw Naam door alle geslachten.

  Gij dan, sta op, en wees Sion genadig,~\sep\ want het is tijd haar genadig te zijn; ja, het uur is gekomen.

  Want Uw dienaren hebben haar stenen lief,~\sep\ en zij bejammeren haar puinen.

  De volken, O Heer, zullen Uw Naam eerbiedigen,~\sep\ en alle koningen der aarde Uw heerlijkheid,

  Als de Heer Sion zal hebben herbouwd,~\sep\ en verschenen zal zijn in Zijn glorie,

  Zich tot de bede der armen geneigd heeft,~\sep\ en hun gebed niet heeft versmaad.

  Men schrijve dit op voor het volgend geslacht;~\sep\ het volk, dat geschapen wordt, prijze de Heer.

  Want de Heer zag neer uit Zijn verheven heiligdom,~\sep\ Hij blikte uit de hemel op de aarde neer,

  Om het zuchten der gevangenen te horen,~\sep\ te verlossen die ten dode zijn opgeschreven.

  Opdat de Naam van de Heer worde verkondigd in Sion,~\sep\ en in Jeruzalem Zijn lof,

  Als de volken zich zullen verzamelen,~\sep\ met de koninkrijken om de Heer te dienen.

  Nog op de weg heeft Hij mijn krachten verteerd,~\sep\ mijn dagen verkort.

  Toch bid ik: Mijn God, neem mij niet weg in het midden van mijn dagen;~\sep\ Uw jaren duren door alle geslachten.

  In den beginne hebt Gij de aarde gegrondvest,~\sep\ en de hemel is het werk van Uw handen.

  Zij zullen vergaan, maar Gij zult blijven,~\sep\ en als een kleed zal alles verslijten.

  Als kleding verwisselt Gij ze, en ze worden verwisseld,~\sep\ maar Gij blijft dezelfde en Uw jaren nemen geen einde.

  De zonen van Uw dienaren zullen in veiligheid wonen,~\sep\ en hun nageslacht blijft bestaan voor Uw aanschijn.

  \psalm{\Ps{103}} Loof, mijn ziel, de Heer, en al wat in mij is, Zijn heilige Naam!

  Loof, mijn ziel, de Heer,~\sep\ en vergeet toch al Zijn weldaden niet.

  Hij vergeeft al uw zonden,~\sep\ en al uw ziekten geneest Hij.

  Hij redt uw leven van de dood,~\sep\ en kroont u met liefde en ontferming.

  Hij vult met goederen uw leven;~\sep\ uw jeugd keert weer als bij een adelaar.

  De Heer oefent werken van gerechtigheid uit,~\sep\ en schaft recht aan alle verdrukten.

  Aan Moses heeft Hij Zijn wegen getoond,~\sep\ Zijn werken aan Israëls zonen.

  Meedogend en genadig is de Heer,~\sep\ lankmoedig en rijk aan ontferming.

  Neen, Hij is niet immer vergramd,~\sep\ en blijft niet eeuwig toornen.

  Hij behandelt ons niet naar onze zonden,~\sep\ en naar onze schuld vergeldt Hij ons niet.

  Want zo hoog als de hemel zich boven de aarde verheft,~\sep\ zo groot is Zijn barmhartigheid voor hen, die Hem vrezen.

  Zo ver het oosten van het westen ligt,~\sep\ zo ver werpt Hij onze schuld van ons af.

  Gelijk een vader zich ontfermt over zijn kinderen,~\sep\ zo ontfermt Zich de Heer over hen, die Hem vrezen.

  Want Hij weet van welk maaksel wij zijn,~\sep\ en bedenkt dat wij stof zijn.

  Als gras zijn de dagen van de mens,~\sep\ hij bloeit als een bloem op het veld:

  Nauwelijks woei er de wind overheen, of ze is er niet meer,~\sep\ en haar plaats erkent haar niet verder.

  Maar van eeuw tot eeuw duurt de erbarming van de Heer voor die Hem vrezen,~\sep\ en Zijn gerechtigheid voor de kinderen van hun kinderen.

  Voor hen, die Zijn verbond onderhouden,~\sep\ er aan denken Zijn geboden na te leven.

  De Heer heeft Zijn troon in de hemel gevestigd,~\sep\ Zijn heerschappij omvat het heelal.

  Looft de Heer, gij, al Zijn engelen, die, machtig en sterk, Zijn bevelen volvoert,~\sep\ om te gehoorzamen aan Zijn woord.

  Looft de Heer, gij, al Zijn legerscharen,~\sep\ gij, Zijn dienaren, die Zijn Wil volbrengt.

  Looft de Heer, gij, al Zijn werken, over heel het gebied van Zijn macht,~\sep\ loof, mijn ziel, de Heer!
\end{halfparskip}

\begin{halfparskip}
  \markedsubsectionrubricwithhint{Vrijdagen ``na'': Marmita 46*}{(origineel: hulala 18)}

  \psalm{\Ps{112}} Gelukkig de man, die de Heer vreest,~\sep\ zich in Zijn geboden ten zeerste verheugt.

  \liturgicalhint{Alleluia, Alleluia, Alleluia.~--- Eerste vers.}

  Zijn kroost zal machtig zijn op aarde;~\sep\ het geslacht der vromen zal worden gezegend.

  Schatten en rijkdommen zijn in zijn huis,~\sep\ en immer zal zijn vrijgevigheid duren.

  Als een licht in het duister rijst op voor de goeden~\sep\ de Zachtmoedige, Barmhartige en Rechtvaardige.

  Goed vergaat het de man, die zich ontfermt en uitleent,~\sep\ die zijn zaken rechtvaardig beheert.

  In eeuwigheid zal hij niet wankelen;~\sep\ in eeuwig aandenken blijft de rechtvaardige.

  Voor kwade tijding zal hij niet vrezen;~\sep\ zijn hart is onwrikbaar, vertrouwend op God.

  Zijn hart is standvastig, hij zal niet vrezen,~\sep\ totdat hij zijn vijanden vernederd ziet.

  Hij deelt uit, geeft aan de armen, zijn mildheid duurt immer voort;~\sep\ zijn hoorn zal zich in luister verheffen.

  De zondaar zal het zien en zich ergeren, knarsen op de tanden en wegteren;~\sep\ het verlangen van de zondaars vergaat.

  \psalm{\Ps{113}} Looft, gij dienaren van de Heer,~\sep\ looft de Naam van de Heer.

  De Naam van de Heer zij geprezen,~\sep\ en nu en tot in eeuwigheid.

  Van het rijzen der zon tot haar dalen,~\sep\ zij de Naam van de Heer geprezen

  Hoogverheven is de Heer boven alle volken;~\sep\ boven de hemelen schittert Zijn glorie.

  Wie is gelijk aan de Heer, onze God, die troont in de hoge,~\sep\ en neerziet op hemel en aarde?

  Hij heft de behoeftige op uit het stof,~\sep\ uit het slijk verheft Hij de arme,

  Om hem te plaatsen onder de vorsten,~\sep\ onder de vorsten van zijn volk.

  Hij doet de onvruchtbare wonen in haar huis,~\sep\ als blijde moeder van kinderen.

  \psalm{\Ps{114}} Toen Israël uit Egypte trok,~\sep\ het huis van Jacob uit een volk van barbaren.

  Werd Juda zijn heiligdom,~\sep\ en Israël zijn rijksgebied.

  De zee zag het en vluchtte,~\sep\ de Jordaan week terug.

  Als rammen sprongen de bergen op,~\sep\ de heuvelen als lammetjes.

  Wat is er, O zee, dat gij vlucht,~\sep\ en gij, Jordaan, dat gij terugwijkt?

  Bergen, dat gij opspringt als rammen,~\sep\ en gij, heuvelen, als lammetjes?

  Beef, gij aarde, voor het aanschijn van de Heer,~\sep\ voor het aanschijn van Jacobs God,

  Die de rots herschiep in een waterpoel,~\sep\ en de klip in een waterbron.

  \liturgicalhint{B.} Niet ons, O Heer, niet ons, maar geef glorie aan Uw Naam,~\sep\ om Uw erbarming en trouw.

  Waarom zouden de heidenen zeggen:~\sep\ ``Waar is toch hun God?''

  Onze God is in de hemel;~\sep\ Hij deed al wat Hij wilde.

  Hun goden zijn zilver en goud,~\sep\ maaksels van mensenhanden.

  Ze hebben een mond en spreken niet;~\sep\ ze hebben ogen en kunnen niet zien.

  Ze hebben oren en horen niet;~\sep\ ze hebben een neus en ruiken niet;

  Ze hebben handen en tasten niet;~\sep\ ze hebben voeten en wandelen niet; geen geluid komt uit hun keel.

  Daaraan worden gelijk, die ze maken,~\sep\ en allen, die er vertrouwen in stellen.

  Het huis van Israël vertrouwt op de Heer:~\sep\ Hij is hun Helper en schild.

  Het huis van Aäron vertrouwt op de Heer,~\sep\ Hij is hun Helper en schild.

  Die de Heer vrezen, vertrouwen op de Heer:~\sep\ Hij is hun Helper en schild.

  De Heer is ons indachtig,~\sep\ en zal ons zegenen;

  Zegenen zal Hij Israëls huis;~\sep\ zegenen zal Hij het huis van Aäron.

  Zegenen hen die de Heer vrezen,~\sep\ zowel kleinen als groten.

  De Heer zal u vermenigvuldigen,~\sep\ u en uw kinderen.

  Weest gezegend door de Heer,~\sep\ die hemel en aarde gemaakt heeft.

  De hemel is de hemel van de Heer,~\sep\ maar de aarde gaf Hij aan de kinderen der mensen.

  Niet de doden prijzen de Heer,~\sep\ noch hij, die neerdaalt in het rijk der doden.

  Maar wij, wij prijzen de Heer,~\sep\ en nu en tot in eeuwigheid.
\end{halfparskip}

\begin{halfparskip}
  \markedsubsectionrubricwithhint{Zaterdagen ``na'': Marmita 55*}{(origineel: hulala 21)}

  \psalm{\Ps{141}} Luid roep ik tot de Heer,~\sep\ luid smeek ik de Heer;

  \liturgicalhint{Alleluia, Alleluia, Alleluia.~--- Eerste vers.}

  Voor Hem stort ik mijn zorgen uit,~\sep\ voor Hem leg ik mijn kommer bloot.

  Als mijn geest in mij is beangst,~\sep\ kent Gij mijn weg.

  Op het pad, waarlangs ik ga,~\sep\ heeft men mij heimelijk een strik gelegd.

  Ik wend mij naar rechts en zie uit,~\sep\ maar niet één, die om mij zich bekommert,

  Er is voor mij geen uitweg meer,~\sep\ en niemand draagt zorg voor mijn leven.

  Ik roep tot U, O Heer, ik zeg: Gij zijt mijn toevlucht,~\sep\ mijn aandeel in het land der levenden.

  Geef acht op mijn geroep,~\sep\ want diep ellendig ben ik geworden.

  Ontruk mij aan die mij vervolgen,~\sep\ want sterker zijn ze dan ik.

  Leid mij uit de kerker,~\sep\ opdat ik Uw Naam moge danken.

  De rechtvaardigen zullen mij omringen,~\sep\ wanneer Gij mij hebt welgedaan.

  \psalm{\Ps{142}} Heer, luister naar mijn bede, hoor mijn smeken om Uw trouw,~\sep\ verhoor mij toch om Uw gerechtigheid!

  Daag Uw dienstknecht niet voor het gericht,~\sep\ want niemand, die leeft, is rechtvaardig voor U.

  Zie, de vijand vervolgt mij: hij wierp mij neer op de grond,~\sep\ hij heeft mij geplaatst in het duister, als hen, die reeds lang zijn gestorven.

  Mijn geest bezwijkt in mij,~\sep\ ontsteld was mijn hart in mijn boezem.

  1k gedenk de aloude dagen, over al Uw daden denk ik na,~\sep\ ik overweeg de werken van Uw handen.

  Ik strek mijn handen naar U uit;~\sep\ als dorre aarde dorst mijn ziel naar U.

  Heer, verhoor mij spoedig,~\sep\ want mijn geest bezwijkt.

  Wil Uw aanschijn voor mij niet verbergen,~\sep\ opdat ik niet worde als zij, die neerdalen in het graf.

  Laat mij spoedig Uw goedheid ontwaren,~\sep\ daar ik vertrouw op U.

  Toon mij de weg, die ik moet gaan,~\sep\ daar ik mijn ziel verhef tot U.

  Verlos mij van mijn vijanden, Heer:~\sep\ op U stel ik mijn hoop.

  Leer mij Uw wil volbrengen,~\sep\ want Gij zijt mijn God.

  Goed is Uw Geest,~\sep\ Hij geleide mij op effen grond.

  Om Uw Naam, O Heer, behoud mij in het leven;~\sep\ om Uw goedertierenheid red mij uit de nood.

  Verdelg in Uw goedheid mijn vijanden, en richt allen, die mij kwellen, te gronde,~\sep\ want ik ben Uw dienstknecht.

  \psalm{\Ps{143}} Geprezen zij de Heer, mijn Rots,~\sep\ die mijn handen africht ten krijg, mijn vingers ten oorlog,

  Mijn Barmhartigheid en mijn Burcht,~\sep\ mijn Schuts en mijn Bevrijder,

  Mijn Schild en mijn Toeverlaat,~\sep\ die de volken aan mij onderwerpt.

  Heer, wat is de mens, dat Gij zorg voor hem draagt,~\sep\ het mensenkind, dat Gij het gedenkt?

  De mens is als een bries,~\sep\ zijn dagen als een vluchtige schaduw.

  Heer, laat Uw hemel neer en daal af,~\sep\ raak de bergen aan, en zij zullen roken.

  Slinger de bliksem, verstrooi hen,~\sep\ schiet Uw pijlen af, jaag hen uiteen.

  Reik uit de hoge Uw hand,~\sep\ ontruk en bevrijd mij uit de watervloed, uit de hand van vreemdelingen,

  Wier mond slechts leugentaal spreekt,~\sep\ en wier rechter valselijk zweert.

  Een nieuw lied, O God, zal ik U zingen,~\sep\ op de tiensnarige harp spelen voor U,

  Die aan koningen de zege schenkt,~\sep\ die David, Uw dienaar, bevrijd hebt.

  Aan het moordende zwaard ontruk mij,~\sep\ en bevrijd mij uit de hand van vreemdelingen,

  Wier mond slechts leugentaal spreekt,~\sep\ en wier rechter valselijk zweert.

  Onze zonen mogen zijn als planten,~\sep\ opgroeiend in hun jeugd;

  Onze dochters als hoekpijlers,~\sep\ gehouwen als tempelzuilen.

  Dat onze schuren vol zijn,~\sep\ ja, overvol van allerlei vruchten;

  Dat onze schapen, in duizendvoudige vruchtbaarheid, zich bij tienduizenden vermenigvuldigen op onze velden;~\sep\ dat onze runderen beladen zijn.

  Er zij geen bres of doorgang in de muren,~\sep\ en geen weeklacht in onze straten.

  Gelukkig het volk, wie zo'n lot ten deel valt,~\sep\ gelukkig het volk, wiens God de Heer is.
\end{halfparskip}

\begin{halfparskip}
  \liturgicalOption{Alle dagen:}~\sep\ \dd~Alleluia, alleluia; laat ons bidden; vrede zij met ons.

  \cc~Moge het gebed van onze broosheid U behagen, o onze Heer en onze God, moge het verzoek van onze zwakheid tot bij U komen, moge Uw barmhartigheid een grote voorspraak zijn voor onze zondigheid; en moge uit de grote schat van Uw mededogen de smeekbeden van onze nood in alle seizoenen en tijden worden beantwoord, Heer van alles...
\end{halfparskip}

% % % % % % % % % % % % % % % % % % % % % % % % % % % % % % % % % % % % % % % %

\markedsection{Onyata D'Mawbta}

\begin{halfparskip}
  \liturgicalhint{Maandag: Onita d'basaliqe; Dinsdag: d'lelya; Donderdag: d'sapra; Vrijdag: eigen; Zaterdag: d'raze.}
\end{halfparskip}

\begin{halfparskip}
  \markedsubsectionrubric{Woensdagen ``voor''.}

  Koningen van de aarde en alle volkeren.~--- Alle naties noemen gezegend de maagd Maria, de moeder van God

  Alle dingen achter en tevoren.~--- Alle naties noemen gezegend de maagd Maria, de moeder van Christus.

  U bent goed en een zegen voor uw ziel.~--- Wel bent u, Johannes, en altijd zult u wel zijn, wanneer de Heer komt, die u liefhad en diende.

  Laat de rechtvaardigen met eer gesterkt worden.~--- Predikers van de Geest en pilaren van licht waren Petrus en Paulus in de schepping.

  Hun evangelie is over de hele aarde uitgegaan.~--- Mattheus, Markus, Lukas en Johannes, moge uw gebed een muur zijn voor onze ziel.

  En Ik zal hem ook de eerstgeborene maken.~--- Gezegend is de dood van de eerstgeborene der martelaren. Stefanus, de vriend van Christus.

  Hij zal voor altijd voor God staan.~--- Laten we ijverig de herdenking der leraren vieren, die Christus liefhadden en Zijn geboden hebben onderhouden.

  Zoals Hij onze voorvaderen bevolen heeft.~--- Laten we de overwinningen van onze geestelijke vaders herdenken, die voor ons hebben gezwoegd door de liefde van Christus.

  Uw herdenking door alle generaties heen.~--- Uw herdenking, o onze vader [patroon], is op het heilig altaar, met de heiligen die overwonnen hebben en de martelaren die gekroond zijn.

  De kleinen zowel als de groten.~--- Zie, al onze overledenen zijn ontslapen in Uw hoop, dat U hen in Uw glorieuze verrijzenis zou opwekken in Uw heerlijkheid.

  Stort uw harten voor Hem uit.~--- Door vasten, gebed en bekering van de ziel, laten we Christus en Zijn Vader en Zijn Geest gunstig stemmen.

  Ik zal de Heer altijd zegenen.~--- Gezegend is Christus die Satan veroordeelde, en onze natuur deed overwinnen door Zijn heilig vasten.

  Kijkt naar Hem en vertrouwt op Hem.~--- Met het oog van de Geest keken de kinderen naar u toen U binnentrok, en ze zongen lof.

  Verblijdt u, zingt en dankt.~--- Uw verrijzenis, o onze Verlosser, heeft de schepping verblijd, en vernietigde de [heidense] altaren en bevestigde de Kerken.

  De aarde beefde en werd bewogen.~--- De aarde beefde toen ze riepen: Kruisigt, kruisigt de Koning der Joden.

  Vreugde in de hele aarde.~--- In de verrijzenis van de Zoon verheugen de schepselen zich, want verzoening was volbracht en de verrijzenis begon te heersen.

  Zoekt de Heer en weest sterk.~--- Martelaren, bidt voor erbarmen voor de wereld, die toevlucht zocht in de kracht van uw beenderen.

  Smeek de Heer en bid vóór Hem.~--- Vraag voor ons van uw Heer, martelaar Joris, mededogen, barmhartigheid en vergeving van zonden.

  God is opgegaan in glorie, de Heer met het geluid van de bazuin.~--- Gezegend is de Koning die opgestegen is en door Zijn hemelvaart de engelen en mensen heeft verblijd, en die de schepselen deed juichen.

  Laat ons de Heer zegenen die ons gemaakt heeft.~--- Gezegend is de nederdaling van de Geest, die Zijn apostelen wijs maakte (\translationoptionNl{onderwees}), en ze de overwinning in de vier windstreken bezorgde.

  Ons hart zal zich in Hem verheugen.~--- Moge uw Kruis, o onze Verlosser, voor ons een wapen zijn; en mogen we door dat Kruis de boze en al zijn listen overwinnen.

  Met lofprijzingen.~--- Ziet, de Kerk roept met heilige stemmen; en in haar prijzen zij de Heer der schepselen.
\end{halfparskip}

\begin{halfparskip}
  \markedsubsectionrubric{Woensdagen ``na''.}

  Hij is uw Heer, breng Hem uw hulde.~--- Maria, heilige maagd, smeek en bid Christus, dat Hij de wereld genadig mag zijn, die toevlucht zoekt in uw gebed. En laat de Kerk zich verheugen in uw feest; moge haar kinderen behoed zijn voor schade en tegenstand en de oppositie van de duivel, de afvallige.


  Zoek Hem en smeek Hem dat Hij medelijden met ons heeft.~--- Maria, heilige maagd...

  Hij deelt uit, geeft aan de armen.~--- Open de schat die uw Heer u heeft gegeven, o Johannes de prediker, en geef hulp aan de armen, die hun toevlucht bij U zoeken en uw naam aanroepen. Geef de zieke genezing, wees een troost voor wie bedroefd is en moge Uw gebed een muur zijn voor al wie gekweld wordt door de boze.

  Op de rots heeft Hij mijn voeten geplaatst.~--- De apostelen, onbeweeglijke rotsen, bouwden een onvergankelijk gebouw door de kracht die zij van hun Heer ontvingen. Ze roeiden het heidendom uit en bouwden de Kerk. Wees gegroet, leerlingen van de waarheid, die uw gebouw hebben voltooid en opgetrokken, en tempels van de Geest hebben versierd en gebouwd in de zielen der gelovigen.

  Zoek de Heer en wees sterk.~--- Uitverkoren en heilige apostelen, bidt en tracht te verkrijgen van de Heer dat er vrede mag zijn in de schepping, dat oorlogen en twisten ophouden, dat er eendracht mag zijn onder priesters, dat verzoening onder de koningen zich mag vermenigvuldigen, en dat in de vier delen van de wereld er grote vrede en rust mag heersen.

  Bied hem lofoffers aan.~--- De uitverkoren Stefanus offerde zijn lichaam en ziel op nobele wijze op, want hij zag zijn Heer gebonden worden voor de rechter die hem zou laten geselen. Ook hij offerde zijn lichaam ter dood door steniging, ontving de kroon van de overwinning en was de eerstgeborene van het martelaarschap.

  Uit de mond van kind en zuigeling.~--- Lof aan U, onze Verlosser, want door Uw kracht werd de overwinning gegeven aan drie geweldige mannen, zuivere tempels van de Heilige Geest. Mar Joris, die werkte door discussie, en de illustere mar Basil, en mar Johannes. Moge hun gebed een wal voor ons zijn.

  Als blijde moeder van kinderen.~--- Bij der herdenking der priesters, zie, de glorieuze en heilige Kerk verheugt zich, want zij stonden in haar op als pilaren, zodat er onder haar zonen leraren zouden zijn. In elk tijdperk overwint hun leer en hun ware geloof, want zij berispten koningen en werden niet overweldigd door hun goddeloze bedreigingen.

  De Heer gedenke al uw offergaven.~--- O onze vader, die overwon in de wedstrijd, zie, uw beloning is in de hemel. Christus, voor wie U uzelf hebt versierd; heeft uw herdenking in Zijn Kerk verheven. Uw liefde was een puur wierookvat en u stemde uw Heer gunstig door uw inspanningen. Smeek met ons, dat wanneer Hij wordt geopenbaard in grote heerlijkheid, Hij ons genadig moge zijn.

  Tot U komt alle vlees.~--- Tot U, mijn Heer, die de zonden aan allen vergaf, komt alle vlees. Mogen de lichamen, verontreinigd door de zonde, met Uw hysop wit worden gemaakt. Komt, stervelingen, beladen met lasten, legt het gewicht van uw zonden neer. Neemt van het altaar de steenkool die de profeet vergiffenis schonk, en weest vrijgesproken.

  Hij zei: ``Bekeert u, mensenkinderen''.~--- Bij de prediking van de profeet werd Nineve gered door te vasten. En door voorbede en gebed hield het de engel des doods tegen. Laten we, in lijden en tranen van berouw, onze toevlucht zoeken in de tempel, en roepen om genade om ons te helpen tot het licht vervaagt.

  Aan U, o God, komt een lofzang toe in Sion.~--- Lof aan U, onze Verlosser; toen U Jeruzalem binnenkwam, en jongens met olijftakken voor U zongen en zeiden: ``Hosanna voor U in de hoge; hosanna voor de Zoon van David. Gezegend is Uw komst naar ons in de kracht en glorie van Uw engelen.''

  Die mijn brood at, op wie ik vertrouwde, heeft mij erg bedrogen.~--- Onze Heiland brak brood en gaf het aan de bedrieglijke leerling; en Satan voer in Hem, en hij werd een nutteloos vat en veranderde niet van gedachten, maar stond toe dat bedrog in hem doordrong. Hij werd een vreemde voor zijn roeping en werd niet geholpen door zijn zegen.

  De Heer heeft Zijn heil doen kennen.~--- Onze Heer maakte Zichzelf bekend en toonde hoe zwaar (Zijn) lijden was; en Hij waakte en werkte, zodat, als het mogelijk was, dat uur aan Hem voorbij zou gaan. Hij waarschuwde ons: ``Ontwaakt en bidt, dat je niet in bekoring komt''. Onze Heer getuigde dat lijden zwaar is, want Zijn zweet was in bloed veranderd.

  Hij is tot de hoeksteen van het gebouw geworden.~--- Uw verrijzenis, o onze Heer, was leven voor het mensenras dat vernietigd was. Want ze zijn opgestaan en levend gemaakt en verworven leven door de grote kracht van het Kruis. Zij zwoeren af en verwierpen het heidendom en alle afgoderij. En ze knielden en aanbaden het ene Wezen, Die door Zijn levende Zoon de wereld heeft gered.

  Hij redde hen niet van de dood.~--- Gij die werd vermoord en uw Schepper liefhad, en uit eigen vrije wil de dood aanvaardde, en zoenoffers waart voor Christus Koning, die u kroonde, biedt met ons een verzoek aan, dat wij op de grote dag van beproeving (\translationoptionNl{bezoeking}) gered mogen worden van kwellingen en het eeuwig leven mogen beërven.

  Gord uw zwaard om de heup, gij machtige held.~--- Een sterke reus was Mar Joris, die de dood, het zwaard, allerlei soorten bittere verminkingen en striemen verachtte, en tekens en wonderen verrichtte. Hij bekeerde alle mensen naar de waarheid. Gezegend is Hij die de atleten de overwinning schonk, door Wiens kracht we de dwaling zullen overwinnen.

  Ik verheerlijk het Woord van God.~--- Het Woord van de Vader wilde in Zijn liefde ons ras redden dat vernietigd was. Hij nam van ons de gelijkenis van een dienaar aan, en steeg op en ging aan de rechterhand zitten. Verheerlijkt (Hem) nu, jullie stervelingen, want zie, de Zoon van ons ras is in de hemel, geprezen door de verheven wezens, de cherubijnen, serafijnen en engelen.

  Zingt een nieuw lied voor de Heer.~--- De apostelen verrichtten een nieuw wonder in de tempel van Jeruzalem, toen zij in de Naam van Jezus de man lieten lopen die kreupel was vanaf de schoot van zijn moeder. De menigte der kruisigers verwonderde zich toen ze de grootsheid zagen van het wonder, dat in de Naam van Jezus, Die ze hadden gekruisigd, de kreupele man deed lopen.

  Uw bliksems verlichtten het aardrijk.~--- Het Kruis van licht in de hemel, dat aan Constantijn werd getoond, ging als een bevelhebber aan het hoofd van het kamp ten strijde en beroerde en beangstigde de groepen heidenen die schepselen aanbaden; zij verlieten de dwaling van het heidendom en aanbaden en vereerden het Kruis.

  De koningsdochter stond in volle glorie.~--- O Kerk, de bruid van Christus, die U door Zijn bloed van dwaling heeft gered en U door Zijn verrijzenis leven en onvergankelijke zegeningen heeft beloofd, versier uzelf met pracht, zeg dank, en belijd zonder twijfelen het ware geloof.
\end{halfparskip}

\begin{halfparskip}
  \markedsubsectionrubric{[Woensdagen ``voor'' en ``na''.]}

  Eer aan de Vader, de Zoon en de Heilige Geest.~--- Door het gebed van de Gezegende moge er vrede heersen in de schepping; en op verzoek van de maagd, mogen de kinderen van de Kerk beschermd worden.

  Moge de Macht, die neerdaalde van boven en haar zodanig heiligde en versierde voor Zijn eer dat zij het ware Licht voortbracht, en de hoop en het leven der schepselen is, bij ons en onder ons zijn alle dagen van ons leven, en de zieken en zwakken genezen, en degenen die in bekoringen zijn gevallen; en hen die op verre reizen zijn in vrede terugbrengen naar hun huizen, dat ze niet geschaad worden door de boze.

  Mogen degenen die over zee reizen gered worden van de golven, en zij die op het droge gaan verlost worden van de barbaren. Mogen degenen die gevangen zijn genomen bevrijd worden van hun banden, en zij die met geweld zijn meegenomen getroost worden in hun verdriet door Uw mededogen. Als iemand door de boze wordt gekweld, moge Uw grote macht hem dan berispen. Als iemand in zonden volhardt, vergeef en scheld kwijt zijn overtredingen. Als iemand offers heeft gebracht, moge Uw Godheid tevreden zijn met hen. En in Uw goedertierenheid maak levend en breng tot leven zij die in Uw hoop zijn ontslapen.

  Mogen wij die onze toevlucht hebben gezocht tot het gebed van de gezegende Maria, de heilige maagd, de moeder van Jezus, onze Verlosser, daardoor beschermd worden van de boze en al zijn listen overwinnen.

  En mogen wij op die grote dag van beproeving (\translationoptionNl{bezoeking}), wanneer de goeden worden gescheiden van de bozen, samen met haar waardig zijn om vreugde te hebben in de bruidskamer van het hemelse koninkrijk en het drievoudig lied van glorie te zingen voor de Vader, Zoon en Heilige Geest.

  Vanaf het begin en in alle eeuwigheid.~--- Zie, de afdelingen en orden der geestelijke wezens, samen met de priesters in de Kerk, zingen lof voor de herdenking van de heilige martelaar, Mar Joris, die zegevierde, overwon, en gekroond werd. Hij leed onder moeilijkheden en kwellingen. De vervolgers onderwierpen hem aan vuur, zwaard en steniging, en verschillende kwellingen. Hij beschaamde de boze koning, die de goede dienaren vervolgde; en verachtte zijn verheven macht en de goden die hij aanbad: Zeus, Apollo, Artemis, het werk van mensenhanden.

  De krachtige reus, Mar Joris riep en zei tot de edelen van de koning: ``Jullie mogen geen afgoden aanbidden, gesneden en bewerkt door ambachtslieden. Want zie, Christus is de Koning der koningen en Heer van alle goden. Hij geeft een erfenis aan allen die Hem vrezen: een bruidskamer en een altijddurende zegen; en Hij bekleedt met glorie in Zijn koninkrijk de illustere martelaren die in Hem hebben geloofd.''

  Knielend in gebed voor zijn Heer, bad en deed hij een verzoek, zeggende: ``Neem in Uw goedheid van iedereen die deze dag van mijn vervolging herdenkt hagel, hongersnood, pest, sprinkhaan, jonge sprinkhaan, larven en de verschroeiende hitte weg die de velden vernietigt, de terreur van de nacht en alle kwade plagen, en bewaar de hele wereld door de grote kracht van Uw Godheid.''

  Laat al het volk zeggen: Amen en amen.~--- Met de met licht omhulde gezelschappen en de engelenkoren is de bruidskamer voor U vervlochten\footnote{Om de bruidskamer af te sluiten met vlechtwerk.} in de hemel aan de rechterhand van de Koning, Christus onze Verlosser.

  Onze nobele en heilige vader [moeder] Mar[t]~\NN\ de illustere, die een werk van gerechtigheid verrichtte in de Kerk, de echtgenote van Christus, en hield van vasten, gebed en volmaakte en ware liefde. Zie, vanuit de plaats waar uw edele lichaam is neergelegd stroomt hulp en genezing naar allen die getroffen zijn, en toevlucht zoeken in uw gebeden.

  Groot is de kracht waarmee u de tegenstander [duivel] hebt overwonnen, en die u naar het land heeft gebracht dat vol is van zegeningen. Smeek nu voor ons allen dat we samen met u in het koninkrijk een nieuwe eeuwigdurende lof mogen zingen aan de Vader, Zoon en Heilige Geest.

  God, onze eigen God, zal ons Zijn zegen geven. God zal ons zegenen. Moge Hij die de gerechtige Abram zegende en Isaak redde, bij ons en onder ons zijn. Moge God, die naar de zuivere Jacob keek, onze vergadering zegenen met Zijn rechterhand.

  Moge de Heer, die bij Jozef was, de bewaker van onze wegen zijn. Moge de Machtige, die naar Mozes keek, ons te allen tijde vergezellen. Moge Hij die de overwinning gaf aan Jozua, de zoon van Nun, met Zijn rechterhand ons overschaduwen. Moge Hij die allen hoedt en David heeft uitgekozen, ons behoeden voor de boze en zijn leger.

  Moge de Heer, die Salomo verrijkte. de rijkdom van onze velden vergroten. Moge de Allerhoogste, die naar Jesaja keek, Zijn vrede onder ons laten wonen. Mogen onze gebeden aanvaard worden zoals die van Elia en Elisa, en moge ons verzoek gehoord worden als dat van de rechtvaardige Daniël.

  Moge Jezus, de Heer der profeten en Kroon der apostelen, ons bewaren, die in Zijn Naam hebben geloofd.

  Moge de glorieuze Drie-eenheid voor altijd bij ons zijn.

  Hij bracht hen bijeen uit alle landen.~--- Uit Cush, Egypte en Thebais vertrokken de heiligen en gingen verder: Mar Augin en Mar Shalita\footnote{Eugenius: Egyptische monnik die het monnikendom naar Syrië bracht; Shalita was één van zijn 72 leerlingen.}, architecten van de heilige Kerk;~\sep\ met het gezelschap van edele oude mannen, de gezegende 72 rechtvaardigen, zuiver en maagdelijk, machtig in roem, uitverkoren vaten van eer en begeerlijke tempels van zuiverheid, havens van vrede en rust, en schatkamers van liefde en eendracht, overstromende bronnen van barmhartigheid, en gezegende fonteinen van medelijden, zeeën vol wijsheid en rivieren van kennis, schatten van vasten en gebed, beroemde verblijfplaatsen van de Heilige Geest.

  Gelijkenissen en types der profeten, beelden en portretten der apostelen, afbeeldsels en weergaven der leraren, belichamingen en perfecties der priesters, beroemd onder de vaders, ijverig onder de asceten, jukgenoten der edele martelaren, collega's der belijders, makkers der kluizenaars en metgezellen der rouwenden [=monniken].

  Aanbiddelijk is de Vader die hen heeft uitverkoren, heilig is de Zoon die hen de overwinning gaf, glorieus is de Geest die hen aanmoedigde om te volharden in Zijn leer. Helder is het licht waarmee zij gekleed zijn, uitmuntend is de heerlijkheid waarin zij behagen scheppen.

  Sterk, stevig, gevestigd en hoog is de hoge muur van hun gebeden, die de bijeenkomsten van hun zonen omringt. Gezegend is Hij van Wiens liefde de rechtvaardigen dronken zijn, en die Zijn waarheid in de schepping hebben gepredikt.

  De berg waar het God behaagd heeft te wonen.~--- Op de moeilijke berg Mirda, Izla, die ontoegankelijker, moeilijker en uitgedroogder is dan alle bergen ter wereld, behaagde het de goddelijke goedheid mensen van vuur en geest te maken\footnote{Bergkam nabij Nisibis, bekend om vele vroege kloosters. De hymne verwijst naar de hartstocht en ascetische levensstijl der monniken die daar woonden en zware omstandigheden doorstonden om hun geloof na te streven.}.

  Reuzen gekleed met waarheid en metgezellen der engelen. Allereerst is er Mar Augin, wiens naam zoet betekent. Zijn naam getuigt van zijn overwinningen, in vertaling is hij een goede man, geestelijk in waarheid, die zijn koers daar versnelde. En 72 gezegenden kwamen met hem mee uit Egypte: na korte tijd Mar Andreas en Mar Ulugh, de beroemde oude man, en Mar Johannes, de Arabier van Khirta naar ras en familie, en spiritueel in waarheid, die wonderbaarlijke wonderen deed en een dode man tot leven wekte door zijn gebed\footnote{Johannes was een monnik van een rijke familie in Ḥirta; een klooster werd gebouwd over zijn graf.}.

  En Mar Abraham van Kashkar, zoon uit het land van de grote Abram, veel groter dan kan worden uitgedrukt is de genade die in hem woonde. En Mar Babai, de leerling van de waarheid, die op de ware weg wandelde, en niet laks was en niet geschokt (\translationoptionNl{verstoord}) werd door het conflict met de macht van de boze. En van `s~morgens tot `s~avonds doorstond hij hitte en kou\footnote{Abraham, ° 492 in Kashkar, Perzië, stichtte een klooster op de berg Izla, het ``Grote Klooster''. / Babai: ° 551; formuleerde Assyrische christologie; monastieke vitator en coadjutor van Mar Aba; derde abt van het Groot Klooster.}.

  En Mar Kudahwai de illustere, die blij was in de ranken [der martelaren] en werd opgetild op de vleugels van de Geest naar het goede en glorieuze land.

  Door hen werd vervuld wat is geschreven: dat van de top der bergen alle generaties en stammen, oude mannen, jonge mannen en knapen, die naar hun aangewezen woningen komen, zullen roepen en prijzen tot de Heer; en dat ze daar hulp mogen ontvangen; dat wij ook door hun gebeden waardig mogen worden om met hen vreugde te hebben.

  De hoop van alle grenzen der aarde.~--- Christus, hoop van Uw gelovigen, de profeten, apostelen, martelaren, priesters en leraren, bewaar onze zielen door hun gebeden, en sta ons toe hun paden te bewandelen.

  Zegen, Heer, de heilige Kerk waarin hun beenderen opgeborgen zijn. Zegen en bewaar uw kudde, die aan hun handen was toevertrouwd. Zegen en bewaar Uw erfdeel, dat door hun inspanningen verdiend werd. Zegen en bewaar de gelovigen, die hun feesten vieren. Zegen en bewaar de rijken, die hun graven eren. Zegen en bewaar de armen, die hun toevlucht zoeken in hun gebeden. Zegen en genees de zieken, die schuilen onder de vleugels van hun beenderen. Zegen en bewaar allen die reizen te land of over zee. Zegen en bewaar alle standen. Ondersteun de zwakken door Uw wil. Genees de zieken en zwakken en zij die in bekoringen vallen.

  Zegen de cyclus van de oogsten. Maak vredig de temperatuur van de lucht. Bewaar de priesters en koningen in liefde en geloof.

  Zegen, bewaar, bescherm en red allen die gekweld en bedroefd zijn en toevlucht nemen in het gebed van Uw vrienden: de profeten, apostelen, leraren, martelaren, priesters en monniken. Zegen en bewaar in Uw goedheid de vergadering die de dag van hun herdenkingen plechtig viert.
\end{halfparskip}

% % % % % % % % % % % % % % % % % % % % % % % % % % % % % % % % % % % % % % % %

\CLEARPAGEAV

\liturgicalhint{Behalve op Woensdagen, Vasten, Rogatie der Ninivieten eindigt de onyata d-mawtba op ferias met:}

\markedsection{Qala D'Udrane}

\begin{halfparskip}
  \liturgicaloption{Voor recitatie.} Barmhartige God, heb medelijden met Uw volk, en geef Uw erfenis niet weg aan de minachting en spot der naties. Kijk vanuit Uw verblijfplaats en zie het geweld dat wij ondergaan vanwege de vijanden. Laat Uw genade spoedig over ons schijnen, zodat de naties niet zeggen: Waar is hun God?

  Barmhartige God, bescherm Uw dienaren van hen die (ons) zonder reden haten, en willen grijpen wat hen niet toebehoort. Omhein ons met Uw sterke waarheid, zodat zij die (ons) haten het zien en zich schamen. Omring ons met Uw zorg als met een wal, dat wij daarin beschermd mogen worden en Uw goedheid mogen belijden.

  Door het gebed van haar die U gebaard heeft, Heer Jezus, schenk vrede aan de hele wereld, die verontrust en verward is door haar zonden; stop oorlogen en  vijandigheden op aarde; zaai vrede tussen priesters en vorsten, zodat in harmonie en liefde alle dagen de herdenking kan worden gevierd van haar die U heeft gebaard.

  Mogen de rechtvaardigen die U behagen, Heer Jezus, de profeten, apostelen, martelaren en leraren in elk land, U voor onze zielen smeken dat U met ons allen genadig bent; en maak ons waardig om U met hen te belijden op de dag dat zij de beloning voor hun daden ontvangen.

  Gezegend bent u, onze glorierijke, heilige vader; die uw loopbaan zegevierend hebt voltooid; en in de wijngaard van Christus hebt gewerkt en de beloning ontvangen hebt voor uw daden, de cent der hemelse goederen. Mogen uw gebeden op ons allen rusten en mogen we waardig zijn om met u het koninkrijk te beërven.
\end{halfparskip}

\CLEARPAGEAV
\liturgicaloption{Of om te zingen.} Vanaf het begin en in alle eeuwigheid, amen en amen.

\begin{singlecolsong}
  \textsizexi

  \begin{halfparskip}
    \songline{1. Full of mercy, You, our Lord}
    \songline{look with favour upon us,}
    \songline{we who are your own people.}
    \songline{You know well, O Lord our God,}
    \songline{the oppression of our foes:}
    \songline{pour forth Your grace upon us.}

    \songindentv
    \songline{2. Sea of mercy, You, our Lord,}
    \songline{protect us, Your children well}
    \songline{from all those who exploit us;}
    \songline{let the truth be our weapon}
    \songline{and it shall be our fortress}
    \songline{that we may sing Your glory.}

    \songindentv
    \songline{3. O Lord Jesus, sow Your peace}
    \songline{in this world of disorder}
    \songline{through the prayers of Mary.}
    \songline{Remove from this world, O Lord,}
    \songline{armament and enmity}
    \songline{that we bless her with one voice.}

    \songindentv
    \songline{4. Lord Jesus, may Your prophets,}
    \songline{apostles and Church-fathers,}
    \songline{doctors and martyrs for faith,}
    \songline{pray for us that with one voice}
    \songline{we may sing with all of them,}
    \songline{praising Your great Majesty.}

    \songindentv
    \songline{5. India's Sliha, mar Thoma,}
    \songline{intercede now for us all,}
    \songline{with our Lord and God Jesus,}
    \songline{that your children may enjoy}
    \songline{bliss of promised adoption}
    \songline{in the kingdom at your side.}

    \songindentv
    \songline{6. He (She) who gained a rich reward}
    \songline{working with true joy and love}
    \songline{in the vineyard of the Lord.}
    \songline{May his (her) supplications bring}
    \songline{to this parish (\liturgicaloption{or:} \emph{ashram}, \emph{bhavan}, \emph{household}) of us all}
    \songline{God's blessing and His mercy.}

    \songindentv
    \songline{7. Recall not our many sins,}
    \songline{we who received Your body}
    \songline{and have drunk from Your chalice.}
    \songline{On that day when You judge us,}
    \songline{may Your mercy raise us up}
    \songline{with all those who welcome You.}
  \end{halfparskip}
\end{singlecolsong}

\begin{halfparskip}
  \dd~Laat ons bidden, vrede zij met ons.

  \cc~Aan U zij de glorie van hen die in de hoge zijn, de belijdenis van hen die beneden zijn; aanbidding, lof en verheerlijking van allen die U in de hemel en op aarde hebt geschapen en gevormd, Heer van alles...
\end{halfparskip}

% % % % % % % % % % % % % % % % % % % % % % % % % % % % % % % % % % % % % % % %

\markedsection{Subaha}

\begin{halfparskip}
  \liturgicalOption{Maandag.} \psalm{\Ps{13}}

  Hoelang nog, Heer, zult Gij mij geheel vergeten,~\sep\ hoelang nog voor mij Uw aanschijn verbergen?

  Eer aan U, O God \liturgicalhint{(3x)}~---  \liturgicalhint{Eerste vers.}

  Hoelang nog zal ik de smart overdenken in mijn ziel,~\sep\ en het wee in mijn hart van dag tot dag?

  Hoelang nog zal mijn vijand zich boven mij verheffen?~\sep\ Zie neer en verhoor mij, O Heer, mijn God!

  Stort licht in mijn ogen, opdat ik de doodsslaap niet inga,~\sep\ en mijn vijand niet zegge: ``Ik heb hem overwonnen'';

  En mijn weerstrevers niet juichen over mijn val,~\sep\ daar ik op Uw erbarming vertrouwde.

  Nu juiche mijn hart om Uw hulp!~\sep\ De Heer wil ik bezingen, die mij heeft welgedaan.
\end{halfparskip}

\begin{halfparskip}
  \liturgicalOption{Dinsdag.} \psalm{\Ps{28}}

  Tot U roep ik, O Heer;~\sep\ mijn Rots, wees niet doof voor mij.

  Opdat ik niet, als Gij niet hoort naar mij,~\sep\ gelijk worde aan hen, die in de grafkuil dalen.

  Eer aan U, O God \liturgicalhint{(3x)}~---  \liturgicalhint{Eerste vers.}

  Hoor de stem van mijn smeken, nu ik roep tot U,~\sep\ nu ik mijn handen ophef naar Uw heilige tempel.

  Ruk mij niet weg met de zondaars,~\sep\ met hen, die kwaad bedrijven,

  Die vriendelijk spreken met hun naaste,~\sep\ maar in hun hart kwade bedoelingen koesteren.

  Handel met hen naar hun daden,~\sep\ en naar de boosheid van hun werken.

  Zet hun het werk van hun handen betaald,~\sep\ vergeld ze hun daden.

  Want ze slaan geen acht op de daden van de Heer en het werk van Zijn handen;~\sep\ Hij richte hen te gronde en heffe hen niet op.

  Gezegend de Heer, want Hij hoorde mijn dringende bede;~\sep\ de Heer, mijn kracht en mijn schild,

  Op Hem vertrouwde mijn hart, en ik ben geholpen;~\sep\ daarom jubelt mijn hart en prijs ik Hem met mijn zang.

  De Heer is een kracht voor Zijn volk,~\sep\ en voor Zijn Gezalfde een heilzame schutse.

  Red Uw volk en zegen Uw erfdeel;~\sep\ weid hen en draag hen voor eeuwig.
\end{halfparskip}

\begin{halfparskip}
  \liturgicalOption{Woensdag.} \psalm{\Ps{67}}

  God zij ons genadig en zegene ons;~\sep\ Hij tone ons Zijn vredig gelaat,~\sep

  Eer aan U, O God \liturgicalhint{(3x)}~---  \liturgicalhint{Eerste vers.}

  Opdat men op aarde Zijn weg lere kennen,~\sep\ onder alle volken Zijn heil.

  Dat de volken U prijzen, O God,~\sep\ dat alle volken U prijzen!

  Laat juichen en jubelen de naties, omdat Gij met rechtvaardigheid de volken regeert,~\sep\ en de naties op aarde bestuurt.

  Dat de volken U prijzen, O God,~\sep\ dat alle volken U prijzen!

  De aarde heeft haar vrucht gegeven;~\sep\ God, onze God, heeft ons gezegend.

  Dat God ons zegene,~\sep\ en dat alle grenzen der aarde Hem vrezen!
\end{halfparskip}

\begin{halfparskip}
  \liturgicalOption{Donderdag.} \psalm{\Ps{54}}

  Red mij, o God, door Uw Naam,~\sep\ en treed in Uw kracht voor mijn rechtszaak op!

  Eer aan U, O God \liturgicalhint{(3x)}~---  \liturgicalhint{Eerste vers.}

  Luister naar mijn bede, O God,~\sep\ hoor naar de woorden van mijn mond!

  Want trotsaards zijn tegen mij opgestaan, en geweldenaars stonden mij naar het leven;~\sep\ zij hielden God niet voor ogen.

  Zie, God komt mij te hulp,~\sep\ de Heer behoudt mijn leven.

  Wend op mijn vijanden de rampen af,~\sep\ vernietig hen omwille van Uw trouw.

  Van harte wil ik U offers brengen;~\sep\ Uw Naam zal ik prijzen, O Heer, want Hij is goed.

  Want Hij heeft mij verlost uit alle nood,~\sep\ en mijn oog zag mijn vijanden beschaamd staan.
\end{halfparskip}

\begin{halfparskip}
  \liturgicalOption{Vrijdag.} \liturgicalhint{Eigen of Ps 95,1--8.}

  Komt, laten wij jubelen voor de Heer,~\sep\ laten wij juichen voor de Rots van ons heil.

  Eer aan U, O God \liturgicalhint{(3x)}~---  \liturgicalhint{Eerste vers.}

  Treden wij voor Zijn aanschijn met lofzangen,~\sep\ juichen wij Hem met liederen toe!

  Want de Heer is een grote God,~\sep\ en een grote Koning boven alle goden.

  Hij houdt in Zijn hand de diepten der aarde,~\sep\ en de toppen der bergen behoren Hem toe.

  Van Hem is de zee; Hij heeft ze geschapen,~\sep\ en het vaste land, door Zijn handen gevormd.

  Komt, laten wij aanbidden en ons neerwerpen,~\sep\ de knieën buigen voor de Heer, die ons heeft gemaakt.

  Want Hij is onze God;~\sep\ wij het volk van Zijn weide en de schapen van Zijn hand.

  Mocht gij toch heden Zijn stem vernemen; ``Wilt niet Uw harten verstokken als bij Meriba, als op de dag van Massa in de woestijn.''
\end{halfparskip}

\begin{halfparskip}
  \liturgicalOption{Zaterdag.} \psalm{\Ps{150}}

  Looft de Heer in Zijn heiligdom,~\sep\ looft Hem in Zijn verheven firmament.~\sep

  Eer aan U, O God \liturgicalhint{(3x)}~---  \liturgicalhint{Eerste vers.}

  Looft Hem om Zijn grootse werken,~\sep\ looft Hem om Zijn hoogste majesteit.

  Looft Hem met bazuingeschal,~\sep\ looft Hem met harp en citer.

  Looft Hem met pauken en reidans,~\sep\ looft Hem met snaarinstrument en schalmei.

  Looft Hem met welluidende cimbalen, looft Hem met rinkelende cimbalen:~\sep\ al wat adem heeft, love de Heer!
\end{halfparskip}

\begin{halfparskip}
  \liturgicalOption{Alle dagen.} Eer aan... Vanaf...~--- Eer aan U, O God \liturgicalhint{(3x)}.
\end{halfparskip}

% % % % % % % % % % % % % % % % % % % % % % % % % % % % % % % % % % % % % % % %

\markedsection{Tesbohta D'Lelya}

\begin{halfparskip}
  \liturgicalOption{Maandag.}

  Wend u tot het gebed van Uw dienaren, Verlosser, en ontvang ons verzoek en beantwoord onze smeekbeden.

  Want U bent de toevlucht en hoop voor de zwakken; mogen wij door Uw hulp de listen van de boze overwinnen.

  Want zie, hij legt te allen tijde strikken voor onze ziel; en wenst, in zijn sluwheid, ons te verontreinigen met zijn slechtheid.

  Zend, mijn Heer, Uw kracht, en laat die zijn netten afsnijden, zodat hij ons niet te pakken krijgt en we niet in de afgrond van de zonde vallen.

  Moge de kracht van Uw genade onze zwakheid versterken; en mogen wij U behagen in onze daden, O Barmhartige.

  Maak ons waardig om op de dag van Uw komst voor U de onophoudelijke lof te zingen, amen.
\end{halfparskip}

\begin{halfparskip}
  \liturgicalOption{Dinsdag.}

  De hoogten en diepten en alles wat daarin is, zijn niet voldoende om Uw Essentie te belijden, O Wezen dat alles vormt.

  Ze zijn te klein om Uw liefde voor ons en de grootheid van Uw genade en Uw ontelbare barmhartigheden uit te drukken.

  Die U aan ons ras heeft toegekend, hoewel we onwaardig waren, U die goed en lief bent en onze natuur hebt aangenomen.

  En ons redde van de dood en naar de hemel opsteeg; en ging zitten hoger dan elke heer en heerser.

  Zie, de koren der engelen aanbidden U; en roepen zonder ophouden, allemaal eenstemmig,

  ``Heilig bent U, Vader, Zoon en Heilige Geest; U zij de glorie van allen, voor eeuwig en altijd. Amen''.
\end{halfparskip}

\begin{halfparskip}
  \liturgicalOption{Woensdag.}

  Moge Uw genade op onze overtredingen rusten, Christus, die de stem van de boeteling bemint.

  Luister naar ons verzoek; genees ons, onze goede Dokter, en verwijder van ons de zweer van onze boosheid.

  Want U kent het lijden van ons ras; verzorg onze wonden met Uw goede medicijn.

  Moge Uw genade onze littekens verzachten; en genees ons, onze Heer, zoals U gewend bent.

  Mogen wij met de dauw van Uw genade onze vlekken reinigen volgens Uw wil.

  Geef ons dat wij unaniem Uw Naam, Vader, Zoon en Heilige Geest belijden.

  Voor altijd en eeuwig, amen en amen.
\end{halfparskip}

\begin{halfparskip}
  \liturgicalOption{Donderdag.}

  Aanvaard, onze Heer, het verzoek van ons allen, die met gebed onze offers aan U brengen.

  Luister, o God, naar de stem van Uw dienaren, en naar de smeekbeden van hen die U verheerlijken.

  Want U bent onze Koning, en in Uw grote Naam hebben wij hoop en vertrouwen.

  Geef ons unanimiteit, zonder verdeeldheid (\translationoptionNl{twijfel}) in het geloof, door te belijden dat door Uw wil alle dingen uit het niets zijn gevormd.

  De Natuur van Uw bestaan kan niet worden begrepen door schepselen, o verborgen Wezen,

  Die in het licht woont, en zoals wie er geen is, en die mensen niet kunnen benaderen.

  Door Uw scheppingswerken, Heer, wordt de grootheid van Uw rijkdom, het werk van Uw handen bekend gemaakt.

  Want U bent Heer en Schepper, die almachtig is en alles bestuurt (\translationoptionNl{voor alles zorgt}).

  Wij verlangen vurig naar de vergeving der overtredingen; schenk het aan ons, zoals U gewend bent te doen.

  En geef ons de kans dat wijzelf het medicijn op onze wonden mogen aanbrengen.

  Wij vragen om genade, Heer van alles, schenk Uw rijkdom aan onze behoefte (\translationoptionNl{armoede}).

  Voor de koppigheid van zij die dwalen door de fout van de list van de vijand

  moge Uw barmhartigheid hun gids zijn; en de weg vrijmaken voor hun geweten,

  zodat zij weten dat het door Uw zorg is dat ons gevangen ras de vrijheid heeft verworven.

  En mogen wij met één enkel, volmaakt en zuiver hart U dienen volgens Uw wil.

  En te allen tijde zorgvuldig doen wat aan Uw Godheid behaagt.

  En eenstemmig de eer geven aan Vader, Zoon en Heilige Geest.

  Die ons heeft gered door de Eersteling die van ons is, en die ons onze ondankbaarheid niet heeft vergolden.

  Aan Hem zij de lof van Zijn aanbidders voor eeuwig en altijd, amen en amen.
\end{halfparskip}

\begin{halfparskip}
  \liturgicalOption{Vrijdag.}

  Eer aan de Goede die in Zijn liefde Zijn glorie openbaarde aan de mensen.

  Hij schiep een stomme natuur uit stof; en versierde het met een ziel begiftigd met schatten.

  Hij plaatste Zijn lof in een nederig (\translationliteralNl{verachtelijk}) lichaam, zodat de hele schepping Zijn glorie zou zingen.

  Komt, jullie, begiftigd met spraak, zingt glorie voor Hem, voordat we de slaap des doods slapen.

  Laten we in de lange nacht de dood gedenken, die onze mond sluit en ons tot stilte brengt.

  De rechtvaardigen die Hem `s~nachts verheerlijkten, zelfs als ze dood zijn, leven.

  En de goddelozen die Zijn grote glorie hebben ontkend, zelfs terwijl ze nog leefden, zijn dood.

  Laten we ons lichaam wakker maken door gebeden; en door alleluias van verborgen macht.

  Dat wij metgezellen mogen worden van de wijze maagden die onze Heer prees.

  En dat wij in de nacht, wanneer Hij de werelden doet beven, waakzaam mogen zijn en de Zoon mogen zien.

  Dat we niet verzinken in (kwade) verlangens, maar dat we Zijn glorie mogen zien op de dag waarop Hij verschijnt;

  en dat wij voor Hem waakzame dienaren mogen zijn in het uur waarop Hij de zonen van Zijn bruidskamer leidt,

  en de goddelozen blijven gemarteld; en plotseling is de deur van genade gesloten.

  Laten we tijdens ons leven ons een beetje inspannen, want na de dood is de dag van de vergelding.

  Het lichaam dat zich vermoeit met gebeden vliegt op de dag van de opstanding door de lucht,

  en ziet onze Heer zonder schaamte; en betreedt met Hem het huis van het koninkrijk.

  De engelen en rechtvaardigen hebben Hem lief, zij die waakzaam zijn geweest en hebben gebeden.

  Gezegend is Hij die ons vaten van Zijn glorie heeft gemaakt en Zijn lof heeft geplaatst in een mond van stof.

  Eer aan Zijn genade die de aardse met de geestelijke wezens associeerde.

  Dat ze alle nachten en te allen tijde ``Heilig'' voor Zijn Naam mogen zingen.

  En laten we hem allemaal voor eeuwig en altijd loven, amen en amen.
\end{halfparskip}

\begin{halfparskip}
  \liturgicalOption{Zaterdag.}

  Gezegend is het Wezen dat ons geschapen en gered heeft door Zijn Christus; en ons bracht tot de kennis van de Drie-eenheid.

  God het Woord, de Eniggeborene van de Vader, moge ik U alleen behagen in onze menselijke natuur die U heeft aangenomen.

  En mag ik nooit treuzelen of ophouden, Christus onze Verlosser, Uw Naam te belijden.
\end{halfparskip}

% % % % % % % % % % % % % % % % % % % % % % % % % % % % % % % % % % % % % % % %

\markedsection{Karozuta}

\begin{halfparskip}
  \dd~Laat ons allen ordelijk staan met berouw en zorg; laat ons bidden en zeggen: Heer, ontferm U over ons.

  \rr~Heer, ontferm U over ons. \liturgicalhint{(Herhaald na elke aanroeping.)}

  Machtige Heer, eeuwig Wezen, die op de hoogste hoogten woont, wij bidden U.

  U die, in Uw grote liefde waarmee U ons liefhad, de vorming van ons ras naar het beeld van Uw glorie hebt geëerd, wij bidden U.

  U die aan de trouwe Abraham goede dingen beloofde aan hen die U liefhebben, en die door de openbaring van Christus aan Uw Kerk bekend werd gemaakt, wij bidden U.

  U die niet wilt dat onze natuur ten onder gaat, maar dat zij zich bekeert van de dwaling van de duisternis naar de kennis van de waarheid, wij bidden U.

  U die alleen de Maker en Vormer der geschapen dingen bent en in het voortreffelijke licht verblijft, wij bidden U,

  Voor de gezondheid van onze heilige vaders, Paus \NN , hoofd van de hele Kerk van Christus, van Patriarch \NN , van onze Catholicos \NN , van onze Metropoliet \NN , van onze Bisschop \NN , en van al hun helpers, wij bidden U,

  Barmhartige God, die met Uw liefde alles bestuurt, wij bidden U,

  Gij die in de hemel wordt geprezen en op aarde wordt aanbeden, wij bidden U,

  Geef ons de overwinning, Christus onze Heer, bij Uw komst, en geef vrede aan Uw Kerk, gered door Uw kostbaar bloed, en ontferm U over ons.

  \liturgicaloption{Slota.} \dd~Laat ons bidden, vrede zij met ons.

  \cc~Medelevende, Barmhartige en Meelijwekkende, wiens deur openstaat voor wie berouw heeft en die altijd zondaars roept om in berouw tot U te naderen; open, onze Heer en onze God, de deur van barmhartigheid voor ons gebed; ontvang ons verzoek en verhoor in Uw barmhartigheid onze smeekbeden uit Uw rijke en overvloedige schatkist, U die goed bent en niet Uw barmhartigheid en gaven aan de behoeftigen en verdrukten weigert, Uw dienaren, die U aanroepen en U smeken in alle seizoenen en tijden, Heer van alles...
\end{halfparskip}

% % % % % % % % % % % % % % % % % % % % % % % % % % % % % % % % % % % % % % % %

\end{document}