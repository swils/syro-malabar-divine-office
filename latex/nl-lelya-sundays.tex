\documentclass[12pt,twoside,a5paper]{article}

\usepackage{multicol}

\usepackage[main=dutch]{babel}
\usepackage{divine-office}

% % % % % % % % % % % % % % % % % % % % % % % % % % % % % % % % % % % % % % % %

% Version: 2024-12-08
\begin{document}

\title{Lelya~--- zondagen}
\author{}
\date{}
\maketitle

% The following prevents footnotes and paracol from interacting in bad ways.
% Not really an idea why...
% See: https://stackoverflow.com/questions/61779911/paracol-and-footnote-placing-in-latex
\footnotelayout{\ }

% % % % % % % % % % % % % % % % % % % % % % % % % % % % % % % % % % % % % % % %

\begin{halfparskip}
  \cc~Eer aan God in den hoge \liturgicalhint{(3x)}. En op aarde vrede en goede hoop aan de mensen, altijd en in eeuwigheid.

  [Amen.]~--- \rr~Zegen, Heer.~--- \liturgicalhint{[Vredekus.]}

  \cc~Onze Vader die in de hemelen zijt,
  \rr~Geheiligd zij Uw Naam. Uw rijk kome, heilig, heilig, heilig zijt Gij. Onze Vader die in de hemelen zijt, de hemel en de aarde zijn gevuld met Uw onmetelijke glorie; de engelen en de mensen roepen U toe: heilig, heilig, heilig zijt Gij.~--- Onze Vader die in de hemelen zijt, geheiligd zij Uw Naam. Uw rijk kome, Uw wil geschiede op aarde zoals in de hemel. Geef ons heden het brood dat we nodig hebben en vergeef ons onze schulden en zonden zoals wij ook vergeven hebben aan onze schuldenaren. En leid ons niet in bekoring, maar verlos ons van de Kwade. Want van U is het koninkrijk en de kracht en de heerlijkheid in eeuwigheid, amen.

  \cc~Eer aan de Vader, de Zoon, en de Heilige Geest.

  \rr~Vanaf het begin en in alle eeuwigheid, amen en amen. Onze Vader die in de hemelen zijt, geheiligd zij Uw naam, Uw rijk kome, heilig, heilig, heilig zijt Gij. Onze Vader die in de hemelen zijt, de hemel en de aarde zijn gevuld met Uw onmetelijke glorie; de engelen en de mensen roepen U toe: heilig, heilig, heilig zijt Gij.

  \dd~Laat ons opstaan om te bidden, vrede zij met ons.

  \cc~Laat ons opstaan, o Heer, in Uw kracht, en bevestigd worden in Uw hoop, mogen we opgetild en gesterkt worden door de hoge arm van Uw macht; en mogen we waardig zijn, met de hulp van Uw goedertierenheid, om te allen tijde lof, eer, belijdenis en aanbidding tot U te verheffen, Heer van alles, Vader...
\end{halfparskip}

% % % % % % % % % % % % % % % % % % % % % % % % % % % % % % % % % % % % % % % %

\markedsection{PSALMEN}

% % % % % % % % % % % % % % % % % % % % % % % % % % % % % % % % % % % % % % % %

\liturgicalhint{Op Zondagen worden 3 hulale gebeden met alleluia (respectievelijk hulale 5-7; of 9-11 + Ps 81; of 12-14 of 16-18 + Ps 129)\footnote{In dit boek vervangen we elke hulala door 1 marmita, de marmiyata gebruikt in ramsa uitsluitend. We volgen het oude systeem waarin de laatste hulala (hier marmita) beëindigd wordt met de psalm die de qalta begeleidt: \Pss{81 \& 129}.}, gevolgd op Zondagen ``voor'' met Ps. 81, op Zondagen ``na'' met Ps. 129, met alleluia na elk vers.}

\markedsubsectionrubricwithhint{1. Zondagen ``voor'':}{Marmita 29 (\Pss{75--77})}

\PSALMtitle{75}{Gods oordeel komt}

\begin{halfparskip}
  Wij loven U, Heer, wij loven U,~\sep\ en prijzen Uw Naam,

  \liturgicalhint{Alleluia, alleluia, alleluia.}~--- \liturgicalhint{Eerste vers:} Wij loven U, Heer, wij loven U,~\sep\ en prijzen Uw Naam,

  verhalen Uw wonderen.~\sep

  ``Op de tijd, die Ik zal bepalen,~\sep\ zal Ik oordelen volgens recht.

  Al wankelt de aarde met al haar bewoners,~\sep\ Ik heb haar zuilen bevestigd.

  De trotsen roep Ik toe: ``Legt Uw hoogmoed af'';~\sep\ en de goddeloze: ``Steekt Uw hoorn niet op'';

  Steekt Uw hoorn niet op tegen de Allerhoogste,~\sep\ uit tegen God geen schaamteloze taal.

  Want noch van het oosten, noch van het westen;~\sep\ noch uit de woestijn, noch van de bergen:

  Maar God is de Rechter:~\sep\ de een drukt Hij neer, de ander verheft Hij.

  Want in de hand van de Heer is een beker,~\sep\ vol kruiden, die schuimt van wijn,

  Hij geeft er uit te drinken; tot de droesem zal men hem ledigen,~\sep\ alle bozen der aarde zullen ervan drinken.''

  Maar ik zal in eeuwigheid juichen,~\sep\ voor de God van Jacob de citer bespelen.

  En alle hoornen der bozen zal ik verbreken,~\sep\ maar de hoornen der gerechtigen worden verheven.
\end{halfparskip}

\PSALMtitle{76}{Overwinningslied}

\begin{halfparskip}
  \psalmsubtitle{a) Gij hebt de vijand verdelgd, O Heer!}

  God is in Juda bekend,~\sep\ groot is Zijn Naam in Israël.

  Zijn tent staat in Salem,~\sep\ en Zijn woning in Sion.

  Daar brak Hij stuk de schichten van de boog,~\sep\ schild en zwaard en wapentuig.

  Gij, Machtige, schitterend van licht, zijt gekomen,~\sep\ van de eeuwige bergen.

  Ontwapend zijn de stoutmoedigen, zij slapen hun doodsslaap;~\sep\ en de handen van al die helden vielen slap.

  Door Uw dreigen, God van Jacob,~\sep\ werden wagens en paarden verlamd.
\end{halfparskip}

\begin{halfparskip}
  \psalmsubtitle{b) Gij zijt ontzagwekkend, O Heer}

  Schrikwekkend zijt Gij, en wie zal U weerstaan,~\sep\ bij het geweld van Uw toorn?

  Vanuit de hemel hebt Gij Uw vonnis doen horen:~\sep\ de aarde ontstelde en zweeg,

  Toen God oprees ten oordeel,~\sep\ om alle verdrukten van het land te redden.

  Want de woede van Edom zal U tot glorie strekken,~\sep\ en die in Emath overbleven, zullen feesten om U.

  Doet geloften aan de Heer Uw God, en komt ze na,~\sep\ dat allen rondom Hem heen aan de Ontzagwekkende een offer brengen.

  Aan Hem, die de trots der vorsten fnuikt,~\sep\ die schrikwekkend is voor de koningen der aarde.
\end{halfparskip}

\PSALMtitle{77}{Gebed in nood}

\begin{halfparskip}
  \psalmsubtitle{a) Groot is mijn droefheid}
  Luid verheft zich mijn stem tot God, mijn stem tot God, opdat Hij mij hore;~\sep

  op de dag van mijn kwelling zoek ik de Heer.

  Onvermoeid strekken bij nacht mijn handen zich uit;~\sep\ mijn ziel is ontroostbaar.

  Denk ik aan God, dan moet ik zuchten,~\sep\ peins ik na, dan verlies ik de moed.

  Gij houdt mijn ogen geopend;~\sep\ ik ben ontsteld en kan niet meer spreken.

  Ik overpeins de vroegere dagen,~\sep\ aan vervlogen jaren denk ik terug.

  Ik overweeg 's nachts in mijn hart,~\sep\ ik peins na, en mijn geest tracht uit te vorsen:
\end{halfparskip}

\begin{halfparskip}
  \psalmsubtitle{b) Heeft God Zijn volk verlaten?}

  ``Zou God dan voor eeuwig verwerpen,~\sep\ en nooit meer genadig zijn?

  Zou Zijn liefde voorgoed zijn verdwenen,~\sep\ Zijn belofte verijdeld voor alle geslachten?

  Heeft God soms vergeten Zich te erbarmen,~\sep\ of in Zijn toorn Zijn ontferming bedwongen?''

  Dan zeg ik: ``Dit is mijn smart,~\sep\ dat de rechterhand van de Allerhoogste is veranderd''.

  Ik denk terug aan de werken van de Heer,~\sep\ ja, ik denk terug aan Uw aloude wonderen.

  Ik overweeg al Uw werken,~\sep\ en overpeins Uw daden.
\end{halfparskip}

\begin{halfparskip}
  \psalmsubtitle{c) Gij hebt ons steeds gered, O God}

  O God, Uw weg is heilig:~\sep\ welke god is groot als onze God?

  Gij zijt de God, die wonderen doet,~\sep\ hebt Uw macht aan de volken doen kennen.

  Door Uw arm hebt Gij Uw volk verlost:~\sep\ de zonen van Jacob en Jozef.

  De wateren zagen U, O God, de wateren zagen U: zij beefden,~\sep\ en de golven werden onstuimig.

  Het zwerk stortte zijn stromen uit, de wolken verhieven hun stem,~\sep\ en Uw flitsen doorkliefden de lucht.

  Uw donder ratelde in de wervelwind, Uw bliksems verlichtten het aardrijk:~\sep\ de aarde sidderde en beefde.

  Uw weg werd gebaand door de zee en Uw pad door de machtige wateren,~\sep\ maar Uw sporen bleven onzichtbaar.

  Als een kudde hebt Gij Uw volk geleid,~\sep\ door de hand van Moses en Aäron.
\end{halfparskip}

\markedsubsectionrubricwithhint{1. Zondagen ``na'':}{Marmita 49 (\Ps{118b})}

\begin{halfparskip}
  \acrosticletter{Lamed} Eeuwig, O Heer, blijft Uw woord,~\sep\ het staat vast als de hemel.

  \liturgicalhint{Alleluia, alleluia, alleluia.}~--- \liturgicalhint{Eerste vers.}

  Van geslacht tot geslacht blijft Uw trouw;~\sep\ Gij hebt de aarde gegrondvest en zij houdt stand.

  Volgens Uw besluiten blijven zij immer bestaan,~\sep\ omdat alles U dienstbaar is.

  Als niet Uw Wet mijn vreugde was,~\sep\ reeds was ik in mijn ellende vergaan.

  Nimmer zal ik Uw bevelen vergeten,~\sep\ want daardoor deedt Gij mij leven.

  Ik ben de uwe: wees mij tot redding,~\sep\ omdat ik uitzag naar Uw bevelen.

  Zondaars wachten mij op om mij te verderven;~\sep\ ik geef op Uw voorschriften acht.

  Begrensd zag ik alle volmaaktheid,~\sep\ maar onbeperkt strekt Uw gebod zich uit.
\end{halfparskip}

\begin{halfparskip}
  \acrosticletter{Men} Hoe lief heb ik Uw Wet, O Heer;~\sep\ de gehele dag overweeg ik haar.

  Uw gebod maakte mij wijzer dan mijn vijanden,~\sep\ want het staat mij eeuwig ter zijde.

  Verstandiger ben ik dan al mijn leraars,~\sep\ omdat ik Uw voorschriften overweeg.

  Ik ben scherper van inzicht dan grijsaards,~\sep\ omdat ik Uw bevelen onderhoud.

  Van alle verkeerde wegen houd ik mijn schreden af,~\sep\ om Uw woorden na te leven.

  Ik wijk niet af van Uw besluiten,~\sep\ want Gij hebt mij onderwezen.

  Hoe zoet voor mijn gehemelte zijn Uw uitspraken,~\sep\ zoeter dan honing voor mijn mond!

  Door Uw bevelen krijg ik inzicht,~\sep\ daarom haat ik iedere weg van ongerechtigheid.
\end{halfparskip}

\begin{halfparskip}
  \acrosticletter{Nun} Uw woord is een lamp voor mijn voeten,~\sep\ en een licht op mijn pad.

  Ik zweer en neem mij voor,~\sep\ Uw rechtvaardige besluiten na te leven.

  Ik ben in de diepste ellende, O Heer;~\sep\ spaar mijn leven naar Uw woord.

  Aanvaard, O Heer, de offers van mijn mond,~\sep\ en leer mij Uw besluiten.

  In voortdurend gevaar is mijn leven,~\sep\ maar Uw Wet vergeet ik niet.

  De bozen hebben mij een strik gelegd,~\sep\ maar van Uw bevelen week ik niet af.

  Uw voorschriften zijn mijn erfdeel voor eeuwig,~\sep\ want ze zijn de vreugde van mijn hart.

  Ik heb er mijn hart op gezet Uw verordeningen na te komen,~\sep\ voortdurend en stipt.
\end{halfparskip}

\begin{halfparskip}
  \acrosticletter{Samech} Ik haat de wankelmoedigen,~\sep\ maar Uw Wet heb ik lief.

  Gij zijt mijn Beschermer en mijn schild,~\sep\ ik vertrouw op Uw woord.

  Gij, bozen, gaat van mij heen:~\sep\ en ik zal de geboden van mijn God onderhouden.

  Sterk mij naar Uw belofte, opdat ik leef;~\sep\ stel mijn hoop niet teleur.

  Sta mij bij, en ik zal behouden zijn,~\sep\ en op Uw verordeningen zal ik acht slaan altijd.

  Die Uw verordeningen verlaten, verwerpt Gij,~\sep\ want hun gedachten zijn bedrieglijk.

  Als afval beschouwt Gij alle bozen op aarde,~\sep\ daarom heb ik Uw voorschriften lief.

  Van vreze beeft mijn vlees voor U,~\sep\ en ik heb ontzag voor Uw besluiten.
\end{halfparskip}

\begin{halfparskip}
  \acrosticletter{Ain} Recht en gerechtigheid heb ik beoefend;~\sep\ lever mij niet over aan mijn verdrukkers.

  Sta borg voor het welzijn van Uw dienaar,~\sep\ opdat de trotsen mij niet verdrukken.

  Mijn ogen kwijnen van verlangen naar Uw hulp,~\sep\ en naar Uw rechtvaardige uitspraak.

  Handel met Uw dienaar naar Uw goedheid,~\sep\ en leer mij Uw verordeningen.

  Ik ben Uw dienstknecht, onderricht mij,~\sep\ opdat ik Uw voorschriften kenne.

  Voor de Heer is het tijd om te handelen:~\sep\ zij hebben Uw Wet verkracht.

  Daarom heb ik Uw geboden lief,~\sep\ meer dan goud en het edelst metaal.

  Daarom koos ik al Uw bevelen tot mijn deel;~\sep\ van iedere dwaalweg heb ik een afschuw.
\end{halfparskip}

\begin{halfparskip}
  \acrosticletter{Phe} Wonderbaar zijn Uw voorschriften,~\sep\ daarom onderhoudt ze mijn ziel.

  De openbaring van Uw woorden geeft licht,~\sep\ de onbedrevenen onderricht zij.

  Ik open smachtend mijn mond,~\sep\ want naar Uw geboden verlang ik.

  Wend U tot mij en wees mij genadig,~\sep\ zoals Gij gewoon zijt voor wie Uw Naam beminnen.

  Richt mijn schreden naar Uw uitspraak,~\sep\ en laat mij geen onrecht beheersen.

  Verlos mij van de verdrukking der mensen;~\sep\ en Uw bevelen zal ik onderhouden.

  Toon Uw dienaar Uw vredig gelaat,~\sep\ en leer mij Uw verordeningen.

  Stromen van tranen ontwelden mijn ogen,~\sep\ omdat men Uw Wet niet onderhield.
\end{halfparskip}

\begin{halfparskip}
  \acrosticletter{Sade} Rechtvaardig zijt Gij, O Heer,~\sep\ en Uw oordeel is billijk.

  Uw voorschriften gaaft Gij in gerechtigheid,~\sep\ en met grote kracht.

  Mijn ijver verteert mij,~\sep\ omdat mijn weerstrevers Uw woorden vergeten.

  Terdege beproefd is Uw uitspraak,~\sep\ en Uw dienaar heeft ze lief.

  Al ben ik dan klein en veracht,~\sep\ Uw bevelen vergeet ik niet.

  Uw gerechtigheid is gerechtigheid voor eeuwig,~\sep\ en onveranderlijk is Uw Wet.

  Al troffen mij kommer en kwelling,~\sep\ Uw geboden zijn mijn geneugte.

  De rechtvaardigheid van Uw voorschriften is eeuwig,~\sep\ onderricht mij, opdat ik mag leven.
\end{halfparskip}

\begin{halfparskip}
  \acrosticletter{Coph} Ik roep uit heel mijn hart: verhoor mij, Heer;~\sep\ Uw verordeningen leef ik na.

  Ik roep tot U; behoud mij,~\sep\ en ik zal Uw voorschriften onderhouden.

  Ik kom bij de dageraad en roep om Uw hulp,~\sep\ ik vertrouw op Uw woorden.

  Vóór de nachtwaken zijn mijn ogen geopend,~\sep\ om Uw uitspraak te overwegen.

  Hoor mijn smeken, Heer, naar Uw barmhartigheid,~\sep\ en schenk mij leven naar Uw besluit.

  Die mij boosaardig vervolgen, naderen mij,~\sep\ ver zijn zij verwijderd van Uw Wet.

  Gij zijt nabij, O Heer,~\sep\ en waarachtig zijn al Uw geboden.

  Reeds vroeger heb ik uit Uw voorschriften begrepen,~\sep\ dat Gij ze gegeven hebt voor eeuwig.
\end{halfparskip}

\begin{halfparskip}
  \acrosticletter{Res} Zie mijn ellende en bevrijd mij,~\sep\ want Uw Wet heb ik niet vergeten.

  Verdedig mijn zaak, en verlos mij;~\sep\ naar Uw uitspraak schenk mij het leven.

  Ver blijft het heil van de zondaars,~\sep\ want zij storen zich niet aan Uw verordeningen.

  Groot is Uw erbarming, O Heer;~\sep\ schenk mij het leven naar Uw besluiten.

  Velen vervolgen en kwellen mij:~\sep\ van Uw voorschriften wijk ik niet af.

  Ik zag overtreders en het walgde mij,~\sep\ want Uw uitspraak volgden zij niet.

  Zie, Heer, ik heb Uw bevelen lief,~\sep\ spaar mijn leven naar Uw barmhartigheid.

  Geheel Uw woord ligt vervat in standvastigheid;~\sep\ en ieder besluit van Uw gerechtigheid is eeuwig.
\end{halfparskip}

\begin{halfparskip}
  \acrosticletter{Sin} Vorsten vervolgen mij zonder reden,~\sep\ maar mijn hart eerbiedigt Uw woorden.

  Ik verheug mij over Uw uitspraken,~\sep\ als iemand, die rijke buit heeft gemaakt.

  Ongerechtigheid haat en verfoei ik,~\sep\ Uw Wet heb ik lief.

  Zevenmaal daags breng ik U lof,~\sep\ om Uw rechtvaardige oordelen.

  Veel vrede is weggelegd voor die Uw Wet beminnen:~\sep\ geen struikelblok ligt ooit op hun weg.

  Van U, O Heer, verwacht ik hulp,~\sep\ en ik onderhoud Uw geboden.

  Ik leef Uw voorschriften na,~\sep\ en heb ze van harte lief.

  Ik onderhoud Uw bevelen en geboden,~\sep\ want heel mijn weg ligt open voor U.
\end{halfparskip}

\begin{halfparskip}
  \acrosticletter{Tau} Mijn geroep kome tot U, O Heer,~\sep\ geef mij inzicht naar Uw woord.

  Mijn bede dringe door tot U;~\sep\ red mij naar Uw uitspraak.

  Van mijn lippen moge een lofzang vloeien,~\sep\ als Gij mij Uw verordeningen  zult hebben geleerd.

  Mijn tong bezinge Uw uitspraak,~\sep\ want rechtvaardig zijn al Uw geboden.

  Uw hand zij gereed mij te helpen,~\sep\ want Uw bevelen heb ik verkoren.

  Van U verwacht ik redding, O Heer,~\sep\ en Uw Wet is mijn geneugte.

  Leve mijn ziel om U te prijzen,~\sep\ en dat Uw besluiten mij helpen!

  Als een verloren schaap dool ik rond; zoek toch Uw dienaar op,~\sep\ want Uw geboden heb ik niet vergeten.
\end{halfparskip}

\begin{halfparskip}
  \dd~Eer aan...~--- Alleluia, alleluia; Eer aan U, God, alleluia; Eer aan U, God, alleluia; Heer, ontferm U over ons. Laat ons bidden; vrede zij met ons.

  \cc~Versterk, onze Heer en onze God, in Uw mededogen onze zwakheid; bemoedig (\translationoptionNl{troost}) en help in Uw genade de armzaligheid van onze ziel; wek de slaperigheid van onze geest; verlicht (\translationoptionNl{neem weg}) de last van onze ledematen; was en reinig het vuil van onze schulden en zonden; verlicht de duisternis van ons intellect; strek een helpende hand uit en geef ons kracht, zodat we daardoor mogen opstaan om U onophoudelijk te belijden en te verheerlijken, alle dagen van ons leven, Heer van alles...
\end{halfparskip}

\markedsubsectionrubricwithhint{2. Zondagen ``voor'':}{Marmita 30 (\Ps{78})}

\PSALMtitle{78}{De zonden der vaderen, een les voor de kinderen}

\begin{halfparskip}
  \psalmsubtitle{a) Het verleden, een spiegel voor het heden}

  Luister, mijn volk, naar mijn leer,~\sep\ neig uw oren naar de woorden van mijn mond.

  \liturgicalhint{Alleluia, alleluia, alleluia.}~--- \liturgicalhint{Eerste vers:} Luister, mijn volk, naar mijn leer...

  Ik ga mijn mond voor wijze spreuken openen,~\sep\ diepzinnige lessen uit de oudheid verkondigen.

  Wat wij hoorden en vernamen,~\sep\ en wat onze vaderen ons hebben verhaald,

  Zullen wij niet voor hun kinderen verbergen,~\sep\ maar aan het nageslacht verhalen:

  De lof van de Heer en Zijn macht,~\sep\ en de wonderen, die Hij wrochtte.

  Want Hij maakte het tot voorschrift voor Jacob,~\sep\ en stelde het tot wet voor Israël:

  Wat Hij onze vaderen heeft geboden,~\sep\ zouden zij leren aan hun kinderen,

  Opdat het komend geslacht het zou weten, de kinderen, die worden geboren,~\sep\ opdat ook deze op zouden staan en het aan hun kinderen verhalen,

  Zodat ze hun vertrouwen op God blijven stellen en de werken van God niet zouden vergeten,~\sep\ maar zich houden aan Zijn geboden.

  Zij moesten niet als hun vaderen worden:~\sep\ een opstandig en weerspannig geslacht;

  Een geslacht, dat niet deugdzaam is van hart,~\sep\ en trouweloos van geest jegens God.
\end{halfparskip}

\begin{halfparskip}
  \psalmsubtitle{b) Ontrouw bij de Rode Zee en in de woestijn}

  Efraïms zonen, de boogschutters,~\sep\ sloegen op de vlucht op de dag van de strijd.

  Het verbond met God onderhielden zij niet,~\sep\ zij weigerden volgens Zijn Wet te wandelen,

  Zijn werken waren zij niet indachtig,~\sep\ noch Zijn wonderen, aan hen betoond.

  Voor het oog van hun vaderen wrochtte Hij wonderen,~\sep\ in het land van Egypte, in de vlakte van Tanis.

  Hij scheidde de zee, voerde hen er doorheen,~\sep\ en vast deed Hij de wateren staan als een dam.

  Hij leidde hen door een wolk bij dag,~\sep\ heel de nacht door een lichtend vuur.

  Hij spleet rotsen in de woestijn,~\sep\ en laafde hen overvloedig als aan stromen.

  Uit de rots deed Hij beken ontspringen,~\sep\ en liet als rivieren het water vloeien.
\end{halfparskip}


\begin{halfparskip}
  \psalmsubtitle{c) Gemor en straffen in de woestijn}

  Maar zij bleven tegen Hem zondigen,~\sep\ bleven de Allerhoogste tarten in de woestijn:

  In hun hart stelden zij God op de proef,~\sep\ door spijs te eisen naar hun begeerte;

  Zij verhieven hun stem tegen God;~\sep\ zij zeiden: ``Zou God in de woestijn wel een dis kunnen bereiden?

  Wel sloeg Hij op de rots, en de wateren vloeiden, en beken ontsprongen;~\sep\ maar zou Hij ook brood kunnen geven, of vlees verschaffen aan Zijn volk?''

  Toen dan de Heer dit vernam, ontstak Hij in gramschap,~\sep\ ontvlamde er een vuur tegen Jacob, en toorn bruiste op tegen Israël.

  Omdat ze niet in God geloofden,~\sep\ en niet op Zijn hulp vertrouwden.

  Toch gaf Hij de wolken daarboven bevel,~\sep\ en ontsloot Hij de poorten van de hemel;

  Hij regende manna als spijs op hen neer,~\sep\ en gaf hun brood uit de hemel.

  Het brood der sterken at de mens,~\sep\ Hij zond hun spijs tot verzadigings toe.

  Hij joeg uit de hemel de oostenwind op~\sep , en voerde de zuidenwind aan door Zijn kracht.

  En Hij regende vlees als stof op hen neer,~\sep\ en als zand van de zee gevleugelde vogels.

  Ze vielen neer in hun legerplaats,~\sep\ rondom de tenten, waarin zij woonden;

  Zij aten, en werden ten volle verzadigd;~\sep\ Hij had aan hun begeerte voldaan.

  Maar nog was hun lust niet bevredigd, nog was de spijs in hun mond,~\sep\ of daar barstte Gods toorn tegen hen los.

  Hij richtte een slachting aan onder hun groten,~\sep\ en sloeg de jonge mannen van Israël neer.
\end{halfparskip}

\begin{halfparskip}
  \psalmsubtitle{d) Schijnbekering van Israël. God blijft barmhartig}

  Maar toch bleven zij zondigen,~\sep\ en aan Zijn wonderen geloofden zij niet.

  Snel deed Hij hun dagen vergaan,~\sep\ en hun jaren door een plotseling verderf.

  Sloeg Hij hen neer, dan zochten zij Hem,~\sep\ dan keerden zij terug en vroegen om God;

  Zij herinnerden zich dat God hun Rots,~\sep\ de allerhoogste God hun Redder was.

  Maar met hun mond bedrogen zij Hem,~\sep\ en belogen Hem met hun tong.

  Hun hart was niet oprecht jegens Hem,~\sep\ zij bleven niet trouw aan Zijn verbond.

  Maar Hij vergaf meedogend hun schuld, en verdelgde hen niet;~\sep\ zo dikwijls weerhield Hij Zijn toorn, stortte niet heel Zijn gramschap uit.

  Hij dacht er aan dat ze maar vlees zijn,~\sep\ een zucht, die vervliegt en niet weerkeert.
\end{halfparskip}

\begin{halfparskip}
  \psalmsubtitle{e) De verlossing uit Egypte had hun niets geleerd}

  Hoe dikwijls tergden zij Hem in de wildernis,~\sep\ bedroefden zij Hem in de woestijn.

  En telkens opnieuw beproefden zij God,~\sep\ en tergden de Heilige van Israël.

  Zij dachten niet meer aan Zijn hand,~\sep\ noch aan de dag, waarop Hij hen uit de greep van de vijand verloste,

  Toen Hij in Egypte Zijn tekenen deed,~\sep\ en Zijn wonderen in de vlakte van Tanis:

  Hij veranderde in bloed hun stromen,~\sep\ en hun beken, opdat ze niet konden drinken.

  Hij zond muggen op hen af, die hen verslonden,~\sep\ en kikvorsen, die op hen aanvielen;

  Hun gewas gaf Hij prijs aan de kever,~\sep\ aan de sprinkhaan de vrucht van hun werk.

  Met hagel sloeg Hij hun wijngaarden,~\sep\ en hun vijgenbomen met rijp;

  Ook hun vee gaf Hij prijs aan de hagel,~\sep\ en aan de bliksem hun kudden.

  Hij zond op hen af de gloed van Zijn toorn, verbolgenheid en gramschap en kwelling:~\sep\ een menigte van onheilstichters.

  Hij liet Zijn toorn de vrije loop, Hij redde hen niet van de dood,~\sep\ en hun vee gaf Hij prijs aan de pest.

  En Hij sloeg alle eerstgeborenen in Egypte,~\sep\ hun eerste kinderen in de tenten van Cham.

  Hij voerde Zijn Volk als schapen weg,~\sep\ en leidde hen in de woestijn als een kudde.

  Hij leidde hen veilig, ze hadden niets te vrezen:~\sep\ de zee bedekte hun vijanden.

  Hij voerde hen naar Zijn heilig land,~\sep\ naar de bergen, door Zijn rechterhand veroverd.

  Hij joeg de volken voor hen uit, wees hen door het lot als erfdeel aan,~\sep\ en deed Israëls stammen in hun tenten wonen.
\end{halfparskip}

\begin{halfparskip}
  \psalmsubtitle{f) Afgoderij in Canaän: Gods straffen}

  Maar zij beproefden en tergden God, de Allerhoogste,~\sep\ en Zijn geboden onderhielden zij niet.

  Trouweloos vielen zij af zoals hun vaderen;~\sep\ zij faalden als een onbetrouwbare boog.

  Zij zetten Hem aan tot toorn door hun offerhoogten,~\sep\ en wekten door hun beelden Zijn na-ijver op.

  God hoorde het en ontstak in toorn,~\sep\ en met geweld verwierp Hij Israël.

  Hij verliet Zijn tent in Silo, de tent,~\sep\ waarin Hij onder de mensen woonde.

  Hij liet Zijn kracht in gevangenschap gaan,~\sep\ Zijn luister in de macht van de vijand.

  Zijn volk gaf Hij prijs aan het zwaard,~\sep\ Hij was tegen Zijn erfdeel verbitterd.

  Het vuur verteerde hun jonge mannen,~\sep\ en hun maagden werden niet verloofd.

  Hun priesters vielen door het zwaard,~\sep\ en hun weduwen hieven geen rouwklacht aan.
\end{halfparskip}

\begin{halfparskip}
  \psalmsubtitle{g) Juda om Ephraïms ontrouw uitverkoren}

  Toen ontwaakte de Heer als uit een slaap,~\sep\ als een krijger, door wijn bevangen.

  Hij sloeg Zijn vijanden van achteren;~\sep\ eeuwige smaad deed Hij hun aan.

  Hij verwierp de tent van Jozef,~\sep\ de stam van Efraïm verkoos Hij niet meer,

  Maar Hij koos de stam van Juda uit,~\sep\ de Sionsberg, die Hij beminde.

  Hoog als de hemel trok Hij Zijn heiligdom op,~\sep\ vast als de aarde, die Hij voor immer gegrondvest heeft.

  Hij koos David, Zijn dienaar, uit,~\sep\ en nam hem van de schaapskooien weg.

  Hij riep hem van achter de zogende schapen,~\sep\ om Jacob te weiden, Zijn volk en Israël, Zijn erfdeel.

  Hij heeft hen geweid in onschuld van het hart,~\sep\ en met bekwame hand hen geleid.
\end{halfparskip}

\markedsubsectionrubricwithhint{2. Zondagen ``na'':}{Marmita 50 (\Pss{119--124})}

\PSALMtitle{119}{Tegen valse tongen}

\begin{halfparskip}
  \psalmsubtitle{a) Verlos mij, Heer, van boze tongen}

  Ik riep tot de Heer in mijn kwelling,~\sep\ en Hij heeft mij verhoord.

  \liturgicalhint{Alleluia, alleluia, alleluia.}~--- \liturgicalhint{Eerste vers:} Ik riep tot de Heer in mijn kwelling...

  Bevrijd mij, Heer, van boze lippen,~\sep\ en van de valse tong.
\end{halfparskip}

\begin{halfparskip}
  \psalmsubtitle{b) Straf van de lasteraar}

  Wat zal Hij u geven of wat daaraan toevoegen,~\sep\ O valse tong?

  Scherpe pijlen van een oorlogsheld,~\sep\ en kolen van de brem.
\end{halfparskip}

\begin{halfparskip}
  \psalmsubtitle{c) Zo lang moet ik wonen onder bozen!}

  Wee mij, dat Ik moet leven in Mosoch,~\sep\ wonen in de tenten van Cedar!

  Te lang heb ik gewoond,~\sep\ met hen, die de vrede haten.

  Spreek ik van vrede,~\sep\ zij dringen tot strijd.
\end{halfparskip}

\PSALMtitle{120}{Onder Gods hoede}

\begin{halfparskip}
  \psalmsubtitle{a) Gij, O God, zijt mijn hulp}

  Ik hef mijn ogen naar de bergen;~\sep\ vanwaar zal er hulp voor mij dagen?

  Mijn hulp komt van de Heer,~\sep\ die hemel en aarde gemaakt heeft.
\end{halfparskip}

\begin{halfparskip}
  \psalmsubtitle{b) Gij, O God, beschermt mij}

  Hij zal uw voet niet laten struikelen,~\sep\ Hij zal niet sluimeren, die u bewaakt.

  Neen, Hij zal niet sluimeren of slapen,~\sep\ die Israël bewaakt.

  De Heer behoedt u,~\sep\ de Heer is uw bescherming aan uw rechterzijde.

  Bij dag zal de zon u niet deren,~\sep\ en de maan niet in de nacht.

  De Heer zal u behoeden voor alle kwaad,~\sep\ Hij zal uw leven bewaken.

  De Heer zal waken over uw gaan en komen,~\sep\ en nu en tot in eeuwigheid.
\end{halfparskip}

\PSALMtitle{121}{Begroeting van Jeruzalem}

\begin{halfparskip}
  \psalmsubtitle{a) Blijde pelgrimstocht naar Jeruzalem}

  Ik was verheugd, omdat men mij zei:~\sep\ ``Wij trekken op naar het huis van de Heer.''

  Reeds staan onze voeten,~\sep\ binnen uw poorten, Jeruzalem,

  Jeruzalem, als een stad gebouwd,~\sep\ geheel aaneengesloten.

  Daarheen trekken de stammen op, de stammen van de Heer,~\sep\ om, volgens een wet in Israël, de Naam van de Heer te prijzen.

  Daar staan de zetels voor het gericht,~\sep\ de zetels van Davids huis.
\end{halfparskip}

\begin{halfparskip}
  \psalmsubtitle{b) Zegenwens voor de stad}

  Vraagt wat Jeruzalem tot vrede strekt;~\sep\ mogen veilig zijn, die u beminnen!

  Vrede zij binnen uw muren,~\sep\ veiligheid in uw paleizen.

  Omwille van mijn broeders en vrienden,~\sep\ wil ik spreken: Vrede zij u!

  Omwille van het huis van de Heer, onze God,~\sep\ zal ik voor u bidden om heil.
\end{halfparskip}

\PSALMtitle{122}{Ontferm U, Heer!}

\begin{halfparskip}
  \psalmsubtitle{a) Heer, sta mij bij!}

  Ik hef mijn ogen op tot U,~\sep\ die in de hemel woont.

  Zie, als de ogen van dienaars,~\sep\ op de handen van hun meesters,

  Als de ogen der dienstmaagd,~\sep\ op de handen van haar meesteres,

  Zo zijn onze ogen gericht op de Heer, onze God,~\sep\ tot Hij zich over ons ontfermt.
\end{halfparskip}

\begin{halfparskip}
  \psalmsubtitle{b) Heer, bevrijd mij van de vijanden!}

  Wees ons genadig, O Heer, wees ons genadig,~\sep\ want wij zijn oververzadigd van smaad;

  Oververzadigd is onze ziel,~\sep\ van het spotten der rijken, van het smaden der trotsen.
\end{halfparskip}

\PSALMtitle{123}{Opgangslied}

\begin{halfparskip}
  \psalmsubtitle{a) Heer, Gij hebt ons gered}

  Was de Heer niet met ons geweest,~\sep\ zo moge Israël nu zeggen;~\sep\ Was de Heer niet met ons geweest,

  toen de mensen tegen ons opstonden,~\sep\ dan hadden zij ons levend verslonden.

  Toen hun toorn tegen ons ontbrandde,~\sep\ zou het water ons hebben verzwolgen,

  Zou een stortvloed over ons zijn heengegaan,~\sep\ zouden over ons zijn heengegaan de bruisende wateren.
\end{halfparskip}

\begin{halfparskip}
  \psalmsubtitle{b) Dank, Heer, voor de redding!}

  Geprezen zij de Heer, die ons niet prijsgaf,~\sep\ ten prooi aan hun tanden.

  Als een vogel zijn wij ontsnapt,~\sep\ aan de strik van de jagers:

  De strik is verbroken,~\sep\ en wij zijn bevrijd !

  Onze hulp is in de Naam van de Heer,~\sep\ die hemel en aarde gemaakt heeft.
\end{halfparskip}

\PSALMtitle{124}{Opgangslied. Vertrouwen op God}

\begin{halfparskip}
  Wie op de Heer vertrouwen, zijn als de berg Sion,~\sep\ die niet wankelt, maar staan blijft voor eeuwig.

  Bergen omgeven Jeruzalem,~\sep\ zo omgeeft de Heer Zijn volk, en nu en in eeuwigheid.

  Neen, de scepter der bozen blijft niet drukken,~\sep\ op het erfdeel der rechtvaardigen,

  Opdat de rechtvaardigen niet strekken,~\sep\ hun handen naar ongerechtigheid.

  Doe wel, Heer, aan de goeden,~\sep\ en aan de oprechten van hart.

  Maar die hun slinkse wegen gaan moge de Heer met de bozen verdrijven;~\sep\ vrede over Israël!
\end{halfparskip}

\begin{halfparskip}
  \dd~Eer aan...~--- Alleluia, alleluia; Eer aan U, God, alleluia; Eer aan U, God, alleluia; Heer, ontferm U over ons. Laat ons bidden; vrede zij met ons.

  \cc~Moge de verborgen kracht, o mijn Heer, van Uw Godheid, de wonderbaarlijke hulp van Uw Heerschap, en de grote hulp van Uw barmhartigheid de broosheid van onze zwakke natuur versterken zodat wij U te allen tijde glorie, eer, belijdenis en aanbidding kunnen verheffen, Heer van alles...
\end{halfparskip}

\markedsubsectionrubricwithhint{3. Zondagen ``voor'':}{Marmita 31 (\Pss{79--81})}

\PSALMtitle{79}{Klaaglied over Sions verwoesting}

\begin{halfparskip}
  \psalmsubtitle{a) Jeruzalem, O Heer, is verwoest!}

  De heidenen, O God, zijn Uw erfdeel binnengedrongen,~\sep\ zij hebben Uw heilige tempel ontwijd,

  \liturgicalhint{Alleluia, alleluia, alleluia.}~--- \liturgicalhint{Eerste vers:} De heidenen, O God...

  Jeruzalem tot een puinhoop gemaakt.

  De lichamen van Uw dienaars wierpen ze als aas voor de vogels van de hemel,~\sep\ het vlees van Uw heiligen voor de dieren van het veld.

  Hun bloed vergoten zij als water rondom Jeruzalem,~\sep\ en niemand was er, die hen begroef.

  Wij zijn tot smaad geworden voor onze buren,~\sep\ tot spot en hoon voor hen, die ons omringen.
\end{halfparskip}

\begin{halfparskip}
  \psalmsubtitle{b) Straf de vijand, Heer; spaar Uw volk!}

  Hoe lang nog, Heer; zult Gij dan eeuwig toornen?~\sep\ zal Uw ijverzucht branden als vuur?

  Stort Uw gramschap uit over de heidenen, die U niet erkennen,~\sep\ en over de koninkrijken, die Uw Naam niet aanroepen.

  Want zij hebben Jacob verslonden,~\sep\ en zijn woonplaats verwoest.

  Ach, reken ons de schulden van onze vaderen niet aan; Uw ontferming trede ons snel tegemoet,~\sep\ want wij zijn uiterst ellendig.
\end{halfparskip}

\begin{halfparskip}
  \psalmsubtitle{c) Help ons, Heer!}

  Help ons, God, onze Redder, om de eer van Uw Naam,~\sep\ en om Uw Naam bevrijd ons, en vergeef onze zonden.

  Waarom moeten de heidenen zeggen:~\sep\ ``Waar is hun God?''

  Laat de heidenen voor onze ogen zien,~\sep\ de wraak voor het vergoten bloed van Uw dienaars.

  Het zuchten der gevangenen dringe tot U door;~\sep\ verlos door de kracht van Uw arm die ten dode zijn gewijd.

  Werp zevenvoudig in de schoot van onze buren,~\sep\ de smaad, die zij U aandeden, O Heer.

  Maar wij, Uw volk en de schapen van Uw weide,~\sep\ wij zullen U eeuwig verheerlijken, en verkondigen Uw lof van geslacht tot geslacht.
\end{halfparskip}

\PSALMtitle{80}{Herstel uw volk, Heer.}

\begin{halfparskip}
  \psalmsubtitle{a) Heer, bevrijd Uw volk!}

  Herder van Israël, luister aandachtig,~\sep\ Gij, die Jozef leidt als een kudde;

  Die zetelt op cherubs, laat stralen Uw licht,~\sep\ voor Efraïm, Benjamin en Manasse.

  Wek op Uw macht,~\sep\ en kom ons verlossen.

  O God, richt ons weer op,~\sep\ en toon ons Uw vredig gelaat, opdat wij worden gered.

  O God der heerscharen, hoe lang nog zult Gij toornen,~\sep\ daar Uw volk toch bidt?

  Gij hebt het gevoed met tranenbrood,~\sep\ en overvloedig met tranen gelaafd.

  Gij hebt ons gemaakt tot twistappel voor onze buren,~\sep\ en onze vijanden spotten met ons.

  O God der heerscharen, richt ons weer op,~\sep\ en toon ons Uw vredig gelaat, opdat wij worden gered.
\end{halfparskip}

\begin{halfparskip}
  \psalmsubtitle{b) Herstel, O Heer, Uw verwoeste wijnstok!}

  Een wijnstok hebt Gij uit Egypte weggenomen,~\sep\ volken uitgeworpen om hem te planten.

  Gij hebt de grond voor hem bereid:~\sep\ hij schoot wortel en begroeide het land;

  Zijn schaduw bedekte de bergen,~\sep\ zijn ranken de ceders van God.

  Hij strekte zijn twijgen uit tot de zee,~\sep\ tot aan de stroom zijn loten.

  Waarom hebt Gij zijn omheining vernield,~\sep\ zodat elke voorbijganger hem plukt,

  De ever van het woud hem omwroet,~\sep\ en de dieren van het veld hem kaal vreten?

  God der heerscharen, keer terug,~\sep\ blik neer uit de hemel en zie, en bezoek deze wijnstok;

  Bescherm hem, die Uw rechterhand heeft geplant,~\sep\ de twijg, die Gij voor U hebt krachtig gemaakt.

  Mogen zij, die in vuur hem verbrandden en vernielden,~\sep\ door Uw dreigende blikken vergaan.

  Uw hand moge rusten op de man aan Uw rechterzijde,~\sep\ op het mensenkind, dat Gij sterk deedt worden voor U.

  En nimmermeer zullen wij U verlaten;~\sep\ Gij zult ons in leven houden, en Uw Naam zullen wij prijzen.

  Heer, God der heerscharen, richt ons weer op,~\sep\ en toon ons Uw vredig gelaat, opdat wij worden gered.
\end{halfparskip}

\PSALMtitle{81}{Het feest der bazuinen}

\begin{halfparskip}
  \psalmsubtitle{a) Wil waardig Gods feesttij vieren!}

  Jubelt voor God, onze Helper,~\sep\ juicht de God van Jacob toe.

  Doet het psalter weerklinken en slaat de pauken,~\sep\ tokkelt de welluidende citer en de lier.

  Steekt de bazuin bij de Nieuwe Maan,~\sep\ bij volle maan op onze plechtige feestdag.

  Want het is een voorschrift voor Israël,~\sep\ een bevel van Jacobs God.

  Dit stelde Hij Jozef tot wet,~\sep\ toen hij optrok tegen het land van Egypte.
\end{halfparskip}

\begin{halfparskip}
  \psalmsubtitle{b) Mijn volk, blijf Mij trouw. Ik zal u zegenen}

  Ik hoorde een taal, die ik niet kende: ``Ik heb de last van zijn schouders genomen;~\sep\ zijn handen lieten de draagkorf los.

  Ge riept in uw nood en Ik heb u bevrijd, uit de donderwolk heb Ik u antwoord gegeven,~\sep\ u beproefd bij het water van Meriba.

  Hoor, mijn volk, en Ik zal u vermanen;~\sep\ ach, Israël, wil toch luisteren naar Mij:

  Er zij geen andere god onder u,~\sep\ geen vreemde god zult gij aanbidden.

  Ik ben de Heer, Uw God, die u leidde uit het land van Egypte:~\sep\ open uw mond, en Ik zal hem vullen.

  Maar Mijn volk luisterde niet naar Mijn stem,~\sep\ en Israël was Mij niet onderdanig.

  Daarom gaf Ik hen prijs aan de verstoktheid van hun harten:~\sep\ dat zij nu handelen volgens hun grillen!

  Mocht toch Mijn volk naar Mij luisteren,~\sep\ Israël Mijn wegen bewandelen!

  Dan zou Ik aanstonds hun haters bedwingen,~\sep\ Mijn hand zou Ik keren tegen hun weerstrevers.

  De vijanden van de Heer zouden zich buigen voor Hem,~\sep\ en hun lot zou vaststaan voor eeuwig.

  Maar hem zou ik spijzen met bloem van tarwe,~\sep\ met honing uit de rots hem verzadigen.
\end{halfparskip}

\markedsubsectionrubricwithhint{3. Zondagen ``na'':}{Marmita 51 (\Pss{125--129})}

\PSALMtitle{125}{Opgangslied. Richt ons weer op}

\begin{halfparskip}
  \psalmsubtitle{a) Vreugde na de ballingschap}

  Toen de Heer de gevangenen van Sion deed wederkeren,~\sep\ was het of wij droomden.

  \liturgicalhint{Alleluia, alleluia, alleluia.}~--- \liturgicalhint{Eerste vers:} Toen de Heer de gevangenen van Sion deed wederkeren...

  Toen werd onze mond met lachen gevuld,~\sep\ en onze tong met gejubel.

  Toen zei men onder de volken:~\sep\ ``De Heer heeft grote dingen aan hen gedaan.''

  Ja, grote dingen heeft de Heer aan ons gedaan;~\sep\ wij zijn nu van vreugde vervuld.
\end{halfparskip}

\begin{halfparskip}
  \psalmsubtitle{b) Voltooi de Verlossing, O Heer!}

  Wend ons lot ten beste, O Heer,~\sep\ als de bergstromen in het Zuiderland.

  Wie met tranen zaaien,~\sep\ zullen met gejubel maaien.

  Wenend trekken zij op,~\sep\ met het zaad, dat gezaaid moet worden;

  Juichend keren zij weer,~\sep\ met hun garven beladen.
\end{halfparskip}

\PSALMtitle{126}{Opgangslied. Alle zegen komt van God}

\begin{halfparskip}
  \psalmsubtitle{a) Niets gedijt zonder God}

  Als de Heer het huis niet bouwt,~\sep\ werken vergeefs die er aan bouwen.

  Als de Heer de stad niet behoedt,~\sep\ waakt de wachter tevergeefs.

  Het heeft voor u geen zin vóór het daglicht op te staan,~\sep\ en op te blijven tot diep in de nacht,

  Voor u, die het brood van harde arbeid eet,~\sep\ want Zijn geliefden schenkt Hij gaven in hun slaap.
\end{halfparskip}

\begin{halfparskip}
  \psalmsubtitle{b) Kinderen zijn een zegen van God}

  Waarlijk, kinderen zijn een gave van de Heer,~\sep\ een loon is de vrucht van de schoot.

  Als pijlen in de hand van de strijder,~\sep\ zo zijn de zonen, verwekt in de jeugd.

  Gelukkig de man, die er zijn koker mee vulde:~\sep\ dan wordt men niet te schande, als men met vijanden twist in de poort.
\end{halfparskip}

\PSALMtitle{127}{Opgangslied. Het huiselijk geluk van de vrome}

\begin{halfparskip}
  Gelukkig, gij allen, die de Heer vreest,~\sep\ en die Zijn wegen bewandelt.

  Want van de arbeid van uw handen zult gij eten;~\sep\ gelukkig zult gij zijn en het zal u welgaan.

  Uw huisvrouw zal zijn als een wijnstok vol vruchten,~\sep\ binnen uw huis,

  Uw kinderen zullen zijn als loten van olijven,~\sep\ rondom uw dis.

  Ja, zo wordt de man gezegend,~\sep\ die de Heer vreest.

  Moge de Heer u uit Sion zegenen,~\sep\ opdat gij al uw levensdagen Jeruzalems welvaart moogt aanschouwen;

  En nog de kinderen van uw kinderen moogt zien:~\sep\ vrede over Israël!
\end{halfparskip}

\PSALMtitle{128}{Opgangslied. God was ons tot heil}

\begin{halfparskip}
  \psalmsubtitle{a) Heer, Gij hebt ons gered in het verleden}

  Hevig bestreden zij mij vanaf mijn jeugd,~\sep\ zo moge Israël nu zeggen,~\sep

  hevig bestreden zij mij vanaf mijn jeugd,~\sep

  maar zij overwonnen mij niet.

  Ploegers hebben mijn rug beploegd,~\sep\ hun lange voren getrokken.

  Maar de rechtvaardige Heer,~\sep\ sneed de strengen der bozen stuk.
\end{halfparskip}

\begin{halfparskip}
  \psalmsubtitle{b) Sta ons bij, Heer, in de toekomst!}

  Dat zij vol schande terugwijken,~\sep\ allen, die Sion haten!

  Mogen ze worden als gras op de daken,~\sep\ dat, nog voor men uitrukt, verdort.

  Waarmee de maaier zijn hand niet vult,~\sep\ noch de schovenbinder zijn schoot.

  En die voorbijgaan, zeggen niet: ``De zegen van de Heer zij over u!''~\sep\ ``Wij zegenen u in de Naam van de Heer!''
\end{halfparskip}

\PSALMtitle{129}{Opgangslied. De profundis}

\begin{halfparskip}
  \psalmsubtitle{a) Heer, wees mij, zondaar, genadig}

  Uit de diepten roep ik tot U, O Heer,~\sep\ Heer, luister naar mijn klagen!

  Laat Uw oren zich neigen,~\sep\ naar de stem van mijn smeken.

  Als Gij de zonden blijft gedenken, Heer,~\sep\ Heer, wie zal dan staande blijven?

  Maar bij U is vergeving van zonden,~\sep\ opdat men vol eerbied U diene.
\end{halfparskip}

\begin{halfparskip}
  \psalmsubtitle{b) Op U, O Heer, vertrouw ik}

  Ik stel mijn hoop op de Heer,~\sep\ mijn ziel hoopt op Zijn woord;

  Verlangend zie ik uit naar de Heer,~\sep\ meer dan wachters naar de dageraad.

  Meer dan wachters naar de dageraad,~\sep\ ziet Israël verlangend uit naar de Heer.

  Want bij de Heer is barmhartigheid,~\sep\ en bij Hem overvloedige verlossing.

  Hij zal Israël verlossen,~\sep\ van al Zijn ongerechtigheden.
\end{halfparskip}

\PSALMtitle{130}{Opgangslied. Berusting}

\begin{halfparskip}
  Heer, mijn hart is niet hoogmoedig,~\sep\ en mijn ogen zien niet op in trots.

  Geen grootse dingen jaag Ik na,~\sep\ noch zaken, voor mij te verheven.

  Ja, rustig en vredig,~\sep\ stemde ik mijn ziel;

  Gelijk een kindje op de schoot van zijn moeder,~\sep\ gelijk een kindje, zo is in mij mijn ziel gestemd.

  Hoop, Israël, op de Heer,~\sep\ nu en tot in eeuwigheid.
\end{halfparskip}

\markedsubsectionrubric{\sep~\underline{Alle dagen:}}

\begin{halfparskip}
  \dd~Eer aan...~--- Alleluia, alleluia; laat ons bidden; vrede zij met ons.

  \cc~Mijn Heer, aan Uw Natuur, die verborgen, onbegrijpelijk en onbeperkt is voor het denken en begrijpen van alle schepselen, is glorie passend en het geluid van lof gepast; aanbidding is geschikt en verschuldigd door allen die U in de hemel en op aarde hebt geschapen en gevormd, Heer van alles...
\end{halfparskip}

% % % % % % % % % % % % % % % % % % % % % % % % % % % % % % % % % % % % % % % %

\markedsection{Qalta \markedsectionhint{(Eigen tekst, zie Hudra.)}}

% % % % % % % % % % % % % % % % % % % % % % % % % % % % % % % % % % % % % % % %

\markedsection{Qanona \markedsectionhint{(Psalm van de Qalta.)}}

\markedsubsectionrubricwithhint{Zondagen van Aankondiging ``voor'':}{\Ps{40,1--15}.}

\begin{halfparskip}
  Ik heb gehoopt, gehoopt op de Heer, alleluia, alleluia, alleluia.

  Ik heb gehoopt, gehoopt op de Heer, en Hij boog zich naar mij,~\sep\ en verhoorde mijn smeken.

  Hij trok mij op uit de kuil van de dood, uit modder en slijk; op de rots heeft Hij mijn voeten geplaatst,~\sep\ mijn schreden heeft Hij gesteund.

  Hij legde een nieuw lied in mijn mond,~\sep\ een lofzang voor onze God.

  Velen zullen het zien, vervuld van ontzag,~\sep\ en zullen op de Heer vertrouwen.

  Gelukkig de man, die op de Heer zijn hoop heeft gesteld,~\sep\ geen dienaars van afgoden volgt, noch hen, die tot verzinsels zich wenden.

  Talrijk, O Heer, mijn God, hebt Gij Uw wonderwerken gemaakt,~\sep\ en in Uw raadsbesluiten over ons is niemand U gelijk.

  Wilde ik ze verhalen en verkondigen:~\sep\ ze zijn te talrijk om te worden geteld.

  Slacht-- noch spijsoffer hebt Gij gewild,~\sep\ maar Gij hebt mij de oren geopend.

  Brand-- noch zoenoffer hebt Gij voor de zonde geëist:~\sep\ toen heb ik gezegd: ``Zie, ik kom; in de boekrol staat over mij geschreven:

  Het is mijn geneugte, mijn God, Uw wil te volbrengen,~\sep\ en diep in mijn hart staat Uw Wet gegrift.''

  De gerechtigheid heb ik verkondigd in de volle vergadering;~\sep\ neen, Heer, Gij weet het: mijn lippen hield ik niet gesloten.

  Uw gerechtigheid verborg ik niet in mijn hart;~\sep\ Uw trouw en Uw hulp heb ik verkondigd;

  Uw goedheid hield ik niet geheim,~\sep\ noch in de volle vergadering Uw trouw.

  Gij dan, O Heer, onthoud mij Uw erbarming niet;~\sep\ laat Uw genade en trouw mij immer behoeden.
\end{halfparskip}

\markedsubsectionrubricwithhint{Zondagen van Aankondiging ``na'':}{\Ps{48,1--11}.}

\begin{halfparskip}
  Groot is de Heer en hoogst lofwaardig, alleluia, alleluia, alleluia.

  Groot is de Heer en hoogst lofwaardig,~\sep\ in de stad van onze God.

  Zijn heilige, zijn roemvolle heuvel,~\sep\ is de vreugde van heel het aardrijk.

  De berg Sion, het uiterste noorden,~\sep\ is de stad van de grote Koning.

  God in haar burchten,~\sep\ toonde zich een veilige schutse.

  Want ziet, de koningen sloten een verbond,~\sep\ en rukten gezamenlijk op.

  Eén blik! Ze staan verbijsterd,~\sep\ ze sidderen en stuiven uiteen.

  Daar grijpt ontzetting hen aan,~\sep\ smart als van een barende,

  Zoals wanneer de oostenwind,~\sep\ de schepen van Tharsis verbrijzelt.

  Gelijk wij het hoorden, zo hebben wij het nu gezien,~\sep\ in de stad van de Heer der heerscharen,

  In de stad van onze God:~\sep\ God houdt haar eeuwig in stand.

  Wij gedenken Uw barmhartigheid, O God,~\sep\ binnen Uw tempel.

  Zoals Uw Naam, O God, zo ook Uw lof:~\sep\ hij reikt tot de grenzen der aarde.

  Vol gerechtigheid is Uw rechterhand,~\sep\ dat de berg Sion zich verblijde,

  Dat Juda's steden jubelen,~\sep\ om Uw gerichten!
\end{halfparskip}

\markedsubsectionrubricwithhint{Zondagen gedurende het jaar ``voor'':}{\Ps{86}.}

\begin{halfparskip}
  Neig Uw oor, O Heer; verhoor mij, alleluia, alleluia, alleluia.

  Neig Uw oor, O Heer; verhoor mij,~\sep\ want ik ben ellendig en arm.

  Bescherm mij, want ik ben U toegewijd;~\sep\ red Uw dienaar, die op U hoopt.

  Mijn God zijt Gij; wees mij genadig, O Heer,~\sep\ want almaar door roep ik tot U.

  Verblijd de ziel van Uw dienaar,~\sep\ want tot U, O Heer, verhef ik mijn ziel.

  Want Gij, O Heer, zijt goed en genadig,~\sep\ vol erbarming voor al wie U aanroept.

  Luister, Heer, naar mijn bede,~\sep\ en geef acht op de stem van mijn smeken.

  Op de dag van mijn kwelling riep ik tot U,~\sep\ omdat Gij mij verhoren zult.

  Onder de goden, O Heer, is er geen als Gij,~\sep\ en geen werk is gelijk aan het Uwe.

  Alle volken, door U geschapen, zullen komen, en U aanbidden, O Heer,~\sep\ en verheerlijken Uw Naam.

  Want Gij zijt groot en Gij doet wonderwerken:~\sep\ Gij zijt God, en Gij alleen.

  Toon mij Uw weg, O Heer, opdat ik wandele in Uw waarheid,~\sep\ richt mijn hart op de vrees voor Uw Naam.

  Ik zal U prijzen, Heer, mijn God, uit heel mijn hart:~\sep\ en eeuwig Uw Naam verheerlijken.

  Want Uw erbarming voor mij was groot,~\sep\ en uit de diepten van het dodenrijk hebt Gij mij opgehaald.

  Trotsen, O God, zijn tegen mij opgestaan, een bende geweldenaars staat mij naar het leven,~\sep\ zij houden U niet voor ogen.

  Maar Gij, O Heer, zijt een barmhartige en liefdevolle God,~\sep\ lankmoedig, rijk aan ontferming en trouw.

  Blik op mij neer en wees mij genadig;~\sep\ schenk aan Uw dienaar Uw kracht, en red de zoon van Uw dienstmaagd.

  Geef mij een teken van Uw gunst, opdat die mij haten, Heer, vol schaamte zien,~\sep\ dat Gij, O Heer, mij hulp en troost hebt geschonken.
\end{halfparskip}

\markedsubsectionrubricwithhint{Zondagen gedurende het jaar ``na'':}{\Ps{91}.}

\begin{halfparskip}
  Gij, die onder de hoede van de Allerhoogste leeft, alleluia, alleluia, alleluia.

  Gij, die onder de hoede van de Allerhoogste leeft,~\sep\ in de schaduw van de Almachtige woont,

  Zeg tot de Heer: ``Mijn toevlucht en burcht,~\sep\ mijn God, op wie ik vertrouw.''

  Want Hij zal u bevrijden uit de strik van de jagers,~\sep\ van de verderfelijke pest.

  Hij zal u met Zijn vleugels beschermen, en onder Zijn wieken zult gij vluchten;~\sep\ een beukelaar en schild is Zijn trouw.

  Geen verschrikking bij nacht zult gij vrezen,~\sep\ geen pijl, die voortsnort bij dag.

  Geen pest, die rondwaart in het duister,~\sep\ geen verwoestend verderf op de middag.

  Al vallen er duizend aan uw zijde, en tienduizend aan uw rechterhand,~\sep\ het zal tot u niet genaken.

  Ja, met eigen ogen zult gij het zien,~\sep\ en de vergelding der bozen aanschouwen.

  Want de Heer is uw toevlucht,~\sep\ de Allerhoogste hebt gij gemaakt tot uw schutse.

  Geen ramp zal u genaken,~\sep\ en geen plaag zal naderen tot uw tent.

  Want Hij gaf over u een bevel aan Zijn engelen,~\sep\ u te behoeden op al uw wegen.

  Zij zullen u op de handen dragen,~\sep\ opdat gij aan geen steen uw voet zoudt stoten.

  Op slang en adder zult gij trappen,~\sep\ leeuw en draak zult gij vertreden,

  Omdat hij Mij aanhangt, zal Ik hem redden;~\sep\ omdat hij Mijn Naam kent, hem beschermen.

  Hij zal Mij aanroepen, en Ik zal hem verhoren, in de nood zal Ik met hem zijn,~\sep\ Ik zal hem redden en hem eren.

  Ik zal hem verzadigen met lengte van dagen,~\sep\ en hem tonen Mijn heil.
\end{halfparskip}

\markedsubsectionrubricwithhint{Zondagen van de Kerkwijding ``voor'':}{\Ps{45}.}

\begin{halfparskip}
  Een heerlijk lied welt op uit mijn hart, alleluia, alleluia, alleluia.

  Een heerlijk lied welt op uit mijn hart: de Koning wijd ik mijn zang;~\sep\ mijn tong is de stift van een vaardige schrijver.

  Gij zijt de schoonste onder de kinderen der mensen; bevalligheid ligt op uw lippen:~\sep\ daarom heeft God u voor eeuwig gezegend.

  Gord uw zwaard om de heup, gij, machtige held,~\sep\ uw sieraad en luister.

  Ruk zegerijk uit voor waarheid en recht;~\sep\ uw rechterhand lere u roemrijke daden.

  Uw pijlen zijn scherp; volkeren worden aan u onderworpen;~\sep\ aan de vijanden van de Koning ontzinkt de moed.

  In de eeuwen der eeuwen staat Uw troon, O God,~\sep\ een scepter van recht is de scepter van Uw rijk.

  Gij hebt de gerechtigheid lief en haat de boosheid; daarom heeft God, uw God, u gezalfd,~\sep\ met de olie der vreugde boven uw genoten.

  Van mirre en aloë en cassia geuren uw gewaden;~\sep\ uit ivoren paleizen klinkt u blij het harpgeluid tegen.

  Koningsdochters treden u tegemoet,~\sep\ de Koningin staat aan uw rechterhand, met goud uit Ofir getooid.

  Hoor, dochter, en zie, en neig uw oor,~\sep\ en vergeet uw volk en het huis van uw vader.

  Dan zal aan de Koning uw schoonheid behagen;~\sep\ Hij is uw Heer, breng Hem uw hulde.

  Dan komt met geschenken het volk van Tyrus,~\sep\ de voornamen onder het volk dingen om uw gunst.

  In volle luister treedt de dochter van de Koning binnen;~\sep\ met goud doorweven is haar gewaad.

  In een kleurige mantel wordt zij voor de Koning geleid,~\sep\ in haar gevolg worden maagden, haar gezellinnen, tot u gevoerd;

  Zij worden voorgeleid in blijde jubel,~\sep\ en treden het paleis van de Koning binnen.

  In de plaats van uw vaderen komen uw zonen:~\sep\ gij zult hen aanstellen tot vorsten over heel de wereld.

  Ik zal uw naam doen gedenken bij alle geslachten;~\sep\ daarom zullen de volken u prijzen in de eeuwen der eeuwen.
\end{halfparskip}

\markedsubsectionrubricwithhint{Zondagen van de Kerkwijding ``na'':}{\Ps{84}.}

\begin{halfparskip}
  Hoe liefelijk is Uw woonstede, Heer der legerscharen, alleluia, alleluia, alleluia.

  Hoe liefelijk is Uw woonstede, Heer der legerscharen;~\sep\ mijn ziel verlangt, ziet smachtend uit naar de voorhoven van de Heer.

  Mijn hart en mijn lichaam,~\sep\ juichen voor de levende God.

  Ook de mus vindt een woning,~\sep\ en de zwaluw een nest, waar ze haar jongen in neerlegt.

  Uw altaren, o Heer der legerscharen,~\sep\ mijn Koning en mijn God!

  Gelukkig zij, die wonen in Uw huis, O Heer;~\sep\ eeuwig loven zij U.

  Gelukkig de man, die hulp krijgt van U,~\sep\ als hij het plan heeft op bedevaart te gaan:

  Trekken zij door een dorre vallei, dan maken zij haar tot bron,~\sep\ en de vroege regen bekleedt haar met zegeningen.

  Al gaande zal hun kracht vermeerderen:~\sep\ de God der goden zullen zij in Sion zien.

  Heer der legerscharen, hoor naar mijn bede,~\sep\ ach, luister toch, o God van Jacob.

  Zie toe, o God, ons schild,~\sep\ en zie op het gelaat van Uw gezalfde.

  Waarlijk, één dag in Uw voorhoven is beter,~\sep\ dan duizend andere.

  Liever blijf ik staan op de drempel van het huis van mijn God,~\sep\ dan te toeven in de tenten der bozen.

  Want een zon en een schild is God de Heer;~\sep\ de Heer schenkt genade en glorie.

  Hij weigert het goede niet,~\sep\ aan die in onschuld wandelen.

  Heer der legerscharen,~\sep\ gelukkig de mens, die op U vertrouwt.
\end{halfparskip}

\markedsubsectionrubricwithhint{\sep~\underline{Alle zondagen ``voor''}:}{\Ps{120,1--8}.}

\begin{halfparskip}
  Ik hef mijn ogen naar de bergen;~\sep\ vanwaar zal er hulp voor mij dagen?

  Mijn hulp komt van de Heer,~\sep\ die hemel en aarde gemaakt heeft.

  Hij zal uw voet niet laten struikelen,~\sep\ Hij zal niet sluimeren, die u bewaakt.

  Neen, Hij zal niet sluimeren of slapen,~\sep\ die Israël bewaakt.

  De Heer behoedt u,~\sep\ de Heer is uw bescherming aan uw rechterzijde.

  Bij dag zal de zon u niet deren,~\sep\ en de maan niet in de nacht.

  De Heer zal u behoeden voor alle kwaad,~\sep\ Hij zal uw leven bewaken.

  De Heer zal waken over uw gaan en komen,~\sep\ en nu en tot in eeuwigheid.

  \liturgicalhint{\Ps{88,10v}.} Of doet Gij voor doden nog wonderen,~\sep\ of zullen gestorvenen, herrijzend, U loven?

  Of wordt Uw goedheid in het graf verkondigd,~\sep\ Uw trouw in het dodenrijk?

  \liturgicalhint{\Ps{137,7v}.} Gij strekt Uw hand naar mijn woedende vijanden uit;~\sep\ Uw rechterhand is mijn redding.

  Wat de Heer begon, voltooit Hij voor mij; Heer, Uw goedheid blijft eeuwig;~\sep\ trek u niet terug van het werk van Uw handen.

  Eer aan...~--- \liturgicalhint{3x alleluia.} Vanaf het begin...~--- \liturgicalhint{Herhaal de eerste twee petgame (halfverzen):}

  \liturgicaloption{Aankondiging ``voor'':} Ik heb gehoopt, gehoopt op de Heer, en Hij boog zich naar mij,~\sep\ en verhoorde mijn smeken, alleluia.

  \liturgicaloption{Kerkwijding ``voor'':} Een heerlijk lied welt op uit mijn hart: de Koning wijd ik mijn zang, alleluia.

  \liturgicaloption{Tijdens het jaar ``voor'':} Ik hef mijn ogen naar de bergen;~\sep\ vanwaar zal er hulp voor mij dagen, alleluia.
\end{halfparskip}

\markedsubsectionrubricwithhint{\sep~\underline{Alle zondagen ``na''}:}{\Ps{122,1--4}.}

\begin{halfparskip}
  Ik hef mijn ogen op tot U,~\sep\ die in de hemel woont.

  Zie, als de ogen van dienaars,~\sep\ op de handen van hun meesters,

  Als de ogen der dienstmaagd,~\sep\ op de handen van haar meesteres,

  Zo zijn onze ogen gericht op de Heer, onze God,~\sep\ tot Hij zich over ons ontfermt.

  Wees ons genadig, O Heer, wees ons genadig.

  \liturgicalhint{\Ps{102,26v}.} Zij zullen vergaan, maar Gij zult blijven,~\sep\ en als een kleed zal alles verslijten.

  Als kleding verwisselt Gij ze, en ze worden verwisseld,~\sep\ maar Gij blijft dezelfde en Uw jaren nemen geen einde.

  \liturgicalhint{\Ps{137,7v} (versie Pius XII):} Gij houdt mij in leven, en strekt Uw hand naar mijn woedende vijanden uit;~\sep\ Uw rechterhand is mijn redding.~\sep\ Wat de Heer begon, voltooit Hij voor mij; Heer, Uw goedheid blijft eeuwig;~\sep\ trek u niet terug van het werk van Uw handen.

  \liturgicalhint{(Peshita):} U zult Uw hand uitstrekken en mij redden.~\sep\ Laat Uw rechterhand, o Heer, op mij rusten.

  O Heer, Uw barmhartigheden zijn voor altijd;~\sep\ laat het werk van Uw handen niet in de steek.

  Eer aan...~--- \liturgicalhint{3x alleluia.} Vanaf het begin...~--- \liturgicalhint{Herhaal de eerste twee petgame (halfverzen):}

  \liturgicaloption{Aankondiging ``na'':} Groot is de Heer en hoogst lofwaardig,~\sep\ in de stad van onze God, alleluia.

  \liturgicaloption{Kerkwijding ``na'':} Hoe liefelijk is Uw woonstede, Heer der legerscharen;~\sep\ mijn ziel verlangt, ziet smachtend uit naar de voorhoven van de Heer, alleluia.

  \liturgicaloption{Tijdens het jaar ``na'':} Gij, die onder de hoede van de Allerhoogste leeft,~\sep\ in de schaduw van de Almachtige woont, alleluia.
\end{halfparskip}

\markedsubsectionrubric{Slota voor de mawtba.}

\begin{halfparskip}
  \dd~Laat ons bidden; vrede zij met ons.

  \cc~Moge ons gebed, Heer, U behagen, moge ons verzoek voor U komen, en mogen vanuit de grote schatkamer van Uw barmhartigheid onze smeekbeden voor onze behoeften worden beantwoord, altijd en in eeuwigheid, Heer van alles...
\end{halfparskip}

% % % % % % % % % % % % % % % % % % % % % % % % % % % % % % % % % % % % % % % %

\markedsection{Mawtba \markedsectionhint{(Eigen tekst, zie Hudra, gevolgd door Qala d-udrane\footnote{Als de Hudra meerdere qale (``slawata'') aangeeft en er onvoldoende tijd is, zeg dan de eerste 2 strofen van beide + de laatste strofen van Maria, Kruis, Heiligen, Patroonheilige en de laatste van de overledenen (volledig), en dan Eer aan.}.)}}

\begin{halfparskip}
  \dd~Laat ons bidden, vrede zij met ons.

  \liturgicaloption{Slota: Advent, Seizoenen van Epifanie en de Verrijzenis en alle feesten van onze Heer.}

  \cc~Voor Uw wonderbaarlijke en onuitsprekelijke heilsbestel, Heer, dat in barmhartigheid en mededogen werd vervolmaakt, voltooid en vervuld voor de vernieuwing en redding van onze zwakke natuur, in de Eersteling die van ons was, brengen wij lof, eer, belijdenis en aanbidding, te allen tijde, Heer van alles...

  \liturgicaloption{Grote vasten, Zomer, Elia tot de feestdag van het Heilig Kruis en ferias:}

  Heb medelijden met ons, O Medelevende, in Uw genade wend u tot ons, O Barmhartige, en wend Uw blik en zorg niet van ons af, Heer; want in U is onze hoop en vertrouwen altijd en in eeuwigheid, Heer van alles...

  \liturgicaloption{Seizoen der apostelen:}

  Moge het gebed der heilige apostelen, mijn Heer, het verzoek der ware predikers, het smeken en bidden der beroemde (\translationoptionNl{triomfantelijke}) atleten, de verkondigers van de waarheid, de boodschappers van gerechtigheid en de zaaiers van vrede in de schepping, voortdurend bij ons zijn, in alle tijden en seizoenen, Heer van alles...

  \liturgicaloption{Seizoen van het Kruis:}

  Laat Uw vrede in alle gebieden wonen, verhef Uw Kerk door Uw Kruis, en bewaar haar kinderen in Uw goedheid, zodat zij in haar altijd lof, eer, belijdenis en aanbidding tot U mogen verheffen, Heer van alles...

  \liturgicaloption{Seizoen van de Kerkwijding:}

  Maak, mijn Heer, in Uw mededogen de fundamenten van Uw Kerk stevig, in Uw genade versterk haar barrières, en laat Uw heerlijkheid wonen in de tempel die apart is gezet ter ere van Uw dienst, in alle dagen van de wereld, Heer van alles...
\end{halfparskip}

% % % % % % % % % % % % % % % % % % % % % % % % % % % % % % % % % % % % % % % %

\markedsection{Qanona \markedsectionhint{(eigen\footnote{Zeg de eerste 2 petgame, dan het eerste deel van de qanona (refrein); zeg de hele psalm; herhaal het eerste deel van de qanona; Eer aan...; zeg het tweede deel; herhaal de eerste 2 petgame; eindig met het derde deel van de qanona.})}~--- Tesbohta \markedsectionhint{(eigen)}}

% % % % % % % % % % % % % % % % % % % % % % % % % % % % % % % % % % % % % % % %

\markedsection{Karozuta \markedsectionhint{(eigen, of:)}}

\begin{halfparskip}
  \dd~Laat ons allen ordelijk staan met vreugde en vrolijkheid; laat ons bidden en zeggen: Heer, ontferm U over ons.~--- \rr~Heer, ontferm U over ons. \liturgicalhint{(Wordt herhaald na elke aanroeping.)}

  Machtige Heer, eeuwig Wezen, die op de hoogste hoogten woont, wij bidden U.

  U die, in Uw grote liefde waarmee U ons liefhebt, de vorming van ons ras naar het beeld van Uw glorie hebt geëerd, wij bidden U.

  U die aan de trouwe Abraham goede dingen beloofde aan hen die U liefhebben, en die door de openbaring van Christus aan Uw Kerk bekend werden gemaakt, wij bidden U.

  U die niet wilt dat onze natuur ten onder gaat, maar dat zij zich bekeert van de dwaling van de duisternis naar de kennis van de waarheid, wij bidden U.

  U die alleen de Maker en Vormer der geschapen dingen bent en in het voortreffelijke licht verblijft, wij bidden U,

  Voor de gezondheid van onze heilige vaders, Paus~\NN , hoofd van de hele Kerk van Christus, van Patriarch~\NN , van onze Catholicos~\NN , van onze Metropoliet~\NN , van onze Bisschop~\NN , en van al hun helpers, wij bidden U,

  Barmhartige God, die met Uw liefde alles bestuurt, wij bidden U,

  U die in de hemel wordt geprezen en op aarde wordt aanbeden, wij bidden U,

  Geef ons de overwinning, Christus onze Heer, bij Uw komst, en geef vrede aan Uw Kerk, gered door Uw kostbaar bloed, en ontferm U over ons.

  \cc~Van U, die vol genade en mededogen bent, van de grote rijkdom van Uw liefdevolle goedheid en de overvloedige schat van Uw mededogen, vragen wij hulp, kracht, verlossing, behoud en genezing voor de pijnen van onze lichamen en zielen. Schenk ons dit in Uw genade en barmhartigheid, zoals U gewend bent, te allen tijde, Heer van alles...

  \cc~Gezegend, aanbiddelijk, hoog, verheven en onbegrijpelijk is de eeuwige genade van Uw glorieuze Drie-eenheid, die medelijden heeft met de zondaars, o onze goede Hoop en Toevlucht vol barmhartigheid, die overtredingen en zonden vergeeft, Heer van alles...
\end{halfparskip}

% % % % % % % % % % % % % % % % % % % % % % % % % % % % % % % % % % % % % % % %

\markedsection{Madrasa \markedsectionhint{(Eigen tekst, zie Hudra, geen madrasa van Pasen tot Elia.)}}

\begin{halfparskip}
  \cc~Mijn Heer, aan Uw Natuur, die verborgen, onbegrijpelijk en onbeperkt is voor het denken en begrijpen van alle schepselen, komt glorie toe en is het geluid van lof gepast; aanbidding is geschikt en verschuldigd door allen die U in de hemel en op aarde hebt geschapen en gevormd, Heer van alles...
\end{halfparskip}

% % % % % % % % % % % % % % % % % % % % % % % % % % % % % % % % % % % % % % % %

\markedsection{Suyake\footnote{Vermeld in Breviarium voor de eerste 2 zondagen van Subara en in de Vasten, maar verondersteld op alle zondagen, behalve tussen Pasen en Elia. In de praktijk worden ze geannuleerd of ingekort (Brev., p.~25 spreekt van de eerste en tweede ``psalm'').}}

\begin{halfparskip}
  \liturgicalOption{Eerste suyaka:} \liturgicalhint{Zondagen ``voor'': hulala~12 (vasten: hulala~19). Zondagen ``na'': hulala~19 (vasten: hulala~8).}

  \cc~Wij belijden, aanbidden en verheerlijken U die verborgen bent in Uw Wezen, geheim in Uw Godheid, en onuitsprekelijk in Uw glorie, grote Koning der glorie, Wezen dat van eeuwigheid is, te allen tijde, Heer van alles...

  \liturgicalOption{Tweede suyaka:} \liturgicalhint{Zondagen ``voor'': hulala~13 (vasten: hulala~20). Zondagen ``na'': hulala 20 (vasten: hulala 9).}

  \cc~Mogen de klanken van onze alleluias en de melodieën van onze liederen, onze Heer en onze God, U behagen; en aanvaard van ons in Uw liefderijke goedheid de redelijke vruchten van onze lippen die wij met lof aan Uw glorieuze Drie-eenheid aanbieden, dag en nacht, Heer van alles...
\end{halfparskip}

% % % % % % % % % % % % % % % % % % % % % % % % % % % % % % % % % % % % % % % %

\newpage
\title{Qale d'sahra}
\inlinemaketitle

\markedsubsectionrubricwithhint{1. Zondagen ``voor''\footnote{Hulale~14 (marmiyata~37--39) voor zondagen ``voor'' is vervangen in deze verkorte uitgave door marmiyata~35--37 daar marmiyata~38 en 39 al gebruikt worden in ramsa op Donderdag. Hulala~21~(58--60) voor zondagen ``na'' is behouden.}:}{Marmita~35 (\Ps{89}).}

\PSALMtitle{89}{Jubel over Gods belofte aan David}

\begin{halfparskip}
  \psalmsubtitle{a) Heer, Gij hebt een belofte aan David gedaan}

  De gunsten van de Heer wil ik eeuwig bezingen,~\sep\ door alle geslachten heen zal mijn mond Uw trouw verkondigen,

  \liturgicalhint{Alleluia, alleluia, alleluia.}~--- \liturgicalhint{Eerste vers:} De gunsten van de Heer wil ik eeuwig bezingen...

  Want Gij hebt gezegd; ``De genade staat eeuwig vast'';~\sep\ in de hemel hebt Gij Uw trouw gegrondvest.

  ``Een verbond ging Ik aan met Mijn uitverkorene;~\sep\ aan David, Mijn dienaar, zwoer Ik een eed:

  Ik zal uw nazaat voor eeuwig bevestigen,~\sep\ en uw troon in stand houden door alle geslachten.''
\end{halfparskip}

\begin{halfparskip}
  \psalmsubtitle{b) Gij zijt almachtig, Heer, en wilt Uw belofte vervullen}

  De hemelen loven Uw wonderen, O Heer,~\sep\ en Uw trouw in de kring der heiligen.

  Want wie in de wolken zal de Heer evenaren,~\sep\ wie onder Gods zonen is gelijk aan de Heer?

  Ontzagwekkend is God in de gemeenschap der heiligen,~\sep\ groot en vreeswekkend boven allen om Hem heen.

  O Heer, God der legerscharen, wie is U gelijk?~\sep\ Machtig zijt Gij, O Heer, en van Uw trouw omgeven,

  Gij beheerst de trotse zee,~\sep\ Gij breekt haar onstuimige golven.

  Gij hebt Rahab doorstoken en vertreden,~\sep\ met Uw machtige arm Uw vijanden verstrooid.

  Van U zijn de hemelen, van U is de aarde;~\sep\ Gij grondvestte de wereld met wat ze bevat.

  Noord en zuid hebt Gij geschapen:~\sep\ Thabor en Hermon juichen om Uw Naam.

  Gij hebt een krachtige arm,~\sep\ sterk is Uw hand, Uw rechter opgeheven.

  Gerechtigheid en recht zijn de grondslag van Uw troon;~\sep\ genade en trouw gaan voor U uit.

  Gelukkig het volk, dat weet te jubelen;~\sep\ het wandelt, o Heer, in het licht van Uw aanschijn,

  Het verheugt zich voor immer om Uw Naam,~\sep\ en roemt in Uw gerechtigheid,

  Want Gij zijt de glans van hun kracht,~\sep\ door Uw gunst verheft zich onze hoorn.

  Ja, van de Heer komt ons schild,~\sep\ van Israëls Heilige komt onze koning.
\end{halfparskip}

\begin{halfparskip}
  \psalmsubtitle{c) Dit hebt Gij David beloofd:}

  Eens hebt Gij in een visioen tot Uw heiligen het woord gesproken:~\sep\ ``Ik heb een held de kroon opgezet, een uit het volk verkoren en verheven.

  Ik heb David gevonden, Mijn dienaar,~\sep\ hem gezalfd met Mijn heilige olie,

  Opdat Mijn hand voor immer met hem zij,~\sep\ en Mijn arm hem sterke.

  Geen vijand zal hem misleiden,~\sep\ geen booswicht hem verdrukken;

  Maar zijn weerstrevers zal Ik voor zijn aanschijn verpletteren,~\sep\ en die hem haten, doorsteken.

  Mijn trouw en Mijn genade zullen met hem zijn,~\sep\ en in Mijn Naam zal zijn hoorn zich verheffen.

  Zijn hand zal Ik uitstrekken over de zee,~\sep\ en over de stromen zijn rechter.

  Hij zal tot Mij roepen: ``Mijn Vader zijt Gij,~\sep\ mijn God en de Rots van mijn heil.''

  En Ik zal hem maken tot eerstgeborene,~\sep\ tot de hoogste onder de koningen der aarde.

  Ik zal hem eeuwig Mijn goedgunstigheid bewaren,~\sep\ en Mijn verbond met hem zal duurzaam zijn.

  Eeuwigdurend maak Ik zijn geslacht,~\sep\ en zijn troon als de dagen van de hemel.
\end{halfparskip}

\begin{halfparskip}
  \psalmsubtitle{d) Uw trouw, Heer, is onwankelbaar}

  Mochten zijn zonen Mijn wet verlaten,~\sep\ en niet wandelen naar Mijn geboden;

  Mochten zij Mijn voorschriften schenden,~\sep\ Mijn geboden niet bewaren,

  Dan zal Ik met de roede hun misdaad bestraffen,~\sep\ en met gesels hun schuld;

  Maar Mijn genade zal Ik hem niet onttrekken,~\sep\ en Ik zal niet breken Mijn trouw.

  Mijn verbond zal Ik niet schenden,~\sep\ noch de uitspraak van Mijn lippen veranderen.

  Eens en voor immer heb Ik bij Mijn heiligheid gezworen:~\sep\ Nooit breek Ik David Mijn woord,

  Zijn nakroost zal blijven in eeuwigheid,~\sep\ en zijn troon zal voor Mijn aanschijn zijn als de zon,

  Als de maan, die eeuwig blijft,~\sep\ een trouwe getuige aan de hemel."
\end{halfparskip}

\begin{halfparskip}
  \psalmsubtitle{e) Gij schijnt ontrouw, O Heer}

  Toch hebt Gij Uw gezalfde verstoten en verworpen,~\sep\ hevig tegen hem getoornd;

  Gij hebt het verbond met Uw dienaar versmaad,~\sep\ zijn kroon ontwijd in het stof,

  Gij hebt al zijn wallen geslecht,~\sep\ zijn vestingwerken in puin gelegd.

  Iedere voorbijganger heeft hem beroofd,~\sep\ hij is de spot van zijn buren geworden.

  Gij hebt de rechterhand van zijn vijanden verheven,~\sep\ al zijn weerstrevers met vreugde vervuld.

  De snee van zijn zwaard hebt Gij afgestompt,~\sep\ hem niet gesteund in de strijd.

  Zijn luister hebt Gij doen tanen,~\sep\ en zijn troon ter aarde geworpen.

  De dagen van zijn jeugd hebt Gij verkort,~\sep\ Gij hebt hem met schande bedekt.
\end{halfparskip}

\begin{halfparskip}
  \psalmsubtitle{f) Vervul, Heer, Uw belofte!}

  Hoe lang nog, Heer; zult Gij U dan immer verbergen!~\sep\ Moet Uw misnoegen als een vuur blijven branden?

  Bedenk hoe kort mijn leven is,~\sep\ hoe zwak Gij alle mensen geschapen hebt.

  Wie is er, die leeft, en de dood niet ziet,~\sep\ die zich aan de macht van de onderwereld onttrekt?

  Waar is Uw aloude goedgunstigheid, Heer,~\sep\ die Gij David bij Uw trouw hebt gezworen?

  Gedenk toch, Heer, de smaad van Uw dienaars;~\sep\ ik draag in mijn boezem al de vijandschap van de volkeren,

  Waarmee Uw weerstrevers U honen, O Heer,~\sep\ en bij iedere schrede Uw gezalfde verguizen.

  Gezegend de Heer in eeuwigheid;~\sep\ zo zij het, zo zij het!
\end{halfparskip}

\markedsubsectionrubricwithhint{1. Zondagen ``na'':}{Hulala~21 (marmiyata~58--60; canticles)}

\subsubsection*{Marmita 58. \normalfont{\emph{Ex 15,1--21}.}\nopagebreak}

\begin{halfparskip}
  Toen zongen Mozes en de kinderen van Israël~\sep\ dit lied voor de Heer, zeggende:

  \liturgicalhint{Alleluia, alleluia, alleluia.}~--- \liturgicalhint{Eerste vers:} Toen zongen Mozes en de kinderen van Israël...

  \psalmsubtitle{a) Gij hebt de Egyptenaren in zee geworpen, O Heer}

  ``Ik wil de Heer bezingen, want Hij is hoogverheven:~\sep\ ros en strijdwagen wierp Hij in zee.

  Mijn kracht en sterkte is de Heer,~\sep\ Hij is mij tot Redder geworden.

  Hij is mijn God, en ik zal Hem prijzen,~\sep\ de God van mijn vader, en ik zal Hem loven.

  De Heer is een krijgsheld:~\sep\ ``Heer'' is Zijn Naam.

  De wagens van Farao en zijn legermacht wierp Hij in zee,~\sep\ in de Rode Zee werd de bloem van zijn veldheren verzwolgen.

  De golven bedekten hen,~\sep\ ze zonken als een steen in de diepte.
\end{halfparskip}

\begin{halfparskip}
  \psalmsubtitle{b) Uw macht, O Heer, heeft overwonnen}

  Uw rechterhand, O Heer, die geweldig zijt in kracht,~\sep\ Uw rechterhand, O Heer, heeft de vijand verslagen.

  Ja, in de volheid van Uw Majesteit hebt Gij Uw weerstrevers vernietigd;~\sep\ Gij liet de vrije loop aan Uw toorn, die hen als stoppels verteerde.

  Door de adem van Uw gramschap hoopten de wateren zich op, bleven de golven staan als een dam,~\sep\ stolden de baren in het midden der zee.

  De vijand sprak: Ik zal hen vervolgen, hen aangrijpen, de buit verdelen,~\sep\ en mijn woede zal worden verzadigd.

  Mijn zwaard zal ik trekken,~\sep\ mijn hand zal hen beroven.

  Met de adem van Uw wind hebt Gij geblazen, de zee heeft hen bedekt,~\sep\ ze zonken als lood in de geweldige wateren.

  Wie onder de goden is U gelijk, Heer, wie gelijk aan U, die uitschittert door heiligheid,~\sep\ Gij, Lofwaardige, die wonderen doet?

  Gij hebt Uw rechterhand uitgestrekt:~\sep\ de aarde heeft hen verslonden.

  In Uw goedheid hebt Gij het volk geleid dat Ge hadt verlost,~\sep\ Gij hebt het geleid door Uw macht naar Uw heilige woonstede.
\end{halfparskip}

\begin{halfparskip}
  \psalmsubtitle{c) Gij zelf, O Heer, leidt Uw volk in het Beloofde Land}

  De volken hoorden het, ze beefden:~\sep\ angst greep de bewoners van Filistea aan.

  Toen werden de vorsten van Edom ontsteld,~\sep\ ontzetting greep de veldheren van Moab aan;

  De kracht ontzonk aan alle bewoners van Chanaän,~\sep\ siddering en ontsteltenis overweldigden hen.

  Vanwege de kracht van Uw arm,~\sep\ stonden zij als versteend,

  Tot Uw volk was doorgetrokken, Heer,~\sep\ tot het volk was doorgetrokken, dat Gij hebt verworven.

  Gij hebt hen binnengevoerd en gevestigd op de berg, die Uw eigendom is, in het oord, dat Gij tot woonplaats hebt bereid, Heer,~\sep\ in het heiligdom, Heer, door Uw handen gegrondvest.

  De Heer zal heersen,~\sep\ voor eeuwig en immer.

  Want toen de paarden van farao de zee introkken met zijn wagens en ruiters, bedolf de Heer hen met de golven der zee,~\sep\ maar Israëls kinderen trokken er droogvoets doorheen.

  En Maria, de profetes, de zuster van Aäron, nam de tamboerijn in de hand,~\sep\ en terwijl alle vrouwen met tamboerijnen haar dansende volgden, herhaalde Maria voor hen het refrein:

  ``Ik wil de Heer bezingen, want Hij is hoogverheven:~\sep\ ros en strijdwagen wierp Hij in zee.''
\end{halfparskip}

\begin{halfparskip}
  \liturgicalhint{Is 42.} Zingt een nieuw lied ter ere van de Heer;~\sep\ heft een lofzang voor Hem aan op de grenzen der aarde.

  Gij, die de zee beploegt en bevolkt,~\sep\ met de eilanden, en die er op wonen.

  De steppe jubele met haar steden, de legerplaats waar Kedar woont;~\sep\ laat de bewoners van Sela juichen, jubelen van de toppen der bergen.

  Laat hen glorie brengen aan de Heer,~\sep\ en aan de eilanden Zijn lof verkonden!

  De Heer rukt uit als een held, als een krijger blakend van strijdlust;~\sep\ bulderend heft Hij de strijdkreet aan, en daagt Zijn vijanden uit.

  Hemelen, dauwt uit de hoge; wolken, laat de gerechtigheid stromen;~\sep\ aarde, open uw schoot, om vrucht van verlossing te dragen, en gerechtigheid te laten ontspruiten. Ik, de Heer, heb het gewrocht.
\end{halfparskip}

\markedsubsectionrubric{Alle zondagen:}

\begin{halfparskip}
  Eer aan...~--- Vanaf het begin...~--- Alleluia, alleluia, eer aan U, O God, alleluia, alleluia, eer aan U, O God, alleluia, alleluia, O Heer, ontferm U over ons.

  \dd~Laat ons bidden; vrede zij met ons.

  \cc~Maak ons waardig, onze Heer en onze God, om samen met de waakzame wezens en de koren van engelen, met stemmen vol belijdenis, Uw glorieuze Drie-eenheid dag en nacht te prijzen, Heer van alles...
\end{halfparskip}

\markedsubsectionrubricwithhint{2. Zondagen ``voor'':}{Marmita~36 (\Pss{90--92}).}

\PSALMtitle{90}{Ons leven is kort en vergankelijk}

\begin{halfparskip}
  \psalmsubtitle{a) God van eeuwigheid!}

  Heer, Gij waart ons tot toevlucht,~\sep\ van geslacht tot geslacht.~\sep\ Eer de bergen werden verwekt,

  \liturgicalhint{Alleluia, alleluia, alleluia.}~--- \liturgicalhint{Eerste vers:} Heer, Gij waart ons tot toevlucht...

  eer aarde en wereld geboren werden:~\sep\ van eeuwigheid tot eeuwigheid zijt Gij, O God.

  De stervelingen beveelt Gij terug te keren tot stof,~\sep\ Gij spreekt: ``Keert terug, gij kinderen der mensen.''

  Want duizend jaren zijn in Uw ogen als de dag van gister, die verstreek,~\sep\ en als een wake in de nacht.

  Gij rukt ze weg: zij worden als een droom in de morgen,~\sep\ als het groenende gras.

  `s~Morgens bloeit het en groeit het,~\sep\ `s~avonds wordt het gemaaid en verdort.
\end{halfparskip}

\begin{halfparskip}
  \psalmsubtitle{b) Om onze zonden, O God, zijn wij in ellende}

  Waarlijk, door Uw toorn zijn wij verteerd,~\sep\ door Uw verbolgenheid ontsteld;

  Gij hebt onze schuld U voor ogen gesteld,~\sep\ onze geheime zonden in het licht van Uw aanschijn.

  Ja, door Uw toorn vloden al onze dagen voorbij,~\sep\ vergingen als een zucht onze jaren.

  Zeventig jaren duurt hoogstens ons leven,~\sep\ of tachtig, als we krachtig zijn;

  De meeste daarvan zijn kommer en schijn,~\sep\ want snel gaan zij heen en zij vlieden voorbij.

  Wie kan de kracht van Uw gramschap meten,~\sep\ en Uw toorn naar de vreze, verschuldigd aan U?
\end{halfparskip}

\begin{halfparskip}
  \psalmsubtitle{c) Geef blijdschap, Heer, om het doorgestane leed!}

  Leer ons onze dagen tellen,~\sep\ opdat wij de wijsheid van het hart verwerven.

  Keer terug, O Heer; hoelang nog?~\sep\ en wees Uw dienaars genadig.

  Verzadig ons spoedig met Uw ontferming,~\sep\ opdat wij juichen en blij zijn al onze dagen.

  Geef ons vreugde voor de dagen toen Gij ons hebt gekastijd,~\sep\ voor de jaren, waarin wij ellende doorstonden.

  Laat Uw werk voor Uw dienaren stralen,~\sep\ en voor hun kinderen Uw glorie!

  Dale op ons neer de goedheid van de Heer, onze God; doe het werk van onze handen gedijen voor ons,~\sep\ doe het werk van onze handen gedijen!
\end{halfparskip}

\PSALMtitle{91}{Onder Gods hoede}

\begin{halfparskip}
  \psalmsubtitle{a) De Allerhoogste, uw Behoeder in alle nood}

  Gij, die onder de hoede van de Allerhoogste leeft,~\sep\ in de schaduw van de Almachtige woont,

  Zeg tot de Heer: ``Mijn toevlucht en burcht,~\sep\ mijn God, op wie ik vertrouw.''

  Want Hij zal u bevrijden uit de strik van de jagers,~\sep\ van de verderfelijke pest.

  Hij zal u met Zijn vleugels beschermen, en onder Zijn wieken zult gij vluchten;~\sep\ een beukelaar en schild is Zijn trouw.

  Geen verschrikking bij nacht zult gij vrezen,~\sep\ geen pijl, die voortsnort bij dag.

  Geen pest, die rondwaart in het duister,~\sep\ geen verwoestend verderf op de middag.

  Al vallen er duizend aan uw zijde, en tienduizend aan uw rechterhand,~\sep\ het zal tot u niet genaken.

  Ja, met eigen ogen zult gij het zien,~\sep\ en de vergelding der bozen aanschouwen.

  Want de Heer is uw toevlucht,~\sep\ de Allerhoogste hebt gij gemaakt tot uw schutse.

  Geen ramp zal u genaken,~\sep\ en geen plaag zal naderen tot uw tent.

  Want Hij gaf over u een bevel aan Zijn engelen,~\sep\ u te behoeden op al uw wegen.

  Zij zullen u op de handen dragen,~\sep\ opdat gij aan geen steen uw voet zoudt stoten.

  Op slang en adder zult gij trappen,~\sep\ leeuw en draak zult gij vertreden,
\end{halfparskip}

\begin{halfparskip}
  \psalmsubtitle{b) Ik zal hem bevrijden en verheerlijken}

  Omdat hij Mij aanhangt, zal Ik hem redden;~\sep\ omdat hij Mijn Naam kent, hem beschermen.

  Hij zal Mij aanroepen, en Ik zal hem verhoren, in de nood zal Ik met hem zijn,~\sep\ Ik zal hem redden en hem eren.

  Ik zal hem verzadigen met lengte van dagen,~\sep\ en hem tonen Mijn heil.
\end{halfparskip}

\PSALMtitle{92}{Vreugde over Gods rechtvaardigheid}

\begin{halfparskip}
  \psalmsubtitle{a) Ik wil U loven, O Heer!}

  Heerlijk is het de Heer te loven,~\sep\ Uw Naam te bezingen, O Allerhoogste;

  vroeg in de morgen Uw erbarming te verkondigen,~\sep\ en Uw trouw gedurende de nacht,

  Op de tiensnarige harp en de lier,~\sep\ met zang bij citerspel.

  Want door Uw daden, Heer, verheugt Gij mij,~\sep\ ik juich om het werk van Uw handen.
\end{halfparskip}

\begin{halfparskip}
  \psalmsubtitle{b) De boze miskent Uw Voorzienigheid en vergaat}

  Hoe luisterrijk, Heer, zijn Uw werken,~\sep\ hoe diepzinnig zijn Uw gedachten!

  De onverstandige beseft het niet,~\sep\ en de dwaze ziet het niet in.

  Al bloeien de goddelozen als gras,~\sep\ al schitteren allen, die kwaad bedrijven,

  Tot eeuwige ondergang zijn zij gedoemd;~\sep\ maar Gij, O Heer, zijt eeuwig verheven.

  Want zie, Uw vijanden, Heer, Uw vijanden zullen vergaan:~\sep\ alle bozen zullen worden verstrooid.

  Als de hoorn van een buffel hebt Gij mijn hoorn verheven,~\sep\ mij gezalfd met de zuiverste olie.

  Mijn oog zag op mijn vijanden neer,~\sep\ en over de bozen, die tegen mij opstonden, vernamen mijn oren verblijdende dingen.
\end{halfparskip}

\begin{halfparskip}
  \psalmsubtitle{c) Maar de vrome bloeit als een palmboom}

  De rechtvaardige zal als een palmboom bloeien,~\sep\ als een ceder van de Libanon gedijen.

  Die staan geplant in het huis van de Heer,~\sep\ zullen bloeien in de voorhoven van onze God.

  Tot in hun ouderdom dragen zij vrucht,~\sep\ blijven zij sappig en fris,

  Om te verkondigen hoe rechtvaardig de Heer is,~\sep\ mijn Rots, en dat in Hem geen onrecht is.
\end{halfparskip}

\markedsubsectionrubricwithhint{2. Zondagen ``na'':}{Marmita~59.}

\begin{halfparskip}
  \psalmsubtitle{a) Luistert naar mijn woorden!}

  \liturgicalhint{Dt 32,1--21a.} Hoort, hemelen, want ik ga spreken;~\sep\ en de aarde luistere naar de woorden van mijn mond.

  \liturgicalhint{Alleluia, alleluia, alleluia.}~--- \liturgicalhint{Eerste vers:} Hoort, hemelen, want ik ga spreken...

  Moge mijn leer als regen neerdalen,~\sep\ mijn rede neerdruppelen als dauw,

  Als stortregen op het gras,~\sep\ als regen op de kruiden.

  Want de Naam van de Heer wil ik prijzen:~\sep\ verheerlijkt onze God!
\end{halfparskip}

\begin{halfparskip}
  \psalmsubtitle{b) Grondgedachte: Gods trouw en rechtvaardigheid; Israëls boosheid}

  Hij is de rots: volmaakt zijn Zijn werken,~\sep\ want al Zijn wegen zijn rechtvaardig.

  God is getrouw en geen onrecht is in Hem;~\sep\ Hij is rechtvaardig en oprecht.

  Slecht hebben ontaarde zonen zich jegens Hem gedragen;~\sep\ het is een bedorven en verworden geslacht.

  Vergeldt gij aldus de Heer,~\sep\ gij, dwaas en onzinnig volk?

  Is Hij uw Vader niet die u heeft voortgebracht,~\sep\ heeft Hij u niet gemaakt en bevestigd?
\end{halfparskip}

\begin{halfparskip}
  \psalmsubtitle{c) Gods weldaden voor Israël}

  Denk aan de aloude dagen,~\sep\ doorloopt de jaren van alle geslachten;

  Vraag het uw vader, en hij zal het u tonen,~\sep\ uw grijsaards, en zij zullen het u zeggen:

  Toen de Allerhoogste de volken hun gebied aanwees,~\sep\ toen Hij de zonen van Adam scheidde,

  Stelde Hij de grenzen der volkeren vast,~\sep\ naar het getal van Israëls zonen.

  Want het deel van de Heer is Zijn volk,~\sep\ Jacob het erfdeel, voor Hem bestemd.

  Hij vond hem in een onbewoond land,~\sep\ in een woeste streek, bij het huilen der wildernis.

  Hij heeft hem gekoesterd en verzorgd,~\sep\ hem behoed als de appel van Zijn oog.

  Gelijk de adelaar zijn broedsel lokt,~\sep\ cirkelt boven zijn jongen,

  Zo spreidde Hij Zijn vleugels uit en nam hem op,~\sep\ en droeg hem op Zijn wieken.

  De Heer alleen heeft hem geleid,~\sep\ en geen enkele vreemde god was met Hem.

  Hij droeg hem op de hoogte in het land,~\sep\ voedde hem met de vruchten der velden.

  Hij liet hem honing zuigen uit de rots,~\sep\ en olie uit allerhardste steen.

  Hij gaf hem boter van het vee en melk van schapen,~\sep\ met het vet van lammeren en rammen,

  Stieren van Basan en bokken, met de fijnste bloem van tarwe;~\sep\ en het sap van de druif hebt gij gedronken, een edele wijn.
\end{halfparskip}

\begin{halfparskip}
  \psalmsubtitle{d) Ontrouw van Israël}

  Gegeten heeft Jacob en zich verzadigd, de lieveling werd vet en sloeg achteruit:~\sep\ hij werd vet, dik en gezet.

  God, zijn Schepper, heeft hij verlaten,~\sep\ en de Rots van zijn heil versmaad.

  Zij tergden Hem met vreemde goden,~\sep\ tartten Hem door hun gruwelen.

  Zij offerden aan duivels, die geen goden zijn,~\sep\ aan goden, hun tot dan toe onbekend,

  Aan nieuwe, onlangs opgekomen,~\sep\ die uw vaderen niet hadden vereerd.

  De Rots, die u voortbracht, hebt gij verlaten,~\sep\ en God, uw Schepper, vergeten.
\end{halfparskip}

\begin{halfparskip}
  \psalmsubtitle{e) God dreigt Israël met straffen}

  God zag het,~\sep\ en in gramschap ontstoken verwierp Hij Zijn zonen en dochters.

  Hij sprak: ``Ik zal hun Mijn aanschijn verbergen, en wil zien wat hun einde zal zijn;~\sep\ want ze zijn een bedorven geslacht, trouweloze kinderen!''.
\end{halfparskip}

\markedsubsectionrubric{Alle zondagen:}

\begin{halfparskip}
  Eer aan...~--- Vanaf het begin...~--- Alleluia, alleluia, eer aan U, O God, alleluia, alleluia, eer aan U, O God, alleluia, alleluia, O Heer, ontferm U over ons.

  \dd~Laat ons bidden; vrede zij met ons.

  \cc~Elke natuur van rationele wezens die U hebt geschapen, Heer, moet voortdurende glorie, onophoudelijke alleluia's, eindeloze lofzangen en stemmen vol dankzegging, dag en nacht verheffen tot Uw glorieuze Drie-eenheid, Heer van alles...
\end{halfparskip}

\markedsubsectionrubricwithhint{3. Zondagen ``voor'':}{Marmita~37 (\Pss{93--95}).}

\PSALMtitle{93}{God is koning voor eeuwig}

\begin{halfparskip}
  De Heer is Koning, met majesteit bekleed;~\sep\ bekleed is de Heer met macht, Hij heeft Zich omgord;

  \liturgicalhint{Alleluia, alleluia, alleluia.}~--- \liturgicalhint{Eerste vers:} De Heer is Koning...

  Hij heeft het aardrijk bevestigd,~\sep\ dat niet zal wankelen.

  Hecht staat Uw troon van ouds;~\sep\ Gij zijt van eeuwigheid.

  De stromen verheffen, O Heer, de stromen verheffen hun stem,~\sep\ de stromen verheffen hun bruisen.

  Maar boven de stem der wijde wateren, boven de branding der zee~\sep\ is machtig de Heer in de hoge.

  Betrouwbaar bovenmate zijn Uw getuigenissen;~\sep\ Uw huis, Heer, past heiligheid in lengte van dagen.
\end{halfparskip}

\PSALMtitle{94}{Beroep op Gods rechtvaardigheid}

\begin{halfparskip}
  \psalmsubtitle{a) Straf, Heer, onze goddeloze verdrukkers!}

  Wrekende God, O Heer,~\sep\ wrekende God, verschijn!

  Rijs op, Gij, Rechter der aarde,~\sep\ vergeld naar verdienste de trotsen!

  Hoelang nog zullen de bozen, O Heer,~\sep\ hoelang nog zullen de bozen roemen,

  Zullen zij snoeven, onbeschaamd spreken,~\sep\ pochen, die het kwade bedrijven?

  Uw volk, Heer, vertrappen zij,~\sep\ Uw erfdeel drukken zij neer;

  Zij doden weduwe en vreemdeling,~\sep\ en wezen brengen zij om.

  En dan zeggen ze nog: ``De Heer ziet het niet,~\sep\ de God van Jacob merkt het niet op.''
\end{halfparskip}

\begin{halfparskip}
  \psalmsubtitle{b) Gij, Heer, ziet en bestraft de bozen}

  Komt toch tot inzicht, gij dwazen onder het volk,~\sep\ onverstandigen, wanneer wordt gij wijs?

  Zou Hij, die het oor heeft geplant, niet horen,~\sep\ of die het oog heeft gevormd, niet zien?

  Zou Hij, die de volken opvoedt, niet straffen,~\sep\ Hij, die de mensen inzicht geeft?

  De Heer kent de gedachten der mensen:~\sep\ want ja, ze zijn ijdel.
\end{halfparskip}

\begin{halfparskip}
  \psalmsubtitle{c) Gij, de trouwe Behoeder van het recht}

  Gelukkig de man, die Gij onderricht, O Heer,~\sep\ en die Gij onderwijst door Uw Wet,

  Om hem rust te schenken in tijden van nood,~\sep\ tot voor de boze het graf is gedolven.

  Want nooit zal de Heer Zijn volk verstoten,~\sep\ noch Zijn erfdeel verlaten;

  Maar weer zal er recht in de rechtspraak zijn,~\sep\ de rechtschapenen van hart zullen allen het volgen.

  Wie treedt voor mij tegen de boosdoeners op,~\sep\ wie staat mij tegen de booswichten bij?

  Stond de Heer mij niet bij,~\sep\ dan woonde ik spoedig in het oord van stilte.

  Als ik denk: ``Nu wankelt mijn voet'',~\sep\ dan steunt mij Uw goedheid, Heer.

  Als kommer steeds meer mijn hart benauwt,~\sep\ dan verkwikt Uw vertroosting mijn ziel.

  Heeft met U iets gemeen een partijdige rechtbank,~\sep\ die kwellingen wekt onder schijn van wet?

  Men mag dan de gerechte het leven belagen,~\sep\ en onschuldig bloed voor schuldig verklaren,

  De Heer zal mij zeker tot beschutting zijn,~\sep\ en mijn God mijn beschermende rots.

  Maar hun zal hij hun onrecht vergelden, hen door eigen boosheid te gronde doen gaan;~\sep\ de Heer onze God zal hen verderven.
\end{halfparskip}

\PSALMtitle{95}{Eert God de Schepper!}

\begin{halfparskip}
  \psalmsubtitle{a) Looft God!}

  Komt, laten wij jubelen voor de Heer,~\sep\ laten wij juichen voor de Rots van ons heil.

  Treden wij voor Zijn aanschijn met lofzangen,~\sep\ juichen wij Hem met liederen toe!

  Want de Heer is een grote God,~\sep\ en een grote Koning boven alle goden.

  Hij houdt in Zijn hand de diepten der aarde,~\sep\ en de toppen der bergen behoren Hem toe.

  Van Hem is de zee; Hij heeft ze geschapen,~\sep\ en het vaste land, door Zijn handen gevormd.

  Komt, laten wij aanbidden en ons neerwerpen,~\sep\ de knieën buigen voor de Heer, die ons heeft gemaakt.

  Want Hij is onze God;~\sep\ wij het volk van Zijn weide en de schapen van Zijn hand.
\end{halfparskip}

\begin{halfparskip}
  \psalmsubtitle{b) Gehoorzaamt God!}

  Mocht gij toch heden Zijn stem vernemen; "Wilt niet Uw harten verstokken als bij Meriba, als op de dag van Massa in de woestijn,~\sep\ waar Uw vaderen Mij tergden, Mij beproefden, hoewel zij Mijn werken zagen.

  Veertig jaar was dat geslacht Mij een walg en Ik zei:~\sep\ Het is een volk dat dwaalt in zijn hart en Mijn wegen kennen zij niet.

  Daarom heb Ik in Mijn gramschap gezworen:~\sep\ Neen, zij zullen Mijn rust niet binnengaan.''
\end{halfparskip}

\markedsubsectionrubricwithhint{3. Zondagen ``na'':}{Marmita~60.}

\begin{halfparskip}
  \liturgicalhint{Dt. 32,21--43.} Ze hebben Mij geprikkeld door een god van niets, Mij door hun ijdelheden getart;~\sep\ maar Ik zal hen prikkelen door een volk van niets, hen tarten door een ijdel volk.

  \liturgicalhint{Alleluia, alleluia, alleluia.}~--- \liturgicalhint{Eerste vers:} Ze hebben Mij geprikkeld door een god van niets...

  Want een vuur is ontvlamd in Mijn woede, dat tot het diepst van het dodenrijk brandt!~\sep\ Het zal de aarde met haar gewassen verteren, de grondvesten der bergen verzengen.

  Ik zal hen overstelpen met rampen,~\sep\ op hen Mijn pijlen verschieten;

  Ze zullen uitgeput worden door honger, verteerd door koorts en giftige pest.~\sep\ Tanden van wilde beesten laat Ik tegen hen los, met venijn van serpenten in het stof;

  Buiten moordt het zwaard hen uit, de schrik binnenshuis:~\sep\ jongemannen als maagden, zuigelingen met grijsaards.
\end{halfparskip}

\begin{halfparskip}
  \psalmsubtitle{f) Israël zal niet geheel vernietigd worden}

  Ik had zeker gezegd: Ik vaag hen weg,~\sep\ laat zelfs hun gedachtenis onder de mensen verdwijnen,

  Zo Ik de hoon van de vijand niet vreesde, hun tegenstanders het niet verkeerd zouden verstaan, en zeggen:~\sep\ ``Het was onze machtige hand, niet de Heer heeft dit alles gedaan!''

  Want ze zijn een volk, dat het begrip heeft verloren,~\sep\ en zonder verstand;

  Waren ze wijs, ze zouden het hebben begrepen,~\sep\ en hun krijgsgeluk hebben verstaan.

  Hoe toch zou één er duizend hebben vervolgd, en twee er tienduizend op de vlucht kunnen jagen,~\sep\ zo hun Rots hen niet had prijsgegeven, en de Heer hen niet had overgeleverd?

  Want niet als onze Rots is de hunne:~\sep\ dat erkennen onze vijanden zelf!

  Neen, van Sodoma's wijnstok stammen hun ranken, Gomorra's wingerd:~\sep\ hun druiven zijn giftige bessen; hun trossen vol bitterheid;

  Drakengif is hun wijn,~\sep\ dodelijk addervenijn.

  Ligt dat niet bij Mij bewaard,~\sep\ in Mijn schatkamers verzegeld,

  Voor de dag van wraak en vergelding; voor de tijd dat hun voeten wankelen?~\sep\ Want nabij is de dag van hun ondergang, wat hun bereid is, snelt toe!
\end{halfparskip}

\begin{halfparskip}
  \psalmsubtitle{g) Israël gestraft, opdat het Zijn God erkenne}

  Want de Heer schaft recht aan Zijn volk, en ontfermt Zich over Zijn dienaars,~\sep\ wanneer Hij ziet dat hun kracht is geweken, en er geen slaaf en geen vrije meer is.

  Dan zal Hij zeggen:~\sep\ ``Waar zijn nu hun goden, de rots, tot wie ze hun toevlucht namen:

  Die het vet van hun slachtoffers aten, en de wijn van hun plengoffers dronken?~\sep\ Laat hen opstaan en u helpen, een schutse voor u zijn!

  Ziet nu dat Ik, dat Ik het ben, en dat er geen God naast Mij is:~\sep\ Ik dood en maak levend, verbrijzel en heel!

   En er is niemand die redt uit Mijn hand!
\end{halfparskip}

\begin{halfparskip}
  \psalmsubtitle{h) Trotse heidenen}

  Waarachtig, Ik hef Mijn hand naar de hemel,~\sep\ en zeg: Zowaar Ik eeuwig leef!

  Wanneer Ik Mijn bliksemend zwaard heb gewet, en Mijn hand naar het strafgericht grijpt,~\sep\ zal Ik Mij wreken op Mijn vijand, en die Mij haten doen boeten.

  Dan maak Ik Mijn pijlen dronken van bloed, en Mijn zwaard zal vlees gaan verslinden:~\sep\ Van het bloed der verslagenen en gevangenen, van het hoofd der vijandelijke vorsten.

  Stemt, naties, een jubelzang aan voor Zijn volk: omdat Hij het bloed van Zijn dienaren wreekt,~\sep\ wraak aan Zijn vijanden oefent, maar het land van zijn volk vergiffenis schenkt!
\end{halfparskip}

\markedsubsectionrubric{Alle zondagen:}

\begin{halfparskip}
  Eer aan...~--- Vanaf het begin...~--- Alleluia, alleluia, alleluia.

  \dd~Laat ons bidden; vrede zij met ons.

  \cc~Moge de Naam van Uw Godheid en van Uw Majesteit, o Heer, worden aanbeden, verheerlijkt, geëerd, verheven, beleden en gezegend in de hemel en op aarde door de redelijke monden die U hebt geschapen, door de verheerlijkende tongen die U hebt gevormd, en door alle gezelschappen van boven en beneden, Heer van alles...
\end{halfparskip}

% % % % % % % % % % % % % % % % % % % % % % % % % % % % % % % % % % % % % % % %

\markedsection{Onita d-lelya \markedsectionhint{(Eigen, zie Hudra.)}}

\begin{halfparskip}
  \liturgicalhint{Op Zondagen de 2 laatste verzen zijn altijd als volgt:}

  Vanaf het begin en in eeuwigheid.~--- Christus, verwaarloos ons niet, laat Uw aanbidders niet in de steek, want bij U, mijn Heer, hebben wij onze toevlucht gezocht. Leid ons in Uw levenswijze, zodat wij allen U mogen prijzen, Heer God.

  Laat al het volk zeggen, Amen and amen.~--- Maria, heilige maagd, moeder van Jezus onze Verlosser; bid voor ons bij Christus, dat Hij Zijn vrede onder ons laat wonen en ons dag en nacht beschermt tegen alle kwaad.

  \liturgicalhint{Van Vasten tot Pinksteren worden deze verzen weggelaten; in Aankondiging wordt het vers ``Laat al het volk'' wegegelaten.}

  \dd~Laat ons bidden, vrede zij met ons.

  \cc~Glorie aan U uit alle monden, dankzegging uit alle tongen, aanbidding, eer en verheerlijking van alle schepselen, o verborgen en glorieus Wezen, dat op de meest verheven hoogten woont, Heer van alles...
\end{halfparskip}

% % % % % % % % % % % % % % % % % % % % % % % % % % % % % % % % % % % % % % % %

\markedsection{Subaha \markedsectionhint{(Eigen teksten, met 3x ``Eer aan U, O God'', gevolgd door de Hpakta, zie Hudra.)}}

% % % % % % % % % % % % % % % % % % % % % % % % % % % % % % % % % % % % % % % %

\markedsection{Tesbohta \markedsectionhint{(van Mar Narsai)}}

\begin{halfparskip}
  Lof aan de Goede die ons ras heeft bevrijd van de slavernij van de boze en van de dood.

  En die vrede heeft gesloten tussen ons en de koren van boven, die boos waren vanwege onze ongerechtigheid.

  Gezegend is de Medelevende, die, ook al zochten wij Hem niet, naar voren kwam om ons te zoeken en Zich in ons leven verheugde.

  Hij schilderde een gelijkenis van onze afdwaling en onze terugkeer in het verloren schaap.

  Hij noemde onze natuur ``erfgenaam'' en ``zoon'', die afdwaalde en terugkeerde, die stierf en weer tot leven kwam.

  Hij heeft de geestelijke koren blij gemaakt door ons berouw en onze verrijzenis.

  Onuitsprekelijk is de grote liefde die de Vriend van ons ras jegens ons heeft getoond,

  Die van ons ras een Middelaar nam, en de wereld verzoende met Zijn Majesteit.

  Ver hoog boven ons en alle schepselen staat dit nieuwe ding dat Hij voor onze mensheid heeft volbracht:

  dat Hij van ons lichaam een heilige tempel heeft gemaakt, zodat Hij daarin de aanbidding van allen zou kunnen vervolmaken.

  Kom, aardse en hemelse wezens, verwondert en staat versteld van de grootsheid van de stap (\translationoptionNl{waardigheid}), want ons ras heeft de grote hoogten van de onbereikbare Godheid bereikt.

  Mogen hemel en aarde, en alles wat daarin is, samen met ons Hem belijden die ons ras heeft verheven.

  Hij heeft ons beeld vernieuwd en onze ongerechtigheid uitgewist; Hij riep ons bij Zijn Naam en maakte alle dingen aan ons onderworpen.

  Hij is waardig om door alle monden geprezen te worden, Hij heeft ons boven allen en alles verheven.

  Laten we allen Hem loven, voor altijd en eeuwig, amen en amen.
\end{halfparskip}

% % % % % % % % % % % % % % % % % % % % % % % % % % % % % % % % % % % % % % % %

\markedsection{Karozuta \markedsectionhint{(Eigen tekst of:)}}

\begin{halfparskip}
  \dd~Laat ons allen ordelijk staan met vreugde en vrolijkheid; laat ons bidden en zeggen: Heer, ontferm U over ons.~--- \rr~Heer, ontferm U over ons. \liturgicalhint{(Wordt herhaald na elke aanroeping.)}

  U die ons geleerd hebt te bidden en traagheid te vermijden, wij bidden U.

  U die de nacht hebt doorgebracht in gebed tot God voor de redding van ons ras, wij bidden U.

  U die ons door onze aardse vaders een bewijs van Uw barmhartigheid heeft gegeven, wij bidden U.

  U die ons van de machtige dood hebt gered, en op wie wij vertrouwen om ons te redden, wij bidden U.

  U die ons gered heeft van de macht van de duisternis en ons naar het koninkrijk van Uw geliefde Zoon hebt gebracht, wij bidden U.

  U die zei: ``Vraag en u zal gegeven worden, zoek en u zult vinden, klop en de schat van barmhartigheid zal voor u opengaan'', wij bidden U.

  Voor de gezondheid van onze heilige vaders, Paus~\NN , hoofd van de hele Kerk van Christus, van Patriarch~\NN , van onze Catholicos~\NN , van onze Metropoliet~\NN , van onze Bisschop~\NN , en van al hun helpers, wij bidden U,

  Barmhartige God, die met Uw liefde alles bestuurt, wij bidden U,

  U die in de hemel wordt geprezen en op aarde wordt aanbeden, wij bidden U,

  Geef ons de overwinning, Christus onze Heer, bij Uw komst, en geef vrede aan Uw Kerk, gered door Uw kostbare bloed, en ontferm U over ons.

  \liturgicalhint{\textbf{Slota} (eigen tekst ofwel de volgende:)} \dd~Laat ons bidden; vrede zij met ons.

  \cc~Maak ons waardig, onze Heer en onze God, U te dienen volgens Uw goddelijke wil en Uw glorieuze Majesteit, puur en nobel, waakzaam en ernstig, rechtvaardig en oprecht, heilig en onberispelijk. En moge onze dienst, mijn Heer, U behagen, ons gebed en onze wake U overtuigen, ons verzoek U gunstig stemmen, ons smeken U eren, onze smeekbede U verzoenen; en moge Uw goddelijke barmhartigheid en mededogen de overtredingen van Uw volk vergeven en de zonden kwijtschelden van alle schapen van Uw weide, die U voor Uzelf hebt uitgekozen in Uw genade en barmhartigheid, Gij goede Vriend der mensen, Heer van alles...
\end{halfparskip}

% % % % % % % % % % % % % % % % % % % % % % % % % % % % % % % % % % % % % % % %

\end{document}