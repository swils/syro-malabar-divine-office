\documentclass[12pt,twoside,a5paper]{article}

\usepackage{multicol}

\usepackage[main=dutch]{babel}
\usepackage{divine-office}

% % % % % % % % % % % % % % % % % % % % % % % % % % % % % % % % % % % % % % % %

% Version: 2024-07-13

\begin{document}

\title{Ramsa~--- weekdagen}
\author{}
\date{}
\maketitle

% % % % % % % % % % % % % % % % % % % % % % % % % % % % % % % % % % % % % % % %

\begin{halfparskip}
  \cc~Eer aan God in den hoge \liturgicalhint{(3x)}. En op aarde vrede en goede hoop aan de mensen, altijd en in eeuwigheid. [Amen\footnote{De Hudra en de Syro-Malabar liturgie voegen hier ``Amen'' aan toe, in tegenstelling tot Breviarium.}.]~--- \rr~Zegen, Heer.~--- \liturgicalhint{Vredekus.}

  \cc~Onze Vader die in de hemelen zijt,

  \rr~Geheiligd zij Uw Naam. Uw rijk kome, heilig, heilig, heilig zijt Gij. Onze Vader die in de hemelen zijt, de hemel en de aarde zijn gevuld met Uw onmetelijke glorie; de engelen en de mensen roepen U toe: heilig, heilig, heilig zijt Gij.~--- Onze Vader die in de hemelen zijt, geheiligd zij Uw Naam. Uw rijk kome, Uw wil geschiede op aarde zoals in de hemel. Geef ons heden het brood dat we nodig hebben en vergeef ons onze schulden en zonden zoals wij ook vergeven hebben aan onze schuldenaren. En leid ons niet in bekoring, maar verlos ons van de Kwade. Want van U is het koninkrijk en de kracht en de heerlijkheid in eeuwigheid, amen.

  \cc~Eer aan de Vader, de Zoon, en de Heilige Geest.

  \rr~Vanaf het begin en in alle eeuwigheid, amen en amen. Onze Vader die in de hemelen zijt, geheiligd zij Uw naam, Uw rijk kome, heilig, heilig, heilig zijt Gij. Onze Vader die in de hemelen zijt, de hemel en de aarde zijn gevuld met Uw onmetelijke glorie; de engelen en de mensen roepen U toe: heilig, heilig, heilig zijt Gij.

  \dd~Laat ons bidden, vrede zij met ons.

  \cc~We willen, Heer, Uw Godheid prijzen \liturgicalhint{(herhaal)} en Uw Majesteit aanbidden, en aan Uw glorierijke Drievuldigheid eeuwige, altijddurende lof brengen, Heer van alles, Vader... in alle eeuwigheid.~--- \rr~Amen.
\end{halfparskip}

% % % % % % % % % % % % % % % % % % % % % % % % % % % % % % % % % % % % % % % %

\markedsection{Marmita}

\liturgicalhint{Na de eerste zin van de eerste psalm van elke marmita, zeg 3x alleluja, en herhaal het eerste vers.}

\markedsubsectionrubricwithhint{Maandagen ``voor''.}{Marmita 4 (\Pss{11--14}).}

\begin{halfparskip}
  \psalm{\Ps{11}} Ik vlucht tot de Heer; hoe kunt gij mij zeggen:~\sep\ ``Vlieg weg als een vogel naar het gebergte.

  \qanona{Alleluia, alleluia, alleluia.}~--- \liturgicalhint{Eerste vers.}

  Want zie, de bozen spannen de boog; ze zetten de pijl op de pees,~\sep\ om de oprechten van hart in het duister te treffen.

  Als zelfs de grondvesten worden gesloopt,~\sep\ wat zal de gerechtige dan nog vermogen?''

  De Heer woont in Zijn heilige tempel,~\sep\ de Heer heeft in de hemel Zijn troon.

  Zijn ogen zien rond,~\sep\ Zijn wimpers doorvorsen de kinderen der mensen.

  De Heer doorvorst de gerechte en de boze;~\sep\ die het onrecht liefheeft, is Hem een gruwel.

  Hij zal op de zondaars gloeiende kolen en zwavel doen regenen;~\sep\ een verzengende wind is de dronk van hun beker.

  Want de Heer is rechtvaardig en heeft de gerechtigheid lief;~\sep\ de goeden zullen Zijn aanschijn aanschouwen.
\end{halfparskip}

\begin{halfparskip}
  \psalm{\Ps{12}} Heer, schenk redding, want er zijn geen vromen meer;~\sep\ verdwenen is de trouw onder de kinderen der mensen.

  Allen liegen ze elkander voor,~\sep\ ze spreken met bedrieglijke lippen en vals gemoed.

  De Heer rukke al die bedrieglijke lippen uit,~\sep\ die grootsprekende tong,

  Hen die zeggen: ``Sterk zijn wij door onze tong;~\sep\ wij hebben onze lippen met ons, wie kan ons overmeesteren?''

  ``Om de nood der verdrukten en het gejammer der armen zal Ik nu opstaan,'' zegt de Heer:~\sep\ ``redding zal Ik brengen aan wie er naar smacht.''

  De woorden van de Heer zijn oprechte woorden,~\sep\ zuiver zilver, van stof ontdaan, tot zevenmaal gelouterd.

  Gij, Heer, zult ons behouden,~\sep\ ons eeuwig beschermen tegen dit geslacht.

  De bozen zwermen om ons heen,~\sep\ terwijl de heffe van het volk oprijst.
\end{halfparskip}

\begin{halfparskip}
  \psalm{\Ps{13}} Hoelang nog, Heer, zult Gij mij geheel vergeten,~\sep\ hoelang nog voor mij Uw aanschijn verbergen?

  Hoelang nog zal ik de smart overdenken in mijn ziel,~\sep\ en het wee in mijn hart van dag tot dag?

  Hoelang nog zal mijn vijand zich boven mij verheffen?~\sep\ Zie neer en verhoor mij, o Heer, mijn God!

  Stort licht in mijn ogen opdat ik de doodsslaap niet inga~\sep\ en mijn vijand niet zegge: ``Ik heb hem overwonnen'',

  En mijn weerstrevers niet juichen over mijn val,~\sep\ daar ik op Uw erbarming vertrouwde.

  Nu juiche mijn hart om Uw hulp!~\sep\ De Heer wil ik bezingen, die mij heeft welgedaan.
\end{halfparskip}

\begin{halfparskip}
  \psalm{\Ps{14}} De dwaas zegt bij zichzelf:~\sep\ ``Er is geen God.''

  Ze zijn bedorven, gruwelen hebben ze bedreven;~\sep\ daar is er niet één, die deugdzaam handelt.

  De Heer blikt uit de hemel neer op de kinderen der mensen,~\sep\ om te zien of er wel één is met verstand, wel één die God zoekt.

  Allen zonder uitzondering zijn ze afgedwaald, allen diep bedorven;~\sep\ daar is er niet één die deugdzaam handelt, niet één.

  Zullen al die bozen dan nimmer tot inzicht komen,~\sep\ zij, die mijn volk verslinden, als aten zij brood?

  Zij riepen de Heer niet aan. Eens zullen zij sidderen van angst,~\sep\ want God is met het geslacht der rechtvaardigen.

  Het beleid van de verdrukte wilt gij beschamen,~\sep\ maar de Heer is zijn toevlucht.

  O, mocht er uit Sion toch heil voor Israël dagen! Als de Heer het lot van Zijn volk ten goede keert,~\sep\ zal er gejubel zijn in Jacob, vreugde in Israël.
\end{halfparskip}

\begin{halfparskip}
  \liturgicalhint{Eer}~--- \liturgicalhint{3x Alleluia.}~--- \liturgicalhint{Eerste vers:} Ik vlucht tot de Heer; hoe kunt gij mij zeggen:~\sep\ ``Vlieg weg als een vogel naar het gebergte.
\end{halfparskip}

\markedsubsectionrubricwithhint{Dinsdagen ``voor''.}{Marmita 9 (\Pss{25--27}).}

\begin{halfparskip}
  \psalm{\Ps{25}} Tot U verhef ik mijn ziel,~\sep\ o Heer, mijn God.~\sep\ Op U vertrouw ik; laat mij niet te schande worden;

  \qanona{Alleluia, alleluia, alleluia.}~--- \liturgicalhint{Eerste vers.}

  dat mijn vijanden niet over mij juichen!

  Want van wie op U hopen, wordt niemand beschaamd,~\sep\ maar wel worden te schande, die hun woord vermetel breken.

  Toon mij Uw wegen, o Heer,~\sep\ en leer mij Uw paden kennen.

  Leid mij in Uw waarheid, en geef mij onderricht, omdat Gij, God, mijn Redder zijt,~\sep\ en immer hoop ik op U.

  Gedenk Uw ontferming, o Heer,~\sep\ en Uw barmhartigheid, die van oudsher zijn.

  De zonden van mijn jeugd en mijn misslagen, gedenk ze niet; wees mij naar Uw erbarming indachtig,~\sep\ vanwege Uw goedheid, Heer.

  Goed en rechtvaardig is de Heer;~\sep\ daarom wijst Hij de zondaars de weg.

  De nederigen leidt Hij in gerechtigheid,~\sep\ de nederigen toont Hij Zijn weg.

  Alle wegen van de Heer zijn goedheid en trouw,~\sep\ voor wie Zijn Verbond en Zijn wetten bewaren.

  Omwille van Uw Naam, o Heer,~\sep\ vergeef mij mijn zonde, want zij is groot.

  Wie is de man, die de Heer vreest?~\sep\ Hij wijst hem de weg, die hij moet kiezen.

  Hij zelf zal in voorspoed leven,~\sep\ en zijn geslacht het land bezitten.

  De Heer is een vriend voor hen, die Hem vrezen:~\sep\ Zijn Verbond doet Hij hun kennen.

  Mijn ogen zijn immer gericht op de Heer,~\sep\ want uit de strik zal Hij mijn voeten bevrijden.

  Zie op mij neer en wees mij genadig,~\sep\ want eenzaam ben ik en ellendig.

  Verlicht de druk van mijn hart,~\sep\ en bevrijd mij van mijn angsten.

  Zie mijn ellende en mijn kwelling;~\sep\ en vergeef mij al mijn zonden.

  Let op mijn vijanden, want ze zijn talrijk,~\sep\ en haten mij met felle haat.

  Bescherm mijn leven en red mij;~\sep\ het zij mij niet tot schande, dat ik bij U mijn toevlucht zocht.

  Dat mijn onschuld en deugd mij beschermen,~\sep\ daar ik hoop op U, o Heer.

  Verlos Israël, o God,~\sep\ uit al zijn kommernissen!
\end{halfparskip}

\begin{halfparskip}
  \psalm{\Ps{26}} Heer, schaf mij recht, want ik leefde in onschuld;~\sep\ vertrouwend op de Heer, heb ik niet gewankeld.

  Onderzoek mij, Heer, en stel mij op de proef;~\sep\ doorgrond mijn nieren en mijn hart.

  Want Uw welwillendheid staat mij voor ogen,~\sep\ en ik wandel naar Uw waarheid.

  Met ongerechtigen zit ik niet neer,~\sep\ en met bedriegers kom ik niet samen.

  Ik haat het gezelschap van bozen,~\sep\ en met goddelozen zit ik niet samen.

  In onschuld was ik mijn handen,~\sep\ en ga rond Uw altaar, o Heer.

  Om openlijk Uw lof te verkondigen,~\sep\ en al Uw wonderen te verhalen.

  Heer, ik heb lief het verblijf van Uw huis,~\sep\ en de woontent van Uw heerlijkheid.

  Ruk mijn ziel niet weg met de zondaars,~\sep\ noch mijn leven met bloeddorstige mannen,

  Aan wier handen de misdaad kleeft,~\sep\ en wier rechterhand met geschenken gevuld is.

  Ik echter wandel in onschuld:~\sep\ red mij, en wees mij genadig.

  Mijn voet staat op effen baan;~\sep\ ik zal de Heer in de vergadering loven.
\end{halfparskip}

\begin{halfparskip}
  \psalm{\Ps{27}} De Heer is mijn licht en mijn heil: wie zou Ik vrezen?~\sep\ De Heer is de schuts van mijn leven: voor wie zou ik sidderen?

  Als de bozen mij bestormen om mijn vlees te verslinden,~\sep\ mijn vijanden en haters, zij struikelen en vallen.

  Al stond er een krijgsmacht tegenover mij, mijn hart zou niet vrezen;~\sep\ al brak er een oorlog tegen mij uit, dan nog zou ik vertrouwen.

  Dit alleen vraag ik de Heer, dit alleen streef ik na:~\sep\ te wonen in het huis van de Heer alle dagen van mijn leven,

  Te genieten de zoetheid van de Heer,~\sep\ en Zijn tempel te aanschouwen.

  Want in Zijn woontent zal Hij mij bergen in tijden van nood,~\sep\ Hij zal mij doen schuilen diep in Zijn tent, mij plaatsen boven op de rots.

  Nu verheft zich mijn hoofd,~\sep\ boven de vijanden, die mij omringen;

  Jubeloffers zal ik brengen in Zijn tent,~\sep\ zingen voor de Heer en spelen op de citer.

  Heer, luister naar mijn stem, waarmee ik luid roep,~\sep\ ontferm u over mij, en schenk mij verhoring!

  Tot U spreekt mijn hart, U zoeken mijn ogen;~\sep\ ik zoek Uw aanschijn, o Heer.

  Verberg mij Uw aanschijn niet,~\sep\ stoot Uw dienaar niet af in Uw toorn!

  Gij zijt mijn hulp; verwerp mij niet!~\sep\ Verlaat mij niet, o God, mijn Redder!

  Zou mijn vader en moeder mij ook verlaten,~\sep\ dan nog neemt de Heer mij op.

  Wijs mij Uw weg, o Heer,~\sep\ en leid mij op effen baan omwille van mijn weerstrevers.

  Geef mij niet prijs aan de moedwil van mijn vijanden,~\sep\ want valse getuigen en geweldenaars stonden tegen mij op.

  Ik ben er zeker van de weldaden van de Heer te zien,~\sep\ in het land der levenden.

  Zie uit naar de Heer, wees onversaagd;~\sep\ sterk zij Uw hart, zie uit naar de Heer.
\end{halfparskip}

\begin{halfparskip}
  \liturgicalhint{Eer}~--- \liturgicalhint{3x Alleluia.}~--- \liturgicalhint{Eerste vers:} Tot U verhef ik mijn ziel,~\sep\ o Heer, mijn God.~\sep\ Op U vertrouw ik; laat mij niet te schande worden.
\end{halfparskip}

\markedsubsectionrubricwithhint{Woensdagen ``voor''.}{Marmita 23 (\Pss{62--64}).}

\begin{halfparskip}
  \psalm{\Ps{62}} In God alleen rust mijn ziel,~\sep\ van Hem komt mijn heil.

  \qanona{Alleluia, alleluia, alleluia.}~--- \liturgicalhint{Eerste vers.}

  Hij alleen is mijn rots en mijn heil,~\sep\ mijn bescherming: neen, ik zal niet wankelen.

  Hoe lang stormt gij op één man los, werpt gij hem met u allen omver,~\sep\ als een hellende wand, als een
  wankelende muur?

  Ja waarlijk, zij beramen hoe zij mij van mijn verheven plaats zullen stoten;~\sep\ zij verlustigen zich in de leugens;

  Ze zegenen met hun mond,~\sep\ maar met hun hart vervloeken zij.

  Rust in God alleen, mijn ziel,~\sep\ want van Hem komt wat ik hoop.

  Hij alleen is mijn rots en mijn heil,~\sep\ mijn bescherming: neen, ik zal niet wankelen.

  Bij God is mijn heil en mijn roem,~\sep\ mijn sterke rots: in God is mijn schuilplaats.

  Hoop op Hem, o volk, ten allen tijde: stort voor Hem uw harten uit:~\sep\ God is ons tot toevlucht.

  Een ademtocht slechts zijn de kinderen der mensen,~\sep\ bedrieglijk de zonen van mannen.

  Zij rijzen op de weegschaal omhoog:~\sep\ lichter dan een zucht zijn zij allen tezamen.

  Verwacht niets van verdrukking, en roem niet ijdel op roof;~\sep\ als uw vermogen toeneemt, wilt er uw hart niet aan hechten.

  Eén zaak heeft God gezegd; deze twee dingen vernam ik: ``Bij God berust de macht, en bij U de goedheid, o Heer,~\sep\ want een ieder zult Gij naar werk vergelden.''
\end{halfparskip}

\begin{halfparskip}
  \psalm{\Ps{63}} God, mijn God zijt Gij:~\sep\ met aandrang zoek ik U.

  Naar U dorst mijn ziel, naar U smacht mijn lichaam,~\sep\ als een dor en dorstig, waterloos land.

  Zo blijf ik U beschouwen in Uw heiligdom,~\sep\ om Uw macht te zien en Uw glorie.

  Daar Uw genade meer dan het leven geldt,~\sep\ zullen mijn lippen U loven.

  Zo wil ik U prijzen mijn leven lang,~\sep\ in Uw Naam mijn handen verheffen.

  Als met merg en vet zal ik verzadigd worden,~\sep\ en met jubelende lippen zal mijn mond U loven.

  Als ik aan U denk op mijn legerstede,~\sep\ blijf ik in de nachtwaken peinzen over U.

  Want Gij zijt mijn Helper geworden,~\sep\ en ik juich in de schaduw van Uw vleugelen.

  Mijn ziel hecht zich aan U,~\sep\ Uw rechterhand is mij tot steun.

  Maar zij, die mij naar het leven staan,~\sep\ zullen verzinken in de diepten der aarde.

  Ze zullen vallen in de macht van het zwaard,~\sep\ en de prooi van vossen worden.

  Maar de koning zal zich verblijden in God; een ieder, die bij Hem zweert, zal roemen,~\sep\ omdat de mond van de lasteraars zal worden gestopt.
\end{halfparskip}

\begin{halfparskip}
  \psalm{\Ps{64}} Luister, o God, als ik klaag, naar mijn stem;~\sep\ behoed mijn leven voor de schrik van de vijand.

  Bescherm mij tegen de samenscholing van bozen,~\sep\ tegen het woelen van hen, die onrecht bedrijven.

  Die hun tongen scherpen als een zwaard,~\sep\ als pijlen hun giftige woorden richten,

  Om vanuit een schuilplaats de onschuldige te treffen,~\sep\ hem onverhoeds te treffen, zonder iets te duchten.

  Vastberaden smeden zij boze plannen, en werken samen om heimelijk strikken te spannen;~\sep\ zij zeggen: ``Wie slaat er acht op ons?''

  Zij denken misdaden uit, verbergen hun weloverwogen gedachten;~\sep\ een afgrond is hun geest en hart.

  Maar God treft hen met Zijn pijlen:~\sep\ onverhoeds worden zij met wonden geslagen,

  En hun eigen tong brengt hun verderf;~\sep\ allen, die hen zien, schudden het hoofd.

  En allen zijn vol ontzag en prijzen Gods werk,~\sep\ en overwegen Zijn daden.

  De rechtvaardige juicht in de Heer en vlucht tot Hem,~\sep\ en allen roemen, die oprecht van harte zijn.
\end{halfparskip}

\begin{halfparskip}
  \liturgicalhint{Eer}~--- \liturgicalhint{3x Alleluia.}~--- \liturgicalhint{Eerste vers:} In God alleen rust mijn ziel,~\sep\ van Hem komt mijn heil.
\end{halfparskip}

\markedsubsectionrubricwithhint{Donderdagen ``voor''.}{Marmita 38 (\Pss{96--98}).}

\begin{halfparskip}
  \psalm{\Ps{96}} Zingt een nieuw lied voor de Heer,~\sep\ zingt voor de Heer alle landen!

  \qanona{Alleluia, alleluia, alleluia.}~--- \liturgicalhint{Eerste vers.}

  Zingt voor de Heer, zegent Zijn Naam,~\sep\ verkondigt Zijn heil van dag tot dag!

  Maakt onder de heidenen Zijn glorie bekend,~\sep\ onder alle volken Zijn wonderen.

  Want groot is de Heer en hoog te prijzen,~\sep\ meer te duchten dan alle goden.

  Want de goden der heidenen zijn allen verzinsels,~\sep\ maar de Heer heeft de hemel geschapen.

  Majesteit en pracht gaan vóór Hem uit;~\sep\ macht en luister zijn in Zijn heilige woning.

  Kent toe aan de Heer, geslachten der volken, kent toe aan de Heer glorie en macht;~\sep\ kent toe aan de Heer de roem van Zijn Naam!

  Treedt met Uw offer Zijn voorhoven binnen;~\sep\ aanbidt de Heer in heilige feesttooi!

  Beef voor Zijn aanschijn, geheel de aarde;~\sep\ roept tot de volken: De Heer is Koning!

  Hij maakte de aarde onwankelbaar vast,~\sep\ heerst over de volken met billijkheid.

  Dat de hemelen juichen en de aarde jubele, laat bruisen de zee met wat ze bevat,~\sep\ laat jubelen het veld met wat er op groeit.

  Dan zullen juichen alle bomen van het woud voor het aanschijn van de Heer, want Hij komt,~\sep\ want Hij komt de aarde regeren.

  Hij zal met rechtvaardigheid de wereld regeren,~\sep\ en de volkeren volgens Zijn trouw.
\end{halfparskip}

\begin{halfparskip}
  \psalm{\Ps{97}} De Heer is Koning, dat de aarde jubele,~\sep\ dat de vele eilanden juichen!

  Wolken en duisternis omgeven Hem,~\sep\ gerechtigheid en recht zijn de steun van Zijn troon.

  Vuur gaat vóór Hem uit,~\sep\ en verbrandt Zijn vijanden om Hem heen.

  Zijn bliksems verlichten het aardrijk;~\sep\ de aarde ziet het en beeft.

  Bergen versmelten als was voor de Heer,~\sep\ voor de Beheerser van heel de aarde.

  De hemelen verkondigen Zijn gerechtigheid,~\sep\ en alle volken aanschouwen Zijn glorie.

  Beschaamd staan allen die beelden vereren, die roemen op valse goden:~\sep\ voor Hem werpen alle goden zich neer.

  Sion hoort het vol vreugde, en de steden van Juda juichen,~\sep\ om Uw oordelen, Heer.

  Want Gij, Heer, zijt verheven boven heel de aarde,~\sep\ hoogverheven onder alle goden.

  De Heer heeft lief die het kwade haten, Hij behoedt het leven van Zijn heiligen,~\sep\ en redt hen uit de hand der bozen.

  Een licht rijst op voor de rechtvaardige,~\sep\ en blijdschap voor de oprechten van hart.

  Verheugt u, rechtschapenen, in de Heer,~\sep\ en verheerlijkt Zijn heilige Naam!
\end{halfparskip}

\begin{halfparskip}
  \psalm{\Ps{98}} Zingt een nieuw lied voor de Heer,~\sep\ want wonderen heeft Hij gewrocht;

  Zege bracht Hem Zijn rechterhand,~\sep\ Zijn heilige arm.

  De Heer heeft Zijn heil doen kennen,~\sep\ voor het oog van de volken Zijn gerechtigheid getoond.

  Zijn liefde en trouw was Hij indachtig,~\sep\ ten gunste van Israëls huis.

  Alle grenzen der aarde hebben aanschouwd,~\sep\ het heil van onze God.

  Juicht voor de Heer, alle landen,~\sep\ weest blij, verheugt u en tokkelt de snaren.

  Zingt voor de Heer bij citerspel,~\sep\ en de klank van de harp,

  Met trompetten en bazuingeschal,~\sep\ jubelt voor het aanschijn van de Koning en Heer!

  De zee, met wat ze bevat, verheffe haar stem,~\sep\ de aarde en die haar bewonen;

  Dat de stromen in de handen klappen,~\sep\ en tegelijk de bergen juichen.

  Voor het aanschijn van de Heer, want Hij komt,~\sep\ want Hij komt de aarde regeren.

  Hij zal met rechtvaardigheid de wereld regeren,~\sep\ en de volkeren met billijkheid.
\end{halfparskip}

\begin{halfparskip}
  \liturgicalhint{Eer}~--- \liturgicalhint{3x Alleluia.}~--- \liturgicalhint{Eerste vers:} Zingt een nieuw lied voor de Heer,~\sep\ zingt voor de Heer alle landen!
\end{halfparskip}

\markedsubsectionrubricwithhint{Vrijdagen ``voor''.}{Marmita 33 (\Pss{85--86}).}

\begin{halfparskip}
  \psalm{\Ps{85}} Gij zijt Uw land genadig geweest, o Heer,~\sep\ hebt het lot van Jacob ten goede gekeerd.

  \qanona{Alleluia, alleluia, alleluia.}~--- \liturgicalhint{Eerste vers.}

  Vergeven hebt Gij de schuld van Uw volk,~\sep\ en al zijn zonden bedekt.

  Uw gramschap hebt Gij geheel bedwongen,~\sep\ de gloed van Uw toorn gestild.

  Herstel ons, o God, onze Redder,~\sep\ en leg Uw wrevel tegen ons af.

  Zult Gij dan eeuwig tegen ons toornen,~\sep\ of verbolgen blijven van geslacht tot geslacht?

  Zult Gij ons dan niet opnieuw doen leven,~\sep\ opdat Uw volk zich verblijde in U?

  Toon ons, Heer, Uw barmhartigheid,~\sep\ en schenk ons Uw heil!

  Ik wil horen naar wat de Heer God spreekt:~\sep\ vrede voorzeker kondigt Hij aan.

  Voor Zijn volk en Zijn heiligen,~\sep\ en voor hen, die zich van harte keren tot Hem.

  Ja waarlijk, Zijn heil is nabij voor wie Hem vrezen,~\sep\ en zo zal er glorie wonen in ons land:

  Barmhartigheid en trouw zullen elkander ontmoeten,~\sep\ gerechtigheid en vrede elkander de kus geven.

  Trouw zal aan de aarde ontspruiten,~\sep\ en gerechtigheid neerzien vanuit de hemel.

  De Heer zelf zal zegen schenken,~\sep\ en ons land zijn vruchten geven.

  Gerechtigheid zal vóór Hem uitgaan,~\sep\ en heil zijn schreden volgen.
\end{halfparskip}

\begin{halfparskip}
  \psalm{\Ps{86}} Neig Uw oor, o Heer; verhoor mij,~\sep\ want ik ben ellendig en arm.

  Bescherm mij, want ik ben U toegewijd;~\sep\ red Uw dienaar, die op U hoopt.

  Mijn God zijt Gij; wees mij genadig, o Heer,~\sep\ want almaar door roep ik tot U.

  Verblijd de ziel van Uw dienaar,~\sep\ want tot U, o Heer, verhef ik mijn ziel.

  Want Gij, o Heer, zijt goed en genadig,~\sep\ vol erbarming voor al wie U aanroept.

  Luister, Heer, naar mijn bede,~\sep\ en geef acht op de stem van mijn smeken.

  Op de dag van mijn kwelling riep ik tot U,~\sep\ omdat Gij mij verhoren zult.

  Onder de goden, o Heer, is er geen als Gij,~\sep\ en geen werk is gelijk aan het Uwe.

  Alle volken, door U geschapen, zullen komen, en U aanbidden, O Heer,~\sep\ en verheerlijken Uw Naam.

  Want Gij zijt groot en Gij doet wonderwerken:~\sep\ Gij zijt God, en Gij alleen.

  Toon mij Uw weg, o Heer, opdat ik wandele in Uw waarheid,~\sep\ richt mijn hart op de vrees voor Uw Naam.

  Ik zal U prijzen, Heer, mijn God, uit heel mijn hart:~\sep\ en eeuwig Uw Naam verheerlijken.

  Want Uw erbarming voor mij was groot,~\sep\ en uit de diepten van het dodenrijk hebt Gij mij opgehaald.

  Trotsen, o God, zijn tegen mij opgestaan, een bende geweldenaars staat mij naar het leven,~\sep\ zij houden U niet voor ogen.

  Maar Gij, o Heer, zijt een barmhartige en liefdevolle God,~\sep\ lankmoedig, rijk aan ontferming en trouw.

  Blik op mij neer en wees mij genadig;~\sep\ schenk aan Uw dienaar Uw kracht, en red de zoon van Uw
  dienstmaagd.

  Geef mij een teken van Uw gunst, opdat die mij haten, Heer, vol schaamte zien,~\sep\ dat Gij, o Heer, mij hulp en troost hebt geschonken.
\end{halfparskip}

\begin{halfparskip}
  \liturgicalhint{Eer}~--- \liturgicalhint{3x Alleluia.}~--- \liturgicalhint{Eerste vers:} Gij zijt Uw land genadig geweest, o Heer,~\sep\ hebt het lot van Jacob ten goede gekeerd.
\end{halfparskip}

\markedsubsectionrubricwithhint{Vrijdagen ``midden''.}{Marmita 33 of 34 (\Pss{85--86} of 87--88).}

\markedsubsectionrubricwithhint{Zaterdagen ``voor''.}{Marmita 56 (\Pss{144--146}).}

\begin{halfparskip}
  \psalm{\Ps{144}} Ik wil U roemen, mijn God, de Koning,~\sep\ en prijzen Uw Naam in eeuwen der eeuwen.

  \qanona{Alleluia, alleluia, alleluia.}~--- \liturgicalhint{Eerste vers.}

  Dag aan dag wil ik U prijzen,~\sep\ en loven Uw Naam in de eeuwen der eeuwen.

  Groot is de Heer en hoogst lofwaardig,~\sep\ en Zijn grootheid is niet te meten.

  Het ene geslacht roemt bij het andere Uw werken,~\sep\ en verkondigt Uw macht.

  Zij spreken over de heerlijke luister van Uw majesteit,~\sep\ en maken Uw wonderwerken bekend.

  Zij bezingen de kracht van Uw ontzagwekkende daden,~\sep\ en verhalen Uw grootheid.

  Zij verkondigen de lof van Uw grote goedheid,~\sep\ en juichen om Uw gerechtigheid.

  Zachtmoedig en barmhartig is de Heer,~\sep\ lankmoedig en genaderijk.

  Goed is de Heer voor allen,~\sep\ en barmhartig voor al Zijn werken.

  Dat al Uw werken U roemen, Heer,~\sep\ en dat Uw heiligen U prijzen,

  Dat zij de glorie van Uw rijk bezingen,~\sep\ en Uw macht verkondigen,

  Om aan de zonen der mensen Uw macht te doen kennen,~\sep\ en de roem van Uw luistervol rijk.

  Uw rijk is een rijk van alle eeuwen,~\sep\ en Uw heerschappij duurt voort door alle geslachten heen.

  Trouw is de Heer in al Zijn woorden,~\sep\ en heilig in al Zijn werken.

  De Heer steunt allen, die dreigen te vallen,~\sep\ en alle terneergedrukten beurt Hij op.

  De ogen van allen zien vertrouwvol naar U,~\sep\ en Gij schenkt hun spijs te bekwamen tijde,

  Gij opent Uw hand,~\sep\ en verzadigt vol goedheid al wat leeft.

  Rechtvaardig is de Heer in al Zijn wegen,~\sep\ en heilig in al Zijn werken.

  De Heer is allen, die Hem aanroepen, nabij,~\sep\ allen, die Hem aanroepen in oprechtheid.

  Hij zal de wensen vervullen van hen, die Hem vrezen,~\sep\ hun smeken aanhoren en hen redden.

  De Heer behoedt allen, die Hem beminnen,~\sep\ maar alle bozen zal Hij verdelgen.

  Laat mijn mond verkondigen de lof van de Heer,~\sep\ en alle vlees prijzen zijn heilige Naam in alle eeuwen der eeuwen.
\end{halfparskip}

\begin{halfparskip}
  \psalm{\Ps{145}} Loof, mijn ziel, de Heer! Loven zal ik de Heer mijn leven lang,~\sep\ mijn God bezingen zolang ik besta.

  Vertrouwt niet op vorsten,~\sep\ niet op een mens, door wie geen redding komt.

  Als zijn geest is geweken, keert hij terug tot zijn stof;~\sep\ dan vallen al zijn plannen in duigen.

  Gelukkig hij, die Jacobs God tot helper heeft,~\sep\ wiens hoop is gevestigd op de Heer, zijn God,

  Die hemel en aarde gemaakt heeft,~\sep\ en de zee met al wat zij bevatten,

  Die trouw blijft voor eeuwig, recht schaft aan de verdrukten,~\sep\ en brood aan de hongerigen schenkt.

  De Heer bevrijdt de gevangenen,~\sep\ de Heer opent blinden de ogen;

  De Heer richt gebukten op,~\sep\ de Heer heeft rechtvaardigen lief.

  De Heer behoedt de vreemden, is voor wees en weduwe een steun;~\sep\ maar de weg der bozen verstoort Hij.

  De Heer zal heersen in eeuwigheid,~\sep\ uw God, o Sion, van geslacht tot geslacht.
\end{halfparskip}

\begin{halfparskip}
  \psalm{\Ps{146}} Looft de Heer want Hij is goed, bezingt onze God want Hij is liefelijk:~\sep\ Hem past lofbetuiging.

  De Heer bouwt Jeruzalem op,~\sep\ verzamelt de verstrooiden van Israël.

  Hij geneest de gebrokenen van harte,~\sep\ en verbindt hun wonden.

  Hij bepaalt het getal der sterren,~\sep\ Hij noemt ze elk bij hun naam.

  Groot is onze Heer en machtig in kracht;~\sep\ Zijn wijsheid kent geen grenzen.

  De Heer heft de nederigen op,~\sep\ drukt de bozen neer in het stof.

  Zingt de Heer een danklied toe,~\sep\ speelt voor onze God op de citer,

  Die de hemel bedekt met wolken,~\sep\ voor de aarde de regen bereidt,

  Die op de bergen het gras doet ontspruiten,~\sep\ en kruid ten dienste der mensen,

  Die voedsel schenkt aan het vee,~\sep\ aan de jongen der raven, die roepen tot Hem.

  Niet in de kracht van het ros heeft Hij behagen,~\sep\ noch in de schenkels van de man Zijn welgevallen.

  Behagen heeft de Heer in hen, die Hem vrezen,~\sep\ en op Zijn goedheid vertrouwen.
\end{halfparskip}

\begin{halfparskip}
  \liturgicalhint{Eer}~--- \liturgicalhint{3x Alleluia.}~--- \liturgicalhint{Eerste vers:} Ik wil U roemen, mijn God, de Koning,~\sep\ en prijzen Uw Naam in eeuwen der eeuwen.
\end{halfparskip}

\markedsubsectionrubricwithhint{Maandagen ``na''.}{Marmita 5 (\Pss{15--17}).}

\begin{halfparskip}
  \psalm{\Ps{15}} Heer, wie mag in Uw tent verblijven,~\sep\ wie wonen op Uw heilige berg?

  \qanona{Alleluia, alleluia, alleluia.}~--- \liturgicalhint{Eerste vers.}

  Die vlekkeloos wandelt en deugdzaam leeft, en in zijn hart wat goed is denkt,~\sep\ en niet lastert met zijn tong;

  Die zijn evenmens geen kwaad berokkent,~\sep\ en zijn nabuur geen smaad aandoet;

  Die de boze voor verachtelijk houdt,~\sep\ maar eert die vrezen de Heer;

  Die aan een schadelijke eed niet tornt, zijn geld niet uitleent met woeker,~\sep\ en onschuldigen ten koste geen steekpenning aanvaardt.

  Wie zo handelt,~\sep\ zal niet wankelen in eeuwigheid.
\end{halfparskip}

\begin{halfparskip}
  \psalm{\Ps{16}} Bewaar mij, God, want ik vlucht tot U.~\sep\ Ik zeg tot de Heer: Mijn Heer zijt Gij; geen geluk voor mij zonder U.

  Hoe schonk Hij mij voor de heiligen in Zijn land,~\sep\ een wondergrote liefde!

  Zij vermeerderen hun smarten,~\sep\ die achter vreemde goden lopen.

  Ik zal niet delen in hun plengoffers van bloed,~\sep\ noch hun namen op mijn lippen nemen.

  De Heer is mijn erfdeel, de dronk van mijn beker;~\sep\ Gij zijt het, die mijn lot in handen houdt.

  Voor mij viel het meetsnoer op heerlijke velden;~\sep\ ja, mijn erfdeel behaagt mij ten volle.

  Ik prijs de Heer, daar Hij mij inzicht gaf,~\sep\ en zelfs 's~nachts mijn hart vermaant.

  De Heer houd ik immer voor ogen;~\sep\ omdat Hij staat aan mijn rechterzijde, zal ik niet wankelen.

  Daarom verheugt zich mijn hart en jubelt mijn ziel,~\sep\ zelfs mijn vlees zal in veiligheid rusten.

  Want mijn ziel zult Gij niet in het dodenrijk laten,~\sep\ Uw heilige het bederf niet doen zien.

  Gij zult de weg naar het leven mij tonen, overvloedige vreugden bij U,~\sep\ en geneugten voor eeuwig aan Uw rechterhand.
\end{halfparskip}

\begin{halfparskip}
  \psalm{\Ps{17}} Luister, Heer, naar een rechtvaardige zaak, geef acht op mijn geroep,~\sep\ hoor de bede van argeloze lippen.

  Van Uw aanschijn ga over mij het oordeel uit:~\sep\ Uw ogen zien wat recht is.

  Peil mijn hart, doorvors het 's~nachts, beproef mij met vuur,~\sep\ geen onrecht zult Gij in mij vinden.

  Mijn mond misdeed niet zoals mensen gewoon zijn;~\sep\ naar de woorden van Uw lippen heb ik de wegen der Wet gevolgd.

  Vast drukten mijn schreden Uw paden,~\sep\ mijn voeten struikelden niet.

  Ik roep U aan, o God, want Gij zult mij verhoren;~\sep\ neig Uw oor naar mij en luister naar mijn bede!

  Toon U wonderbaar in Uw erbarmen,~\sep\ Gij die redt van weerstrevers al wie aan Uw zijde zijn toevlucht zoekt.

  Behoed me als de appel van het oog, verberg me in de schaduw van Uw vleugels,~\sep\ voor de zondaars die me geweld aandoen.

  Mijn vijanden omringen mij woedend; zij sluiten hun zinnelijk hart,~\sep\ hun mond spreekt trotse woorden.

  Hun schreden omringen mij thans;~\sep\ zij loeren om mij ter aarde te werpen.

  Ze zijn als de leeuw, die de muil spert naar prooi,~\sep\ als een leeuwenwelp, die in hinderlaag ligt.

  Rijs op, Heer, hem tegemoet en vel hem terneer, red mij door Uw zwaard van de boze,~\sep\ door Uw hand van mensen, o Heer,

  Van mensen, wier deel dit leven is,~\sep\ en wier schoot Gij vult met Uw schatten;

  Wier zonen zich verzadigen,~\sep\ en wat hun overblijft aan hun kinderen achterlaten.

  Ik echter zal door gerechtigheid Uw aanschijn aanschouwen,~\sep\ en mij bij het ontwaken met Uw aanblik verzadigen.
\end{halfparskip}

\begin{halfparskip}
  \liturgicalhint{Eer}~--- \liturgicalhint{3x Alleluia.}~--- \liturgicalhint{Eerste vers:} Heer, wie mag in Uw tent verblijven,~\sep\ wie wonen op Uw heilige berg?
\end{halfparskip}

\markedsubsectionrubricwithhint{Dinsdagen ``na''.}{Marmita 10 (\Pss{28--30}).}

\begin{halfparskip}
  \psalm{\Ps{28}} Tot U roep ik, o Heer;~\sep\ mijn Rots, wees niet doof voor mij.

  \qanona{Alleluia, alleluia, alleluia.}~--- \liturgicalhint{Eerste vers.}

  Opdat ik niet, als Gij niet hoort naar mij,~\sep\ gelijk worde aan hen, die in de grafkuil dalen.

  Hoor de stem van mijn smeken, nu ik roep tot U,~\sep\ nu ik mijn handen ophef naar Uw heilige tempel.

  Ruk mij niet weg met de zondaars,~\sep\ met hen, die kwaad bedrijven,

  Die vriendelijk spreken met hun naaste,~\sep\ maar in hun hart kwade bedoelingen koesteren.

  Handel met hen naar hun daden,~\sep\ en naar de boosheid van hun werken.

  Zet hun het werk van hun handen betaald,~\sep\ vergeld ze hun daden.

  Want ze slaan geen acht op de daden van de Heer en het werk van Zijn handen;~\sep\ Hij richte hen te gronde en heffe hen niet op.

  Gezegend de Heer, want Hij hoorde mijn dringende bede;~\sep\ de Heer, mijn kracht en mijn schild,

  Op Hem vertrouwde mijn hart, en ik ben geholpen;~\sep\ daarom jubelt mijn hart en prijs ik Hem met mijn zang.

  De Heer is een kracht voor Zijn volk,~\sep\ en voor Zijn Gezalfde een heilzame schutse.

  Red Uw volk en zegen Uw erfdeel;~\sep\ weid hen en draag hen voor eeuwig.
\end{halfparskip}

\begin{halfparskip}
  \psalm{\Ps{29}} Kent toe aan de Heer, zonen van God,~\sep\ kent toe aan de Heer glorie en macht!

  Kent toe aan de Heer de roem van Zijn Naam,~\sep\ aanbidt de Heer in heilige feesttooi.

  De stem van de Heer over de wateren! De God van majesteit doet de donder rollen:~\sep\ de Heer over de wijde wateren!

  De stem van de Heer vol kracht,~\sep\ de stem van de Heer vol majesteit!

  De stem van de Heer verbrijzelt de ceders,~\sep\ de Heer verbrijzelt de ceders van de Libanon.

  Hij doet de Libanon opspringen als een kalf,~\sep\ en de Sarion als het jong van een buffel.

  De stem van de Heer schiet vlammende schichten, de stem van de Heer doet de wildernis beven,~\sep\ de Heer doet Cades' wildernis beven.

  De stem van de Heer buigt eiken krom en ontschorst de bomen der wouden:~\sep\ en in Zijn tempel roepen allen: Glorie!

  De Heer troonde boven de watervloed,~\sep\ en de Heer zal tronen als Koning voor eeuwig.

  De Heer zal sterkte schenken aan Zijn volk,~\sep\ de Heer zal Zijn volk met vrede zegenen.
\end{halfparskip}

\begin{halfparskip}
  \psalm{\Ps{30}} Ik wil U roemen, o Heer, daar Gij mij gered hebt,~\sep\ en niet mijn vijanden over mij liet juichen.

  Heer, mijn God, ik riep tot U,~\sep\ en Gij hebt mij genezen.

  Heer, uit het dodenrijk hebt Gij mij weggevoerd,~\sep\ mij gered uit hen, die ten grave dalen.

  Speelt op de citer voor de Heer, gij, Zijn heiligen;~\sep\ en dankt Zijn heilige Naam.

  Want Zijn toorn duurt slechts een ogenblik,~\sep\ maar Zijn welwillendheid het hele leven door.

  's~Avonds komt er geween te gast,~\sep\ maar 's~morgens is er gejubel.

  In overmoed nu heb ik gezegd:~\sep\ ``In eeuwigheid zal ik niet wankelen.''

  Het was Uw gunst, o Heer, die mij ere schonk en macht;~\sep\ maar toen Gij Uw aanschijn verborgen hieldt, werd ik ontsteld.

  Ik roep tot U, o Heer,~\sep\ en smeek bij mijn God om erbarming:

  ``Wat kan mijn bloed U baten,~\sep\ of mijn neerdalen in het graf?

  Zal het stof U soms prijzen,~\sep\ of roemen Uw trouw?''

  Luister, o Heer, en wees mij genadig;~\sep\ o Heer, wees toch mijn helper!

  Gij hebt mijn rouw in een reidans veranderd,~\sep\ mijn rouwkleed verscheurd, mij met vreugde omgord,

  Opdat mijn ziel U zou prijzen en nimmermeer zwijgen.~\sep\ Heer, mijn God, ik zal U loven voor eeuwig!
\end{halfparskip}

\begin{halfparskip}
  \liturgicalhint{Eer}~--- \liturgicalhint{3x Alleluia.}~--- \liturgicalhint{Eerste vers:} Tot U roep ik, o Heer;~\sep\ mijn Rots, wees niet doof voor mij.
\end{halfparskip}

\markedsubsectionrubricwithhint{Woensdagen ``na''.}{Marmita 24 (\Pss{65--67}).}

\begin{halfparskip}
  \psalm{\Ps{65}} Aan U, o God, komt een lofzang toe in Sion;~\sep\ men volbrenge zijn gelofte aan U, die de bede verhoort.

  \qanona{Alleluia, alleluia, alleluia.}~--- \liturgicalhint{Eerste vers.}

  Tot U komt alle vlees,~\sep\ omwille der ongerechtigheden.

  Onze misdaden drukken ons neer:~\sep\ Gij scheldt ze kwijt.

  Gelukkig die Gij uitkiest en tot U neemt:~\sep\ hij woont in Uw voorhoven.

  Dat wij verzadigd worden met de goederen van Uw huis,~\sep\ met de heiligheid van Uw tempel.

  Met gerechtigheid verhoort Gij ons door wondere tekenen,~\sep\ God, onze Redder,

  Gij zijt de hoop van alle grenzen der aarde,~\sep\ en van de verre zeeën;

  Die de bergen vastlegt door Uw kracht,~\sep\ die met macht zijt omgord,

  Die het bulderen der zee bedwingt,~\sep\ het bulderen van haar golven en het woelen der naties.

  En die de grenzen der aarde bewonen, huiveren om Uw tekenen:~\sep\ met vreugde vervult Gij het uiterste oosten en westen.

  Gij hebt de aarde bezocht en haar besproeid,~\sep\ met rijkdommen haar overstelpt.

  De stroom van God is met water gevuld: Gij hebt hun graan bereid;~\sep\ zo hebt Gij haar gereed gemaakt:

  Haar voren hebt Gij besproeid,~\sep\ geëffend haar kluiten,

  Door stortregens hebt Gij haar geweekt,~\sep\ en haar gewas gezegend.

  Met Uw mildheid hebt Gij het jaar gekroond,~\sep\ en Uw wegen druipen van vet.

  De weiden der woestijn druipen ervan,~\sep\ en de heuvelen omgorden zich met jubel.

  De weiden zijn met kudden bekleed en de dalen met koren bedekt:~\sep\ zij juichen U toe en zingen!
\end{halfparskip}

\begin{halfparskip}
  \psalm{\Ps{66}} Juicht God toe, alle landen, bezingt de glorie van zijn Naam,~\sep\ heft voor Hem een heerlijk loflied aan.

  Zegt tot God: Hoe ontzagwekkend zijn Uw werken!~\sep\ Om Uw geweldige kracht brengen Uw vijanden U vleiend hulde.

  Dat heel de aarde U aanbidde en voor U zinge,~\sep\ dat zij bezinge Uw Naam.

  Komt en ziet de werken van God:~\sep\ wondere daden volbracht Hij onder de kinderen der mensen!

  De zee veranderde Hij in land, en zij trokken te voet door de stroom;~\sep\ laten wij daarom over Hem ons verheugen.

  Eeuwig heerst Hij door Zijn macht; Zijn ogen slaan de volken gade:~\sep\ dat de weerspannigen zich niet verheffen.

  Zegent, gij volken, onze God,~\sep\ en verkondigt Zijn wijd verbreide lof.

  Hij behield ons in leven,~\sep\ en liet onze voeten niet wankelen.

  Want Gij hebt ons beproefd, o God,~\sep\ met vuur ons gelouterd, zoals men zilver loutert.

  Gij liet ons de strik inlopen,~\sep\ een zware last hebt Gij op onze heupen gelegd.

  Mensen liet Gij over onze hoofden treden; wij zijn door vuur en water gegaan,~\sep\ maar Gij hebt ons uitkomst gebracht.

  Met brandoffers wil ik Uw huis betreden,~\sep\ U mijn geloften inlossen,

  Die mijn lippen hebben uitgesproken,~\sep\ en mijn mond heeft beloofd in mijn kwelling.

  Brandoffers wil ik U brengen van vette schapen met het vet van rammen;~\sep\ runderen en bokken zal ik offeren.

  Komt en hoort, gij allen, die God vreest,~\sep\ ik wil U verhalen hoe grote dingen Hij aan mij gedaan heeft!

  Ik riep Hem aan met mijn mond,~\sep\ en prees Hem met mijn tong.

  Had ik in mijn hart op boosheid gezonnen,~\sep\ dan had de Heer mij niet verhoord.

  Maar God heeft mij verhoord,~\sep\ heeft gelet op de stem van mijn smeken.

  Gezegend zij God, die mijn bede niet heeft versmaad,~\sep\ mij Zijn ontferming niet heeft onthouden.
\end{halfparskip}

\begin{halfparskip}
  \psalm{\Ps{67}} God zij ons genadig en zegene ons;~\sep\ Hij tone ons Zijn vredig gelaat.

  Opdat men op aarde Zijn weg lere kennen,~\sep\ onder alle volken Zijn heil.

  Dat de volken U prijzen, o God,~\sep\ dat alle volken U prijzen!

  Laat juichen en jubelen de naties, omdat Gij met rechtvaardigheid de volken regeert,~\sep\ en de naties op aarde bestuurt.

  Dat de volken U prijzen, o God,~\sep\ dat alle volken U prijzen!

  De aarde heeft haar vrucht gegeven;~\sep\ God, onze God, heeft ons gezegend.

  Dat God ons zegene,~\sep\ en dat alle grenzen der aarde Hem vrezen!
\end{halfparskip}

\begin{halfparskip}
  \liturgicalhint{Eer}~--- \liturgicalhint{3x Alleluia.}~--- \liturgicalhint{Eerste vers:} Aan U, o God, komt een lofzang toe in Sion;~\sep\ men volbrenge zijn gelofte aan
  U, die de bede verhoort.
\end{halfparskip}

\markedsubsectionrubricwithhint{Donderdagen ``na''.}{Marmita 39* (\Pss{99--101}).}

\begin{halfparskip}
  \psalm{\Ps{99}} De Heer is Koning: de volken sidderen;~\sep\ Hij troont op de cherubs: de aarde beeft.

  \qanona{Alleluia, alleluia, alleluia.}~--- \liturgicalhint{Eerste vers.}

  Groot is de Heer op de Sion,~\sep\ en boven alle volken verheven.

  Dat zij prijzen Uw grote en ontzagwekkende Naam:~\sep\ Hij toch is heilig.

  De Machtige heerst, die rechtvaardigheid bemint: wat recht is, hebt Gij gegrondvest,~\sep\ Gij handhaaft in Jacob gerechtigheid en recht.

  Prijst de Heer, onze God, en werpt u neer voor Zijn voetbank:~\sep\ Hij toch is heilig.

  Moses en Aäron zijn onder Zijn priesters, en Samuël onder hen die Zijn Naam aanriepen;~\sep\ zij riepen tot de Heer, en Hij schonk hun verhoring.

  In een wolkkolom sprak Hij tot hen:~\sep\ zij luisterden naar Zijn geboden, en naar de wet, die Hij hun gaf.

  Heer, onze God, Gij hebt hen verhoord;~\sep\ Gij waart hun genadig, o God, maar hun fouten hebt Gij gestraft.

  Prijst de Heer, onze God, en werpt u neer voor Zijn heilige berg,~\sep\ want heilig is de Heer, onze God.
\end{halfparskip}

\begin{halfparskip}
  \psalm{\Ps{100}} Juicht voor de Heer, alle landen,~\sep\ dient de Heer met vreugde;

  Treedt voor Zijn aanschijn~\sep\ met gejubel!

  Weet het wel: de Heer is God; Hij heeft ons gemaakt, Hem behoren we toe,~\sep\ Zijn volk zijn wij en de schapen van Zijn weide.

  Treedt Zijn poorten met lofzang binnen, Zijn voorhoven met jubelzang;~\sep\ brengt Hem hulde en zegent Zijn Naam.

  Want goed is de Heer: eeuwig duurt Zijn barmhartigheid,~\sep\ en Zijn trouw van geslacht tot geslacht.
\end{halfparskip}

\begin{halfparskip}
  \psalm{\Ps{101}} Van liefde en recht wil ik zingen,~\sep\ voor U de citer bespelen, o Heer.

  Op de weg der onschuld zal ik wandelen;~\sep\ wanneer zult Gij tot mij komen?

  Rein van hart wil ik leven,~\sep\ binnen mijn huis.

  Mijn ogen zal ik niet vestigen,~\sep\ op ongerechtigheid;

  Wie onrecht pleegt, is mij een gruwel:~\sep\ hij zal met mij geen omgang hebben.

  Een bedorven hart blijft verre van mij;~\sep\ van kwaad wil ik niets weten.

  Wie heimelijk zijn naaste belastert,~\sep\ die zal ik te gronde richten.

  De trotse blik en het hovaardig hart,~\sep\ zal ik niet dulden.

  Mijn ogen zien uit naar de getrouwen in het land,~\sep\ opdat ze bij mij wonen.

  Wie wandelt langs de goede weg,~\sep\ die zal mijn dienaar zijn.

  In mijn huis zal niet verblijven,~\sep\ die aan bedrog zich schuldig maakt.

  Wie leugens spreekt,~\sep\ houdt het niet uit onder mijn ogen.

  Dag aan dag zal ik verdelgen,~\sep\ alle boosdoeners in het land,

  Verbannen uit de stad van de Heer,~\sep\ allen die kwaad bedrijven.
\end{halfparskip}

\begin{halfparskip}
  \liturgicalhint{Eer}~--- \liturgicalhint{3x Alleluia.}~--- \liturgicalhint{Eerste vers:} De Heer is Koning: de volken sidderen; Hij troont op de cherubs: de aarde beeft.
\end{halfparskip}

\markedsubsectionrubricwithhint{Vrijdagen ``na''.}{Marmita 34 (\Pss{87--88}).}

\begin{halfparskip}
  \psalm{\Ps{87}} Zijn stichting op de heilige bergen bemint de Heer:~\sep\ de poorten van Sion boven alle tenten van Jacob.

  \qanona{Alleluia, alleluia, alleluia.}~--- \liturgicalhint{Eerste vers.}

  Roemrijke dingen verhaalt men van u,~\sep\ O stad van God!

  Rahab en Babel zal Ik tot Mijn vereerders rekenen:~\sep\ zie, Filistea en Tyrus en het volk der Ethiopiërs: daar zijn ze geboren!

  Over Sion zal men zeggen: ``Allen, man voor man, zijn in haar geboren,~\sep\ en de Allerhoogste zelf heeft haar bevestigd.''

  De Heer zal schrijven in het boek der volken:~\sep\ ``Daar zijn ze geboren.''

  En in reidans zullen zij zingen:~\sep\ ``Al mijn bronnen zijn in u.''
\end{halfparskip}

\begin{halfparskip}
  \psalm{\Ps{88}} Heer, mijn God, ik roep overdag,~\sep\ en ik jammer 's~nachts voor Uw aanschijn.

  Dringe mijn bede toch door tot U,~\sep\ neig Uw oor naar mijn klagen!

  Want mijn ziel is verzadigd met rampen,~\sep\ mijn leven is het dodenrijk nabij.

  Ik word gerekend onder hen, die ten grave dalen,~\sep\ ik ben als een man zonder kracht.

  Onder de doden is mijn legerstede,~\sep\ als van verslagenen, die liggen in het graf,

  Aan wie Gij niet meer denkt,~\sep\ die aan Uw zorgen zijn onttrokken.

  In een diepe groeve hebt Gij mij neergelegd,~\sep\ in duisternis, in een diep ravijn.

  Uw verontwaardiging drukt zwaar op mij,~\sep\ met al Uw golven slaat Gij mij neer.

  Gij hebt mijn vrienden van mij vervreemd, mij tot afschuw voor hen gemaakt;~\sep\ ik zit gevangen, en kan niet ontkomen.

  Van ellende verkwijnen mijn ogen; iedere dag roep ik tot U, o Heer,~\sep\ naar U strek ik mijn handen uit.

  Of doet Gij voor doden nog wonderen,~\sep\ of zullen gestorvenen, herrijzend, U loven?

  Of wordt Uw goedheid in het graf verkondigd,~\sep\ Uw trouw in het dodenrijk?

  Openbaart men in het duister Uw wonderen,~\sep\ in het land der vergetelheid Uw genade?

  Ik echter roep tot U, o Heer,~\sep\ mijn bede stijgt tot U op in de morgen.

  Waarom toch, o Heer, verstoot Gij mij,~\sep\ verbergt Gij voor mij Uw gelaat?

  Van jongsaf ben ik ellendig en stervend,~\sep\ ik torste Uw verschrikkingen en kwijnde.

  Uw toorn is over mij heengegaan,~\sep\ Uw verschrikkingen sloegen mij neer.

  Zij omgeven mij immer als water,~\sep\ omringen mij alle tezamen.

  Vriend en makker hebt Gij van mij vervreemd,~\sep\ mijn vertrouweling is de duisternis.
\end{halfparskip}

\begin{halfparskip}
  \liturgicalhint{Eer}~--- \liturgicalhint{3x Alleluia.}~--- \liturgicalhint{Eerste vers:} Zijn stichting op de heilige bergen bemint de Heer:~\sep\ de poorten van Sion
  boven alle tenten van Jacob.
\end{halfparskip}

\markedsubsectionrubricwithhint{Zaterdagen ``na''.}{Marmita 57 (\Pss{147:12--150}).}

\begin{halfparskip}
  \psalm{\Ps{147}} Loof, Jeruzalem, de Heer,~\sep\ loof Uw God, o Sion.

  \qanona{Alleluia, alleluia, alleluia.}~--- \liturgicalhint{Eerste vers.}

  Want hecht heeft Hij gemaakt de grendels van uw poorten,~\sep\ uw zonen in u gezegend.

  Hij schonk vrede aan uw gebied,~\sep\ verzadigt u met bloem van tarwe.

  Hij zendt Zijn bevel naar de aarde,~\sep\ haastig ijlt Zijn uitspraak heen.

  De sneeuw doet Hij vallen als wol,~\sep\ de rijp spreidt Hij uit als as.

  Zijn ijs werpt Hij neer als kruimels brood,~\sep\ voor Zijn koude stollen de wateren.

  Hij geeft Zijn bevel en doet ze weer smelten;~\sep\ de wind doet Hij waaien, en de wateren stromen.

  Hij maakte Zijn gebod aan Jacob bekend,~\sep\ aan Israël Zijn wetten en bevelen.

  Zo deed Hij voor geen ander volk,~\sep\ hun openbaarde Hij Zijn wetten niet.
\end{halfparskip}

\begin{halfparskip}
  \psalm{\Ps{148}} Looft de Heer in de hemel,~\sep\ looft Hem in de hoge.

  Looft Hem al Zijn engelen,~\sep\ looft Hem al Zijn legerscharen!

  Looft Hem zon en maan,~\sep\ looft Hem alle fonkelende sterren!

  Looft Hem hemelen der hemelen,~\sep\ en gij wateren boven de hemel!

  Dat zij de Naam van de Heer loven,~\sep\ want Hij gebood, en ze waren geschapen;

  En Hij heeft ze gevestigd voor immer en eeuwig:~\sep\ Hij gaf een wet, die niet zal vergaan.

  Looft de Heer op aarde,~\sep\ monsters en alle diepten der zee,

  Vuur, hagel, sneeuw en nevel,~\sep\ stormwind, die Zijn bevel volbrengt,

  Bergen en alle heuvelen,~\sep\ vruchtbomen en alle ceders,

  Wilde dieren en alle vee,~\sep\ kruipende dieren en gevleugelde vogels,

  Koningen der aarde en alle volkeren,~\sep\ vorsten en alle rechters op aarde,

  Jongelingen zowel als maagden,~\sep\ grijsaards tezamen met kinderen:

  Dat zij de Naam van de Heer prijzen,~\sep\ want Zijn Naam alleen is verheven;

  Zijn majesteit gaat aarde en hemel te boven,~\sep\ en Hij heeft de hoorn van zijn volk verheven.

  Hij is de roem van al Zijn getrouwen,~\sep\ van de zonen van Israël, van het volk, dat Hem zo na is.
\end{halfparskip}

\begin{halfparskip}
  \psalm{\Ps{149}} Zingt een nieuw lied voor de Heer,~\sep\ Zijn lof weerklinke in de kring der heiligen!

  Laat Israël zich over zijn Schepper verheugen,~\sep\ de zonen van Sion jubelen over hun Koning.

  Dat zij Zijn Naam met reidans loven,~\sep\ Hem bezingen bij pauk en citer,

  Want de Heer bemint Zijn volk,~\sep\ en de nederigen kroont Hij met zege.

  Dat de heiligen juichen om de glorie,~\sep\ zich op hun legersteden verblijden!

  Gods lof zij in hun mond,~\sep\ en het tweesnijdend zwaard in hun hand,

  Om zich te wreken op de heidenen,~\sep\ om de volken te tuchtigen;

  Om hun koningen in ketens te slaan,~\sep\ en hun prinsen in ijzeren boeien,

  Om het vastgestelde vonnis aan hen te voltrekken:~\sep\ dat is de roem van al Zijn heiligen.
\end{halfparskip}

\begin{halfparskip}
  \psalm{\Ps{150}} Looft de Heer in Zijn heiligdom,~\sep\ looft Hem in Zijn verheven firmament.

  Looft Hem om Zijn grootse werken,~\sep\ looft Hem om Zijn hoogste majesteit.

  Looft Hem met bazuingeschal,~\sep\ looft Hem met harp en citer.

  Looft Hem met pauken en reidans,~\sep\ looft Hem met snaarinstrument en schalmei.

  Looft Hem met welluidende cimbalen,~\sep\ looft Hem met rinkelende cimbalen: al wat adem heeft, love de Heer!
\end{halfparskip}

\begin{halfparskip}
  \liturgicalhint{Eer}~--- \liturgicalhint{3x Alleluia.}~--- \liturgicalhint{Eerste vers:} Loof, Jeruzalem, de Heer,~\sep\ loof Uw God, o Sion.
\end{halfparskip}

\begin{halfparskip}
  \dd~Laat ons bidden, vrede zij met ons.

  \cc~Voor al Uw hulp en genaden die Gij ons gegeven hebt, waarvoor wij U nooit genoeg danken kunnen, prijzen en verheerlijken we U onophoudelijk in Uw gekroonde Kerk, voorzien van alle hulp en zegeningen, want Gij zijt de Heer en Schepper van alles, Vader, Zoon en Heilige Geest, in alle eeuwigheid.~--- \rr~Amen.
\end{halfparskip}

% % % % % % % % % % % % % % % % % % % % % % % % % % % % % % % % % % % % % % % %

\markedsection{Laku Mara}

\vspace{0.4em}
\begin{doublecols}
  \textsizexi

  \englishl \rr~You, Lord of all, we worship You~/ Jesus Christ, we exalt You~/ You give life to our bodies~/ and salvation to our souls!

  \dutchc{1} \rr~U, Heer van alles prijzen wij; U, Jezus Christus loven wij; U bent de Levendmaker van onze lichamen; U bent de Verlosser van onze zielen.
\end{doublecols}

\begin{halfparskip}
  \dd~Ik was blij toen me mij zei: Wij gaan op naar het huis van de Heer.~--- \rr~U, Heer van alles...

  (\liturgicaloption{Buiten kerken.} \dd~In elke plaats zijt Gij, Heer, aanvaard onze smeekbede.~--- \rr~U, Heer van alles...)

  \cc~Eer aan de Vader, de Zoon, en de Heilige Geest. Vanaf het begin en in alle eeuwigheid, amen en amen.

  \rr~U, Heer van alles...

  \dd~Laat ons bidden. Vrede zij met ons.

  \cc~U bent waarlijk de Levendmaker van onze lichamen, de goede Verlosser van onze zielen, en de trouwe Bewaker van onze levens. U moeten wij altijd loven, aanbidden en verheerlijken, Heer van alles in alle eeuwigheid.~--- \rr~Amen.
\end{halfparskip}

% % % % % % % % % % % % % % % % % % % % % % % % % % % % % % % % % % % % % % % %

\markedsection{Suraya D'Qdam}

\liturgicalhint{Na de eerste zin, zeg 3x alleluja, en herhaal het eerste vers. Op het einde: Eer aan... 3x alleluja.}

\markedsubsectionrubricwithhint{Maandagen ``voor''.}{(\Ps{12:1--7})}

\begin{halfparskip}
  Heer, schenk redding, want er zijn geen vromen meer;~\sep\ verdwenen is de trouw onder de kinderen der mensen.

  \qanona{Alleluia, alleluia, alleluia.}~--- \liturgicalhint{Eerste vers.}

  Allen liegen ze elkander voor,~\sep\ ze spreken met bedrieglijke lippen en vals gemoed.

  De Heer rukke al die bedrieglijke lippen uit,~\sep\ die grootsprekende tong,

  Hen, die zeggen: ``Sterk zijn wij door onze tong;~\sep\ wij hebben onze lippen met ons: wie kan ons overmeesteren?''

  ``Om de nood der verdrukten en het gejammer der armen zal Ik nu opstaan,'' zegt de Heer:~\sep\ ``redding zal Ik brengen aan wie er naar smacht.''
\end{halfparskip}

\markedsubsectionrubricwithhint{Dinsdagen``voor''.}{(\Ps{17:1--6})}

\begin{halfparskip}
  Luister, Heer, naar een rechtvaardige zaak, geef acht op mijn geroep,~\sep\ hoor de bede van argeloze lippen.

  \qanona{Alleluia, alleluia, alleluia.}~--- \liturgicalhint{Eerste vers.}

  Van Uw aanschijn ga over mij het oordeel uit:~\sep\ Uw ogen zien wat recht is.

  Peil mijn hart, doorvors het 's~nachts, beproef mij met vuur,~\sep\ geen onrecht zult Gij in mij vinden.

  Mijn mond misdeed niet zoals mensen gewoon zijn;~\sep\ naar de woorden van Uw lippen heb ik de wegen der Wet gevolgd.

  Vast drukten mijn schreden Uw paden,~\sep\ mijn voeten struikelden niet.

  Ik roep U aan, o God, want Gij zult mij verhoren;~\sep\ neig Uw oor naar mij en luister naar mijn bede!
\end{halfparskip}

\markedsubsectionrubricwithhint{Woensdagen ``voor''.}{(\Ps{23:1--5})}

\begin{halfparskip}
  De Heer is mijn Herder: het ontbreekt mij aan niets;~\sep\ in groenende beemden laat Hij mij sluimeren.

  \qanona{Alleluia, alleluia, alleluia.}~--- \liturgicalhint{Eerste vers.}

  Hij voert mij naar wateren, waar ik kan rusten;~\sep\ Hij verkwikt mijn ziel.

  Hij leidt mij langs rechte wegen,~\sep\ omwille van Zijn Naam.

  Al schrijd ik dan voort in een donker dal,~\sep\ geen kwaad zal ik vrezen, omdat Gij met mij zijt.

  Uw roede en Uw herdersstaf,~\sep\ zijn mij tot troost.
\end{halfparskip}

\markedsubsectionrubricwithhint{Donderdagen ``voor''.}{(\Ps{25:1--5})}

\begin{halfparskip}
  Tot U verhef ik mijn ziel, O Heer, mijn God.~\sep

  \qanona{Alleluia, alleluia, alleluia.}~--- \liturgicalhint{Eerste vers.}

  Op U vertrouw ik; laat mij niet te schande worden;~\sep\ dat mijn vijanden niet over mij juichen!

  Want van wie op U hopen, wordt niemand beschaamd,~\sep\ maar wel worden te schande, die hun woord vermetel breken.

  Toon mij Uw wegen, O Heer,~\sep\ leer mij Uw paden kennen.

  Leid mij in Uw waarheid, en geef mij onderricht, omdat Gij, God, mijn Redder zijt,~\sep\ en immer hoop ik op U.
\end{halfparskip}

\markedsubsectionrubricwithhint{Vrijdagen ``voor''.}{(\Ps{75:1--5})}

\begin{halfparskip}
  Wij loven U, Heer, wij loven U, en prijzen Uw Naam, verhalen Uw wonderen.

  \qanona{Alleluia, alleluia, alleluia.}~--- \liturgicalhint{Eerste vers.}

  ``Op de tijd, die Ik zal bepalen, zal Ik oordelen volgens recht.

  Al wankelt de aarde met al haar bewoners,~\sep\ Ik heb haar zuilen bevestigd.''
\end{halfparskip}

\markedsubsectionrubricwithhint{Vrijdagen ``midden''.}{(\Ps{95:1--8})}

\begin{halfparskip}
  Komt, laten wij jubelen voor de Heer,~\sep\ laten wij juichen voor de Rots van ons heil.

  \qanona{Alleluia, alleluia, alleluia.}~--- \liturgicalhint{Eerste vers.}

  Treden wij voor Zijn aanschijn met lofzangen,~\sep\ juichen wij Hem met liederen toe!

  Want de Heer is een grote God,~\sep\ en een grote Koning boven alle goden.

  Hij houdt in Zijn hand de diepten der aarde,~\sep\ en de toppen der bergen behoren Hem toe.

  Van Hem is de zee; Hij heeft ze geschapen,~\sep\ en het vaste land, door Zijn handen gevormd.

  Komt, laten wij aanbidden en ons neerwerpen,~\sep\ de knieën buigen voor de Heer, die ons heeft gemaakt.

  Want Hij is onze God;~\sep\ wij het volk van Zijn weide en de schapen van Zijn hand.
\end{halfparskip}

\markedsubsectionrubricwithhint{Zaterdagen ``voor''.}{(\Ps{30:1--5})}

\begin{halfparskip}
  Ik wil U roemen, o Heer, daar Gij mij gered hebt,~\sep\ en niet mijn vijanden over mij liet juichen.

  \qanona{Alleluia, alleluia, alleluia.}~--- \liturgicalhint{Eerste vers.}

  Heer, mijn God, ik riep tot U,~\sep\ en Gij hebt mij genezen.

  Heer, uit het dodenrijk hebt Gij mij weggevoerd,~\sep\ mij gered uit hen, die ten grave dalen.

  Speelt op de citer voor de Heer, gij, Zijn heiligen;~\sep\ en dankt Zijn heilige Naam.

  Want Zijn toorn duurt slechts een ogenblik,~\sep\ maar Zijn welwillendheid het hele leven door.

  's~Avonds komt er geween te gast,~\sep\ maar 's~morgens is er gejubel.
\end{halfparskip}

\markedsubsectionrubricwithhint{Maandagen ``na''.}{(\Ps{42:1--5})}

\begin{halfparskip}
  Gelijk de hinde smacht naar waterstromen,~\sep\ zo smacht mijn ziel naar U, o God.

  \qanona{Alleluia, alleluia, alleluia.}~--- \liturgicalhint{Eerste vers.}

  Naar God dorst mijn ziel, naar de levende God;~\sep\ wanneer mag ik komen en Gods aanschijn aanschouwen?

  Mijn tranen zijn mij tot spijs geworden dag en nacht,~\sep\ terwijl men steeds tot mij zegt: ``Waar is Uw God?''

  Ik denk er met diepe weemoed aan terug, hoe ik eenmaal voortschreed met de schare,~\sep\ ja, hen voorging naar Gods huis.
\end{halfparskip}

\markedsubsectionrubricwithhint{Dinsdagen ``na''.}{(\Ps{67:1--6})}

\begin{halfparskip}
  God zij ons genadig en zegene ons;~\sep\ Hij tone ons zijn vredig gelaat.

  \qanona{Alleluia, alleluia, alleluia.}~--- \liturgicalhint{Eerste vers.}

  Opdat men op aarde Zijn weg lere kennen,~\sep\ onder alle volken Zijn heil.

  Dat de volken U prijzen, o God,~\sep\ dat alle volken U prijzen!

  Laat juichen en jubelen de naties, omdat Gij met rechtvaardigheid de volken regeert,~\sep\ en de naties op aarde bestuurt.

  Dat de volken U prijzen, o God,~\sep\ dat alle volken U prijzen!
\end{halfparskip}

\markedsubsectionrubricwithhint{Woensdagen ``na''.}{(\Ps{72:1--5})}

\begin{halfparskip}
  Geef aan de Koning Uw rechtsmacht, o God,~\sep\ en Uw rechtvaardigheidszin aan de Zoon van de Koning.

  \qanona{Alleluia, alleluia, alleluia.}~--- \liturgicalhint{Eerste vers.}

  Hij moge Uw volk met rechtvaardigheid besturen,~\sep\ en Uw geringen naar billijkheid.

  De bergen zullen vrede brengen aan het volk,~\sep\ en de heuvelen gerechtigheid.

  De geringen uit het volk zal Hij beschermen, redding brengen aan de zonen der armen~\sep\ en de verdrukker met voeten treden.
\end{halfparskip}

\markedsubsectionrubricwithhint{Donderdagen ``na''.}{(\Ps{118:41--49})}

\begin{halfparskip}
  Dat Uw ontferming, Heer, op mij neerdale,~\sep\ en Uw hulp, zoals Gij beloofd hebt.

  \qanona{Alleluia, alleluia, alleluia.}~--- \liturgicalhint{Eerste vers.}

  En Ik zal, die mij honen, te woord staan,~\sep\ omdat ik op Uw woorden hoop.

  Neem het woord der waarheid niet uit mijn mond,~\sep\ daar ik mijn hoop stel op Uw besluiten.

  Uw Wet zal ik steeds onderhouden,~\sep\ voor eeuwig en immer.

  Op een brede weg zal ik wandelen,~\sep\ omdat ik Uw bevelen doorzoek.

  Ik zal voor koningen over Uw voorschriften spreken,~\sep\ en er mij niet over schamen,

  Ik zal mij verheugen in Uw geboden,~\sep\ die ik bemin.

  Opheffen zal ik mijn handen naar Uw geboden,~\sep\ en Uw verordeningen overwegen.
\end{halfparskip}

\markedsubsectionrubricwithhint{Vrijdagen ``na''.}{(\Ps{144:1--7})}

\begin{halfparskip}
  Ik wil U roemen, mijn God, de Koning,~\sep\ en prijzen Uw Naam in eeuwen der eeuwen.

  \qanona{Alleluia, alleluia, alleluia.}~--- \liturgicalhint{Eerste vers.}

  Dag aan dag wil ik U prijzen,~\sep\ en loven Uw Naam in de eeuwen der eeuwen.

  Groot is de Heer en hoogst lofwaardig,~\sep\ en Zijn grootheid is niet te meten.

  Het ene geslacht roemt bij het andere Uw werken,~\sep\ en verkondigt Uw macht.

  Zij spreken over de heerlijke luister van Uw majesteit,~\sep\ en maken Uw wonderwerken bekend.

  Zij bezingen de kracht van Uw ontzagwekkende daden,~\sep\ en verhalen Uw grootheid.

  Zij verkondigen de lof van Uw grote goedheid,~\sep\ en juichen om Uw gerechtigheid.
\end{halfparskip}

\markedsubsectionrubricwithhint{Zaterdagen ``na''.}{(\Ps{123:1--6})}

\begin{halfparskip}
  Was de Heer niet met ons geweest,~\sep\ zo moge Israël nu zeggen;

  \qanona{Alleluia, alleluia, alleluia.}~--- \liturgicalhint{Eerste vers.}

  Was de Heer niet met ons geweest, toen de mensen tegen ons opstonden,~\sep\ dan hadden zij ons levend verslonden.

  Toen hun toorn tegen ons ontbrandde,~\sep\ zou het water ons hebben verzwolgen,

  Zou een stortvloed over ons zijn heengegaan,~\sep\ zouden over ons zijn heengegaan de bruisende wateren.

  Geprezen zij de Heer, die ons niet prijsgaf,~\sep\ ten prooi aan hun tanden.
\end{halfparskip}

\markedday{Alle dagen.} Alleluia, alleluia, alleluia.

% % % % % % % % % % % % % % % % % % % % % % % % % % % % % % % % % % % % % % % %

\markedsection{Onita D'Qdam}

\markedsubsectionrubric{Maandagen ``voor''}

\begin{halfparskip}
  Heer, schenk redding, want er zijn geen vromen meer.~--- Zie, de gerechtigen zijn uitgestorven, maar de rechtvaardigen zijn verdwenen, de vrome is omgekomen, en er is geen eerlijke onder de mensen. Iedere man spreekt met zijn naaste met een dubbel hart en met verdeeldheid zaaiende lippen. Onze innerlijke mens is vol afgunst, bedrog en laster. Liefde, het voornaamste gebod, is uit onze geest ontworteld. Daarom is er grote vrees dat wij in Uw toorn omkomen, want Uw gerechtigheid heeft bij ons gezwegen en plaats gemaakt voor Uw genade. O onze Heer, heb medelijden met ons.

  De Heer is trouw in Zijn woorden.~--- Onze Heiland gaf de belofte van leven aan wie naar Zijn liefde verlangen, en maakte hen rijk in Zijn kennis, en vervulde hen met de wijsheid van Zijn kracht, en leerde hen altijd dit te bidden: Onze Vader die in de hemelen zijt, geheiligd zij Uw Naam, Uw rijk kome naar ons toe, Uw wil geschiede op aarde, zoals in de hemel. Geef ons het brood dat we nodig hebben, en leid ons niet in bekoring, maar verlos ons van de boze; want van U is het koninkrijk, en de kracht en de glorie.

  Eer...~--- Maria, die het Geneesmiddel van het Leven baarde voor de kinderen van Adam, wij willen onze toevlucht nemen in uw bede, en met de gebeden van mar Johannes zullen we de boze en zijn leger overwinnen. En door het gebed der profeten, apostelen, martelaren, vaders en leraren, en door het gebed van onze heilige vader \NN~\liturgicalhint{[patroon]}, van de belijders, en van mar Joris; en de grote kracht van het Kruis, en de heiliging van de heilige Kerk, zullen we Christus smeken om genade en mededogen te hebben met onze zielen.
\end{halfparskip}

\markedsubsectionrubric{Dinsdagen ``voor''}

\begin{halfparskip}
  Hoor, God, en heb medelijden met mij~--- U die luistert en U niet afwendt, maar antwoordt, redt en verlost, luister, O onze Heer, naar ons verzoek, en in Uw genade verhoor onze beden.

  Getrouw is de Heer in Zijn woorden~--- O Heer, U hebt gezegd: Voor iedereen die aan de deur van Mijn majesteit klopt, wordt de deur geopend, en zijn verzoeken zullen beantwoord worden.

  Eer...~--- Moge het gebed van de maagd Maria, de moeder van Christus, een bolwerk zijn voor ons, en ons behoeden voor de boze.

  Vanaf het begin en in alle eeuwigheid.~--- Profeten, apostelen, martelaren, priesters en leraren, moge uw gebed een muur voor ons zijn bij nacht en bij dag.

  Dat al het volk zegge: amen en amen.~--- Onze heilige vader, vriend van de hemelse Bruidegom, smeek voor ons om genade, van uw Heer wiens liefde U verlangde.
\end{halfparskip}

\markedsubsectionrubric{Woensdagen ``voor''}

\begin{halfparskip}
  Dag en nacht.~--- Moge het gebed van de maagd Maria, de moeder van Jezus, onze Heiland, voortdurend een bolwerk voor ons zijn bij nacht en bij dag.

  In alle seizoenen en tijden.~--- Moge het gebed..., \liturgicalhint{zoals hierboven}.

  Eer...~--- Profeten, bidt voor vrede; apostelen voor rust; en martelaren, priesters en leraren, mogen uw gebeden een wal voor ons zijn.

  Vanaf het begin en in alle eeuwigheid.~--- Bid en smeek, onze eerbiedwaardige vader, Christus wiens liefde u verlangde, dat de gemeente die uw herdenking viert, door uw gebeden worde geholpen.

  Dat al het volk zegge: Amen en amen.~--- U die de gebeden van Uw dienaren hoort en de verzoeken van Uw aanbidders beantwoordt, in Uw goedheid hoor ons gebed en de stem van ons verzoek en verhoor onze bede.
\end{halfparskip}

\markedsubsectionrubric{Donderdagen ``voor''}

\begin{halfparskip}
   Heer, U weet.~--- Meer dan allen weet U, Heer, wat ons helpt. In Uw goedertierenheid beheer ons leven, en moge Uw genade medelijden hebben met onze overtredingen. Moge Uw mededogen een geneesheer voor ons zijn, en Uw liefde een meester en leraar. Aan U zij glorie, en op ons zij Uw genade.

  Uw genade, Heer, duurt voor altijd.~--- Wij roepen de overvloedige genade van Uw goedheid, Koning Christus, voor bijstand in onze zwakheid; want de tijden zijn verontrustend, en de wereld is in verwarring door haar zonden. Geef ons onwankelbare vrede, zodat we door haar U altijd mogen belijden.

  Eer...~--- Maria, de heilige maagd, moeder van Jezus, onze Heiland, bid en smeek om genade voor de zondaars, die in uw gebeden hun toevlucht zoeken, dat ze niet vernietigd zouden worden. Moge uw gebed een bolwerk voor ons zijn, in deze wereld en in de toekomstige.

  Vanaf het begin en in alle eeuwigheid.~--- Door het gebed der gerechtigen die U welgevallig waren, en de rechtvaardigen die U behaagden, de profeten, apostelen, leraren, martelaren, priesters en monniken, behoud de vergadering van Uw aanbidders, zodat ze een nieuw lied voor U mogen doen opstijgen, Vader, Zoon en Heilige Geest.

  Dat al het volk zegge: amen en amen.~--- Laat ons allen zorgvuldig de glorieuze dag eren van de herdenking van onze eerbiedwaardige en heilige vader, die een vat vol genade was en waardig zijn Heer te dienen. Elke dag schijnt zijn lamp; moge zijn gebed een wal voor ons zijn.
\end{halfparskip}

\markedsubsectionrubric{Vrijdagen ``voor''}

\begin{halfparskip}
  Brengt Hem hulde en zegent Zijn Naam.~--- Belijdt, stervelingen, de Zoon die ons heeft gered van de heerschappij van de dood, die ons in onze zonden hield. Want de enige reden waarom Hij naar Sheol afdaalde, was dat Hij de doden uit de graven tot leven zou kunnen wekken. Wie kan de goedheid terugbetalen die Hij aan het ras der stervelingen heeft getoond?

  Mijn mond zal van wijsheid spreken.~--- Zoekt toevlucht, zondaars, in berouw, want de tijd is kort. De wereld gedijt en gaat voorbij. Zegen voor de boeteling, maar oordeel voor de bozen die niet worden gerechtvaardigd (\translationoptionNl{vrijgesproken}). Want als U, mijn Heer, rechtvaardig oordeelt, wie kan dan in het tribunaal onschuldig worden verklaard?

  Eer...~--- Wij smeken Uw goedheid, Christus Koning, gedenk niet de overtredingen van Uw dienaren, die het Mysterie van Uw lichaam hebben ontvangen; moge U voor hen opkomen op de dag van de opstanding, en mogen zij verlost worden van Gehenna; en mogen zij samen met de engelen worden verheven om met grote glorie U tegemoet te treden in de hoogste hemel.
\end{halfparskip}

\markedsubsectionrubric{Vrijdagen ``midden''}

\begin{halfparskip}
  Komt, laten we de Heer verheerlijken.~--- Komt, stervelingen, laten we belijden en verheerlijken Hem die door Zijn dood de heerschappij van de dood teniet deed, en leven en opstanding beloofde aan het hele ras van stervelingen.

  Komt, laten we knielen en Hem aanbidden.~--- U aanbidden wij, Christus onze Verlosser, want u bent de Bezieler en Verlosser van alle overledenen, die gedoopt werden in Uw Naam, en Uw Kruis en Uw dood beleden.

  Eer...~--- Glorie aan U die door Uw opstanding leven en verkwikking beloofde voor het hele ras van
  stervelingen. Wij zullen U belijden en verheerlijken, want U bent de bezieler der overledenen.
\end{halfparskip}

\markedsubsectionrubric{Zaterdagen ``voor''}

\begin{halfparskip}
  Ons hart zal zich in Hem verheugen.~--- Moge het Kruis, dat voor ons de oorzaak van zegeningen was, en waardoor ons sterfelijk ras werd bevrijd, voor ons een sterke muur zijn, mijn Heer, en mogen we door haar de boze en al zijn listen overwinnen.

  Omdat we hebben gehoopt op Zijn heilige Naam.~--- Moge het Kruis..., \liturgicalhint{zoals hierboven}.

  Eer...~--- Heilig, Heiland, Uw Kerk in Uw mededogen, en laat Uw goedheid wonen in de tempel die tot Uw eer is gereserveerd; en plaats daarin Uw edele altaar, waarop Uw lichaam en bloed worden gevierd, o Heer.

  Vanaf het begin en in alle eeuwigheid.~--- Onze Heer, die in Uw barmhartigheid aan Uw dienaren heeft beloofd: ``Wie vraagt, zal ontvangen, en wie zoekt, zal vinden''. Van U vragen wij kracht en hulp, dat we de wil van Uw Majesteit mogen vervullen door onze daden.
\end{halfparskip}

\markedsubsectionrubric{Maandagen ``na''}

\begin{halfparskip}
  Zoals het hert schreeuwt om de waterbeek.~--- Wie zal mij een fontein van tranen en een zuiver hart geven, om te huilen, klagen en kreunen met luid zuchten voor de jaren van mijn leven die verspild zijn in ijdelheden zonder nut? En ik was fout in mijn gedrag.

  Wee mij dat mijn verblijf verlengd wordt.~--- Wee mij dat het mij is overkomen, dat de boze een strik voor mij heeft gespannen, en ik ben erin gevallen. U, mijn Heer, hebt mij bevrijd. En ik heb Uw geboden veracht. En toen de boze zag dat ik nalatig was, spande hij een strik voor mij, en ik viel erin. Verlos mij, o mijn Heer, van
  de boze die mij gevangen heeft genomen.

  Eer...~--- Door de gebeden van de maagd Maria, de gezegende moeder, mogen Uw aanbidders, Heer, worden bewaard van de listen van de sluwe. En geef dat we Uw wil vervullen, zowel in woord als in daden, en dat we U ten allen tijde mogen lofzingen.

  Vanaf het begin en in alle eeuwigheid.~--- Door het gebed van Uw heiligen die Uw geboden hebben
  onderhouden, Christus onze Verlosser, profeten, apostelen, leraren, martelaren, priesters en monniken, bewaar de vergadering van Uw aanbidders van de listen van de sluwe. En sterk ons om Uw wil te vervullen.

  Dat al het volk zegge: amen en amen.~--- Een kleed van genade geweven door de Heilige Geest bent u, onze vader; een fontein van hemelse zegeningen hebt u uitgestort door uw standvastige daden; en u hebt de kudde van uw weide te drinken gegeven van het woord van geestelijk leven. Zie, de overwinningskroon is voor u geweven.
\end{halfparskip}

\markedsubsectionrubric{Dinsdagen ``na''}

\begin{halfparskip}
  Toon het licht van Uw aangezicht en we zullen worden gered.~--- Medelevende en Erbarmingsvolle, verslap niet Uw waakzaamheid op ons. Stuur ons vanuit Uw schatkist mededogen, barmhartigheid en redding.

  Totdat U ons genadig bent.~--- Aan uw deur, o Heer, kloppen we, en vragen U om genade. Open voor ons en beantwoord onze verzoeken, U die Uw goedheid en genade niet tegenhoudt.

  Eer...~--- Moge het gebed van de maagd Maria, de moeder van Jezus, onze Heiland, voortdurend voor ons een muur zijn bij dag en nacht.

  Vanaf het begin en in alle eeuwigheid.~--- Profeten, bidt voor vrede, apostelen, voor rust; martelaren, priesters en leraren, moge uw gebed een muur voor ons zijn.

  Dat al het volk zegge: amen en amen.~--- Vraag en smeek, onze eerbiedwaardige vader, Christus wiens liefde u verlangde, dat de vergadering, die uw herdenking heeft gevierd, door uw gebeden mogen geholpen worden.
\end{halfparskip}

\markedsubsectionrubric{Woensdagen ``na''}

\begin{halfparskip}
  Daar zal ik de hoorn van David laten opstaan.~--- Uit het huis van David en Abraham koos de Schepper een maagd, en deed Zijn verborgen kracht in haar wonen. Door de kracht van de Heilige Geest ontving ze en baarde de Heiland, Christus, de Rechter van de hoogten en de diepten.

  En heeft haar uitgekozen als woning voor Zichzelf.~--- Uit het huis... \liturgicalhint{zoals hierboven}.

  Eer...~--- Hoe passend is het dat we dit heilige huis zouden verheerlijken, waarin profeten, en apostelen zijn, en martelaren, priesters en leraren, en waarin de heilige tafel is opgesteld voor de vergeving der kinderen van Adam.

  Vanaf het begin en in alle eeuwigheid.~--- Gezegend is hij die handel bedreef zoals U handel bedreef, o onze vader; en die geestelijke rijkdom heeft verzameld, en zijn schip vulde met alle zegeningen, en gedijde en vertrok naar het toevluchtsoord, naar de plaats die is aangewezen voor alle rechtvaardigen.

  Dat al het volk zegge: amen en amen.~--- Onze Heer, uw koninkrijk kome, Uw wil geschiede op aarde, zoals het is in de hemel; en geef ons het brood van onze nood, en leid ons niet in bekoring, maar verlos ons van de boze.
\end{halfparskip}

\markedsubsectionrubric{Donderdagen ``na''}

\begin{halfparskip}
  Laat Uw barmhartigheden tot mij komen, Heer, en Uw zaligheid waarover U sprak.~--- Laat Uw barmhartigheden ons te hulp komen, want zij hebben ons geschapen. En laten ze onze ziekte genezen met het medicijn van Uw mededogen.

  Wend u tot mij en heb medelijden met mij.~--- Kom, mijn Heer, ter hulp, en versterk onze zwakheid. Want in U is onze hoop 's~nachts en overdag.

  Eer...~--- Laat er op het heilig altaar een herdenking zijn van de maagd Maria, de moeder van Christus.

  Vanaf het begin en in alle eeuwigheid.~--- Apostelen van de Zoon en vrienden van de Eniggeborene, bidt dat er vrede mag zijn in de schepping.

  Dat al het volk zegge: Amen en amen.~--- Uw gedenkteken, o onze vader, staat op het heilige altaar; met de rechtvaardigen die overwonnen hebben en de martelaren die werden gekroond.
\end{halfparskip}

\markedsubsectionrubric{Laatste vrijdagen}

\begin{halfparskip}
  Ik zal U verheerlijken, mijn Heer de Koning.~--- Christus Koning, onze Redder. Maak mij levend op de dag van Uw komst. En breng mij naar Uw rechterhand op de dag dat Uw majesteit verschijnt.

  Mijn God, ik heb mijn vertrouwen gesteld in U, laat mij niet beschaamd worden.~--- Wij vereren Uw Kruis, mijn Heer, waardoor wij overeind staan, en waardoor we levend worden gemaakt. Daardoor worden onze overledenen levend gemaakt, en hun lichamen bekleden de heerlijkheid.

  Eer...~--- God de Vader, geef mij leven. Christus de Zoon, maak mij levend die dood ben. Heilige Geest de Parakleet, breng me naar het land van licht.
\end{halfparskip}

\markedsubsectionrubric{Zaterdagen ``na''}

\begin{halfparskip}
  En ons hart zal zich in Hem verheugen.~--- Moge Uw Kruis, o Verlosser, dat vrede heeft gebracht tussen de hoogte en de diepte, door Zijn grote, aanbiddelijke en glorieuze kracht, vrede brengen in de wereld, die verstoord en verward is door haar zonden. En laat Uw vrede wonen in de vier windrichtingen.

  Hij doet oorlogen in de hele wereld eindigen.~--- Moge uw Kruis..., \liturgicalhint{zoals hierboven}.

  Eer...~--- Bewaar Uw heilige Kerk voor kwaad, Heer Christus, die kwam om ons te redden, en maak Uw aanbidders waardig om in angst en beven Uw Godheid te prijzen.

  Vanaf het begin en in alle eeuwigheid.~--- Geef ons, onze Levengever, dat waarvan u weet dat het ons zal helpen, want we weten niet wat we vragen. Maar één ding verzoeken wij U, dat Uw wil in ons vervuld mag worden, en dat Uw barmhartigheid voor ons allemaal mag bemiddelen.
\end{halfparskip}

\begin{halfparskip}
  \markedday{Alle dagen.} \dd~Laat ons bidden; vrede zij met ons.

  \cc~Wij moeten altijd Uw barmhartigheid en de zorg van Uw goede wil jegens ons, onze Heer en onze God, erkennen, aanbidden en eren, Heer van alles, Vader, Zoon en Heilige Geest in eeuwigheid.~--- \rr~Amen.
\end{halfparskip}

% % % % % % % % % % % % % % % % % % % % % % % % % % % % % % % % % % % % % % % %

\markedsection{Maria Qretak \markedsectionhint{(\Ps{140--141;118:105--112;116} onder één Eer aan de Vader.)}}

\begin{halfparskip}
  \psalm{\Ps{140}} Ik roep tot U, Heer; snel mij te hulp.~--- \qanona{Luister naar mijn smeken, wanneer ik tot U roep.}

  Ik roep tot U, Heer; snel mij te hulp;~\sep\ luister naar mijn smeken, wanneer ik tot U roep.

  Laat mijn bede als een reukoffer opgaan tot U,~\sep\ het heffen van mijn handen als een avondoffer zijn.

  Heer, zet een wacht voor mijn mond,~\sep\ een post voor de deur van mijn lippen.

  Neig mijn hart niet tot kwaad,~\sep\ om boze daden te stellen;

  En geef dat ik nooit met boosdoeners,~\sep\ hun uitgezochte spijzen eet.

  Laat de rechtvaardige mij slaan: dat is liefde;~\sep\ mij berispen: dat is olie op mijn hoofd,

  Die mijn hoofd niet zal weigeren;~\sep\ maar immer zal ik bidden onder hun kastijding.

  Hun vorsten werden neergelaten langs de rots,~\sep\ en zij hoorden hoe zachtzinnig mijn woorden waren.

  Zoals wanneer men de grond doorploegt en scheurt,~\sep\ zo liggen hun beenderen verstrooid bij de poort van het dodenrijk.

  Maar op U, Heer God, zijn mijn ogen gericht,~\sep\ naar U vlucht ik heen: laat mij niet vergaan.

  Behoed mij voor het net, dat ze mij spanden;~\sep\ en voor de valstrikken van hen, die het kwade bedrijven.

  Laat de bozen tezamen in hun eigen netten vallen,~\sep\ terwijl ik behouden ontkom.
\end{halfparskip}

\begin{halfparskip}
  \psalm{\Ps{141}} Luid roep ik tot de Heer,~\sep\ luid smeek ik de Heer.

  Voor Hem stort ik mijn zorgen uit,~\sep\ voor Hem leg ik mijn kommer bloot.

  Als mijn geest in mij is beangst,~\sep\ kent Gij mijn weg.

  Op het pad waarlangs ik ga,~\sep\ heeft men mij heimelijk een strik gelegd.

  Ik wend mij naar rechts en zie uit,~\sep\ maar niet één die om mij zich bekommert,

  Er is voor mij geen uitweg meer,~\sep\ en niemand draagt zorg voor mijn leven.

  Ik roep tot U, o Heer, ik zeg: Gij zijt mijn toevlucht,~\sep\ mijn aandeel in het land der levenden.

  Geef acht op mijn geroep,~\sep\ want diep ellendig ben ik geworden.

  Ontruk mij aan die mij vervolgen,~\sep\ want sterker zijn ze dan ik.

  Leid mij uit de kerker,~\sep\ opdat ik Uw Naam moge danken.

  De rechtvaardigen zullen mij omringen,~\sep\ wanneer Gij mij hebt welgedaan.
\end{halfparskip}

\begin{halfparskip}
  \psalm{\Ps{118:105vv}} Uw woord is een lamp voor mijn voeten,~\sep\ en een licht op mijn pad.

  Ik zweer en neem mij voor,~\sep\ Uw rechtvaardige besluiten na te leven.

  Ik ben in de diepste ellende, o Heer;~\sep\ spaar mijn leven naar Uw woord.

  Aanvaard, o Heer, de offers van mijn mond,~\sep\ en leer mij Uw besluiten.

  In voortdurend gevaar is mijn leven,~\sep\ maar Uw Wet vergeet ik niet.

  De bozen hebben mij een strik gelegd,~\sep\ maar van Uw bevelen week ik niet af.

  Uw voorschriften zijn mijn erfdeel voor eeuwig,~\sep\ want ze zijn de vreugde van mijn hart.

  Ik heb er mijn hart op gezet Uw verordeningen na te komen,~\sep\ voortdurend en stipt.
\end{halfparskip}

\begin{halfparskip}
  \psalm{\Ps{116}} Looft de Heer, alle volkeren,~\sep\ alle naties, verheerlijkt Hem,

  Want Zijn erbarming blijft ons verzekerd,~\sep\ en de trouw van de Heer duurt eeuwig.

  Eer...~--- Ik roep tot U, o Heer; snel mij te hulp;~\sep\ luister naar mijn smeken, wanneer ik tot U roep.
\end{halfparskip}

\begin{halfparskip}
  \dd~Laat ons bidden; vrede zij met ons.

  \cc~Hoor, onze Heer en onze God, het gebed van Uw dienaars in uw mededogen, ontvang de petitie van Uw aanbidders in Uw barmhartigheid, en heb medelijden met onze zondigheid in Uw goedheid en barmhartigheid, O Geneesheer van ons lichaam en goede Hoop van onze ziel, Heer van alles, Vader...~--- \rr~Amen.
\end{halfparskip}

% % % % % % % % % % % % % % % % % % % % % % % % % % % % % % % % % % % % % % % %

\markedsection{Suraya D'Batar}

\liturgicalhint{Na de eerste zin, zeg 3x alleluja, en herhaal het eerste vers. Op het einde: Eer aan... 3x alleluja.}

\markedsubsectionrubric{Maandagen ``voor''}

\begin{halfparskip}
  \psalm{\Ps{15,1--5}} Heer, wie mag in Uw tent verblijven,~\sep\ wie wonen op Uw heilige berg?

  \qanona{Alleluia, alleluia, alleluia.}~--- \liturgicalhint{Eerste vers.}

  Die vlekkeloos wandelt en deugdzaam leeft, en in zijn hart wat goed is denkt,~\sep\ en niet lastert met zijn tong;

  Die zijn evenmens geen kwaad berokkent,~\sep\ en zijn nabuur geen smaad aandoet;

  Die de boze voor verachtelijk houdt,~\sep\ maar eert die vrezen de Heer;
\end{halfparskip}

\markedsubsectionrubric{Dinsdagen ``voor''}

\begin{halfparskip}
  \psalm{\Ps{21,1--5}} Over uw macht, o Heer, verheugt zich de koning,~\sep\ hoe uitbundig jubelt hij over Uw hulp!

  \qanona{Alleluia, alleluia, alleluia.}~--- \liturgicalhint{Eerste vers.}

  Zijn hartewens hebt Gij verhoord,~\sep\ en de bede van zijn lippen niet afgewezen.

  Ja, Gij hebt hem voorkomen met rijke zegen,~\sep\ een kroon van zuiver goud hem op het hoofd gedrukt.

  Leven vroeg hij U; Gij hebt hem gegeven,~\sep\ lengte van dagen voor immer.

  Groot is zijn roem, dank aan uw hulp;~\sep\ met majesteit en luister hebt Gij hem getooid,
\end{halfparskip}

\markedsubsectionrubric{Woensdagen ``voor''}

\begin{halfparskip}
  \psalm{\Ps{24,1--6}} De Heer behoort de aarde met al wat zij bevat,~\sep\ de wereld en die er op wonen.

  \qanona{Alleluia, alleluia, alleluia.}~--- \liturgicalhint{Eerste vers.}

  Want Hij heeft haar op de zeeën gegrondvest,~\sep\ en legde haar vast op de stromen.

  Wie mag de berg van de Heer bestijgen,~\sep\ of wie verwijlen in Zijn heilige plaats?

  Die rein is van handen en zuiver van hart, zijn geest niet richt op ijdele dingen,~\sep\ en tegen zijn naaste geen meineed zweert.

  Die zal zegen ontvangen van de Heer,~\sep\ en loon van God, zijn Redder.

  Dit is het geslacht van die naar Hem zoeken,~\sep\ van die zoeken het aanschijn van Jacobs God.
\end{halfparskip}

\markedsubsectionrubric{Donderdagen ``voor''}

\begin{halfparskip}
  \psalm{\Ps{28,1--8}} Tot U roep ik, o Heer;~\sep\ mijn Rots, wees niet doof voor mij.

  \qanona{Alleluia, alleluia, alleluia.}~--- \liturgicalhint{Eerste vers.}

  Opdat ik niet, als Gij niet hoort naar mij,~\sep\ gelijk worde aan hen, die in de grafkuil dalen.

  Hoor de stem van mijn smeken, nu ik roep tot U,~\sep\ nu ik mijn handen ophef naar Uw heilige tempel.

  Ruk mij niet weg met de zondaars,~\sep\ met hen, die kwaad bedrijven,

  Die vriendelijk spreken met hun naaste,~\sep\ maar in hun hart kwade bedoelingen koesteren.

  Handel met hen naar hun daden,~\sep\ en naar de boosheid van hun werken.

  Zet hun het werk van hun handen betaald,~\sep\ vergeld ze hun daden.

  Want ze slaan geen acht op de daden van de Heer en het werk van Zijn handen;~\sep\ Hij richte hen te gronde en heffe hen niet op.

  Gezegend de Heer, want Hij hoorde mijn dringende bede.
\end{halfparskip}

\markedsubsectionrubric{Vrijdagen ``voor''}

\begin{halfparskip}
  \psalm{\Ps{82,1--5}} God rijst op in de goddelijke raad,~\sep\ Hij houdt gericht te midden der goden.

  \qanona{Alleluia, alleluia, alleluia.}~--- \liturgicalhint{Eerste vers.}

  ``Hoe lang nog zult gij onrechtvaardig oordelen,~\sep\ en de zaak der bozen begunstigen?

  Verdedigt verdrukten en wezen,~\sep\ geeft aan ellendigen en armen hun recht,

  Bevrijdt verdrukten en behoeftigen:~\sep\ ontrukt ze aan de hand van de bozen.''
\end{halfparskip}

\markedsubsectionrubric{Midden vrijdagen}

\begin{halfparskip}
  \psalm{\Ps{139,1--5}} Verlos mij, Heer, van de boze,~\sep\ behoed mij voor de geweldenaar,

  \qanona{Alleluia, alleluia, alleluia.}~--- \liturgicalhint{Eerste vers.}

  Voor wie kwaad beramen in hun hart,~\sep\ twist verwekken dag aan dag,

  Die als slangen hun tongen scherpen:~\sep\ addergif schuilt er onder hun lippen.

  Red mij, Heer, uit de hand van de boze,~\sep\ behoed mij voor de geweldenaar;

  Die mij de voet willen lichten, de trotsaards,~\sep\ zij spannen mij heimelijk een strik,

  Zij spannen als een net hun koorden,~\sep\ leggen mij valstrikken langs de weg.
\end{halfparskip}

\markedsubsectionrubric{Zaterdagen ``voor''}

\begin{halfparskip}
  \psalm{\Ps{54,1--5}} Red mij, o God, door Uw Naam,~\sep\ en treed in Uw kracht voor mijn rechtszaak op!

  \qanona{Alleluia, alleluia, alleluia.}~--- \liturgicalhint{Eerste vers.}

  Luister naar mijn bede, o God,~\sep\ hoor naar de woorden van mijn mond!

  Want trotsaards zijn tegen mij opgestaan en geweldenaars stonden mij naar het leven;~\sep\ zij hielden God niet voor ogen.

  Zie, God komt mij te hulp,~\sep\ de Heer behoudt mijn leven.
\end{halfparskip}

\markedsubsectionrubric{Maandagen ``na''}

\begin{halfparskip}
  \psalm{\Ps{123,1--3}} Was de Heer niet met ons geweest,~\sep\ zo moge Israël nu zeggen;

  \qanona{Alleluia, alleluia, alleluia.}~--- \liturgicalhint{Eerste vers.}

  Was de Heer niet met ons geweest,~\sep\ toen de mensen tegen ons opstonden.

  Dan hadden zij ons levend verslonden,~\sep\ toen hun toorn tegen ons ontbrandde.
\end{halfparskip}

\markedsubsectionrubric{Dinsdagen ``na''}

\begin{halfparskip}
  \psalm{\Ps{70,1--4}} Gewaardig U, o God, mij te verlossen;~\sep\ Heer, snel mij te hulp.

  \qanona{Alleluia, alleluia, alleluia.}~--- \liturgicalhint{Eerste vers.}

  Laat schande en beschaming hen treffen,~\sep\ die mij naar het leven staan.

  Dat zij vol schaamte terugdeinzen,~\sep\ die zich over mijn rampen verheugen,

  Terugwijken, met schaamte overladen,~\sep\ die mij toeroepen: Ha, ha!

  Over U mogen jubelen en zich verblijden,~\sep\ zij allen, die U zoeken.

  En steeds herhalen: ``Hooggeprezen zij God''.
\end{halfparskip}

\markedsubsectionrubric{Woensdagen ``na''}

\begin{halfparskip}
  \psalm{\Ps{101,1--6}} Van liefde en recht wil ik zingen,~\sep\ voor U de citer bespelen, o Heer.

  \qanona{Alleluia, alleluia, alleluia.}~--- \liturgicalhint{Eerste vers.}

  Op de weg der onschuld zal ik wandelen;~\sep\ wanneer zult Gij tot mij komen?

  Rein van hart wil ik leven,~\sep\ binnen mijn huis.

  Mijn ogen zal ik niet vestigen,~\sep\ op ongerechtigheid;

  Wie onrecht pleegt, is mij een gruwel:~\sep\ hij zal met mij geen omgang hebben.

  Een bedorven hart blijft verre van mij;~\sep\ van kwaad wil ik niets weten.

  Wie heimelijk zijn naaste belastert,~\sep\ die zal ik te gronde richten.

  De trotse blik en het hovaardig hart,~\sep\ zal ik niet dulden.

  Mijn ogen zien uit naar de getrouwen in het land,~\sep\ opdat ze bij mij wonen.

  Wie wandelt langs de goede weg,~\sep\ die zal mijn dienaar zijn.
\end{halfparskip}

\markedsubsectionrubric{Donderdagen ``na''}

\begin{halfparskip}
  \psalm{\Ps{118,121-129}} Recht en gerechtigheid heb ik beoefend;~\sep\ lever mij niet over aan mijn verdrukkers.

  \qanona{Alleluia, alleluia, alleluia.}~--- \liturgicalhint{Eerste vers.}

  Sta borg voor het welzijn van Uw dienaar,~\sep\ opdat de trotsen mij niet verdrukken.

  Mijn ogen kwijnen van verlangen naar Uw hulp,~\sep\ en naar Uw rechtvaardige uitspraak.

  Handel met Uw dienaar naar Uw goedheid,~\sep\ en leer mij Uw verordeningen.

  Ik ben Uw dienstknecht, onderricht mij,~\sep\ opdat ik Uw voorschriften kenne.

  Voor de Heer is het tijd om te handelen:~\sep\ zij hebben Uw Wet verkracht.

  Daarom heb ik Uw geboden lief,~\sep\ meer dan goud en het edelst metaal.

  Daarom koos ik al Uw bevelen tot mijn deel;~\sep\ van iedere dwaalweg heb ik een afschuw.
\end{halfparskip}

\markedsubsectionrubric{Laatste vrijdagen}

\begin{halfparskip}
  \psalm{\Ps{144,18ff}} De Heer is allen, die Hem aanroepen, nabij,~\sep\ allen, die Hem aanroepen in oprechtheid.

  \qanona{Alleluia, alleluia, alleluia.}~--- \liturgicalhint{Eerste vers.}

  Hij zal de wensen vervullen van hen, die Hem vrezen,~\sep\ hun smeken aanhoren en hen redden.

  De Heer behoedt allen, die Hem beminnen,~\sep\ maar alle bozen zal Hij verdelgen.

  Laat mijn mond verkondigen de lof van de Heer,~\sep\ en alle vlees prijzen Zijn heilige Naam in alle eeuwen der eeuwen.
\end{halfparskip}

\markedsubsectionrubric{Zaterdagen ``na''}

\begin{halfparskip}
  \psalm{\Ps{124,1--3}} Wie op de Heer vertrouwen, zijn als de berg Sion,~\sep\ die niet wankelt, maar staan blijft voor eeuwig.

  \qanona{Alleluia, alleluia, alleluia.}~--- \liturgicalhint{Eerste vers.}

  Bergen omgeven Jeruzalem,~\sep\ zo omgeeft de Heer Zijn volk, en nu en in eeuwigheid.
\end{halfparskip}

\begin{halfparskip}
  \markedday{Alle dagen.} Alleluia, alleluia, alleluia.
\end{halfparskip}

% % % % % % % % % % % % % % % % % % % % % % % % % % % % % % % % % % % % % % % %

\markedsection{Onita D'Batar}

\markedsubsectionrubric{Maandagen ``voor''}

\begin{halfparskip}
  Hij die voor de werelden is.~--- Medelevende Vader, barmhartige Zoon, en lankmoedige Geest, de Heiliger van het onreine, heilig ons lichaam en heb medelijden met ons.

  Heilig en vreselijk is Zijn Naam.~--- Moge de lofprijzingen en het zingen van alleluja's der hemelse scharen, door de bede van de Zoon aan de rechterhand van de Vader voor ons bij Uw gerechtigheid smeken en ons genadig zijn.

  Eer...~--- Maria, moeder van de Koning, de Koning der koningen, smeek Christus, die uit uw boezem is verschenen, dat Hij medelijden met ons moge hebben in Zijn goedheid, en ons Zijn koninkrijk waardig maken.

  Vanaf het begin en in alle eeuwigheid.~--- Christus onze Verlosser, door het gebed van Uw heiligen, de profeten, apostelen, martelaren, en alle rechtvaardigen, bescherm het gezelschap van Uw aanbidders tegen alle kwaad.

  Dat al het volk zegge: amen en amen.~--- Onze heilige vader, wees een gids voor ons in goede daden, die uw Heer gunstig stemmen, dat we door uw gebeden kunnen worden geholpen, en met u vreugde mogen hebben.
\end{halfparskip}

\markedsubsectionrubric{Dinsdagen ``voor''}

\begin{halfparskip}
  Heer, hoor mijn gebed~--- Wij verzoeken U, Heer, en smeken Uw Majesteit. Zoals U ons in Uw mededogen hebt geschapen, geef ons zo het leven door Uw komst; want U hebt medelijden met zondaars, en erbarmen voor hen die zich bekeren. Laat in Uw liefderijke barmhartigheid de menigte zonden voorbijgaan.

  Luister naar mijn woorden en aanvaard (ze).~--- God, die het lam heeft aangenomen van Abel, een perfect lam, en het offer van de rechtvaardige Noach, en van de trouwe Abraham, aanvaard, Heer, ons verzoek, en verhoor in Uw genade onze verzoeken, en moge Uw vrede onder ons wonen, al onze dagen.

  Eer...~--- Maria, de heilige maagd, moeder van Jezus onze Heiland, moge uw gebed een toevluchtsoord zijn voor het gezelschap der gelovigen, en mogen onze gebeden door u verhoord worden, als een hulp voor onze zwakheid, en mogen wij met u Christus zien, op de dag van Zijn openbaring.

  Vanaf het begin en in alle eeuwigheid.~--- We houden de herdenking der rechtvaardigen, en nemen onze toevlucht in hun gebeden. En via hen roepen we U aan: o Heer, heb medelijden met ons; bevestig in ons Uw liefde zoals die in hen is; en laat onze mond hun waarheid prediken; en bevestig in ons hun geloof, de hoop van uw gelovigen.

  Dat al het volk zegge: amen en amen.~--- Hoe uitmuntend bent u, gezegende; en begeerd is de kroon van al uw overwinningen, die u door deugdzame daden hebt gewonnen in de hemel. U hebt de tegenstander veroordeeld, die met u gevochten heeft in de wedloop. En zie, uw herdenking wordt gevierd in de hemel en op aarde.
\end{halfparskip}

\markedsubsectionrubric{Woensdagen ``voor''}

\begin{halfparskip}
  En het land is ermee gevuld.~--- Maria, die het Geneesmiddel van het leven baarde voor de kinderen van Adam, mogen wij door uw bede genade vinden op de dag van de verkwikking.

  De heuvels waren bedekt met de schaduw ervan.~--- Maria, \liturgicalhint{zoals hierboven}.

  Eer...~--- Heer, U hebt van alle trouwe zonen van Uw mysteries die Uw Naam liefhadden voor ons een bron van leven gemaakt.

  Vanaf het begin en in alle eeuwigheid.~--- Onze eerbiedwaardige en heilige vader, vriend van de Zoon, smeek van uw Heer medeleven en barmhartigheid voor onze zielen.

  Dat al het volk zegge: amen en amen.~--- O Heer, help ons, zend vrede, verstrooi de bozen; behoud Uw Kerk, bevrijd haar kinderen van kwaad.
\end{halfparskip}

\markedsubsectionrubric{Donderdagen ``voor''}

\begin{halfparskip}
  Heer, ik heb U dagelijks aangeroepen.~--- Tot U roepen de gekwelden, O Medelijdende; en bij U zoeken de bedroefden hun toevlucht, o Mensenvriend. In Uw medelijden, wees een bewaker van hun leven; en red hen van de Boze, want op U hopen zij.

  In het diepste van zijn hart.~--- Jona riep tot U vanuit de vis, en U verhoorde hem; het gezelschap van Ananias in de vuuroven, en U bevrijdde hen. De hele schepping roept U al zuchtend aan: Heb medelijden met haar en ontferm U over haar, zoals u gewoon bent.

  Eer...~--- Maria, moeder van de Koning, de Koning der koningen, offer samen met ons een verzoek aan de Zoon, die van U is; dat Hij Zijn vrede en kalmte doet wonen in de schepping, en dat de Kerk en haar kinderen van kwaad bewaard blijven.

  Vanaf het begin en in alle eeuwigheid.~--- Vrede aan uw gezelschap, o gezegenden, kooplieden die het leven brengen naar de mensen, opent de schatkist van uw gebeden voor de behoeftigen; beschermt het land waar jullie woonden van schade.

  Dat al het volk zegge: amen en amen.~--- Verzoek voor ons allemaal bij uw Heer, onze eerbiedwaardige en heilige vader, vriend van de Zoon. Mogen door uw gebeden worden geholpen en gered allen die gekweld en bedroefd zijn en hun toevlucht bij u zoeken.
\end{halfparskip}

\markedsubsectionrubric{Vrijdagen ``voor''}

\begin{halfparskip}
  Ik zal de Heer altijd zegenen.~--- Gezegend is Uw dag, Zoon van de Heer van alles, die komt en de boezem van Sheol scheurt. Glorieus is Uw verrijzenis, waarnaar verlangen de generaties die zijn verdwenen en zij die overblijven.

  Ze gaan voorbij, maar U volhardt.~--- Kijk, deze wereld gaat voorbij, en al haar verlangens lopen uit op niets.

  Gezegend is hij die voorraad voor de wereld die niet voorbijgaat voor zichzelf heeft klaargelegd.

  Eer...~--- We hebben geen hoop waarin we kunnen roemen, tenzij Uw Kruis dat onze overtredingen vergeeft. Het is voor ons een hoge muur, en verlost ons van schade.
\end{halfparskip}

\markedsubsectionrubric{Vrijdagen ``midden''}

\begin{halfparskip}
  U hebt mij gevormd en Uw hand op mij gelegd.~--- Op vrijdag in het begin vormde God Adam uit stof, blies de geest in hem; en maakte hem tot een redelijk wezen, zodat hij Hem zou loven.

  Dwaze en onverstandige mensen.~--- Op een vrijdag kruisigden de Joden onze Heer op de top van Golgotha, en op een vrijdag doodde de Slachter de dood, en verhief onze natuur.

  Eer...~--- Laten we in een vurig gebed vragen, en oprecht om genade smeken, en vergeving vragen aan de Barmhartige, wiens deur openstaat voor allen die naar Hem terugkeren en zich bekeren.
\end{halfparskip}

\markedsubsectionrubric{Zaterdagen ``voor''}

\begin{halfparskip}
  Van het ene uiteinde van de aarde naar het andere.~--- O Kruis, dat de vier uiteinden van de schepping draagt, bewaar in Uw goddelijk mededogen Uw aanbidders.

  Van het opkomen tot aan het ondergaan van de zon.~--- O Kruis..., zoals hierboven.

  Eer...~--- Op de top van Golgotha zag de Kerk Christus, en knielde, aanbad Hem en gaf eer aan Wie Hem zond.

  Vanaf het begin en in alle eeuwigheid.~--- Laat het gebed van Uw dienaren, o Heer, niet tevergeefs zijn, maar moge het zijn voor verzoening, en vergeving van zonden.
\end{halfparskip}

\markedsubsectionrubric{Maandagen ``na''}

\begin{halfparskip}
  Tot U hief ik mijn ogen op, U die in de hemelen woont.~--- Tot U hief ik mijn ogen op, U die in de hemel woont. Want U heeft ons tot bestaan gebracht, en door Uw wil ons geschapen. Zend Uw kracht en genees mijn ziekte, en genees mijn pijnen met het medicijn van Uw medelijden; want U bent de ware Geneesheer die zonder betaling geneest. Genees in Uw genade mijn pijnen en ziekten.

  Moge Uw barmhartigheid ons snel voorafgaan.~--- Moge Uw genade, barmhartige Heer, ons snel voorafgaan. Want zie, onze handen zijn naar U uitgestrekt en ons hart naar de hemel, om U te verzoeken en te smeken dat U onze overtredingen zou vergeven en onze zonden kwijtschelden. Red ons van de Boze die dreigt ons te vernietigen; en versterk ons om Uw wil te kunnen vervullen.

  Eer...~--- Door het gebed van Maria die U heeft gebaard en Johannes die U heeft gedoopt, van Petrus en Paulus, de predikers, de vier evangelisten, Stefanus en de groep leraren; van onze eerbiedwaardige vader en al onze overledenen; van de twaalf belijders en de martelaar Joris, bewaar, onze Heer, ons land en zij die erin wonen.

  Vanaf het begin en in alle eeuwigheid.~--- Door Uw hemelvaart, Heer Jezus, hebt U ons stof(felijke natuur) verheven; en in Uw liefde deed U ons aan de rechterhand van de Vader in de hemel zitten. Door de nederdaling van Uw Geest hebt U onze kindsheid wijs gemaakt. Door het Kruis van Uw licht hebt U onze kennis verlicht. Door de heiliging van de heilige Kerk hebt U onze natuur geheiligd. Aanbiddelijk was Uw leven voor onze
  redding.
\end{halfparskip}

\markedsubsectionrubric{Dinsdagen ``na''}

\begin{halfparskip}
  Red mij en verlos mij voor altijd van deze generatie.~--- God, die het huis van Ezechias redde, en Jeruzalem verloste van de Assyriërs, laat Uw rechterhand Uw aanbidders overschaduwen en hen redden, en laat de voet van de trotse, O Heer, ons niet vertreden.

  In het diepst van zijn hart.~--- Toen de zoon van Isaï tot U riep, Heer, antwoordde U hem. U verlichtte zijn ellende en wierp zijn vijanden omver. Wij roepen tot U: vernietig onze vervolgers, en geef sterkte aan de kerken die Uw Kruis eren.

  Eer...~--- De heilige maagd is een groot toevluchtsoord voor de gelovigen die altijd haar gebeden vragen. Moge onze gemeenschap gezegend worden door de kracht van haar gebed, en moge de Kerk rijk worden in vrede en eendracht.

  Vanaf het begin en in alle eeuwigheid.~--- Profeten, apostelen, martelaren en leraren, smeekt Gods genade voor de wereld, eendracht voor de priesters, vrede voor de koningen, vergeving van overtredingen voor de Kerk en haar kinderen.

  Dat al het volk zegge: amen en amen.~--- Onze heilige vader, de vriend van Christus, die het koninkrijk heeft gekocht door uw deugdzame daden, smeek om genade voor ons bij uw Heer, die u heeft uitgekozen, dat Hij ons genadig mag zijn en onze ziel mag redden.
\end{halfparskip}

\markedsubsectionrubric{Woensdagen ``na''}

\begin{halfparskip}
  Want Hij is uw Heer, aanbid Hem.~--- Maria, moeder van de Koning, de Koning der koningen, smeek Christus, die uit Uw boezem is verschenen, dat Hij medelijden met ons moge hebben in Zijn goedertierenheid, en ons Zijn koninkrijk waardig maken.

  Smeekt Hem en bidt Hem dat Hij medelijden met ons moge hebben.~--- Maria, moeder...

  Eer...~--- Vrede voor uw gezelschap, o gezegenden, kooplieden die leven brengt aan de mensen. Opent de schatkist van uw gebeden voor de behoeftigen, en behoedt het land waar u woonde voor schade.

  Vanaf het begin en in alle eeuwigheid.~--- Doe een verzoek voor ons allen bij uw Heer, o onze eerbiedwaardige en heilige vader, vriend van de Zoon. Moge door uw gebeden worden geholpen en gered allen die gekweld en bedroefd zijn en hun toevlucht bij u zoeken.

  Dat al het volk zegge: Amen en amen.~--- God, die medelijden had met Nineve, heb medelijden met ons, en wend Uw blik niet af van onze slechte generatie. En indien U Uw deur dicht houdt voor ons, zondaars, aan wiens deur gaan wij dan kloppen, o Mensenvriend?
\end{halfparskip}

\markedsubsectionrubric{Donderdagen ``na''}

\begin{halfparskip}
  U bent rechtvaardig, Heer, en Uw oordelen zijn zeer waarachtig.~--- U riep niet de rechtvaardigen tot bekering, maar U beval zondaars zich te bekeren. Bekeer ons in Uw mededogen, Christus onze Verlosser, en in Uw goedertierenheid vergeef onze overtredingen en onze zonden.

  Wend U tot mij en heb medelijden met mij.~--- Wend U tot het gebed van Uw dienaren, Heer, en aanvaard ons verzoek en antwoord onze verzoeken. En in Uw mededogen doe de neiging van de sluwe teniet, dat hij onze samenkomst niet verstoort met de gemeenheid van zijn afgunst.

  Eer...~--- Maria, moeder van de Koning, de Koning der koningen, smeek Christus, die verscheen uit uw boezem, dat Hij medelijden heeft met onze laagheid en dat Hij oorlogen in alle delen van de aarde teniet doet, en de kroon van het jaar zegene in Zijn goedertierenheid.

  Vanaf het begin en in alle eeuwigheid.~--- Door het gebed van Uw heiligen, Christus onze Verlosser, de profeten, apostelen, martelaren en alle rechtvaardigen, bescherm de gemeenschap van Uw aanbidders van alle kwaad, zodat zij U 's~nachts en overdag mogen prijzen.

  Dat al het volk zegge: amen en amen.~--- Onze vader, vriend van de hemelse Bruidegom, moge uw gebed voor ons een muur en een toevluchtsoord zijn, en op de dag dat uw Heer verschijnt in heerlijkheid, smeek Hem dat wij met u het koninkrijk mogen beërven.
\end{halfparskip}

\markedsubsectionrubric{Laatste vrijdag}

\begin{halfparskip}
  Zijn redding is nabij hen die Hem vrezen.~--- De tijd dat deze wereld voorbijgaat, is nabij; en al haar verlangens zullen vernietigd worden en vervallen. Hoort, stervelingen, en komt tot inkeer voordat de stervensdag komt, en iedere man zal vergolden worden naar wat hij heeft gedaan in het vreselijke gerecht.

  Als het ware een op hol geslagen en brullende leeuw.~--- Daar de dood alle mensen treft en verslindt, ben ik zeer ontroerd en doodsbang als het op mij afkomt, want ik ben van mijn overtredingen overtuigd dat ze niet zijn als die van andere mensen, want Hij zag dat ik Zijn deel was. U bent mijn hulp, Verlosser, op U vertrouw ik.

  Eer...~--- Glorie aan U, Jezus, onze zegevierende Koning, die door Uw Kruis ons ras van dwaling hebt gered. Moge Uw grote kracht ons wezen vernieuwen; en moge de dood teniet worden gedaan en de verrijzenis heersen. En mogen we door Uw wil Uw barmhartigheid waardig worden, o Koning en Levendmaker.
\end{halfparskip}

\markedsubsectionrubric{Zaterdagen ``na''}

\begin{halfparskip}
  Moge Hij Zijn kracht tonen.~--- O Kruis, dat prachtige wonderen liet zien aan de mensen, houd het kwade weg van de zielen die met U zijn getekend.

  Toon Uw kracht en kom ons redden.~--- O Kruis..., \liturgicalhint{zoals hierboven}.

  Eer...~--- Heer, wees een muur voor de Kerken in elke plaats; moge Uw waarheid een sterk bolwerk voor hen zijn.

  Vanaf het begin en in alle eeuwigheid.~--- Heer, help, zend vrede, verstrooi de bozen, bewaar Uw Kerk, bevrijd haar kinderen van het kwaad.
\end{halfparskip}

% % % % % % % % % % % % % % % % % % % % % % % % % % % % % % % % % % % % % % % %

\markedsection{Karozuta I}

\begin{halfparskip}
  \dd~Laat ons allen ordelijk staan met berouw en ijver; laat ons bidden en zeggen: Heer, ontferm U over ons.~--- \rr~Heer, ontferm U over ons. \liturgicalhint{(Wordt herhaald na elke aanroeping.)}

  \dd~Vader van barmhartigheid en God van alle troost, wij bidden U,

  \dd~Onze Verlosser, die voor ons zorgt en in alle dingen voorziet, wij bidden U,

  \dd~Voor vrede, eendracht en bestendigheid in de hele wereld en in alle kerken, wij bidden U,

  \dd~Voor ons land, voor alle landen en de gelovigen die er wonen, wij bidden U,

  \dd~Voor de temperatuur van de lucht, een voorspoedig (\translationoptionNl{vruchtbaar}) jaar, een goede oogst (\translationoptionNl{opbrengst}) van fruit en het behoud (\translationoptionNl{stabiliteit}) van de hele wereld, wij bidden U,

  \dd~Voor de gezondheid van onze heilige vaders, Paus \NN , hoofd van de hele Kerk van Christus, van Patriarch \NN , van onze Catholicos \NN , van onze Metropoliet \NN , van onze Bisschop \NN , en van al hun helpers, wij bidden U,

  \optionalindentedpar{\footnote{De invoegingen tussen [\ ] kunnen worden weggelaten, behalve op feesten van Christus en vastenzondagen (Hudra, 414, voetnoot). Breviarium Chaldaicum somt alle aanroepingen zonder enig onderscheid op.}\dd~Voor de heersers die de macht hebben in deze wereld, wij bidden U,}

  \dd~Barmhartige God, die met Uw liefde alles bestuurt, wij bidden U,

  \optionalindentedpar{\dd~Voor rechtgelovige priesters en diakens, en al onze broeders in Christus, wij bidden U,}

  \dd~Gij, rijk in barmhartigheid, en overvloedig in goedheid, wij bidden U,

  \optionalindentedpar{\dd~Gij, die van vóór alle tijden zijt, en wiens macht (\translationoptionNl{rijk}) eeuwig blijft, wij bidden U,}

  \dd~Gij, goed van natuur en Gever van alle goed, wij bidden U,

  \optionalindentedpar{\dd~Gij, die geen behagen schept in de dood van de zondaar, maar wilt dat hij berouw heeft over zijn boosheid en leeft, wij bidden U,}

  \dd~Gij die in de hemel wordt geprezen en op aarde wordt aanbeden, wij bidden U,

  \optionalindentedpar{\dd~Gij, die door Uw heilige~--- geboorte / verschijning / vasten / intrede / verrijzenis / hemelvaart / nederdaling / Kruis~--- de aarde deed verheugen en de hemelen blij zijn, wij bidden U,}

  \dd~Gij, die onsterfelijk zijt van natuur, en die in stralend licht woont, wij bidden U,

  \optionalindentedpar{\dd~Redder van alle mensen en in het bijzonder van hen die in U geloven, wij bidden U,}

  \dd~Verlos ons allemaal, Christus onze Heer, in Uw genade, doe in ons Uw vrede en rust toenemen en ontferm U over ons.
\end{halfparskip}

% % % % % % % % % % % % % % % % % % % % % % % % % % % % % % % % % % % % % % % %

\CLEARPAGEAV

\markedsection{Karozuta II}

\begin{halfparskip}
  \liturgicalLbracket\dd~Laat ons bidden, vrede zij met ons, laat ons God, de Heer van alles bidden en vragen.~--- \rr~Amen\footnote{Het volk antwoordt ``Amen'' na ``Heer van alles'' volgens Breviarium en Hudra \emph{of} op het einde van elke aanroeping. Beiden gewoonten bestaan.}.

  \dd~Dat Hij de stem van ons gebed moge horen, onze bede ontvangen en Zich over ons ontfermen.

  \dd~Voor de heilige katholieke Kerk, hier en in elke plaats, laat ons bidden en vragen God, de Heer van alles, dat Zijn heil en vrede mag vertoeven tot aan het einde van de wereld.

  \dd~Voor onze Vaders, de bisschoppen, laat ons bidden en vragen God, de Heer van alles, dat zij aan het hoofd van hun bisdommen mogen staan zonder blaam of vlek alle dagen van hun leven.

  \dd~Bijzonder voor het welzijn van onze heilige vaders, de Paus, de Patriarch, de Catholicos, de Metropoliet; de Bisschop, laat ons bidden en vragen God, de Heer van alles, dat Hij hen mag bewaren en behouden aan het hoofd van al hun bisdommen, dat zij mogen weiden, dienen en gereed maken voor de Heer, een volmaakt volk, ijverig in goede en mooie werken:

  \dd~Voor de priesters en diakens die in deze dienst van de waarheid zijn, laat ons bidden en vragen God, de Heer van alles, dat zij met een goed hart en zuivere gedachten vóór Hem mogen dienen.

  \dd~Voor alle kuise en heilige leden van het verbond, de kinderen van de heilige katholieke Kerk, laat ons bidden en vragen God, de Heer van alles, dat zij hun goede en heilige levensloop mogen voleinden, en van de Heer de hoop en de belofte in het land van het Leven ontvangen.

  \dd~Voor de gedachtenis van de gezegende mart Maria, de heilige maagd, de moeder van Christus, onze Redder en Levengever, laat ons bidden en vragen God de Heer van alles, dat de H. Geest die in haar verbleef ons moge heiligen in zijn goedheid, in ons Zijn wil vervullen, en ons zegelen in de waarheid alle dagen van ons leven.

  \dd~Voor de herdenking der profeten, apostelen, martelaren en belijders, laat ons bidden en vragen God de Heer van alles, dat door hun gebeden en lijden Hij ons geve met hen: goede hoop en redding, en ons waardig maken van hun gezegende gedachtenis en hun levende en ware belofte in het rijk der hemelen

  \dd~Voor de gedachtenis van onze heilige vaderen, mar Gregorius [Nazianze], mar Basilius, mar Johannes [Chrysostomus], bisschoppen en leraren van de waarheid, mar Efrem, mar Narsai en mar Abraham, en alle heilige, oude en ware leraren, laat ons bidden en vragen God, de Heer van alles, dat door hun gebeden en smeekbeden de zuivere waarheid van de leer van hun godsdienst en van hun geloof bewaard kunnen blijven in heel de heilige katholieke Kerk tot aan het einde van de wereld.

  \dd~Voor de gedachtenis van onze vaders en broers, ware gelovigen, die zijn vertrokken en gegaan uit deze wereld in dit ware geloof en de orthodoxe godsdienst, laat ons bidden en vragen God, de Heer van alles, dat Hij hun overtredingen en fouten moge vergeven en wegnemen en ze waardig maken vreugde te hebben met de heiligen en rechtvaardigen die door Zijn wil zijn goedgekeurd.

  \dd~Voor dit land en haar inwoners; voor deze stad/dorp en degenen die erin wonen, voor dit huis en zij die er zorg voor dragen, en vooral voor deze gemeenschap, laat ons bidden en vragen God, de Heer van alles, dat Hij in Zijn goedheid van ons moge verwijderen het zwaard, gevangenschap, diefstal, aardbevingen, hongersnood, hagel, pest en alle boze plagen die vijandig zijn aan het lichaam.

  \dd~Voor degenen die dwalen van dit ware geloof en worden gehouden in de strikken van Satan, laat ons bidden en vragen God, de Heer van alles, dat Hij de hardheid van hun harten mogen doen omkeren, en hen doen weten dat God is één, de Vader van de waarheid, en Zijn Zoon Jezus Christus onze Heer.

  \dd~Voor hen die zwaar ziek zijn en bekoord door boze geesten, laat ons bidden en vragen God, de Heer van alles, dat Hij hen, in de overvloed van Zijn goedheid en barmhartigheid, Zijn engel van barmhartigheid en genezing moge sturen, om hen te bezoeken, te genezen, te sterken, te helpen en te troosten.

  \dd~Voor de armen, verdrukten, wezen, weduwen, gekwelden, verwarden en bedroefden in geest in deze wereld, laat ons bidden en vragen God de Heer van alles, dat Hij hen in Zijn goedheid moge geven wat ze nodig hebben, in Zijn genade hen verzorgen, in Zijn mededogen hen troosten, en hen redden van hem die hen schandelijk misbruikt.

  \dd~Bidt en vraagt God de Heer van alles dat gij voor Hem een koninkrijk van heilige priesters en volk moge zijn. Roept tot de almachtige Heer God met heel uw hart en gans uw ziel. Want Hij is God de Vader van mededogen, barmhartig en genadig, die niet wil dat zij die Hij heeft gevormd verloren zouden gaan, maar bekeren en leven vóór Hem. Vooral moeten we bidden, belijden, aanbidden, verheerlijken, eren en verheffen de ene God, de aanbiddelijke Vader, Heer van alles, die door zijn Christus goede hoop en verlossing gaf aan onze zielen, dat Hij moge vervullen in ons Zijn goedheid, barmhartigheid en mededogen tot het einde.~--- \rr~Amen.\liturgicalRbracket
\end{halfparskip}

% % % % % % % % % % % % % % % % % % % % % % % % % % % % % % % % % % % % % % % %

\markedsection{Karozuta III}

\begin{halfparskip}
  \liturgicalLbracket\dd~Laat ons vragen door gebed en smeking voor de engel van vrede en barmhartigheid.

  \rr~Van U, Heer. \liturgicalhint{(na elke aanroeping)}

  \dd~Dag en nacht, alle dagen van ons leven vragen wij het behoud van de vrede voor de Kerk en een leven zonder zonde.

  \dd~De eenheid in de liefde, de band van volmaaktheid, vragen wij door de werking van de Heilige Geest,

  \dd~Vergeving der zonden en alles wat nodig is voor ons leven en dat behaagt aan Uw Godheid vragen wij,

  \dd~De barmhartigheid van de Heer en Zijn goedheid vragen wij altijd,\liturgicalRbracket

  \dd~Wij bevelen onszelf en ieder van ons aan, aan de Vader, de Zoon en de H. Geest.~--- \rr~Aan U, Heer.

  \cc~Aan U, Heer, almachtige God, vertrouwen wij onze lichamen en zielen toe; en van U, onze Heer en onze God, vragen we vergeving van onze overtredingen en zonden; geef ons dit in Uw goedheid en barmhartigheid, zoals Gij gewoon zijt, ten allen tijde, Heer van alles, Vader....~--- \rr~Amen.
\end{halfparskip}

% % % % % % % % % % % % % % % % % % % % % % % % % % % % % % % % % % % % % % % %

\markedsection{Trisagion \&\ conclusie van Ramsa}

\dd~Verheft uw stem, geheel het volk, en prijst de levende God.

\vspace{\parskip}
\begin{doublecols}
  \fontsize{11}{12}\selectfont

  \dutchl \rr~Qadisha Alaha / qadisha Hailthana / qadisha Lamayotha, ethrahaim alein.

  \dutchc{1} \rr~Heilige God, heilige Machtige, heilige Onsterfelijke, ontferm U over ons.
\end{doublecols}

\begin{halfparskip}
  Eer aan...~--- Heilige God...~--- Vanaf het begin en in alle eeuwigheid, amen.~--- Heilige God...

  \dd~Laat ons bidden; vrede zij met ons.

  \cc~Heilige, Roemrijke, Almachtige en Onsterfelijke, Gij die leeft temidden van Uw heiligen, Gij vindt Uw behagen in hun gezelschap. Wij smeken U, kom naar ons met barmhartigheid en ontferm U over ons, zoals Gij altijd doet, Heer van alles, Vader, Zoon en Heilige Geest, in alle eeuwigheid.~--- \rr~Amen.

  \dd~Zegen, mijn Heer. Buigt uw hoofden voor de handoplegging en ontvang de zegen.

  \cc~\liturgicalhint{(zegent de diaken.)} Moge Christus uw dienst heerlijk maken in het koninkrijk der hemelen \liturgicalhint{(De voorhand van het heiligdom wordt nu geopend.)}

  \cc~Mogen onze zielen worden vervolmaakt in het ene volledige geloof van Uw glorieuze Drie-eenheid, en mogen wij allen in één eenheid van liefde waardig zijn om ten allen tijde heerlijkheid, eer, belijdenis en aanbidding tot U te verheffen, Heer van alles...
\end{halfparskip}

% % % % % % % % % % % % % % % % % % % % % % % % % % % % % % % % % % % % % % % %

\markedsection{Onita D'Ramsa \markedsectionhint{(Eigen tekst van de dag: zie Hudra of Gaza, behalve op Woensdagen\footnote{We hebben de oniyata d'ramsa van Maclean toegevoegd als vervanging voor het geval de volledige liturgische tekst niet beschikbaar is voor de gebruiker. Die voor de woensdagen staan vast, aangezien er voor individuele woensdagen geen eigen teksten worden gegeven.}.)}}

\markedsubsectionrubric{Maandagen ``voor''}

\begin{halfparskip}
  Ik wachtte geduldig op de Heer.~--- Het lichaam van Christus en Zijn kostbaar bloed zijn op het heilig altaar. Laat ons allen naderbij komen, met vrees en liefde, en met de engelen Hem toezingen: Heilig, heilig, heilig, de Heer God.

  De armen zullen eten en verzadigd zijn.~--- Het lichaam van Christus..., \liturgicalhint{zoals hierboven}.

  Eer...~--- Christus, toevlucht en ware hoop der lijdenden, wees, mijn Heer, een muur voor Uw aanbidders, en bewaar hen voor de Boze. Genees en verzacht hun pijn in het mededogen van Uw Godheid, o Barmhartige en Vergever van zonden.
\end{halfparskip}

\markedsubsectionrubric{Dinsdagen ``voor''}

\begin{halfparskip}
  Onze hulp is in de Naam van de Heer.~--- Onze hulp komt van God, die ons allen in Zijn barmhartigheid straft, want Hij is de Gever van ons leven. De hoop op de redding van onze ziel zal nooit meer worden afgesneden, maar laten we roepen en zeggen: Bewaar ons, Heer, in Uw mededogen, en heb medelijden met ons.

  En onze Helper in tijden van nood.~--- Onze hulp..., \liturgicalhint{zoals hierboven}.

  Eer...~--- Christus, die bij Uw komst de hele schepping verzoend hebt met Hem die U gezonden heeft, heb medelijden met Uw Kerk, gered door Uw bloed. Maak een einde in haar aan de tweedracht en verdeeldheid, die de duivel toelaten het wonderbaarlijke bestel van Uw menswording te betreden; en doe in haar priesters opstaan om het gezonde geloof te prediken.
\end{halfparskip}

\markedsubsectionrubric{Woensdagen ``voor''}

\begin{halfparskip}
  Heilig is de tabernakel van de Allerhoogste.~--- Maria droeg met grote glorie de tempel van de Zoon, het Woord, in haar boezem. Zij was de moeder en dienstmaagd van Jezus de Verlosser van allen. Daarom verblijden alle wezens zich op de dag van haar feest, en worden uitgenodigd naar de bruidkamer van licht en eindeloze vreugde. En wij allemaal, met alle generaties, zullen haar ``meest gezegend'' noemen. Eer aan Hem die haar heeft uitgekozen als de verblijfplaats van Zijn glorieus Beeld.

  Ik zal het Woord van God verheerlijken.~--- Maria droeg...

  Eer...~--- De beenderen der heiligen zijn als mooie parels, vastgehecht in de kroon van de Koning, en hun mooiheid schittert in de schepping. Komt, laat ons allemaal met volmaakte liefde de dag van hun herdenking eren met gezangen van de Heilige Geest, van 's~morgens tot 's~avonds; en laten we van hun Heer om genade smeken, dat Hij zijn vrede mag doen wonen in de schepping; en dat de Kerk en haar kinderen door hun gebeden
  mogen behouden blijven.

  Vanaf het begin en in alle eeuwigheid.~--- Gezegend is uw gedachtenis, onze eerbiedwaardige vader, want om de waarheid werd u vervolgd en doorstond u leed en beproevingen, opdat u een erfgenaam van het koninkrijk moogt worden. Wie kan de loop van uw geestelijke daden verwoorden? Want u was versierd met zuiverheid, waken, vasten en gebed. Moge uw gebed een toevluchtsoord zijn voor zondaars die hun toevlucht bij u zoeken, en mogen wij waardig zijn uw Heer, die u heeft verheven, te eren.

  Dat al het volk zegge: amen en amen.~--- Zoals het wierookvat dat Aaron aanbood, moge de zoete geur van onze gemeenschap U behagen. En zoals de petitie der Ninevieten, ontvang het gebed van Uw dienaren, Heer. En in Uw barmhartigheid geef een antwoord op onze verzoeken uit Uw rijke schatkist. En zoals U Daniël verhoorde in de leeuwenkuil, verhoor, Heer, en help Uw aanbidders in deze tijden van ellende. Want in U is onze hoop, o Heer, die Uw dienaren liefheeft.
\end{halfparskip}

\markedsubsectionrubric{Donderdagen ``voor''}

\begin{halfparskip}
  We zullen u altijd danken.~--- Wij danken U, onze Schepper, want U ontfermt U over heel de schepping. In Uw goedertierenheid hebt U zowel het goede als het kwade tot bestaan gebracht, en U biedt overvloedig hulp aan de mensen. Eer aan U, die overvloedige barmhartigheid hebt.

  Om Uw liefdevolle barmhartigheid en omwille van Uw waarheid.~--- Wij danken U...

  Eer...~--- Open voor ons, o Heer, de grote schat van Uw genade, dat we barmhartigheid en redding mogen ontvangen voor ons arm ras, en genees de pijnen van onze zonden door het grote medicijn van Uw mededogen, zodat ons zwakke ras door Uw genade erbarmen mag ontvangen.
\end{halfparskip}

\markedsubsectionrubric{Vrijdagen ``voor''}

\begin{halfparskip}
  Wie kan de nobele daden van de Heer tot uitdrukking brengen?~--- Wie kan, o onze Maker, U voldoende dankzeggen voor Uw genade? Gij, die ons in het begin vormde naar Uw schitterend beeld, en die in de laatste tijden Uzelf met ons bekleed hebt en ons tot kennis van U gebracht. U die ons ras verheft, eer aan U.

  Wie is zoals U, Heer?~--- Wie kan de nobele, \liturgicalhint{enz}.~--- Eer aan...~--- Wie kan de nobele, \liturgicalhint{enz}.
\end{halfparskip}

\markedsubsectionrubric{Vrijdagen ``midden''}

\begin{halfparskip}
  Wie kan de nobele daden van de Heer uitdrukken?~--- Wie kan voldoende Uw wonderen verhalen, Christus Koning, onze Verlosser? Want U, glorieuzer dan allen, werd geopenbaard in de eerste Vrucht die van ons is, want daarom wilde U zij die in U geloven redden, en de vorm aannemen van een slaaf, en verschijnen in de wereld in een lichaam.

  Wie is zoals U, Heer?~--- Wie kan... \liturgicalhint{als hierboven}.~--- Eer aan...~--- Wie kan...
\end{halfparskip}

\markedsubsectionrubric{Zaterdagen ``voor''}

\begin{halfparskip}
  Heer, ik heb dagelijks tot U geroepen.~--- Tot U roepen we: Wees ons genadig, Christus onze Verlosser.

  Tot U heb ik geroepen, Heer mijn God, zwijg niet.~--- Tot U roepen we...

  Eer...~--- Laten we de Zoon, die ons heeft gered door Zijn Kruis, glorie brengen en belijden.

  Vanaf het begin en in alle eeuwigheid.~--- Moge ons gebed, o Heer Uw goede wil behagen, al onze dagen.
\end{halfparskip}

\markedsubsectionrubric{Maandagen ``na''}

\begin{halfparskip}
  God, mijn hart is gereed, mijn hart is gereed.~--- Laten we klaar staan in angst en liefde voor de vreeswekkende gave van de mysteries van Christus, en onze ziel versieren met daden, waarmee we de Rechter van allen gunstig kunnen stemmen, dat Hij medelijden met ons moge hebben wanneer Hij de families
  van de aarde oordeelt.

  En we zijn opgestaan en staan klaar.~--- Laten we klaar staan..., \liturgicalhint{zoals hierboven}.

  Eer...~--- Door onze eigen wilskeuze zijn we omgeven door het lijden dat zonde met zich meebrengt\footnote{\emph{Letterlijk,} Uit vrije wil heeft de lijden van de zonde ons omringd.}. In Uw goedheid open de deur voor ons gebed. Onze natuur is zwak en geneigd tot zonde. Reik ons nu een hand, want we zijn laag gevallen. Wellicht zal het nu zijn ware aard inzien door de barmhartigheid van Uw genade.
\end{halfparskip}

\markedsubsectionrubric{Dinsdagen ``na''}

\begin{halfparskip}
  Wie kan de nobele daden van de Heer uitdrukken?~--- Wie kan, o onze Maker, U voldoende dankzeggen voor Uw genade? Gij, die ons in het begin vormde naar Uw schitterend beeld, en die in de laatste tijden Uzelf met ons bekleed hebt en ons tot kennis van U gebracht. Eer aan U die ons ras verheft.

  Omwille van Uw goedertierenheid en waarheid.~--- Wie kan..., \liturgicalhint{zoals hierboven}.

  Eer...~--- Open voor ons, o Heer, de grote schat van Uw genade, dat we barmhartigheid en redding mogen ontvangen voor ons arme ras, en genees de pijnen van onze zonden door het grote medicijn van Uw mededogen, zodat ons zwakke ras door Uw genade erbarmen mag ontvangen.
\end{halfparskip}

\markedsubsectionrubric{Woensdagen ``na''}

\begin{halfparskip}
  Bescherm mij tegen de slechtheid van de Aanvaller.~--- Onder de vleugels van uw gebeden, zuivere Maria, zoeken we altijd onze toevlucht. Mogen ze ten allen tijde bij ons blijven; en mogen we dankzij hen genade en barmhartigheid op de oordeelsdag vinden.

  Bewaar en bescherm mij tegen mijn vijanden.~--- Onder de vleugels..., \liturgicalhint{zoals hierboven}.

  Eer...~--- Door het gebed der profeten die Uw mysteries hebben voorspeld; en door het gebed der apostelen die Uw evangelie predikten; en door het gebed der martelaren, priesters en leraren, Christus, behoed Uw aanbidders voor het kwaad.

  Vanaf het begin en in alle eeuwigheid.~--- Onze vader, bid dat onze Heer in Zijn genade alle pijn mag helen; moge uw gebed zijn als de geur van zoete wierook en een wierookvat dat God gunstig stemt voor de zondaars.

  Laat al het volk zeggen: Amen en amen.~--- Luister naar ons verzoek, o Hoop van ons leven, en in Uw genade beantwoord het verzoek van onze ziel. En geef ons in (Uw) goedheid heling van onze ziekten, opdat wij Uw heilige Naam mogen belijden, en ontferm U over ons.
\end{halfparskip}

\markedsubsectionrubric{Donderdagen ``na''}

\begin{halfparskip}
  Hoort dit allemaal, volkeren.~--- Natuur der stervelingen, laat uw hoop toenemen, want uw Levendmaker is gekomen. De mysteries van de profetie zijn tot hun einde gekomen, want het Licht werd naar de wereld gestuurd. De hemelse wezens hebben het evangelie van vrede gepreekt aan de aardse wezens. Gezegend is Hij die de naties van dwaling heeft afgekeerd.

  De armen zullen ervan horen en zich verheugen.~--- Natuur der stervelingen...

  Eer...~--- Christus, toevlucht en ware hoop der lijdenden, wees, O mijn Heer, een muur voor Uw aanbidders, en bewaar ze van de Boze. Genees en verzacht hun pijnen, in Uw goddelijk mededogen, O Barmhartige, die de zonden vergeeft.
\end{halfparskip}

\markedsubsectionrubric{Laatste vrijdagen}

\begin{halfparskip}
  Vertel van dag tot dag over Zijn verlossing.~--- Het evangelie van vrede en liefde in barmhartigheid is aan ons gepredikt, door de glorieuze geboorte [/openbaring] van Christus. Daardoor hebben we geleerd dat de Schepper met ons verzoend is; en dat Hij in ons Zijn glorieus beeld, dat wij vernietigd hadden, heeft hersteld en aan ons heeft teruggegeven, door de eerste Vrucht van ons aan te nemen, en door onze natuur te doen deelhebben aan Zijn Majesteit in de glorie.

  De hoop van alle uiteinden van de aarde.~--- Het evangelie van vrede...~--- Eer...~--- Het evangelie van vrede...
\end{halfparskip}

\markedsubsectionrubric{Zaterdagen ``na''}

\begin{halfparskip}
  Ik zal de Heer altijd zegenen.~--- Gezegend is de Hogepriester die onze overtredingen heeft vergeven door het offer van Zichzelf.

  En gezegend zij de Naam van Zijn majesteit in eeuwigheid.~--- Gezegend is de Hogepriester...

  Eer...~--- Laten we de Zoon eren en danken, Hij die ons heeft gered door Zijn Kruis.

  Vanaf het begin en in alle eeuwigheid.~--- Moge ons gebed, Heer, behagen aan Uw goede wil, al onze dagen.
\end{halfparskip}

\begin{halfparskip}
  \liturgicalhint{Gewone weekdagen.} \cc~Heb medelijden met ons, o Barmhartige, in Uw goedheid, en wend u tot ons, O Genadevolle. Wend Uw ogen niet af van ons, O Heer, want in U is onze hoop en ons vertrouwen in alle seizoenen en tijden, Heer van alles, Vader...
\end{halfparskip}

\begin{halfparskip}
  \liturgicalhint{Woensdagen.} Rust ons uit, Heer onze God, met een sterk en ondoordringbaar harnas, door de gebeden van Uw gezegende moeder, en geef ons met haar een portie en een deel in Uw hemelse bruidkamer, Heer van alles in alle eeuwigheid.
\end{halfparskip}

\begin{halfparskip}
  \liturgicalhint{Vrijdagen.} Maak levend, Heer, onze overledenen in Uw mededogen, en zet hen aan Uw rechterhand. Bekleed hen met voortreffelijke heerlijkheid in Uw koninkrijk, en voeg hen bij de rechtvaardigen en gerechtigen die Uw wil vervullen in het hemelse Jeruzalem, Heer van onze dood en van ons leven, Vader... Amen.
\end{halfparskip}

% % % % % % % % % % % % % % % % % % % % % % % % % % % % % % % % % % % % % % % %

\markedsection{Suraya / Athwatha (letterpsalm) \markedsectionhint{Met 3x alleluia, herhaal eerste vers; op het einde: Eer... 3x alleluia.}}

\markedsubsectionrubricwithhint{Maandagen ``voor''.}{\Ps{118,1--16}.}

\begin{halfparskip}
  Gelukkig zij, die smetteloos zijn op hun levensweg,

  \liturgicalhint{3x alleluia. --- Eerste vers.}

  \sep\ die wandelen naar de Wet van de Heer.

  Gelukkig zij, die Zijn voorschriften volbrengen,~\sep\ Hem zoeken met heel hun hart,

  Die geen ongerechtigheid plegen,~\sep\ maar op Zijn wegen wandelen.

  Gij hebt Uw bevelen gegeven,~\sep\ opdat men ze stipt zou volbrengen.

  Mochten mijn wegen standvastig zijn,~\sep\ om Uw verordeningen na te leven.

  Dan zal ik niet te schande worden,~\sep\ als ik acht geef op al Uw geboden.

  In oprechtheid des harten zal ik U prijzen,~\sep\ als ik de besluiten van Uw gerechtigheid ken.

  Uw verordeningen zal Ik volbrengen;~\sep\ wil mij in het geheel niet verlaten!

  Hoe houdt een jongeling zijn levenspad rein?~\sep\ Door het onderhouden van Uw woord.

  Met heel mijn hart dan zoek ik U;~\sep\ laat mij van Uw geboden niet wijken.

  Ik houd in mijn hart Uw uitspraak verborgen,~\sep\ om niet te zondigen tegen U.

  Gezegend zijt Gij, o Heer;~\sep\ leer mij toch Uw verordeningen.

  Met mijn lippen verkondig ik~\sep\ al de besluiten van Uw mond.

  Over de weg van Uw voorschriften verheug ik mij,~\sep\ als over alle rijkdom.

  Uw bevelen zal ik overpeinzen,~\sep\ en acht geven op Uw wegen.

  Ik zal mij verheugen over Uw verordeningen;~\sep\ en Uw woorden niet vergeten!
\end{halfparskip}

\markedsubsectionrubricwithhint{Dinsdagen ``voor''.}{\Ps{118,17--32}.}

\begin{halfparskip}
  Doe wel aan Uw dienaar, opdat ik leve,~\sep\ en Ik zal Uw woorden onderhouden.

  \liturgicalhint{3x alleluia. --- Eerste vers.}

  Open mijn ogen,~\sep\ om de wonderen van Uw Wet te beschouwen.

  Ik ben een vreemdeling op aarde,~\sep\ verberg mij Uw geboden niet.

  Mijn ziel kwijnt weg~\sep\ van aanhoudend verlangen naar Uw besluiten.

  Gij hebt de trotsen berispt;~\sep\ vervloekt, die Uw geboden verlaten.

  Neem smaad en verachting van mij weg;~\sep\ want Uw voorschriften volg ik op.

  Zelfs als vorsten vergaderen en tegen mij plannen beramen,~\sep\ dan nog overweegt Uw dienaar Uw verordeningen.

  Want Uw voorschriften zijn mijn geneugte,~\sep\ Uw verordeningen mijn raadgevers.

  Mijn ziel ligt neer in het stof;~\sep\ geef mij nieuw leven, volgens Uw belofte.

  Ik heb mijn wegen opengelegd, en Gij hebt mij verhoord;~\sep\ leer mij Uw verordeningen!

  Onderricht mij in de weg van Uw geboden,~\sep\ en ik zal Uw wonderwerken overwegen.

  Ik stort tranen van droefheid;~\sep\ richt mij op naar Uw woord.

  Houd mij van dwaalwegen af,~\sep\ maar schenk mij vrijgevig Uw Wet.

  Ik heb de weg der waarheid gekozen,~\sep\ Uw besluiten mij voor ogen opengesteld.

  Ik houd mij vast aan Uw voorschriften;~\sep\ maak mij niet te schande, o Heer.

  Ik zal de weg van Uw geboden bewandelen,~\sep\ als Gij mijn hart hebt verruimd.
\end{halfparskip}

\markedsubsectionrubricwithhint{Woensdagen ``voor''.}{(Suraya.) \Ps{45,14--17}.}

\begin{halfparskip}
  In volle luister treedt de dochter van de Koning binnen;~\sep\ met goud doorweven is haar gewaad.

  \liturgicalhint{3x alleluia. --- Eerste vers.}

  In een kleurige mantel wordt zij voor de Koning geleid,~\sep\ in haar gevolg worden maagden, haar gezellinnen, tot u gevoerd;

  Zij worden voorgeleid in blijde jubel,~\sep\ en treden het paleis van de Koning binnen.
\end{halfparskip}

\markedsubsectionrubricwithhint{Donderdagen ``voor''.}{\Ps{118,49--64}.}

\begin{halfparskip}
  Gedenk Uw woord tot Uw dienstknecht,~\sep\ waardoor Gij mij hoop hebt gegeven.

  \liturgicalhint{3x alleluia. --- Eerste vers.}

  Dit is mijn troost in mijn droefheid,~\sep\ dat Uw uitspraak mij het leven schenkt.

  Trotsen honen mij zeer,~\sep\ maar van Uw Wet wijk ik niet af.

  Ik gedenk Uw aloude uitspraken, O Heer,~\sep\ en ik voel mij getroost.

  Verontwaardiging grijpt mij aan vanwege de zondaars,~\sep\ die Uw Wet verlaten.

  Uw verordeningen zijn mij gezangen geworden,~\sep\ in het oord van mijn ballingschap.

  Des nachts, O Heer, gedenk ik Uw Naam,~\sep\ en ik wil Uw Wet onderhouden.

  Dit is mij ten deel gevallen,~\sep\ omdat ik Uw bevelen heb nageleefd.

  Mijn deel, O Heer, zo sprak ik,~\sep\ is het onderhouden van Uw woorden.

  Van ganser harte bid ik tot Uw aanschijn:~\sep\ wees mij genadig naar Uw belofte.

  Ik dacht over mijn levensweg na,~\sep\ en richtte naar Uw voorschriften mijn schreden.

  Ik heb mij gehaast, en gedraald heb ik niet,~\sep\ Uw geboden te onderhouden.

  De strikken der bozen hielden mij omstrengeld;~\sep\ Uw Wet vergat ik niet.

  Te middernacht sta ik op om U te prijzen,~\sep\ voor Uw rechtvaardige uitspraken.

  Ik ben de vriend van allen, die U vrezen,~\sep\ en Uw bevelen onderhouden.

  De aarde is vol van Uw goedheid, o Heer;~\sep\ leer mij Uw verordeningen.
\end{halfparskip}

\noindent\begin{minipage}{\textwidth}
  \markedsubsectionrubricwithhint{Vrijdagen ``voor''.}{(Suraya of eigen aan de dag.)~\Ps{115,11--13}.}
\end{minipage}

\begin{halfparskip}
  Wat zal ik de Heer weergeven,~\sep\ voor alles, wat Hij mij schonk?

  \liturgicalhint{3x alleluia. --- Eerste vers.}

  De kelk van het heil zal ik heffen,~\sep\ en aanroepen de Naam van de Heer.

  Ik zal de Heer mijn geloften volbrengen,~\sep\ in het bijzijn van geheel Zijn volk.
\end{halfparskip}

\markedsubsectionrubricwithhint{Vrijdagen ``midden''.}{(Suraya.)~\Ps{40,6--8}.}

\begin{halfparskip}
  Wilde ik ze verhalen en verkondigen:~\sep\ ze zijn te talrijk om te worden geteld.

  \liturgicalhint{3x alleluia. --- Eerste vers.}

  Slacht-- noch spijsoffer hebt Gij gewild,~\sep\ maar Gij hebt mij de oren geopend.

  Brand-- noch zoenoffer hebt Gij voor de zonde geëist:~\sep\ toen heb ik gezegd: ``Zie, ik kom; in de boekrol staat over mij geschreven''.
\end{halfparskip}

\markedsubsectionrubricwithhint{Zaterdagen ``voor''.}{\Ps{118,65--88}.}

\begin{halfparskip}
  Gij hebt Uw dienstknecht welgedaan,~\sep\ o Heer, zoals Gij beloofd hebt.

  \liturgicalhint{3x alleluia. --- Eerste vers.}

  Geef mij oordeel en inzicht,~\sep\ want ik vertrouw op Uw geboden.

  Eer ik getuchtigd werd, dwaalde ik af;~\sep\ maar nu houd ik mij aan Uw uitspraak.

  Gij zijt goed en weldadig;~\sep\ leer mij Uw verordeningen.

  De trotsen beramen listige plannen tegen mij,~\sep\ toch leef ik van ganser harte Uw bevelen na.

  Gestold als vet is hun hart;~\sep\ ik vind in Uw Wet mijn vreugde.

  Het was goed voor mij, dat ik werd gekastijd,~\sep\ opdat ik Uw verordeningen zou leren.

  De Wet van Uw mond is mij meer,~\sep\ dan duizenden in goud en zilver.

  Uw handen hebben mij gemaakt en gevormd;~\sep\ onderricht mij, opdat ik Uw geboden leer kennen.

  Die U vrezen, zullen mij zien en zich verheugen,~\sep\ omdat ik op Uw woord heb vertrouwd.

  Ik weet, o Heer, dat Uw besluiten rechtvaardig zijn,~\sep\ en terecht hebt Gij mij gekastijd.

  Uw barmhartigheid sta mij bij om mij te troosten,~\sep\ volgens de belofte, aan Uw dienaar gedaan.

  Dat Uw ontferming op mij neerdale,~\sep\ opdat ik leve, daar Uw Wet mijn vreugde is.

  Schande aan de trotsen, want ten onrechte kwellen zij mij;~\sep\ ik zal Uw bevelen overwegen.

  Mogen zich tot mij wenden, die U vrezen,~\sep\ en op Uw voorschriften bedacht zijn.

  Mijn hart zij volmaakt in Uw verordeningen,~\sep\ opdat ik niet te schande worde.

  Mijn ziel kwijnt van verlangen naar Uw hulp;~\sep\ ik stel mijn hoop op Uw woord.

  Mijn ogen kwijnen van verlangen naar Uw uitspraak;~\sep\ wanneer zult Gij mij troosten?

  Want ik ben als een lederen zak in de rook,~\sep\ toch vergat ik Uw verordeningen niet.

  Hoeveel dagen zijn Uw dienaar beschoren?~\sep\ Wanneer voltrekt Gij Uw oordeel aan mijn vervolgers?

  De trotsen hebben mij kuilen gegraven,~\sep\ zij, die niet handelen naar Uw Wet.

  Waarachtig zijn al Uw geboden;~\sep\ men vervolgt mij ten onrechte; kom mij te hulp!

  Men heeft mij haast van de aarde verdelgd;~\sep\ maar Uw bevelen verwaarloosde ik niet.

  Spaar mijn leven naar Uw erbarming,~\sep\ en de voorschriften van Uw mond zal ik volgen.
\end{halfparskip}

\markedsubsectionrubricwithhint{Maandagen ``na''.}{\Ps{118,89--104}.}

\begin{halfparskip}
  Eeuwig, o Heer, blijft Uw woord,~\sep\ het staat vast als de hemel.

  \liturgicalhint{3x alleluia. --- Eerste vers.}

  Van geslacht tot geslacht blijft Uw trouw;~\sep\ Gij hebt de aarde gegrondvest en zij houdt stand.

  Volgens Uw besluiten blijven zij immer bestaan,~\sep\ omdat alles U dienstbaar is.

  Als niet Uw Wet mijn vreugde was,~\sep\ reeds was ik in mijn ellende vergaan.

  Nimmer zal ik Uw bevelen vergeten,~\sep\ want daardoor deedt Gij mij leven.

  Ik ben de uwe: wees mij tot redding,~\sep\ omdat ik uitzag naar Uw bevelen.

  Zondaars wachten mij op om mij te verderven;~\sep\ ik geef op Uw voorschriften acht.

  Begrensd zag ik alle volmaaktheid,~\sep\ maar onbeperkt strekt Uw gebod zich uit.

  Hoe lief heb ik Uw Wet, o Heer;~\sep\ de gehele dag overweeg ik haar.

  Uw gebod maakte mij wijzer dan mijn vijanden,~\sep\ want het staat mij eeuwig ter zijde.

  Verstandiger ben ik dan al mijn leraars,~\sep\ omdat ik Uw voorschriften overweeg.

  Ik ben scherper van inzicht dan grijsaards,~\sep\ omdat ik Uw bevelen onderhoud.

  Van alle verkeerde wegen houd ik mijn schreden af,~\sep\ om Uw woorden na te leven.

  Ik wijk niet af van Uw besluiten,~\sep\ want Gij hebt mij onderwezen.

  Hoe zoet voor mijn gehemelte zijn Uw uitspraken,~\sep\ zoeter dan honing voor mijn mond!

  Door Uw bevelen krijg ik inzicht,~\sep\ daarom haat ik iedere weg van ongerechtigheid.
\end{halfparskip}

\markedsubsectionrubricwithhint{Dinsdagen ``na''.}{\Ps{118,113--128}.}

\begin{halfparskip}
  Ik haat de wankelmoedigen,~\sep\ maar Uw Wet heb ik lief.

  \liturgicalhint{3x alleluia. --- Eerste vers.}

  Gij zijt mijn Beschermer en mijn schild,~\sep\ ik vertrouw op Uw woord.

  Gij, bozen, gaat van mij heen:~\sep\ en ik zal de geboden van mijn God onderhouden.

  Sterk mij naar Uw belofte, opdat ik leef;~\sep\ stel mijn hoop niet teleur.

  Sta mij bij, en ik zal behouden zijn,~\sep\ en op Uw verordeningen zal ik acht slaan altijd.

  Die Uw verordeningen verlaten, verwerpt Gij,~\sep\ want hun gedachten zijn bedrieglijk.

  Als afval beschouwt Gij alle bozen op aarde,~\sep\ daarom heb ik Uw voorschriften lief.

  Van vreze beeft mijn vlees voor U,~\sep\ en ik heb ontzag voor Uw besluiten.

  Recht en gerechtigheid heb ik beoefend;~\sep\ lever mij niet over aan mijn verdrukkers.

  Sta borg voor het welzijn van Uw dienaar,~\sep\ opdat de trotsen mij niet verdrukken.

  Mijn ogen kwijnen van verlangen naar Uw hulp,~\sep\ en naar Uw rechtvaardige uitspraak.

  Handel met Uw dienaar naar Uw goedheid,~\sep\ en leer mij Uw verordeningen.

  Ik ben Uw dienstknecht, onderricht mij,~\sep\ opdat ik Uw voorschriften kenne.

  Voor de Heer is het tijd om te handelen:~\sep\ zij hebben Uw Wet verkracht.

  Daarom heb ik Uw geboden lief,~\sep\ meer dan goud en het edelst metaal.

  Daarom koos ik al Uw bevelen tot mijn deel;~\sep\ van iedere dwaalweg heb ik een afschuw.
\end{halfparskip}

\markedsubsectionrubricwithhint{Woensdagen ``na''.}{(Suraya.)~\Ps{Ex 15,20v}.}

\begin{halfparskip}
  Maria, de profetes, de zuster van Aäron, nam de tamboerijn in de hand,~\sep\ en terwijl alle vrouwen met
  tamboerijnen haar dansende volgden, herhaalde Maria voor hen het refrein:

  \liturgicalhint{3x alleluia. --- Eerste vers.}

  ``Ik wil de Heer bezingen, want Hij is hoogverheven.
\end{halfparskip}

\markedsubsectionrubricwithhint{Donderdagen ``na''.}{\Ps{118,145--160}.}

\begin{halfparskip}
  Ik roep uit heel mijn hart: verhoor mij, Heer;~\sep\ Uw verordeningen leef ik na.

  \liturgicalhint{3x alleluia. --- Eerste vers.}

  Ik roep tot U; behoud mij,~\sep\ en ik zal Uw voorschriften onderhouden.

  Ik kom bij de dageraad en roep om Uw hulp,~\sep\ ik vertrouw op Uw woorden.

  Vóór de nachtwaken zijn mijn ogen geopend,~\sep\ om Uw uitspraak te overwegen.

  Hoor mijn smeken, Heer, naar Uw barmhartigheid,~\sep\ en schenk mij leven naar Uw besluit.

  Die mij boosaardig vervolgen, naderen mij,~\sep\ ver zijn zij verwijderd van Uw Wet.

  Gij zijt nabij, o Heer,~\sep\ en waarachtig zijn al Uw geboden.

  Reeds vroeger heb ik uit Uw voorschriften begrepen,~\sep\ dat Gij ze gegeven hebt voor eeuwig.

  Zie mijn ellende en bevrijd mij,~\sep\ want Uw Wet heb ik niet vergeten.

  Verdedig mijn zaak, en verlos mij;~\sep\ naar Uw uitspraak schenk mij het leven.

  Ver blijft het heil van de zondaars,~\sep\ want zij storen zich niet aan Uw verordeningen.

  Groot is Uw erbarming, o Heer;~\sep\ schenk mij het leven naar Uw besluiten.

  Velen vervolgen en kwellen mij:~\sep\ van Uw voorschriften wijk ik niet af.

  Ik zag overtreders en het walgde mij,~\sep\ want Uw uitspraak volgden zij niet.

  Zie, Heer, ik heb Uw bevelen lief,~\sep\ spaar mijn leven naar Uw barmhartigheid.

  Geheel Uw woord ligt vervat in standvastigheid;~\sep\ en ieder besluit van Uw gerechtigheid is eeuwig.
\end{halfparskip}

\markedsubsectionrubricwithhint{Laatste vrijdag.}{(Suraya of eigen psalm.)~\Ps{31,19--21}.}

\begin{halfparskip}
  Hoe groot, Heer, is Uw goedheid,~\sep\ die Gij hebt weggelegd voor hen, die U vrezen,

  \liturgicalhint{3x alleluia. --- Eerste vers.}

  Die Gij bewijst aan hen, die vluchten tot U,~\sep\ ten aanschouwen der mensen.

  Gij beschermt hen onder de schutse van Uw aanschijn,~\sep\ tegen het samenzweren der mannen,

  Gij verbergt hen in Uw tent,~\sep\ tegen het schelden der tongen.

  Gezegend de Heer,~\sep\ want Hij bewees mij Zijn wondere barmhartigheid in de versterkte stad.
\end{halfparskip}

\markedsubsectionrubricwithhint{Zaterdagen ``na''.}{\Ps{118,161--176}.}

\begin{halfparskip}
  Vorsten vervolgen mij zonder reden,~\sep\ maar mijn hart eerbiedigt Uw woorden.

  \liturgicalhint{3x alleluia. --- Eerste vers.}

  Ik verheug mij over Uw uitspraken,~\sep\ als iemand, die rijke buit heeft gemaakt.

  Ongerechtigheid haat en verfoei ik,~\sep\ Uw Wet heb ik lief.

  Zevenmaal daags breng ik U lof,~\sep\ om Uw rechtvaardige oordelen.

  Veel vrede is weggelegd voor die Uw Wet beminnen:~\sep\ geen struikelblok ligt ooit op hun weg.

  Van U, o Heer, verwacht ik hulp,~\sep\ en ik onderhoud Uw geboden.

  Ik leef Uw voorschriften na,~\sep\ en heb ze van harte lief.

  Ik onderhoud Uw bevelen en geboden,~\sep\ want heel mijn weg ligt open voor U.

  \acrosticletter{Tau} Mijn geroep kome tot U, o Heer,~\sep\ geef mij inzicht naar Uw woord.

  Mijn bede dringe door tot U;~\sep\ red mij naar Uw uitspraak.

  Van mijn lippen moge een lofzang vloeien,~\sep\ als Gij mij Uw verordeningen zult hebben geleerd.

  Mijn tong bezinge Uw uitspraak,~\sep\ want rechtvaardig zijn al Uw geboden.

  Uw hand zij gereed mij te helpen,~\sep\ want Uw bevelen heb ik verkoren.

  Van U verwacht ik redding, o Heer,~\sep\ en Uw Wet is mijn geneugte.

  Leve mijn ziel om U te prijzen,~\sep\ en dat Uw besluiten mij helpen!

  Als een verloren schaap dool ik rond; zoek toch Uw dienaar op,~\sep\ want Uw geboden heb ik niet vergeten.
\end{halfparskip}

\begin{halfparskip}
  \markedday{Alle dagen.} \liturgicalhint{Eer. --- 3x alleluia.}
\end{halfparskip}

% % % % % % % % % % % % % % % % % % % % % % % % % % % % % % % % % % % % % % % %

\markedsection{Onze Vader \markedsectionhint{met antifoon. (Niet in de Vasten en Rogatie der Ninevieten, maar begin dan de Suba'a.)}}

\begin{halfparskip}
  \cc~Mogen Uw Naam verheerlijkt worden, onze heilige Heer en God, Uw Godheid aanbeden, Uw Majesteit geëerd, Uw Grootheid gevierd en Uw Wezen verheven; en moge de eeuwige genade van Uw glorieuze Drie-eenheid over Uw volk en de schapen van Uw weide ten allen tijde uitgegoten worden, Heer van alles, Vader...

  \cc~In de hemel en op aarde, mijn Heer, is Uw Godheid gezegend en Uw Majesteit aanbeden. Heilig, glorieus, verheerlijkt, hoog en verheven is ten allen tijde de aanbiddelijke en glorieuze Naam van Uw roemrijke Drie-eenheid, Heer van alles, Vader...
\end{halfparskip}

% % % % % % % % % % % % % % % % % % % % % % % % % % % % % % % % % % % % % % % %

\markedsection{Onita D-Sahde (Hymnen der Martelaren)}

\markedsubsectionrubric{Maandagen}

\begin{halfparskip}
  Jubelt, rechtvaardigen, in de Heer.~--- Heilige martelaren, bidt voor vrede, dat we uw feesten met vreugde mogen vieren.

  De rechtschapenen past een lofzang.~--- De martelaren die verlangden Christus te zien verkregen vleugels door het zwaard en vlogen naar de hemel.

  Zingt voor Hem in welluidende klanken.~--- De martelaren zeggen in hun liefde tot Christus: Voor U sterven we dagelijks.

  Zoekt de Heer en weest sterk.~--- Martelaren, vraagt genade voor de wereld die het ware toevluchtsoord zoekt in de kracht van uw beenderen.

  Zij riepen tot de Heer, en Hij schonk hun verhoring.~--- Laat ons de martelaren aanroepen en bij hen toevlucht zoeken, dat ze voor ons mogen bidden.

  Als een stad gebouwd, geheel aaneengesloten.~--- Jullie zijn degenen die de slachtingen stoppen (\translationoptionNl{de bressen afsluiten}) voor de vervolgers in tijden van verdrukking.

  Nu en tot in eeuwigheid.~--- Mogen de gebeden der martelaren een wal voor ons zijn en de aanvallen van de sluwe van ons wegjagen.

  Dat zij Hem dankoffers brengen.~--- Martelaren, jullie waren offers voor de hogepriesters. Moge uw gebed een bolwerk voor onze zielen zijn.

  Prijzen wil ik de Heer te allen tijde.~--- Gezegend is uw strijd, heilige martelaren, want door het bloed van jullie nekken heb jullie het koninkrijk gewonnen.

  Steeds zal mijn mond Hem loven.~--- Gezegend is Christus, die Zijn heiligen loofde, hier op aarde en boven in de hemel.

  Zij die de Heer liefhebben, haten het kwade.~--- Martelaren van de Zoon en vrienden van de Eniggeborene, bidt dat er vrede mag heersen in de schepping.

  Ziet naar Hem op en vertrouwt op Hem.~--- De martelaren zagen de Zoon gekruisigd aan de boom, en bogen hun nek voor het zwaard, en werden gekroond.

  Neig Uw oor, O Heer; verhoor mij.~--- Het Kruis van onze Heer was met bloed besmeurd. De martelaren zagen Het en bogen hun nekken.

  Hij scheidde de zee, voerde hen er doorheen.~--- Het Kruis van Christus was een brug naar de martelaren; de rechtvaardigen zijn er overgegaan naar het land waar geen angst heerst.

  Te verkiezen boven goud en kostbare stenen (\translationoptionNl{schatten van het edelst metaal}).~--- De martelaren zijn als parels, want hun afbeeldingen zijn gemonteerd in de kroon van de Koning.

  De koningsdochter stond in glorie.~--- De trouwe Kerk is een parel; de martelaren in haar zijn zoenoffers.

  Bij de poorten der dochter van Sion.~--- De martelaren zagen een parel in Sion, en renden en kochten die met het bloed van hun nekken.

  Gij zijt de schoonste onder de kinderen der mensen.~--- De roos in de tuinen is prachtig om te zien, maar nog mooier waren de martelaren in hun dood.

  Hoe goed en heerlijk.~--- Edelstenen en berylen bent u, martelaren, in de kroon van de Zoon van de Koning.

  Zingt voor de Heer bij citerspel, en de klank van de harp.~--- Ik hoorde de stem der martelaren lof zingen met de harpen van David rond het paradijs.

  Zijn majesteit gaat aarde en hemel te boven.~--- Lof aan die Stem die tot de martelaren zei: Vermeng uw bloed met Mijn bloed en Mijn leven met uw leven.

  Aanhoort het, alle volkeren.~--- De martelaren werden korenaren, de koningen oogstten ze; en de Heer plaatste ze in de schuur van Zijn koninkrijk.

  Juichend keren zij weer.~--- De heilige martelaren en priesters vertrekken om onze Heer te ontmoeten op de dag van Zijn komst.

  Uw gebeden zijn op ons allen.~--- Heilige martelaren, vraagt genade voor ons, dat wij door uw gebeden vergeving mogen ontvangen.

  Ik sprak over vrede.~--- Vrede zij u, Mar Pithiun, uitverkoren door Christus, die al het lijden droeg voor de waarheid van uw Heer.

  Hij tuchtigde koningen omwille van hen.~--- Hoe passend was het voor de jonge Cyriac toen hij de onrechtvaardige koning terechtwees.

  Als blijde moeder van kinderen.~--- De trouwe Shmuni moedigt haar zoons aan: ``Mijn geliefde zonen, vertrekt in vrede''.

  Verlaat u op de Heer, en stel uw hoop op Hem.~--- Vraag voor ons aan uw Heer, martelaar Joris, mededogen, genade en vergeving der zonden.

  Koningen der aarde en alle volkeren.~--- Alle rassen noemen gezegend de maagd Maria, de moeder van Christus.

  Eer...~--- Vrede zij u, heilige martelaren, zaaiers van vrede in de vier uiteinden [der aarde].

  Vanaf het begin en in alle eeuwigheid.~--- Uw gedachtenis, onze vader, is op het heilig altaar, met de rechtvaardigen die overwonnen hebben en de martelaren die gekroond werden.

  Dat al het volk zegge: amen en amen.~--- O God, wiens barmhartigheid groot is, laat ons gecorrigeerd worden door U, en niet door mensen.

  Kom, Heer, help ons en versterk onze zwakheid, want in U is onze hoop 's~nachts en overdag.

  Christus, die niemand negeert die U aanroept, in Uw genade verwerp niet het verzoek van hen die U aanbidden.

  Want naar U heffen zij en keren hun ogen, dat U hun overtredingen zou vergeven en hun zonden uitwissen.

  Heer, werp door Uw rechterhand Satan omver die dronken maakt zonder wijn en doet uitglijden zonder modder.

  Met de tollenaar vragen wij genade. Heb medelijden met ons, o God, en wees ons genadig. Uw Kruis heeft ons gered, Uw Kruis redt ons. Moge Uw Kruis een wal zijn voor onze zielen.

  Mogen oorlogen eindigen en geschillen tot rust worden gebracht; en moge vrede heersen in de vier uiteinden.

  Heer, bewaar ons, want wij zijn als schapen onder slangen die erger zijn dan wolven.

  Heer, zaai Uw vrede en rust in de wereld; en verwijder de roede van correctie van ons.

  Heer, zegen en bewaar onze gemeente; en laat Uw vrede en rust in haar wonen.

  Heer, schenk vrede aan de vier windstreken; en wend de vervolgers af die ons tegenwerken.

  Heer, sluit de mond der onrechtvaardigen, zodat zij geen kwaad spreken tegen de kinderen van de Kerk.
\end{halfparskip}

\markedsubsectionrubric{Dinsdagen}

\begin{halfparskip}
  In eeuwig aandenken blijft de rechtvaardige.~--- Stefanus betrad de weg; de martelaren traden in zijn voetsporen. Zij verheugen zich met de Bruidegom in de bruidskamer van onvergankelijk licht.

  Voor kwade tijding zal hij niet vrezen.~--- Toen Stefanus werd gestenigd, zag hij de glorie van zijn Heer in de hemel, en de Heilige Geest die een kroon weefde voor het Hoofd van de gelovigen.

  Hij vroeg U om leven, en U gaf het hem.~--- Stefanus vroeg om genade voor de mensen die naderbij kwamen voor zijn steniging. ``Vergeef, Heer, en scheld het hun kwijt, want ze weten niet wat ze doen''.

  Tegen een onbarmhartig volk.~--- Stefanus, toen hij werd gedood, vroeg genade voor zij die hem vermoordden, zoals zijn Heer toen Hij werd gekruisigd door de wrede Joden.

  Zingt voor Hem in welluidende klanken.~--- Met luide stem riepen de martelaren tot de rechters en zeiden: ``Wij zullen Christus, die voor ons de dood heeft geproefd, niet verloochenen''.

  Help ons, God, onze Redder.~--- Christus, die met de martelaren mee afdaalde en hen hielp in hun strijd, wees voor ons een onoverwinnelijk pantser en een onneembare muur.

  Heer, er is geen als U, en geen werk is gelijk aan het Uwe.~--- Geen enkele beproeving of verdriet zal ons dwingen Uw aanbidding te ontkennen of op te geven. Want de goden der heidenen lopen op niets uit; U alleen kent geen einde.

  Die staan geplant in het huis van de Heer.~--- Zoals bomen in een tuin worden de martelaren in de tempel geplaatst. Boven hen werd het altaar geplaatst en de Heilige Geest bedient hen.

  Gij, Zijn dienaren, die Zijn wil volbrengt.~--- De engelen dalen af van hun plaats en loven in hun gezangen de beenderen der gelovigen die de wil van hun Schepper hebben volbracht.

  Dat zij Hem dankoffers brengen.~--- De martelaren die verbrand werden waren wierook en vielen als kruiden (\translationoptionNl{wortels}) in het vuur. En de geur van hun verbranding was zoet als aromaten in de tempel.

  De ceders van de Libanon, die door Hem zijn geplant.~--- De martelaren zijn ceders die hun lichaam niet voor de rechters hebben gebogen. Op hun graven baden koningen, dat ze uit hun beenderen hulp konden krijgen.

  Als een stad gebouwd, geheel aaneengesloten.~--- De martelaren zijn muren die niet instorten, gezegende fonteinen die niet te kort schieten. Smeek door uw gebeden om genade van de Medelevende voor de wereld.

  Neig Uw oor, O Heer; verhoor mij.~--- Mozes bad en verdeelde de zee. Simon bad en overwon de boze. Onze Heer bad, proefde de dood en gaf leven aan Adam die was omgekomen.

  Wie in de wolken zal de Heer evenaren?~--- Voor de gelovigen die geleid worden door Uw Naam\footnote{Letterlijk ``Uw Naam nagewandeld hebben''.}, kan het lijden van deze tijd niet vergeleken worden met het koninkrijk dat vanaf het begin voor hen bereid was.

  Ziet, hoe goed en heerlijk het is.~--- Hoe zoet is de geur van die tuin, geplaatst in de tempel van Jeruzalem. De martelaren komen binnen en hebben vreugde in de schaduw van hun takken.

  Gelukkig zij, die smetteloos zijn op hun levensweg.~--- Gezegend zijn de martelaren als zij de stem van de Vader tot hen horen zeggen: ``Komt, gaat binnen en beërft het koninkrijk dat u vanaf het begin verwacht''.

  Die wandelen naar de Wet van de Heer.~--- Onze Heer Jezus heeft een grote zegen beloofd die niet voorbijgaat aan degenen die Hem hebben liefgehad, in Hem hebben geloofd en al Zijn geboden hebben onderhouden.

  Komt, laten wij aanbidden en ons neerwerpen.~--- Komt, laten we gezegend worden door de martelaren; komt, laten we gezegend worden door de priesters. Moge het gebed der martelaren en priesters een wal voor onze zielen zijn.

  O God, wie is als Gij?~--- De dood der atleten is als de dood die onze Heer stierf. Zijn (dood) was aan het Kruis hangen, die van hen het zwaard en de steniging.

  Aan de oprechten van hart.~--- De heilige martelaren zijn uitgenodigd voor gezegende gunsten waaraan geen einde komt. In plaats van het lijden dat ze hier hebben doorstaan, erven ze het koninkrijk in de hemel.

  Op de wieken van de wind werd Hij gedragen.~--- Dezelfde wagen die Elia droeg, steeg op (\translationoptionNl{bewoog}) en daalde af. En daarin zullen de heiligen opstijgen om onze Heer te ontmoeten wanneer Hij komt.

  Ziet, hoe goed en heerlijk het is.~--- Hoe mooi zijn de rijen martelaren terwijl ze bidden, en zoet is de klank van hun liederen, wanneer zij om genade vragen voor de wereld.

  Hij is rechtvaardig en oprecht.~--- Glorieuze en heilige martelaar, beroemde mar Cyriacus, smeek uw Heer om genade voor ons, dat we vergeving van zonden waardig mogen zijn.

  Mogen Uw gebeden met ons allen zijn.~--- Vrede zij u, mar Pethion de martelaar, geestelijke penningmeester, verschaf welvaart aan de behoeftigen, die hun toevlucht zoeken in uw gebeden.

  Zowel kleinen als groten.~--- Laten we onze toevlucht zoeken bij mar Joris, dat door de kracht van zijn gebeden onze Heer onze paden recht maakt en de zwaarte van onze ledematen verlicht."

  Ik zal spreken: Vrede zij u.~--- De engel zei tot Maria: ``Vrede aan u, vol genade''; de Heer zij met u, o dochter der stervelingen, van wie de Verlosser is voortgekomen.

  Eer...~--- Vrede zij u, architecten (\translationoptionNl{metselaars}), die een citadel bouwden die niet instort, en in de Naam van Jezus een onverwoestbare bruidskamer in de hemel versierden en bouwden.

  Vanaf het begin en in alle eeuwigheid.~--- O onze vader, de kroon van de overwinning van uw strijd, die u op de dag van uw dood hebt ontvangen van de rechterhand van uw Heer die u heeft verheerlijkt, is geweven in de hemel.

  Dat al het volk zegge: amen en amen.~--- U die het verloop van de dag voltooit en de nacht voor rust geeft, vervul in ons Uw liefderijke goedheid, Heer, bij nacht en bij dag.

  Christus, die ons overdag heeft behoed en in Uw genade ons (veilig) naar de avond heeft geleid, geef ons een rustgevende nacht, zodat we U kunnen belijden voor Uw goedheid.

  Onze Heer, geef vrede aan ons land; zegen al ons werk, zodat we genade, mededogen en vertrouwen (\translationliteralNl{open gezicht}) in het tribunaal waardig mogen zijn.

  Geef in Uw overvloedige genade een goede nachtrust en slaap aan alle zieken en lijdenden, die U aanroepen in hun ellende.
\end{halfparskip}

\markedsubsectionrubric{Woensdagen}

\begin{halfparskip}
  Ja, om Uwentwil blijft men ons dagelijks doden.~--- Heilige martelaren, die werden gedood uit liefde voor Christus, wij smeken jullie: smeek God om genade voor ons allen.

  In hemel en op aarde.~--- Jullie daden zijn briljant, en de gedachtenis aan jullie strijd verheugt ons. Jullie werden vermoord omwille van Christus; en met Hem zullen jullie in de hemel regeren.

  Zoekt de Heer en weest sterk.~--- Martelaren, smeekt om genade voor de wereld, die haar toevlucht zoekt in uw beenderen, opdat zij door uw verzoeken en gebeden genade mag vinden op de dag des oordeels.

  Ziet, hoe goed en heerlijk het is.~--- Het land waar u woont is gevuld met vrede, rust en eendracht. Moge het door uw gebeden van alle kwaad worden behoed, o gezegenden.

  Prijzen wil ik de Heer te allen tijde.~--- Gezegend is Hij die van u, gezegenden, echte artsen heeft gemaakt. Uit uw beenderen stroomt hulp uit voor hem die in hen zijn toevlucht zoekt.

  De engel van de Heer slaat een legerplaats op.~--- In opeenvolgende groepen komen de martelaren binnen voor hun vervolgers, roepend en zeggend: ``Heer, help Uw dienaren in tijden van verdrukking''.

  Ziet naar Hem op, en hoopt in Hem.~--- Zij zagen Christus en verlangden naar Zijn liefde, Hij die de dood aanvaardde (\translationliteralNl{proefde}) ter wille van Zijn Kerk. Zij renden naar Hem en offerden hun ziel om bij Hem te zijn wanneer Hij Zich openbaart.

  Dat zij Hem dankoffers brengen.~--- Verzoenende wierookvaten en zuivere offers waren jullie, gezegenden, in wie de Vader zich verheugde, aanvaard door de Zoon en gekroond door de Heilige Geest.

  Machtig is de Heer in de hoge.~--- Boven in de hemel wordt de glorie bewaard die de martelaren aantrekken bij de verrijzenis in het glorieuze land vol zegeningen, waar het leven ver verheven is boven alle gevaar.

  Te verkiezen boven goud en schatten van het edelst metaal.~--- Christus zal kronen op uw hoofd plaatsen die beter zijn dan goud, edelstenen en parels, en u vreugde schenken in Zijn bruidskamer.

  Houdt moed en weest sterk van hart.~--- Martelaren, jullie waren niet bang voor het vuur, noch voor de vreselijke zwaarden der rechters, want jullie waren bekleed met de liefde van Christus en verachtten alle kwellingen.

  Zingt voor Hem een nieuw lied.~--- Christus bekleedt met nieuwe glorie de vrienden die Hij voor Zichzelf heeft uitgekozen. En in plaats van de verdrukkingen die zij moesten doorstaan, laat Hij hen de bruidskamer in de hemel beërven.

  Ontsluit mij de poorten der gerechtigheid.~--- Open uw schatten, gezegenden, en schenk hulp aan de behoeftigen, die uitkijken naar en wachten op uw gebed, zodat zij daardoor beschermd mogen worden tegen de boze.

  Gelukkig zij, die smetteloos zijn op hun levensweg.~--- Gezegend bent u, heilige martelaren, vrienden van de hemelse Bruidegom, want zie, u bent uitgenodigd in het koninkrijk voor het eindeloos leven en eeuwige zegen.

  Die wandelen naar de Wet van de Heer.~--- Onze Heiland heeft een oneindige zegen en een oneindig en onvergankelijk koninkrijk beloofd aan Zijn heiligen die Hem hebben liefgehad en zijn geboden hebben onderhouden.

  Jozef, verkocht in slavernij.~--- De beenderen van de illustere Jozef stopten (\translationoptionNl{vormden een heg voor}) de slachtingen der Egyptenaren. Jullie beenderen, heilige martelaren, brachten mededogen voor de hele schepping.

  U hebt een glorierijke kroon (\translationoptionNl{van zuiver goud}) hem op het hoofd gedrukt.~--- De Heer der glorie plaatste een prachtige en eervolle kroon op de illustere martelaar Hormizd, die lijden verdroeg voor de waarheid.

  Rechtvaardig en oprecht.~--- Moge het gebed van de martelaar Pethion, die door zijn kracht de boze en zijn dienaren in verwarring bracht, onze gemeente beschermen tegen de boze en zijn trawanten.

  De Heer zal sterkte schenken aan Zijn volk.~--- Moge de macht die neerdaalde in de strijd en de martelaar Mar Joris beroemd (\translationoptionNl{triomfantelijk}) maakte, ons bewaren van de boze en zijn legers.

  Het is zuiver zilver, van stof ontdaan.~--- Moge uw zuivere lichaam, O Maria, voor ons een schat aan hulp zijn, zodat het door zijn overvloedige hulp onze behoeftigheid kan verrijken.

  Eer.~--- Eer aan de Heer die u heeft verheerlijkt en u tot een schatkamer van leven heeft gemaakt voor de ellendigen en noodlijdenden, die bij U hun toevlucht zoeken en gered worden.

  Vanaf het begin en in alle eeuwigheid.~--- Zegevierende en heldhaftige atleet, O onze vader, erfgenaam van het koninkrijk, smeek Christus dat Hij in Zijn barmhartigheid Zijn vrede mag doen wonen in de schepping.

  Dat al het volk zegge: amen en amen.~--- Aan U, Heer God, brengen wij de avond lofprijzing, en wij vragen dat wij van U vergeving van onze overtredingen mogen ontvangen.

  Heer, geef ons een zondeloos leven, liefde, vrede en eendracht; mogen al onze smeekbeden beantwoord worden, zoals Uw Godheid het goed acht.

  Zaai in ons, Heer, een verlangen naar gebed; moge ons verzoek naar U opstijgen; en geef ons een tweevoudige gezondheid, het onderhoud van lichaam en geest.
\end{halfparskip}

\markedsubsectionrubric{Donderdagen}

\begin{halfparskip}
  Gij zijt de gezegenden van de Heer.~--- De gezegende martelaren beschouwden de dood als een grote winst en aanvaardden geselingen en martelingen als eerbetoon en geschenken. Nu, na hun dood, schenken ze de wereld weldaden en schatkamers vol hulp.

  Zij versmaadden het begeerlijke land.~--- De martelaren, die zagen hoe de wereld voorbijging en de waarheid bleef bestaan, verlieten gebouwen, rijkdommen en bezittingen, die ijdelheid zijn. Zij verlangden de vreze Gods en gaven hun nek over aan het zwaard. En zie, zij werden uitgenodigd om het koninkrijk te beërven.

  Zingt voor Hem in welluidende klanken.~--- De martelaren zeggen: ``Onze kroon is voorbereid (\translationoptionNl{gefixeerd}) en onze beloning bewaard''. Christus, de Koning, die wij hebben bemind, zal ons Zijn koninkrijk laten betreden.

  Omdat we vuur en het zwaard hebben geleden, zal Christus onze ellende troosten in het vreugdevol paradijs.

  Uw rijk is een rijk van alle eeuwen.~--- De martelaren zijn uitgenodigd in het koninkrijk in de hoge en in het eindeloze leven, zoals geschreven staat: ``Dat wat het oor niet heeft gehoord en het oog niet heeft gezien, en wat het hart van de mens te boven gaat''. Er bestaat geen zegen hoger dan die waarin de edele zielen, die Christus liefhadden, verblijven.

  Grijp schild en beukelaar, en rijs op als mijn helper.~--- De martelaren van de waarheid werden bekleed met een geestelijke wapenrusting toen ze naar de strijd gingen. Jonge mannen en knapen waren bijeengekomen om een ongewoon schouwspel te aanschouwen: mannen die de dood verachtten en overwonnen, hoewel ze gedood werden. Een wonder, groter dan woorden kunnen uitdrukken.

  Ziet naar Hem op en vertrouwt op Hem.~--- Heilige martelaren, jullie hebben Christus gezien in het hemelse koninkrijk. Vanuit het tijdelijke leven hebben jullie de volmaaktheid bereikt en zijn jullie waardig geacht om samen met de geestelijke gezelschappen God, de Heer der schepselen, te verheerlijken.

  De ogen van de Heer zien op de rechtvaardigen neer.~--- Met het oog van de ziel werd Christus gezien door de heilige martelaren die Hem navolgden door hun lijden. Want zij hoorden die stem: ``Iedereen, die hier op aarde lijden door mensen verdraagt ter wille van Mij, zal vreugde hebben in de bruidskamer van het koninkrijk.''

  Hij spreidde een wolk tot dekking uit.~--- Wolken van licht dragen de heiligen op de dag van de verrijzenis, wanneer de grootheid van de hemelse Koning der koningen wordt geopenbaard. Boven, in de hoge, aan de rechterhand van Christus, zijn allen die Hem hebben bemind en Zijn geboden hebben onderhouden in vreugde.

  Zij hebben inzicht noch verstand.~--- De koning beval drie kinderen in de oven te gooien. De vierde onder hen besprenkelde hun gezichten met dauw. Zolang het vuur brandde, straalden hun gezichten. Gezegend zij de Heer die Zijn heiligen heeft verheerlijkt.

  In het bijzijn van geheel Zijn volk.~--- De martelaren zeggen: ``Wij zullen de Zoon van God niet verloochenen, want wij zijn het zaad van Abraham en zonen van Isaak's erfenis. Voor de God van onze vaderen aanvaarden wij een tijdelijke dood en beërven we een leven zonder einde''.

  Uit de vuist van de goddeloze en van de verdrukker.~--- De slechte koning, woedend van woede en afgunst, liet scherpe spijkers brengen en dreef ze in het lichaam van de jongen, maar Christus hielp hem en sterkte Mar Cyriacus, de overwinnaar. Mogen zijn gebeden een wal voor ons zijn.

  Zijn hart is onwrikbaar, vertrouwend op God.~--- De martelaar Mar Pethion stond moedig voor de heidense magiërs en berispte hun arrogantie: ``Aanbidt geen afgoden, maar aanbidt de ene God, want hemel en aarde behoren Hem. Zijn heerschappij duurt eeuwig.

  Tegen een onbarmhartig volk.~--- De heilige martelaar, de zegevierende Mar Joris, heeft verschrikkelijk lijden, bittere kwellingen, en vele soorten dood doorstaan uit liefde voor Christus, omdat hij Hem meer liefhad dan zijn eigen leven. Gezegend is Hij die de martelaar met de overwinning kroonde.

  De getrouwen behoedt de Heer.~--- Mogen door de gebeden van de maagd Maria, de gezegende moeder, Uw aanbidders, Heer, worden bewaard voor de listen van de sluwe. Verleen ons Uw wil te vervullen, zowel in woorden als daden, en dat wij te allen tijde voor U glorie mogen zingen.

  Eer.~--- Vrede zij u, heilige martelaren, vrienden van Christus, die overwonnen, zegevierden en gekroond werden, en de boze in de strijd in verwarring brachten. Gezegend bent u op de dag van de Zoon, wanneer de grootheid van Zijn heerlijkheid zal schijnen, en u met Hem de bruidskamer zult binnengaan.

  Vanaf het begin en in alle eeuwigheid.~--- O onze vader, u bent een kleed van genade geweven door de Heilige Geest. Door uw standvastige daden hebt u ervoor gezorgd dat er een fontein van hemelse zegeningen is ontstaan, en u hebt de kudde op uw weide te drinken gegeven van het woord van geestelijk leven. Zie, een kroon van overwinning is voor u geweven.

  Dat al het volk zegge: amen en amen.~--- Over zee en land roepen zij U aan, Heer, dat U hen komt helpen. U antwoordt hem die over zee roept, en U negeert niet hem die op aarde is. Wij roepen tot U, Heer, kom, help ons, red ons en verlos ons van de boze en zijn leger.

  Christus, die hemel en aarde (\translationliteralNl{de hoogte en de diepte}) met Uw eigen bloed heeft verzoend; zaai vrede, Heer, tussen priesters en koningen; in Uw mededogen bevestig Uw Kerk, en in Uw genade laat het enige teken van geloof heersen in de hele wereld.

  In Uw genade, barmhartige Heer, zegen en behoed dit land en zijn inwoners voor duivels en slechte mensen. Vermeerder in ons land de tijdelijke steun, de liefde, vrede en eendracht, en de gezondheid van lichaam en ziel.

  Verjaag Satan, die de vijand van alle gerechtigheid is, uit het huis waarin wij verblijven. Laat hem het niet betreden en er heersen. Vestig, onze Heer, zijn fundamenten op de rots van het geloof en doe het eeuwig leven toenemen in ons verblijf.
\end{halfparskip}

\markedsubsectionrubric{Vrijdagen}

\begin{halfparskip}
  Dit alles kwam over ons; en toch wij zijn U niet vergeten.~--- De martelaren beminden en vertrouwden Christus, de Verlosser van de wereld, de grote Koning der heerlijkheid, dus beschaamden ze de duivel. Met de engelen verheugen zij zich in de hemel en staan voor God, en hebben de vijand en zijn leger onder hun voeten vertrapt.

  En hebben Uw verbond niet geschonden.~--- Christus is de Verlosser van de wereld die in Zijn goedheid voor ons heeft geleden en de weg naar het koninkrijk heeft betreden. De martelaren volgden in Zijn voetsporen en leverden hun lichamen over aan kwellingen, vlammen en martelingen, en verkregen door het bloed van hun nekken het beloofde eeuwige leven.

  Er klinkt een juich-- en zegekreet in de tenten der rechtvaardigen.~--- Bij het horen van de heilige martelaren, zaaiers van vrede in de schepping, vliegen de geestelijke wezens en dalen eervol neer in het heiligdom. Koren van geestelijke wezens roepen, samen met de martelaren de lofzang, ``Heilig bent U, Heer van alles, die het menselijk ras hebt verheven''.

  Want wonderen heeft Hij gewrocht.~--- De zegevierende atleten en martelaren boden ons een groot wonder aan, want zij zagen hoe de zwaarden flitsten en de beulen dreigden, maar hun gemoed verroerde noch week terug. Vanwege hun grote liefde voor hun Heer aanvaardden zij de dood als een geschenk en deden geen afstand van hun getuigenis van Hem.

  Hij redde hen niet van de dood.~--- Vermoorde zielen, die uw Schepper beminden en uit eigen vrije wil de dood omhelst hebben en zoenoffers zijn geworden, biedt samen met ons aan Christus, de Koning die u kroont, de bede dat we op de grote dag van beproeving mogen worden gered van martelingen en het eeuwige leven mogen beërven.

  Dat zij Hem dankoffers brengen.~--- Martelaren, ware zoenoffers, die zich hebben verzoend met uw Heer, met het bloed dat uw nekken hebben uitgegoten, hebben jullie het onvergankelijk koninkrijk verworven, smeekt en bidt de Heer voor de zondaars, die in jullie gebeden hun toevlucht zoeken, dat zij niet vergaan. Moge vrede en rust voor hen vermenigvuldigd worden.

  Zijn woord zond Hij uit om hen te genezen.~--- De martelaren waren ware artsen in de wereld, en genazen en heelden de zielen, die verontreinigd waren met zonde. Lof aan de Heer die u heeft uitverkoren en Zijn macht in uw beenderen liet wonen, zodat u voor het ras der stervelingen een toevluchtsoord van vrede moogt zijn midden in de wereld.

  Hij voerde hen naar de verlangde haven.~--- Een aangename en uitstekende haven is de schat van de beenderen der heiligen. Uw teken (\translationoptionNl{wil}), o onze Verlosser, heeft (daar) een bron van genade en genezing gemaakt. Moge de kracht die uit de hemel neerdaalt en altijd hun beenderen bezoekt, de samenkomst van Uw aanbidders beschermen tegen de listen van de sluwe.

  Ik wil U roemen, mijn God, de Koning.~--- De Koning van de hemel hielp met Zijn soldaten (\translationoptionNl{dienaren}) het gezelschap der gelovigen. Het bevel ging uit dat de rechtvaardige martelaren met het zwaard moesten worden gedood. De Chaldeeën [= Magiërs] waren verbaasd, stonden op, hieven hun handen op en zeiden: ``Groot is de God der gelovigen. Hoewel Hij niet wordt gezien, redt Hij hen''.

  Een grote Koning boven alle goden.~--- De Koning van de hoogste hoogten bouwde voor Zichzelf een citadel in de hemel en gaf die de naam Jeruzalem van de Eerstgeborene, geschreven in de hemel. Hij heeft een ladder van leven gevestigd in Zijn Kerk en leidt de leden van Zijn huisgezin, en verheft hen naar de voortreffelijke woonplaats van de hemel, zodat zij erfgenamen van het koninkrijk kunnen zijn.

  Grijp schild en beukelaar, en rijs op als mijn helper.~--- In hun strijd werden de martelaren bekleed met de wapenrusting van de Heilige Geest, toen zij hun nekken bogen voor de zwaarden der vervolgers, en de geselingen en martelingen verachtten uit de liefde van Christus, en zij door het bloed van hun nekken het onverwoestbare koninkrijk erfden.

  De koningsdochter stond in glorie.~--- De Kerk is als de ark, en het heilig altaar als de troon, en de martelaren als het koor der Vorstendommen, die Christus dienen. Het is gebouwd met jaspisstenen, saffieren en kristal. De architecten zijn Petrus en Paulus, Johannes en Andreas.

  Ik zal spreken over U.~--- Vrede aan u, martelaar Cyriacus, atleet van Christus Koning. Vrede aan uw graf, dat gezondheid geeft aan de zieken. Vrede aan uw ledematen, waaruit hulp stroomt. Vrede aan u en vrede aan uw moeder, en lof aan Hem, die u de overwinning heeft gegeven.

  Verlaat u op de Heer, en stel uw hoop op Hem.~--- Wees gegroet, geestelijke koopman, nobele martelaar Mar Sargis! Een parel zonder gebreken, een licht scheen in uw ziel. U hebt het met uw bloed gekocht en bent daardoor rijk geworden. U hebt onvergankelijke rijkdom verworven. Vraag voor de Kerk en haar kinderen liefde, vrede en eendracht.

  Gord uw zwaard om de heup, gij, machtige held.~--- Sterke held, Mar Joris, die de dood, het zwaard, wrede sneden en elke soort kastijding verachtte, en wonderen verrichtte en alle mensen tot de waarheid bracht, gezegend is Hij die deze atleet liet zegevieren, die door Zijn kracht de dwaling zal overwinnen.

  Toen zei men onder de volken.~--- Gezegend zijt gij, heilige maagd, gezegend zijt gij, moeder van God (\emph{Hudra:} Christus). Gezegend zijt gij die alle generaties en rassen ``gezegend'' noemen. Gezegend zijt gij in wie de Vader welbehagen had. Gezegend zijt gij in wie de Eerstgeborene woonde. Gezegend zijt gij, want de Heilige Geest heeft uw naam laten zegevieren in de schepping.

  Eer.~--- Eer aan de Vader die jullie heeft uitverkoren, o zegevierende en heilige martelaren. Eer aan de Zoon door wiens kracht u ter wille van Hem in de strijd stond. Eer aan de Heilige Geest, die uw kronen versierde (en) weefde. Moge uw gebed een wal voor ons zijn tot het einde van de wereld.

  Vanaf het begin en in alle eeuwigheid.~--- Onze vader, die zegevierde in de strijd, zie, in de hemel is uw beloning. Christus, voor (\translationoptionNl{met}) wie u uzelf versierde, heeft uw gedachtenis in Zijn Kerk verheerlijkt. Uw liefde is pure wierook, en door uw daden hebt u uw Heer behaagd (\translationoptionNl{gunstig gestemd}). Smeek Hem met ons, wanneer Hij in grote heerlijkheid verschijnen zal, dat Hij medelijden met ons heeft.

  Dat al het volk zegge: amen en amen.~--- Koning der koningen, onze Helper, Christus onze Verlosser, heb medelijden met Uw dienaren die U aanroepen in deze tijden van verdrukking. Want zie, ellende en verschrikkingen omringen ons aan alle kanten. Laat Uw genade ons snel voorafgaan, en laat Uw gezicht stralen en ons redden.

  U bent medelevend van alle eeuwigheid en barmhartig voor altijd. Wat is de boosheid van de schepping, vergeleken met de overvloedige genade van Uw liefderijke goedheid? Besprinkel het gelaat van onze (menselijke) natuur met de dauw van barmhartigheid en medelijden, en red ons uit de hand van de boze, en van het onkruid, de zonen van de dwaling.

  Moge Adam en het kamp der rechtvaardigen, Mozes en de keten der profeten, Petrus en de groep apostelen, Stefanus en alle martelaren. Efrem en de vergadering der leraren, en Antonius en de kluizenaars U smeken, O onze Heer Jezus, dat U medelijden moge hebben met de wereld.

  Onze Heer, mogen de lichamen der overledenen, die zich met U in het water van het doopsel hebben bekleed, door U worden gereinigd van de verontreinigingen van de zonde. Schenk aan de overledenen die Uw lichaam hebben genuttigd en genoten hebben van Uw levend bloed, O onze Heer, een herdenking in het land waar de rechtvaardigen verblijven.
\end{halfparskip}

\markedsubsectionrubric{Zaterdagen}

\begin{halfparskip}
  Over heel de wereld golft hun sein.~--- De heilige martelaren, bekleed met licht, gingen naar de vier uiteinden van de aarde om de glorieuze Drie-eenheid te prediken: Vader, Zoon en Heilige Geest.

  En tot de grenzen der aarde hun uitspraak.~--- De martelaren imiteerden de engelen, want terwijl ze op aarde rondliepen in het uiterlijk van alle mensen, woonden ze in hun geest boven in de hemel tussen de engelen.

  Looft de Heer, gij, al Zijn engelen.~--- De martelaren, vrienden van Christus, die de Drie-eenheid verkondigden, volgden het voorbeeld der engelen: zij verachtten en verzaakten aan alle aardse dingen.

  Zingt voor de Heer bij citerspel, en de klank van de harp.~--- Met liederen en heilige stemmen zingen de martelaren alleluja voor de hemelse Bruidegom en roepen Hem toe: ``Heilig, heilig, heilig is Hij die de overwinning geeft aan Uw vrienden.

  Daar staken er op schepen in zee.~--- Heil, kooplieden, die met het bloed van hun nekken onvergankelijke rijkdom hebben verworven, smeekt en bidt Christus dat Hij Zijn vrede doet verblijven in de wereld.

  Gij, Zijn dienaren, die Zijn wil volbrengt.~--- De engelen dalen af uit het land van licht en zingen een hymne over de beenderen der heiligen tot Hem die de stervelingen waardig heeft gemaakt om vreugde te hebben met de geestelijke wezens.

  Te verkiezen boven goud en kostbare stenen.~--- Zoals goud en kostbare stenen worden de beenderen der heiligen in de kerken van Christus gelegd. Ze verspreiden en delen hulp uit aan hem die zijn toevlucht tot hen neemt.

  Als een licht in het duister rijst op voor de goeden.~--- Zoals de zon aan de hemel schijnt, schijnen de daden der heiligen in de kerken van de Koningszoon. De duisternis heeft geen vat op hen die hun toevlucht tot hen nemen.

  Majesteit en pracht gaan vóór Hem uit.~--- De gezegende martelaren zagen Uw glans, o onze Heer. Toen zij omwille van Uw Naam aan het kruis leden, riepen allen en zeiden: ``Voor U sterven wij''.

  Hij spreidde een wolk tot dekking uit.~--- In de vuuroven doofden de drie jongelingen uit het huis van Ananias de vlam. Moge hun gebed voor ons een wal zijn tegen de listen van de sluwe.

  Gelukkig zij, die smetteloos zijn op hun levensweg.~--- Gezegend zijn de martelaren die Christus liefhadden en in het vuur van Zijn liefde de zichtbare dingen haatten. Zie, Hij die hen beloont, is gekomen om de beloning (\translationoptionNl{wedde}) voor hun daden te betalen.

  Die wandelen naar de Wet van de Heer.~--- Gezegend zijn jullie, heilige martelaren, atleten, die hebben getriomfeerd en overwonnen in de geestelijke strijd. Kijkt, jullie lichamen zijn in de kerk, en jullie geesten te midden der engelen.

  De Heer zal sterkte schenken aan Zijn volk.~--- Christus, die de knaap Cyriacus zo heeft gesterkt, dat hij in de talrijke martelingen niet van de waarheid afvalt, bewaak onze gemeente door zijn gebeden altijd tegen de boze.

  De Heer is een kracht voor Zijn volk.~--- Christus, die de martelaar mar Pethion heeft versterkt om het kwaad der magiërs, de zonen van de dwaling, te weerstaan, versterk onze gemeenschap zodat zij kan zegevieren over lijden en verleidingen.

  Gelukkig zult gij zijn en het zal u welgaan.~--- Gezegend is uw geest, martelaar Joris, die uw Heer diende volgens Zijn wil in deze voorbijgaande wereld. Zie, u hebt voor uw inspanningen de beloning ontvangen met de heiligen in het koninkrijk.

  Laat ze juichen, jubelen van de toppen der bergen.~--- Gezegend zijt gij, Maria, gezegende moeder, die volgens de voorspelling der profeten en de verzegeling der apostelen, zonder echtgenoot de Verlosser baarde door de kracht van de Heilige Geest.

  Eer...~--- Vrede zij u, heilige martelaren. Vrede voor u die overwonnen hebt in de geestelijke strijd. Vrede zij u die Christus liefhad met een zuiver hart.

  Vanaf het begin en in alle eeuwigheid.~--- Onze edele vader kwam tot de vervolmaking van alle heiligen door de hulp van de genade. Christus, die Uw heilige verheerlijkt, bewaak onze bijeenkomst door zijn gebed.

  Dat al het volk zegge: amen en amen.~--- Hoor ons, Heer, hoor ons, Hoop van Uw huisgezin. Hoor ons, onze Verlosser, en luister naar de stem van ons verzoek, en beantwoord in Uw genade onze smeekbeden, o Algoede, die Uw barmhartigheid niet weigert.

  Moge Uw Kruis, onze Heer, een bewaker zijn van de geliefde gemeenten die Uzelf in Uw Katholieke Kerk hebt uitgekozen, zodat zij waardig mogen zijn om Uw wil te doen op aarde, zoals in de hemel.

  Christus, onze Verlosser, laat het gebed en verzoek tot U doordringen dat Uw aanbidders aan Uw Majesteit hebben aangeboden, en moge Uw genade ons helpen het hoofd van de rebel te vertrappen.

  De leer is een lamp en licht, getuigt David onder de profeten, en Paulus onder de apostelen. Komt, laten we vreugde hebben door ervan te drinken en de weg der geboden te bewandelen.

  Open voor ons, o onze Heer, de schatkamer van Uw genade, dat wij genade en redding mogen ontvangen voor onze verontreinigde zielen, zoals de rover die U liefhad en aan wie U het koninkrijk beloofde.
\end{halfparskip}

\markedsubsectionrubric{Alle dagen}

\begin{halfparskip}
  \cc~Wij smeken U, die Uw Kerk laat groeien, die kroont die U liefhebben, die Uw atleten laat zegevieren, en Uw heiligen helpt in hun glorieuze, heilige, levengevende en goddelijke wedstrijden; keer U naar ons, Heer, heb medelijden en wees ons genadig, zoals U dat altijd doet, Heer van alles...

  \cc~Laat U door het gebed van Uw heiligen, onze Heer en onze God, met ons verzoenen. Op verzoek van Uw gelovigen, laat onze zonden voorbijgaan, vergeef ons wat in ons ontbreekt; maak onze vijanden vredig, doe onze overledenen verrijzen; en maak ons de voortreffelijke glorie van Uw koninkrijk waardig, samen met de gerechtigen en rechtvaardigen die Uw wil vervullen in het hemelse Jeruzalem, Heer van alles...
\end{halfparskip}

% % % % % % % % % % % % % % % % % % % % % % % % % % % % % % % % % % % % % % % %

\markedsection{Suba'a \markedsectionhint{(Op gedachtenissen, in de Vasten, Rogatie der Nineviten, niet op Zondagen en feesten van de Heer.)}}

\begin{halfparskip}\begin{sfpar}
  \cc~Maak ons waardig, onze Heer en God, van een vredige avond, een rustgevende nacht, een ochtend waarin goede dingen worden verkondigd en een dag van goede daden van gerechtigheid; opdat wij daardoor Uw Godheid gunstig kunnen stemmen, alle dagen van ons leven, Heer van alles...

  \liturgicalhint{Zeg nu de Suba'a (\&\ qanona \&\ tesbohta).}

  \dd~Laat ons allen ordelijk staan met vreugde en vrolijkheid (\liturgicalhint{vasten:} met berouw en zorg); laat ons bidden en zeggen: Heer, ontferm U over ons.~--- \rr~Heer, ontferm U over ons. \liturgicalhint{(Wordt herhaald na elke aanroeping.)}

  \dd~Machtige Heer, almachtige God van onze vaderen, wij bidden U.

  \dd~Heilige en glorieuze, die onder de heiligen woont en wiens wil (door hen) gunstig wordt gestemd, wij bidden wij U,

  \dd~Koning der koningen en Heer der heren, die in het voortreffelijke licht woont, wij bidden U,

  \dd~U die niemand heeft gezien, noch kan zien, wij bidden U,

  \dd~U die wilt dat alle mensen leven en zich tot de kennis van de waarheid wenden, wij bidden U,

  \dd~Voor de gezondheid van onze heilige vaders, Paus \NN, hoofd van de hele Kerk van Christus, van Patriarch \NN, van onze Catholicos \NN, van onze Metropoliet \NN, van onze Bisschop \NN, en van al hun helpers, wij bidden U,

  \dd~Barmhartige God, die met Uw liefde alles bestuurt, wij bidden U,

  \dd~Gij die in de hemel wordt geprezen en op aarde wordt aanbeden, wij bidden U,

  \dd~Laat Uw vrede en rust wonen in de vergadering van Uw aanbidders, Christus onze Verlosser, en ontferm U over ons.

  \liturgicalOption{Trisagion} (Heilige God...) en \liturgicalOption{Onze Vader}...

  \cc~U die Uw deur opent voor hen die erop kloppen en antwoord geeft op de smeekbeden van hen die U vragen, open, onze Heer en onze God, de deur van barmhartigheid voor ons gebed; ontvang ons verzoek, en antwoord in Uw barmhartigheid op onze smeekbeden uit Uw rijke en overvloedige schatkist, U die goed bent: en laat Uw barmhartigheid en gaven aan de behoeftigen en verdrukten niet achterwege, Uw dienaren die U aanroepen en U smeken, in alle seizoenen en tijden, Heer van alles...

  \cc~U die de stem hoort der gerechtigen en rechtvaardigen die U voortdurend gunstig stemmen, en die de wens vervult van hen die U vrezen, hoor, mijn Heer, in Uw mededogen het gebed van Uw dienaren, en ontvang in Uw mededogen het verzoek van Uw aanbidders. en heb medelijden met de getroffenen en gekwelden, Uw dienaren die U aanroepen en U smeken, in alle seizoenen en tijden, Heer van alles...

  \fullline
\end{sfpar}\end{halfparskip}

% % % % % % % % % % % % % % % % % % % % % % % % % % % % % % % % % % % % % % % %

\markedsection{Gebeden voor hulp}

\liturgicalhint{In aanwezigheid van meerdere priesters in volgorde van prioriteit. Elk gebed wordt afgesloten met ``Amen, zegen, Heer''. Het eerste wordt altijd gezegd indien het officie plaats heeft in een kerk.}

\begin{halfparskip}
  [\cc~Moge, Heer, Uw barmhartige hulp, de grote bijstand van Uw goedheid, de verborgen en glorieuze kracht van Uw glorieuze Drie-eenheid, en Uw rechterhand vol erbarmen en genade, de zwakheid van Uw aanbidders beschutten en begeleid worden vanuit Uw heilig huis dat vol is van alle hulp en alle zegeningen, door het gebed van de zalige Maria en van alle heiligen die U gunstig stemmen, Heer van alles, Vader...

  \liturgicalhint{2.} Onze Heer en God, mogen Uw dienaren gezegend worden door Uw zegen, en Uw aanbidders worden beschermd door de zorg van Uw wil; Heer, moge de voortdurende vrede van Uw goddelijkheid en de blijvende vrede van Uw Goddelijkheid heersen over Uw volk en in Uw Kerk, alle dagen van de wereld, Heer van alles...

  \liturgicalhint{3.} Moge de zegen van Hem die alles zegent, de vrede van Hem die alles tot bedaren brengt, de genade van Hem die iedereen genadig is, de bescherming van onze aanbiddelijke God, met ons, onder ons en rondom ons zijn en ons beschermen van de boze en van zijn krachten altijd en in eeuwigheid, Heer van alles...

  \liturgicalhint{4.} Onze Heer en onze God, moge wij gezegend zijn door Uw zegen, mogen wij beschermd worden door Uw voorzienigheid, moge Uw kracht ons ondersteunen, moge Uw hulp ons vergezellen, moge Uw rechterhand ons overschaduwen, moge Uw vrede heersen onder ons, moge Uw Kruis een hoge vesting (\translationoptionNl{muur}) en toevlucht voor ons zijn en mogen wij onder zijn vleugels worden verdedigd tegen de boze en zijn legers, altijd en in eeuwigheid, Heer van alles...

  \liturgicalhint{5.} Heer, gezegend is de genade van Uw goedheid, aanbiddelijk zijn de beloften van Uw Heerlijkheid, die ons leren om altijd naar U te kijken en in U te roemen. Laat onze hoop niet van U gescheiden worden alle dagen van ons leven, Heer van alles...

  \liturgicalhint{6.} Onze Heer en God, moge Uw zegen rusten op Uw volk, en moge Uw genade voortdurend op ons, zwakke zondaars zijn; onze goede Hoop, onze barmhartige Toevlucht en Vergever van schulden en zonden, Heer van alles...

  \liturgicalhint{7.} Moge de vrede van de Vader met ons zijn, en de liefde van de Zoon onder ons, en moge de Heilige Geest ons leiden volgens Zijn wil, en mogen Zijn barmhartigheid en medeleven altijd en in eeuwigheid op ons zijn, Heer van alles...

  \liturgicalhint{8.} Heer, moge Uw vrede in ons wonen en Uw rust in ons heersen, en moge Uw liefde onder ons toenemen alle dagen van ons leven, Heer van alles...

  \liturgicalhint{9.} Heer, bescherm ons door Uw rechterhand, verdedig ons onder Uw vleugels en laat Uw hulp ons alle dagen van ons leven vergezellen, Heer van alles...

  \liturgicalhint{10.} Heer, geef ons Uw voortdurende vrede, liefde, liefde voor kennis, leven, geluk en vreugde, en laat Uw zorg over ons geen enkele dag van ons leven ontbreken, Heer van alles...

  \liturgicalhint{11.} Wees een slapeloze Bewaker van het bolwerk waarin Uw schapen wonen, opdat ze niet worden gekwetst door de wolven die dorsten naar het bloed van Uw kudde, want U bent de zee die niet zal afnemen, Heer van alles...

  \liturgicalhint{12.} Heer, zegen ons met Uw zegeningen, omring ons met de vesting (\translationoptionNl{muur}) van Uw zorg, beroof ons niet van het goede en laat ons aan tafel liggen in Uw stralend bruidsfeest, Heer van alles...

  \liturgicalhint{13.} Heer, kom ons te hulp in Uw barmhartigheid, openbaar onze verlossing in Uw mededogen, en leid onze stappen op de paden van gerechtigheid alle dagen van ons leven, Heer van alles...

  \liturgicalhint{14.} Heer, laat Uw goedheid tot ons doordringen wanneer Uw gerechtigheid ons oordeelt, en laat Uw genade ons te hulp komen op de dag waarop Uw Majesteit zal verschijnen, Heer van alles...

  \liturgicalhint{15.} Heer van alles, laat Uw zegen en Uw genade, Uw rechterhand vol barmhartigheid en mededogen, de gemeenschappen van Uw aanbidders die U altijd aanroepen en smeken, overschaduwen en begeleiden, Heer van alles...]
\end{halfparskip}

% % % % % % % % % % % % % % % % % % % % % % % % % % % % % % % % % % % % % % % %

\markedsection{Finale gebeden}

\begin{halfparskip}
  \liturgicaloption{(1)} \liturgicaloption{Maria.} Moge het gebed, Heer, van de heilige maagd, het verzoek van de gezegende moeder, het verzoeken en smeken van haar die vol genade is, de gezegende mart Maria, de grote kracht van het zegevierende Kruis, de goddelijke hulp, en de bede van mar Johannes de Doper altijd met ons zijn in alle eeuwigheid, Heer van alles...

  \liturgicaloption{Apostelen.} Moge het gebed, Heer, van de heilige apostelen, de bede der ware predikers, het verzoeken en smeken der beroemde atleten, de verkondigers van gerechtigheid, de predikers van heiligheid, de zaaiers van vrede in de schepping, altijd met ons zijn in alle eeuwigheid, Heer van alles, Vader...

  \liturgicaloption{Heiligen.} Moge het gebed, verzoek, smeken en vragen van onze beroemde en heilige vader, mar Thomas apostel, van mar Adai en mar Mari, leraren van het Oosten, mar Stefanus, de eerstgeborene der martelaren, mar Simon bar Sabbae, mar Jacob, mar Efrem, van de krachtige reus, mar Joris, de beroemde martelaar, van mar Cyriacus, mar Pethion, mar Hormizd, van de gezegende mar Eugenius en geheel zijn geestelijk gezelschap, van sint Barbara en van Shmuni en haar zonen, van Meskenta en haar twee zonen, van alle martelaren en heiligen van onze Heer, altijd voor ons een hoge muur en een stevig huis van toevlucht zijn, om onze lichamen en zielen te verlossen, te bevrijden, te redden en te bewaren van de Boze en zijn legers, in alle eeuwigheid, Heer van alles.
\end{halfparskip}

\begin{halfparskip}
  \fullline
  \liturgicaloption{(2) Of korte versie:} Moge het gebed, Heer, van de heilige maagd, het verzoek van de gezegende moeder, het smeken en bidden van haar die vol genade is, de gezegende mart Maria, de grote kracht van het zegevierende Kruis, de goddelijke hulp, de bede van mar Johannes de Doper, het gebed der heilige apostelen, de bede der ware predikers, het verzoeken en smeken der beroemde atleten, de verkondigers van gerechtigheid, de predikers van heiligheid, de zaaiers van vrede in de schepping; het gebed van onze beroemde en heilige vader mar Thomas apostel, van mar Adai en mar Mari, leraren van het Oosten, mar Stefanus, de eerstgeborene der martelaren, mar Simon bar Sabbae, mar Jacob, mar Efrem, de krachtige reus, mar Joris, de beroemde martelaar, van mar Cyriacus, mar Pethion, mar Hormizd, de gezegende mar Eugenius en geheel zijn geestelijk gezelschap, sint Barbara en Shmuni en haar zonen, Meskenta en haar twee zonen, van alle martelaren en heiligen van onze Heer, altijd voor ons een hoge muur en een stevig huis van toevlucht zijn, om onze lichamen en zielen te verlossen, te bevrijden, te redden en te bewaren van de Boze en zijn legers, in alle eeuwigheid, Heer van alles...

  \liturgicalhint{Hudra: De priester neemt het Kruis in de hand; wendt zich tot de mensen en zegt: ``Zegen, mijn Heer; op Uw bevel''. Zij antwoorden: ``Op bevel van Christus en verheerlijk Zijn heilige Naam''. Ze buigen hun hoofd en slaan op de borst tijdens de zegening.}
\end{halfparskip}

% % % % % % % % % % % % % % % % % % % % % % % % % % % % % % % % % % % % % % % %

\markedsection{Hutama (Eindzegen)}

\begin{halfparskip}
  \liturgicalhint{1.} Glorie aan U, Jezus, onze zegevierende Koning, de Glans van de eeuwige Vader, verwekt zonder begin, vóór alle tijden en (geschapen) dingen; we hebben geen hoop en verwachting tenzij U, de Schepper. Door het gebed der rechtvaardigen en uitverkorenen die vanaf het begin U hebben behaagd (\translationoptionNl{door U goedgekeurd waren}), vergeef onze zonden, scheld kwijt onze overtredingen, verlos ons van onze ellende, verhoor onze verzoeken, breng ons naar het glanzende licht, en bewaar ons door Uw levend Kruis van alle kwaad, verborgen en open, Christus de Hoop van onze natuur, nu en altijd en in eeuwigheid. \liturgicalOption{Of:}

  \liturgicalhint{2.} Moge het gebed van Uw zwakke dienstknechten, onze Heer en onze God, aangenomen worden voor de troon van Uw Godheid; en moge deze, onze samenkomst, behagen aan de Wil van Uw majesteit; dat wij van U de gift van een goede gezondheid voor het lichaam en bescherming voor de ziel mogen ontvangen; toename van voedsel; vergeving van schulden en kwijtschelding van zonden, en eeuwige vrede, o Heer, en langdurige rust; eenheid in liefde die niet voorbijgaat en niet uit ons midden verdwijnt, in elk tijdperk van deze wereld, nu en altijd en in eeuwigheid.

  \liturgicalhint{3.} Gezegend zij God voor altijd, en verheerlijkt zij Zijn heilige Naam tot in alle eeuwigheid. Aan Hem doen wij een verzoek, en wij smeken de overstromende zee van Zijn barmhartigheid, dat Hij ons waardig zou maken van de verheven heerlijkheid van Zijn rijk, van de zaligheid met Zijn heilige engelen en de onthulling van gelaat voor Hem [= vertrouwen], en het staan aan Zijn rechterhand in het hemelse Jeruzalem, in Zijn goedheid en barmhartigheid, nu en altijd en in eeuwigheid.

  \liturgicalhint{4.} Moge God, de Heer van alles, in wiens huis we zijn samengekomen en voor wiens Majesteit we gebeden hebben, in de grote hoop op Zijn genade, ons gebed in Zijn mededogen horen, en ons verzoek in Zijn medelijden aanvaarden; en moge Hij het vuil van onze overtredingen en zonden wassen en reinigen met de hysop van Zijn overvloedige medelijden, en rust geven aan de zielen der overledenen in de glorieuze woningen van Zijn koninkrijk. Moge Hij ons allen besprenkelen met de dauw van Zijn zoetheid. En moge de rechterhand van Zijn zorg ons en alle schepselen overschaduwen in Zijn liefdevolle goedertierenheid en barmhartigheid, nu en altijd en in eeuwigheid.

  \liturgicalhint{5.} Moge God, de Heer van alles, die Zijn lofprijzingen aan onze mond heeft toevertrouwd, Zijn liederen aan onze tong, Zijn lofzangen aan onze kelen, Zijn belijdenis aan onze lippen, Zijn geloof aan ons harten, onze gebeden verhoren, onze beden aannemen, verzoend worden door (\translationoptionNl{Zijn behagen vinden in}) ons smeken, onze schulden kwijtschelden, onze verzoeken inwilligen met weldaden en niet met terechtwijzing; en moge Hij uit de grote schatkamer van Zijn barmhartigheid Zijn erbarmen en mededogen over ons en over de hele wereld storten, nu en altijd en in eeuwigheid.

  \liturgicalhint{6.} Moge de Naam van God, de Heer van alles, die tijden en seizoenen ordent, onder ons verheerlijkt worden; en moge de rechterhand van de zorg van Zijn genade ons overschaduwen, die zwak en zondig zijn, en de hele wereld, de heilige Kerk en haar kinderen, onze vaders, broeders, oversten en leraren, onze overledenen die van ons gescheiden zijn en uit ons midden zijn genomen, en heel onze broederschap in Christus, nu en altijd en in eeuwigheid.

  \liturgicalhint{7.} Aan God zij glorie, aan de engelen eer, aan Satan beschaming, aan het Kruis verering, aan de Kerk verheerlijking, aan de overledenen verkwikking, aan de boetvaardigen opname, aan de gevangenen vrijlating, aan de zieken en zwakken herstel en genezing, en aan de vier uiteinden van de wereld grote vrede en rust. Ook over ons, die zwak en zondig zijn, mogen de barmhartigheid en genade van onze aanbiddelijke God komen, mogen zij ons overschaduwen, over ons stromen, en standvastig blijven en voortdurend regeren, nu en altijd en in eeuwigheid.

  \liturgicalhint{8.} Bij de rechterhand van Uw Majesteit, onze Vader die in de hemel is, zegen ons allen, o mijn Heer; behoud ons allen; help ons allen; steun en bescherm ons allen; verkwik de overledenen; laat Uw rechterhand ons allen overschaduwen; mogen Uw genade en barmhartigheid over ons allen worden uitgestort; en moge voortdurende lof, eer, belijdenis, aanbidding en dankzegging tot U opstijgen uit de mond van ons allen, nu en altijd en in eeuwigheid

  \liturgicalhint{9.} \liturgicaloption{In een klooster of huis:} Moge God, de Heer van alles, in Zijn goedheid onze gemeenschap zegenen, en in de overvloedige menigte van Zijn barmhartigheid ons behoeden van te vallen; moge Hij onze verzoeken vanuit Zijn schatkamer beantwoorden; en moge over de hele wereld, over de heilige Kerk en haar kinderen, over dit land en haar inwoners, over deze woning en zij die erin wonen, en over ons allemaal en ieder van ons samen, de barmhartigheid en genade van ons goede God komen en voortdurend worden uitgestort, nu en altijd en in eeuwigheid.

  \rr~Amen.
\end{halfparskip}

% % % % % % % % % % % % % % % % % % % % % % % % % % % % % % % % % % % % % % % %

\end{document}