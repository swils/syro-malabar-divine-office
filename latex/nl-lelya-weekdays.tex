\documentclass[12pt,twoside,a5paper]{article}

\usepackage{multicol}

\usepackage[main=dutch]{babel}
\usepackage{divine-office}

% % % % % % % % % % % % % % % % % % % % % % % % % % % % % % % % % % % % % % % %

% Version: 2024-07-13
\begin{document}

\title{Lelya~--- weekdagen}
\author{}
\date{}
\maketitle

% The following prevents footnotes and paracol from interacting in bad ways.
% Not really an idea why...
% See: https://stackoverflow.com/questions/61779911/paracol-and-footnote-placing-in-latex
\footnotelayout{\ }

% % % % % % % % % % % % % % % % % % % % % % % % % % % % % % % % % % % % % % % %

\begin{halfparskip}
  \cc~Eer aan God in den hoge \liturgicalhint{(3x)}. En op aarde vrede en goede hoop aan de mensen, altijd en in eeuwigheid.

  [Amen].~--- \rr~Zegen, Heer.~--- \liturgicalhint{[vredekus].}

  \cc~Onze Vader die in de hemelen zijt,

  \rr~Geheiligd zij Uw Naam. Uw rijk kome, heilig, heilig, heilig zijt Gij. Onze Vader die in de hemelen zijt, de hemel en de aarde zijn gevuld met Uw onmetelijke glorie; de engelen en de mensen roepen U toe: heilig, heilig, heilig zijt Gij.~--- Onze Vader die in de hemelen zijt, geheiligd zij Uw Naam. Uw rijk kome, Uw wil geschiede op aarde zoals in de hemel. Geef ons heden het brood dat we nodig hebben en vergeef ons onze schulden en zonden zoals wij ook vergeven hebben aan onze schuldenaren. En leid ons niet in bekoring, maar verlos ons van de Kwade. Want van U is het koninkrijk en de kracht en de heerlijkheid in eeuwigheid, amen.

  \cc~Eer aan de Vader, de Zoon, en de Heilige Geest.

  \rr~Vanaf het begin en in alle eeuwigheid, amen en amen. Onze Vader die in de hemelen zijt, geheiligd zij Uw naam, Uw rijk kome, heilig, heilig, heilig zijt Gij. Onze Vader die in de hemelen zijt, de hemel en de aarde zijn gevuld met Uw onmetelijke glorie; de engelen en de mensen roepen U toe: heilig, heilig, heilig zijt Gij.

  \dd~Laat ons opstaan om te bidden, vrede zij met ons.\footnote{In deze verkorte editie hebben we elke hulala door een marmita vervangen.}

  \cc~Laat ons opstaan, o Heer, in Uw kracht, en bevestigd worden in Uw hoop, mogen we opgetild en gesterkt worden door de hoge arm van Uw macht; en mogen we waardig zijn, met de hulp van Uw goedertierenheid, om te allen tijde lof, eer, belijdenis en aanbidding tot U te verheffen, Heer van alles, Vader...
\end{halfparskip}

% % % % % % % % % % % % % % % % % % % % % % % % % % % % % % % % % % % % % % % %

\markedsection{PSALMEN}

\markedsection{Eerste marmita\footnote{Breviarium: alle Maandagen: hulala 1--3; Dinsdagen: 4--6; Woensdagen: 7--9; Donderdagen: 10--11 en 15; Vrijdagen: 16--18; Zaterdagen: 19--21; Hulale 12--14 zijn gereserveerd voor feesten en gedachtenissen.}\markedsectionhint{(origineel: 3 hullale)}}

\liturgicalhint{Eer aan... na elke marmita.}

\markedsubsectionrubricwithhint{Maandagen ``voor'': Marmita 1*}{(origineel: hulala 1)}

\begin{halfparskip}
  \psalm{\Ps{1}} Zalig de man die de raad van goddelozen niet volgt,~\sep\ die de weg van zondaars niet inslaat,

  \liturgicalhint{Alleluia, Alleluia, Alleluia.~--- Eerste vers.}

  noch neerzit in de kring van spotters.

  Maar die zijn vreugde vindt in de Wet van de Heer,~\sep\ en Zijn Wet overweegt bij dag en bij nacht.

  Hij is als een boom,~\sep\ geplant aan waterstromen.

  Die vrucht geeft op zijn tijd, en wiens lover niet verwelkt;~\sep\ ja, al wat hij doet, gedijt.

  Niet zo de goddelozen, niet zo;~\sep\ ze zijn als kaf, dat de wind verstrooit.

  Daarom zullen de goddelozen niet standhouden in het oordeel,~\sep\ noch de zondaars in de kring der
  rechtvaardigen.

  Want de Heer draagt zorg voor de weg der rechtvaardigen,~\sep\ maar de weg der goddelozen loopt uit op verderf.

  \psalm{\Ps{2}} Waarom woelen de heidenen,~\sep\ en smeden de naties ijdele plannen?

  De koningen der aarde rijzen samen op,~\sep\ en de vorsten spannen tezamen tegen de Heer en Zijn Gezalfde:

  ``Laten wij Hun boeien verbreken,~\sep\ en werpen wij Hun kluisters van ons af''.

  Die in de hemelen woont, Hij lacht hen uit,~\sep\ de Heer drijft de spot met hen.

  Dan spreekt Hij hen toe in Zijn toorn,~\sep\ in Zijn gramschap doet Hij hen sidderen:

  ``Maar Ik, Ik heb Mijn Koning aangesteld,~\sep\ op de Sion, Mijn heilige berg!''

  Ik wil het besluit van de Heer bekendmaken: de Heer sprak tot mij:~\sep\ ``Mijn Zoon zijt Gij, Ik heb U heden voortgebracht.

  Vraag Mij, en Ik geef U de volken tot erfdeel,~\sep\ en tot Uw bezit de grenzen der aarde.

  Gij zult hen regeren met ijzeren scepter,~\sep\ hen in stukken slaan als het vat van een pottenbakker''.

  Nu dan, koningen, komt tot inzicht,~\sep\ laat u gezeggen, die de wereld bestuurt.

  Dient de Heer in vreze en juicht Hem toe;~\sep\ met huivering Hem uw hulde gebracht!

  Opdat Hij niet toorne en gij van de weg af vergaat als weldra Zijn toorn zal zijn ontbrand;~\sep\ zalig allen die vluchten tot Hem.

  \psalm{\Ps{3}} Heer, hoe talrijk zijn zij, die mij kwellen!~\sep\ Velen staan tegen mij op!

  Velen zijn er, die van mij zeggen:~\sep\ ``Voor hem is er geen redding bij God!''

  Maar Gij, Heer, zijt mijn schild,~\sep\ Gij mijn roem, die mijn hoofd opbeurt.

  Ik riep tot de Heer met luide stem,~\sep\ en Hij verhoorde mij van Zijn heilige berg.

  Ik legde mij neer en ik sliep in;~\sep\ dan stond ik op, want de Heer is mijn steun.

  Neen, nu vrees ik de drommen van duizenden niet,~\sep\ die zich opstellen rondom mij heen.

  Verhef U, Heer!~\sep\ red mij, mijn God!

  Want al mijn weerstrevers hebt Gij op de kaken geslagen,~\sep\ de tanden der bozen hebt Gij verbrijzeld.

  Bij de Heer is redding:~\sep\ Uw zegen zij over Uw volk!

  \psalm{\Ps{4}} Verhoor mij, als ik U aanroep, mijn rechtvaardige God,

  die mij in kwelling verlichting bracht;~\sep\ wees mij genadig en verhoor mijn gebed.

  Mannen, hoelang nog blijft gij verstokt van hart;~\sep\ waarom ijdelheid bemind en leugen gezocht?

  Weet dat de Heer jegens Zijn heiligen wonderbaar handelt;~\sep\ de Heer zal mij verhoren, als ik Hem aanroep.

  Siddert en wilt niet zondigen,~\sep\ denkt na bij u zelf, op uw sponde, en zwijgt!

  Brengt gerechte offers,~\sep\ en hoopt op de Heer.

  Velen zeggen: ``Wie zal ons voorspoed doen zien?''~\sep\ Doe opgaan over ons, Heer, het licht van Uw gelaat!

  Gij hebt mij een vreugde in het hart gestort,~\sep\ groter dan bij overvloed van tarwe en wijn.

  Zodra ik mij neerleg, slaap ik in vrede,~\sep\ want Gij alleen, Heer, stelt mij in veiligheid.
\end{halfparskip}

\markedsubsectionrubricwithhint{Dinsdagen ``voor'': Marmita 11*}{(origineel: hulala 4)}

\begin{halfparskip}
  \psalm{\Ps{31}} Ik vlucht tot U, O Heer: dat ik nimmer te schande worde;~\sep\ bevrijd mij toch in Uw gerechtigheid!

  \liturgicalhint{Alleluia, Alleluia, Alleluia.~--- Eerste vers.}

  Neig Uw oor naar mij;~\sep\ haast U mij te redden!

  Wees mij een rots, waar ik vluchten kan,~\sep\ een versterkte burcht tot mijn behoud.

  Want Gij zijt mijn Rots en mijn burcht,~\sep\ en omwille van Uw Naam zult Gij mijn Leider zijn en Gids.

  Gij zult mij trekken uit het net, dat zij heimelijk mij spanden,~\sep\ want Gij zijt mijn toevlucht.

  In Uw handen beveel ik mijn geest:~\sep\ Gij zult mij bevrijden, O Heer, getrouwe God.

  Gij haat, die nietige afgoden dienen,~\sep\ maar ik vertrouw op de Heer.

  Vol vreugde zal ik juichen om Uw ontferming, omdat Gij op mijn ellende hebt neergezien:~\sep\ Gij waart mijn hulp in tijden van nood.

  Gij gaaft mij niet prijs aan de macht van een vijand,~\sep\ maar plaatste mijn voeten op ruime baan.

  O Heer, wees mij genadig, want ik ben in nood;~\sep\ van droefheid kwijnt mijn oog, mijn ziel en mijn lichaam.

  Ja, mijn leven teert weg in smart,~\sep\ en mijn jaren in geween.

  Van verdriet is mijn kracht gebroken,~\sep\ en mijn gebeente verdord.

  Voor al mijn vijanden ben ik tot smaad geworden, voor mijn buren tot spot, en tot afschrik voor mijn bekenden;~\sep\ die mij buiten zien, vluchten van mij heen.

  Ik ben als een dode door vergetelheid uit het hart gewist,~\sep\ en ik werd als een vat in scherven.

  Ja, ik hoorde het fluiten van velen - verschrikking van alle zijden!~\sep\ Zij schoolden tegen mij samen en zonnen op moord.

  Maar ik, O Heer, vertrouw op U,~\sep\ en zeg: Gij zijt mijn God.

  Mijn levenslot ligt in Uw hand,~\sep\ ontruk mij aan de hand van mijn vijanden en vervolgers.

  Toon aan Uw dienstknecht Uw vredig gelaat,~\sep\ red mij in Uw barmhartigheid.

  Heer, laat mij niet beschaamd worden, want U riep ik aan;~\sep\ maar dat de bozen zich schamen en zwijgen, voortgedreven naar het dodenrijk.

  Dat de leugenlippen verstommen,~\sep\ die vermetel tegen de rechtvaardige spreken, in trots en verachting.

  Hoe groot, Heer, is Uw goedheid,~\sep\ die Gij hebt weggelegd voor hen, die U vrezen,

  Die Gij bewijst aan hen, die vluchten tot U,~\sep\ ten aanschouwen der mensen.

  Gij beschermt hen onder de schutse van Uw aanschijn,~\sep\ tegen het samenzweren der mannen,

  Gij verbergt hen in Uw tent,~\sep\ tegen het schelden der tongen.

  Gezegend de Heer, want Hij bewees mij~\sep\ Zijn wondere barmhartigheid in de versterkte stad.

  Wel sprak ik in mijn onrust:~\sep\ ``Ik ben van Uw aanschijn verstoten.''

  Maar Gij hebt de stem van mijn smeken gehoord,~\sep\ daar ik tot U riep.

  Bemint de Heer, gij, al Zijn heiligen:~\sep\ de getrouwen behoedt de Heer,

  Maar overvloedig vergeldt Hij,~\sep\ die handelen in trots.

  Houdt moed, en weest sterk van hart,~\sep\ gij allen, die hoopt op de Heer.

  \psalm{\Ps{32}} Gelukkig hij, wiens misdaad vergeven,~\sep\ wiens zonde bedekt is.

  Gelukkig de mens, wie de Heer zijn schuld niet toerekent,~\sep\ en in wiens geest geen bedrog is.

  Zolang ik bleef zwijgen, werd mijn gebeente verteerd,~\sep\ onder mijn aanhoudend gezucht.

  Want dag en nacht drukte Uw hand op mij;~\sep\ mijn kracht teerde weg als bij zomerse hitte.

  Mijn zonde heb ik beleden voor U,~\sep\ en mijn schuld hield ik niet verborgen;

  Ik sprak: ``Ik belijd mijn boosheid voor de Heer,''~\sep\ en Gij hebt de schuld van mijn zonde vergeven.

  Daarom moet iedere vrome bidden tot U,~\sep\ in tijden van nood.

  En al breekt dan de watervloed los,~\sep\ hem zal hij niet genaken.

  Gij zijt mijn Toeverlaat, Gij zult voor nood mij behoeden,~\sep\ en mij omgeven met vreugde over mijn redding.

  Ik zal u leren, en de weg wijzen, die gij moet gaan,~\sep\ u onderrichten en vast Mijn ogen richten op u.

  Weest niet als paard en muilezel zonder verstand, wier onstuimigheid men bedwingt met toom en gebit,~\sep\ anders komen ze niet naar u toe.

  Veel smart valt de boze ten deel,~\sep\ maar erbarming omgeeft wie vertrouwt op de Heer.

  Verheugt u in de Heer, weest blijde, gij, rechtvaardigen,~\sep\ en jubelt, gij allen, die oprecht zijt van harte.
\end{halfparskip}

\markedsubsectionrubricwithhint{Woensdagen ``voor'': Marmita 17*}{(origineel: hulala 7)}

\begin{halfparskip}
  \psalm{\Ps{44}} Met eigen oren, O God, hebben wij het gehoord,~\sep\ onze vaderen hebben het ons verhaald:

  \liturgicalhint{Alleluia, Alleluia, Alleluia.~--- Eerste vers.}

  Het werk, dat Gij gewrocht hebt in hun dagen,~\sep\ in overoude tijden.

  Met eigen hand hebt Gij de heidenen verdreven, maar hen geplant;~\sep\ om hen te doen bloeien hebt Gij volken verslagen.

  Neen, hun zwaard was het niet, dat het land heeft veroverd,~\sep\ noch hun arm, die hun redding bracht.

  Maar Uw rechter en Uw arm,~\sep\ en het licht van Uw gelaat, want Gij hadt hen lief.

  Gij zijt mijn Koning, mijn God,~\sep\ die de zege verleende aan Jacob.

  Door U hebben wij onze tegenstanders verdreven,~\sep\ en in Uw Naam vertrapten wij die tegen ons waren opgestaan.

  Neen, ik heb niet vertrouwd op mijn boog,~\sep\ noch was het mijn zwaard, dat mij redding bracht.

  Maar Gij hebt ons verlost van onze weerstrevers,~\sep\ en die ons haten, hebt Gij beschaamd.

  In God roemden wij ten allen tijde,~\sep\ en Uw Naam prezen wij immer.

  Maar nu hebt Gij ons verstoten en te schande gemaakt,~\sep\ en Gij trekt niet meer op, God, met onze legerscharen.

  Gij hebt ons doen wijken voor onze tegenstanders;~\sep\ die ons haten, hebben ons uitgeplunderd.

  Als slachtschapen hebt Gij ons overgeleverd,~\sep\ en onder de volken verstrooid.

  Voor een spotprijs hebt Gij Uw volk verkocht,~\sep\ en weinig winst heeft U de verkoop gebracht.

  Gij hebt ons te schande gemaakt voor onze buren,~\sep\ tot spot en hoon voor hen, die ons omringen.

  Gij hebt ons tot spreekwoord gemaakt onder de heidenen;~\sep\ de volken schudden het hoofd over ons.

  Mijn schande staat mij immer voor ogen,~\sep\ en schaamte bedekt mijn gelaat,

  Om de praatjes van schimper en spotter,~\sep\ om vijand en tegenstander.

  Dit alles kwam over ons; en toch: wij zijn U niet vergeten;~\sep\ en hebben Uw verbond niet geschonden;

  Ook viel ons hart niet van U af,~\sep\ en onze schreden weken niet af van Uw wegen,

  Toen Gij ons gebroken hebt in het oord der kwelling,~\sep\ en ons met duisternis hebt omhuld.

  Hadden wij de Naam van onze God vergeten,~\sep\ en naar een vreemde god onze handen uitgestrekt,

  Zou God dat niet hebben ontdekt?~\sep\ Hij toch doorschouwt de geheimen van het hart.

  Ja, om Uwentwil blijft men ons doden,~\sep\ worden wij als slachtschapen beschouwd.

  Ontwaak dan, Heer, wat slaapt Gij?~\sep\ Waak op, blijf ons niet eeuwig verstoten!

  Waarom verbergt Gij Uw aanschijn,~\sep\ vergeet Gij onze ellende en onze verdrukking?

  Want onze ziel ligt neer in het stof,~\sep\ ons lichaam kleeft aan de aarde.

  Sta op om ons te helpen,~\sep\ en bevrijd ons om Uw barmhartigheid.

  \psalm{\Ps{45}} Een heerlijk lied welt op uit mijn hart: de Koning wijd ik mijn zang;~\sep\ mijn tong is de stift van een vaardige schrijver.

  Gij zijt de schoonste onder de kinderen der mensen; bevalligheid ligt op uw lippen:~\sep\ daarom heeft God u voor eeuwig gezegend.

  Gord uw zwaard om de heup, gij, machtige held,~\sep\ uw sieraad en luister.

  Ruk zegerijk uit voor waarheid en recht;~\sep\ uw rechterhand lere u roemrijke daden.

  Uw pijlen zijn scherp; volkeren worden aan u onderworpen;~\sep\ aan de vijanden van de Koning ontzinkt de moed.

  In de eeuwen der eeuwen staat Uw troon, O God,~\sep\ een scepter van recht is de scepter van Uw rijk.

  Gij hebt de gerechtigheid lief en haat de boosheid; daarom heeft God, uw God, u gezalfd,~\sep\ met de olie der vreugde boven uw genoten.

  Van mirre en aloë en cassia geuren uw gewaden;~\sep\ uit ivoren paleizen klinkt u blij het harpgeluid tegen.

  Koningsdochters treden u tegemoet,~\sep\ de Koningin staat aan uw rechterhand, met goud uit Ofir getooid.

  Hoor, dochter, en zie, en neig uw oor,~\sep\ en vergeet uw volk en het huis van uw vader.

  Dan zal aan de Koning uw schoonheid behagen;~\sep\ Hij is uw Heer, breng Hem uw hulde.

  Dan komt met geschenken het volk van Tyrus,~\sep\ de voornamen onder het volk dingen om uw gunst.

  In volle luister treedt de dochter van de Koning binnen;~\sep\ met goud doorweven is haar gewaad.

  In een kleurige mantel wordt zij voor de Koning geleid,~\sep\ in haar gevolg worden maagden, haar
  gezellinnen, tot u gevoerd;

  Zij worden voorgeleid in blijde jubel,~\sep\ en treden het paleis van de Koning binnen.

  In de plaats van uw vaderen komen uw zonen:~\sep\ gij zult hen aanstellen tot vorsten over heel de wereld.

  Ik zal uw naam doen gedenken bij alle geslachten;~\sep\ daarom zullen de volken u prijzen in de eeuwen der eeuwen.

  \psalm{\Ps{46}} God is ons een Toevlucht en Kracht;~\sep\ een machtige Helper toonde Hij zich in de nood.

  Daarom vrezen wij niet, al wordt ook de aarde geschokt,~\sep\ al storten de bergen midden in zee.

  Laat bruisen en koken haar wateren,~\sep\ laat schudden de bergen door haar geweld:

  De Heer der heerscharen is met ons,~\sep\ de God van Jacob is onze bescherming.

  De armen van de stroom verblijden de stad van God,~\sep\ de heilige woontent van de Allerhoogste.

  God woont daarbinnen, zij zal niet wankelen;~\sep\ God staat haar bij van de vroege dageraad af.

  Volkeren woedden, en koninkrijken werden geschokt:~\sep\ daar galmde Zijn donderstem, en weg vloeide de aarde.

  De Heer der heerscharen is met ons,~\sep\ de God van Jacob is onze bescherming.

  Komt en aanschouwt de werken van de Heer,~\sep\ de wonderen, die Hij op aarde gewrocht heeft.

  Hij bedwingt de oorlogen tot aan het einde der aarde,~\sep\ Hij verbrijzelt de bogen, breekt lansen stuk, in het vuur verbrandt Hij de schilden.

  Houdt op, en erkent Mij als God,~\sep\ verheven onder de volken, verheven op aarde.

  De Heer der heerscharen is met ons,~\sep\ de God van Jacob is onze bescherming.
\end{halfparskip}

% % % % % % % % % % % % % % % % % % % % % % % % % % % % % % % % % % % % % % % %

\end{document}