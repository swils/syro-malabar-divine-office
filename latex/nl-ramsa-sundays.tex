\documentclass[12pt,twoside,a5paper]{article}

\usepackage{multicol}

\usepackage[main=dutch]{babel}
\usepackage{divine-office}

% % % % % % % % % % % % % % % % % % % % % % % % % % % % % % % % % % % % % % % %

% Version: 2024-07-13
\begin{document}

\title{Ramsa~--- Zondagen, feesten en gedachtenissen}
\author{}
\date{}
\maketitle

% The following prevents footnotes and paracol from interacting in bad ways.
% Not really an idea why...
% See: https://stackoverflow.com/questions/61779911/paracol-and-footnote-placing-in-latex
\footnotelayout{\ }

% % % % % % % % % % % % % % % % % % % % % % % % % % % % % % % % % % % % % % % %

\begin{halfparskip}
  \cc~Eer aan God in de hoge \liturgicalhint{(3x)}. En op aarde vrede en goede hoop aan de mensen, altijd en in eeuwigheid.

  \rr~Amen. Zegen Heer. \liturgicalhint{(Vredekus)}

  \cc~Onze Vader die in de hemelen zijt,

  \rr~Geheiligd zij Uw Naam. Uw rijk kome, heilig, heilig, heilig zijt Gij. Onze Vader die in de hemelen zijt, de hemel en de aarde zijn gevuld met Uw onmetelijke glorie; de engelen en de mensen roepen U toe: heilig, heilig, heilig zijt Gij.~--- Onze Vader die in de hemelen zijt, geheiligd zij Uw Naam. Uw rijk kome, Uw wil geschiede op aarde zoals in de hemel. Geef ons heden het brood dat we nodig hebben en vergeef ons onze schulden en zonden zoals wij ook vergeven hebben aan onze schuldenaren. En leid ons niet in bekoring, maar verlos ons van de Kwade. Want van U is het koninkrijk en de kracht en de heerlijkheid in eeuwigheid, amen.

  \cc~Eer aan de Vader, de Zoon, en de Heilige Geest.

  \rr~Vanaf het begin en in alle eeuwigheid, amen en amen. Onze Vader die in de hemelen zijt, geheiligd zij Uw naam, Uw rijk kome, heilig, heilig, heilig zijt Gij. Onze Vader die in de hemelen zijt, de hemel en de aarde zijn gevuld met Uw onmetelijke glorie; de engelen en de mensen roepen U toe: heilig, heilig, heilig zijt Gij.

  \dd~Laat ons bidden, vrede zij met ons.

  \liturgicaloption{Zondagen en feesten:} \cc~Laat ons, mijn Heer, Uw Godheid met geestelijke lofzangen belijden, Uw Majesteit aanbidden met aardse aanbiddingen, en Uw geheime en verborgen Natuur verheerlijken met reine en heilige gedachten, Heer van alles, Vader, Zoon en Heilige Geest in alle eeuwigheid.~--- \rr~Amen.

  \liturgicaloption{Gedachtenissen:} \cc~We willen, Heer, Uw Godheid prijzen (herhaal) en Uw Majesteit aanbidden en aan Uw glorierijke Drievuldigheid onophoudelijk lof brengen voor altijd, Heer van alles, Vader...~--- \rr~Amen.
\end{halfparskip}

% % % % % % % % % % % % % % % % % % % % % % % % % % % % % % % % % % % % % % % %

\markedsection{Marmita}

\begin{halfparskip}
  \liturgicalhint{Begin de marmita van de dag als aangegeven in de Hudra of Gaza. Elke psalm heeft een eigen antifoon op feesten en gedachtenissen; de antifoon op zondagen en feries zijn 3 alleluias. Herhaal het eerste vers.}
\end{halfparskip}

\begin{halfparskip}
  \markedday{a) Zondagen en feesten van Advent tot Epifanie, \Pss{87--88}.}
\end{halfparskip}

\begin{halfparskip}
  \psalm{\Ps{87}} Zijn stichting op de heilige bergen bemint de Heer:~\sep\ de poorten van Sion~--- \liturgicalhint{alleluia, alleluia, alleluia}.

  Zijn stichting op de heilige bergen bemint de Heer:~\sep\ de poorten van Sion boven alle tenten van Jacob.

  Roemrijke dingen verhaalt men van u,~\sep\ O stad van God!

  Rahab en Babel zal Ik tot Mijn vereerders rekenen:~\sep\ zie, Filistea en Tyrus en het volk der Ethiopiërs: daar zijn ze geboren!

  Over Sion zal men zeggen: ``Allen, man voor man, zijn in haar geboren,~\sep\ en de Allerhoogste zelf heeft haar bevestigd.''

  De Heer zal schrijven in het boek der volken:~\sep\ ``Daar zijn ze geboren.''

  En in reidans zullen zij zingen:~\sep\ ``Al mijn bronnen zijn in u.''
\end{halfparskip}

\begin{halfparskip}
  \psalm{\Ps{88}} Heer, mijn God, ik roep overdag,~\sep\ en ik jammer 's nachts voor Uw aanschijn.

  Dringe mijn bede toch door tot U,~\sep\ neig Uw oor naar mijn klagen!

  Want mijn ziel is verzadigd met rampen,~\sep\ mijn leven is het dodenrijk nabij.

  Ik word gerekend onder hen, die ten grave dalen,~\sep\ ik ben als een man zonder kracht.

  Onder de doden is mijn legerstede,~\sep\ als van verslagenen, die liggen in het graf,

  Aan wie Gij niet meer denkt,~\sep\ die aan Uw zorgen zijn onttrokken.

  In een diepe groeve hebt Gij mij neergelegd,~\sep\ in duisternis, in een diep ravijn.

  Uw verontwaardiging drukt zwaar op mij,~\sep\ met al Uw golven slaat Gij mij neer.

  Gij hebt mijn vrienden van mij vervreemd, mij tot afschuw voor hen gemaakt;~\sep\ ik zit gevangen, en kan niet ontkomen.

  Van ellende verkwijnen mijn ogen; iedere dag roep ik tot U, O Heer,~\sep\ naar U strek ik mijn handen uit.

  Of doet Gij voor doden nog wonderen,~\sep\ of zullen gestorvenen, herrijzend, U loven?

  Of wordt Uw goedheid in het graf verkondigd,~\sep\ Uw trouw in het dodenrijk?

  Openbaart men in het duister Uw wonderen,~\sep\ in het land der vergetelheid Uw genade?

  Ik echter roep tot U, O Heer,~\sep\ mijn bede stijgt tot U op in de morgen.

  Waarom toch, O Heer, verstoot Gij mij,~\sep\ verbergt Gij voor mij Uw gelaat?

  Van jongs af ben ik ellendig en stervend,~\sep\ ik torste Uw verschrikkingen en kwijnde.

  Uw toorn is over mij heengegaan,~\sep\ Uw verschrikkingen sloegen mij neer.

  Zij omgeven mij immer als water,~\sep\ omringen mij alle tezamen.

  Vriend en makker hebt Gij van mij vervreemd,~\sep\ mijn vertrouweling is de duisternis.

  \liturgicalhint{Eer...}~--- \liturgicalhint{3x alleluia.}~--- Zijn stichting op de heilige bergen bemint de Heer:~\sep\ de poorten van Sion.
\end{halfparskip}

\begin{halfparskip}
  \markedday{b) Andere feesten en zondagen, \Pss{65--67}.}
\end{halfparskip}

\begin{halfparskip}
  \psalm{\Ps{65}} Aan U, O God, komt een lofzang toe in Sion;~\sep\ men volbrenge zijn gelofte aan U, die de bede verhoort.

  \liturgicalhint{3x alleluia.}~--- \liturgicalhint{Eerste vers.}

  Tot U komt alle vlees,~\sep\ omwille der ongerechtigheden.

  Onze misdaden drukken ons neer:~\sep\ Gij scheldt ze kwijt.

  Gelukkig die Gij uitkiest en tot U neemt:~\sep\ hij woont in Uw voorhoven.

  Dat wij verzadigd worden met de goederen van Uw huis,~\sep\ met de heiligheid van Uw tempel.

  Met gerechtigheid verhoort Gij ons door wondere tekenen,~\sep\ God, onze Redder,

  Gij zijt de Hoop van alle grenzen der aarde,~\sep\ en van de verre zeeën;

  Die de bergen vastlegt door Uw kracht,~\sep\ die met macht zijt omgord,

  Die het bulderen der zee bedwingt,~\sep\ het bulderen van haar golven en het woelen der naties.

  En die de grenzen der aarde bewonen, huiveren om Uw tekenen:~\sep\ met vreugde vervult Gij het uiterste oosten en westen.

  Gij hebt de aarde bezocht en haar besproeid,~\sep\ met rijkdommen haar overstelpt.

  De stroom van God is met water gevuld: Gij hebt hun graan bereid;~\sep\ zo hebt Gij haar gereed gemaakt:

  Haar voren hebt Gij besproeid,~\sep\ geëffend haar kluiten,

  Door stortregens hebt Gij haar geweekt,~\sep\ en haar gewas gezegend.

  Met Uw mildheid hebt Gij het jaar gekroond,~\sep\ en Uw wegen druipen van vet.

  De weiden der woestijn druipen ervan,~\sep\ en de heuvelen omgorden zich met jubel.

  De weiden zijn met kudden bekleed en de dalen met koren bedekt:~\sep\ zij juichen U toe en zingen!
\end{halfparskip}

\begin{halfparskip}
  \psalm{\Ps{66}} Juicht God toe, alle landen, bezingt de glorie van Zijn Naam,~\sep\ heft voor Hem een heerlijk loflied aan.

  Zegt tot God: Hoe ontzagwekkend zijn Uw werken!~\sep\ Om Uw geweldige kracht brengen Uw vijanden U vleiend hulde.

  Dat heel de aarde U aanbidde en voor U zinge,~\sep\ dat zij bezinge Uw Naam.

  Komt en ziet de werken van God:~\sep\ wondere daden volbracht Hij onder de kinderen der mensen!

  De zee veranderde Hij in land, en zij trokken te voet door de stroom;~\sep\ laten wij daarom over Hem ons verheugen.

  Eeuwig heerst Hij door Zijn macht; Zijn ogen slaan de volken gade:~\sep\ dat de weerspannigen zich niet verheffen.

  Zegent, gij volken, onze God,~\sep\ en verkondigt Zijn wijd verbreide lof.

  Hij behield ons in leven,~\sep\ en liet onze voeten niet wankelen.

  Want Gij hebt ons beproefd, O God,~\sep\ met vuur ons gelouterd, zoals men zilver loutert.

  Gij liet ons de strik inlopen,~\sep\ een zware last hebt Gij op onze heupen gelegd.

  Mensen liet Gij over onze hoofden treden; wij zijn door vuur en water gegaan,~\sep\ maar Gij hebt ons uitkomst gebracht.

  Met brandoffers wil ik Uw huis betreden,~\sep\ U mijn geloften inlossen,

  Die mijn lippen hebben uitgesproken,~\sep\ en mijn mond heeft beloofd in mijn kwelling.

  Brandoffers wil ik U brengen van vette schapen met het vet van rammen,~\sep\ runderen en bokken zal ik offeren.

  Komt en hoort, gij allen, die God vreest,~\sep\ ik wil U verhalen hoe grote dingen Hij aan mij gedaan heeft!

  Ik riep Hem aan met mijn mond,~\sep\ en prees Hem met mijn tong.

  Had ik in mijn hart op boosheid gezonnen,~\sep\ dan had de Heer mij niet verhoord.

  Maar God heeft mij verhoord,~\sep\ heeft gelet op de stem van mijn smeken.

  Gezegend zij God, die mijn bede niet heeft versmaad,~\sep\ mij Zijn ontferming niet heeft onthouden.
\end{halfparskip}

\begin{halfparskip}
  \psalm{\Ps{67}} God zij ons genadig en zegene ons;~\sep\ Hij tone ons Zijn vredig gelaat.

  Opdat men op aarde Zijn weg lere kennen,~\sep\ onder alle volken Zijn heil.

  Dat de volken U prijzen, O God,~\sep\ dat alle volken U prijzen!

  Laat juichen en jubelen de naties, omdat Gij met rechtvaardigheid de volken regeert,~\sep\ en de naties op aarde bestuurt.

  Dat de volken U prijzen, O God,~\sep\ dat alle volken U prijzen!

  De aarde heeft haar vrucht gegeven;~\sep\ God, onze God, heeft ons gezegend.

  Dat God ons zegene,~\sep\ en dat alle grenzen der aarde Hem vrezen!

  \liturgicalhint{Eer...}~--- \liturgicalhint{3x alleluia.}~--- Aan U, O God, komt een lofzang toe in Sion;~\sep\ men volbrenge zijn gelofte aan U, die de bede verhoort.
\end{halfparskip}

\begin{halfparskip}
  \markedday{c) Gedachtenissen op Vrijdagen, \Pss{85--86}.}
\end{halfparskip}

\begin{halfparskip}
  \psalm{\Ps{85}} Gij zijt Uw land genadig geweest, O Heer,~\sep\ hebt het lot van Jacob ten goede gekeerd.

  \liturgicalhint{3x alleluia.}~--- Gij zijt Uw land genadig geweest, O Heer,~\sep\ hebt het lot van Jacob ten goede gekeerd.

  Vergeven hebt Gij de schuld van Uw volk,~\sep\ en al zijn zonden bedekt.

  Uw gramschap hebt Gij geheel bedwongen,~\sep\ de gloed van Uw toorn gestild.

  Herstel ons, o God, onze Redder,~\sep\ en leg Uw wrevel tegen ons af.

  Zult Gij dan eeuwig tegen ons toornen,~\sep\ of verbolgen blijven van geslacht tot geslacht?

  Zult Gij ons dan niet opnieuw doen leven,~\sep\ opdat Uw volk zich verblijde in U?

  Toon ons, Heer, Uw barmhartigheid,~\sep\ en schenk ons Uw heil!

  Ik wil horen naar wat de Heer God spreekt:~\sep\ vrede voorzeker kondigt Hij aan.

  Voor Zijn volk en Zijn heiligen,~\sep\ en voor hen, die zich van harte keren tot Hem.

  Ja waarlijk, Zijn heil is nabij voor wie Hem vrezen,~\sep\ en zo zal er glorie wonen in ons land:

  Barmhartigheid en trouw zullen elkander ontmoeten,~\sep\ gerechtigheid en vrede elkander de kus geven.

  Trouw zal aan de aarde ontspruiten,~\sep\ en gerechtigheid neerzien vanuit de hemel.

  De Heer zelf zal zegen schenken,~\sep\ en ons land zijn vruchten geven.

  Gerechtigheid zal vóór Hem uitgaan,~\sep\ en heil zijn schreden volgen.
\end{halfparskip}

\begin{halfparskip}
  \psalm{\Ps{86}} Neig Uw oor, O Heer; verhoor mij,~\sep\ want ik ben ellendig en arm.

  Bescherm mij, want ik ben U toegewijd;~\sep\ red Uw dienaar, die op U hoopt.

  Mijn God zijt Gij; wees mij genadig, O Heer,~\sep\ want almaar door roep ik tot U.

  Verblijd de ziel van Uw dienaar,~\sep\ want tot U, O Heer, verhef ik mijn ziel.

  Want Gij, O Heer, zijt goed en genadig,~\sep\ vol erbarming voor al wie U aanroept.

  Luister, Heer, naar mijn bede,~\sep\ en geef acht op de stem van mijn smeken.

  Op de dag van mijn kwelling riep ik tot U,~\sep\ omdat Gij mij verhoren zult.

  Onder de goden, O Heer, is er geen als Gij,~\sep\ en geen werk is gelijk aan het Uwe.

  Alle volken, door U geschapen, zullen komen, en U aanbidden, O Heer,~\sep\ en verheerlijken Uw Naam.

  Want Gij zijt groot en Gij doet wonderwerken:~\sep\ Gij zijt God, en Gij alleen.

  Toon mij Uw weg, O Heer, opdat ik wandele in Uw waarheid,~\sep\ richt mijn hart op de vrees voor Uw Naam.

  Ik zal U prijzen, Heer, mijn God, uit heel mijn hart:~\sep\ en eeuwig Uw Naam verheerlijken.

  Want Uw erbarming voor mij was groot,~\sep\ en uit de diepten van het dodenrijk hebt Gij mij opgehaald.

  Trotsen, O God, zijn tegen mij opgestaan, een bende geweldenaars staat mij naar het leven,~\sep\ zij houden U niet voor ogen.

  Maar Gij, O Heer, zijt een barmhartige en liefdevolle God,~\sep\ lankmoedig, rijk aan ontferming en trouw.

  Blik op mij neer en wees mij genadig;~\sep\ schenk aan Uw dienaar Uw kracht, en red de zoon van Uw dienstmaagd.

  Geef mij een teken van Uw gunst, opdat die mij haten, Heer, vol schaamte zien,~\sep\ dat Gij, O Heer, mij hulp en troost hebt geschonken.

  \liturgicalhint{Eer...}~--- \liturgicalhint{3x alleluia.}~--- Gij zijt Uw land genadig geweest, O Heer,~\sep\ hebt het lot van Jacob ten goede gekeerd.
\end{halfparskip}

\begin{halfparskip}
  \markedday{d) Gedachtenissen op andere dagen, \Pss{15--17}}
\end{halfparskip}

\begin{halfparskip}
  \liturgicalhint{Zie Psalterium.}

  \liturgicalhint{Eer...}~--- \liturgicalhint{Alleluia, alleluia, alleluia.}~--- \liturgicalhint{Eerste vers.}
\end{halfparskip}

\begin{halfparskip}
  \dd~Laat ons bidden; vrede zij met ons.

  \liturgicalOption{Zondagen en feesten.} \cc~Wij moeten altijd de grote, vreeswekkende, heilige, gezegende, uitmuntende en onbegrijpelijke Naam van Uw roemrijke Drieëenheid en Uw goedheid jegens ons geslacht belijden, aanbidden
  en verheerlijken, Heer van alles, Vader, Zoon en H.~Geest in eeuwigheid.~--- \rr~Amen.

  \liturgicalOption{Gedachtenissen.} \cc~U, goede, vriendelijke, medelevende, vol van barmhartigheden, grote Koning van glorie, Wezen dat van eeuwigheid is, belijden, aanbidden en verheerlijken wij te allen tijde, Heer van alles...
\end{halfparskip}

% % % % % % % % % % % % % % % % % % % % % % % % % % % % % % % % % % % % % % % %

\markedsection{Wierookhymne (Aik etra)}

\begin{halfparskip}
  \liturgicalOption{1. Zon- en feestdagen.} 1) Hoe geliefd zijn Uw woningen, U Heer der heerscharen\footnote{\liturgicalhint{Wierookzegen:} Wij willen te allen tijde Uw glorieuze Drie-eenheid loven, amen. \liturgicalhint{Of:} Christus, die het bloed der martelaren aanvaardde op de dag dat ze werden gedood, aanvaard in de goedheid van Uw mededogen deze wierook voor altijd uit mijn zwakke handen, amen.}.

  \liturgicalhint{Tune: Bahar Lemba~/ Misiha Kartave}
\end{halfparskip}

\begin{doublecols}
  \englishl Our Lord Jesus Christ, mankind's Saviour~/ accept in mercy our supplications~/ like sweet smelling frankincense~/ which we offer You.

  \dutchc{1} Als de geur van zoete wierook en de geur van een aangenaam wierookvat, ontvang, Christus onze Verlosser, het verzoek en gebed van Uw dienaren.
\end{doublecols}

\begin{halfparskip}
  2) Mijn ziel verlangt en smacht naar de voorhoven van de Heer.~--- \liturgicalhint{Our Lord...}~/ \liturgicalhint{Als de geur...}
\end{halfparskip}

\begin{halfparskip}
  \liturgicalOption{2. Gedachtenissen.} 1) Ik zal altijd de Heer zegenen.~--- \liturgicalhint{Our Lord Jesus...} \liturgicaloption{of:} \liturgicalhint{Als de geur...}

  2) En Zijn lof zal altijd in mijn mond zijn.~--- \liturgicalhint{Our Lord Jesus...} \liturgicaloption{of:} \liturgicalhint{Als de geur...}
\end{halfparskip}

\begin{halfparskip}
  \liturgicalOption{3. Feesten van de Heer.} 3) Mijn hart en mijn lichaam juichen voor de levende God.

  4) Mijn Koning en mijn God, gezegend zijn zij die in Uw huis verblijven.
\end{halfparskip}

\begin{halfparskip}
  \liturgicalOptionUl{Alle dagen.}

  \liturgicalhint{Eer...}~--- Vanaf het begin en in alle eeuwigheid, amen en amen.~--- \liturgicalhint{Our Lord...}~/ \liturgicalhint{Als de geur...}

  \dd~Vrede zij met ons.

  \cc~Voor al Uw hulp en genaden aan ons, die niet terugbetaald kunnen worden, zullen wij U onophoudelijk belijden en verheerlijken in Uw gekroonde Kerk, die vol is van alle hulp en alle zegeningen, want U bent de Heer en Schepper van alles, Vader...~--- \rr~Amen.
\end{halfparskip}

% % % % % % % % % % % % % % % % % % % % % % % % % % % % % % % % % % % % % % % %

\markedsection{Laku Mara}

\vspace{0.4em}
\begin{doublecols}
  \textsizex

  \englishl 1. You, Lord of all, we worship You~/ Jesus Christ, we exalt You~/ You give life to our bodies~/ and salvation to our souls.

  \dutchc{1} 1. U, Heer van alles, prijzen wij; U, Jezus Christus loven wij; U bent de Levendmaker van onze lichamen; U bent de Verlosser van onze zielen.
\end{doublecols}

\begin{halfparskip}
  2. Ik was blij toen men me zei, Wij gaan op naar het huis van de Heer.~--- \liturgicalhint{You, Lord...}~/ \liturgicalhint{U, Heer van alles...}

  \liturgicallbracket\liturgicaloption{In een privaat huis (Assyrische Hudra):} In elke plaats zijt Gij, God, aanvaard onze bede.\liturgicalrbracket

  3. \liturgicalhint{Eer...}~--- Vanaf het begin en in alle eeuwigheid, amen en amen.~--- \liturgicalhint{You, Lord...}~/ \liturgicalhint{U, Heer van alles...}

  \liturgicalOption{Op feesten van de Heer.} 2. Ik was blij toen men me zei.~--- \liturgicalhint{You, Lord...}~/ \liturgicalhint{U, Heer van alles...}

  3. Wij gaan op naar het huis van de Heer.~--- \liturgicalhint{You, Lord...}~/ \liturgicalhint{U, Heer van alles...}

  4. Eer aan de Vader, de Zoon en de Heilige Geest.~--- \liturgicalhint{You, Lord...}~/ \liturgicalhint{U, Heer van alles...}

  5. Vanaf het begin en in alle eeuwigheid, amen en amen.~--- \liturgicalhint{You, Lord...}~/ \liturgicalhint{U, Heer van alles...}

  \dd~Laat ons bidden; vrede zij met ons.

  \cc~U bent waarlijk de Levendmaker van onze lichamen, de goede Verlosser van onze zielen, en de trouwe Bewaker van onze levens. U moeten wij altijd loven, aanbidden en verheerlijken, Heer van alles in alle eeuwigheid.~--- \rr~Amen.
\end{halfparskip}

% % % % % % % % % % % % % % % % % % % % % % % % % % % % % % % % % % % % % % % %

\markedsection{Suraya Da'Qdam}

\liturgicalhint{(Op zondagen, niet op feesten en gedachtenissen. Met 3x alleluia in het begin en op het einde.)}


\markedday{Eerste Zondag van elk seizoen, \Ps{47,1--5}.} \liturgicalhint{Van Hemelvaart tot Aankondiging voeg er vv.5-9 aan toe.}

\begin{halfparskip}
  Volken, gij alle, klapt in de handen,~\sep\ juicht God toe met jubelzang!

  \liturgicalhint{3x Alleluia.}~--- \liturgicalhint{Herhaal het eerste vers.}

  Want hoogverheven, ontzagwekkend is de Heer,~\sep\ de grote Koning van heel de aarde.

  Hij onderwerpt ons de volken,~\sep\ en legt de naties onder onze voeten.

  Ons erfdeel kiest Hij voor ons uit,~\sep\ de roem van Jacob, die Hij liefheeft.

  Hemelvaart tot Aankondiging: God stijgt op onder gejubel,~\sep\ de Heer onder bazuingeschal.

  Zingt voor God, zingt Hem toe,~\sep\ zingt voor onze Koning, zingt Hem toe!

  Want God is Koning over heel de aarde:~\sep\ zingt een lofzang.

  God heerst over de volkeren,~\sep\ God zetelt op Zijn heilige troon.
\end{halfparskip}

\markedday{Tweede Zondag, \Ps{65,1--5}.}

\begin{halfparskip}
  Aan U, o God, komt een lofzang toe in Sion;~\sep\ men volbrenge zijn gelofte aan U, die de bede verhoort.

  \liturgicalhint{3x Alleluia.}~--- \liturgicalhint{Herhaal het eerste vers.}

  Tot U komt alle vlees,~\sep\ omwille der ongerechtigheden.

  Onze misdaden drukken ons neer:~\sep\ Gij scheldt ze kwijt.

  Gelukkig die Gij uitkiest en tot U neemt:~\sep\ hij woont in Uw voorhoven.

  Dat wij verzadigd worden met de goederen van Uw huis,~\sep\ met de heiligheid van Uw tempel.

  Met gerechtigheid verhoort Gij ons door wondere tekenen,~\sep\ God, onze Redder,

  Gij zijt de Hoop van alle grenzen der aarde,~\sep\ en van de verre zeeën;
\end{halfparskip}

\markedday{Derde Zondag, \Ps{89,1--5}.}

\begin{halfparskip}
  De gunsten van de Heer wil ik eeuwig bezingen,~\sep\ door alle geslachten heen zal mijn mond Uw trouw verkondigen,

  \liturgicalhint{3x Alleluia.}~--- \liturgicalhint{Herhaal het eerste vers.}

  Want Gij hebt gezegd; ``De genade staat eeuwig vast'';~\sep\ in de hemel hebt Gij Uw trouw gegrondvest.

  ``Een verbond ging Ik aan met Mijn uitverkorene;~\sep\ aan David, Mijn dienaar, zwoer Ik een eed:

  Ik zal uw nazaat voor eeuwig bevestigen,~\sep\ en uw troon in stand houden door alle geslachten.''
\end{halfparskip}

\markedday{Vierde Zondag, \Ps{93}.}

\begin{halfparskip}
  De Heer is Koning, met majesteit bekleed;~\sep\ bekleed is de Heer met macht, Hij heeft Zich omgord;

  \liturgicalhint{3x Alleluia.}~--- \liturgicalhint{Herhaal het eerste vers.}

  Hij heeft het aardrijk bevestigd,~\sep\ dat niet zal wankelen.

  Hecht staat Uw troon van ouds;~\sep\ Gij zijt van eeuwigheid.

  De stromen verheffen, o Heer, de stromen verheffen hun stem,~\sep\ de stromen verheffen hun bruisen.

  Maar boven de stem der wijde wateren, boven de branding der zee~\sep\ is machtig de Heer in de hoge.

  Betrouwbaar bovenmate zijn Uw getuigenissen;~\sep\ Uw huis, Heer, past heiligheid in lengte van dagen.
\end{halfparskip}

\markedday{Vijfde Zondag, \Ps{125,1--7}.}

\begin{halfparskip}
  Toen de Heer de gevangenen van Sion deed wederkeren,~\sep\ was het of wij droomden.

  \liturgicalhint{3x Alleluia.}~--- \liturgicalhint{Herhaal het eerste vers.}

  Toen werd onze mond met lachen gevuld,~\sep\ en onze tong met gejubel.

  Toen zei men onder de volken:~\sep\ ``De Heer heeft grote dingen aan hen gedaan.''

  Ja, grote dingen heeft de Heer aan ons gedaan;~\sep\ wij zijn nu van vreugde vervuld.

  Wend ons lot ten beste, o Heer,~\sep\ als de bergstromen in het Zuiderland.

  Wie met tranen zaaien,~\sep\ zullen met gejubel maaien.
\end{halfparskip}

\markedday{Zesde Zondag, \Ps{49,1--5}.}

\begin{halfparskip}
  Aanhoort het, alle volkeren;~\sep\ luistert, alle bewoners der aarde:

  \liturgicalhint{3x Alleluia.}~--- \liturgicalhint{Herhaal het eerste vers.}

  Zowel geringen als edelen,~\sep\ rijken als armen op eenzelfde wijze.

  Mijn mond gaat wijsheid verkondigen,~\sep\ de overweging van mijn hart brengt inzicht.

  Mijn oor wil ik neigen naar een leer van wijsheid,~\sep\ bij het spel van de citer mijn raadsel onthullen.
\end{halfparskip}

\markedday{Zevende Zondag, \Ps{136,1--4}.}

\begin{halfparskip}
  Aan de stromen van Babylon, daar zaten wij en weenden,~\sep\ als wij aan Sion dachten.

  \liturgicalhint{3x Alleluia.}~--- \liturgicalhint{Herhaal het eerste vers.}

  Aan de wilgen van dat land,~\sep\ hingen wij onze citers op.

  Want die ons hadden weggevoerd, vroegen ons daarginds om liederen, en die ons verdrukten, om een jubelzang:~\sep\ ``Zingt ons uit Sions liederen!''
\end{halfparskip}

\liturgicaloption{* Alle zondagen:} \liturgicalhint{Eer.~--- 3x alleluia}

% % % % % % % % % % % % % % % % % % % % % % % % % % % % % % % % % % % % % % % %

\markedsection{Onita D'Qdam \markedsectionhint{(Alle Zondagen.)}}

\begin{halfparskip}
  \dd~Verblijd de ziel van Uw dienaar.

  \rr~Met al Uw heiligen, laat, Christus de Koning, de geest van Uw dienaren in vrede rusten, waar lijden niet regeert, noch leed, noch verdriet, maar het beloofde leven zonder einde.

  \dd~Want hoogverheven, ontzagwekkend is de Heer.

  \rr~Ons vertrouwen is in God, de Maker van onze vader Adam, de Hoop van onze dood en ons leven. De wereld is niets, en niets zijn haar genoegens; maar Hij doet ons verrijzen en geeft ons leven in Zijn goedheid.

  \cc~Eer aan de Vader...~--- \rr~Christus de Koning, onze Verlosser, doe ons opstaan op de dag van Uw komst, en laat ons staan aan Uw rechterhand, met de rechtvaardigen die U hebben behaagd, en in Uw Kruis geloofden en Het beleden, dat we met hen het eeuwige leven mogen beërven.

  \liturgicalhint{*} \cc~Wij moeten altijd Uw barmhartigheid en de zorg van Uw goede wil jegens ons, onze Heer en onze God, erkennen, aanbidden en eren, Heer van alles, Vader, Zoon en Heilige Geest in eeuwigheid.~--- \rr~Amen.
\end{halfparskip}

% % % % % % % % % % % % % % % % % % % % % % % % % % % % % % % % % % % % % % % %

\end{document}