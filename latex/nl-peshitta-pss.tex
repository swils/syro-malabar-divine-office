\documentclass[12pt,twoside,a5paper]{article}

\usepackage[main=dutch]{babel}
\usepackage{divine-office}

\setlength{\parskip}{0.6\origparskip}

% % % % % % % % % % % % % % % % % % % % % % % % % % % % % % % % % % % % % % % %

\begin{document}

% Translation based on R.A. Taylor's translation, revised by Fr. R. Matheus with the help of Dr. S. Brock.

\title{Dawidaya --- Psalterium + ondertitels}
\author{}
\date{}
\maketitle

% % % % % % % % % % % % % % % % % % % % % % % % % % % % % % % % % % % % % % % %

\hulala{1}

\Slota{Maak ons waardig, onze Heer en God, dat we mogen handelen met (\translationoptionNl{geleid worden in}) een deugdzaam gedrag dat Uw Majesteit behaagt, en dat onze wil volgens Uw wet mag zijn, en dat we er dag en nacht over mogen mediteren, Heer van alles, Vader, Zoon, en Heilige Geest in eeuwigheid. --- Amen.}

\marmita{1}

\PSALMtitle{1}{Verschillend levenslot}

\psalmsubtitle{a) De vromen ontvangen Gods zegen}

\begin{halfparskip}
  Zalig de man die de raad van goddelozen niet volgt,~\sep\ die de weg van zondaars niet inslaat, noch neerzit in de kring van spotters.

  \qanona{Heer, gezegend is hij, die Uw juk draagt, en dag en nacht Uw wet overweegt.}

  Maar die zijn vreugde vindt in de Wet van de Heer,~\sep\ en Zijn Wet overweegt bij dag en bij nacht.

  Hij is als een boom,~\sep\ geplant aan waterstromen.

  Die vrucht geeft op zijn tijd, en wiens lover niet verwelkt;~\sep\ ja, al wat hij doet, gedijt.
\end{halfparskip}

\psalmsubtitle{b) De goddelozen zullen vergaan}

\begin{halfparskip}
  Niet zo de goddelozen, niet zo;~\sep\ ze zijn als kaf, dat de wind verstrooit.

  Daarom zullen de goddelozen niet standhouden in het oordeel,~\sep\ noch de zondaars in de kring der rechtvaardigen.

  Want de Heer draagt zorg voor de weg der rechtvaardigen,~\sep\ maar de weg der goddelozen loopt uit op verderf.
\end{halfparskip}

\PSALMtitle{2}{DE Messias, opperste koning}

\psalmsubtitle{a) Tevergeefs staan de heidenen op tegen Christus}

\begin{halfparskip}
  Waarom woelen de heidenen,~\sep\ en smeden de naties ijdele plannen?

  \qanona{Als paarden zonder onderscheidingsvermogen raasden de arroganten en kruisigden de Messias.}

  De koningen der aarde rijzen samen op,~\sep\ en de vorsten spannen tezamen tegen de Heer en Zijn Gezalfde:

  ``Laten wij Hun boeien verbreken,~\sep\ en werpen wij Hun kluisters van ons af''.
\end{halfparskip}

\psalmsubtitle{b) De Heer spot met hen}

\begin{halfparskip}
  Die in de hemelen woont, Hij lacht hen uit,~\sep\ de Heer drijft de spot met hen.

  Dan spreekt Hij hen toe in Zijn toorn,~\sep\ in Zijn gramschap doet Hij hen sidderen:

  ``Maar Ik, Ik heb Mijn Koning aangesteld,~\sep\ op de Sion, Mijn heilige berg!''
\end{halfparskip}

\psalmsubtitle{c) Opperheerschappij van Christus}

\begin{halfparskip}
  Ik wil het besluit van de Heer bekendmaken: de Heer sprak tot mij:~\sep\ ``Mijn Zoon zijt Gij, Ik heb U heden voortgebracht.

  Vraag Mij, en Ik geef U de volken tot erfdeel,~\sep\ en tot Uw bezit de grenzen der aarde.

  Gij zult hen regeren met ijzeren scepter,~\sep\ hen in stukken slaan als het vat van een pottenbakker''.
\end{halfparskip}

\psalmsubtitle{d) Koningen, dient de Heer}

\begin{halfparskip}
  Nu dan, koningen, komt tot inzicht,~\sep\ laat u gezeggen, die de wereld bestuurt.

  Dient de Heer in vreze en juicht Hem toe;~\sep\ met huivering Hem uw hulde gebracht!

  Opdat Hij niet toorne en gij van de weg af vergaat als weldra Zijn toorn zal zijn ontbrandt;~\sep\ zalig allen die vluchten tot Hem.
\end{halfparskip}

\PSALMtitle{3}{God is mijn beschermer}

\psalmsubtitle{a) Talrijk zijn mijn vijanden; Gij, Heer, helpt mij}

\begin{halfparskip}
  Heer, hoe talrijk zijn zij, die mij kwellen!~\sep\ Velen staan tegen mij op!

  \qanona{Terwijl ik sprak over Uw waarheid, O Heer, stonden de goddelozen tegen mij op. Red mij van hun geweld.}

  Velen zijn er, die van mij zeggen:~\sep\ ``Voor hem is er geen redding bij God!''

  Maar Gij, Heer, zijt mijn schild,~\sep\ Gij mijn roem, die mijn hoofd opbeurt.
\end{halfparskip}

\psalmsubtitle{b) Op U, O God, vertrouw ik}

\begin{halfparskip}
  Ik riep tot de Heer met luide stem,~\sep\ en Hij verhoorde mij van Zijn heilige berg.

  Ik legde mij neer en ik sliep in;~\sep\ dan stond ik op, want de Heer is mijn steun.

  Neen, nu vrees ik de drommen van duizenden niet,~\sep\ die zich opstellen rondom mij heen.

  Verhef U, Heer!~\sep\ red mij, mijn God!

  Want al mijn weerstrevers hebt Gij op de kaken geslagen,~\sep\ de tanden der bozen hebt Gij verbrijzeld.

Bij de Heer is redding:~\sep\ Uw zegen zij over Uw volk!
\end{halfparskip}

\PSALMtitle{4}{Vertrouwvol avondgebed}

\begin{halfparskip}
  Verhoor mij, als ik U aanroep, mijn rechtvaardige God,

  \qanona{Er is niemand zoals de Heer, in wie ik heb vertrouwd. Hij redt mij uit de strikken en listen der bozen.}

  die mij in kwelling verlichting bracht;~\sep\ wees mij genadig en verhoor mijn gebed.
\end{halfparskip}

\psalmsubtitle{a) Vermaning tot bekering}

\begin{halfparskip}
  Mannen, hoelang nog blijft gij verstokt van hart;~\sep\ waarom ijdelheid bemind en leugen gezocht?

  Weet dat de Heer jegens Zijn heiligen wonderbaar handelt;~\sep\ de Heer zal mij verhoren, als ik Hem aanroep.

  Siddert en wilt niet zondigen,~\sep\ denkt na bij u zelf, op uw sponde, en zwijgt!

  Brengt gerechte offers,~\sep\ en hoopt op de Heer.
\end{halfparskip}

\psalmsubtitle{b) Gij helpt de vromen, O God}

\begin{halfparskip}
  Velen zeggen: ``Wie zal ons voorspoed doen zien?''~\sep\ Doe opgaan over ons, Heer, het licht van Uw gelaat!

  Gij hebt mij een vreugde in het hart gestort,~\sep\ groter dan bij overvloed van tarwe en wijn.

  Zodra ik mij neerleg, slaap ik in vrede,~\sep\ want Gij alleen, Heer, stelt mij in veiligheid.
\end{halfparskip}

\Slota{Hoor de woorden van onze gebeden, Heer, neig Uw oor naar de klank van ons geween; en wend U niet af van het geluid van onze smeekbeden, O Goede, in wie wij altijd ons vertrouwen stellen, Heer van alles...}

\marmita{2}

\PSALMtitle{5}{Morgenbede in Gods tempel}

\psalmsubtitle{a) Heer, schenk mij Uw hulp!}

\begin{halfparskip}
  Luister, Heer, naar mijn woorden, geef acht op mijn zuchten.~\sep\ Let op mijn bede, mijn Koning en God!

  \qanona{U hebt mij terechtgewezen, Heer, om mij wijs te maken; wijs mijn verzoek niet af.}

  Tot U toch richt ik mijn bede, Heer; in de morgen hoort Gij mijn stem,~\sep\ in de morgen leg ik mijn bede voor U neer en wacht.
\end{halfparskip}

\psalmsubtitle{b) Heer, Gij vergeldt naar werken}

\begin{halfparskip}
  Neen, geen God zijt Gij, aan wie de boosheid behaagt; geen boze mag blijven bij U,~\sep\ noch houden goddelozen voor Uw aanschijn stand.

  Gij haat allen, die ongerechtigheid plegen,~\sep\ en stort alle leugenaars in het verderf;

  De bloeddorstige en de bedrieger~\sep\ zijn een afschuw voor de Heer.

  Maar dank aan Uw vele genaden,~\sep\ zal ik binnentreden in Uw huis,

  En mij neerwerpen voor Uw heilige tempel,~\sep\ in ontzag voor U, Heer.

  Geleid mij in Uw gerechtigheid omwille van mijn vijanden;~\sep\ baan Uw weg voor mij uit.

  Want geen oprechtheid is in hun mond,~\sep\ hun hart zint op bedrog;

  Een open graf is hun keel,~\sep\ vleitaal spreekt hun tong.

  Kastijd hen, o God;~\sep\ dat zij falen in hun plannen;

  Verdrijf hen om hun vele misdaden,~\sep\ want ze zijn weerspannig tegen U.

  Mogen allen zich verblijden, die vluchten tot U,~\sep\ en juichen voor immer!

  Wil hen beschermen, en dat in U zich verblijden,~\sep\ die Uw Naam beminnen!

  Want Gij, o Heer, zult de gerechtige zegenen;~\sep\ met welwillendheid hem omgeven als met een schild.
\end{halfparskip}

\PSALMtitle{6}{Boetgebed}

\psalmsubtitle{a) O God, wees mij genadig!}

\begin{halfparskip}
  Heer, straf mij niet in Uw toorn,~\sep\ en in Uw gramschap kastijd mij niet.

  \qanona{Heb medelijden met mijn zwakheid, mijn Schepper, en kastijd mij in Uw liefde.}

  Wees mij genadig, o Heer, omdat ik zwak ben;~\sep\ genees mij, Heer, want ontwricht is mijn gebeente.

  En mijn ziel is diep geschokt;~\sep\ maar Gij, Heer, hoelang nog?

  Keer terug, Heer, en bevrijd mij;~\sep\ red mij om Uw barmhartigheid.

  Want in de dood denkt niemand aan U;~\sep\ wie looft U in het dodenrijk?

  Door mijn zuchten ben ik afgetobd, ik besproei mijn sponde iedere nacht door mijn geween,~\sep\ en drenk met mijn tranen mijn rustbed.

  Mijn ogen staan dof van verdriet,~\sep\ en flets om allen, die mij haten.
\end{halfparskip}

\psalmsubtitle{b) Laat mijn vijanden afdeinzen, O Heer!}

\begin{halfparskip}
  Gaat weg van mij, gij allen, die onrecht pleegt,~\sep\ want de Heer heeft mijn schreien gehoord.

  De Heer heeft naar mijn smeken geluisterd,~\sep\ de Heer heeft mijn bede aanvaard.

  Dat al mijn vijanden zich schamen en hevig ontstellen,~\sep\ haastig vluchten, met schande overdekt!
\end{halfparskip}

\PSALMtitle{7}{Beroep op God}

\psalmsubtitle{a) Red mij, Heer, om mijn onschuld}

\begin{halfparskip}
  Heer, mijn God, naar U vlucht ik heen;~\sep\ verlos en bevrijd mij van al mijn vervolgers,

  \qanona{Geloofd zij God, die Zijn dienaren corrigeert en troost.}

  Opdat er geen als een leeuw mij het leven ontrove,~\sep\ mij verscheure, en niemand mij redt.

  Heer, mijn God, als ik dat heb gedaan,~\sep\ als er onrecht kleeft aan mijn handen,

  Als ik mijn vriend soms kwaad heb berokkend,~\sep\ terwijl ik toch spaarde die mij onrechtmatig bestreden:

  Dan mag de vijand mij achtervolgen en grijpen, op de grond mij vertreden,~\sep\ en mijn eer vergooien in het stof.
\end{halfparskip}

\psalmsubtitle{b) Verschaf mij recht, O Heer}

\begin{halfparskip}
  Rijs op, Heer, in Uw toorn, verhef U tegen de woede van mijn verdrukkers,~\sep\ en treed voor mij op in het gericht, door U bepaald.

  De vergaderde volken mogen U omringen;~\sep\ zetel boven hen uit in de hoge.

  De Heer is de Rechter der volken: doe mij recht, O Heer, naar mijn gerechtigheid,~\sep\ en naar de onschuld van mijn hart.

  De snoodheid der bozen neme een einde, maar geef de rechtvaardige kracht,~\sep\ Gij, rechtvaardige God, die harten en nieren doorgrondt.
\end{halfparskip}

\psalmsubtitle{c) God straft de onboetvaardige}

\begin{halfparskip}
  God is mij tot schild;~\sep\ Hij redt de oprechten van hart.

  God is een rechtvaardige Rechter,~\sep\ een God, die voortdurend bedreigt.

  Bekeert men zich niet, dan scherpt Hij Zijn zwaard,~\sep\ dan spant en richt Hij Zijn boog,

  Moordende schichten bereidt Hij voor hen,~\sep\ en gloeiend maakt Hij Zijn pijlen.

  Zie, door ongerechtigheid werd hij bevrucht, hij gaat zwanger van boosheid;~\sep\ en wat hij baart, is bedrog.

  Hij groef een kuil, en diepte hem uit,~\sep\ maar viel zelf in de groeve, die hij had gedolven.

  Op eigen hoofd zal zijn boosheid wederkeren,~\sep\ en neerdalen op eigen schedel zijn wreedheid.

  Maar ik zal de Heer om Zijn gerechtigheid prijzen,~\sep\ en de Naam van de allerhoogste Heer bezingen onder citerspel.
\end{halfparskip}

\Slota{Heer, onze Heer, verborgen in Uw Wezen, die door de mond van kleintjes en kinderen Uw glorie heeft gevestigd, wij moeten U belijden, aanbidden en verheerlijken in alle seizoenen en tijden, Heer van alles...}

\marmita{3}

\PSALMtitle{8}{Hoe groot is de mens!}

\psalmsubtitle{a) Wat zijt Gij groot, O God, in de schepping!}

\begin{halfparskip}
  Heer, onze Heer, hoe wonderbaar is Uw Naam over heel de aarde;~\sep\ boven de hemelen hebt Gij Uw Majesteit verheven.

  \qanona{O Zoon, die door de kinderen in Jeruzalem werd geprezen met hun hosanna's, wij vragen U: red degenen die U aanbidden.}

  Ten spijt van Uw weerstrevers hebt Gij U lof bereid uit de mond van kind en zuigeling,~\sep\ om te beteugelen Uw vijand en hater.

  Als ik Uw hemelen zie, het werk van Uw vingeren,~\sep\ maan en sterren, die Gij hebt gegrondvest;

  Want is dan de mens, dat Gij hem gedenkt,~\sep\ of een mensenkind, dat Gij zorg voor hem draagt?
\end{halfparskip}

\psalmsubtitle{b) Wat hebt Gij de mens hoog verheven!}

\begin{halfparskip}
  Toch hebt Gij hem weinig minder dan de engelen gemaakt,~\sep\ met glorie en eer hem gekroond.

  Gij schonkt hem macht over de werken van Uw handen,~\sep\ alles hebt Gij onder zijn voeten gelegd:

  Alle schapen en runderen,~\sep\ ook de dieren in het wild,

  De vogels in de lucht en de vissen in de zee:~\sep\ al wat de paden der zeeën doorwandelt.

  Heer, onze Heer,~\sep\ hoe wonderbaar is Uw Naam over heel de aarde!
\end{halfparskip}

\PSALMtitle{9}{Gods rechtvaardig wereldbestuur}

\psalmsubtitle{a) Dank, Heer, voor de overwinning!}

\begin{halfparskip}
  Prijzen wil ik U, Heer, uit heel mijn hart,~\sep\ verhalen al Uw wonderwerken.

  \qanona{Wij danken U, want in Uw erbarmen hebt U ons gekeerd naar Uw wijsheid, laat onze tegenstanders worden beschaamd.}

  Om U wil ik juichen en jubelen,~\sep\ Uw Naam, Allerhoogste, bezingen.

  Want teruggeweken zijn mijn vijanden,~\sep\ ze zijn gevallen en kwamen om voor Uw aanschijn.

  Want Gij hebt mijn recht en mijn rechtszaak behartigd,~\sep\ Gij waart gezeten op Uw troon, als rechtvaardige Rechter.

  Gij hebt de heidenen getuchtigd, de goddeloze doen omkomen,~\sep\ hun naam uitgewist voor eeuwig.

  De vijanden kwamen om, voor immer ten onder gebracht;~\sep\ de steden hebt Gij verwoest: de gedachtenis aan hen is vergaan.

  Maar de Heer troont in eeuwigheid,~\sep\ en heeft Zijn rechterstoel gevestigd ten oordeel.

  En Hij zelf zal de wereld oordelen volgens recht,~\sep\ de volken richten volgens billijkheid.

  De Heer zal voor de verdrukte een toevlucht zijn,~\sep\ een veilige toevlucht in bange tijden.

  En die Uw Naam kennen, zullen hopen op U,~\sep\ want die U zoeken, Heer, verlaat Gij niet.
\end{halfparskip}

\psalmsubtitle{b) God, help mij in de strijd!}

\begin{halfparskip}
  Bezingt de Heer, die woont in Sion,~\sep\ verkondigt aan de volken Zijn daden.

  Want Hij, die bloedschuld wreekt, was hun indachtig,~\sep\ de kreten der armen vergeet Hij niet.

  Wees mij genadig, Heer, zie de ellende die ik lijd vanwege mijn vijanden,~\sep\ Gij die mij terugvoert van de poorten van de dood;

  Opdat ik al Uw lof verkondige bij de poorten der dochter van Sion,~\sep\ en juiche over Uw hulp.

  Bedolven zijn de volken in de kuil, die zij groeven;~\sep\ hun voet zit verward in de strik, die zij heimelijk hebben gelegd.

  De Heer heeft Zich geopenbaard, Hij heeft recht gesproken;~\sep\ in de werken van zijn handen ligt de zondaar verstrikt.

  Dat de bozen neerdalen in het dodenrijk,~\sep\ alle volken, die God zijn vergeten.

  Neen, niet voor immer zal de arme aan de vergetelheid worden prijsgegeven,~\sep\ niet voor immer de hoop der ellendigen worden beschaamd.

  Heer, sta op, laat de mens niet overmachtig worden;~\sep\ laten de heidenen voor Uw aanschijn worden gericht.

  Sla hen met ontzetting, Heer;~\sep\ dat de heidenen inzien, dat zij maar mensen zijn.
\end{halfparskip}

\PSALMtitle{10}{Bede voor hulp tegen vijanden}

\psalmsubtitle{a) De verdrukker minacht Gods gerechtigheid}

\begin{halfparskip}
  Waarom, Heer, blijft Gij veraf,~\sep\ verbergt Gij U in bange tijden.

  \qanona{Omdat de goddelozen hinderlagen hebben gelegd voor (\translationoptionNl{verraderlijk hebben gehandeld jegens}) de rechtvaardigen en Uw Naam hebben ontheiligd, vernietig, Heer, hun plannen!}

  Terwijl de goddeloze zich trots verheft, de arme verdrukt wordt~\sep\ en verstrikt ligt in de listen, die hij heeft uitgedacht.

  Ziet, de boze gaat groot op zijn driften,~\sep\ en de woekeraar lastert en minacht de Heer.

  De goddeloze spreekt in trots gemoed: ``Hij zal niet straffen;~\sep\ er bestaat geen God'': dit is heel zijn gedachtengang.

  Zijn wandel is immer voorspoedig,~\sep\ hij bekommert zich niet om Uw gerichten, al zijn weerstrevers veracht hij.

  Hij zegt bij zichzelf: ``Ik zal niet wankelen,~\sep\ van geslacht tot geslacht zal geen onheil mij treffen.''

  Zijn mond is vol verwensingen, vol list en bedrog;~\sep\ smart en kwelling kleeft aan zijn tong.

  Hij ligt bij de dorpen in hinderlaag, doodt de onschuldige in het geheim;~\sep\ zijn ogen bespieden de ongelukkige.

  Hij ligt op de loer in zijn schuilplaats als een leeuw in zijn hol; hij bespiedt de ongelukkige om hem te grijpen:~\sep\ hij sleept de rampzalige weg en trekt hem in zijn net.

  Hij bukt, werpt zich op de grond,~\sep\ en onder zijn geweld bezwijken de armen.

  Hij zegt bij zichzelf: ``God denkt er niet aan,~\sep\ Hij wendt Zijn aangezicht af, ziet nooit naar hem om.''
\end{halfparskip}

\psalmsubtitle{b) Heer, wreek de verdrukte!}

\begin{halfparskip}
  Rijs op, Heer, God, hef Uw hand omhoog;~\sep\ vergeet de armen niet!

  Waarom blijft de boze God tergen,~\sep\ en spreekt hij bij zichzelf: ``Neen, Hij zal toch niet straffen?''

  Maar Gij ziet toe: lijden en smart staan U voor ogen,~\sep\ om het in Uw handen te nemen.

  Op U verlaat zich de ongelukkige,~\sep\ Gij zijt de Helper der wezen.

  Verbrijzel de arm van zondaar en boze;~\sep\ zijn boosheid zult Gij straffen, en geen spoor blijve er van over.

  De Heer is Koning in eeuwigheid,~\sep\ in Zijn land zijn de heidenen vernietigd.

  Het verlangen der ellendigen hebt Gij gehoord, Heer;~\sep\ Gij hebt hun hart versterkt, Uw oor naar hen geneigd,

  Om het recht te beschermen van wees en verdrukte,~\sep\ zodat geen mens ter wereld hen nog vrees aanjaagt.
\end{halfparskip}

% % % % % % % % % % % % % % % % % % % % % % % % % % % % % % % % % % % % % % % %

\hulala{2}

\Slota{Bevestig, O Heer, Uw vertrouwen in ons, en vul onze ziel met Uw hulp; moge Uw genade onze zonden vergeven, en mogen de eeuwige barmhartigheid van Uw glorieuze Drie-eenheid Uw aanbidders die U aanroepen en U smeken, te hulp komen in alle seizoenen en tijden, Heer van alles...}

\marmita{4}

\PSALMtitle{11}{Wankelt niet}

\psalmsubtitle{a) Het gevaar dringt tot vluchten}

\begin{halfparskip}
  Ik vlucht tot de Heer; hoe kunt gij mij zeggen:~\sep\ ``Vlieg weg als een vogel naar het gebergte.

  \qanona{De zondaars hebben mij bedrogen; Ik heb op U vertrouwd, Heer!}

  Want zie, de bozen spannen de boog, ze zetten de pijl op de pees,~\sep\ om de oprechten van hart in het duister te treffen.

  Als zelfs de grondvesten worden gesloopt,~\sep\ wat zal de gerechtige dan nog vermogen?''
\end{halfparskip}

\psalmsubtitle{b) Gij, Heer, zijt de Beschermer van het recht}

\begin{halfparskip}
  De Heer woont in Zijn heilige tempel,~\sep\ de Heer heeft in de hemel Zijn troon.

  Zijn ogen zien rond,~\sep\ Zijn wimpers doorvorsen de kinderen der mensen.

  De Heer doorvorst de gerechte en de boze;~\sep\ die het onrecht liefheeft, is Hem een gruwel.

  Hij zal op de zondaars gloeiende kolen en zwavel doen regenen;~\sep\ een verzengende wind is de dronk van hun beker.

  Want de Heer is rechtvaardig en heeft de gerechtigheid lief;~\sep\ de goeden zullen Zijn aanschijn aanschouwen.
\end{halfparskip}

\PSALMtitle{12}{Troost in ontrouw}

\psalmsubtitle{a) De wereld is trouweloos}

\begin{halfparskip}
  Heer, schenk redding, want er zijn geen vromen meer;~\sep\ verdwenen is de trouw onder de kinderen der mensen.

  \qanona{Het bedrog is toegenomen en de liefde afgenomen; keer u niet van ons af, O Messias!}

  Allen liegen ze elkander voor,~\sep\ ze spreken met bedrieglijke lippen en vals gemoed.

  De Heer rukke al die bedrieglijke lippen uit,~\sep\ die grootsprekende tong,

  Hen, die zeggen: ``Sterk zijn wij door onze tong;~\sep\ wij hebben onze lippen met ons: wie kan ons overmeesteren?''
\end{halfparskip}

\psalmsubtitle{b) Gij, Heer, zijt getrouw}

\begin{halfparskip}
  ``Om de nood der verdrukten en het gejammer der armen zal Ik nu opstaan,'' zegt de Heer:~\sep\ ``redding zal Ik brengen aan wie er naar smacht.''

  De woorden van de Heer zijn oprechte woorden,~\sep\ zuiver zilver, van stof ontdaan, tot zevenmaal gelouterd.

  Gij, Heer, zult ons behouden,~\sep\ ons eeuwig beschermen tegen dit geslacht.

  De bozen zwermen om ons heen,~\sep\ terwijl de heffe van het volk oprijst.
\end{halfparskip}

\PSALMtitle{13}{Vertrouwvolle noodkreet}

\psalmsubtitle{a) Groot is mijn ellende}

\begin{halfparskip}
  Hoelang nog, Heer, zult Gij mij geheel vergeten,~\sep\ hoelang nog voor mij Uw aanschijn verbergen?

  \qanona{Verzoen u met mij, Heer, en red mij, zodat ik U kan prijzen.}

  Hoelang nog zal ik de smart overdenken in mijn ziel,~\sep\ en het wee in mijn hart van dag tot dag?

  Hoelang nog zal mijn vijand zich boven mij verheffen?~\sep\ Zie neer en verhoor mij, O Heer, mijn God!
\end{halfparskip}

\psalmsubtitle{b) Red mij, Heer!}

\begin{halfparskip}
  Stort licht in mijn ogen, opdat ik de doodsslaap niet inga,~\sep\ en mijn vijand niet zegge: ``Ik heb hem overwonnen'';

  En mijn weerstrevers niet juichen over mijn val,~\sep\ daar ik op Uw erbarming vertrouwde.

  Nu juiche mijn hart om Uw hulp!~\sep\ De Heer wil ik bezingen, die mij heeft welgedaan.
\end{halfparskip}

\PSALMtitle{14}{Het bederf is groot}

\psalmsubtitle{a) Boos zijn de mensen}

\begin{halfparskip}
  De dwaas zegt bij zichzelf:~\sep\ ``Er is geen God.''~\sep\ Ze zijn bedorven, gruwelen hebben ze bedreven;~\sep

  \qanona{Red Uw Kerk van de bozen, Heer van alle schepselen!}

  daar is er niet één, die deugdzaam handelt.

  De Heer blikt uit de hemel neer op de kinderen der mensen,~\sep\ om te zien of er wel één is met verstand, wel één die God zoekt.

  Allen zonder uitzondering zijn ze afgedwaald, allen diep bedorven;~\sep\ daar is er niet één, die deugdzaam handelt, niet één.
\end{halfparskip}

\psalmsubtitle{b) Gij, Heer, straft de bozen}

\begin{halfparskip}
  Zullen al die bozen dan nimmer tot inzicht komen,~\sep\ zij, die Mijn volk verslinden, als aten zij brood?

  Zij riepen de Heer niet aan. Eens zullen zij sidderen van angst,~\sep\ want God is met het geslacht der rechtvaardigen.

  Het beleid van de verdrukte wilt gij beschamen,~\sep\ maar de Heer is zijn toevlucht.

  O, mocht er uit Sion toch heil voor Israël dagen! Als de Heer het lot van Zijn volk ten goede keert,~\sep\ zal er gejubel zijn in Jacob, vreugde in Israël.
\end{halfparskip}

\Slota{Maak ons waardig, onze Heer en onze God, met een geweten gezuiverd en geheiligd door Uw waarheid, in Uw heilig tabernakel te wonen en onberispelijk Uw weg te bewandelen alle dagen van ons leven, Heer van alles...}

\marmita{5}

\PSALMtitle{15}{'s Heren gast}

\begin{halfparskip}
  Heer, wie mag in Uw tent verblijven,~\sep\ wie wonen op Uw heilige berg?

  \qanona{Laat mij staan met zuivere gedachten voor Uw heilig altaar, O Heer.}

  Die vlekkeloos wandelt en deugdzaam leeft, en in zijn hart wat goed is denkt,~\sep\ en niet lastert met zijn tong;

  Die zijn evenmens geen kwaad berokkent,~\sep\ en zijn nabuur geen smaad aandoet;

  Die de boze voor verachtelijk houdt,~\sep\ maar eert die vrezen de Heer;

  Die aan een schadelijke eed niet tornt, zijn geld niet uitleent met woeker,~\sep\ en onschuldigen ten koste geen steekpenning aanvaardt.

  Wie zo handelt,~\sep\ zal niet wankelen in eeuwigheid.
\end{halfparskip}

\PSALMtitle{16}{God is mijn hoop voor hier en hiernamaals}

\psalmsubtitle{a) Gij, God, zijt mijn enig goed!}

\begin{halfparskip}
  Bewaar mij, God, want ik vlucht tot U;~\sep\ ik zeg tot de Heer: ``Mijn Heer zijt Gij; geen geluk voor mij zonder U''.

  \qanona{Glorierijk is het vertrouwen in U, O onze Schepper, want in haar verheugt zich mijn zwakheid.}

  Hoe schonk Hij mij voor de heiligen in Zijn land,~\sep\ een wondergrote liefde!

  Zij vermeerderen hun smarten,~\sep\ die achter vreemde goden lopen.

  Ik zal niet delen in hun plengoffers van bloed,~\sep\ noch hun namen op mijn lippen nemen.

  De Heer is mijn erfdeel, de dronk van mijn beker;~\sep\ Gij zijt het, die mijn lot in handen houdt.

  Voor mij viel het meetsnoer op heerlijke velden;~\sep\ ja, mijn erfdeel behaagt mij ten volle.
\end{halfparskip}

\psalmsubtitle{b) Mijn erfdeel, O Heer, van eeuwig leven}

\begin{halfparskip}
  Ik prijs de Heer, daar Hij mij inzicht gaf,~\sep\ en zelfs 's nachts mijn hart vermaant.

  De Heer houd ik immer voor ogen;~\sep\ omdat Hij staat aan mijn rechterzijde, zal ik niet wankelen.

  Daarom verheugt zich mijn hart en jubelt mijn ziel,~\sep\ zelfs mijn vlees zal in veiligheid rusten.

  Want mijn ziel zult Gij niet in het dodenrijk laten,~\sep\ Uw heilige het bederf niet doen zien.

  Gij zult de weg naar het leven mij tonen, overvloedige vreugden bij U,~\sep\ en geneugten voor eeuwig aan Uw rechterhand.
\end{halfparskip}

\PSALMtitle{17}{Vertrouwvol gebed in uiterste nood}

\psalmsubtitle{a) Heer, wreek mijn onschuld}

\begin{halfparskip}
  Luister, Heer, naar een rechtvaardige zaak, geef acht op mijn geroep,~\sep\ hoor de bede van argeloze lippen.

  \qanona{Mijn Heer en mijn God, heb medelijden met mij; want ik word onrechtvaardig vervolgd!}

  Van Uw aanschijn ga over mij het oordeel uit:~\sep\ Uw ogen zien wat recht is.

  Peil mijn hart, doorvors het 's nachts, beproef mij met vuur,~\sep\ geen onrecht zult Gij in mij vinden.

  Mijn mond misdeed niet zoals mensen gewoon zijn;~\sep\ naar de woorden van Uw lippen heb ik de wegen der Wet gevolgd.

  Vast drukten mijn schreden Uw paden,~\sep\ mijn voeten struikelden niet.
\end{halfparskip}

\psalmsubtitle{b) God, red mij van de bozen!}

\begin{halfparskip}
  Ik roep U aan, o God, want Gij zult mij verhoren;~\sep\ neig Uw oor naar mij en luister naar mijn bede!

  Toon U wonderbaar in Uw erbarmen,~\sep\ Gij die redt van weerstrevers al wie aan Uw zijde zijn toevlucht zoekt.

  Behoed me als de appel van het oog, verberg me in de schaduw van Uw vleugels,~\sep\ voor de zondaars die me geweld aandoen.

  Mijn vijanden omringen mij woedend; zij sluiten hun zinnelijk hart,~\sep\ hun mond spreekt trotse woorden.

  Hun schreden omringen mij thans;~\sep\ zij loeren om mij ter aarde te werpen.

  Ze zijn als de leeuw, die de muil spert naar prooi,~\sep\ als een leeuwenwelp, die in hinderlaag ligt.
\end{halfparskip}

\psalmsubtitle{c) Wees, Heer, een God van vergelding}

\begin{halfparskip}
  Rijs op, Heer, hem tegemoet en vel hem terneer, red mij door Uw zwaard van de boze,~\sep\ door Uw hand van mensen, O Heer,

  Van mensen, wier deel dit leven is,~\sep\ en wier schoot Gij vult met Uw schatten;

  Wier zonen zich verzadigen,~\sep\ en wat hun overblijft aan hun kinderen achterlaten.

  Ik echter zal door gerechtigheid Uw aanschijn aanschouwen,~\sep\ en mij bij het ontwaken met Uw aanblik verzadigen.
\end{halfparskip}

\Slota{Wij moeten U belijden, aanbidden en verheerlijken, O glorieuze Kracht van Uw dienaren, sterke Hoop van Uw aanbidders, en machtige Toevlucht van hen die U vrezen, Helper die de hoorn van onze verlossing verheft, in alle seizoenen en tijden, Heer van alles...}

\marmita{6}

\PSALMtitle{18}{Davids triomfantelijk lied}

\psalmsubtitle{a) God, mijn Redder, ik bemin U}

\begin{halfparskip}
  Ik heb U lief, o Heer, mijn Sterkte,~\sep\ Heer, mijn Rots, mijn Burcht, mijn Bevrijder;

  \qanona{Hemel en aarde en alles wat in hen is, de hemelse en aardse wezens, knielen en aanbidden God, hun Schepper.}

  Mijn God, mijn Rotswand, waarheen ik vlucht,~\sep\ mijn Schild, de Hoorn van mijn heil, mijn Toeverlaat.

  Aanroepen zal ik de Heer, de Lofwaardige,~\sep\ en van mijn vijanden worden verlost.

  Mij omspoelden de golven van de dood,~\sep\ en vernietigende stromen ontstelden mij.

  De strikken van het dodenrijk omknelden mij,~\sep\ de boeien van de dood vielen op mij neer.

  In mijn nood riep ik tot de Heer,~\sep\ en mijn geschrei steeg op tot mijn God;

  En Hij hoorde mijn stem vanuit Zijn tempel,~\sep\ en mijn hulpgeroep drong door tot Zijn oren.
\end{halfparskip}

\psalmsubtitle{b) Gods ingrijpen onder het beeld van een storm}

\begin{halfparskip}
  Daar schudde de aarde en beefde; de grondvesten der bergen werden geschokt,~\sep\ en zij dreunden want Hij brandde van toorn.

  Rook steeg uit Zijn neusgaten op, verslindend vuur uit Zijn mond,~\sep\ gloeiende kolen sprongen van Hem uit.

  Hij haalde de wolkenhemel neer en daalde af,~\sep\ en zwarte wolken hingen onder Zijn voeten.

  Hij voer op de cherub en vloog,~\sep\ op de wieken van de wind werd Hij gedragen.

  Hij omhulde zich met duisternis als met een kleed,~\sep\ met donkere nevels en dichte wolken als met een mantel.

  Door de gloed vóór Hem uit,~\sep\ ontbrandden gloeiende kolen.

  En de Heer deed de donder rollen uit de hemel,~\sep\ en weergalmen deed de Allerhoogste Zijn stem.

  En Hij schoot Zijn pijlen af en dreef hen uiteen,~\sep\ talloze flitsen, en Hij velde hen neer.

  En de bodem der zee kwam te voorschijn,~\sep\ en het fundament der aarde lag bloot.

  Door het dreigen van de Heer,~\sep\ door de ademtocht van Zijn toorn.
\end{halfparskip}

\psalmsubtitle{c) Redding uit diepste nood}

\begin{halfparskip}
  Hij strekte Zijn hand uit de hoge, Hij greep mij aan,~\sep\ en trok mij op uit de watervloed.

  Hij bevrijdde mij van mijn geweldige vijand,~\sep\ en van hen, die mij haatten, die machtiger waren dan ik.

  Zij overvielen mij op de dag van mijn rampspoed,~\sep\ maar tot bescherming was mij de Heer.

  En Hij leidde mij uit in het vrije veld;~\sep\ Hij heeft mij gered, omdat Hij mij liefheeft.
\end{halfparskip}

\psalmsubtitle{d) Mijn God, ik was U trouw}

\begin{halfparskip}
  Zo loonde mij de Heer naar mijn gerechtigheid;~\sep\ naar de reinheid van mijn handen vergold Hij mij.

  Want de wegen van de Heer heb ik gevolgd,~\sep\ door geen zonde ben ik afgeweken van mijn God.

  Ja, al Zijn geboden hield ik voor ogen,~\sep\ en Zijn wetten wierp ik niet van mij af.

  Maar voor Zijn aanschijn was ik rein,~\sep\ en ik heb mij behoed voor de zonde.

  Zo vergold mij de Heer naar mijn gerechtigheid,~\sep\ naar de reinheid van mijn handen voor Zijn ogen.
\end{halfparskip}

\psalmsubtitle{e) Gij, God, vergeldt naar werken}

\begin{halfparskip}
  Met de vrome handelt Gij liefdevol,~\sep\ met de rechtschapene rechtschapen;

  Voor de reine toont Gij U rein,~\sep\ met de sluwe handelt Gij slim.

  Want Gij redt het nederige volk,~\sep\ maar trotse blikken slaat Gij neer.

  Ja, Gij doet mijn lamp schijnen, O Heer;~\sep\ mijn God, mijn duisternis maakt Gij tot licht.

  Ja, met U storm ik los op de drommen der vijanden,~\sep\ en met mijn God bespring ik de wallen.

  Gods wegen zijn volmaakt, het woord van de Heer is door het vuur gelouterd;~\sep\ Hij is een schild voor allen, die vluchten tot Hem.
\end{halfparskip}

\psalmsubtitle{f) Gij, God, hebt mij vaardig gemaakt}

\begin{halfparskip}
  Wie is God buiten de Heer,~\sep\ of wie een rots buiten onze God?

  God, die mij met kracht heeft omgord,~\sep\ en mij een veilige weg heeft gebaand;

  Die aan mijn voeten de snelheid der hinden gaf,~\sep\ en mij plaatste op de hoogten,

  Die mijn handen oefende tot de strijd,~\sep\ en tot het spannen van de koperen boog mijn armen.
\end{halfparskip}

\psalmsubtitle{g) Van U, Heer, kwam de zege}

\begin{halfparskip}
  Gij schonkt mij Uw schild, dat redding brengt, en Uw rechterhand heeft mij staande gehouden,~\sep\ en Uw zorgzame liefde maakte mij groot.

  Gij hebt de weg voor mijn schreden verbreed,~\sep\ en mijn voeten wankelden niet.

  Ik zette mijn vijanden na, en greep ze aan,~\sep\ en ik keerde niet terug, eer ik ze had vernietigd.

  Ik heb ze verpletterd en opstaan konden ze niet,~\sep\ ze bleven liggen onder mijn voeten.

  Ja, Gij hebt mij met kracht omgord tot de strijd;~\sep\ en die mij weerstaan, hebt Gij voor mij doen bukken.

  Gij hebt mijn vijanden op de vlucht gedreven,~\sep\ en die mij haten, hebt Gij verdelgd.

  Zij schreeuwden het uit - maar niemand schonk redding -~\sep\ tot de Heer, maar Hij verhoorde hen niet.

  En ik heb ze vergruisd als stof voor de wind,~\sep\ vertrapt als slijk in de straten.

  Gij hebt mij ontrukt aan het muitende volk,~\sep\ mij gesteld aan het hoofd van de naties.

  Een volk, dat mij vreemd was, werd mij dienstbaar;~\sep\ nauwelijks hoorde het van mij, of het was mij onderdanig.

  Vreemden brachten mij vleiend hulde,~\sep\ vreemden, geslagen met schrik, kropen sidderend uit hun burchten.
\end{halfparskip}

\psalmsubtitle{h) U, Heer, zij lof!}

\begin{halfparskip}
  Leve de Heer, mijn Rots zij gezegend;~\sep\ hooggeprezen zij God, mijn Redder!

  God, die mij de wraak in handen gaf,~\sep\ en mij de volkeren onderwierp,

  Gij, die mij van mijn vijanden hebt bevrijd, en mij verheven hebt boven mijn weerstrevers,~\sep\ mij hebt ontrukt aan de geweldenaar.

  Daarom zal ik U prijzen onder de volken, O Heer,~\sep\ en verheerlijken Uw Naam.

  Gij hebt Uw koning een schitterende zege verleend,~\sep\ en barmhartigheid bewezen aan Uw gezalfde, aan David en zijn geslacht voor eeuwig.
\end{halfparskip}

\Slota{Wij moeten U belijden, aanbidden en verheerlijken, U die voor allen in Uw Wezen verborgen zijt maar Uzelf hebt geopenbaard in de wonderbaarlijke daden van Uw heilsbestel. Hemel en aarde verkondigen Uw kracht in alle seizoenen en tijden, Heer van alles...}

\marmita{7}

\PSALMtitle{19}{Gods wet, de zon van onze ziel}

\psalmsubtitle{a) Heerlijkheid der natuur}

\begin{halfparskip}
  De hemelen verhalen de glorie van God,~\sep\ en het uitspansel roemt het werk van Zijn handen.

  \qanona{Aanbiddenswaardig is de eeuwige God, die de rationele wezens geschapen heeft om Zijn andere werken te begrijpen en Hem te prijzen!}

  De dag galmt het uit aan de dag,~\sep\ en de nacht geeft het door aan de nacht.

  Dat is geen taal, dat zijn geen woorden,~\sep\ waarvan de klank niet wordt vernomen.

  Over heel de wereld golft hun sein,~\sep\ en tot de grenzen der aarde hun uitspraak.

  Daar sloeg Hij Zijn tent op voor de zon, die als een bruidegom uit zijn bruidskamer treedt,~\sep\ en als een juichende reus zijn baan doorloopt.

  Aan het einde van de hemel is zijn opgang, en zijn kringloop reikt tot het einde van de hemel,~\sep\ aan zijn gloed kan niets zich onttrekken.
\end{halfparskip}

\psalmsubtitle{b) Voortreffelijk, Heer, is Uw Wet}

\begin{halfparskip}
  De Wet van de Heer is volmaakt: zij schenkt aan de ziel nieuw leven;~\sep\ het gebod van de Heer staat vast, het onderricht de eenvoudige.

  De voorschriften van de Heer zijn rechtmatig: een vreugde voor het hart;~\sep\ het bevel van de Heer is rein: een licht voor de ogen.

  De vrees van de Heer is zuiver: zij blijft eeuwig bestaan;~\sep\ de oordelen van de Heer zijn waarachtig: alle even rechtvaardig,

  Te verkiezen boven goud en schatten van het edelst metaal,~\sep\ en zoeter dan honing en druipend honingzeem.

  Al wijdt er Uw dienaar zijn aandacht aan,~\sep\ en onderhoudt hij ze vol ijver,

  Wie kent er zijn fouten?~\sep\ Reinig mij van die mij verborgen zijn!

  Behoed ook Uw dienaar voor hoogmoed;~\sep\ dat hij mij niet beheerse!

  Dan zal ik zuiver zijn,~\sep\ en rein van zware misdaad.

  Mogen de woorden van mijn mond en de overweging van mijn hart welgevallig zijn,~\sep\ voor Uw aanschijn, O Heer, mijn Rots en mijn Redder!
\end{halfparskip}

\PSALMtitle{20}{Gebed voor de koning, die optrekt ten strijd}

\psalmsubtitle{a) Heer, help hem in de strijd!}

\begin{halfparskip}
  Dat de Heer u verhore in tijden van nood,~\sep\ u bescherme de Naam van Jacobs God!

  \qanona{Laten we ons vertrouwen in God plaatsen, want Hij redt de nederigen.}

  Hij zende u hulp uit het heiligdom,~\sep\ en uit Sion steune Hij u.

  Hij gedenke al uw offergaven,~\sep\ en uw brandoffer behage Hem.

  Hij schenke u wat uw hart begeert,~\sep\ en doe al uw plannen slagen.

  Mogen we over uw zegepraal juichen, in de Naam van onze God de banieren verheffen;~\sep\ dat de Heer al uw beden vervulle!
\end{halfparskip}

\psalmsubtitle{b) Gij, God, schonkt reeds verhoring}

\begin{halfparskip}
  Reeds weet ik dat de Heer Zijn gezalfde de zegepraal schonk,~\sep\ hem heeft verhoord vanuit Zijn heilige hemel door de kracht van Zijn verwinnende rechterhand.

  Dat anderen vertrouwen op strijdwagens, anderen op rossen,~\sep\ wij echter zijn sterk door de Naam van de Heer, onze God.

  Zij zijn gevallen en liggen terneer,~\sep\ maar wij houden onwankelbaar stand.

  O Heer, schenk de koning de zege,~\sep\ en verhoor ons op de dag van ons smeken.
\end{halfparskip}

\PSALMtitle{21}{Dank voor de overwinning}

\psalmsubtitle{a) Dank, Heer, voor de zege van de koning!}

\begin{halfparskip}
  Over Uw macht, O Heer, verheugt zich de koning,~\sep\ hoe uitbundig jubelt hij over Uw hulp!

  \qanona{De Heer neemt het leed weg van Zijn dienaren, en verblijdt hen door Zijn macht.}

  Zijn hartewens hebt Gij verhoord,~\sep\ en de bede van zijn lippen niet afgewezen.

  Ja, Gij hebt hem voorkomen met rijke zegen,~\sep\ een kroon van zuiver goud hem op het hoofd gedrukt.

  Leven vroeg hij U; Gij hebt hem gegeven,~\sep\ lengte van dagen voor immer.

  Groot is zijn roem, dank aan Uw hulp;~\sep\ met majesteit en luister hebt Gij hem getooid,

  Ja, Gij hebt hem voor eeuwig overladen met zegening,~\sep\ hem met vreugde overstelpt voor Uw aanschijn.

  Want de koning vertrouwt op de Heer,~\sep\ en door de gunst van de Allerhoogste zal hij niet wankelen.
\end{halfparskip}

\psalmsubtitle{b) Sla de vijand neer, O God!}

\begin{halfparskip}
  Moge Uw hand al Uw vijanden treffen,~\sep\ Uw rechter aangrijpen al die U haten.

  Maak ze als tot een gloeiende oven,~\sep\ wanneer Gij Uw aanschijn zult tonen.

  Dat de Heer hen in Zijn toorn vertere,~\sep\ en het vuur hen verslinde.

  Verdelg hun kroost op aarde,~\sep\ en hun zaad onder de kinderen der mensen.

  Als zij U kwaad willen doen, listige plannen beramen,~\sep\ zullen zij niets vermogen;

  Want Gij zult ze doen vluchten,~\sep\ Uw boog op hun aangezicht richten.

  Rijs op, o Heer, in Uw kracht!~\sep\ Wij zullen Uw macht bezingen en prijzen.
\end{halfparskip}

% % % % % % % % % % % % % % % % % % % % % % % % % % % % % % % % % % % % % % % %

\hulala{3}

\Slota{We moeten Uw glorieuze Godheid, vol barmhartigheid, mededogen, hoop, leven en redding voor alle schepselen, belijden, aanbidden en verheerlijken in alle seizoenen en tijden, Heer van alles...}

\marmita{8}

\PSALMtitle{22}{Van God verlaten}

\psalmsubtitle{a) In diepe verlatenheid}

\begin{halfparskip}
  Mijn God, mijn God, waarom hebt Gij mij verlaten?~\sep\ Ver houdt Gij U af van mijn bede, van mijn noodgeschrei.

  \qanona{Mijn God, mijn God, verwerp mij niet samen met de mensen, die U niet kennen.}

  Bij dag roep ik U aan, mijn God, en Gij verhoort mij niet;~\sep\ bij nacht, en Gij slaat geen acht op mij.

  Toch woont Gij in het heiligdom,~\sep\ Gij, de roem van Israël.

  Onze vaderen hoopten op U,~\sep\ zij hoopten op U, en Gij hebt hen bevrijd;

  Zij riepen U aan, en werden gered;~\sep\ zij hoopten op U, en zijn niet beschaamd.
\end{halfparskip}

\psalmsubtitle{b) Veracht en bespot}

\begin{halfparskip}
  Maar ik ben een worm en geen mens,~\sep\ de smaad der mensen en de verachting van het volk.

  Allen, die mij zien, spotten met mij,~\sep\ vertrekken de lippen en schudden het hoofd:

  ``Hij vertrouwt op de Heer; laat Die hem bevrijden,~\sep\ laat Die hem verlossen, zo Hij hem bemint.''

  Ja, Gij hebt mij geleid van de moederschoot af,~\sep\ mij veilig gelegd aan de borst van mijn moeder.

  U werd ik toevertrouwd vanaf mijn geboorte,~\sep\ vanaf de schoot van mijn moeder zijt Gij mijn God.

  Blijf toch niet ver van mij, want ik word gekweld;~\sep\ wees mij nabij, want er is geen helper.
\end{halfparskip}

\psalmsubtitle{c) In uiterste doodsnood}

\begin{halfparskip}
  Jonge stieren stuwen in menigte om mij heen,~\sep\ stieren van Basan omsingelen mij.

  Zij sperren hun muil tegen mij open,~\sep\ als een roofzuchtige en brullende leeuw.

  Als water ben ik uitgestort,~\sep\ en al mijn beenderen zijn ontwricht.

  Mijn hart is geworden als was,~\sep\ het smelt in mijn binnenste weg.

  Mijn keel is droog als een potscherf, en mijn tong kleeft vast aan mijn gehemelte;~\sep\ Gij hebt mij gebracht tot het stof van de dood.

  Want vele honden staan om mij heen,~\sep\ een bende boosdoeners houdt mij omsingeld.

  Zij hebben mijn handen en voeten doorboord,~\sep\ ik kan al mijn beenderen tellen.

  Zij slaan mij gade, en bij die aanblik verheugen zij zich; zij verdelen mijn klederen onder elkander,~\sep\ en werpen het lot over mijn gewaad.
\end{halfparskip}

\psalmsubtitle{d) Red mij van de dood!}

\begin{halfparskip}
  Gij nu, Heer, blijf niet van verre staan;~\sep\ mijn Bijstand, snel mij te hulp!

  Ontruk mijn ziel aan het zwaard,~\sep\ aan de greep van de hond mijn leven.

  Red mij uit de muil van de leeuw,~\sep\ mij, ongelukkige, van de hoornen der buffels.
\end{halfparskip}

\psalmsubtitle{e) Eeuwige dank aan God}

\begin{halfparskip}
  Ik zal mijn broeders Uw Naam verkondigen,~\sep\ in volle vergadering U prijzen:

  ``Looft de Heer, gij, die Hem vreest, heel Jacobs geslacht, verheerlijk Hem:~\sep\ vreest Hem, alle kinderen van Israël!

  Want Hij heeft niet versmaad, noch geminacht de ellende van de verdrukte, en Hij hield zijn aanschijn voor hem niet verborgen;~\sep\ Hij heeft hem aanhoord, toen hij riep tot Hem.''

  Van U komt mijn lof in de volle vergadering;~\sep\ ten aanschouwen van Zijn vereerders zal ik mijn geloften volbrengen.

  De armen zullen eten en zich verzadigen; die de Heer zoeken, zullen Hem loven:~\sep\ ``dat uw harten leven in eeuwigheid!''
\end{halfparskip}

\psalmsubtitle{f) Alle volken zullen Hem eren}

\begin{halfparskip}
  Dit indachtig, zullen tot de Heer zich bekeren,~\sep\ alle grenzen der aarde;

  En voor Zijn aanschijn zullen neervallen,~\sep\ alle stammen der heidenen,

  Want aan de Heer behoort het koningschap,~\sep\ Hij is het, die over de volkeren heerst;

  Hem alleen zullen allen aanbidden die onder de aarde rusten,~\sep\ voor Hem zullen allen zich buigen, die neerdalen in het stof.

  En mijn ziel zal leven voor Hem,~\sep\ mijn nageslacht Hem dienen;

  Het zal van de Heer verhalen aan het geslacht, dat komen zal,~\sep\ en Zijn gerechtigheid zal men vermelden aan het volk, dat wordt geboren: ``Dit heeft de Heer gedaan.''
\end{halfparskip}

\PSALMtitle{23}{God is mijn herder}

\psalmsubtitle{a) Gij zijt een goede Herder, O Heer}

\begin{halfparskip}
  De Heer is mijn Herder: het ontbreekt mij aan niets;~\sep\ in groenende beemden laat Hij mij sluimeren.

  \qanona{Laten we onze zorgen toevertrouwen aan de Heer, de Zorgdrager voor Zijn gezin.}

  Hij voert mij naar wateren, waar ik kan rusten;~\sep\ Hij verkwikt mijn ziel.

  Hij leidt mij langs rechte wegen,~\sep\ omwille van Zijn Naam.

  Al schrijd ik dan voort in een donker dal,~\sep\ geen kwaad zal ik vrezen, omdat Gij met mij zijt.

  Uw roede en Uw herdersstaf,~\sep\ zijn mij tot troost.
\end{halfparskip}

\psalmsubtitle{b) Mijn God, Gij zijt een milde Gastheer}

\begin{halfparskip}
  Gij richt voor mij een maaltijd aan,~\sep\ ten aanschouwen van mijn weerstrevers.

  Met olie zalft Gij mijn hoofd;~\sep\ mijn beker is overvol.

  Goedertierenheid en genade zullen mij volgen,~\sep\ al de dagen van mijn leven;

  En wonen zal ik in het huis van de Heer,~\sep\ tot in de verre toekomst.
\end{halfparskip}

\PSALMtitle{24}{Feestlied}

\psalmsubtitle{a) Uw huis, O Heer, vraagt heiligheid}

\begin{halfparskip}
  De Heer behoort de aarde met al wat zij bevat,~\sep\ de wereld en die er op wonen.

  \qanona{Laat ons zorgvuldig zijn in onze plicht, want de Almachtige heeft ons gered.}

  Want Hij heeft haar op de zeeën gegrondvest,~\sep\ en legde haar vast op de stromen.

  Wie mag de berg van de Heer bestijgen,~\sep\ of wie verwijlen in Zijn heilige plaats?

  Die rein is van handen en zuiver van hart, zijn geest niet richt op ijdele dingen,~\sep\ en tegen zijn naaste geen meineed zweert.

  Die zal zegen ontvangen van de Heer,~\sep\ en loon van God, zijn Redder.

  Dit is het geslacht van die naar Hem zoeken,~\sep\ van die zoeken het aanschijn van Jacobs God.
\end{halfparskip}

\psalmsubtitle{b) De intrede van de Heer in Zijn heiligdom}

\begin{halfparskip}
  Poorten, uw bogen omhoog, omhoog, gij, aloude poorten,~\sep\ opdat de Koning der glorie Zijn intrede doe!

  ``Wie is die Koning der glorie?''~\sep\ ``De Heer, de Sterke en de Machtige, de Heer, de Held in de strijd.''

  Poorten, uw bogen omhoog, omhoog, gij, aloude poorten,~\sep\ opdat de Koning der glorie Zijn intrede doe!

  ``Wie is die Koning der glorie?''~\sep\ De Heer der heerscharen, Hij is de Koning der glorie.''
\end{halfparskip}

\Slota{Naar U, onze Heer en onze God, zijn de ogen van onze zielen opgeheven; in U is onze hoop en ons vertrouwen; en van U vragen wij vergeving voor onze overtredingen. Schenk ons dit altijd in Uw liefderijke goedheid en barmhartigheid, zoals U gewend bent, Heer van alles, Vader...}

\marmita{9}

\PSALMtitle{25}{Gebed om bescherming en vergeving}

\psalmsubtitle{a) Vergeef mij, Heer, mijn zonden!}

\begin{halfparskip}
  Tot U verhef ik mijn ziel, O Heer, mijn God.~\sep\ Op U vertrouw ik; laat mij niet te schande worden;~\sep

  \qanona{Mijn Heer, ik hef mijn ogen op tot U, want Gij zijt mijn ware Hoop!}

  dat mijn vijanden niet over mij juichen!

  Want van wie op U hopen, wordt niemand beschaamd,~\sep\ maar wel worden te schande, die hun woord vermetel breken.

  Toon mij Uw wegen, O Heer,~\sep\ en leer mij Uw paden kennen.

  Leid mij in Uw waarheid en geef mij onderricht, omdat Gij, God, mijn Redder zijt,~\sep\ en immer hoop ik op U.

  Gedenk Uw ontferming, o Heer,~\sep\ en Uw barmhartigheid, die van oudsher zijn.

  De zonden van mijn jeugd en mijn misslagen, gedenk ze niet; wees mij naar Uw erbarming indachtig,~\sep\ vanwege Uw goedheid, Heer.
\end{halfparskip}

\psalmsubtitle{b) God wijst de nederige de rechte weg}

\begin{halfparskip}
  Goed en rechtvaardig is de Heer;~\sep\ daarom wijst Hij de zondaars de weg.

  De nederigen leidt Hij in gerechtigheid,~\sep\ de nederigen toont Hij Zijn weg.

  Alle wegen van de Heer zijn goedheid en trouw,~\sep\ voor wie Zijn Verbond en Zijn wetten bewaren.

  Omwille van Uw Naam, Heer,~\sep\ vergeef mij mijn zonde, want zij is groot.

  Wie is de man, die de Heer vreest?~\sep\ Hij wijst hem de weg, die hij moet kiezen.

  Hij zelf zal in voorspoed leven,~\sep\ en zijn geslacht het land bezitten.

  De Heer is een vriend voor hen, die Hem vrezen:~\sep\ Zijn Verbond doet Hij hun kennen.

  Mijn ogen zijn immer gericht op de Heer,~\sep\ want uit de strik zal Hij mijn voeten bevrijden.
\end{halfparskip}

\psalmsubtitle{c) Bevrijd mij, Heer, van kwellingen!}

\begin{halfparskip}
  Zie op mij neer en wees mij genadig,~\sep\ want eenzaam ben ik en ellendig.

  Verlicht de druk van mijn hart,~\sep\ en bevrijd mij van mijn angsten.

  Zie mijn ellende en mijn kwelling;~\sep\ en vergeef mij al mijn zonden.

  Let op mijn vijanden, want ze zijn talrijk,~\sep\ en haten mij met felle haat.

  Bescherm mijn leven en red mij;~\sep\ het zij mij niet tot schande, dat ik bij U mijn toevlucht zocht.

  Dat mijn onschuld en deugd mij beschermen,~\sep\ daar ik hoop op U, O Heer.

  Verlos Israël, o God,~\sep\ uit al zijn kommernissen!
\end{halfparskip}

\PSALMtitle{26}{Vertrouwen van een goed geweten}

\psalmsubtitle{a) Schaf mij recht, O God, om mijn onschuld!}

\begin{halfparskip}
  Heer, schaf mij recht, want ik leefde in onschuld;~\sep\ vertrouwend op de Heer, heb ik niet gewankeld.

  \qanona{O Rechter, hoogste der rechters, druk mijn hoofd niet terneer in Uw oordeel!}

  Onderzoek mij, Heer, en stel mij op de proef;~\sep\ doorgrond mijn nieren en mijn hart.

  Want Uw welwillendheid staat mij voor ogen,~\sep\ en ik wandel naar Uw waarheid.

  Met ongerechtigen zit ik niet neer,~\sep\ en met bedriegers kom ik niet samen.

  Ik haat het gezelschap van bozen,~\sep\ en met goddelozen zit ik niet samen.

  In onschuld was ik mijn handen,~\sep\ en ga rond Uw altaar, O Heer.

  Om openlijk Uw lof te verkondigen,~\sep\ en al Uw wonderen te verhalen.

  Heer, ik heb lief het verblijf van Uw huis,~\sep\ en de woontent van Uw heerlijkheid.
\end{halfparskip}

\psalmsubtitle{b) Verderf mij niet, Heer, met de bozen!}

\begin{halfparskip}
  Ruk mijn ziel niet weg met de zondaars,~\sep\ noch mijn leven met bloeddorstige mannen,

  Aan wier handen de misdaad kleeft,~\sep\ en wier rechterhand met geschenken gevuld is.

  Ik echter wandel in onschuld:~\sep\ red mij, en wees mij genadig.

  Mijn voet staat op effen baan;~\sep\ ik zal de Heer in de vergadering loven.
\end{halfparskip}

\PSALMtitle{27}{Vertrouwen op God}

\psalmsubtitle{a) Met God vrees ik geen mens}

\begin{halfparskip}
  De Heer is mijn Licht en mijn Heil: wie zou Ik vrezen?~\sep\ De Heer is de Schuts van mijn leven: voor wie zou ik sidderen?

  \qanona{Verjaag mij niet van voor Uw aangezicht, Gij die de geheime dingen doorgrondt!}

  Als de bozen mij bestormen om mijn vlees te verslinden,~\sep\ mijn vijanden en haters, zij struikelen en vallen.

  Al stond er een krijgsmacht tegenover mij, mijn hart zou niet vrezen;~\sep\ al brak er een oorlog tegen mij uit, dan nog zou ik vertrouwen.
\end{halfparskip}

\psalmsubtitle{b) Maar één hartewens: het huis van de Heer}

\begin{halfparskip}
  Dit alleen vraag ik de Heer, dit alleen streef ik na:~\sep\ te wonen in het huis van de Heer alle dagen van mijn leven,

  Te genieten de zoetheid van de Heer,~\sep\ en Zijn tempel te aanschouwen.

  Want in Zijn woontent zal Hij mij bergen in tijden van nood,~\sep\ Hij zal mij doen schuilen diep in Zijn tent, mij plaatsen boven op de rots.

  Nu verheft zich mijn hoofd,~\sep\ boven de vijanden, die mij omringen;

  Jubeloffers zal ik brengen in Zijn tent,~\sep\ zingen voor de Heer en spelen op de citer.
\end{halfparskip}

\psalmsubtitle{c) Uw aanschijn zoek ik, O Heer}

\begin{halfparskip}
  Heer, luister naar mijn stem, waarmee ik luid roep;~\sep\ ontferm U over mij, en schenk mij verhoring!

  Tot U spreekt mijn hart, U zoeken mijn ogen;~\sep\ ik zoek Uw aanschijn, o Heer.

  Verberg mij Uw aanschijn niet,~\sep\ stoot Uw dienaar niet af in Uw toorn!

  Gij zijt mijn hulp; verwerp mij niet!~\sep\ Verlaat mij niet, O God, mijn Redder!

  Zou mijn vader en moeder mij ook verlaten,~\sep\ dan nog neemt de Heer mij op.

  Wijs mij Uw weg, o Heer,~\sep\ en leid mij op effen baan omwille van mijn weerstrevers.

  Geef mij niet prijs aan de moedwil van mijn vijanden,~\sep\ want valse getuigen en geweldenaars stonden tegen mij op.

  Ik ben er zeker van de weldaden van de Heer te zien,~\sep\ in het land der levenden.

  Zie uit naar de Heer, wees onversaagd;~\sep\ sterk zij Uw hart, zie uit naar de Heer.
\end{halfparskip}

\Slota{Tot U, Heer, roepen wij; bij U zoeken wij onze toevlucht, en aan U vragen wij vergeving van onze overtredingen en zonden; schenk ons dit in Uw genade en barmhartigheid zoals U gewend bent, te allen tijde, Heer van alles...}

\marmita{10}

\PSALMtitle{28}{Smeek- en dankgebed}

\psalmsubtitle{a) Heer, verwerp mij niet met de bozen!}

\begin{halfparskip}
  Tot U roep ik, O Heer;~\sep\ mijn Rots, wees niet doof voor mij.

  Opdat ik niet, als Gij niet hoort naar mij,~\sep\ gelijk worde aan hen, die in de grafkuil dalen.

  \qanona{Onze zielen roepen tot U: kom ons te hulp en red ons!}

  Hoor de stem van mijn smeken, nu ik roep tot U,~\sep\ nu ik mijn handen ophef naar Uw heilige tempel.

  Ruk mij niet weg met de zondaars,~\sep\ met hen, die kwaad bedrijven,

  Die vriendelijk spreken met hun naaste,~\sep\ maar in hun hart kwade bedoelingen koesteren.

  Handel met hen naar hun daden,~\sep\ en naar de boosheid van hun werken.

  Zet hun het werk van hun handen betaald,~\sep\ vergeld ze hun daden.

  Want ze slaan geen acht op de daden van de Heer en het werk van Zijn handen;~\sep\ Hij richte hen te gronde en heffe hen niet op.
\end{halfparskip}

\psalmsubtitle{b) Ja, Hij heeft mij verhoord}

\begin{halfparskip}
  Gezegend de Heer, want Hij hoorde mijn dringende bede;~\sep\ de Heer, mijn kracht en mijn schild,

  Op Hem vertrouwde mijn hart, en ik ben geholpen;~\sep\ daarom jubelt mijn hart en prijs ik Hem met mijn zang.

  De Heer is een kracht voor Zijn volk,~\sep\ en voor Zijn Gezalfde een heilzame schutse.

  Red Uw volk en zegen Uw erfdeel;~\sep\ weid hen en draag hen voor eeuwig.
\end{halfparskip}

\PSALMtitle{29}{Gods majesteit in het onweer}

\begin{halfparskip}
  Kent toe aan de Heer, zonen van God,~\sep\ kent toe aan de Heer glorie en macht!

  \qanona{Gij, Goede, Barmhartige, lof komt U toe.}

  Kent toe aan de Heer de roem van Zijn Naam,~\sep\ aanbidt de Heer in heilige feesttooi.
\end{halfparskip}

\psalmsubtitle{a) Gods Majesteit in het onweer}

\begin{halfparskip}
  De stem van de Heer over de wateren! De God van Majesteit doet de donder rollen:~\sep\ de Heer over de wijde wateren!

  De stem van de Heer vol kracht,~\sep\ de stem van de Heer vol majesteit!

  De stem van de Heer verbrijzelt de ceders,~\sep\ de Heer verbrijzelt de ceders van de Libanon.

  Hij doet de Libanon opspringen als een kalf,~\sep\ en de Sarion als het jong van een buffel.

  De stem van de Heer schiet vlammende schichten, de stem van de Heer doet de wildernis beven,~\sep\ de Heer doet Cades' wildernis beven.

  De stem van de Heer buigt eiken krom en ontschorst de bomen der wouden:~\sep\ en in Zijn tempel roepen allen: Glorie!

  De Heer troonde boven de watervloed,~\sep\ en de Heer zal tronen als Koning voor eeuwig.

  De Heer zal sterkte schenken aan Zijn volk,~\sep\ de Heer zal Zijn volk met vrede zegenen.
\end{halfparskip}

\PSALMtitle{30}{Dank na herstel van ziekte}

\psalmsubtitle{a) Dank, Heer, voor mijn genezing!}

\begin{halfparskip}
  Ik wil U roemen, o Heer, daar Gij mij gered hebt,~\sep\ en niet mijn vijanden over mij liet juichen.

  \qanona{Wij zullen Uw Naam prijzen, want U hebt ons gered; en door Uw kracht hebt U onze vijanden verpletterd.}

  Heer, mijn God,~\sep\ ik riep tot U, en Gij hebt mij genezen.

  Heer, uit het dodenrijk hebt Gij mij weggevoerd,~\sep\ mij gered uit hen, die ten grave dalen.

  Speelt op de citer voor de Heer, gij, Zijn heiligen;~\sep\ en dankt Zijn heilige Naam.

  Want Zijn toorn duurt slechts een ogenblik,~\sep\ maar Zijn welwillendheid het hele leven door.

  's Avonds komt er geween te gast,~\sep\ maar 's morgens is er gejubel.
\end{halfparskip}

\psalmsubtitle{b) Gij, Heer, waart mij genadig}

\begin{halfparskip}
  In overmoed nu heb ik gezegd:~\sep\ ``In eeuwigheid zal ik niet wankelen.''

  Het was Uw gunst, o Heer, die mij ere schonk en macht;~\sep\ maar toen Gij Uw aanschijn verborgen hieldt, werd ik ontsteld.

  Ik roep tot U, O Heer,~\sep\ en smeek bij mijn God om erbarming:

  ``Wat kan mijn bloed U baten,~\sep\ of mijn neerdalen in het graf?

  Zal het stof U soms prijzen,~\sep\ of roemen Uw trouw?''

  Luister, O Heer, en wees mij genadig;~\sep\ O Heer, wees toch mijn Helper !

  Gij hebt mijn rouw in een reidans veranderd,~\sep\ mijn rouwkleed verscheurd, mij met vreugde omgord,

  Opdat mijn ziel U zou prijzen en nimmermeer zwijgen.~\sep\ Heer, mijn God, ik zal U loven voor eeuwig!
\end{halfparskip}

% % % % % % % % % % % % % % % % % % % % % % % % % % % % % % % % % % % % % % % %

\hulala{4}

\Slota{Wij smeken U, die onze Hoop en ons Vertrouwen bent, onze Helper, onze Zorgverlener en de grote Toevlucht voor onze zwakheid; keer u naar ons, O Heer, heb medelijden met ons, en wees ons altijd genadig, zoals U gewend bent, Heer van alles...}

\marmita{11}

\PSALMtitle{31}{Gebed in nood}

\psalmsubtitle{a) Heer, ik vertrouw op U}

\begin{halfparskip}
  Ik vlucht tot U, O Heer: dat ik nimmer te schande worde;~\sep\ bevrijd mij toch in Uw gerechtigheid!

  \qanona{Ze hebben mij in mijn ellende vergeten als iemand onder de doden, maar de Heer troost mij.}

  Neig Uw oor naar mij;~\sep\ haast U mij te redden!

  Wees mij een rots, waar ik vluchten kan,~\sep\ een versterkte burcht tot mijn behoud.

  Want Gij zijt mijn Rots en mijn Burcht,~\sep\ en omwille van Uw Naam zult Gij mijn Leider zijn en Gids.

  Gij zult mij trekken uit het net, dat zij heimelijk mij spanden,~\sep\ want Gij zijt mijn Toevlucht.

  In Uw handen beveel ik mijn geest:~\sep\ Gij zult mij bevrijden, O Heer, getrouwe God.

  Gij haat, die nietige afgoden dienen,~\sep\ maar ik vertrouw op de Heer.

  Vol vreugde zal ik juichen om Uw ontferming, omdat Gij op mijn ellende hebt neergezien:~\sep\ Gij waart mijn Hulp in tijden van nood.

  Gij gaaft mij niet prijs aan de macht van een vijand,~\sep\ maar plaatste mijn voeten op ruime baan.
\end{halfparskip}

\psalmsubtitle{b) Red mij, O God, in mijn nood!}

\begin{halfparskip}
  O Heer, wees mij genadig, want ik ben in nood;~\sep\ van droefheid kwijnt mijn oog, mijn ziel en mijn lichaam.

  Ja, mijn leven teert weg in smart,~\sep\ en mijn jaren in geween.

  Van verdriet is mijn kracht gebroken,~\sep\ en mijn gebeente verdord.

  Voor al mijn vijanden ben ik tot smaad geworden, voor mijn buren tot spot, en tot afschrik voor mijn bekenden;~\sep\ die mij buiten zien, vluchten van mij heen.

  Ik ben als een dode door vergetelheid uit het hart gewist,~\sep\ en ik werd als een vat in scherven.

  Ja, ik hoorde het fluiten van velen - verschrikking van alle zijden!~\sep\ Zij schoolden tegen mij samen en zonnen op moord.
\end{halfparskip}

\psalmsubtitle{c) Maar mijn vertrouwen blijft ongeschokt}

\begin{halfparskip}
  Maar ik, O Heer, vertrouw op U,~\sep\ en zeg: Gij zijt mijn God.

  Mijn levenslot ligt in Uw hand,~\sep\ ontruk mij aan de hand van mijn vijanden en vervolgers.

  Toon aan Uw dienstknecht Uw vredig gelaat,~\sep\ red mij in Uw barmhartigheid.

  Heer, laat mij niet beschaamd worden, want U riep ik aan;~\sep\ maar dat de bozen zich schamen en zwijgen, voortgedreven naar het dodenrijk.

  Dat de leugenlippen verstommen,~\sep\ die vermetel tegen de rechtvaardige spreken, in trots en verachting.
\end{halfparskip}

\psalmsubtitle{d) Wat is God goed voor die Hem vrezen!}

\begin{halfparskip}
  Hoe groot, Heer, is Uw goedheid,~\sep\ die Gij hebt weggelegd voor hen, die U vrezen,

  Die Gij bewijst aan hen, die vluchten tot U,~\sep\ ten aanschouwen der mensen.

  Gij beschermt hen onder de schutse van Uw aanschijn,~\sep\ tegen het samenzweren der mannen,

  Gij verbergt hen in Uw tent,~\sep\ tegen het schelden der tongen.

  Gezegend de Heer, want Hij bewees mij~\sep\ Zijn wondere barmhartigheid in de versterkte stad.

  Wel sprak ik in mijn onrust:~\sep\ ``Ik ben van Uw aanschijn verstoten.''

  Maar Gij hebt de stem van mijn smeken gehoord,~\sep\ daar ik tot U riep.

  Bemint de Heer, gij, al Zijn heiligen:~\sep\ de getrouwen behoedt de Heer,

  Maar overvloedig vergeldt Hij,~\sep\ die handelen in trots.

  Houdt moed, en weest sterk van hart,~\sep\ gij allen, die hoopt op de Heer.
\end{halfparskip}

\PSALMtitle{32}{Herademing na vergeven schuld}

\psalmsubtitle{a) Gelukkig hij, wiens zonde is vergeven}

\begin{halfparskip}
  Gelukkig hij, wiens misdaad vergeven,~\sep\ wiens zonde bedekt is.

  \qanona{Zoals Hij die de mensheid liefheeft ons heeft gered, laten we Hem behagen, zodat Hij medelijden met ons mag hebben.}

  Gelukkig de mens, wie de Heer zijn schuld niet toerekent,~\sep\ en in wiens geest geen bedrog is.
\end{halfparskip}

\psalmsubtitle{b) Ongeluk van het verzwijgen, geluk van de belijdenis der zonde}

\begin{halfparskip}
  Zolang ik bleef zwijgen, werd mijn gebeente verteerd,~\sep\ onder mijn aanhoudend gezucht.

  Want dag en nacht drukte Uw hand op mij;~\sep\ mijn kracht teerde weg als bij zomerse hitte.

  Mijn zonde heb ik beleden voor U,~\sep\ en mijn schuld hield ik niet verborgen;

  Ik sprak: ``Ik belijd mijn boosheid voor de Heer,''~\sep\ en Gij hebt de schuld van mijn zonde vergeven.

  Daarom moet iedere vrome bidden tot U,~\sep\ in tijden van nood.

  En al breekt dan de watervloed los,~\sep\ hem zal hij niet genaken.

  Gij zijt mijn Toeverlaat, Gij zult voor nood mij behoeden,~\sep\ en mij omgeven met vreugde over mijn redding.
\end{halfparskip}

\psalmsubtitle{c) Bekeert u!}

\begin{halfparskip}
  Ik zal u leren, en de weg wijzen, die gij moet gaan,~\sep\ u onderrichten en vast Mijn ogen richten op u.

  Weest niet als paard en muilezel zonder verstand, wier onstuimigheid men bedwingt met toom en gebit,~\sep\ anders komen ze niet naar u toe.

  Veel smart valt de boze ten deel,~\sep\ maar erbarming omgeeft wie vertrouwt op de Heer.

  Verheugt u in de Heer, weest blijde, gij, rechtvaardigen,~\sep\ en jubelt, gij allen, die oprecht zijt van harte.
\end{halfparskip}

\Slota{U die verheerlijkt wordt door de rechtvaardigen, en aanbeden door de oprechten van hart, en beleden en gezegend in de hemel en op aarde, wij moeten U belijden, aanbidden en verheerlijken in alle seizoenen en tijden, Heer van alles...}

\marmita{12}

\PSALMtitle{33}{Loflied op Gods macht}

\psalmsubtitle{a) Looft uw Schepper!}

\begin{halfparskip}
  Jubelt, rechtvaardigen, in de Heer:~\sep\ de rechtschapenen past een lofzang.

  \qanona{De rechtschapenen past een lofzang, en ook dankzegging.}

  Looft de Heer met de citer,~\sep\ tokkelt voor Hem de tiensnarige harp.

  Zingt voor Hem een nieuw lied,~\sep\ zingt voor Hem in welluidende klanken.

  Want het woord van de Heer is oprecht,~\sep\ en al wat Hij doet, is betrouwbaar.

  Gerechtigheid en recht heeft Hij lief,~\sep\ van de goedheid van de Heer is de aarde vervuld.

  Door het woord van de Heer zijn de hemelen gemaakt,~\sep\ door de adem van Zijn mond heel hun legerschaar.

  Als in een lederen zak verzamelt Hij de wateren der zee,~\sep\ de vloeden bergt Hij op in schuren.

  Heel de aarde vreze de Heer,~\sep\ dat alle bewoners der wereld Hem duchten.

  Want Hij sprak: en het was;~\sep\ Hij beval: en het bestond.

  De Heer verijdelt de raadslagen der naties,~\sep\ doet de plannen der volken te niet.

  Het besluit van de Heer staat voor eeuwig vast,~\sep\ de gedachten van Zijn hart van geslacht tot geslacht.

  Gelukkig het volk, wiens God de Heer is,~\sep\ de natie, die Hij tot erfdeel koos.
\end{halfparskip}

\psalmsubtitle{b) Uw Voorzienigheid waakt over ons}

\begin{halfparskip}
  De Heer blikt neer uit de hemel,~\sep\ en ziet alle kinderen der mensen.

  Uit Zijn woonplaats ziet Hij van verre neer,~\sep\ op allen, die de aarde bewonen:

  Hij, die aller hart geschapen heeft,~\sep\ die let op al hun werken.

  Geen koning verwint door een machtig leger,~\sep\ geen strijder vindt heil in geweldige kracht.

  Het ros is onmachtig de zege te schenken,~\sep\ en brengt geen redding bij al zijn kracht.

  Maar de ogen van de Heer rusten op hen, die Hem vrezen,~\sep\ op hen die op Zijn goedheid hopen.

  Om hen te ontrukken aan de dood,~\sep\ en bij hongersnood te voeden.
\end{halfparskip}

\psalmsubtitle{c) Heer, ik vertrouw op U}

\begin{halfparskip}
  Onze ziel stelt haar hoop op de Heer,~\sep\ Hij is onze Helper en ons Schild.

  Daarom verblijdt zich ons hart in Hem,~\sep\ en vertrouwen wij op Zijn heilige Naam.

  Uw barmhartigheid, Heer, kome over ons,~\sep\ naarmate wij hopen op U.
\end{halfparskip}

\PSALMtitle{34}{God beschermt de rechtvaardigen}

\psalmsubtitle{a) Dank, Heer, voor de redding!}

\begin{halfparskip}
  Prijzen wil ik de Heer te allen tijde,~\sep\ steeds zal mijn mond Hem loven.

  \qanona{Gezegend is de Koning, die de overwinning verleende aan Zijn atleten in hun wedstrijden!}

  Moge mijn ziel op de Heer zich beroemen;~\sep\ dat de geringen het horen en juichen!

  Verheerlijkt de Heer met mij,~\sep\ en prijzen wij samen Zijn Naam!

  Ik heb de Heer gezocht, en Hij heeft mij verhoord,~\sep\ mij bevrijd van al mijn angsten.

  Ziet naar Hem op, opdat ge van blijdschap moogt stralen,~\sep\ en geen schaamrood Uw aanschijn bedekke.

  Zie, een bedrukte riep luid, en de Heer heeft geluisterd,~\sep\ en hem van al zijn kommer bevrijd.

  De engel van de Heer slaat een legerplaats op,~\sep\ rond hen, die Hem vrezen: en hij bevrijdt hen.

  Smaakt en ziet hoe goed de Heer is;~\sep\ gelukkig de man, die heenvlucht naar Hem.

  Vreest de Heer, gij, Zijn heiligen,~\sep\ want die Hem vrezen, lijden geen nood.

  Machtigen werden arm en moesten honger lijden,~\sep\ maar die de Heer zoeken, zal niets goeds ontbreken.
\end{halfparskip}

\psalmsubtitle{b) Godsvrucht maakt gelukkig}

\begin{halfparskip}
  Komt, kinderen, luistert naar mij,~\sep\ ik zal u leren de vreze van de Heer.

  Wie is de mens, die van het leven houdt,~\sep\ naar dagen verlangt om het goede te smaken?

  Behoed uw tong voor het kwaad,~\sep\ en uw lippen voor bedrieglijke woorden.

  Wijk terug van het kwaad en doe het goede;~\sep\ zoek de vrede en jaag hem na!

  De ogen van de Heer zien op de rechtvaardigen neer,~\sep\ en Zijn oren luisteren naar hun geroep.

  Het aanschijn van de Heer is tegen de bozen gericht,~\sep\ om de herinnering aan hen van de aarde te verdelgen.

  De rechtvaardigen riepen luid, en de Heer heeft hen verhoord,~\sep\ en hen van al hun angsten bevrijd.

  De vermorzelden van harte is de Heer nabij,~\sep\ de terneergeslagenen van geest schenkt Hij redding.

  Vele rampen treffen de rechtvaardige,~\sep\ maar uit alle bevrijdt hem de Heer.

  Al zijn beenderen behoedt Hij,~\sep\ niet één ervan zal worden gebroken.

  De boosheid drijft de goddeloze in de dood,~\sep\ die de rechtvaardige haten, worden gestraft.

  De Heer spaart Zijn dienaars het leven,~\sep\ en wie tot Hem vlucht, blijft vrij van straf.
\end{halfparskip}

\Slota{Wij smeken U, die het juiste oordeel velt en wiens onderzoek vol gerechtigheid is, en wiens wraak vol barmhartigheid en medelijden is, wend u naar ons, mijn Heer, heb medelijden met ons en wees ons altijd genadig, zoals U gewend bent, Heer van alles...}

\marmita{13}

\PSALMtitle{35}{Gebed om hulp}

\psalmsubtitle{a) Heer, help mij tegen mijn vijanden!}

\begin{halfparskip}
  Bestrijd, Heer, die mij bestrijden, bekamp die mij bekampen.~\sep\ Grijp schild en beukelaar, en rijs op om mij te helpen.

  \qanona{Zij die branden van ijver voor U worden vervolgd; verwaarloos ze niet, O Messias!}

  Slinger de lans en bedwing mijn vervolgers,~\sep\ zeg tot mijn ziel: ``Ik ben Uw redding.''

  Laat smaad en schande hen treffen, die mijn leven belagen,~\sep\ en vol schaamte terugdeinzen die kwaad tegen mij beramen.

  Dat zij worden als kaf voor de wind,~\sep\ wanneer de engel van de Heer hen voortdrijft.

  Duister en glibberig worde hun pad,~\sep\ wanneer de engel van de Heer hen nazet.

  Want zonder reden hebben zij mij hun net gespannen,~\sep\ zonder reden een kuil voor mij gegraven.

  Moge onverhoeds de ondergang hen treffen, en het net, dat zij mij spanden, hen zelf vangen;~\sep\ mogen zij zelf neerstorten in de kuil, die zij groeven.

  Dan zal ik juichen in de Heer,~\sep\ mij verblijden over Zijn hulp.

  Met geheel mijn wezen zal ik zeggen:~\sep\ ``Heer, wie is U gelijk,

  Die de zwakke bevrijdt van de overmachtige,~\sep\ de zwakke en arme van zijn berover.''

  Brutale getuigen stonden tegen mij op;~\sep\ wat ik mij niet was bewust, legden zij mij ten laste.

  Zij vergolden mij goed met kwaad:~\sep\ lieten mij alleen en verlaten.
\end{halfparskip}

\psalmsubtitle{b) Ondankbaar, Heer, zijn mijn vijanden}

\begin{halfparskip}
  En toch toen zij ziek lagen, trok ik het boetekleed aan,~\sep\ putte mij uit door vasten, en bad, diep voorovergebogen.

  Als gold het een vriend of mijn broeder, zo schreed ik droevig voort;~\sep\ als een, die rouwt over zijn moeder, zo ging ik van droefheid gebukt.

  Maar toen ik wankelde, waren zij blij en liepen te hoop,~\sep\ zij schoolden tegen mij samen en sloegen mij, die geen kwaad vermoedde;

  Zonder ophouden verscheurden zij mij, zij vielen mij aan, bespotten mij,~\sep\ terwijl ze tegen mij knarsetandden.
\end{halfparskip}

\psalmsubtitle{c) Heer, boos zijn mijn vijanden}

\begin{halfparskip}
  Heer, hoe lang nog zult Gij het aanzien?~\sep\ Bevrijd mijn ziel van die brullende dieren en van die leeuwen mijn leven.

  Dan zal ik U danken in de volle vergadering,~\sep\ voor de talloze scharen U prijzen.

  Gun geen leedvermaak aan hen, die zonder reden mijn vijanden zijn,~\sep\ laat hen elkander met de ogen niet toewenken, die mij onverdiend haten.

  Want zij spreken geen woorden van vrede,~\sep\ en tegen de rustige bewoners van het land zinnen zij op bedrog.

  En hun mond sperren zij tegen mij open,~\sep\ zeggende: ``Ha, ha, met eigen ogen hebben wij het gezien!''

  Heer, Gij hebt het gezien, zwijg niet langer!~\sep\ Neen, Heer, houd U niet verre van mij!

  Ontwaak en blijf wakker om mij te verdedigen,~\sep\ ten gunste van mijn rechtszaak, mijn God en mijn Heer!

  Oordeel mij naar Uw gerechtigheid, Heer;~\sep\ mijn God, dat zij niet over mij juichen!

  Laat ze niet denken bij zichzelf: ``Ha, wat gaat het ons naar wens!''~\sep\ Laat ze niet zeggen: ``Wij hebben hem verslonden.''

  Dat ze te schande worden en zich schamen allen tezamen,~\sep\ die zich verheugen over mijn ongeluk;

  Met schaamte en schande worden bedekt,~\sep\ die zich tegen mij verheffen.

  Laat juichen en zich verblijden hen, die mijn zaak gunstig gezind zijn,~\sep\ en laat hen immerdoor spreken:

  ``Verheerlijkt zij de Heer,~\sep\ die het heil van Zijn dienaar bevordert.''

  Dan zal mijn tong Uw gerechtigheid verkondigen,~\sep\ en ten allen tijde Uw roem.
\end{halfparskip}

\PSALMtitle{36}{Menselijke bedorvenheid en Gods mildheid}

\psalmsubtitle{a) Boos is de goddeloze}

\begin{halfparskip}
  De zonde spreekt tot het hart van de goddeloze;~\sep\ geen vreze van God staat hem voor ogen.

  \qanona{Machtige Heer, U bent goed, rechtvaardig en wijs.}

  Want bedrieglijk stelt hij zich voor,~\sep\ dat zijn zonde niet wordt bemerkt, noch wordt verafschuwd.

  De woorden van zijn mond zijn boosheid en bedrog;~\sep\ niet langer gedraagt hij zich wijs en behoorlijk.

  Zelfs op zijn sponde zint hij op boosheid;~\sep\ op het slechte pad houdt hij zich op, wendt zich niet af van het kwaad.
\end{halfparskip}

\psalmsubtitle{b) Gij draagt zorg, Heer, voor alle schepselen}

\begin{halfparskip}
  Uw erbarming, o Heer, reikt tot de hemel,~\sep\ en tot de wolken zelf Uw trouw.

  Uw gerechtigheid is als de bergen van God, Uw oordelen zijn diep als de zee;~\sep\ van mens en dier, O Heer, zijt Gij het behoud.

  Wat is Uw genade kostbaar, O God!~\sep\ in de schaduw van Uw vleugels bergen zich de kinderen der mensen.

  Zij worden verzadigd door de weelde van Uw huis,~\sep\ en met de stroom van Uw geneugten laaft Gij hen.

  Want bij U is de bron van het leven,~\sep\ en in Uw licht zien wij het licht.

  Blijf Uw genade schenken aan hen, die U dienen,~\sep\ en Uw gerechtigheid aan de oprechten van hart.

  Laat de voet van de trotse mij niet vertreden,~\sep\ en mij niet schokken de hand van de zondaar.

  Ziet reeds zijn de booswichten gevallen;~\sep\ neergeworpen zijn ze en kunnen niet meer opstaan.
\end{halfparskip}

% % % % % % % % % % % % % % % % % % % % % % % % % % % % % % % % % % % % % % % %

\hulala{5}

\Slota{Wij belijden, aanbidden en verheerlijken U, die goed en vriendelijk bent, meelevend en barmhartig, grote Koning der glorie, die van eeuwigheid bent, in alle seizoenen en tijden, Heer van alles, Vader...}

\marmita{14}

\PSALMtitle{37}{Het lot van bozen en braven}

\psalmsubtitle{a) Vertrouw op Gods Voorzienigheid}

\begin{halfparskip}
  Ontbrand niet in toorn vanwege de zondaars,~\sep\ en benijd de boosdoeners niet.

  \qanona{Gerechtigheid verschijnt plotseling en vernietigt de goddelozen.}

  Spoedig toch vallen zij neer als hooi,~\sep\ en verwelken als het groene gras.

  Vertrouw op de Heer, en doe het goede,~\sep\ om het Land te bewonen en een veilig bestaan te genieten.

  Stel uw vreugde in de Heer,~\sep\ en Hij zal u schenken wat uw hart maar begeert.

  Vertrouw de Heer uw levensweg toe,~\sep\ en hoop op Hem, Hij zal wel zorgen.

  Hij zal als het licht uw gerechtigheid doen opgaan,~\sep\ en als de middagzon uw recht.

  Verlaat u op de Heer,~\sep\ en stel uw hoop op Hem.

  Vertoorn u niet op hem, die het wel gaat in het leven,~\sep\ op de mens, die het kwade beraamt.

  Leg uw verbolgenheid af en laat varen uw gramschap,~\sep\ ontbrand niet in toorn om geen kwaad te bedrijven.

  Immers: de boosdoeners worden te gronde gericht,~\sep\ maar die hopen op de Heer, zullen het land bezitten.

  Nog een weinig tijds, en weg is de boze;~\sep\ en zoekt ge zijn plaats, hij is er niet meer.

  Maar de zachtmoedigen zullen het Land bezitten,~\sep\ en een overvloedige vrede genieten.
\end{halfparskip}

\psalmsubtitle{b) De boze gaat spoedig te gronde}

\begin{halfparskip}
  De goddeloze zint op onheil tegen de rechtvaardige,~\sep\ en knarst tegen hem met de tanden;

  De Heer spot met hem,~\sep\ omdat Hij zijn dag ziet naderen.

  De bozen trekken het zwaard en spannen de boog, om de ellendige en arme neer te vellen,~\sep\ te doden die gaan langs de rechte weg.

  Maar hun zwaard zal hun eigen hart doorboren,~\sep\ en hun bogen zullen worden gebroken.

  Beter het schamele, dat de gerechte bezit,~\sep\ dan de grote rijkdom der bozen.

  Want de armen der bozen zullen worden gebroken,~\sep\ maar de Heer is een steun voor de rechtvaardigen.

  De Heer draagt zorg voor het leven der vromen,~\sep\ en hun erfdeel blijft eeuwig bestaan.

  Zij zullen bij rampspoed niet worden beschaamd,~\sep\ maar verzadigd worden bij hongersnood.

  Maar de goddelozen zullen vergaan, en de vijanden van de Heer als de tooi der weiden verwelken:~\sep\ ze zullen vervliegen als rook.

  De goddeloze leent en geeft niet terug,~\sep\ maar de gerechte is genadig en geeft.
\end{halfparskip}

\psalmsubtitle{c) Duurzaam is het geluk der vromen}

\begin{halfparskip}
  Want die Hij zegent, zullen het Land bezitten,~\sep\ maar die Hij vloekt, zullen vergaan.

  Door de Heer worden de schreden van de mens ondersteund;~\sep\ en in zijn wandel schept Hij behagen.

  Mocht hij vallen, hij valt niet languit,~\sep\ want de Heer houdt hem vast bij de hand,

  Eens was ik een kind en nu ben ik een grijsaard, maar nooit heb ik een rechtvaardige verlaten gezien,~\sep\ noch zag ik zijn kinderen bedelen om brood.

  Steeds is hij meedogend en geeft hij te leen,~\sep\ en zijn nakroost zal gezegend worden.

  Houd u af van het kwaad en doe het goede,~\sep\ opdat gij voor eeuwig moogt leven.

  Want de Heer heeft de gerechtigheid lief,~\sep\ en Hij verlaat Zijn heiligen niet.

  De bozen worden vernietigd,~\sep\ het geslacht der goddelozen verdelgd.

  De rechtvaardigen zullen het Land bezitten,~\sep\ en zullen daar wonen voor immer.
\end{halfparskip}

\psalmsubtitle{d) De vrome is gelukkig}

\begin{halfparskip}
  De mond van de rechtvaardige spreekt wijsheid,~\sep\ en wat recht is, verkondigt zijn tong.

  Hij draagt de Wet van zijn God in zijn hart,~\sep\ en zijn schreden wankelen niet.

  De goddeloze bespiedt de rechtvaardige,~\sep\ en zoekt hem te doden.

  De Heer laat hem niet in zijn macht,~\sep\ en veroordeelt hem niet in het gericht.

  Vertrouw op de Heer,~\sep\ en bewandel Zijn weg;

  Dan helpt Hij u voort om het Land te bezitten,~\sep\ en gij zult vol vreugde de verdelging der bozen aanschouwen.
\end{halfparskip}

\psalmsubtitle{e) De boze is ongelukkig}

\begin{halfparskip}
  Ik heb de goddeloze gezien in zijn trots:~\sep\ hij breidde zich uit als een bladerrijke ceder.

  En ik ging voorbij, doch zie, hij was er niet meer,~\sep\ ik zocht naar hem, maar hij was niet te vinden.

  Let op de vrome en beschouw de rechtvaardige,~\sep\ want een vreedzaam man heeft een nageslacht.

  Maar de zondaars gaan allen te gronde,~\sep\ het nakroost der goddelozen zal worden verdelgd.

  Het heil der rechtvaardigen komt van de Heer;~\sep\ hun Toevlucht is Hij ten tijde van rampspoed.

  De Heer is hun Helper, Hij schenkt hun bevrijding,~\sep\ Hij bevrijdt hen van bozen en brengt hun de redding, daar zij hun toevlucht nemen tot Hem.
\end{halfparskip}

\Slota{Tuchtig ons niet, O Heer, in Uw woede en toorn; vergeld ons niet zoals onze overtredingen verdienen, maar in Uw barmhartigheid en medelijden, O Heer, wend U tot ons, heb erbarmen en medelijden met ons, zoals U altijd gewend bent, Heer van alles...}

\marmita{15}

\PSALMtitle{38}{Boetepsalm}

\psalmsubtitle{a) Mijn lijden, Heer, is de straf voor mijn zonden}

\begin{halfparskip}
  Heer, straf mij niet in Uw toorn,~\sep\ en in Uw gramschap kastijd mij niet.

  \qanona{Medelijdende, laat onze kastijding getemperd worden door Uw genade.}

  Want Uw pijlen zijn in mij doorgedrongen,~\sep\ en Uw hand is op mij neergedaald.

  Door Uw toorn is er geen plek meer gezond aan mijn vlees,~\sep\ in mijn gebeente niets gaafs vanwege mijn zonde.

  Want mijn zonden zijn mij boven het hoofd gestegen:~\sep\ al te zeer drukken ze mij als een zware last.
\end{halfparskip}

\psalmsubtitle{b) Hevig, O God, is mijn lijden}

\begin{halfparskip}
  Kwalijk rieken mijn builen in hun ontbinding,~\sep\ vanwege mijn dwaasheid.

  Ik ben gebogen en diep gekromd;~\sep\ ik sleep mij treurend heel de dag voort.

  Want mijn lendenen zijn geheel ontstoken,~\sep\ en geen gezonde plek is er nog aan mijn vlees.

  Ik kwijn weg en ben geheel gebroken;~\sep\ van hartzeer snik ik het uit.

  Heer, voor Uw aanschijn ligt heel mijn verlangen open,~\sep\ en mijn zuchten is niet verborgen voor U.

  Mijn hart klopt hevig, mijn kracht heeft mij begeven,~\sep\ en het licht van mijn ogen, ook dat moet ik derven.

  Mijn vrienden en gezellen houden zich ver van mijn wonde,~\sep\ mijn verwanten blijven op afstand staan.

  Die mijn leven belagen, spannen strikken, die mij kwaad willen dreigen met ondergang;~\sep\ voortdurend zijn ze uit op bedrog.
\end{halfparskip}

\psalmsubtitle{c) Mijn God, ik vertrouw op U}

\begin{halfparskip}
  Ik echter ben als een dove en hoor niet,~\sep\ als een stomme, die zijn mond niet opent.

  Ik ben geworden als een mens, die niet hoort,~\sep\ en wiens mond het antwoord schuldig blijft.

  Op U toch, Heer, vertrouw ik;~\sep\ Gij, O Heer, mijn God, zult mij verhoren.

  Ik zeg immers: ``Laten zij zich niet vrolijk over mij maken,~\sep\ niet schamper tegen mij uitvallen, als mijn voet soms wankelt.''

  Want ja, ik ben de val nabij,~\sep\ en mijn smart staat mij steeds voor ogen.

  En inderdaad, ik belijd mijn schuld,~\sep\ en ben vol kommer vanwege mijn zonde.

  Maar machtig zijn zij, die mij zonder reden bestrijden,~\sep\ en talrijk, die mij ten onrechte haten.

  En die goed met kwaad vergelden,~\sep\ vallen mij aan, omdat ik het goede volg.

  Heer, verlaat mij toch niet,~\sep\ mijn God, sta niet ver van mij af!
\end{halfparskip}

\PSALMtitle{39}{Vast vertrouwen op God}

\psalmsubtitle{a) Zwijgend, Heer, wil ik mijn ellende dragen}

\begin{halfparskip}
  Ik sprak: Over mijn wandel zal ik waken,~\sep\ om niet te zondigen met mijn tong;~\sep

  Mijn mond zal ik beteugelen,~\sep\ zolang de boze voor mij staat.

  \qanona{De goddelozen hebben mij onderdrukt (\translationoptionNl{gekweld}), O onze Heer; in U is mijn ware hoop!}

  Zwijgend bleef ik en stom, verstoken van geluk,~\sep\ maar des te feller werd mijn smart.

  In mijn binnenste gloeide mijn hart; als ik nadacht, laaide het vuur in mij op;~\sep\ mijn tong begon te spreken.
\end{halfparskip}

\psalmsubtitle{b) Kort en nietig is het leven}

\begin{halfparskip}
  Heer, doe mij mijn einde kennen, en het getal van mijn dagen,~\sep\ opdat ik wete hoe vergankelijk ik ben.

  Zie, enkele handpalmen lang hebt Gij mijn dagen gemaakt, en als een niet is mijn leven voor U;~\sep\ als een ademtocht slechts leeft iedere mens.

  Als een schaduw slechts gaat de mens voorbij; hij is in verwarring om niets;~\sep\ schatten stapelt hij op, maar weet niet wie ze zal krijgen.
\end{halfparskip}

\psalmsubtitle{c) Op U, O God, vertrouw ik}

\begin{halfparskip}
  En nu, Heer, wat kan ik nog verwachten?~\sep\ Op U is mijn vertrouwen gesteld.

  Verlos mij van al mijn zonden,~\sep\ lever mij niet over aan de spot van de dwaas.

  Ik zwijg en doe mijn mond niet open,~\sep\ want Gij deedt het mij aan.

  Wend toch Uw slagen van mij af;~\sep\ ik bezwijk onder de druk van Uw hand.

  Met straf voor schuld kastijdt Gij de mens; Gij verteert als de motten zijn kostbaarheden:~\sep\ een ademtocht slechts is iedere mens.

  Luister, Heer, naar mijn gebed, en hoor naar mijn smeken;~\sep\ blijf niet doof voor mijn snikken.

  Immers, ik ben een gast bij U,~\sep\ een pelgrim gelijk al mijn vaderen.

  Wend Uw ogen van mij af, opdat ik nog ademe,~\sep\ eer ik heenga en niet meer ben.
\end{halfparskip}

\PSALMtitle{40}{Gehoorzaamheid het beste offer}

\psalmsubtitle{a) Dank, Heer, voor Uw weldaden!}

\begin{halfparskip}
  Ik heb gehoopt, gehoopt op de Heer, en Hij boog zich naar mij,~\sep\ en verhoorde mijn smeken.

  \qanona{Medelijdende, Uw weldaden voor ons zijn niet te tellen!}

  Hij trok mij op uit de kuil van de dood, uit modder en slijk; op de rots heeft Hij mijn voeten geplaatst,~\sep\ mijn schreden heeft Hij gesteund.

  Hij legde een nieuw lied in mijn mond,~\sep\ een lofzang voor onze God.

  Velen zullen het zien, vervuld van ontzag,~\sep\ en zullen op de Heer vertrouwen.

  Gelukkig de man, die op de Heer zijn hoop heeft gesteld,~\sep\ geen dienaars van afgoden volgt, noch hen, die tot verzinsels zich wenden.

  Talrijk, O Heer, mijn God, hebt Gij Uw wonderwerken gemaakt,~\sep\ en in Uw raadsbesluiten over ons is niemand U gelijk.

  Wilde ik ze verhalen en verkondigen:~\sep\ ze zijn te talrijk om te worden geteld.
\end{halfparskip}

\psalmsubtitle{b) Uw wil, O God, boven offers!}

\begin{halfparskip}
  Slacht- noch spijsoffer hebt Gij gewild,~\sep\ maar Gij hebt mij de oren geopend.

  Brand- noch zoenoffer hebt Gij voor de zonde geëist:~\sep\ toen heb ik gezegd: ``Zie, ik kom; in de boekrol staat over mij geschreven:

  Het is mijn geneugte, mijn God, Uw wil te volbrengen,~\sep\ en diep in mijn hart staat Uw Wet gegrift.''

  De gerechtigheid heb ik verkondigd in de volle vergadering;~\sep\ neen, Heer, Gij weet het: mijn lippen hield ik niet gesloten.

  Uw gerechtigheid verborg ik niet in mijn hart;~\sep\ Uw trouw en Uw hulp heb ik verkondigd;

  Uw goedheid hield ik niet geheim,~\sep\ noch in de volle vergadering Uw trouw.
\end{halfparskip}

\psalmsubtitle{c) Heer, kom mij te hulp!}

\begin{halfparskip}
  Gij dan, O Heer, onthoud mij Uw erbarming niet;~\sep\ laat Uw genade en trouw mij immer behoeden.

  Want talloze rampen hebben mij omgeven, mijn zonden hebben mij aangegrepen,~\sep\ zodat ik niet kan zien;

  Talrijker zijn ze dan de haren op mijn hoofd:~\sep\ en de moed is mij ontzonken.

  Het behage U, O Heer, mij te verlossen;~\sep\ Heer, snel mij te hulp!

  Laten allen te schande en schaamrood worden,~\sep\ die zoeken mij van het leven te beroven.

  Terugwijken vol schaamte mogen zij,~\sep\ die zich over mijn rampen verheugen,

  Dat zij verstommen, met schande bedekt,~\sep\ die mij toeroepen: Ha, ha!

  In U mogen jubelen vol blijdschap allen die U zoeken;~\sep\ dat zij, die uitzien naar Uw hulp, immer herhalen: ``Hooggeprezen zij de Heer.''

  Ach, ik ben ellendig en arm,~\sep\ maar de Heer is vol zorg voor mij.

  Mijn Helper en Redder zijt Gij;~\sep\ mijn God, wil niet toeven!
\end{halfparskip}

% % % % % % % % % % % % % % % % % % % % % % % % % % % % % % % % % % % % % % % %

\hulala{6}

\Slota{Wij belijden, aanbidden en verheerlijken U, rijk in Uw liefde, overvloeiend van mededogen, vriendelijk in Uw goedheid, onuitsprekelijk in Uw glorie, grote Koning der glorie, Wezen dat van eeuwigheid is, in alle seizoenen en tijden, Heer van alles...}

\marmita{16}

\PSALMtitle{41}{Vertrouwvol gebed bij misleiding}

\psalmsubtitle{a) Zegen, Heer, de barmhartige!}

\begin{halfparskip}
  Gelukkig die denkt aan de behoeftige en arme:~\sep\ de Heer zal hem redden op de dag van het onheil.

  \qanona{Gezegend is hij die genade heeft gevonden in de rechtbank van Uw gerechtigheid!}

  De Heer zal hem behoeden en in het leven bewaren, hem gelukkig maken op aarde,~\sep\ en niet prijsgeven aan de moedwil van zijn vijanden.

  De Heer zal hem bijstaan op zijn lijdenssponde:~\sep\ alle zwakheid van hem wegnemen in zijn ziekte.
\end{halfparskip}

\psalmsubtitle{b) Mijn vijanden belagen mij}

\begin{halfparskip}
  Ik bid dan: Heer, wees mij genadig;~\sep\ genees mij, omdat ik tegen U heb gezondigd.

  Kwade dingen zeggen mijn vijanden over mij:~\sep\ ``Wanneer zal hij sterven en zal zijn naam verdwijnen?''

  En wie op bezoek komt, spreekt huichelachtig;~\sep\ stof tot laster verzamelt zijn hart, en hij gaat het buiten vertellen.

  Allen, die mij haten, fluisteren samen over mij,~\sep\ en denken aan de rampspoed, die over mij kwam.

  ``Een boosaardige pest heeft hem besmet,''~\sep\ en: ``Die daar neerligt, zal niet meer opstaan.''

  Zelfs mijn vriend, op wie ik vertrouwde,~\sep\ die mijn brood at, heeft tegen mij zijn hiel geheven.
\end{halfparskip}

\psalmsubtitle{c) Bevrijd mij, Heer, van mijn vijanden!}

\begin{halfparskip}
  Maar Gij, O Heer, wees mij genadig en richt mij op,~\sep\ opdat ik het hun vergelde.

  Hieraan zal ik erkennen, dat Gij mij gunstig gezind zijt,~\sep\ dat mijn vijand niet over mij juicht.

  Na mijn herstel nu zult Gij mij steunen,~\sep\ mij voor eeuwig voor Uw aangezicht plaatsen.

  Gezegend zij de Heer, de God van Israël,~\sep\ van eeuwigheid tot eeuwigheid. Het zij zo, het zij zo.
\end{halfparskip}

\PSALMtitle{42}{Heimwee naar Gods woning}

\psalmsubtitle{a) Naar U, O God, verlangt mijn hart}

\begin{halfparskip}
  Gelijk de hinde smacht naar waterstromen,~\sep\ zo smacht mijn ziel naar U, O God.

  \qanona{Tot U, mijn God, smeek ik: in Uw barmhartigheid, moge ik hersteld (\translationoptionNl{bekeerd}) worden.}

  Naar God dorst mijn ziel, naar de levende God;~\sep\ wanneer mag ik komen en Gods aanschijn aanschouwen?

  Mijn tranen zijn mij tot spijs geworden dag en nacht,~\sep\ terwijl men steeds tot mij zegt: ``Waar is uw God?''

  Ik denk er met diepe weemoed aan terug, hoe ik eenmaal voortschreed met de schare,~\sep\ ja, hen voorging naar Gods huis,

  Onder gejubel en lofzang,~\sep\ in feestelijke stoet.

  Waarom zijt gij bedrukt, mijn ziel,~\sep\ en waarom vol onrust in mij?

  Stel Uw hoop op God, want opnieuw zal ik Hem prijzen,~\sep\ het Heil van mijn aanschijn en mijn God.
\end{halfparskip}

\psalmsubtitle{b) Rampspoed komt over mij}

\begin{halfparskip}
  Mijn ziel is bedrukt in mij:~\sep\ daarom denk ik aan U vanuit het land van de Jordaan en de Hermon, en vanaf de berg Mizar.

  De ene kolk roept de andere op bij het bruisen van Uw watervallen:~\sep\ al Uw golven en baren gingen over mij heen.

  Bij dag schenke de Heer vrijgevig Zijn genade,~\sep\ en bij nacht wil ik voor Hem zingen, en prijzen de God van mijn leven.

  Ik zeg tot God: ``Mijn Rots, waarom vergeet Gij mij;~\sep\ waarom ga ik droevig voort, door de vijand verdrukt?

  Mijn beenderen worden verbrijzeld terwijl mijn weerstrevers mij honen,~\sep\ en mij dagelijks zeggen: ``Waar blijft toch uw God?''

  Waarom zijt gij bedrukt, mijn ziel,~\sep\ en waarom vol onrust in mij?

  Stel uw hoop op God, want opnieuw zal ik Hem prijzen,~\sep\ het Heil van mijn aanschijn en mijn God!
\end{halfparskip}

\PSALMtitle{43}{Smachten naar de tempel}

\psalmsubtitle{a) Uw genade voert mij tot U terug}

\begin{halfparskip}
  Schaf mij recht, God, en verdedig mijn zaak tegen een onheilig volk;~\sep

  \qanona{Schaf mij recht, Heer, tegen mijn tegenstanders en laat Uw hulp mij begeleiden.}

  verlos mij van de bedrieger en de boze,

  Want gij, God, zijt mijn Kracht; waarom hebt Ge me verstoten,~\sep\ waarom ga ik droevig voort terwijl de vijand me verdrukt?
\end{halfparskip}

\psalmsubtitle{b) Ik verlang, Heer, naar Uw heiligdom}

\begin{halfparskip}
  Zend Uw licht en Uw trouw: laat deze mij leiden,~\sep\ mij voeren naar Uw heilige berg en in Uw woontenten.

  Dan zal ik opgaan naar het altaar van God,~\sep\ naar God, mijn Vreugde en mijn Jubel;

  En ik zal U loven op de citer,~\sep\ God, mijn God.

  Waarom zijt gij bedrukt, mijn ziel,~\sep\ en waarom opstandig in mij?

  Vertrouw op God, want opnieuw zal ik Hem prijzen,~\sep\ het Heil van mijn aanschijn en mijn God.
\end{halfparskip}

\Slota{Wij belijden, aanbidden en verheerlijken U in alle seizoenen en tijden, onze Schepper en Weldoener, Hersteller en Gids van onze zielen door het zachte gebod van Uw wil, grote Koning der glorie, Wezen van eeuwigheid, Heer van alles,...}

\marmita{17}

\PSALMtitle{44}{Schep moed in het rijk verleden}

\psalmsubtitle{a) In het verleden, Heer, stond Gij ons bij}

\begin{halfparskip}
  Met eigen oren, O God, hebben wij het gehoord,~\sep\ onze vaderen hebben het ons verhaald:

  \qanona{O Schepper, die onze eerste vaders verloste door Uw kracht, red Uw aanbidders die U aanroepen.}

  Het werk, dat Gij gewrocht hebt in hun dagen,~\sep\ in overoude tijden.

  Met eigen hand hebt Gij de heidenen verdreven, maar hen geplant;~\sep\ om hen te doen bloeien hebt Gij volken verslagen.

  Neen, hun zwaard was het niet, dat het land heeft veroverd,~\sep\ noch hun arm, die hun redding bracht.

  Maar Uw rechter en Uw arm,~\sep\ en het licht van Uw gelaat, want Gij hadt hen lief.

  Gij zijt mijn Koning, mijn God,~\sep\ die de zege verleende aan Jacob.

  Door U hebben wij onze tegenstanders verdreven,~\sep\ en in Uw Naam vertrapten wij die tegen ons waren opgestaan.

  Neen, ik heb niet vertrouwd op mijn boog,~\sep\ noch was het mijn zwaard, dat mij redding bracht.

  Maar Gij hebt ons verlost van onze weerstrevers,~\sep\ en die ons haten, hebt Gij beschaamd.

  In God roemden wij ten allen tijde,~\sep\ en Uw Naam prezen wij immer.
\end{halfparskip}

\psalmsubtitle{b) Groot, Heer, is onze kwelling!}

\begin{halfparskip}
  Maar nu hebt Gij ons verstoten en te schande gemaakt,~\sep\ en Gij trekt niet meer op, God, met onze legerscharen.

  Gij hebt ons doen wijken voor onze tegenstanders;~\sep\ die ons haten, hebben ons uitgeplunderd.

  Als slachtschapen hebt Gij ons overgeleverd,~\sep\ en onder de volken verstrooid.

  Voor een spotprijs hebt Gij Uw volk verkocht,~\sep\ en weinig winst heeft U de verkoop gebracht.

  Gij hebt ons te schande gemaakt voor onze buren,~\sep\ tot spot en hoon voor hen, die ons omringen.

  Gij hebt ons tot spreekwoord gemaakt onder de heidenen;~\sep\ de volken schudden het hoofd over ons.

  Mijn schande staat mij immer voor ogen,~\sep\ en schaamte bedekt mijn gelaat,

  Om de praatjes van schimper en spotter,~\sep\ om vijand en tegenstander.
\end{halfparskip}

\psalmsubtitle{c) Red ons, Heer, om onze trouw!}

\begin{halfparskip}
  Dit alles kwam over ons; en toch: wij zijn U niet vergeten;~\sep\ en hebben Uw verbond niet geschonden;

  Ook viel ons hart niet van U af,~\sep\ en onze schreden weken niet af van Uw wegen,

  Toen Gij ons gebroken hebt in het oord der kwelling,~\sep\ en ons met duisternis hebt omhuld.

  Hadden wij de Naam van onze God vergeten,~\sep\ en naar een vreemde god onze handen uitgestrekt,

  Zou God dat niet hebben ontdekt?~\sep\ Hij toch doorschouwt de geheimen van het hart.

  Ja, om Uwentwil blijft men ons doden,~\sep\ worden wij als slachtschapen beschouwd.

  Ontwaak dan, Heer, wat slaapt Gij?~\sep\ Waak op, blijf ons niet eeuwig verstoten!

  Waarom verbergt Gij Uw aanschijn,~\sep\ vergeet Gij onze ellende en onze verdrukking?

  Want onze ziel ligt neer in het stof,~\sep\ ons lichaam kleeft aan de aarde.

  Sta op om ons te helpen,~\sep\ en bevrijd ons om Uw barmhartigheid.
\end{halfparskip}

\PSALMtitle{45}{Een koningslied}

\begin{halfparskip}
  Een heerlijk lied welt op uit mijn hart: de Koning wijd ik mijn zang;~\sep

  \qanona{Eer aan U, onze Verlosser, want U hebt Uw uitverkoren Kerk geëerd en haar met alle schoonheden versierd.}

  mijn tong is de stift van een vaardige schrijver.
\end{halfparskip}

\psalmsubtitle{a) Voortreffelijkheid van de Bruidegom}

\begin{halfparskip}
  Gij zijt de schoonste onder de kinderen der mensen; bevalligheid ligt op uw lippen:~\sep\ daarom heeft God u voor eeuwig gezegend.

  Gord uw zwaard om de heup, gij, machtige held,~\sep\ uw sieraad en luister.

  Ruk zegerijk uit voor waarheid en recht;~\sep\ uw rechterhand lere u roemrijke daden.

  Uw pijlen zijn scherp; volkeren worden aan u onderworpen;~\sep\ aan de vijanden van de Koning ontzinkt de moed.

  In de eeuwen der eeuwen staat Uw troon, O God,~\sep\ een scepter van recht is de scepter van Uw rijk.


  Gij hebt de gerechtigheid lief en haat de boosheid; daarom heeft God, uw God, u gezalfd,~\sep\ met de olie der vreugde boven uw genoten.

  Van mirre en aloë en cassia geuren uw gewaden;~\sep\ uit ivoren paleizen klinkt u blij het harpgeluid tegen.

  Koningsdochters treden u tegemoet,~\sep\ de Koningin staat aan uw rechterhand, met goud uit Ofir getooid.
\end{halfparskip}

\psalmsubtitle{b) Schoonheid der bruid}

\begin{halfparskip}
  Hoor, dochter, en zie, en neig uw oor,~\sep\ en vergeet uw volk en het huis van uw vader.

  Dan zal aan de Koning uw schoonheid behagen;~\sep\ Hij is uw Heer, breng Hem uw hulde.

  Dan komt met geschenken het volk van Tyrus,~\sep\ de voornamen onder het volk dingen om uw gunst.

  In volle luister treedt de dochter van de Koning binnen;~\sep\ met goud doorweven is haar gewaad.

  In een kleurige mantel wordt zij voor de Koning geleid,~\sep\ in haar gevolg worden maagden, haar gezellinnen, tot u gevoerd;

  Zij worden voorgeleid in blijde jubel,~\sep\ en treden het paleis van de Koning binnen.
\end{halfparskip}

\psalmsubtitle{c) Uw rijk, Heer, is algemeen en eeuwig}

\begin{halfparskip}
  In de plaats van uw vaderen komen uw zonen:~\sep\ gij zult hen aanstellen tot vorsten over heel de wereld.

  Ik zal uw naam doen gedenken bij alle geslachten;~\sep\ daarom zullen de volken u prijzen in de eeuwen der eeuwen.
\end{halfparskip}

\PSALMtitle{46}{God is als een sterke burcht}

\psalmsubtitle{a) God, de Beschermer van Zijn volk}

\begin{halfparskip}
  God is ons een Toevlucht en Kracht;~\sep\ een machtige Helper toonde Hij zich in de nood.

  \qanona{God, onze onoverwinnelijke Helper, vernietig de hoogmoedigen en maak ons blij met Uw verlossing.}

  Daarom vrezen wij niet, al wordt ook de aarde geschokt,~\sep\ al storten de bergen midden in zee.

  Laat bruisen en koken haar wateren,~\sep\ laat schudden de bergen door haar geweld:

  De Heer der heerscharen is met ons,~\sep\ de God van Jacob is onze bescherming.
\end{halfparskip}

\psalmsubtitle{b) God woont in Zijn heilige stad}

\begin{halfparskip}
  De armen van de stroom verblijden de stad van God,~\sep\ de heilige woontent van de Allerhoogste.

  God woont daarbinnen, zij zal niet wankelen;~\sep\ God staat haar bij van de vroege dageraad af.

  Volkeren woedden, en koninkrijken werden geschokt:~\sep\ daar galmde Zijn donderstem, en weg vloeide de aarde.

  De Heer der heerscharen is met ons,~\sep\ de God van Jacob is onze bescherming.
\end{halfparskip}

\psalmsubtitle{c) Zijn rijk is een rijk van vrede}

\begin{halfparskip}
  Komt en aanschouwt de werken van de Heer,~\sep\ de wonderen, die Hij op aarde gewrocht heeft.

  Hij bedwingt de oorlogen tot aan het einde der aarde,~\sep\ Hij verbrijzelt de bogen, breekt lansen stuk, in het vuur verbrandt Hij de schilden.

  Houdt op, en erkent Mij als God,~\sep\ verheven onder de volken, verheven op aarde.

  De Heer der heerscharen is met ons,~\sep\ de God van Jacob is onze bescherming.
\end{halfparskip}

\Slota{Wij moeten U belijden, aanbidden en verheerlijken, luisterrijke en glorierijke Koning, wiens Majesteit de naties en volkeren aanbidden, klappend in hun handen, in alle seizoenen en tijden, Heer van alles...}

\marmita{18}

\PSALMtitle{47}{Gods opperheerschappij}

\psalmsubtitle{a) God schenkt Zijn volk de zege}

\begin{halfparskip}
  Volken, gij alle, klapt in de handen,~\sep\ juicht God toe met jubelzang!

  \qanona{Gezegend is Hij die neerdaalde en ons Zijn lichaam als voedsel gaf en door Zijn bloed de schulden van Zijn kudde uitwiste.}

  Want hoogverheven, ontzagwekkend is de Heer,~\sep\ de grote Koning van heel de aarde.

  Hij onderwerpt ons de volken,~\sep\ en legt de naties onder onze voeten.

  Ons erfdeel kiest Hij voor ons uit,~\sep\ de roem van Jacob, die Hij liefheeft.
\end{halfparskip}

\psalmsubtitle{b) Loof God, heel de aarde!}

\begin{halfparskip}
  God stijgt op onder gejubel,~\sep\ de Heer onder bazuingeschal.

  Zingt voor God, zingt Hem toe,~\sep\ zingt voor onze Koning, zingt Hem toe!

  Want God is Koning over heel de aarde:~\sep\ zingt een lofzang.

  God heerst over de volkeren,~\sep\ God zetelt op Zijn heilige troon.

  De vorsten der volken sloten zich aan,~\sep\ bij het volk van Abrahams God.

  Want aan God behoren de machtigen der aarde:~\sep\ Hij is hoogverheven.
\end{halfparskip}

\PSALMtitle{48}{Onaantastbaarheid van Gods stad}

\psalmsubtitle{a) Gij hebt Jeruzalem gered, O Heer!}

\begin{halfparskip}
  Groot is de Heer en hoogst lofwaardig,~\sep\ in de stad van onze God.

  \qanona{Jubel en verheug u, ras van Adam, dat verheven werd in Jezus, die opgestaan is en de dood heeft overwonnen door Zijn dood.}

  Zijn heilige, zijn roemvolle heuvel,~\sep\ is de vreugde van heel het aardrijk.

  De berg Sion, het uiterste noorden,~\sep\ is de stad van de grote Koning.

  God in haar burchten,~\sep\ toonde zich een veilige schutse.

  Want ziet, de koningen sloten een verbond,~\sep\ en rukten gezamenlijk op.

  Eén blik! Ze staan verbijsterd,~\sep\ ze sidderen en stuiven uiteen.

  Daar grijpt ontzetting hen aan,~\sep\ smart als van een barende,

  Zoals wanneer de oostenwind,~\sep\ de schepen van Tharsis verbrijzelt.
\end{halfparskip}

\psalmsubtitle{b) Dank, Heer, voor de bevrijding}

\begin{halfparskip}
  Gelijk wij het hoorden, zo hebben wij het nu gezien,~\sep\ in de stad van de Heer der heerscharen,

  In de stad van onze God:~\sep\ God houdt haar eeuwig in stand.

  Wij gedenken Uw barmhartigheid, O God,~\sep\ binnen Uw tempel.

  Zoals Uw Naam, O God, zo ook Uw lof:~\sep\ hij reikt tot de grenzen der aarde.

  Vol gerechtigheid is Uw rechterhand,~\sep\ dat de berg Sion zich verblijde,

  Dat Juda's steden jubelen,~\sep\ om Uw gerichten!

  Laat Uw blikken gaan over Sion, en wandelt er omheen,~\sep\ telt zijn torens.

  Beschouwt zijn bolwerken,~\sep\ doorloopt zijn burchten,

  Om het te verhalen aan het nageslacht:~\sep\ ``Zo groot is God,

  Onze God voor eeuwig en immer:~\sep\ Hij zelf zal ons geleiden.''
\end{halfparskip}

\PSALMtitle{49}{Ook de rijken sterven}

\psalmsubtitle{a) Luistert naar Mijn leer!}

\begin{halfparskip}
  Aanhoort het, alle volkeren;~\sep\ luistert, alle bewoners der aarde:

  \qanona{Hoort nu, heersers van het volk: dient met vrees God, de Heer van alles!}

  Zowel geringen als edelen,~\sep\ rijken als armen op eenzelfde wijze.

  Mijn mond gaat wijsheid verkondigen,~\sep\ de overweging van mijn hart brengt inzicht.

  Mijn oor wil ik neigen naar een leer van wijsheid,~\sep\ bij het spel van de citer mijn raadsel onthullen.
\end{halfparskip}

\psalmsubtitle{b) Rijkdom redt de boze niet van de dood}

\begin{halfparskip}
  Waarom zou ik vrezen op dagen van rampspoed,~\sep\ als de boosheid van belagers mij omringt,

  Die op hun rijkdom vertrouwen,~\sep\ en pochen op hun groot bezit?

  Want niemand kan zichzelf bevrijden,~\sep\ noch zijn losgeld betalen aan God.

  Te hoog is de losprijs voor zijn leven, en nimmer toereikend,~\sep\ om eeuwig te leven en de dood te ontgaan.

  Want de wijzen ziet hij sterven, ook de dwaas en de domme ziet hij vergaan,~\sep\ en hun rijkdommen aan vreemden achterlaten.

  Het graf is voor eeuwig hun woning, hun verblijf van geslacht tot geslacht,~\sep\ al hebben zij ook landgoederen naar hun naam genoemd.

  Neen, de mens in weelde blijft niet voortbestaan,~\sep\ hij is als het vee, dat vergaat.
\end{halfparskip}

\psalmsubtitle{c) De boze gaat voor eeuwig te gronde}

\begin{halfparskip}
  Dit is de weg van wie als dwazen vertrouwen,~\sep\ en dit is het eind van wie in hun lot zich verlustigen.

  Zij worden als schapen in het dodenrijk geborgen;~\sep\ de dood is hun herder, de rechtvaardigen zijn hun meesters.

  Spoedig gaat hun gedaante voorbij,~\sep\ het dodenrijk zal hun woonplaats zijn.
\end{halfparskip}

\psalmsubtitle{d) De rechtvaardige zal eeuwig leven}

\begin{halfparskip}
  Doch God zal mijn ziel uit het dodenrijk redden,~\sep\ doordat Hij mij tot Zich neemt.

  Maak u geen zorg, als iemand rijk is geworden,~\sep\ als het vermogen van zijn huis is toegenomen;

  Want bij zijn sterven neemt hij niets met zich mee,~\sep\ zijn schatten dalen niet met hem af.

  Al heeft hij zich bij zijn leven gelukkig geprezen:~\sep\ ``Men zal u roemen, omdat gij u te goed deedt.''

  Afdalen zal hij in de kring van zijn vaderen,~\sep\ die in eeuwigheid het licht niet zullen aanschouwen.

  De mens, die onbezonnen en in weelde leeft,~\sep\ is gelijk aan het vee, dat vergaat.
\end{halfparskip}

% % % % % % % % % % % % % % % % % % % % % % % % % % % % % % % % % % % % % % % %

\hulala{7}

\Slota{Wij belijden, aanbidden en verheerlijken U, God der goden en Heer der heren, grote Koning der heerlijkheid, Wezen dat van eeuwigheid is, in alle seizoenen en tijden, Heer van alles...}

\marmita{19}

\PSALMtitle{50}{Het ware offer}

\psalmsubtitle{a) God komt ten oordeel}

\begin{halfparskip}
  God de Heer heeft gesproken en de wereld gedagvaard,~\sep\ van het rijzen der zon tot haar dalen.

  \qanona{Gezegend is Hij die door het offer van Zijn Welbeminde alle offers heeft afgeschaft en beëindigd!}

  Uit Sion, de volschone, straalde God in heerlijkheid.~\sep\ Onze God verscheen en zwijgt niet.

  Verslindend vuur gaat voor Hem uit,~\sep\ en om Hem heen woedt de stormwind.

  Hij dagvaardt de hemel daarboven en ook de aarde,~\sep\ want oordelen gaat Hij Zijn volk.

  ``Verzamelt vóór Mij Mijn heiligen,~\sep\ die door offers het Verbond met Mij sloten.''

  En de hemelen kondigen Zijn gerechtigheid aan,~\sep\ want God zelf is de Rechter.
\end{halfparskip}

\psalmsubtitle{b) Het ware offer is dat van het hart}

\begin{halfparskip}
  ``Luister, Mijn volk, Ik ga spreken, en tegen u getuigen, o Israël,~\sep\ God, uw God ben Ik.

  Niet om uw offers berisp Ik u,~\sep\ want uw brandoffers stijgen steeds voor Mij op.

  Ik zal geen kalf uit uw stal aanvaarden,~\sep\ en uit uw kudden geen bokken.

  Want Mij behoren alle beesten der bossen,~\sep\ de duizenden dieren op Mijn bergen.

  Ik ken alle vogels in de lucht,~\sep\ en wat zich beweegt op het veld, is Mij bekend.

  Had Ik honger, Ik zou het u niet zeggen;~\sep\ want Mij behoort de aarde met al wat zij bevat.

  Of zou Ik het vlees van stieren eten,~\sep\ of drinken het bloed van bokken?

  Breng aan God een offer van lof,~\sep\ en voldoe uw geloften aan de Allerhoogste.

  En roep Mij dan aan op de dag van kwelling:~\sep\ Ik zal u redden, en gij zult Mij vereren.''
\end{halfparskip}

\psalmsubtitle{c) Wee de schijnheiligen!}

\begin{halfparskip}
  Maar tot de zondaar spreekt God:~\sep\ ``Wat praat gij over Mijn wetten, en hebt de mond vol van Mijn Verbond?

  Gij, die de tucht veracht,~\sep\ en u aan Mijn woorden niet hebt gestoord?

  Zaagt gij een dief, gij liept met hem mee,~\sep\ en met echtbrekers gingt gij vertrouwelijk om.

  Uw mond stond open voor boosheid,~\sep\ en uw tong spon enkel bedrog.

  Gij zat neer, en getuigde tegen uw broeder;~\sep\ de zoon van uw moeder bracht gij in schande:

  Dat hebt gij gedaan, en zou Ik dan zwijgen; meent gij, dat Ik ben zoals gij?~\sep\ Hier is Mijn aanklacht en Ik ga ze voor uw ogen ontvouwen.

  Beseft dit wel, gij, die God zijt vergeten,~\sep\ anders roof Ik u weg en kan geen u nog redden.

  Wie een offer brengt van lof, verheerlijkt Mij;~\sep\ en wie in oprechtheid wandelt, hem zal Ik tonen Gods heil.''
\end{halfparskip}

\PSALMtitle{51}{Miserere}

\psalmsubtitle{a) Heer, vergeef mij mijn zonden!}

\begin{halfparskip}
  Ontferm u over mij, O God, naar Uw barmhartigheid;~\sep\ delg toch mijn misdaad uit volgens Uw grote ontferming.

  \qanona{Moge, met de hysop van Uw barmhartigheid, onze vlekken wit worden, O Barmhartige!}

  Zuiver mij geheel van mijn schuld,~\sep\ en reinig mij van mijn zonde.

  Want mijn boosheid erken ik,~\sep\ en mijn zonde staat mij steeds voor ogen.

  Tegen U alleen heb ik gezondigd,~\sep\ en wat kwaad is in Uw ogen, heb ik gedaan:

  Zo zult Gij rechtvaardig in Uw uitspraak blijken,~\sep\ onberispelijk in Uw oordeel.

  Zie, in schuld ben ik geboren,~\sep\ en in zonde ontving mij mijn moeder.

  Zie, in de oprechtheid van het hart schept Gij behagen;~\sep\ en Gij leert mij de wijsheid in het diepst van mijn hart.
\end{halfparskip}

\psalmsubtitle{b) Heer, reinig mijn ziel!}

\begin{halfparskip}
  Besprenkel mij met hysop, en ik zal gereinigd worden;~\sep\ was mij, en ik zal blanker zijn dan sneeuw.

  Laat mij een blijde tijding vernemen,~\sep\ laat juichen mijn gebeente, dat Gij hebt verbrijzeld.

  Wend van mijn zonden Uw aanschijn af,~\sep\ en delg al mijn schulden uit.
\end{halfparskip}

\psalmsubtitle{c) Laat mij U met een nieuw hart dienen!}

\begin{halfparskip}
  Schep in mij een zuiver hart, O God,~\sep\ en vernieuw in mij een standvastige geest.

  Verwerp mij niet van Uw aanschijn,~\sep\ en neem Uw heilige geest niet van mij weg.

  Schenk mij terug de vreugde van Uw heil,~\sep\ en sterk mij door een edelmoedige geest.

  Dan zal ik de bozen Uw wegen doen kennen,~\sep\ en de zondaars zullen zich bekeren tot U.

  Verlos mij van bloedschuld, O God, God, mijn Redder,~\sep\ en dat mijn tong over Uw gerechtigheid juiche.

  Heer, open mijn lippen,~\sep\ en mijn mond zal Uw lof verkondigen.

  Neen, in een slachtoffer schept Gij geen vreugde,~\sep\ en zo ik een brandoffer bracht, Gij zoudt het niet aanvaarden.

  Mijn offer, O God, is een rouwmoedige geest;~\sep\ een berouwvol en vernederd hart zult Gij niet versmaden, O God.

  Volgens Uw goedheid, O Heer, handel genadig met Sion,~\sep\ en bouw weer de muren van Jeruzalem op.

  Dan zult Gij weer wettige offers, dank- en brandoffers aanvaarden;~\sep\ dan zal men op Uw altaar weer varren opdragen.
\end{halfparskip}

\PSALMtitle{52}{Wee gij slechtaard!}

\psalmsubtitle{a) Boosheid van de aanbrenger}

\begin{halfparskip}
  Wat roemt gij op boosheid,~\sep\ gij, overmachtige eerloze?~\sep\ Al maar door zint gij op bedrog;

  \qanona{Gezegend is Hij die de nederigen verheft en de hoogmoedigen vernedert!}

  scherp als een scheermes is uw tong, O leugenmeester.

  Gij stelt het kwade boven het goede,~\sep\ spreekt liever leugens dan de waarheid.

  Slechts woorden, die verderf aanbrengen, zijn u lief,~\sep\ verraderlijke tong!

  Daarom zal God u doen omkomen,~\sep\ u uit de weg ruimen voor immer.

  U wegsleuren uit uw tent,~\sep\ en u met wortel en al uitrukken uit het land der levenden.
\end{halfparskip}

\psalmsubtitle{b) De vromen zullen juichen en God danken}

\begin{halfparskip}
  De rechtvaardigen zullen het zien en huiveren,~\sep\ en zij zullen lachen met hem:

  ``Dat is nu de man, die niet erkende,~\sep\ dat God zijn bescherming is,

  Maar die op zijn grote rijkdom vertrouwde,~\sep\ door zijn misdaden overmachtig werd.''

  Ik echter ben als een bloeiende olijf in Gods huis;~\sep\ op Gods barmhartigheid vertrouw ik voor immer.

  Eeuwig zal ik U loven, omdat Gij dit hebt gedaan; en verheerlijken zal ik Uw Naam,~\sep\ want Hij is goed, voor het aanschijn van Uw heiligen.
\end{halfparskip}

\Slota{Wij smeken U, die de bozen vernietigt, de hoogmoedigen verstrooit, de sterken omverwerpt en (Uw) verbond en genade hebt bewaard voor hen die Uw heilige Naam vrezen, keer U naar ons, O Heer, heb medelijden en ontferm U altijd over ons, zoals U gewend bent, Heer van alles...}

\marmita{20}

\PSALMtitle{53}{Het bederf is groot}

\psalmsubtitle{a) Boos zijn de mensen}

\begin{halfparskip}
  De dwaas zegt bij zich zelf:~\sep\ ``Er is geen God.''~\sep\ Ze zijn bedorven, gruwelen hebben ze bedreven;~\sep

  \qanona{Gij, die alles weet, red Uw Kerk.}

  daar is er niet één, die deugdzaam handelt.

  God blikt uit de hemel neer op de kinderen der mensen,~\sep\ om te zien of er wel één is met verstand, wel één, die God zoekt.

  Maar allen zonder uitzondering zijn ze afgedwaald, allen diep bedorven;~\sep\ er is er niet één die deugdzaam handelt, niet één.
\end{halfparskip}

\psalmsubtitle{b) Gij, Heer, straft de bozen}

\begin{halfparskip}
  Zullen die bozen dan nimmer tot inzicht komen, zij die Mijn volk verslinden als aten ze brood;~\sep\ roepen zij dan God niet aan?

  Zij sidderden van angst,~\sep\ waar niets viel te vrezen;

  Want God heeft de beenderen verstrooid van die u bestookten;~\sep\ ze werden ontsteld, want God heeft hen verworpen.

  O, mocht er uit Sion toch heil voor Israël dagen! Als God het lot van Zijn volk ten goede keert,~\sep\ zal er gejubel zijn in Jacob en vreugde in Israël!
\end{halfparskip}

\PSALMtitle{54}{Vertrouwvol gebed}

\psalmsubtitle{a) Heer, schenk mij Uw hulp!}

\begin{halfparskip}
  Red mij, o God, door Uw Naam,~\sep\ en treed in Uw kracht voor mijn rechtszaak op!

  \qanona{Gij zijt mijn ware Hoop.}

  Luister naar mijn bede, O God,~\sep\ hoor naar de woorden van mijn mond!

  Want trotsaards zijn tegen mij opgestaan, en geweldenaars stonden mij naar het leven;~\sep\ zij hielden God niet voor ogen.
\end{halfparskip}

\psalmsubtitle{b) U, mijn Redder, zij lof!}

\begin{halfparskip}
  Zie, God komt mij te hulp,~\sep\ de Heer behoudt mijn leven.

  Wend op mijn vijanden de rampen af,~\sep\ vernietig hen omwille van Uw trouw.

  Van harte wil ik U offers brengen;~\sep\ Uw Naam zal ik prijzen, O Heer, want Hij is goed.

  Want Hij heeft mij verlost uit alle nood,~\sep\ en mijn oog zag mijn vijanden beschaamd staan.
\end{halfparskip}

\PSALMtitle{55}{Verraad en bedrog}

\psalmsubtitle{a) Bevrijd mij, O God, van mijn ellende}

\begin{halfparskip}
  Luister, O God, naar mijn bede, en wend U niet af van mijn smeken;~\sep\ geef acht op mij en schenk mij verhoring.

  \qanona{Aan U, God, vertrouw ik mijn rechtgeding (\translationoptionNl{oordeel}) toe.}

  Ik sidder van angst en word ontsteld~\sep\ bij het razen van de vijand, en het geschreeuw van de zondaar.

  Want zij storten rampen over mij uit,~\sep\ en vallen mij grimmig aan.

  Mijn hart is ontsteld in mijn binnenste,~\sep\ en de angst van de dood overvalt mij.

  Vrees en siddering storten op mij neer,~\sep\ en ontzetting grijpt mij aan.

  En ik zeg: Had ik maar vleugelen als de duif,~\sep\ dan vloog ik heen en vond ik rust.

  Ja, ver, ver zou ik heenvluchten,~\sep\ verwijlen in de woestijn.

  Spoedig zou Ik mij een schuilplaats zoeken,~\sep\ tegen stormwind en orkaan.
\end{halfparskip}

\psalmsubtitle{b) Verneder, Heer, mijn trouweloze vriend!}

\begin{halfparskip}
  Verstrooi ze, Heer, verwar hun spraak,~\sep\ want in de stad zie ik geweld en twist.

  Dag en nacht doen zij de ronde op haar wallen,~\sep\ en daarbinnen heerst boosheid en verdrukking.

  Hinderlagen legt men er,~\sep\ en onrecht en bedrog wijken niet van haar straten.

  Och, was het slechts mijn vijand, die mij had gehoond,~\sep\ voorzeker, ik had het verdragen;

  Of was, die mij haatte, tegen mij opgestaan,~\sep\ ik had mij voor hem verborgen.

  Maar gij waart het, mijn disgenoot,~\sep\ mijn vriend en vertrouweling,

  Met wie ik zo gemoedelijk omging;~\sep\ in het huis van God trokken we samen met de feeststoet op.
\end{halfparskip}

\psalmsubtitle{c) Verdelg de vijand, Heer; ik vertrouw op U}

\begin{halfparskip}
  Dat de dood op hen aanstorme, dat zij levend neerdalen in het dodenrijk,~\sep\ want boosheid woont in hun huizen en in hun binnenste!

  Ik echter zal roepen tot God,~\sep\ en de Heer zal mij redden.

  's Avonds en 's morgens en 's middags zal ik weeklagen en jammeren,~\sep\ en mijn stem zal Hij horen.

  Mijn leven zal Hij in veiligheid brengen tegen hen, die mij bestoken:~\sep\ want velen staan tegen mij op.

  God zal het horen, en Die van eeuwigheid regeert, hen neerdrukken;~\sep\ want onverbeterlijk zijn ze en ze vrezen God niet.

  Ieder heft zijn hand op tegen zijn vrienden,~\sep\ en schendt zijn verbond.

  Gladder dan boter is hun gelaat,~\sep\ maar hun hart wil strijd;

  Zachter dan olie zijn hun woorden,~\sep\ maar het zijn getrokken zwaarden.

  Werp Uw zorg op de Heer, en Hij zal Uw steun zijn:~\sep\ nooit zal Hij dulden dat de rechtvaardige wankelt.

  Maar hen, O God, zult Gij neerstorten,~\sep\ in de afgrond van verderf.

  Die bloed vergieten en bedriegen, zullen de helft van hun dagen niet halen;~\sep\ ik echter hoop op U, O Heer.
\end{halfparskip}

\Slota{Wees ons genadig, onze Heer en onze God, vergeef, wis en scheld kwijt onze overtredingen in de overvloedige genade van Uw barmhartigheid, U, Lankmoedige, die schulden en zonden vergeeft, Heer van alles...}

\marmita{21}

\PSALMtitle{56}{Vertrouwvol gebed}

\psalmsubtitle{a) Vertrouwvolle bede tegen vijanden}

\begin{halfparskip}
  Wees mij genadig, O God, want de mensen vertrappen mij;~\sep\ zij bestrijden en verdrukken mij al maar door.

  \qanona{Laat mij niet in de steek, Heer, omwille van het bedrog van verraders.}

  Voortdurend vertreden mij mijn vijanden,~\sep\ want velen zijn het, die tegen mij strijden.

  Allerhoogste, als vrees mij bevangt,~\sep\ stel ik toch mijn vertrouwen op U.

  Op God, wiens belofte ik verheerlijk, op God vertrouw ik, ik zal niet vrezen:~\sep\ wat kan een mens mij aandoen?

  De hele dag kleineren ze mij,~\sep\ tegen mij zijn al hun gedachten ten kwade gericht.

  Zij komen tezamen, belagen mij,~\sep\ gaan mijn gangen na en staan mij naar het leven.

  Zet ze hun boosheid betaald,~\sep\ werp, O God, de volkeren neer in Uw toorn.

  De wegen van mijn ballingschap zijn U bekend; geborgen zijn mijn tranen in Uw lederen zak;~\sep\ zijn ze in Uw boek niet opgetekend?

  Dan zullen mijn vijanden terug moeten wijken, telkens wanneer ik U aanroep;~\sep\ dit weet ik goed: God is met mij.

  Op God, wiens belofte ik verheerlijk, op God vertrouw ik, ik zal niet vrezen:~\sep\ wat kan een mens mij aandoen?
\end{halfparskip}

\psalmsubtitle{b) Heb dank, O Heer}

\begin{halfparskip}
  Ik ben gehouden, O God, door geloften, die ik U heb gedaan:~\sep\ inlossen zal ik U de offers van lof,

  Want Gij hebt mijn leven ontrukt aan de dood, en mijn voet bewaard voor de val,~\sep\ opdat ik wandele voor God in het licht der levenden.
\end{halfparskip}

\PSALMtitle{57}{Godsvertrouwen}

\psalmsubtitle{a) Gij zult mij redden, O God!}

\begin{halfparskip}
  Wees mij genadig, O God, wees mij genadig,~\sep\ want mijn ziel neemt haar toevlucht tot U,~\sep\ En ik schuil in de schaduw van Uw vleugels,~\sep

  \qanona{Mijn Heer en mijn God, bescherm mij van het tumult der mensen!}

  totdat het onheil voorbijgaat.

  Ik roep tot God, de Allerhoogste,~\sep\ tot God, die mij weldoet.

  Hij zende uit de hemel mij redding, en overlade mijn vervolgers met smaad;~\sep\ God zende Zijn liefde en Zijn trouw.

  Ik lig neer te midden van leeuwen,~\sep\ die gulzig de kinderen der mensen verslinden.

  Hun tanden zijn lansen en pijlen,~\sep\ een vlijmend zwaard is hun tong.

  Verschijn hoog boven de hemelen, O God,~\sep\ dat Uw glorie strale over heel de aarde.
\end{halfparskip}

\psalmsubtitle{b) Dank, Heer, voor mijn bevrijding!}

\begin{halfparskip}
  Zij spanden een net voor mijn voeten,~\sep\ en drukten mij neer;

  Zij groeven een kuil voor mij;~\sep\ dat zij zelf er in vallen!

  Gesterkt is mijn hart, O God, gesterkt is mijn hart;~\sep\ ik zal zingen en het psalter bespelen.

  Ontwaak toch, mijn ziel, gij, psalter en citer, ontwaakt;~\sep\ ik wil de dageraad wekken.

  Onder de volken wil ik U prijzen, O Heer,~\sep\ U bezingen onder de naties met snarenspel.

  Want hoog tot de hemel reikt Uw erbarming,~\sep\ en tot de wolken Uw trouw.

  Verschijn hoog boven de hemelen, O God,~\sep\ en dat Uw glorie strale over heel de aarde.
\end{halfparskip}

\PSALMtitle{58}{Onrechtvaardige rechters}

\psalmsubtitle{a) Boosheid der rechters}

\begin{halfparskip}
  Spreekt gij nu waarlijk recht, gij machtigen,~\sep\ oordeelt gij rechtvaardig, gij mensenkinderen?

  \qanona{God beschermt de rechtvaardigen en verwerpt de bozen.}

  Veeleer pleegt gij ongerechtigheid in uw hart;~\sep\ uw handen zaaien onrecht in het land.

  Reeds van de moederschoot weken de goddelozen af van het pad,~\sep\ reeds van hun geboorte dwaalden de leugenaars.

  Hun venijn is als slangengif,~\sep\ als het gif van een dove adder, die haar oren stopt,

  Om de stem niet te horen van de bezweerders,~\sep\ noch de toverban van de bedreven belezer.
\end{halfparskip}

\psalmsubtitle{b) Straf der onrechtvaardige rechters}

\begin{halfparskip}
  O God, verbrijzel de tanden in hun mond,~\sep\ sla de leeuwenkiezen stuk, O Heer.

  Laat ze verdwijnen als water, dat wegvloeit;~\sep\ mogen hun pijlen afstompen, wanneer zij richten.

  Dat ze vergaan als de slak, die wegsmelt,~\sep\ als een misdracht, die het zonlicht niet heeft gezien.

  Eer Uw ketels de doornstruik voelen,~\sep\ terwijl deze nog groen is, vage de gloeiende wervelwind hem weg.

  De rechtvaardige zal zich verblijden bij het zien van de wraak;~\sep\ zijn voeten zal hij wassen in het bloed van de boze.

  En de mensen zullen zeggen: Ja waarlijk, de rechtschapene plukt zijn vruchten;~\sep\ waarlijk, er is nog een God, die rechtspreekt op aarde.
\end{halfparskip}

% % % % % % % % % % % % % % % % % % % % % % % % % % % % % % % % % % % % % % % %

\hulala{8}

\Slota{Bevrijd ons, onze Heer en onze God, van de listen en strikken van de afvallige vijand door Uw grote en sterke macht, en door Uw hoge en onoverwinnelijke arm, U die goed bent en op wie ons vertrouwen is gesteld, in alle tijden en seizoenen, Heer van alles...}

\marmita{22}

\PSALMtitle{59}{Tegen onruststokers}

\psalmsubtitle{a) Verlos mij, O God, van de bozen!}

\begin{halfparskip}
  Ontruk mij, mijn God, aan mijn vijanden,~\sep\ en behoed mij voor hen, die tegen mij opstaan.

  \qanona{De goddelozen hebben mij onderdrukt en mishandeld! Verlos mij, mijn Heer en mijn God.}

  Bevrijd mij van hen, die onrecht plegen,~\sep\ en red mij van mannen, die bloed vergieten.

  Want zie, ze staan mij naar het leven,~\sep\ de machtigen spannen tegen mij samen.

  Geen misdrijf, Heer, geen zonde is er in mij;~\sep\ buiten mijn schuld rukken zij op en vallen zij aan.

  Ontwaak, snel mij te hulp, en zie toe,~\sep\ want, O Heer der legerscharen, Gij zijt Israëls God.

  Ontwaak en tuchtig al de heidenvolkeren;~\sep\ heb geen medelijden met al die trouwelozen.

  's Avonds keren zij terug, ze blaffen als honden,~\sep\ en zwerven rond in de stad.

  Zie, grootspraak is in hun mond en smaad op hun lippen:~\sep\ ``Wie immers hoort het?''

  Maar Gij, Heer, lacht hen uit,~\sep\ Gij drijft de spot met al de heidenen.

  Mijn kracht, U schenk ik mijn aandacht, want Gij, O God, zijt mijn Bescherming,~\sep\ mijn God en mijn Barmhartigheid.
\end{halfparskip}

\psalmsubtitle{b) Op U, O God, vertrouw ik}

\begin{halfparskip}
  Dat God mij helpe,~\sep\ mij doe juichen over mijn vijanden.

  Dood ze, O God, opdat ze mijn volk geen aanstoot geven;~\sep\ breng ze in verwarring door Uw kracht en vel ze neer, Gij, ons Schild, O Heer.

  Een zonde van hun mond zijn de woorden van hun lippen;~\sep\ dat ze in hun trots worden verstrikt, in de lastertaal en leugens, die zij spreken.

  Verdelg hen in Uw toorn, verdelg hen zo, dat zij niet meer bestaan,~\sep\ opdat men wete, dat God heerst in Jacob en tot aan de grenzen der aarde.

  's Avonds keren zij terug, ze blaffen als honden,~\sep\ en zwerven rond in de stad;

  Ze dolen rond om voedsel te zoeken,~\sep\ en huilen, als ze niet zijn verzadigd.

  Maar ik, ik zal Uw macht bezingen,~\sep\ en 's morgens jubelen over Uw barmhartigheid,

  Want Gij zijt mij tot bescherming geworden,~\sep\ en tot toevlucht op de dag van mijn kwelling.

  Mijn kracht, voor U zal ik het psalter bespelen, want Gij, O God, zijt mijn Bescherming,~\sep\ mijn God en mijn Barmhartigheid.
\end{halfparskip}

\PSALMtitle{60}{Houd moed}

\psalmsubtitle{a) Uw volk verslagen, O God, Uw land verwoest!}

\begin{halfparskip}
  O God, Gij hebt ons verstoten, onze gelederen verbroken:~\sep\ Gij zijt vertoornd: herstel ons weer!~\sep\ Het land hebt Gij geschokt, Gij hebt het gescheurd:~\sep

  \qanona{Red hen die U aanbidden, Christus, volgens Uw belofte!}

  herstel zijn kloven, want het wankelt.

  Uw volk hebt Gij harde dingen opgelegd,~\sep\ ons bedwelmende wijn doen drinken.

  Voor die U vrezen hebt Gij een banier geheven,~\sep\ om er te vluchten voor de boog;

  Opdat Uw geliefden zouden worden bevrijd,~\sep\ help ons door Uw rechterhand, en verhoor ons.
\end{halfparskip}

\psalmsubtitle{b) Uw beloften, Heer, zijn mijn hoop}

\begin{halfparskip}
  God heeft gesproken in Zijn heiligdom:~\sep\ ``Ik zal juichen en Sichem verdelen, en het dal van Succoth meten.

  Van Mij is het land Galaäd, van Mij het land Manasse,~\sep\ Efraïm is de helm van Mijn hoofd en Juda Mijn scepter;

  Moab is Mijn wasbekken, op Edom werp Ik Mijn schoeisel,~\sep\ over Filistea zal Ik zegevieren.''

  Wie zal mij binnenvoeren in de versterkte stad,~\sep\ wie mij naar Edom geleiden?

  Zijt Gij het niet, O God, die ons hebt verstoten,~\sep\ die niet meer uittrekt, O God, met onze legerscharen?

  Schenk ons Uw hulp tegen de vijand,~\sep\ want ijdel is de hulp van mensen.

  Met God zullen wij dapper strijden,~\sep\ en Hij zelf zal onze vijanden vertreden.
\end{halfparskip}

\PSALMtitle{61}{Gebed van een balling}

\psalmsubtitle{a) Geleid mij, Heer, naar Uw heiligdom}

\begin{halfparskip}
  Luister, O God, naar mijn smeken,~\sep\ geef acht op mijn bede.

  Van het uiteinde der aarde roep ik tot U,~\sep\ wanneer mijn hart het begeeft.

  \qanona{Gij zijt overal, God, hoor onze beden.}

  Gij zult mij verheffen op de rots, mij rust verschaffen,~\sep\ want Gij zijt mij tot bescherming, een sterke Toren tegen de vijand.

  Mocht ik toch immer wonen in Uw tent,~\sep\ en schuilen onder de dekking van Uw vleugels!
\end{halfparskip}

\psalmsubtitle{b) Laat mij eeuwig heersen, O heer!}

\begin{halfparskip}
  Gij toch, O God, hebt mijn geloften aanhoord,~\sep\ mij het erfdeel gegeven van die Uw Naam vrezen.

  Vermenigvuldig de dagen van de koning,~\sep\ mogen zijn jaren gelijk zijn aan vele geslachten.

  Eeuwig moge hij tronen voor Gods aanschijn;~\sep\ zend tot zijn behoud Uw genade en trouw.

  Zo zal ik Uw Naam voor immer bezingen,~\sep\ en te allen tijde mijn geloften inlossen.
\end{halfparskip}

\Slota{Op U, mijn Heer, wachten onze zielen, naar Uw mededogen kijken de ogen van ons hart, en van U vragen wij vergeving voor onze overtredingen. Schenk ons dit altijd in Uw genade en barmhartigheid, zoals U gewend bent, Heer van alles...}

\marmita{23}

\PSALMtitle{62}{In God is slechts rust}

\psalmsubtitle{a) Op U alleen, O God, vertrouw ik}

\begin{halfparskip}
  In God alleen rust mijn ziel,~\sep\ van Hem komt mijn heil.~\sep\ Hij alleen is mijn Rots en mijn Heil,~\sep

  \qanona{Ik onderwerp mij aan God, want Hij is mijn ware Hoop.}

  mijn Bescherming: neen, ik zal niet wankelen.

  Hoe lang stormt gij op één man los, werpt gij hem met u allen omver,~\sep\ als een hellende wand, als een wankelende muur?

  Ja waarlijk, zij beramen hoe zij mij van mijn verheven plaats zullen stoten;~\sep\ zij verlustigen zich in de leugens;

  Ze zegenen met hun mond,~\sep\ maar met hun hart vervloeken zij.

  Rust in God alleen, mijn ziel,~\sep\ want van Hem komt wat ik hoop.

  Hij alleen is mijn Rots en mijn Heil, mijn Bescherming:~\sep\ neen, ik zal niet wankelen.

  Bij God is mijn heil en mijn roem,~\sep\ mijn sterke rots: in God is mijn schuilplaats.

  Hoop op Hem, O volk, te allen tijde: stort voor Hem uw harten uit:~\sep\ God is ons tot toevlucht.
\end{halfparskip}

\psalmsubtitle{b) IJdel is de hulp van mensen}

\begin{halfparskip}
  Een ademtocht slechts zijn de kinderen der mensen,~\sep\ bedrieglijk de zonen van mannen.

  Zij rijzen op de weegschaal omhoog:~\sep\ lichter dan een zucht zijn zij allen tezamen.

  Verwacht niets van verdrukking, en roem niet ijdel op roof;~\sep\ als uw vermogen toeneemt, wilt er uw hart niet aan hechten.

  Eén zaak heeft God gezegd; deze twee dingen vernam ik: ``Bij God berust de macht, en bij U de goedheid, O  Heer,~\sep\ want een ieder zult Gij naar werk vergelden.
\end{halfparskip}

\PSALMtitle{63}{Drang naar God}

\psalmsubtitle{a) Naar U, O God, verlang ik}

\begin{halfparskip}
  God, mijn God zijt Gij:~\sep\ met aandrang zoek ik U.~\sep\ Naar U dorst mijn ziel, naar U smacht mijn lichaam,~\sep

  \qanona{In Uw barmhartigheid, dat beter is dan het leven, heb medelijden met mij, O Barmhartige!}

  als een dor en dorstig, waterloos land.

  Zo blijf ik U beschouwen in Uw heiligdom,~\sep\ om Uw macht te zien en Uw glorie.

  Daar Uw genade meer dan het leven geldt,~\sep\ zullen mijn lippen U loven.
\end{halfparskip}

\psalmsubtitle{b) Op U, O God, vertrouw ik}

\begin{halfparskip}
  Zo wil ik U prijzen mijn leven lang,~\sep\ in Uw Naam mijn handen verheffen.

  Als met merg en vet zal ik verzadigd worden,~\sep\ en met jubelende lippen zal mijn mond U loven.

  Als ik aan U denk op mijn legerstede,~\sep\ blijf ik in de nachtwaken peinzen over U.

  Want Gij zijt mijn Helper geworden,~\sep\ en ik juich in de schaduw van Uw vleugels.

  Mijn ziel hecht zich aan U,~\sep\ Uw rechterhand is mij tot steun.
\end{halfparskip}

\psalmsubtitle{c) Mijn vijanden, Heer, komen om}

\begin{halfparskip}
  Maar zij, die mij naar het leven staan,~\sep\ zullen verzinken in de diepten der aarde.

  Ze zullen vallen in de macht van het zwaard,~\sep\ en de prooi van vossen worden.

  Maar de koning zal zich verblijden in God; een ieder, die bij Hem zweert, zal roemen,~\sep\ omdat de mond van de lasteraars zal worden gestopt.
\end{halfparskip}

\PSALMtitle{64}{Godsvertrouwen tegen valse vijanden}

\psalmsubtitle{a) Sluwe plannen der vijanden}

\begin{halfparskip}
  Luister, O God, als ik klaag, naar mijn stem;~\sep\ behoed mijn leven voor de schrik van de vijand.

  \qanona{Bescherm mij, mijn Heer en mijn God, van de slechtheid van de sluwe.}

  Bescherm mij tegen de samenscholing van bozen,~\sep\ tegen het woelen van hen, die onrecht bedrijven.

  Die hun tongen scherpen als een zwaard,~\sep\ als pijlen hun giftige woorden richten,

  Om vanuit een schuilplaats de onschuldige te treffen,~\sep\ hem onverhoeds te treffen, zonder iets te duchten.

  Vastberaden smeden zij boze plannen, en werken samen om heimelijk strikken te spannen;~\sep\ zij zeggen: ``Wie slaat er acht op ons?''

  Zij denken misdaden uit, verbergen hun weloverwogen gedachten;~\sep\ een afgrond is hun geest en hart.
\end{halfparskip}

\psalmsubtitle{b) God zal ingrijpen}

\begin{halfparskip}
  Maar God treft hen met Zijn pijlen:~\sep\ onverhoeds worden zij met wonden geslagen,

  En hun eigen tong brengt hun verderf;~\sep\ allen, die hen zien, schudden het hoofd.

  En allen zijn vol ontzag en prijzen Gods werk,~\sep\ en overwegen Zijn daden.

  De rechtvaardige juicht in de Heer en vlucht tot Hem,~\sep\ en allen roemen, die oprecht van harte zijn.
\end{halfparskip}

\Slota{Aan U past lof in de door U uitverkoren Kerk; aan U is belijdenis verschuldigd in het glorierijke Sion; en aan U komt aanbidding toe in Uw verheven woonplaats, Heer van alles...}

\marmita{24}

\PSALMtitle{65}{Danklied}

\psalmsubtitle{a) Dank, O God, voor Uw gaven}

\begin{halfparskip}
  Aan U, O God, komt een lofzang toe in Sion;~\sep\ men volbrenge zijn gelofte aan U, die de bede verhoort.

  \qanona{Het zijn niet onze handen die onze wegen voorspoedig maken, Christus onze Verlosser, want Gij zijt de Schepper (\translationoptionNl{Hersteller}) van onze werken door Uw macht en wijsheid.}

  Tot U komt alle vlees,~\sep\ omwille der ongerechtigheden.

  Onze misdaden drukken ons neer:~\sep\ Gij scheldt ze kwijt.

  Gelukkig die Gij uitkiest en tot U neemt:~\sep\ hij woont in Uw voorhoven.

  Dat wij verzadigd worden met de goederen van Uw huis,~\sep\ met de heiligheid van Uw tempel.
\end{halfparskip}

\psalmsubtitle{b) Gij zijt machtig, O God!}

\begin{halfparskip}
  Met gerechtigheid verhoort Gij ons door wondere tekenen,~\sep\ God, onze Redder,

  Gij zijt de Hoop van alle grenzen der aarde,~\sep\ en van de verre zeeën;

  Die de bergen vastlegt door Uw kracht,~\sep\ die met macht zijt omgord,

  Die het bulderen der zee bedwingt,~\sep\ het bulderen van haar golven en het woelen der naties.

  En die de grenzen der aarde bewonen, huiveren om Uw tekenen:~\sep\ met vreugde vervult Gij het uiterste oosten en westen.
\end{halfparskip}

\psalmsubtitle{c) Gij, Heer, maakt de aarde vruchtbaar}

\begin{halfparskip}
  Gij hebt de aarde bezocht en haar besproeid,~\sep\ met rijkdommen haar overstelpt.

  De stroom van God is met water gevuld: Gij hebt hun graan bereid;~\sep\ zo hebt Gij haar gereed gemaakt:

  Haar voren hebt Gij besproeid,~\sep\ geëffend haar kluiten,

  Door stortregens hebt Gij haar geweekt,~\sep\ en haar gewas gezegend.

  Met Uw mildheid hebt Gij het jaar gekroond,~\sep\ en Uw wegen druipen van vet.

  De weiden der woestijn druipen ervan,~\sep\ en de heuvelen omgorden zich met jubel.

  De weiden zijn met kudden bekleed en de dalen met koren bedekt:~\sep\ zij juichen U toe en zingen!
\end{halfparskip}

\PSALMtitle{66}{Dank voor redding}

\psalmsubtitle{a) Looft God!}

\begin{halfparskip}
  Juicht God toe, alle landen, bezingt de glorie van Zijn Naam,~\sep\ heft voor Hem een heerlijk loflied aan.

  \qanona{Eer aan U, O onze Schepper, die ons 's nachts rust gaf, ons beschermde, en ons 's morgens wekte, zodat we Uw wonderen in het licht konden zien!}

  Zegt tot God: Hoe ontzagwekkend zijn Uw werken!~\sep\ Om Uw geweldige kracht brengen Uw vijanden U
  vleiend hulde.

  Dat heel de aarde U aanbidde en voor U zinge,~\sep\ dat zij bezinge Uw Naam.
\end{halfparskip}

\psalmsubtitle{b) Dank, Heer, voor de redding!}

\begin{halfparskip}
  Komt en ziet de werken van God:~\sep\ wondere daden volbracht Hij onder de kinderen der mensen!

  De zee veranderde Hij in land, en zij trokken te voet door de stroom;~\sep\ laten wij daarom over Hem ons verheugen.

  Eeuwig heerst Hij door Zijn macht; Zijn ogen slaan de volken gade:~\sep\ dat de weerspannigen zich niet verheffen.

  Zegent, gij volken, onze God,~\sep\ en verkondigt Zijn wijd verbreide lof.

  Hij behield ons in leven,~\sep\ en liet onze voeten niet wankelen.

  Want Gij hebt ons beproefd, O God,~\sep\ met vuur ons gelouterd, zoals men zilver loutert.

  Gij liet ons de strik inlopen,~\sep\ een zware last hebt Gij op onze heupen gelegd.

  Mensen liet Gij over onze hoofden treden; wij zijn door vuur en water gegaan,~\sep\ maar Gij hebt ons uitkomst gebracht.
\end{halfparskip}

\psalmsubtitle{c) Heer, ik wil U dankoffers brengen}

\begin{halfparskip}
  Met brandoffers wil ik Uw huis betreden,~\sep\ U mijn geloften inlossen,

  Die mijn lippen hebben uitgesproken,~\sep\ en mijn mond heeft beloofd in mijn kwelling.

  Brandoffers wil ik U brengen van vette schapen met het vet van rammen,~\sep\ runderen en bokken zal ik offeren.

  Komt en hoort, gij allen, die God vreest,~\sep\ ik wil U verhalen hoe grote dingen Hij aan mij gedaan heeft!

  Ik riep Hem aan met mijn mond,~\sep\ en prees Hem met mijn tong.

  Had ik in mijn hart op boosheid gezonnen,~\sep\ dan had de Heer mij niet verhoord.

  Maar God heeft mij verhoord,~\sep\ heeft gelet op de stem van mijn smeken.

  Gezegend zij God, die mijn bede niet heeft versmaad,~\sep\ mij Zijn ontferming niet heeft onthouden.
\end{halfparskip}

\PSALMtitle{67}{Lof en dank}

\psalmsubtitle{Heer, zegen ons!}

\begin{halfparskip}
  God zij ons genadig en zegene ons;~\sep\ Hij tone ons Zijn vredig gelaat.

  \qanona{Christus, die talenten van geestelijk zilver gaf aan Uw dienaren, verleen hulp aan hen die U aanbidden en die Uw gift hebben ontvangen.}

  Opdat men op aarde Zijn weg lere kennen,~\sep\ onder alle volken Zijn heil.

  Dat de volken U prijzen, O God,~\sep\ dat alle volken U prijzen!
\end{halfparskip}

\psalmsubtitle{b) Dat alle volken U prijzen!}

\begin{halfparskip}
  Laat juichen en jubelen de naties, omdat Gij met rechtvaardigheid de volken regeert,~\sep\ en de naties op aarde bestuurt.

  Dat de volken U prijzen, O God,~\sep\ dat alle volken U prijzen!

  De aarde heeft haar vrucht gegeven;~\sep\ God, onze God, heeft ons gezegend.

  Dat God ons zegene,~\sep\ en dat alle grenzen der aarde Hem vrezen!
\end{halfparskip}

% % % % % % % % % % % % % % % % % % % % % % % % % % % % % % % % % % % % % % % %

\hulala{9}

\Slota{Mogen de eeuwige barmhartigheid van Uw glorierijke Drie-eenheid, O onze Heer en onze God, altijd kome over Uw zondige en zwakke dienaren, die U aanroepen en U te allen tijde smeken, Heer van alles...}

\marmita{25}

\PSALMtitle{68}{Gods triomftocht naar Sion}

\psalmsubtitle{a) Aanvang der dankprocessie}

\begin{halfparskip}
  God rijst op: Zijn vijanden stuiven uiteen,~\sep\ en die Hem haten, vluchten weg voor Zijn aanschijn.

  \qanona{De tijd is aangebroken dat de afgoden worden uitgeroeid en dat de ene God, de Heer van alles, wordt aanbeden!}

  Zij verdwijnen, zoals rook verdwijnt;~\sep\ zoals was wegsmelt bij vuur, zo vergaan de zondaars voor Gods aanschijn.

  Maar de rechtvaardigen juichen, springen op voor het aanschijn van God,~\sep\ zijn opgetogen van vreugde.

  Zingt God toe, tokkelt het psalter voor Zijn Naam;~\sep\ baant een weg voor Hem, die voorttrekt door de woestijn,

  Wiens naam is: de Heer,~\sep\ en juicht voor Zijn aanschijn!

  Vader der wezen en Beschermer der weduwen,~\sep\ is God in Zijn heilige woonstede.

  Voor verlatenen bereidt God een woning, gevangenen brengt Hij tot voorspoed;~\sep\ slechts de weerspannigen blijven achter in het verdorde land.
\end{halfparskip}

\psalmsubtitle{b) Triomftocht door de woestijn: manna en kwartels}

\begin{halfparskip}
  Bij Uw uittocht, O God, aan het hoofd van Uw volk,~\sep\ bij Uw optrekken door de woestijn,

  Beefde de aarde en dropen de hemelen voor het aanschijn van God;~\sep\ de Sinaï sidderde voor God, de God van Israël.

  Milde regen hebt Gij, O God, over Uw erfdeel uitgestort;~\sep\ en als het was uitgeput, hebt Gij het verkwikt.

  Uw kudde heeft er gewoond;~\sep\ in Uw goedheid, O God, hebt Gij het de arme bereid.
\end{halfparskip}

\psalmsubtitle{c) Triomftocht door het Beloofde Land}

\begin{halfparskip}
  De Heer doet een uitspraak:~\sep\ en groot is de menigte, die blijde dingen meldt:

  ``De vorsten der legerscharen vluchten, vluchten:~\sep\ en de huisgenoten verdelen de buit.

  Terwijl gij rustte in de stallen der kudden, schitterden van zilver de vleugels der duif,~\sep\ en haar pennen in goudgele glans.

  Terwijl er de Almachtige de vorsten verstrooide,~\sep\ viel er sneeuw op de Salmon.''
\end{halfparskip}

\psalmsubtitle{d) De Sionsberg uitverkoren}

\begin{halfparskip}
  Hoge bergen zijn de bergen van Basan,~\sep\ steile bergen zijn de bergen van Basan.

  Waarom ziet gij afgunstig, gij, steile bergen, naar de berg, waar het God behaagd heeft te wonen,~\sep\ ja, waar de Heer zelfs voor immer zal wonen?

  De strijdwagens van God zijn myriaden in aantal, duizendmaal duizend;~\sep\ van de Sinaï af trekt de Heer naar Zijn heiligdom.

  Gij hebt de hoogte bestegen, gevangenen meegevoerd, mensen als gaven ontvangen,~\sep\ zelfs hen, die weigeren bij God de Heer te wonen.
\end{halfparskip}

\psalmsubtitle{e) God zal de vijand verslaan}

\begin{halfparskip}
  Geprezen zij de Heer, dag aan dag;~\sep\ onze lasten draagt God, ons heil!

  Onze God is een God, die redding brengt,~\sep\ en de Heer God schenkt uitkomst in doodsgevaar.

  Waarlijk, God verbrijzelt de hoofden van Zijn vijanden,~\sep\ de ruige schedel van hem, die de weg der misdaad bewandelt.

  De Heer heeft gesproken: ``Ik haal ze terug uit Basan,~\sep\ ik haal ze op uit de diepte der zee,

  Opdat ge uw voet moogt dopen in bloed,~\sep\ en de tong van uw honden haar deel aan de vijanden heeft!''
\end{halfparskip}

\psalmsubtitle{f) Triomftocht naar het heiligdom}

\begin{halfparskip}
  Zij zien Uw intrede, O God,~\sep\ de intrede van mijn God, mijn Koning, in het heiligdom:

  Voorop gaan de zangers, de harpspelers aan het einde,~\sep\ in het midden slaan de meisjes de pauken.

  ``Looft God in uw feestelijke bijeenkomsten,~\sep\ de Heer, gij, die uit Israël geboren zijt!''

  Daar treedt Benjamin aan, de jongste; hij gaat voor hen uit; dan volgen de vorsten van Juda met hun scharen,~\sep\ de vorsten van Zabulon, de vorsten van Neftali.
\end{halfparskip}

\psalmsubtitle{g) Heer, onderwerp de volkeren!}

\begin{halfparskip}
  Ontplooi Uw macht, O God,~\sep\ Uw macht, O God, Gij, die werkt voor ons!

  Omwille van Uw tempel, die in Jeruzalem staat,~\sep\ mogen de vorsten U gaven brengen!

  Bedwing het wilde beest in het riet,~\sep\ de troep stieren met de runderen der volken;

  Dat zij zich ter aarde werpen met platen van zilver:~\sep\ verstrooi de volken, die zich verlustigen in oorlog.

  Laat de machtigen uit Egypte zich aandienen,~\sep\ dat Ethiopië de handen uitstrekke naar God.
\end{halfparskip}

\psalmsubtitle{h) Volkeren, looft de Heer!}

\begin{halfparskip}
  Koninkrijken der aarde, zingt voor God, bespeelt het psalter voor de Heer,~\sep\ die door de hemelen, de aloude hemelen rijdt!

  Zie, Zijn stem laat Hij horen, Zijn machtige stem:~\sep\ ``Erkent de macht van God'';

  Zijn Majesteit strekt zich uit over Israël,~\sep\ en in de wolken straalt Zijn macht.

  Ontzagwekkend is God in Zijn heiligdom, Hij, de God van Israël. Hij zelf schenkt macht en kracht aan Zijn volk.~\sep\ Geprezen zij God!
\end{halfparskip}

\Slota{Red, mijn Heer, Uw volk van de ondergang en vernietiging, trek onze zielen uit de stormen van de zonde. Moge Uw waarheid ons steunen en doe ons wandelen op de paden van gerechtigheid, alle dagen van ons leven, Heer van alles...}

\marmita{26}

\PSALMtitle{69}{Klacht in nood}

\psalmsubtitle{a) Groot, Heer, is mijn ellende}

\begin{halfparskip}
  Red mij, O God,~\sep\ want de wateren zijn tot aan mijn hals gestegen.

  \qanona{Christus, ontferm U over mij!}

  Ik ben weggezonken in een diepe modderpoel,~\sep\ en er is geen plaats, waar ik mijn voet kan zetten.

  Ik zonk in diepe wateren,~\sep\ en de golven bedelven mij.

  Ik ben uitgeput door het roepen,~\sep\ en schor is mijn keel,

  Mijn ogen zijn vermoeid,~\sep\ van het hoopvol opzien naar mijn God.

  Talrijker dan mijn hoofdharen zijn zij,~\sep\ die zonder reden mij haten;

  Krachtiger dan mijn beenderen, die mij ten onrechte vervolgen;~\sep\ of moet ik soms teruggeven wat ik niet heb geroofd?
\end{halfparskip}

\psalmsubtitle{b) Om Uw zaak, Heer, draag ik dit lijden}

\begin{halfparskip}
  God, Gij kent mijn dwaasheid,~\sep\ en mijn zonden zijn U niet verborgen.

  Laat hen, die op U hopen, niet te schande worden om mij,~\sep\ O Heer, Heer der legerscharen.

  Laat niet beschaamd worden om mij,~\sep\ die U zoeken, God van Israël.

  Want om U leed ik versmading,~\sep\ en bedekte schaamrood mijn gelaat.

  Ik ben een vreemdeling geworden voor mijn broeders,~\sep\ en een onbekende voor de zonen van mijn moeder.

  Want de ijver voor Uw huis heeft mij verteerd,~\sep\ en de smaad van die U smaden viel op mij neer.

  Door vasten heb ik mij gekastijd,~\sep\ en het werd mij tot smaad.

  Als gewaad trok ik een boetekleed aan,~\sep\ en ik werd hun tot spot.

  Die in de poort zitten, bepraten mij,~\sep\ en de wijndrinkers voegen mij schimpwoorden toe.
\end{halfparskip}

\psalmsubtitle{c) Red mij, Heer, uit de nood!}

\begin{halfparskip}
  Maar tot U, Heer, richt ik mijn bede,~\sep\ in de tijd der genade, O God:

  Verhoor mij volgens Uw grote goedheid,~\sep\ volgens Uw trouwe hulp.

  Trek mij op uit slijk, opdat ik niet verzinke; red mij uit de handen van die mij haten,~\sep\ en uit de diepten der wateren.

  Laat de watervloeden mij niet bedelven, en de diepe zee mij niet verzwelgen,~\sep\ noch de kuil zijn mond boven mij sluiten.

  Verhoor mij, Heer, want mild is Uw genade,~\sep\ zie op mij neer volgens Uw grote barmhartigheid.

  Verberg Uw aanschijn niet voor Uw dienstknecht;~\sep\ verhoor mij spoedig, want ik word gekweld.

  Nader tot mij om mij te redden;~\sep\ wil mij bevrijden vanwege mijn vijanden.

  Gij kent mijn smaad en mijn schaamte en mijn schande;~\sep\ allen, die mij kwellen, staan U voor ogen.

  De smaad brak mijn hart en ik kwijnde weg; ik zag uit naar een vol medelijden, maar er was er geen;~\sep\ naar troosters, maar ik vond ze niet.

  Zij mengden mijn spijzen met gal,~\sep\ en in mijn dorst gaven zij mij azijn te drinken.
\end{halfparskip}

\psalmsubtitle{d) Straf mijn vijanden, O Heer!}

\begin{halfparskip}
  Moge hun tafel hun worden tot val,~\sep\ en voor hun vrienden tot strik.

  Dat hun ogen worden verduisterd, zodat zij niet zien;~\sep\ en doe hun schreden immer wankelen!

  Stort Uw verontwaardiging over hen uit,~\sep\ en laat de gloed van Uw toorn hen aangrijpen!

  Hun woonstede worde verwoest,~\sep\ en niemand wone nog in hun tenten.

  Want zij hebben vervolgd, die Gij hebt geslagen,~\sep\ en de smart verhoogd van die Gij hebt gewond.

  Vermeerder hun schuld,~\sep\ en dat zij niet meer gerechtvaardigd worden bij U.

  Men wisse hen weg uit het boek der levenden,~\sep\ schrijve hen niet op met de rechtvaardigen.
\end{halfparskip}

\psalmsubtitle{e) Gij zult mij redden, Heer; ik zal U prijzen.}

\begin{halfparskip}
  Maar ik ben ellendig en treur;~\sep\ Uw hulp, O God, moge mij beschermen!

  Prijzen zal ik de Naam van God met een jubelzang,~\sep\ Hem met een danklied verheerlijken.

  Dit zal God welgevalliger zijn dan een offerstier,~\sep\ dan een rund met horens en hoeven.

  Ziet het, bedrukten, en weest blijde,~\sep\ dan leeft Uw hart weer op, O gij, die God zoekt.

  Want de Heer hoort de armen aan,~\sep\ en versmaadt Zijn gevangenen niet.

  Dat hemel en aarde Hem loven,~\sep\ de zeeën en al wat zich daarin beweegt.

  Want God zal Sion redden, en de steden van Juda herbouwen,~\sep\ daar zal men wonen en het bezitten.

  En het geslacht van Zijn dienaren zal het beërven,~\sep\ en die Zijn Naam liefhebben, zullen er wonen.
\end{halfparskip}

\PSALMtitle{70}{Gebed om hulp}

\begin{halfparskip}
  Gewaardig u, O God, mij te verlossen;~\sep\ Heer, snel mij te hulp.

  \qanona{Mijn Heer en mijn God, help mij!}

  Laat schande en beschaming hen treffen,~\sep\ die mij naar het leven staan.

  Dat zij vol schaamte terugdeinzen,~\sep\ die zich over mijn rampen verheugen,

  Terugwijken, met schaamte overladen,~\sep\ die mij toeroepen: Ha, ha!

  Over U mogen jubelen en zich verblijden,~\sep\ zij allen, die U zoeken.

  En steeds herhalen: ``Hooggeprezen zij God'',~\sep\ die uitzien naar Uw hulp.

  Ik echter ben ellendig en arm,~\sep\ God, sta mij bij!

  Gij zijt mijn Helper en mijn Redder;~\sep\ Heer, wil toch niet toeven!
\end{halfparskip}

% % % % % % % % % % % % % % % % % % % % % % % % % % % % % % % % % % % % % % % %

\hulala{10}

\Slota{Op U, Heer, stellen wij onze hoop, in Uw barmhartigheid stellen wij ons vertrouwen, en aan Uw liefderijke goedheid doen wij een verzoek; wees, Heer, een Helper in onze zwakheid, een Toevluchtsoord in onze verwarring, een Redder in onze ellende, een Vergevingsgezinde in onze zondigheid, verzamel ons die verstrooid zijn, bezorg ons wat we nodig hebben; en wend Uw gelaat niet af van het geluid van onze smeekbeden, U die goed bent en op wie ons vertrouwen is gesteld in alle seizoenen en tijden, Heer van alles...}

\marmita{27}

\PSALMtitle{71}{Verlaat mij niet nu ik oud ben}

\psalmsubtitle{a) Gij, O God, mijn steun van jongsaf}

\begin{halfparskip}
  Tot U neem ik mijn toevlucht, O Heer, moge ik niet beschaamd worden in eeuwigheid.~\sep\ Bevrijd en verlos mij volgens Uw gerechtigheid;~\sep

  \qanona{Komt, broeders, laten we onze toevlucht nemen tot het gebed, want dat is een sterk wapen, en daarmee zullen we Satan, onze vijand, de hater van onze natuur, overwinnen.}

  neig tot mij Uw oor en red mij.

  Wees mij een rots tot toevluchtsoord, een versterkte burcht om mij te redden:~\sep\ want mijn Rots en mijn Burcht zijt Gij.

  Ontruk mij, mijn God, aan de hand van de boze,~\sep\ uit de vuist van de goddeloze en van de verdrukker;

  Want Gij, mijn God, zijt mijn verwachting,~\sep\ mijn hoop, O Heer, vanaf mijn jeugd.

  Gij waart mijn Steun reeds vóór mijn geboorte, van de moederschoot af reeds mijn Beschermer,~\sep\ op U heb ik immer gehoopt.

  Als een wonder geleek ik voor velen,~\sep\ want Gij waart mijn krachtige Helper.

  Mijn mond was vol van Uw lof,~\sep\ de hele dag vol van Uw glorie.
\end{halfparskip}

\psalmsubtitle{b) Verlaat mij niet, Heer, in mijn ouderdom!}

\begin{halfparskip}
  Verwerp mij toch niet in mijn ouderdom,~\sep\ verlaat mij niet, nu mijn kracht mij begeeft.

  Want mijn vijanden bepraten mij,~\sep\ zij gaan mijn gangen na en overleggen tezamen.

  Ze zeggen: ``God heeft hem verlaten; zet hem na en grijpt hem aan,~\sep\ want er is niemand, die hem kan redden.''

  O God, blijf niet ver van mij af;~\sep\ mijn God, snel mij te hulp!

  Dat zij beschaamd worden en vergaan, die mijn leven belagen;~\sep\ dat smaad en schande bedekken, die mijn ongeluk zoeken.

  Ik echter zal immer vertrouwen,~\sep\ en bij al Uw lof nog dagelijks nieuwe voegen.

  Mijn mond zal Uw gerechtigheid verkondigen, en Uw bijstand heel de dag,~\sep\ want de maat ervan ken ik niet.

  Gods macht zal ik verhalen,~\sep\ Uw gerechtigheid roemen, O Heer, de Uwe alleen.
\end{halfparskip}

\psalmsubtitle{c) U zij lof, Heer!}

\begin{halfparskip}
  O God, vanaf mijn jeugd hebt Gij mij onderwezen,~\sep\ en tot heden verhaal ik Uw wonderen.

  Nu ik oud ben en grijs,~\sep\ verlaat mij toch niet, O God.

  Nu ik Uw kracht ga verkondigen aan dit geslacht,~\sep\ Uw macht aan allen, die nog komen,

  En Uw gerechtigheid, O God, die tot de hemel reikt,~\sep\ waardoor Gij zoveel groots hebt volbracht: God, wie is aan U gelijk?

  Vele en zware beproevingen hebt Gij gebracht over mij:~\sep\ weer zult Gij mij doen leven, en weer mij optrekken uit de diepten der aarde.

  Verhoog, mijn waardigheid,~\sep\ en troost mij opnieuw.

  Ook ik, O God, zal dan Uw trouw bij psalterspel roemen,~\sep\ U, Heilige van Israël, op de citer bezingen.

  Mijn lippen zullen juichen, wanneer ik voor U zing,~\sep\ zo ook mijn ziel, die Gij hebt verlost.

  Ook mijn tong zal heel de dag Uw gerechtigheid verkondigen,~\sep\ omdat die mijn ongeluk zoeken, met schande en schaamrood bedekt zijn.
\end{halfparskip}

\PSALMtitle{72}{Het Messiaanse rijk}

\psalmsubtitle{a) Uw rijk, Heer, is rechtvaardig en weldadig}

\begin{halfparskip}
  Geef aan de Koning Uw rechtsmacht, O God,~\sep\ en Uw rechtvaardigheidszin aan de Zoon van de Koning.

  \qanona{De genade van God is van bovenaf over iedereen uitgestort in de komst van Christus, de Verlosser van alle schepselen!}

  Hij moge Uw volk met rechtvaardigheid besturen,~\sep\ en Uw geringen naar billijkheid.

  De bergen zullen vrede brengen aan het volk,~\sep\ en de heuvelen gerechtigheid.

  De geringen uit het volk zal Hij beschermen, redding brengen aan de zonen der armen~\sep\ en de verdrukker met voeten treden.
\end{halfparskip}

\psalmsubtitle{b) Uw rijk is eeuwig}

\begin{halfparskip}
  Hij zal leven zolang als de zon bestaat,~\sep\ en als de maan door alle geslachten.

  Hij zal neerdalen als regen op het gras,~\sep\ als buien, die de aarde besproeien.

  In Zijn dagen zal de gerechtigheid bloeien,~\sep\ en overvloedige vrede, totdat de maan niet meer schijnt.
\end{halfparskip}

\psalmsubtitle{c) Uw rijk is algemeen}

\begin{halfparskip}
  Hij zal heersen van zee tot zee,~\sep\ van de Stroom tot aan de grenzen der aarde.

  Zijn vijanden zullen voor Hem neervallen,~\sep\ en Zijn weerstrevers de aarde kussen.

  Tharsis' koningen en die der eilanden zullen geschenken offeren,~\sep\ de koningen van Arabië en Saba gaven aanbrengen:

  Alle koningen zullen Hem aanbidden,~\sep\ alle volken Hem dienen.
\end{halfparskip}

\psalmsubtitle{d) Uw rijk is zegevol en eeuwig roemrijk}

\begin{halfparskip}
  Ja, bevrijden zal Hij de arme, die tot Hem roept,~\sep\ en de verdrukte, die geen helper heeft.

  Hij zal de behoeftige en arme genadig zijn,~\sep\ en het leven der armen redden,

  Hen van onrecht en verdrukking bevrijden;~\sep\ en hun bloed zal kostbaar zijn in Zijn oog.

  Daarom zal Hij leven en schenkt men Hem goud van Arabië,~\sep\ voortdurend zal men voor Hem bidden, Hem zegenen immerdoor.

  Overvloed van koren zal er zijn in het land; op de toppen der bergen zullen Zijn aren als de Libanon ruisen,~\sep\ en de bewoners der steden zullen bloeien als gras op het veld.

  Zijn Naam zal eeuwig gezegend zijn;~\sep\ zolang de zon zal schijnen, zal ook Zijn Naam bestaan.

  Alle stammen der aarde zullen gezegend worden in Hem,~\sep\ en alle volken Hem gelukkig prijzen.

  Gezegend zij de Heer, de God van Israël,~\sep\ die wonderen doet, Hij alleen.

  En Zijn roemrijke Naam zij gezegend voor eeuwig;~\sep\ en heel de aarde zij van Zijn glorie vervuld. Het zij zo, het zij zo.
\end{halfparskip}

\Slota{Wij belijden, aanbidden en verheerlijken U die geduldig van geest bent in Uw liefdevolle goedheid en een scherpe wreker in Uw gerechtigheid, grote Koning van glorie, Wezen dat van eeuwigheid is, in alle seizoenen en tijden, Heer van alles...}

\marmita{28}

\PSALMtitle{73}{Is er een voorzienigheid?}

\psalmsubtitle{a) Waarom de voorspoed der bozen?}

\begin{halfparskip}
  Hoe goed is Israëls God voor de rechtvaardigen,~\sep\ de Heer voor die rein zijn van hart!~\sep\ Toch wankelden bijna mijn voeten,~\sep

  \qanona{Onze Heer, Uw Geest is geduldig, maar Uw straf is streng.}

  haast gleden mijn schreden uit.

  Daar ik de goddelozen benijdde,~\sep\ toen ik de voorspoed der zondaren zag.

  Want kwellingen kennen zij niet,~\sep\ gezond en gezet is hun lichaam.

  De zorgen der stervelingen delen zij niet,~\sep\ en zij ontkomen de gesels der mensen.

  Daarom omsluit hen de trots als een halssnoer,~\sep\ en bedekt hen geweld als een kleed.

  De misdaad puilt uit hun zinnelijk hart,~\sep\ de verzinsels van hun geest dringen door naar buiten.

  Zij spotten en lasteren,~\sep\ zij dreigen op hoge toon met geweld,

  Zij zetten een mond op tegen de hemel,~\sep\ en hun tongen striemen de aarde.
\end{halfparskip}

\psalmsubtitle{b) Kwade invloed der bozen}

\begin{halfparskip}
  Daarom loopt mijn volk achter hen aan,~\sep\ en slurpen zij water in overvloed.

  En zij zeggen ``Hoe zou God het weten,~\sep\ en zou de Allerhoogste er kennis van dragen?''

  Zie, zo zijn de zondaars,~\sep\ en, steeds ongestoord, vermeerderen zij hun macht.
\end{halfparskip}

\psalmsubtitle{c) Misleidend is de voorspoed der bozen}

\begin{halfparskip}
  Heb ik dan vergeefs mijn hart in reinheid bewaard,~\sep\ en mijn handen in onschuld gewassen?

  Want almaar door word ik gegeseld,~\sep\ en iedere dag gekastijd.

  Had ik gedacht: Laat mij spreken als zij,~\sep\ dan had ik de aard van Uw kinderen verloochend.

  Ik dacht dus na om het te vatten,~\sep\ maar het leek mij een moeilijke zaak,

  Totdat ik binnentrad in Gods heiligdom,~\sep\ en op hun einde ging letten.
\end{halfparskip}

\psalmsubtitle{d) De boze gaat te gronde}

\begin{halfparskip}
  Waarlijk, Gij zet hen op een glibberig pad,~\sep\ stort hen neer in het verderf.

  Hoe zijn ze in een oogwenk ineengestort,~\sep\ verdwenen, in schrikkelijke angst vergaan!

  Als een droombeeld, O Heer, bij hem, die ontwaakt,~\sep\ zo zult Gij hun beeld, als Gij oprijst, versmaden.

  Toen mijn geest verbitterd was,~\sep\ en mijn hart werd geprikkeld,

  Was ik een dwaas zonder enig begrip,~\sep\ als een stuk vee voor Uw aanschijn.
\end{halfparskip}

\psalmsubtitle{e) God, mijn deel voor eeuwig}

\begin{halfparskip}
  Maar ik zal immer bij U zijn:~\sep\ Gij houdt mij vast aan mijn rechterhand.

  Met Uw raad zult Gij mij leiden,~\sep\ en mij opnemen, eens, in de glorie.

  Wie bezit ik in de hemel buiten U;~\sep\ en ben ik bij U, dan geeft mij de aarde geen vreugde.

  Mijn lichaam bezwijkt en mijn hart,~\sep\ de Rots van mijn hart en mijn Aandeel voor eeuwig is God.

  Want zie, die U verlaten, zullen vergaan;~\sep\ die U afvallig worden, verdelgt Gij allen.

  Maar mij is het goed bij God te zijn,~\sep\ mijn toevlucht te nemen bij God, de Heer.

  Al Uw werken zal ik verhalen,~\sep\ in de poorten der dochter Sion.
\end{halfparskip}

\PSALMtitle{74}{Klaagzang over de verwoeste tempel}

\psalmsubtitle{a) De tempel, Heer, is verwoest, de eredienst afgeschaft}

\begin{halfparskip}
  Waarom, O God, hebt Gij ons voor eeuwig verstoten,~\sep\ ontbrandt Uw toorn tegen de schapen van Uw weide?

  \qanona{Heer van alles, aan wie alles is geopenbaard, verstoot niet hen die U aanbidden!}

  Gedenk Uw volksgemeenschap, die Gij in oude tijden gesticht hebt, de stam, die Gij tot Uw bezit hebt vrijgekocht,~\sep\ de Sionsberg, waar Gij Uw zetel hebt gevestigd.

  Richt Uw schreden naar de eeuwige puinen:~\sep\ alles heeft de vijand in het heiligdom verwoest.

  Uw tegenstanders raasden op de plaats van Uw vergadering,~\sep\ en richtten er hun banieren als zegetekens op.

  Ze zijn als zij die met de bijl in het kreupelhout zwaaien;~\sep\ en zie, met houweel en hamer verbrijzelen zij tezamen zijn deuren.

  Aan het vuur hebben zij Uw heiligdom prijsgegeven,~\sep\ de woontent van Uw Naam tot de grond toe ontwijd.

  Zij spraken bij zichzelf: ``Laten wij hen allen tezamen verdelgen;~\sep\ verbrandt alle heiligdommen Gods in het land.''

  Onze tekenen zien wij reeds niet meer, er is geen profeet;~\sep\ en niemand onder ons weet hoe lang nog.

  Hoe lang nog, O God, zal de vijand smaden,~\sep\ de tegenstander Uw Naam maar immer lasteren?

  Waarom wendt Gij Uw hand van ons af,~\sep\ en houdt Gij Uw rechter terug in Uw schoot?
\end{halfparskip}

\psalmsubtitle{b) Red ons, O God, gelijk weleer!}

\begin{halfparskip}
  God toch is van oudsher mijn Koning,~\sep\ die midden op de aarde redding brengt.

  Gij hebt door Uw macht de zee gescheiden,~\sep\ in de wateren de koppen der draken verpletterd.

  Gij hebt de koppen van Leviathan verbrijzeld,~\sep\ hem tot voedsel gegeven aan de monsters der zee.

  Gij liet bronnen en beken ontspringen,~\sep\ Gij hebt waterrijke stromen drooggelegd.

  Van U is de dag en van U is de nacht;~\sep\ maan en zon hebt Gij hun vaste plaats gegeven.

  Alle grenzen der aarde hebt Gij bepaald;~\sep\ Gij hebt zomer en winter geschapen.
\end{halfparskip}

\psalmsubtitle{c) Red Uw volk, O Heer!}

\begin{halfparskip}
  Herinner U dit: de vijand heeft U gehoond, Oo Heer,~\sep\ en een waanzinnig volk heeft Uw Naam gelasterd.

  Geef het leven van Uw tortel niet prijs aan de gier,~\sep\ vergeet het leven van Uw armen niet voor immer.

  Denk aan Uw verbond,~\sep\ want geweld heerst in de schuilhoeken van land en veld.

  Dat geen verdrukte vol schaamte heenga:~\sep\ dat de arme en behoeftige prijzen Uw Naam.

  Rijs op, O God, verdedig Uw zaak,~\sep\ gedenk de smaad, die de dwaze U aandoet dag aan dag.

  Vergeet het geraas van Uw vijanden niet;~\sep\ het geschreeuw van die opstaan tegen U stijgt immer omhoog.
\end{halfparskip}

\Slota{Wij belijden, onze Heer en onze God, Uw genade voor ons, wij aanbidden de zorg van Uw welbehagen jegens ons en brengen U te allen tijde lof, eer, belijdenis en aanbidding, Heer van alles...}

\marmita{29}

\PSALMtitle{75}{Gods oordeel komt}

\begin{halfparskip}
  Wij loven U, Heer, wij loven U,~\sep\ en prijzen Uw Naam,

  \qanona{Laat ons het feest van Uw doop eren, Christus onze Verlosser!}

  verhalen Uw wonderen.~\sep

  ``Op de tijd, die Ik zal bepalen,~\sep\ zal Ik oordelen volgens recht.

  Al wankelt de aarde met al haar bewoners,~\sep\ Ik heb haar zuilen bevestigd.

  De trotsen roep Ik toe: ``Legt Uw hoogmoed af'';~\sep\ en de goddeloze: ``Steekt Uw hoorn niet op'';

  Steekt Uw hoorn niet op tegen de Allerhoogste,~\sep\ uit tegen God geen schaamteloze taal.

  Want noch van het oosten, noch van het westen;~\sep\ noch uit de woestijn, noch van de bergen:

  Maar God is de Rechter:~\sep\ de een drukt Hij neer, de ander verheft Hij.

  Want in de hand van de Heer is een beker,~\sep\ vol kruiden, die schuimt van wijn,

  Hij geeft er uit te drinken; tot de droesem zal men hem ledigen,~\sep\ alle bozen der aarde zullen ervan drinken.''

  Maar ik zal in eeuwigheid juichen,~\sep\ voor de God van Jacob de citer bespelen.

  En alle hoornen der bozen zal ik verbreken,~\sep\ maar de hoornen der gerechtigen worden verheven.
\end{halfparskip}

\PSALMtitle{76}{Overwinningslied}

\psalmsubtitle{a) Gij hebt de vijand verdelgd, O Heer!}

\begin{halfparskip}
  God is in Juda bekend,~\sep\ groot is Zijn Naam in Israël.

  \qanona{Verheven zijt Gij in eeuwigheid, Koning van alle schepselen.}

  Zijn tent staat in Salem,~\sep\ en Zijn woning in Sion.

  Daar brak Hij stuk de schichten van de boog,~\sep\ schild en zwaard en wapentuig.

  Gij, Machtige, schitterend van licht, zijt gekomen,~\sep\ van de eeuwige bergen.

  Ontwapend zijn de stoutmoedigen, zij slapen hun doodsslaap;~\sep\ en de handen van al die helden vielen slap.

  Door Uw dreigen, God van Jacob,~\sep\ werden wagens en paarden verlamd.
\end{halfparskip}

\psalmsubtitle{b) Gij zijt ontzagwekkend, O Heer}

\begin{halfparskip}
  Schrikwekkend zijt Gij, en wie zal U weerstaan,~\sep\ bij het geweld van Uw toorn?

  Vanuit de hemel hebt Gij Uw vonnis doen horen:~\sep\ de aarde ontstelde en zweeg,

  Toen God oprees ten oordeel,~\sep\ om alle verdrukten van het land te redden.

  Want de woede van Edom zal U tot glorie strekken,~\sep\ en die in Emath overbleven, zullen feesten om U.

  Doet geloften aan de Heer Uw God, en komt ze na,~\sep\ dat allen rondom Hem heen aan de Ontzagwekkende een offer brengen.

  Aan Hem, die de trots der vorsten fnuikt,~\sep\ die schrikwekkend is voor de koningen der aarde.
\end{halfparskip}

\PSALMtitle{77}{Gebed in nood}

\psalmsubtitle{a) Groot is mijn droefheid}

\begin{halfparskip}
  Luid verheft zich mijn stem tot God, mijn stem tot God, opdat Hij mij hore;~\sep

  \qanona{Heer van alles, aan wie alles is geopenbaard, verstoot niet Uw aanbidders.}

  op de dag van mijn kwelling zoek ik de Heer.

  Onvermoeid strekken bij nacht mijn handen zich uit;~\sep\ mijn ziel is ontroostbaar.

  Denk ik aan God, dan moet ik zuchten,~\sep\ peins ik na, dan verlies ik de moed.

  Gij houdt mijn ogen geopend;~\sep\ ik ben ontsteld en kan niet meer spreken.

  Ik overpeins de vroegere dagen,~\sep\ aan vervlogen jaren denk ik terug.

  Ik overweeg 's nachts in mijn hart,~\sep\ ik peins na, en mijn geest tracht uit te vorsen:
\end{halfparskip}

\psalmsubtitle{b) Heeft God Zijn volk verlaten?}

\begin{halfparskip}
  ``Zou God dan voor eeuwig verwerpen,~\sep\ en nooit meer genadig zijn?

  Zou Zijn liefde voorgoed zijn verdwenen,~\sep\ Zijn belofte verijdeld voor alle geslachten?

  Heeft God soms vergeten Zich te erbarmen,~\sep\ of in Zijn toorn Zijn ontferming bedwongen?''

  Dan zeg ik: ``Dit is mijn smart,~\sep\ dat de rechterhand van de Allerhoogste is veranderd''.

  Ik denk terug aan de werken van de Heer,~\sep\ ja, ik denk terug aan Uw aloude wonderen.

  Ik overweeg al Uw werken,~\sep\ en overpeins Uw daden.
\end{halfparskip}

\psalmsubtitle{c) Gij hebt ons steeds gered, O God}

\begin{halfparskip}
  O God, Uw weg is heilig:~\sep\ welke god is groot als onze God?

  Gij zijt de God, die wonderen doet,~\sep\ hebt Uw macht aan de volken doen kennen.

  Door Uw arm hebt Gij Uw volk verlost:~\sep\ de zonen van Jacob en Jozef.

  De wateren zagen U, O God, de wateren zagen U: zij beefden,~\sep\ en de golven werden onstuimig.

  Het zwerk stortte zijn stromen uit, de wolken verhieven hun stem,~\sep\ en Uw flitsen doorkliefden de lucht.

  Uw donder ratelde in de wervelwind, Uw bliksems verlichtten het aardrijk:~\sep\ de aarde sidderde en beefde.

  Uw weg werd gebaand door de zee en Uw pad door de machtige wateren,~\sep\ maar Uw sporen bleven  onzichtbaar.

  Als een kudde hebt Gij Uw volk geleid,~\sep\ door de hand van Moses en Aäron.
\end{halfparskip}

% % % % % % % % % % % % % % % % % % % % % % % % % % % % % % % % % % % % % % % %

\hulala{11}

\Slota{Wij smeken U, wijze Heerser en wonderbaarlijke Beheerder van Uw huisgezin, grote Schat die in Uw barmhartigheid alle hulp en weldaden uitstort, keer u naar ons, O Heer, heb medelijden met ons en ontferm U altijd over ons, zoals U gewend bent, Heer van alles...}

\marmita{30}

\PSALMtitle{78}{De zonden der vaderen, een les voor de kinderen}

\psalmsubtitle{a) Het verleden, een spiegel voor het heden}

\begin{halfparskip}
  Luister, mijn volk, naar mijn leer,~\sep\ neig uw oren naar de woorden van mijn mond.

  \qanona{De zonen van Israël, een bedorven volk.}

  Ik ga mijn mond voor wijze spreuken openen,~\sep\ diepzinnige lessen uit de oudheid verkondigen.

  Wat wij hoorden en vernamen,~\sep\ en wat onze vaderen ons hebben verhaald,

  Zullen wij niet voor hun kinderen verbergen,~\sep\ maar aan het nageslacht verhalen:

  De lof van de Heer en Zijn macht,~\sep\ en de wonderen, die Hij wrochtte.

  Want Hij maakte het tot voorschrift voor Jacob,~\sep\ en stelde het tot wet voor Israël:

  Wat Hij onze vaderen heeft geboden,~\sep\ zouden zij leren aan hun kinderen,

  Opdat het komend geslacht het zou weten, de kinderen, die worden geboren,~\sep\ opdat ook deze op zouden staan en het aan hun kinderen verhalen,

  Zodat ze hun vertrouwen op God blijven stellen en de werken van God niet zouden vergeten,~\sep\ maar zich houden aan Zijn geboden.

  Zij moesten niet als hun vaderen worden:~\sep\ een opstandig en weerspannig geslacht;

  Een geslacht, dat niet deugdzaam is van hart,~\sep\ en trouweloos van geest jegens God.
\end{halfparskip}

\psalmsubtitle{b) Ontrouw bij de Rode Zee en in de woestijn}

\begin{halfparskip}
  Efraïms zonen, de boogschutters,~\sep\ sloegen op de vlucht op de dag van de strijd.

  Het verbond met God onderhielden zij niet,~\sep\ zij weigerden volgens Zijn Wet te wandelen,

  Zijn werken waren zij niet indachtig,~\sep\ noch Zijn wonderen, aan hen betoond.

  Voor het oog van hun vaderen wrochtte Hij wonderen,~\sep\ in het land van Egypte, in de vlakte van Tanis.

  Hij scheidde de zee, voerde hen er doorheen,~\sep\ en vast deed Hij de wateren staan als een dam.

  Hij leidde hen door een wolk bij dag,~\sep\ heel de nacht door een lichtend vuur.

  Hij spleet rotsen in de woestijn,~\sep\ en laafde hen overvloedig als aan stromen.

  Uit de rots deed Hij beken ontspringen,~\sep\ en liet als rivieren het water vloeien.
\end{halfparskip}

\psalmsubtitle{c) Gemor en straffen in de woestijn}

\begin{halfparskip}
  Maar zij bleven tegen Hem zondigen,~\sep\ bleven de Allerhoogste tarten in de woestijn:

  In hun hart stelden zij God op de proef,~\sep\ door spijs te eisen naar hun begeerte;

  Zij verhieven hun stem tegen God;~\sep\ zij zeiden: ``Zou God in de woestijn wel een dis kunnen bereiden?

  Wel sloeg Hij op de rots, en de wateren vloeiden, en beken ontsprongen;~\sep\ maar zou Hij ook brood kunnen geven, of vlees verschaffen aan Zijn volk?''

  Toen dan de Heer dit vernam, ontstak Hij in gramschap,~\sep\ ontvlamde er een vuur tegen Jacob, en toorn bruiste op tegen Israël.

  Omdat ze niet in God geloofden,~\sep\ en niet op Zijn hulp vertrouwden.

  Toch gaf Hij de wolken daarboven bevel,~\sep\ en ontsloot Hij de poorten van de hemel;

  Hij regende manna als spijs op hen neer,~\sep\ en gaf hun brood uit de hemel.

  Het brood der sterken at de mens,~\sep\ Hij zond hun spijs tot verzadigings toe.

  Hij joeg uit de hemel de oostenwind op* , en voerde de zuidenwind aan door Zijn kracht.

  En Hij regende vlees als stof op hen neer,~\sep\ en als zand van de zee gevleugelde vogels.

  Ze vielen neer in hun legerplaats,~\sep\ rondom de tenten, waarin zij woonden;

  Zij aten, en werden ten volle verzadigd;~\sep\ Hij had aan hun begeerte voldaan.

  Maar nog was hun lust niet bevredigd, nog was de spijs in hun mond,~\sep\ of daar barstte Gods toorn tegen hen los.

  Hij richtte een slachting aan onder hun groten,~\sep\ en sloeg de jonge mannen van Israël neer.
\end{halfparskip}

\psalmsubtitle{d) Schijnbekering van Israël. God blijft barmhartig}

\begin{halfparskip}
  Maar toch bleven zij zondigen,~\sep\ en aan Zijn wonderen geloofden zij niet.

  Snel deed Hij hun dagen vergaan,~\sep\ en hun jaren door een plotseling verderf.

  Sloeg Hij hen neer, dan zochten zij Hem,~\sep\ dan keerden zij terug en vroegen om God;

  Zij herinnerden zich dat God hun Rots,~\sep\ de allerhoogste God hun Redder was.

  Maar met hun mond bedrogen zij Hem,~\sep\ en belogen Hem met hun tong.

  Hun hart was niet oprecht jegens Hem,~\sep\ zij bleven niet trouw aan Zijn verbond.

  Maar Hij vergaf meedogend hun schuld, en verdelgde hen niet;~\sep\ zo dikwijls weerhield Hij Zijn toorn, stortte niet heel Zijn gramschap uit.

  Hij dacht er aan dat ze maar vlees zijn,~\sep\ een zucht, die vervliegt en niet weerkeert.
\end{halfparskip}

\psalmsubtitle{e) De verlossing uit Egypte had hun niets geleerd}

\begin{halfparskip}
  Hoe dikwijls tergden zij Hem in de wildernis,~\sep\ bedroefden zij Hem in de woestijn.

  En telkens opnieuw beproefden zij God,~\sep\ en tergden de Heilige van Israël.

  Zij dachten niet meer aan Zijn hand,~\sep\ noch aan de dag, waarop Hij hen uit de greep van de vijand verloste,

  Toen Hij in Egypte Zijn tekenen deed,~\sep\ en Zijn wonderen in de vlakte van Tanis:

  Hij veranderde in bloed hun stromen,~\sep\ en hun beken, opdat ze niet konden drinken.

  Hij zond muggen op hen af, die hen verslonden,~\sep\ en kikvorsen, die op hen aanvielen;

  Hun gewas gaf Hij prijs aan de kever,~\sep\ aan de sprinkhaan de vrucht van hun werk.

  Met hagel sloeg Hij hun wijngaarden,~\sep\ en hun vijgenbomen met rijp;

  Ook hun vee gaf Hij prijs aan de hagel,~\sep\ en aan de bliksem hun kudden.

  Hij zond op hen af de gloed van Zijn toorn, verbolgenheid en gramschap en kwelling:~\sep\ een menigte van onheilstichters.

  Hij liet Zijn toorn de vrije loop, Hij redde hen niet van de dood,~\sep\ en hun vee gaf Hij prijs aan de pest.

  En Hij sloeg alle eerstgeborenen in Egypte,~\sep\ hun eerste kinderen in de tenten van Cham.

  Hij voerde Zijn Volk als schapen weg,~\sep\ en leidde hen in de woestijn als een kudde.

  Hij leidde hen veilig, ze hadden niets te vrezen:~\sep\ de zee bedekte hun vijanden.

  Hij voerde hen naar Zijn heilig land,~\sep\ naar de bergen, door Zijn rechterhand veroverd.

  Hij joeg de volken voor hen uit, wees hen door het lot als erfdeel aan,~\sep\ en deed Israëls stammen in hun tenten wonen.
\end{halfparskip}

\psalmsubtitle{f) Afgoderij in Canaän: Gods straffen}

\begin{halfparskip}
  Maar zij beproefden en tergden God, de Allerhoogste,~\sep\ en Zijn geboden onderhielden zij niet.

  Trouweloos vielen zij af zoals hun vaderen;~\sep\ zij faalden als een onbetrouwbare boog.

  Zij zetten Hem aan tot toorn door hun offerhoogten,~\sep\ en wekten door hun beelden Zijn na-ijver op.

  God hoorde het en ontstak in toorn,~\sep\ en met geweld verwierp Hij Israël.

  Hij verliet Zijn tent in Silo, de tent,~\sep\ waarin Hij onder de mensen woonde.

  Hij liet Zijn kracht in gevangenschap gaan,~\sep\ Zijn luister in de macht van de vijand.

  Zijn volk gaf Hij prijs aan het zwaard,~\sep\ Hij was tegen Zijn erfdeel verbitterd.

  Het vuur verteerde hun jonge mannen,~\sep\ en hun maagden werden niet verloofd.

  Hun priesters vielen door het zwaard,~\sep\ en hun weduwen hieven geen rouwklacht aan.
\end{halfparskip}

\psalmsubtitle{g) Juda om Ephraïms ontrouw uitverkoren}

\begin{halfparskip}
  Toen ontwaakte de Heer als uit een slaap,~\sep\ als een krijger, door wijn bevangen.

  Hij sloeg Zijn vijanden van achteren;~\sep\ eeuwige smaad deed Hij hun aan.

  Hij verwierp de tent van Jozef,~\sep\ de stam van Efraïm verkoos Hij niet meer,

  Maar Hij koos de stam van Juda uit,~\sep\ de Sionsberg, die Hij beminde.

  Hoog als de hemel trok Hij Zijn heiligdom op,~\sep\ vast als de aarde, die Hij voor immer gegrondvest heeft.

  Hij koos David, Zijn dienaar, uit,~\sep\ en nam hem van de schaapskooien weg.

  Hij riep hem van achter de zogende schapen,~\sep\ om Jacob te weiden, Zijn volk en Israël, Zijn erfdeel.

  Hij heeft hen geweid in onschuld van het hart,~\sep\ en met bekwame hand hen geleid.
\end{halfparskip}

\Slota{Red, O Heer, Uw volk, zegen Uw erfenis en laat Uw heerlijkheid wonen in de tempel die voor Uw eer is gereserveerd, alle dagen van de wereld, Heer van alles...}

\marmita{31}

\PSALMtitle{79}{Klaaglied over Sions verwoesting}

\psalmsubtitle{a) Jeruzalem, O Heer, is verwoest!}

\begin{halfparskip}
  De heidenen, O God, zijn Uw erfdeel binnengedrongen,~\sep\ zij hebben Uw heilige tempel ontwijd,

  \qanona{Omdat wij gezondigd hebben, hebben onze onderdrukkers ons onderworpen en Uw heilige plaats bezoedeld (\translationoptionNl{geprofaneerd}), Gij, Medelijdende, ontferm u over ons.}

  Jeruzalem tot een puinhoop gemaakt.

  De lichamen van Uw dienaars wierpen ze als aas voor de vogels van de hemel,~\sep\ het vlees van Uw heiligen voor de dieren van het veld.

  Hun bloed vergoten zij als water rondom Jeruzalem,~\sep\ en niemand was er, die hen begroef.

  Wij zijn tot smaad geworden voor onze buren,~\sep\ tot spot en hoon voor hen, die ons omringen.
\end{halfparskip}

\psalmsubtitle{b) Straf de vijand, Heer; spaar Uw volk!}

\begin{halfparskip}
  Hoe lang nog, Heer; zult Gij dan eeuwig toornen?~\sep\ zal Uw ijverzucht branden als vuur?

  Stort Uw gramschap uit over de heidenen, die U niet erkennen,~\sep\ en over de koninkrijken, die Uw Naam niet aanroepen.

  Want zij hebben Jacob verslonden,~\sep\ en zijn woonplaats verwoest.

  Ach, reken ons de schulden van onze vaderen niet aan; Uw ontferming trede ons snel tegemoet,~\sep\ want wij zijn uiterst ellendig.
\end{halfparskip}

\psalmsubtitle{c) Help ons, Heer!}

\begin{halfparskip}
  Help ons, God, onze Redder, om de eer van Uw Naam,~\sep\ en om Uw Naam bevrijd ons, en vergeef onze zonden.

  Waarom moeten de heidenen zeggen:~\sep\ ``Waar is hun God?''

  Laat de heidenen voor onze ogen zien,~\sep\ de wraak voor het vergoten bloed van Uw dienaars.

  Het zuchten der gevangenen dringe tot U door;~\sep\ verlos door de kracht van Uw arm die ten dode zijn gewijd.

  Werp zevenvoudig in de schoot van onze buren,~\sep\ de smaad, die zij U aandeden, O Heer.

  Maar wij, Uw volk en de schapen van Uw weide,~\sep\ wij zullen U eeuwig verheerlijken, en verkondigen Uw lof van geslacht tot geslacht.
\end{halfparskip}

\PSALMtitle{80}{Herstel uw volk, Heer}

\psalmsubtitle{a) Heer, bevrijd Uw volk!}

\begin{halfparskip}
  Herder van Israël, luister aandachtig,~\sep\ Gij, die Jozef leidt als een kudde;

  \qanona{Gij bewaarde in hun geslachten onze vaders, die U behaagden, Gij die alles ziet, red Uw Kerk.}

  Die zetelt op cherubs, laat stralen Uw licht,~\sep\ voor Efraïm, Benjamin en Manasse.

  Wek op Uw macht,~\sep\ en kom ons verlossen.

  O God, richt ons weer op,~\sep\ en toon ons Uw vredig gelaat, opdat wij worden gered.

  O God der heerscharen, hoe lang nog zult Gij toornen,~\sep\ daar Uw volk toch bidt?

  Gij hebt het gevoed met tranenbrood,~\sep\ en overvloedig met tranen gelaafd.

  Gij hebt ons gemaakt tot twistappel voor onze buren,~\sep\ en onze vijanden spotten met ons.

  O God der heerscharen, richt ons weer op,~\sep\ en toon ons Uw vredig gelaat, opdat wij worden gered.
\end{halfparskip}

\psalmsubtitle{b) Herstel, O Heer, Uw verwoeste wijnstok!}

\begin{halfparskip}
  Een wijnstok hebt Gij uit Egypte weggenomen,~\sep\ volken uitgeworpen om hem te planten.

  Gij hebt de grond voor hem bereid:~\sep\ hij schoot wortel en begroeide het land;

  Zijn schaduw bedekte de bergen,~\sep\ zijn ranken de ceders van God.

  Hij strekte zijn twijgen uit tot de zee,~\sep\ tot aan de stroom zijn loten.

  Waarom hebt Gij zijn omheining vernield,~\sep\ zodat elke voorbijganger hem plukt,

  De ever van het woud hem omwroet,~\sep\ en de dieren van het veld hem kaal vreten?

  God der heerscharen, keer terug,~\sep\ blik neer uit de hemel en zie, en bezoek deze wijnstok;

  Bescherm hem, die Uw rechterhand heeft geplant,~\sep\ de twijg, die Gij voor U hebt krachtig gemaakt.

  Mogen zij, die in vuur hem verbrandden en vernielden,~\sep\ door Uw dreigende blikken vergaan.

  Uw hand moge rusten op de man aan Uw rechterzijde,~\sep\ op het mensenkind, dat Gij sterk deedt worden voor U.

  En nimmermeer zullen wij U verlaten;~\sep\ Gij zult ons in leven houden, en Uw Naam zullen wij prijzen.

  Heer, God der heerscharen, richt ons weer op,~\sep\ en toon ons Uw vredig gelaat, opdat wij worden gered.
\end{halfparskip}

\PSALMtitle{81}{Het feest der bazuinen}

\psalmsubtitle{a) Wil waardig Gods feesttij vieren!}

\begin{halfparskip}
  Jubelt voor God, onze Helper,~\sep\ juicht de God van Jacob toe.

  \qanona{Gezegend is de Heer die Zijn heiligen helpt, die Zijn woorden onderhouden, en die wie Hem haat snel vernietigt.}

  Doet het psalter weerklinken en slaat de pauken,~\sep\ tokkelt de welluidende citer en de lier.

  Steekt de bazuin bij de Nieuwe Maan,~\sep\ bij volle maan op onze plechtige feestdag.

  Want het is een voorschrift voor Israël,~\sep\ een bevel van Jacobs God.

  Dit stelde Hij Jozef tot wet,~\sep\ toen hij optrok tegen het land van Egypte.
\end{halfparskip}

\psalmsubtitle{b) Mijn volk, blijf Mij trouw. Ik zal u zegenen}

\begin{halfparskip}
  Ik hoorde een taal, die ik niet kende: ``Ik heb de last van zijn schouders genomen;~\sep\ zijn handen lieten de draagkorf los.

  Ge riept in uw nood en Ik heb u bevrijd, uit de donderwolk heb Ik u antwoord gegeven,~\sep\ u beproefd bij het water van Meriba.

  Hoor, mijn volk, en Ik zal u vermanen;~\sep\ ach, Israël, wil toch luisteren naar Mij:

  Er zij geen andere god onder u,~\sep\ geen vreemde god zult gij aanbidden.

  Ik ben de Heer, Uw God, die u leidde uit het land van Egypte:~\sep\ open uw mond, en Ik zal hem vullen.

  Maar Mijn volk luisterde niet naar Mijn stem,~\sep\ en Israël was Mij niet onderdanig.

  Daarom gaf Ik hen prijs aan de verstoktheid van hun harten:~\sep\ dat zij nu handelen volgens hun grillen!

  Mocht toch Mijn volk naar Mij luisteren,~\sep\ Israël Mijn wegen bewandelen!

  Dan zou Ik aanstonds hun haters bedwingen,~\sep\ Mijn hand zou Ik keren tegen hun weerstrevers.

  De vijanden van de Heer zouden zich buigen voor Hem,~\sep\ en hun lot zou vaststaan voor eeuwig.

  Maar hem zou ik spijzen met bloem van tarwe,~\sep\ met honing uit de rots hem verzadigen.
\end{halfparskip}

% % % % % % % % % % % % % % % % % % % % % % % % % % % % % % % % % % % % % % % %

\PSALMtitle{}{}

\psalmsubtitle{}

\begin{halfparskip}
  \qanona{}
\end{halfparskip}

\psalmsubtitle{}

\begin{halfparskip}
\end{halfparskip}

\psalmsubtitle{}

\begin{halfparskip}
\end{halfparskip}

% % % % % % % % % % % % % % % % % % % % % % % % % % % % % % % % % % % % % % % %

\end{document}