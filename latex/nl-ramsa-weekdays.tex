\documentclass[12pt,twoside,a5paper]{article}

\usepackage{multicol}

\usepackage[main=dutch]{babel}
\usepackage{divine-office}

% % % % % % % % % % % % % % % % % % % % % % % % % % % % % % % % % % % % % % % %

% Version: 2024-06-11

\begin{document}

\title{Ramsa --- weekdagen}
\author{}
\date{}
\maketitle

% % % % % % % % % % % % % % % % % % % % % % % % % % % % % % % % % % % % % % % %

\begin{halfparskip}
  \cc~Eer aan God in den hoge \liturgicalhint{(3x)}. En op aarde vrede en goede hoop aan de mensen, altijd en in eeuwigheid. [Amen\footnote{De Hudra en de Syro-Malabar liturgie voegen hier ``Amen'' aan toe, in tegenstelling tot Breviarium.}.]~--- \rr~Zegen, Heer.~--- \liturgicalhint{Vredekus.}

  \cc~Onze Vader die in de hemelen zijt,

  \rr~Geheiligd zij Uw Naam. Uw rijk kome, heilig, heilig, heilig zijt Gij. Onze Vader die in de hemelen zijt, de hemel en de aarde zijn gevuld met Uw onmetelijke glorie; de engelen en de mensen roepen U toe: heilig, heilig, heilig zijt Gij. --- Onze Vader die in de hemelen zijt, geheiligd zij Uw Naam. Uw rijk kome, Uw wil geschiede op aarde zoals in de hemel. Geef ons heden het brood dat we nodig hebben en vergeef ons onze schulden en zonden zoals wij ook vergeven hebben aan onze schuldenaren. En leid ons niet in bekoring, maar verlos ons van de Kwade. Want van U is het koninkrijk en de kracht en de heerlijkheid in eeuwigheid, amen.

  \cc~Eer aan de Vader, de Zoon, en de Heilige Geest.

  \rr~Vanaf het begin en in alle eeuwigheid, amen en amen. Onze Vader die in de hemelen zijt, geheiligd zij Uw naam, Uw rijk kome, heilig, heilig, heilig zijt Gij. Onze Vader die in de hemelen zijt, de hemel en de aarde zijn gevuld met Uw onmetelijke glorie; de engelen en de mensen roepen U toe: heilig, heilig, heilig zijt Gij.

  \dd~Laat ons bidden, vrede zij met ons.

  \cc~We willen, Heer, Uw Godheid prijzen \liturgicalhint{(herhaal)} en Uw Majesteit aanbidden, en aan Uw glorierijke Drievuldigheid eeuwige, altijddurende lof brengen, Heer van alles, Vader... in alle eeuwigheid. --- \rr~Amen.
\end{halfparskip}

% % % % % % % % % % % % % % % % % % % % % % % % % % % % % % % % % % % % % % % %

\markedsection{Marmita}

\liturgicalhint{Na de eerste zin van de eerste psalm van elke marmita, zeg 3x alleluja, en herhaal het eerste vers.}

\begin{halfparskip}
  \liturgicalOption{Maandagen ``voor'':} \liturgicalhint{Marmita 4 (\Pss{11--14}).}
\end{halfparskip}

\begin{halfparskip}
  \psalm{\Ps{11}} Ik vlucht tot de Heer; hoe kunt gij mij zeggen:~\sep\ ``Vlieg weg als een vogel naar het gebergte.

  \qanona{Alleluia, alleluia, alleluia.} --- \liturgicalhint{Eerste vers.}

  Want zie, de bozen spannen de boog; ze zetten de pijl op de pees,~\sep\ om de oprechten van hart in het duister te treffen.

  Als zelfs de grondvesten worden gesloopt,~\sep\ wat zal de gerechtige dan nog vermogen?''

  De Heer woont in Zijn heilige tempel,~\sep\ de Heer heeft in de hemel Zijn troon.

  Zijn ogen zien rond,~\sep\ Zijn wimpers doorvorsen de kinderen der mensen.

  De Heer doorvorst de gerechte en de boze;~\sep\ die het onrecht liefheeft, is Hem een gruwel.

  Hij zal op de zondaars gloeiende kolen en zwavel doen regenen;~\sep\ een verzengende wind is de dronk van hun beker.

  Want de Heer is rechtvaardig en heeft de gerechtigheid lief;~\sep\ de goeden zullen Zijn aanschijn aanschouwen.
\end{halfparskip}

\begin{halfparskip}
  \psalm{\Ps{12}} Heer, schenk redding, want er zijn geen vromen meer;~\sep\ verdwenen is de trouw onder de kinderen der mensen.

  Allen liegen ze elkander voor,~\sep\ ze spreken met bedrieglijke lippen en vals gemoed.

  De Heer rukke al die bedrieglijke lippen uit,~\sep\ die grootsprekende tong,

  Hen die zeggen: ``Sterk zijn wij door onze tong;~\sep\ wij hebben onze lippen met ons, wie kan ons overmeesteren?''

  ``Om de nood der verdrukten en het gejammer der armen zal Ik nu opstaan,'' zegt de Heer:~\sep\ ``redding zal Ik brengen aan wie er naar smacht.''

  De woorden van de Heer zijn oprechte woorden,~\sep\ zuiver zilver, van stof ontdaan, tot zevenmaal gelouterd.

  Gij, Heer, zult ons behouden,~\sep\ ons eeuwig beschermen tegen dit geslacht.

  De bozen zwermen om ons heen,~\sep\ terwijl de heffe van het volk oprijst.
\end{halfparskip}

\begin{halfparskip}
  \psalm{\Ps{13}} Hoelang nog, Heer, zult Gij mij geheel vergeten,~\sep\ hoelang nog voor mij Uw aanschijn verbergen?

  Hoelang nog zal ik de smart overdenken in mijn ziel,~\sep\ en het wee in mijn hart van dag tot dag?

  Hoelang nog zal mijn vijand zich boven mij verheffen?~\sep\ Zie neer en verhoor mij, o Heer, mijn God!

  Stort licht in mijn ogen opdat ik de doodsslaap niet inga~\sep\ en mijn vijand niet zegge: ``Ik heb hem overwonnen'',

  En mijn weerstrevers niet juichen over mijn val,~\sep\ daar ik op Uw erbarming vertrouwde.

  Nu juiche mijn hart om Uw hulp!~\sep\ De Heer wil ik bezingen, die mij heeft welgedaan.
\end{halfparskip}

\begin{halfparskip}
  \psalm{\Ps{14}} De dwaas zegt bij zichzelf:~\sep\ ``Er is geen God.''

  Ze zijn bedorven, gruwelen hebben ze bedreven;~\sep\ daar is er niet één, die deugdzaam handelt.

  De Heer blikt uit de hemel neer op de kinderen der mensen,~\sep\ om te zien of er wel één is met verstand, wel één die God zoekt.

  Allen zonder uitzondering zijn ze afgedwaald, allen diep bedorven;~\sep\ daar is er niet één die deugdzaam handelt, niet één.

  Zullen al die bozen dan nimmer tot inzicht komen,~\sep\ zij, die mijn volk verslinden, als aten zij brood?

  Zij riepen de Heer niet aan. Eens zullen zij sidderen van angst,~\sep\ want God is met het geslacht der rechtvaardigen.

  Het beleid van de verdrukte wilt gij beschamen,~\sep\ maar de Heer is zijn toevlucht.

  O, mocht er uit Sion toch heil voor Israël dagen! Als de Heer het lot van Zijn volk ten goede keert,~\sep\ zal er gejubel zijn in Jacob, vreugde in Israël.
\end{halfparskip}

\begin{halfparskip}
  \liturgicalhint{Eer} --- \liturgicalhint{3x Alleluia.} --- \liturgicalhint{Eerste vers:} Ik vlucht tot de Heer; hoe kunt gij mij zeggen:~\sep\ ``Vlieg weg als een vogel naar het gebergte.
\end{halfparskip}

\begin{halfparskip}
  \liturgicalOption{Dinsdagen ``voor'':} \liturgicalhint{Marmita 9 (\Pss{25--27}).}
\end{halfparskip}

\begin{halfparskip}
  \psalm{\Ps{25}} Tot U verhef ik mijn ziel,~\sep\ o Heer, mijn God.~\sep\ Op U vertrouw ik; laat mij niet te schande worden;

  \qanona{Alleluia, alleluia, alleluia.} --- \liturgicalhint{Eerste vers.}

  dat mijn vijanden niet over mij juichen!

  Want van wie op U hopen, wordt niemand beschaamd,~\sep\ maar wel worden te schande, die hun woord vermetel breken.

  Toon mij Uw wegen, o Heer,~\sep\ en leer mij Uw paden kennen.

  Leid mij in Uw waarheid, en geef mij onderricht, omdat Gij, God, mijn Redder zijt,~\sep\ en immer hoop ik op U.

  Gedenk Uw ontferming, o Heer,~\sep\ en Uw barmhartigheid, die van oudsher zijn.

  De zonden van mijn jeugd en mijn misslagen, gedenk ze niet; wees mij naar Uw erbarming indachtig, *
  vanwege Uw goedheid, Heer.

  Goed en rechtvaardig is de Heer;~\sep\ daarom wijst Hij de zondaars de weg.

  De nederigen leidt Hij in gerechtigheid,~\sep\ de nederigen toont Hij Zijn weg.

  Alle wegen van de Heer zijn goedheid en trouw,~\sep\ voor wie Zijn Verbond en Zijn wetten bewaren.

  Omwille van Uw Naam, o Heer,~\sep\ vergeef mij mijn zonde, want zij is groot.

  Wie is de man, die de Heer vreest?~\sep\ Hij wijst hem de weg, die hij moet kiezen.

  Hij zelf zal in voorspoed leven,~\sep\ en zijn geslacht het land bezitten.

  De Heer is een vriend voor hen, die Hem vrezen:~\sep\ Zijn Verbond doet Hij hun kennen.

  Mijn ogen zijn immer gericht op de Heer,~\sep\ want uit de strik zal Hij mijn voeten bevrijden.

  Zie op mij neer en wees mij genadig,~\sep\ want eenzaam ben ik en ellendig.

  Verlicht de druk van mijn hart,~\sep\ en bevrijd mij van mijn angsten.

  Zie mijn ellende en mijn kwelling;~\sep\ en vergeef mij al mijn zonden.

  Let op mijn vijanden, want ze zijn talrijk,~\sep\ en haten mij met felle haat.

  Bescherm mijn leven en red mij;~\sep\ het zij mij niet tot schande, dat ik bij U mijn toevlucht zocht.

  Dat mijn onschuld en deugd mij beschermen,~\sep\ daar ik hoop op U, o Heer.

  Verlos Israël, o God,~\sep\ uit al zijn kommernissen!
\end{halfparskip}

\begin{halfparskip}
  \psalm{\Ps{26}} Heer, schaf mij recht, want ik leefde in onschuld;~\sep\ vertrouwend op de Heer, heb ik niet gewankeld.

  Onderzoek mij, Heer, en stel mij op de proef;~\sep\ doorgrond mijn nieren en mijn hart.

  Want Uw welwillendheid staat mij voor ogen,~\sep\ en ik wandel naar Uw waarheid.

  Met ongerechtigen zit ik niet neer,~\sep\ en met bedriegers kom ik niet samen.

  Ik haat het gezelschap van bozen,~\sep\ en met goddelozen zit ik niet samen.

  In onschuld was ik mijn handen,~\sep\ en ga rond Uw altaar, o Heer.

  Om openlijk Uw lof te verkondigen,~\sep\ en al Uw wonderen te verhalen.

  Heer, ik heb lief het verblijf van Uw huis,~\sep\ en de woontent van Uw heerlijkheid.

  Ruk mijn ziel niet weg met de zondaars,~\sep\ noch mijn leven met bloeddorstige mannen,

  Aan wier handen de misdaad kleeft,~\sep\ en wier rechterhand met geschenken gevuld is.

  Ik echter wandel in onschuld:~\sep\ red mij, en wees mij genadig.

  Mijn voet staat op effen baan;~\sep\ ik zal de Heer in de vergadering loven.
\end{halfparskip}

\begin{halfparskip}
  \psalm{\Ps{27}} De Heer is mijn licht en mijn heil: wie zou Ik vrezen?~\sep\ De Heer is de schuts van mijn leven: voor wie zou ik sidderen?

  Als de bozen mij bestormen om mijn vlees te verslinden,~\sep\ mijn vijanden en haters, zij struikelen en vallen.

  Al stond er een krijgsmacht tegenover mij, mijn hart zou niet vrezen;~\sep\ al brak er een oorlog tegen mij uit, dan nog zou ik vertrouwen.

  Dit alleen vraag ik de Heer, dit alleen streef ik na:~\sep\ te wonen in het huis van de Heer alle dagen van mijn leven,

  Te genieten de zoetheid van de Heer,~\sep\ en Zijn tempel te aanschouwen.

  Want in Zijn woontent zal Hij mij bergen in tijden van nood,~\sep\ Hij zal mij doen schuilen diep in Zijn tent, mij plaatsen boven op de rots.

  Nu verheft zich mijn hoofd,~\sep\ boven de vijanden, die mij omringen;

  Jubeloffers zal ik brengen in Zijn tent,~\sep\ zingen voor de Heer en spelen op de citer.

  Heer, luister naar mijn stem, waarmee ik luid roep,~\sep\ ontferm u over mij, en schenk mij verhoring!

  Tot U spreekt mijn hart, U zoeken mijn ogen;~\sep\ ik zoek Uw aanschijn, o Heer.

  Verberg mij Uw aanschijn niet,~\sep\ stoot Uw dienaar niet af in Uw toorn!

  Gij zijt mijn hulp; verwerp mij niet!~\sep\ Verlaat mij niet, o God, mijn Redder!

  Zou mijn vader en moeder mij ook verlaten,~\sep\ dan nog neemt de Heer mij op.

  Wijs mij Uw weg, o Heer,~\sep\ en leid mij op effen baan omwille van mijn weerstrevers.

  Geef mij niet prijs aan de moedwil van mijn vijanden,~\sep\ want valse getuigen en geweldenaars stonden tegen mij op.

  Ik ben er zeker van de weldaden van de Heer te zien,~\sep\ in het land der levenden.

  Zie uit naar de Heer, wees onversaagd;~\sep\ sterk zij Uw hart, zie uit naar de Heer.
\end{halfparskip}

\begin{halfparskip}
  \liturgicalhint{Eer} --- \liturgicalhint{3x Alleluia.} --- \liturgicalhint{Eerste vers:} Tot U verhef ik mijn ziel,~\sep\ o Heer, mijn God.~\sep\ Op U vertrouw ik; laat mij niet te schande worden.
\end{halfparskip}

\begin{halfparskip}
  \liturgicalOption{Woensdagen ``voor'':} \liturgicalhint{Marmita 23 (\Pss{62--64}).}
\end{halfparskip}

\begin{halfparskip}
  \psalm{\Ps{62}} In God alleen rust mijn ziel,~\sep\ van Hem komt mijn heil.

  \qanona{Alleluia, alleluia, alleluia.} --- \liturgicalhint{Eerste vers.}

  Hij alleen is mijn rots en mijn heil,~\sep\ mijn bescherming: neen, ik zal niet wankelen.

  Hoe lang stormt gij op één man los, werpt gij hem met u allen omver,~\sep\ als een hellende wand, als een
  wankelende muur?

  Ja waarlijk, zij beramen hoe zij mij van mijn verheven plaats zullen stoten;~\sep\ zij verlustigen zich in de leugens;

  Ze zegenen met hun mond,~\sep\ maar met hun hart vervloeken zij.

  Rust in God alleen, mijn ziel,~\sep\ want van Hem komt wat ik hoop.

  Hij alleen is mijn rots en mijn heil,~\sep\ mijn bescherming: neen, ik zal niet wankelen.

  Bij God is mijn heil en mijn roem,~\sep\ mijn sterke rots: in God is mijn schuilplaats.

  Hoop op Hem, o volk, ten allen tijde: stort voor Hem uw harten uit:~\sep\ God is ons tot toevlucht.

  Een ademtocht slechts zijn de kinderen der mensen,~\sep\ bedrieglijk de zonen van mannen.

  Zij rijzen op de weegschaal omhoog:~\sep\ lichter dan een zucht zijn zij allen tezamen.

  Verwacht niets van verdrukking, en roem niet ijdel op roof;~\sep\ als uw vermogen toeneemt, wilt er uw hart niet aan hechten.

  Eén zaak heeft God gezegd; deze twee dingen vernam ik: "Bij God berust de macht, en bij U de goedheid, o Heer,~\sep\ want een ieder zult Gij naar werk vergelden.
\end{halfparskip}

\begin{halfparskip}
  \psalm{\Ps{63}} God, mijn God zijt Gij:~\sep\ met aandrang zoek ik U.

  Naar U dorst mijn ziel, naar U smacht mijn lichaam,~\sep\ als een dor en dorstig, waterloos land.

  Zo blijf ik U beschouwen in Uw heiligdom,~\sep\ om Uw macht te zien en Uw glorie.

  Daar Uw genade meer dan het leven geldt,~\sep\ zullen mijn lippen U loven.

  Zo wil ik U prijzen mijn leven lang,~\sep\ in Uw Naam mijn handen verheffen.

  Als met merg en vet zal ik verzadigd worden,~\sep\ en met jubelende lippen zal mijn mond U loven.

  Als ik aan U denk op mijn legerstede,~\sep\ blijf ik in de nachtwaken peinzen over U.

  Want Gij zijt mijn Helper geworden,~\sep\ en ik juich in de schaduw van Uw vleugelen.

  Mijn ziel hecht zich aan U,~\sep\ Uw rechterhand is mij tot steun.

  Maar zij, die mij naar het leven staan,~\sep\ zullen verzinken in de diepten der aarde.

  Ze zullen vallen in de macht van het zwaard,~\sep\ en de prooi van vossen worden.

  Maar de koning zal zich verblijden in God; een ieder, die bij Hem zweert, zal roemen,~\sep\ omdat de mond van de lasteraars zal worden gestopt.
\end{halfparskip}

\begin{halfparskip}
  \psalm{\Ps{64}} Luister, o God, als ik klaag, naar mijn stem;~\sep\ behoed mijn leven voor de schrik van de vijand.

  Bescherm mij tegen de samenscholing van bozen,~\sep\ tegen het woelen van hen, die onrecht bedrijven.

  Die hun tongen scherpen als een zwaard,~\sep\ als pijlen hun giftige woorden richten,

  Om vanuit een schuilplaats de onschuldige te treffen,~\sep\ hem onverhoeds te treffen, zonder iets te duchten.

  Vastberaden smeden zij boze plannen, en werken samen om heimelijk strikken te spannen;~\sep\ zij zeggen: ``Wie slaat er acht op ons?''

  Zij denken misdaden uit, verbergen hun weloverwogen gedachten;~\sep\ een afgrond is hun geest en hart.

  Maar God treft hen met Zijn pijlen:~\sep\ onverhoeds worden zij met wonden geslagen,

  En hun eigen tong brengt hun verderf;~\sep\ allen, die hen zien, schudden het hoofd.

  En allen zijn vol ontzag en prijzen Gods werk,~\sep\ en overwegen Zijn daden.

  De rechtvaardige juicht in de Heer en vlucht tot Hem,~\sep\ en allen roemen, die oprecht van harte zijn.
\end{halfparskip}

\begin{halfparskip}
  \liturgicalhint{Eer} --- \liturgicalhint{3x Alleluia.} --- \liturgicalhint{Eerste vers:} In God alleen rust mijn ziel,~\sep\ van Hem komt mijn heil.
\end{halfparskip}

\begin{halfparskip}
  \liturgicalOption{Donderdagen ``voor'':} \liturgicalhint{Marmita 38 (\Pss{96--98}).}
\end{halfparskip}

\begin{halfparskip}
  \psalm{\Ps{96}} Zingt een nieuw lied voor de Heer,~\sep\ zingt voor de Heer alle landen!

  \qanona{Alleluia, alleluia, alleluia.} --- \liturgicalhint{Eerste vers.}

  Zingt voor de Heer, zegent Zijn Naam,~\sep\ verkondigt Zijn heil van dag tot dag!

  Maakt onder de heidenen Zijn glorie bekend,~\sep\ onder alle volken Zijn wonderen.

  Want groot is de Heer en hoog te prijzen,~\sep\ meer te duchten dan alle goden.

  Want de goden der heidenen zijn allen verzinsels,~\sep\ maar de Heer heeft de hemel geschapen.

  Majesteit en pracht gaan vóór Hem uit;~\sep\ macht en luister zijn in Zijn heilige woning.

  Kent toe aan de Heer, geslachten der volken, kent toe aan de Heer glorie en macht;~\sep\ kent toe aan de Heer de roem van Zijn Naam!

  Treedt met Uw offer Zijn voorhoven binnen;~\sep\ aanbidt de Heer in heilige feesttooi!

  Beef voor Zijn aanschijn, geheel de aarde;~\sep\ roept tot de volken: De Heer is Koning!

  Hij maakte de aarde onwankelbaar vast,~\sep\ heerst over de volken met billijkheid.

  Dat de hemelen juichen en de aarde jubele, laat bruisen de zee met wat ze bevat,~\sep\ laat jubelen het veld met wat er op groeit.

  Dan zullen juichen alle bomen van het woud voor het aanschijn van de Heer, want Hij komt,~\sep\ want Hij komt de aarde regeren.

  Hij zal met rechtvaardigheid de wereld regeren,~\sep\ en de volkeren volgens Zijn trouw.
\end{halfparskip}

\begin{halfparskip}
  \psalm{\Ps{97}} De Heer is Koning, dat de aarde jubele,~\sep\ dat de vele eilanden juichen!

  Wolken en duisternis omgeven Hem,~\sep\ gerechtigheid en recht zijn de steun van Zijn troon.

  Vuur gaat vóór Hem uit,~\sep\ en verbrandt Zijn vijanden om Hem heen.

  Zijn bliksems verlichten het aardrijk;~\sep\ de aarde ziet het en beeft.

  Bergen versmelten als was voor de Heer,~\sep\ voor de Beheerser van heel de aarde.

  De hemelen verkondigen Zijn gerechtigheid,~\sep\ en alle volken aanschouwen Zijn glorie.

  Beschaamd staan allen die beelden vereren, die roemen op valse goden:~\sep\ voor Hem werpen alle goden zich neer.

  Sion hoort het vol vreugde, en de steden van Juda juichen,~\sep\ om Uw oordelen, Heer.

  Want Gij, Heer, zijt verheven boven heel de aarde,~\sep\ hoogverheven onder alle goden.

  De Heer heeft lief die het kwade haten, Hij behoedt het leven van Zijn heiligen,~\sep\ en redt hen uit de hand der bozen.

  Een licht rijst op voor de rechtvaardige,~\sep\ en blijdschap voor de oprechten van hart.

  Verheugt u, rechtschapenen, in de Heer,~\sep\ en verheerlijkt Zijn heilige Naam!
\end{halfparskip}

\begin{halfparskip}
  \psalm{\Ps{98}} Zingt een nieuw lied voor de Heer,~\sep\ want wonderen heeft Hij gewrocht;

  Zege bracht Hem Zijn rechterhand,~\sep\ Zijn heilige arm.

  De Heer heeft Zijn heil doen kennen,~\sep\ voor het oog van de volken Zijn gerechtigheid getoond.

  Zijn liefde en trouw was Hij indachtig,~\sep\ ten gunste van Israëls huis.

  Alle grenzen der aarde hebben aanschouwd,~\sep\ het heil van onze God.

  Juicht voor de Heer, alle landen,~\sep\ weest blij, verheugt u en tokkelt de snaren.

  Zingt voor de Heer bij citerspel,~\sep\ en de klank van de harp,

  Met trompetten en bazuingeschal,~\sep\ jubelt voor het aanschijn van de Koning en Heer!

  De zee, met wat ze bevat, verheffe haar stem,~\sep\ de aarde en die haar bewonen;

  Dat de stromen in de handen klappen,~\sep\ en tegelijk de bergen juichen.

  Voor het aanschijn van de Heer, want Hij komt,~\sep\ want Hij komt de aarde regeren.

  Hij zal met rechtvaardigheid de wereld regeren,~\sep\ en de volkeren met billijkheid.
\end{halfparskip}

\begin{halfparskip}
  \liturgicalhint{Eer} --- \liturgicalhint{3x Alleluia.} --- \liturgicalhint{Eerste vers:} Zingt een nieuw lied voor de Heer,~\sep\ zingt voor de Heer alle landen!
\end{halfparskip}

\begin{halfparskip}
  \liturgicalOption{Vrijdagen ``voor'':} \liturgicalhint{Marmita 33 (\Pss{85--86}).}
\end{halfparskip}

\begin{halfparskip}
  \psalm{\Ps{85}} Gij zijt Uw land genadig geweest, o Heer,~\sep\ hebt het lot van Jacob ten goede gekeerd.

  \qanona{Alleluia, alleluia, alleluia.} --- \liturgicalhint{Eerste vers.}

  Vergeven hebt Gij de schuld van Uw volk,~\sep\ en al zijn zonden bedekt.

  Uw gramschap hebt Gij geheel bedwongen,~\sep\ de gloed van Uw toorn gestild.

  Herstel ons, o God, onze Redder,~\sep\ en leg Uw wrevel tegen ons af.

  Zult Gij dan eeuwig tegen ons toornen,~\sep\ of verbolgen blijven van geslacht tot geslacht?

  Zult Gij ons dan niet opnieuw doen leven,~\sep\ opdat Uw volk zich verblijde in U?

  Toon ons, Heer, Uw barmhartigheid,~\sep\ en schenk ons Uw heil!

  Ik wil horen naar wat de Heer God spreekt:~\sep\ vrede voorzeker kondigt Hij aan.

  Voor Zijn volk en Zijn heiligen,~\sep\ en voor hen, die zich van harte keren tot Hem.

  Ja waarlijk, Zijn heil is nabij voor wie Hem vrezen,~\sep\ en zo zal er glorie wonen in ons land:

  Barmhartigheid en trouw zullen elkander ontmoeten,~\sep\ gerechtigheid en vrede elkander de kus geven.

  Trouw zal aan de aarde ontspruiten,~\sep\ en gerechtigheid neerzien vanuit de hemel.

  De Heer zelf zal zegen schenken,~\sep\ en ons land zijn vruchten geven.

  Gerechtigheid zal vóór Hem uitgaan,~\sep\ en heil zijn schreden volgen.
\end{halfparskip}

\begin{halfparskip}
  \psalm{\Ps{86}} Neig Uw oor, o Heer; verhoor mij,~\sep\ want ik ben ellendig en arm.

  Bescherm mij, want ik ben U toegewijd;~\sep\ red Uw dienaar, die op U hoopt.

  Mijn God zijt Gij; wees mij genadig, o Heer,~\sep\ want almaar door roep ik tot U.

  Verblijd de ziel van Uw dienaar,~\sep\ want tot U, o Heer, verhef ik mijn ziel.

  Want Gij, o Heer, zijt goed en genadig,~\sep\ vol erbarming voor al wie U aanroept.

  Luister, Heer, naar mijn bede,~\sep\ en geef acht op de stem van mijn smeken.

  Op de dag van mijn kwelling riep ik tot U,~\sep\ omdat Gij mij verhoren zult.

  Onder de goden, o Heer, is er geen als Gij,~\sep\ en geen werk is gelijk aan het Uwe.

  Alle volken, door U geschapen, zullen komen, en U aanbidden, O Heer,~\sep\ en verheerlijken Uw Naam.

  Want Gij zijt groot en Gij doet wonderwerken:~\sep\ Gij zijt God, en Gij alleen.

  Toon mij Uw weg, o Heer, opdat ik wandele in Uw waarheid,~\sep\ richt mijn hart op de vrees voor Uw Naam.

  Ik zal U prijzen, Heer, mijn God, uit heel mijn hart:~\sep\ en eeuwig Uw Naam verheerlijken.

  Want Uw erbarming voor mij was groot,~\sep\ en uit de diepten van het dodenrijk hebt Gij mij opgehaald.

  Trotsen, o God, zijn tegen mij opgestaan, een bende geweldenaars staat mij naar het leven,~\sep\ zij houden U niet voor ogen.

  Maar Gij, o Heer, zijt een barmhartige en liefdevolle God,~\sep\ lankmoedig, rijk aan ontferming en trouw.

  Blik op mij neer en wees mij genadig;~\sep\ schenk aan Uw dienaar Uw kracht, en red de zoon van Uw
  dienstmaagd.

  Geef mij een teken van Uw gunst, opdat die mij haten, Heer, vol schaamte zien,~\sep\ dat Gij, o Heer, mij hulp en troost hebt geschonken.
\end{halfparskip}

\begin{halfparskip}
  \liturgicalhint{Eer} --- \liturgicalhint{3x Alleluia.} --- \liturgicalhint{Eerste vers:} Gij zijt Uw land genadig geweest, o Heer,~\sep\ hebt het lot van Jacob ten goede gekeerd.
\end{halfparskip}

\begin{halfparskip}
  \liturgicalOption{Vrijdagen ``midden'':} \liturgicalhint{Marmita 33 of 34 (\Pss{85--86} of 87--88).}
\end{halfparskip}

\begin{halfparskip}
  \liturgicalOption{Zaterdagen ``voor'':} \liturgicalhint{Marmita 56 (\Pss{144--146}).}
\end{halfparskip}

\begin{halfparskip}
  \psalm{\Ps{144}} Ik wil U roemen, mijn God, de Koning,~\sep\ en prijzen Uw Naam in eeuwen der eeuwen.

  \qanona{Alleluia, alleluia, alleluia.} --- \liturgicalhint{Eerste vers.}

  Dag aan dag wil ik U prijzen,~\sep\ en loven Uw Naam in de eeuwen der eeuwen.

  Groot is de Heer en hoogst lofwaardig,~\sep\ en Zijn grootheid is niet te meten.

  Het ene geslacht roemt bij het andere Uw werken,~\sep\ en verkondigt Uw macht.

  Zij spreken over de heerlijke luister van Uw majesteit,~\sep\ en maken Uw wonderwerken bekend.

  Zij bezingen de kracht van Uw ontzagwekkende daden,~\sep\ en verhalen Uw grootheid.

  Zij verkondigen de lof van Uw grote goedheid,~\sep\ en juichen om Uw gerechtigheid.

  Zachtmoedig en barmhartig is de Heer,~\sep\ lankmoedig en genaderijk.

  Goed is de Heer voor allen,~\sep\ en barmhartig voor al Zijn werken.

  Dat al Uw werken U roemen, Heer,~\sep\ en dat Uw heiligen U prijzen,

  Dat zij de glorie van Uw rijk bezingen,~\sep\ en Uw macht verkondigen,

  Om aan de zonen der mensen Uw macht te doen kennen,~\sep\ en de roem van Uw luistervol rijk.

  Uw rijk is een rijk van alle eeuwen,~\sep\ en Uw heerschappij duurt voort door alle geslachten heen.

  Trouw is de Heer in al Zijn woorden,~\sep\ en heilig in al Zijn werken.

  De Heer steunt allen, die dreigen te vallen,~\sep\ en alle terneergedrukten beurt Hij op.

  De ogen van allen zien vertrouwvol naar U,~\sep\ en Gij schenkt hun spijs te bekwamen tijde,

  Gij opent Uw hand,~\sep\ en verzadigt vol goedheid al wat leeft.

  Rechtvaardig is de Heer in al Zijn wegen,~\sep\ en heilig in al Zijn werken.

  De Heer is allen, die Hem aanroepen, nabij,~\sep\ allen, die Hem aanroepen in oprechtheid.

  Hij zal de wensen vervullen van hen, die Hem vrezen,~\sep\ hun smeken aanhoren en hen redden.

  De Heer behoedt allen, die Hem beminnen,~\sep\ maar alle bozen zal Hij verdelgen.

  Laat mijn mond verkondigen de lof van de Heer,~\sep\ en alle vlees prijzen zijn heilige Naam in alle eeuwen der eeuwen.
\end{halfparskip}

\begin{halfparskip}
  \psalm{\Ps{145}} Loof, mijn ziel, de Heer! Loven zal ik de Heer mijn leven lang,~\sep\ mijn God bezingen zolang ik besta.

  Vertrouwt niet op vorsten,~\sep\ niet op een mens, door wie geen redding komt.

  Als zijn geest is geweken, keert hij terug tot zijn stof;~\sep\ dan vallen al zijn plannen in duigen.

  Gelukkig hij, die Jacobs God tot helper heeft,~\sep\ wiens hoop is gevestigd op de Heer, zijn God,

  Die hemel en aarde gemaakt heeft,~\sep\ en de zee met al wat zij bevatten,

  Die trouw blijft voor eeuwig, recht schaft aan de verdrukten,~\sep\ en brood aan de hongerigen schenkt.

  De Heer bevrijdt de gevangenen,~\sep\ de Heer opent blinden de ogen;

  De Heer richt gebukten op,~\sep\ de Heer heeft rechtvaardigen lief.

  De Heer behoedt de vreemden, is voor wees en weduwe een steun;~\sep\ maar de weg der bozen verstoort Hij.

  De Heer zal heersen in eeuwigheid,~\sep\ uw God, o Sion, van geslacht tot geslacht.
\end{halfparskip}

\begin{halfparskip}
  \psalm{\Ps{146}} Looft de Heer want Hij is goed, bezingt onze God want Hij is liefelijk:~\sep\ Hem past lofbetuiging.

  De Heer bouwt Jeruzalem op,~\sep\ verzamelt de verstrooiden van Israël.

  Hij geneest de gebrokenen van harte,~\sep\ en verbindt hun wonden.

  Hij bepaalt het getal der sterren,~\sep\ Hij noemt ze elk bij hun naam.

  Groot is onze Heer en machtig in kracht;~\sep\ Zijn wijsheid kent geen grenzen.

  De Heer heft de nederigen op,~\sep\ drukt de bozen neer in het stof.

  Zingt de Heer een danklied toe,~\sep\ speelt voor onze God op de citer,

  Die de hemel bedekt met wolken,~\sep\ voor de aarde de regen bereidt,

  Die op de bergen het gras doet ontspruiten,~\sep\ en kruid ten dienste der mensen,

  Die voedsel schenkt aan het vee,~\sep\ aan de jongen der raven, die roepen tot Hem.

  Niet in de kracht van het ros heeft Hij behagen,~\sep\ noch in de schenkels van de man Zijn welgevallen.

  Behagen heeft de Heer in hen, die Hem vrezen,~\sep\ en op Zijn goedheid vertrouwen.
\end{halfparskip}

\begin{halfparskip}
  \liturgicalhint{Eer} --- \liturgicalhint{3x Alleluia.} --- \liturgicalhint{Eerste vers:} Ik wil U roemen, mijn God, de Koning,~\sep\ en prijzen Uw Naam in eeuwen der eeuwen.
\end{halfparskip}

\begin{halfparskip}
  \liturgicalOption{Maandagen ``na'':} \liturgicalhint{Marmita 5 (\Pss{15--17}).}
\end{halfparskip}

\begin{halfparskip}
  \psalm{\Ps{15}} Heer, wie mag in Uw tent verblijven,~\sep\ wie wonen op Uw heilige berg?

  \qanona{Alleluia, alleluia, alleluia.} --- \liturgicalhint{Eerste vers.}

  Die vlekkeloos wandelt en deugdzaam leeft, en in zijn hart wat goed is denkt,~\sep\ en niet lastert met zijn tong;

  Die zijn evenmens geen kwaad berokkent,~\sep\ en zijn nabuur geen smaad aandoet;

  Die de boze voor verachtelijk houdt,~\sep\ maar eert die vrezen de Heer;

  Die aan een schadelijke eed niet tornt, zijn geld niet uitleent met woeker,~\sep\ en onschuldigen ten koste geen steekpenning aanvaardt.

  Wie zo handelt,~\sep\ zal niet wankelen in eeuwigheid.
\end{halfparskip}

\begin{halfparskip}
  \psalm{\Ps{16}} Bewaar mij, God, want ik vlucht tot U.~\sep\ Ik zeg tot de Heer: Mijn Heer zijt Gij; geen geluk voor mij zonder U.

  Hoe schonk Hij mij voor de heiligen in Zijn land,~\sep\ een wondergrote liefde!

  Zij vermeerderen hun smarten,~\sep\ die achter vreemde goden lopen.

  Ik zal niet delen in hun plengoffers van bloed,~\sep\ noch hun namen op mijn lippen nemen.

  De Heer is mijn erfdeel, de dronk van mijn beker;~\sep\ Gij zijt het, die mijn lot in handen houdt.

  Voor mij viel het meetsnoer op heerlijke velden;~\sep\ ja, mijn erfdeel behaagt mij ten volle.

  Ik prijs de Heer, daar Hij mij inzicht gaf,~\sep\ en zelfs 's nachts mijn hart vermaant.

  De Heer houd ik immer voor ogen;~\sep\ omdat Hij staat aan mijn rechterzijde, zal ik niet wankelen.

  Daarom verheugt zich mijn hart en jubelt mijn ziel,~\sep\ zelfs mijn vlees zal in veiligheid rusten.

  Want mijn ziel zult Gij niet in het dodenrijk laten,~\sep\ Uw heilige het bederf niet doen zien.

  Gij zult de weg naar het leven mij tonen, overvloedige vreugden bij U,~\sep\ en geneugten voor eeuwig aan Uw rechterhand.
\end{halfparskip}

\begin{halfparskip}
  \psalm{\Ps{17}} Luister, Heer, naar een rechtvaardige zaak, geef acht op mijn geroep,~\sep\ hoor de bede van argeloze lippen.

  Van Uw aanschijn ga over mij het oordeel uit:~\sep\ Uw ogen zien wat recht is.

  Peil mijn hart, doorvors het 's nachts, beproef mij met vuur,~\sep\ geen onrecht zult Gij in mij vinden.

  Mijn mond misdeed niet zoals mensen gewoon zijn;~\sep\ naar de woorden van Uw lippen heb ik de wegen der Wet gevolgd.

  Vast drukten mijn schreden Uw paden,~\sep\ mijn voeten struikelden niet.

  Ik roep U aan, o God, want Gij zult mij verhoren;~\sep\ neig Uw oor naar mij en luister naar mijn bede!

  Toon U wonderbaar in Uw erbarmen,~\sep\ Gij die redt van weerstrevers al wie aan Uw zijde zijn toevlucht zoekt.

  Behoed me als de appel van het oog, verberg me in de schaduw van Uw vleugels,~\sep\ voor de zondaars die me geweld aandoen.

  Mijn vijanden omringen mij woedend; zij sluiten hun zinnelijk hart,~\sep\ hun mond spreekt trotse woorden.

  Hun schreden omringen mij thans;~\sep\ zij loeren om mij ter aarde te werpen.

  Ze zijn als de leeuw, die de muil spert naar prooi,~\sep\ als een leeuwenwelp, die in hinderlaag ligt.

  Rijs op, Heer, hem tegemoet en vel hem terneer, red mij door Uw zwaard van de boze,~\sep\ door Uw hand van mensen, o Heer,

  Van mensen, wier deel dit leven is,~\sep\ en wier schoot Gij vult met Uw schatten;

  Wier zonen zich verzadigen,~\sep\ en wat hun overblijft aan hun kinderen achterlaten.

  Ik echter zal door gerechtigheid Uw aanschijn aanschouwen,~\sep\ en mij bij het ontwaken met Uw aanblik verzadigen.
\end{halfparskip}

\begin{halfparskip}
  \liturgicalhint{Eer} --- \liturgicalhint{3x Alleluia.} --- \liturgicalhint{Eerste vers:} Heer, wie mag in Uw tent verblijven,~\sep\ wie wonen op Uw heilige berg?
\end{halfparskip}

\begin{halfparskip}
  \liturgicalOption{Dinsdagen ``na'':} \liturgicalhint{Marmita 10 (\Pss{28--30}).}
\end{halfparskip}

\begin{halfparskip}
  \psalm{\Ps{28}} Tot U roep ik, o Heer;~\sep\ mijn Rots, wees niet doof voor mij.

  \qanona{Alleluia, alleluia, alleluia.} --- \liturgicalhint{Eerste vers.}

  Opdat ik niet, als Gij niet hoort naar mij,~\sep\ gelijk worde aan hen, die in de grafkuil dalen.

  Hoor de stem van mijn smeken, nu ik roep tot U,~\sep\ nu ik mijn handen ophef naar Uw heilige tempel.

  Ruk mij niet weg met de zondaars,~\sep\ met hen, die kwaad bedrijven,

  Die vriendelijk spreken met hun naaste,~\sep\ maar in hun hart kwade bedoelingen koesteren.

  Handel met hen naar hun daden,~\sep\ en naar de boosheid van hun werken.

  Zet hun het werk van hun handen betaald,~\sep\ vergeld ze hun daden.

  Want ze slaan geen acht op de daden van de Heer en het werk van Zijn handen;~\sep\ Hij richte hen te gronde en heffe hen niet op.

  Gezegend de Heer, want Hij hoorde mijn dringende bede;~\sep\ de Heer, mijn kracht en mijn schild,

  Op Hem vertrouwde mijn hart, en ik ben geholpen;~\sep\ daarom jubelt mijn hart en prijs ik Hem met mijn zang.

  De Heer is een kracht voor Zijn volk,~\sep\ en voor Zijn Gezalfde een heilzame schutse.

  Red Uw volk en zegen Uw erfdeel;~\sep\ weid hen en draag hen voor eeuwig.
\end{halfparskip}

\begin{halfparskip}
  \psalm{\Ps{29}} Kent toe aan de Heer, zonen van God,~\sep\ kent toe aan de Heer glorie en macht!

  Kent toe aan de Heer de roem van Zijn Naam,~\sep\ aanbidt de Heer in heilige feesttooi.

  De stem van de Heer over de wateren! De God van majesteit doet de donder rollen:~\sep\ de Heer over de wijde wateren!

  De stem van de Heer vol kracht,~\sep\ de stem van de Heer vol majesteit!

  De stem van de Heer verbrijzelt de ceders,~\sep\ de Heer verbrijzelt de ceders van de Libanon.

  Hij doet de Libanon opspringen als een kalf,~\sep\ en de Sarion als het jong van een buffel.

  De stem van de Heer schiet vlammende schichten, de stem van de Heer doet de wildernis beven,~\sep\ de Heer doet Cades' wildernis beven.

  De stem van de Heer buigt eiken krom en ontschorst de bomen der wouden:~\sep\ en in Zijn tempel roepen allen: Glorie!

  De Heer troonde boven de watervloed,~\sep\ en de Heer zal tronen als Koning voor eeuwig.

  De Heer zal sterkte schenken aan Zijn volk,~\sep\ de Heer zal Zijn volk met vrede zegenen.
\end{halfparskip}

\begin{halfparskip}
  \psalm{\Ps{30}} Ik wil U roemen, o Heer, daar Gij mij gered hebt,~\sep\ en niet mijn vijanden over mij liet juichen.

  Heer, mijn God, ik riep tot U,~\sep\ en Gij hebt mij genezen.

  Heer, uit het dodenrijk hebt Gij mij weggevoerd,~\sep\ mij gered uit hen, die ten grave dalen.

  Speelt op de citer voor de Heer, gij, Zijn heiligen;~\sep\ en dankt Zijn heilige Naam.

  Want Zijn toorn duurt slechts een ogenblik,~\sep\ maar Zijn welwillendheid het hele leven door.

  's~Avonds komt er geween te gast,~\sep\ maar 's~morgens is er gejubel.

  In overmoed nu heb ik gezegd:~\sep\ ``In eeuwigheid zal ik niet wankelen.''

  Het was Uw gunst, o Heer, die mij ere schonk en macht;~\sep\ maar toen Gij Uw aanschijn verborgen hieldt, werd ik ontsteld.

  Ik roep tot U, o Heer,~\sep\ en smeek bij mijn God om erbarming:

  ``Wat kan mijn bloed U baten,~\sep\ of mijn neerdalen in het graf?

  Zal het stof U soms prijzen,~\sep\ of roemen Uw trouw?''

  Luister, o Heer, en wees mij genadig;~\sep\ o Heer, wees toch mijn helper!

  Gij hebt mijn rouw in een reidans veranderd,~\sep\ mijn rouwkleed verscheurd, mij met vreugde omgord,

  Opdat mijn ziel U zou prijzen en nimmermeer zwijgen.~\sep\ Heer, mijn God, ik zal U loven voor eeuwig!
\end{halfparskip}

\begin{halfparskip}
  \liturgicalhint{Eer} --- \liturgicalhint{3x Alleluia.} --- \liturgicalhint{Eerste vers:} Tot U roep ik, o Heer;~\sep\ mijn Rots, wees niet doof voor mij.
\end{halfparskip}

\begin{halfparskip}
  \liturgicalOption{Woensdagen ``na'':} \liturgicalhint{Marmita 24 (\Pss{65--67}).}
\end{halfparskip}

\begin{halfparskip}
  \psalm{\Ps{65}} Aan U, o God, komt een lofzang toe in Sion;~\sep\ men volbrenge zijn gelofte aan U, die de bede verhoort.

  \qanona{Alleluia, alleluia, alleluia.} --- \liturgicalhint{Eerste vers.}

  Tot U komt alle vlees,~\sep\ omwille der ongerechtigheden.

  Onze misdaden drukken ons neer:~\sep\ Gij scheldt ze kwijt.

  Gelukkig die Gij uitkiest en tot U neemt:~\sep\ hij woont in Uw voorhoven.

  Dat wij verzadigd worden met de goederen van Uw huis,~\sep\ met de heiligheid van Uw tempel.

  Met gerechtigheid verhoort Gij ons door wondere tekenen,~\sep\ God, onze Redder,

  Gij zijt de hoop van alle grenzen der aarde,~\sep\ en van de verre zeeën;

  Die de bergen vastlegt door Uw kracht,~\sep\ die met macht zijt omgord,

  Die het bulderen der zee bedwingt,~\sep\ het bulderen van haar golven en het woelen der naties.

  En die de grenzen der aarde bewonen, huiveren om Uw tekenen:~\sep\ met vreugde vervult Gij het uiterste oosten en westen.

  Gij hebt de aarde bezocht en haar besproeid,~\sep\ met rijkdommen haar overstelpt.

  De stroom van God is met water gevuld: Gij hebt hun graan bereid;~\sep\ zo hebt Gij haar gereed gemaakt:

  Haar voren hebt Gij besproeid,~\sep\ geëffend haar kluiten,

  Door stortregens hebt Gij haar geweekt,~\sep\ en haar gewas gezegend.

  Met Uw mildheid hebt Gij het jaar gekroond,~\sep\ en Uw wegen druipen van vet.

  De weiden der woestijn druipen ervan,~\sep\ en de heuvelen omgorden zich met jubel.

  De weiden zijn met kudden bekleed en de dalen met koren bedekt:~\sep\ zij juichen U toe en zingen!
\end{halfparskip}

\begin{halfparskip}
  \psalm{\Ps{66}} Juicht God toe, alle landen, bezingt de glorie van zijn Naam,~\sep\ heft voor Hem een heerlijk loflied aan.

  Zegt tot God: Hoe ontzagwekkend zijn Uw werken!~\sep\ Om Uw geweldige kracht brengen Uw vijanden U vleiend hulde.

  Dat heel de aarde U aanbidde en voor U zinge,~\sep\ dat zij bezinge Uw Naam.

  Komt en ziet de werken van God:~\sep\ wondere daden volbracht Hij onder de kinderen der mensen!

  De zee veranderde Hij in land, en zij trokken te voet door de stroom;~\sep\ laten wij daarom over Hem ons verheugen.

  Eeuwig heerst Hij door Zijn macht; Zijn ogen slaan de volken gade:~\sep\ dat de weerspannigen zich niet verheffen.

  Zegent, gij volken, onze God,~\sep\ en verkondigt Zijn wijd verbreide lof.

  Hij behield ons in leven,~\sep\ en liet onze voeten niet wankelen.

  Want Gij hebt ons beproefd, o God,~\sep\ met vuur ons gelouterd, zoals men zilver loutert.

  Gij liet ons de strik inlopen,~\sep\ een zware last hebt Gij op onze heupen gelegd.

  Mensen liet Gij over onze hoofden treden; wij zijn door vuur en water gegaan,~\sep\ maar Gij hebt ons uitkomst gebracht.

  Met brandoffers wil ik Uw huis betreden,~\sep\ U mijn geloften inlossen,

  Die mijn lippen hebben uitgesproken,~\sep\ en mijn mond heeft beloofd in mijn kwelling.

  Brandoffers wil ik U brengen van vette schapen met het vet van rammen;~\sep\ runderen en bokken zal ik offeren.

  Komt en hoort, gij allen, die God vreest,~\sep\ ik wil U verhalen hoe grote dingen Hij aan mij gedaan heeft!

  Ik riep Hem aan met mijn mond,~\sep\ en prees Hem met mijn tong.

  Had ik in mijn hart op boosheid gezonnen,~\sep\ dan had de Heer mij niet verhoord.

  Maar God heeft mij verhoord,~\sep\ heeft gelet op de stem van mijn smeken.

  Gezegend zij God, die mijn bede niet heeft versmaad,~\sep\ mij Zijn ontferming niet heeft onthouden.
\end{halfparskip}

\begin{halfparskip}
  \psalm{\Ps{67}} God zij ons genadig en zegene ons;~\sep\ Hij tone ons Zijn vredig gelaat.

  Opdat men op aarde Zijn weg lere kennen,~\sep\ onder alle volken Zijn heil.

  Dat de volken U prijzen, o God,~\sep\ dat alle volken U prijzen!

  Laat juichen en jubelen de naties, omdat Gij met rechtvaardigheid de volken regeert,~\sep\ en de naties op aarde bestuurt.

  Dat de volken U prijzen, o God,~\sep\ dat alle volken U prijzen!

  De aarde heeft haar vrucht gegeven;~\sep\ God, onze God, heeft ons gezegend.

  Dat God ons zegene,~\sep\ en dat alle grenzen der aarde Hem vrezen!
\end{halfparskip}

\begin{halfparskip}
  \liturgicalhint{Eer} --- \liturgicalhint{3x Alleluia.} --- \liturgicalhint{Eerste vers:} Aan U, o God, komt een lofzang toe in Sion;~\sep\ men volbrenge zijn gelofte aan
  U, die de bede verhoort.
\end{halfparskip}

\begin{halfparskip}
  \liturgicalOption{Donderdagen ``na'':} \liturgicalhint{Marmita 39* (\Pss{99--101}).}
\end{halfparskip}

\begin{halfparskip}
  \psalm{\Ps{99}} De Heer is Koning: de volken sidderen;~\sep\ Hij troont op de cherubs: de aarde beeft.

  \qanona{Alleluia, alleluia, alleluia.} --- \liturgicalhint{Eerste vers.}

  Groot is de Heer op de Sion,~\sep\ en boven alle volken verheven.

  Dat zij prijzen Uw grote en ontzagwekkende Naam:~\sep\ Hij toch is heilig.

  De Machtige heerst, die rechtvaardigheid bemint: wat recht is, hebt Gij gegrondvest,~\sep\ Gij handhaaft in Jacob gerechtigheid en recht.

  Prijst de Heer, onze God, en werpt u neer voor Zijn voetbank:~\sep\ Hij toch is heilig.

  Moses en Aäron zijn onder Zijn priesters, en Samuël onder hen die Zijn Naam aanriepen;~\sep\ zij riepen tot de Heer, en Hij schonk hun verhoring.

  In een wolkkolom sprak Hij tot hen:~\sep\ zij luisterden naar Zijn geboden, en naar de wet, die Hij hun gaf.

  Heer, onze God, Gij hebt hen verhoord;~\sep\ Gij waart hun genadig, o God, maar hun fouten hebt Gij gestraft.

  Prijst de Heer, onze God, en werpt u neer voor Zijn heilige berg,~\sep\ want heilig is de Heer, onze God.
\end{halfparskip}

\begin{halfparskip}
  \psalm{\Ps{100}} Juicht voor de Heer, alle landen,~\sep\ dient de Heer met vreugde;

  Treedt voor Zijn aanschijn~\sep\ met gejubel!

  Weet het wel: de Heer is God; Hij heeft ons gemaakt, Hem behoren we toe,~\sep\ Zijn volk zijn wij en de schapen van Zijn weide.

  Treedt Zijn poorten met lofzang binnen, Zijn voorhoven met jubelzang;~\sep\ brengt Hem hulde en zegent Zijn Naam.

  Want goed is de Heer: eeuwig duurt Zijn barmhartigheid,~\sep\ en Zijn trouw van geslacht tot geslacht.
\end{halfparskip}

\begin{halfparskip}
  \psalm{\Ps{101}} Van liefde en recht wil ik zingen,~\sep\ voor U de citer bespelen, o Heer.

  Op de weg der onschuld zal ik wandelen;~\sep\ wanneer zult Gij tot mij komen?

  Rein van hart wil ik leven,~\sep\ binnen mijn huis.

  Mijn ogen zal ik niet vestigen,~\sep\ op ongerechtigheid;

  Wie onrecht pleegt, is mij een gruwel:~\sep\ hij zal met mij geen omgang hebben.

  Een bedorven hart blijft verre van mij;~\sep\ van kwaad wil ik niets weten.

  Wie heimelijk zijn naaste belastert,~\sep\ die zal ik te gronde richten.

  De trotse blik en het hovaardig hart,~\sep\ zal ik niet dulden.

  Mijn ogen zien uit naar de getrouwen in het land,~\sep\ opdat ze bij mij wonen.

  Wie wandelt langs de goede weg,~\sep\ die zal mijn dienaar zijn.

  In mijn huis zal niet verblijven,~\sep\ die aan bedrog zich schuldig maakt.

  Wie leugens spreekt,~\sep\ houdt het niet uit onder mijn ogen.

  Dag aan dag zal ik verdelgen,~\sep\ alle boosdoeners in het land,

  Verbannen uit de stad van de Heer,~\sep\ allen die kwaad bedrijven.
\end{halfparskip}

\begin{halfparskip}
  \liturgicalhint{Eer} --- \liturgicalhint{3x Alleluia.} --- \liturgicalhint{Eerste vers:} De Heer is Koning: de volken sidderen; Hij troont op de cherubs: de aarde beeft.
\end{halfparskip}

\begin{halfparskip}
  \liturgicalOption{Vrijdagen ``na'':} \liturgicalhint{Marmita 34 (\Pss{87--88}).}
\end{halfparskip}

\begin{halfparskip}
  \psalm{\Ps{87}} Zijn stichting op de heilige bergen bemint de Heer:~\sep\ de poorten van Sion boven alle tenten van Jacob.

  \qanona{Alleluia, alleluia, alleluia.} --- \liturgicalhint{Eerste vers.}

  Roemrijke dingen verhaalt men van u,~\sep\ O stad van God!

  Rahab en Babel zal Ik tot Mijn vereerders rekenen:~\sep\ zie, Filistea en Tyrus en het volk der Ethiopiërs: daar zijn ze geboren!

  Over Sion zal men zeggen: ``Allen, man voor man, zijn in haar geboren,~\sep\ en de Allerhoogste zelf heeft haar bevestigd.''

  De Heer zal schrijven in het boek der volken:~\sep\ ``Daar zijn ze geboren.''

  En in reidans zullen zij zingen:~\sep\ ``Al mijn bronnen zijn in u.''
\end{halfparskip}

\begin{halfparskip}
  \psalm{\Ps{88}} Heer, mijn God, ik roep overdag,~\sep\ en ik jammer 's nachts voor Uw aanschijn.

  Dringe mijn bede toch door tot U,~\sep\ neig Uw oor naar mijn klagen!

  Want mijn ziel is verzadigd met rampen,~\sep\ mijn leven is het dodenrijk nabij.

  Ik word gerekend onder hen, die ten grave dalen,~\sep\ ik ben als een man zonder kracht.

  Onder de doden is mijn legerstede,~\sep\ als van verslagenen, die liggen in het graf,

  Aan wie Gij niet meer denkt,~\sep\ die aan Uw zorgen zijn onttrokken.

  In een diepe groeve hebt Gij mij neergelegd,~\sep\ in duisternis, in een diep ravijn.

  Uw verontwaardiging drukt zwaar op mij,~\sep\ met al Uw golven slaat Gij mij neer.

  Gij hebt mijn vrienden van mij vervreemd, mij tot afschuw voor hen gemaakt;~\sep\ ik zit gevangen, en kan niet ontkomen.

  Van ellende verkwijnen mijn ogen; iedere dag roep ik tot U, o Heer,~\sep\ naar U strek ik mijn handen uit.

  Of doet Gij voor doden nog wonderen,~\sep\ of zullen gestorvenen, herrijzend, U loven?

  Of wordt Uw goedheid in het graf verkondigd,~\sep\ Uw trouw in het dodenrijk?

  Openbaart men in het duister Uw wonderen,~\sep\ in het land der vergetelheid Uw genade?

  Ik echter roep tot U, o Heer,~\sep\ mijn bede stijgt tot U op in de morgen.

  Waarom toch, o Heer, verstoot Gij mij,~\sep\ verbergt Gij voor mij Uw gelaat?

  Van jongsaf ben ik ellendig en stervend,~\sep\ ik torste Uw verschrikkingen en kwijnde.

  Uw toorn is over mij heengegaan,~\sep\ Uw verschrikkingen sloegen mij neer.

  Zij omgeven mij immer als water,~\sep\ omringen mij alle tezamen.

  Vriend en makker hebt Gij van mij vervreemd,~\sep\ mijn vertrouweling is de duisternis.
\end{halfparskip}

\begin{halfparskip}
  \liturgicalhint{Eer} --- \liturgicalhint{3x Alleluia.} --- \liturgicalhint{Eerste vers:} Zijn stichting op de heilige bergen bemint de Heer:~\sep\ de poorten van Sion
  boven alle tenten van Jacob.
\end{halfparskip}

\begin{halfparskip}
  \liturgicalOption{Zaterdagen ``na'':} \liturgicalhint{Marmita 57 (\Pss{147:12--150}).}
\end{halfparskip}

\begin{halfparskip}
  \psalm{\Ps{147}} Loof, Jeruzalem, de Heer,~\sep\ loof Uw God, o Sion.

  \qanona{Alleluia, alleluia, alleluia.} --- \liturgicalhint{Eerste vers.}

  Want hecht heeft Hij gemaakt de grendels van uw poorten,~\sep\ uw zonen in u gezegend.

  Hij schonk vrede aan uw gebied,~\sep\ verzadigt u met bloem van tarwe.

  Hij zendt Zijn bevel naar de aarde,~\sep\ haastig ijlt Zijn uitspraak heen.

  De sneeuw doet Hij vallen als wol,~\sep\ de rijp spreidt Hij uit als as.

  Zijn ijs werpt Hij neer als kruimels brood,~\sep\ voor Zijn koude stollen de wateren.

  Hij geeft Zijn bevel en doet ze weer smelten;~\sep\ de wind doet Hij waaien, en de wateren stromen.

  Hij maakte Zijn gebod aan Jacob bekend,~\sep\ aan Israël Zijn wetten en bevelen.

  Zo deed Hij voor geen ander volk,~\sep\ hun openbaarde Hij Zijn wetten niet.
\end{halfparskip}

\begin{halfparskip}
  \psalm{\Ps{148}} Looft de Heer in de hemel,~\sep\ looft Hem in de hoge.

  Looft Hem al Zijn engelen,~\sep\ looft Hem al Zijn legerscharen!

  Looft Hem zon en maan,~\sep\ looft Hem alle fonkelende sterren!

  Looft Hem hemelen der hemelen,~\sep\ en gij wateren boven de hemel!

  Dat zij de Naam van de Heer loven,~\sep\ want Hij gebood, en ze waren geschapen;

  En Hij heeft ze gevestigd voor immer en eeuwig:~\sep\ Hij gaf een wet, die niet zal vergaan.

  Looft de Heer op aarde,~\sep\ monsters en alle diepten der zee,

  Vuur, hagel, sneeuw en nevel,~\sep\ stormwind, die Zijn bevel volbrengt,

  Bergen en alle heuvelen,~\sep\ vruchtbomen en alle ceders,

  Wilde dieren en alle vee,~\sep\ kruipende dieren en gevleugelde vogels,

  Koningen der aarde en alle volkeren,~\sep\ vorsten en alle rechters op aarde,

  Jongelingen zowel als maagden,~\sep\ grijsaards tezamen met kinderen:

  Dat zij de Naam van de Heer prijzen,~\sep\ want Zijn Naam alleen is verheven;

  Zijn majesteit gaat aarde en hemel te boven,~\sep\ en Hij heeft de hoorn van zijn volk verheven.

  Hij is de roem van al Zijn getrouwen,~\sep\ van de zonen van Israël, van het volk, dat Hem zo na is.
\end{halfparskip}

\begin{halfparskip}
  \psalm{\Ps{149}} Zingt een nieuw lied voor de Heer,~\sep\ Zijn lof weerklinke in de kring der heiligen!

  Laat Israël zich over zijn Schepper verheugen,~\sep\ de zonen van Sion jubelen over hun Koning.

  Dat zij Zijn Naam met reidans loven,~\sep\ Hem bezingen bij pauk en citer,

  Want de Heer bemint Zijn volk,~\sep\ en de nederigen kroont Hij met zege.

  Dat de heiligen juichen om de glorie,~\sep\ zich op hun legersteden verblijden!

  Gods lof zij in hun mond,~\sep\ en het tweesnijdend zwaard in hun hand,

  Om zich te wreken op de heidenen,~\sep\ om de volken te tuchtigen;

  Om hun koningen in ketens te slaan,~\sep\ en hun prinsen in ijzeren boeien,

  Om het vastgestelde vonnis aan hen te voltrekken:~\sep\ dat is de roem van al Zijn heiligen.
\end{halfparskip}

\begin{halfparskip}
  \psalm{\Ps{150}} Looft de Heer in Zijn heiligdom,~\sep\ looft Hem in Zijn verheven firmament.

  Looft Hem om Zijn grootse werken,~\sep\ looft Hem om Zijn hoogste majesteit.

  Looft Hem met bazuingeschal,~\sep\ looft Hem met harp en citer.

  Looft Hem met pauken en reidans,~\sep\ looft Hem met snaarinstrument en schalmei.

  Looft Hem met welluidende cimbalen,~\sep\ looft Hem met rinkelende cimbalen: al wat adem heeft, love de Heer!
\end{halfparskip}

\begin{halfparskip}
  \liturgicalhint{Eer} --- \liturgicalhint{3x Alleluia.} --- \liturgicalhint{Eerste vers:} Loof, Jeruzalem, de Heer,~\sep\ loof Uw God, o Sion.
\end{halfparskip}

\begin{halfparskip}
  \dd~Laat ons bidden, vrede zij met ons.

  \cc~Voor al Uw hulp en genaden die Gij ons gegeven hebt, waarvoor wij U nooit genoeg danken kunnen, prijzen en verheerlijken we U onophoudelijk in Uw gekroonde Kerk, voorzien van alle hulp en zegeningen, want Gij zijt de Heer en Schepper van alles, Vader, Zoon en Heilige Geest, in alle eeuwigheid. --- \rr~Amen.
\end{halfparskip}

% % % % % % % % % % % % % % % % % % % % % % % % % % % % % % % % % % % % % % % %

\markedsection{Laku Mara}

\vspace{0.4em}
\begin{doublecols}
  \textsizexi

  \englishl \liturgicalhint{1.}~You, Lord of all, we worship You~/ Jesus Christ, we exalt You~/ You give life to our bodies~/ and salvation to our souls!

  \dutchc{1} \liturgicalhint{1.}~U, Heer van alles prijzen wij; U, Jezus Christus loven wij; U bent de Levendmaker van onze lichamen; U bent de Verlosser van onze zielen.
\end{doublecols}

\begin{halfparskip}
  \dd~Ik was blij toen me mij zei: Wij gaan op naar het huis van de Heer. --- \rr~U, Heer van alles...

  (\liturgicaloption{Buiten kerken:} \dd~In elke plaats zijt Gij, Heer, aanvaard onze smeekbede. --- \rr~U, Heer van alles...)

  \cc~Eer aan de Vader, de Zoon, en de Heilige Geest. Vanaf het begin en in alle eeuwigheid, amen en amen.

  \rr~U, Heer van alles...

  \dd~Laat ons bidden. Vrede zij met ons.

  \cc~U bent waarlijk de Levendmaker van onze lichamen, de goede Verlosser van onze zielen, en de trouwe Bewaker van onze levens. U moeten wij altijd loven, aanbidden en verheerlijken, Heer van alles in alle eeuwigheid. --- \rr~Amen.
\end{halfparskip}

% % % % % % % % % % % % % % % % % % % % % % % % % % % % % % % % % % % % % % % %

\markedsection{Suraya D'Qdam}

\liturgicalhint{Na de eerste zin, zeg 3x alleluja, en herhaal het eerste vers. Op het einde: Eer aan... 3x alleluja.}

% % % % % % % % % % % % % % % % % % % % % % % % % % % % % % % % % % % % % % % %

\end{document}