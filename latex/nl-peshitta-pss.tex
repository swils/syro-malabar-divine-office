\documentclass[12pt,twoside,a5paper]{article}

\usepackage[main=english]{babel}
\usepackage{divine-office}

\setlength{\parskip}{0.6\origparskip}

% % % % % % % % % % % % % % % % % % % % % % % % % % % % % % % % % % % % % % % %

\begin{document}

% Translation based on R.A. Taylor's translation, revised by Fr. R. Matheus with the help of Dr. S. Brock.

\title{Dawidaya --- Psalterium + ondertitels}
\author{}
\date{}
\maketitle

% % % % % % % % % % % % % % % % % % % % % % % % % % % % % % % % % % % % % % % %

\hulala{1}

\Slota{Maak ons waardig, onze Heer en God, dat we mogen handelen met (of: geleid worden in) een deugdzaam gedrag dat Uw Majesteit behaagt, en dat onze wil volgens Uw wet mag zijn, en dat we er dag en nacht over mogen mediteren, Heer van alles, Vader, Zoon, en Heilige Geest in eeuwigheid. --- Amen.}

\marmita{1}

\PSALMtitle{1}{Verschillend levenslot}

\psalmsubtitle{a) De vromen ontvangen Gods zegen}

\begin{halfparskip}
  Zalig de man die de raad van goddelozen niet volgt,~\sep\ die de weg van zondaars niet inslaat, noch neerzit in de kring van spotters.

  \qanona{Heer, gezegend is hij, die Uw juk draagt, en dag en nacht Uw wet overweegt.}

  Maar die zijn vreugde vindt in de Wet van de Heer,~\sep\ en Zijn Wet overweegt bij dag en bij nacht.

  Hij is als een boom,~\sep\ geplant aan waterstromen.

  Die vrucht geeft op zijn tijd, en wiens lover niet verwelkt;~\sep\ ja, al wat hij doet, gedijt.
\end{halfparskip}

\psalmsubtitle{b) De goddelozen zullen vergaan}

\begin{halfparskip}
  Niet zo de goddelozen, niet zo;~\sep\ ze zijn als kaf, dat de wind verstrooit.

  Daarom zullen de goddelozen niet standhouden in het oordeel,~\sep\ noch de zondaars in de kring der rechtvaardigen.

  Want de Heer draagt zorg voor de weg der rechtvaardigen,~\sep\ maar de weg der goddelozen loopt uit op verderf.
\end{halfparskip}

\PSALMtitle{2}{DE Messias, opperste koning}

\psalmsubtitle{a) Tevergeefs staan de heidenen op tegen Christus}

\begin{halfparskip}
  Waarom woelen de heidenen,~\sep\ en smeden de naties ijdele plannen?

  \qanona{Als paarden zonder onderscheidingsvermogen raasden de arroganten en kruisigden de Messias.}

  De koningen der aarde rijzen samen op,~\sep\ en de vorsten spannen tezamen tegen de Heer en Zijn Gezalfde:

  "Laten wij Hun boeien verbreken,~\sep\ en werpen wij Hun kluisters van ons af".
\end{halfparskip}

\psalmsubtitle{b) De Heer spot met hen}

\begin{halfparskip}
  Die in de hemelen woont, Hij lacht hen uit,~\sep\ de Heer drijft de spot met hen.

  Dan spreekt Hij hen toe in Zijn toorn,~\sep\ in Zijn gramschap doet Hij hen sidderen:

  "Maar Ik, Ik heb Mijn Koning aangesteld,~\sep\ op de Sion, Mijn heilige berg!"
\end{halfparskip}

\psalmsubtitle{c) Opperheerschappij van Christus}

\begin{halfparskip}
  Ik wil het besluit van de Heer bekendmaken: de Heer sprak tot mij:~\sep\ "Mijn Zoon zijt Gij, Ik heb U heden voortgebracht.

  Vraag Mij, en Ik geef U de volken tot erfdeel,~\sep\ en tot Uw bezit de grenzen der aarde.

  Gij zult hen regeren met ijzeren scepter,~\sep\ hen in stukken slaan als het vat van een pottenbakker".
\end{halfparskip}

\psalmsubtitle{d) Koningen, dient de Heer}

\begin{halfparskip}
  Nu dan, koningen, komt tot inzicht,~\sep\ laat u gezeggen, die de wereld bestuurt.

  Dient de Heer in vreze en juicht Hem toe;~\sep\ met huivering Hem uw hulde gebracht!

  Opdat Hij niet toorne en gij van de weg af vergaat als weldra Zijn toorn zal zijn ontbrandt;~\sep\ zalig allen die vluchten tot Hem.
\end{halfparskip}

\PSALMtitle{3}{God is mijn beschermer}

\psalmsubtitle{a) Talrijk zijn mijn vijanden; Gij, Heer, helpt mij}

\begin{halfparskip}
  Heer, hoe talrijk zijn zij, die mij kwellen!~\sep\ Velen staan tegen mij op!

  \qanona{Terwijl ik sprak over Uw waarheid, O Heer, stonden de goddelozen tegen mij op. Red mij van hun geweld.}

  Velen zijn er, die van mij zeggen:~\sep\ "Voor hem is er geen redding bij God!"

  Maar Gij, Heer, zijt mijn schild,~\sep\ Gij mijn roem, die mijn hoofd opbeurt.
\end{halfparskip}

\psalmsubtitle{b) Op U, O God, vertrouw ik}

\begin{halfparskip}
  Ik riep tot de Heer met luide stem,~\sep\ en Hij verhoorde mij van Zijn heilige berg.

  Ik legde mij neer en ik sliep in;~\sep\ dan stond ik op, want de Heer is mijn steun.

  Neen, nu vrees ik de drommen van duizenden niet,~\sep\ die zich opstellen rondom mij heen.

  Verhef U, Heer!~\sep\ red mij, mijn God!

  Want al mijn weerstrevers hebt Gij op de kaken geslagen,~\sep\ de tanden der bozen hebt Gij verbrijzeld.

Bij de Heer is redding:~\sep\ Uw zegen zij over Uw volk!
\end{halfparskip}

\PSALMtitle{4}{Vertrouwvol avondgebed}

\begin{halfparskip}
  Verhoor mij, als ik U aanroep, mijn rechtvaardige God,

  \qanona{Er is niemand zoals de Heer, in wie ik heb vertrouwd. Hij redt mij uit de strikken en listen der bozen.}

  die mij in kwelling verlichting bracht;~\sep\ wees mij genadig en verhoor mijn gebed.
\end{halfparskip}

\psalmsubtitle{a) Vermaning tot bekering}

\begin{halfparskip}
  Mannen, hoelang nog blijft gij verstokt van hart;~\sep\ waarom ijdelheid bemind en leugen gezocht?

  Weet dat de Heer jegens Zijn heiligen wonderbaar handelt;~\sep\ de Heer zal mij verhoren, als ik Hem aanroep.

  Siddert en wilt niet zondigen,~\sep\ denkt na bij u zelf, op uw sponde, en zwijgt!

  Brengt gerechte offers,~\sep\ en hoopt op de Heer.
\end{halfparskip}

\psalmsubtitle{b) Gij helpt de vromen, O God}

\begin{halfparskip}
  Velen zeggen: "Wie zal ons voorspoed doen zien?"~\sep\ Doe opgaan over ons, Heer, het licht van Uw gelaat!

  Gij hebt mij een vreugde in het hart gestort,~\sep\ groter dan bij overvloed van tarwe en wijn.

  Zodra ik mij neerleg, slaap ik in vrede,~\sep\ want Gij alleen, Heer, stelt mij in veiligheid.
\end{halfparskip}

\Slota{Hoor de woorden van onze gebeden, Heer, neig Uw oor naar de klank van ons geween; en wend U niet af van het geluid van onze smeekbeden, O Goede, in wie wij altijd ons vertrouwen stellen, Heer van alles...}

\marmita{2}

\PSALMtitle{5}{Morgenbede in Gods tempel}

\psalmsubtitle{a) Heer, schenk mij Uw hulp!}

\begin{halfparskip}
  Luister, Heer, naar mijn woorden, geef acht op mijn zuchten.~\sep\ Let op mijn bede, mijn Koning en God!

  \qanona{U hebt mij terechtgewezen, Heer, om mij wijs te maken; wijs mijn verzoek niet af.}

  Tot U toch richt ik mijn bede, Heer; in de morgen hoort Gij mijn stem,~\sep\ in de morgen leg ik mijn bede voor U neer en wacht.
\end{halfparskip}

\psalmsubtitle{b) Heer, Gij vergeldt naar werken}

\begin{halfparskip}
  Neen, geen God zijt Gij, aan wie de boosheid behaagt; geen boze mag blijven bij U,~\sep\ noch houden goddelozen voor Uw aanschijn stand.

  Gij haat allen, die ongerechtigheid plegen,~\sep\ en stort alle leugenaars in het verderf;

  De bloeddorstige en de bedrieger~\sep\ zijn een afschuw voor de Heer.

  Maar dank aan Uw vele genaden,~\sep\ zal ik binnentreden in Uw huis,

  En mij neerwerpen voor Uw heilige tempel,~\sep\ in ontzag voor U, Heer.

  Geleid mij in Uw gerechtigheid omwille van mijn vijanden;~\sep\ baan Uw weg voor mij uit.

  Want geen oprechtheid is in hun mond,~\sep\ hun hart zint op bedrog;

  Een open graf is hun keel,~\sep\ vleitaal spreekt hun tong.

  Kastijd hen, o God;~\sep\ dat zij falen in hun plannen;

  Verdrijf hen om hun vele misdaden,~\sep\ want ze zijn weerspannig tegen U.

  Mogen allen zich verblijden, die vluchten tot U,~\sep\ en juichen voor immer!

  Wil hen beschermen, en dat in U zich verblijden,~\sep\ die Uw Naam beminnen!

  Want Gij, o Heer, zult de gerechtige zegenen;~\sep\ met welwillendheid hem omgeven als met een schild.
\end{halfparskip}

\PSALMtitle{6}{Boetgebed}

\psalmsubtitle{a) O God, wees mij genadig!}

\begin{halfparskip}
  Heer, straf mij niet in Uw toorn,~\sep\ en in Uw gramschap kastijd mij niet.

  \qanona{Heb medelijden met mijn zwakheid, mijn Schepper, en kastijd mij in Uw liefde.}

  Wees mij genadig, o Heer, omdat ik zwak ben;~\sep\ genees mij, Heer, want ontwricht is mijn gebeente.

  En mijn ziel is diep geschokt;~\sep\ maar Gij, Heer, hoelang nog?

  Keer terug, Heer, en bevrijd mij;~\sep\ red mij om Uw barmhartigheid.

  Want in de dood denkt niemand aan U;~\sep\ wie looft U in het dodenrijk?

  Door mijn zuchten ben ik afgetobd, ik besproei mijn sponde iedere nacht door mijn geween,~\sep\ en drenk met mijn tranen mijn rustbed.

  Mijn ogen staan dof van verdriet,~\sep\ en flets om allen, die mij haten.
\end{halfparskip}

\psalmsubtitle{b) Laat mijn vijanden afdeinzen, O Heer!}

\begin{halfparskip}
  Gaat weg van mij, gij allen, die onrecht pleegt,~\sep\ want de Heer heeft mijn schreien gehoord.

  De Heer heeft naar mijn smeken geluisterd,~\sep\ de Heer heeft mijn bede aanvaard.

  Dat al mijn vijanden zich schamen en hevig ontstellen,~\sep\ haastig vluchten, met schande overdekt!
\end{halfparskip}

\PSALMtitle{7}{Beroep op God}

\psalmsubtitle{a) Red mij, Heer, om mijn onschuld}

\begin{halfparskip}
  Heer, mijn God, naar U vlucht ik heen;~\sep\ verlos en bevrijd mij van al mijn vervolgers,

  \qanona{Geloofd zij God, die Zijn dienaren corrigeert en troost.}

  Opdat er geen als een leeuw mij het leven ontrove,~\sep\ mij verscheure, en niemand mij redt.

  Heer, mijn God, als ik dat heb gedaan,~\sep\ als er onrecht kleeft aan mijn handen,

  Als ik mijn vriend soms kwaad heb berokkend,~\sep\ terwijl ik toch spaarde die mij onrechtmatig bestreden:

  Dan mag de vijand mij achtervolgen en grijpen, op de grond mij vertreden,~\sep\ en mijn eer vergooien in het stof.
\end{halfparskip}

\psalmsubtitle{b) Verschaf mij recht, O Heer}

\begin{halfparskip}
  Rijs op, Heer, in Uw toorn, verhef U tegen de woede van mijn verdrukkers,~\sep\ en treed voor mij op in het gericht, door U bepaald.

  De vergaderde volken mogen U omringen;~\sep\ zetel boven hen uit in de hoge.

  De Heer is de Rechter der volken: doe mij recht, O Heer, naar mijn gerechtigheid,~\sep\ en naar de onschuld van mijn hart.

  De snoodheid der bozen neme een einde, maar geef de rechtvaardige kracht,~\sep\ Gij, rechtvaardige God, die harten en nieren doorgrondt.
\end{halfparskip}

\psalmsubtitle{c) God straft de onboetvaardige}

\begin{halfparskip}
  God is mij tot schild;~\sep\ Hij redt de oprechten van hart.

  God is een rechtvaardige Rechter,~\sep\ een God, die voortdurend bedreigt.

  Bekeert men zich niet, dan scherpt Hij Zijn zwaard,~\sep\ dan spant en richt Hij Zijn boog,

  Moordende schichten bereidt Hij voor hen,~\sep\ en gloeiend maakt Hij Zijn pijlen.

  Zie, door ongerechtigheid werd hij bevrucht, hij gaat zwanger van boosheid;~\sep\ en wat hij baart, is bedrog.

  Hij groef een kuil, en diepte hem uit,~\sep\ maar viel zelf in de groeve, die hij had gedolven.

  Op eigen hoofd zal zijn boosheid wederkeren,~\sep\ en neerdalen op eigen schedel zijn wreedheid.

  Maar ik zal de Heer om Zijn gerechtigheid prijzen,~\sep\ en de Naam van de allerhoogste Heer bezingen onder citerspel.
\end{halfparskip}

\slota{Heer, onze Heer, verborgen in Uw Wezen, die door de mond van kleintjes en kinderen Uw glorie heeft gevestigd, wij moeten U belijden, aanbidden en verheerlijken in alle seizoenen en tijden, Heer van alles...}

% % % % % % % % % % % % % % % % % % % % % % % % % % % % % % % % % % % % % % % %

\end{document}