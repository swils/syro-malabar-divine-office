\documentclass[12pt,twoside,a5paper]{article}

\usepackage[main=english]{babel}
\usepackage{divine-office}

\setlength{\parskip}{0.6\origparskip}

% % % % % % % % % % % % % % % % % % % % % % % % % % % % % % % % % % % % % % % %

\begin{document}

% Translation based on R.A. Taylor's translation, revised by Fr. R. Matheus with the help of Dr. S. Brock.

\title{Dawidaya --- Psalterium + ondertitels}
\author{}
\date{}
\maketitle

% % % % % % % % % % % % % % % % % % % % % % % % % % % % % % % % % % % % % % % %

\hulala{1}

\Slota{Maak ons waardig, onze Heer en God, dat we mogen handelen met (of: geleid worden in) een deugdzaam gedrag dat Uw Majesteit behaagt, en dat onze wil volgens Uw wet mag zijn, en dat we er dag en nacht over mogen mediteren, Heer van alles, Vader, Zoon, en Heilige Geest in eeuwigheid. --- Amen.}

\marmita{1}

\PSALMtitle{1}{Verschillend levenslot}

\psalmsubtitle{a) De vromen ontvangen Gods zegen}

\begin{halfparskip}
  Zalig de man die de raad van goddelozen niet volgt,~\sep\ die de weg van zondaars niet inslaat, noch neerzit in de kring van spotters.

  \qanona{Heer, gezegend is hij, die Uw juk draagt, en dag en nacht Uw wet overweegt.}

  Maar die zijn vreugde vindt in de Wet van de Heer,~\sep\ en Zijn Wet overweegt bij dag en bij nacht.

  Hij is als een boom,~\sep\ geplant aan waterstromen.

  Die vrucht geeft op zijn tijd, en wiens lover niet verwelkt;~\sep\ ja, al wat hij doet, gedijt.
\end{halfparskip}

\psalmsubtitle{b) De goddelozen zullen vergaan}

\begin{halfparskip}
  Niet zo de goddelozen, niet zo;~\sep\ ze zijn als kaf, dat de wind verstrooit.

  Daarom zullen de goddelozen niet standhouden in het oordeel,~\sep\ noch de zondaars in de kring der rechtvaardigen.

  Want de Heer draagt zorg voor de weg der rechtvaardigen,~\sep\ maar de weg der goddelozen loopt uit op verderf.
\end{halfparskip}

\PSALMtitle{2}{DE Messias, opperste koning}

\psalmsubtitle{a) Tevergeefs staan de heidenen op tegen Christus}

\begin{halfparskip}
  Waarom woelen de heidenen,~\sep\ en smeden de naties ijdele plannen?

  \qanona{Als paarden zonder onderscheidingsvermogen raasden de arroganten en kruisigden de Messias.}

  De koningen der aarde rijzen samen op,~\sep\ en de vorsten spannen tezamen tegen de Heer en Zijn Gezalfde:

  ``Laten wij Hun boeien verbreken,~\sep\ en werpen wij Hun kluisters van ons af''.
\end{halfparskip}

\psalmsubtitle{b) De Heer spot met hen}

\begin{halfparskip}
  Die in de hemelen woont, Hij lacht hen uit,~\sep\ de Heer drijft de spot met hen.

  Dan spreekt Hij hen toe in Zijn toorn,~\sep\ in Zijn gramschap doet Hij hen sidderen:

  ``Maar Ik, Ik heb Mijn Koning aangesteld,~\sep\ op de Sion, Mijn heilige berg!''
\end{halfparskip}

\psalmsubtitle{c) Opperheerschappij van Christus}

\begin{halfparskip}
  Ik wil het besluit van de Heer bekendmaken: de Heer sprak tot mij:~\sep\ ``Mijn Zoon zijt Gij, Ik heb U heden voortgebracht.

  Vraag Mij, en Ik geef U de volken tot erfdeel,~\sep\ en tot Uw bezit de grenzen der aarde.

  Gij zult hen regeren met ijzeren scepter,~\sep\ hen in stukken slaan als het vat van een pottenbakker''.
\end{halfparskip}

\psalmsubtitle{d) Koningen, dient de Heer}

\begin{halfparskip}
  Nu dan, koningen, komt tot inzicht,~\sep\ laat u gezeggen, die de wereld bestuurt.

  Dient de Heer in vreze en juicht Hem toe;~\sep\ met huivering Hem uw hulde gebracht!

  Opdat Hij niet toorne en gij van de weg af vergaat als weldra Zijn toorn zal zijn ontbrandt;~\sep\ zalig allen die vluchten tot Hem.
\end{halfparskip}

\PSALMtitle{3}{God is mijn beschermer}

\psalmsubtitle{a) Talrijk zijn mijn vijanden; Gij, Heer, helpt mij}

\begin{halfparskip}
  Heer, hoe talrijk zijn zij, die mij kwellen!~\sep\ Velen staan tegen mij op!

  \qanona{Terwijl ik sprak over Uw waarheid, O Heer, stonden de goddelozen tegen mij op. Red mij van hun geweld.}

  Velen zijn er, die van mij zeggen:~\sep\ ``Voor hem is er geen redding bij God!''

  Maar Gij, Heer, zijt mijn schild,~\sep\ Gij mijn roem, die mijn hoofd opbeurt.
\end{halfparskip}

\psalmsubtitle{b) Op U, O God, vertrouw ik}

\begin{halfparskip}
  Ik riep tot de Heer met luide stem,~\sep\ en Hij verhoorde mij van Zijn heilige berg.

  Ik legde mij neer en ik sliep in;~\sep\ dan stond ik op, want de Heer is mijn steun.

  Neen, nu vrees ik de drommen van duizenden niet,~\sep\ die zich opstellen rondom mij heen.

  Verhef U, Heer!~\sep\ red mij, mijn God!

  Want al mijn weerstrevers hebt Gij op de kaken geslagen,~\sep\ de tanden der bozen hebt Gij verbrijzeld.

Bij de Heer is redding:~\sep\ Uw zegen zij over Uw volk!
\end{halfparskip}

\PSALMtitle{4}{Vertrouwvol avondgebed}

\begin{halfparskip}
  Verhoor mij, als ik U aanroep, mijn rechtvaardige God,

  \qanona{Er is niemand zoals de Heer, in wie ik heb vertrouwd. Hij redt mij uit de strikken en listen der bozen.}

  die mij in kwelling verlichting bracht;~\sep\ wees mij genadig en verhoor mijn gebed.
\end{halfparskip}

\psalmsubtitle{a) Vermaning tot bekering}

\begin{halfparskip}
  Mannen, hoelang nog blijft gij verstokt van hart;~\sep\ waarom ijdelheid bemind en leugen gezocht?

  Weet dat de Heer jegens Zijn heiligen wonderbaar handelt;~\sep\ de Heer zal mij verhoren, als ik Hem aanroep.

  Siddert en wilt niet zondigen,~\sep\ denkt na bij u zelf, op uw sponde, en zwijgt!

  Brengt gerechte offers,~\sep\ en hoopt op de Heer.
\end{halfparskip}

\psalmsubtitle{b) Gij helpt de vromen, O God}

\begin{halfparskip}
  Velen zeggen: ``Wie zal ons voorspoed doen zien?''~\sep\ Doe opgaan over ons, Heer, het licht van Uw gelaat!

  Gij hebt mij een vreugde in het hart gestort,~\sep\ groter dan bij overvloed van tarwe en wijn.

  Zodra ik mij neerleg, slaap ik in vrede,~\sep\ want Gij alleen, Heer, stelt mij in veiligheid.
\end{halfparskip}

\Slota{Hoor de woorden van onze gebeden, Heer, neig Uw oor naar de klank van ons geween; en wend U niet af van het geluid van onze smeekbeden, O Goede, in wie wij altijd ons vertrouwen stellen, Heer van alles...}

\marmita{2}

\PSALMtitle{5}{Morgenbede in Gods tempel}

\psalmsubtitle{a) Heer, schenk mij Uw hulp!}

\begin{halfparskip}
  Luister, Heer, naar mijn woorden, geef acht op mijn zuchten.~\sep\ Let op mijn bede, mijn Koning en God!

  \qanona{U hebt mij terechtgewezen, Heer, om mij wijs te maken; wijs mijn verzoek niet af.}

  Tot U toch richt ik mijn bede, Heer; in de morgen hoort Gij mijn stem,~\sep\ in de morgen leg ik mijn bede voor U neer en wacht.
\end{halfparskip}

\psalmsubtitle{b) Heer, Gij vergeldt naar werken}

\begin{halfparskip}
  Neen, geen God zijt Gij, aan wie de boosheid behaagt; geen boze mag blijven bij U,~\sep\ noch houden goddelozen voor Uw aanschijn stand.

  Gij haat allen, die ongerechtigheid plegen,~\sep\ en stort alle leugenaars in het verderf;

  De bloeddorstige en de bedrieger~\sep\ zijn een afschuw voor de Heer.

  Maar dank aan Uw vele genaden,~\sep\ zal ik binnentreden in Uw huis,

  En mij neerwerpen voor Uw heilige tempel,~\sep\ in ontzag voor U, Heer.

  Geleid mij in Uw gerechtigheid omwille van mijn vijanden;~\sep\ baan Uw weg voor mij uit.

  Want geen oprechtheid is in hun mond,~\sep\ hun hart zint op bedrog;

  Een open graf is hun keel,~\sep\ vleitaal spreekt hun tong.

  Kastijd hen, o God;~\sep\ dat zij falen in hun plannen;

  Verdrijf hen om hun vele misdaden,~\sep\ want ze zijn weerspannig tegen U.

  Mogen allen zich verblijden, die vluchten tot U,~\sep\ en juichen voor immer!

  Wil hen beschermen, en dat in U zich verblijden,~\sep\ die Uw Naam beminnen!

  Want Gij, o Heer, zult de gerechtige zegenen;~\sep\ met welwillendheid hem omgeven als met een schild.
\end{halfparskip}

\PSALMtitle{6}{Boetgebed}

\psalmsubtitle{a) O God, wees mij genadig!}

\begin{halfparskip}
  Heer, straf mij niet in Uw toorn,~\sep\ en in Uw gramschap kastijd mij niet.

  \qanona{Heb medelijden met mijn zwakheid, mijn Schepper, en kastijd mij in Uw liefde.}

  Wees mij genadig, o Heer, omdat ik zwak ben;~\sep\ genees mij, Heer, want ontwricht is mijn gebeente.

  En mijn ziel is diep geschokt;~\sep\ maar Gij, Heer, hoelang nog?

  Keer terug, Heer, en bevrijd mij;~\sep\ red mij om Uw barmhartigheid.

  Want in de dood denkt niemand aan U;~\sep\ wie looft U in het dodenrijk?

  Door mijn zuchten ben ik afgetobd, ik besproei mijn sponde iedere nacht door mijn geween,~\sep\ en drenk met mijn tranen mijn rustbed.

  Mijn ogen staan dof van verdriet,~\sep\ en flets om allen, die mij haten.
\end{halfparskip}

\psalmsubtitle{b) Laat mijn vijanden afdeinzen, O Heer!}

\begin{halfparskip}
  Gaat weg van mij, gij allen, die onrecht pleegt,~\sep\ want de Heer heeft mijn schreien gehoord.

  De Heer heeft naar mijn smeken geluisterd,~\sep\ de Heer heeft mijn bede aanvaard.

  Dat al mijn vijanden zich schamen en hevig ontstellen,~\sep\ haastig vluchten, met schande overdekt!
\end{halfparskip}

\PSALMtitle{7}{Beroep op God}

\psalmsubtitle{a) Red mij, Heer, om mijn onschuld}

\begin{halfparskip}
  Heer, mijn God, naar U vlucht ik heen;~\sep\ verlos en bevrijd mij van al mijn vervolgers,

  \qanona{Geloofd zij God, die Zijn dienaren corrigeert en troost.}

  Opdat er geen als een leeuw mij het leven ontrove,~\sep\ mij verscheure, en niemand mij redt.

  Heer, mijn God, als ik dat heb gedaan,~\sep\ als er onrecht kleeft aan mijn handen,

  Als ik mijn vriend soms kwaad heb berokkend,~\sep\ terwijl ik toch spaarde die mij onrechtmatig bestreden:

  Dan mag de vijand mij achtervolgen en grijpen, op de grond mij vertreden,~\sep\ en mijn eer vergooien in het stof.
\end{halfparskip}

\psalmsubtitle{b) Verschaf mij recht, O Heer}

\begin{halfparskip}
  Rijs op, Heer, in Uw toorn, verhef U tegen de woede van mijn verdrukkers,~\sep\ en treed voor mij op in het gericht, door U bepaald.

  De vergaderde volken mogen U omringen;~\sep\ zetel boven hen uit in de hoge.

  De Heer is de Rechter der volken: doe mij recht, O Heer, naar mijn gerechtigheid,~\sep\ en naar de onschuld van mijn hart.

  De snoodheid der bozen neme een einde, maar geef de rechtvaardige kracht,~\sep\ Gij, rechtvaardige God, die harten en nieren doorgrondt.
\end{halfparskip}

\psalmsubtitle{c) God straft de onboetvaardige}

\begin{halfparskip}
  God is mij tot schild;~\sep\ Hij redt de oprechten van hart.

  God is een rechtvaardige Rechter,~\sep\ een God, die voortdurend bedreigt.

  Bekeert men zich niet, dan scherpt Hij Zijn zwaard,~\sep\ dan spant en richt Hij Zijn boog,

  Moordende schichten bereidt Hij voor hen,~\sep\ en gloeiend maakt Hij Zijn pijlen.

  Zie, door ongerechtigheid werd hij bevrucht, hij gaat zwanger van boosheid;~\sep\ en wat hij baart, is bedrog.

  Hij groef een kuil, en diepte hem uit,~\sep\ maar viel zelf in de groeve, die hij had gedolven.

  Op eigen hoofd zal zijn boosheid wederkeren,~\sep\ en neerdalen op eigen schedel zijn wreedheid.

  Maar ik zal de Heer om Zijn gerechtigheid prijzen,~\sep\ en de Naam van de allerhoogste Heer bezingen onder citerspel.
\end{halfparskip}

\slota{Heer, onze Heer, verborgen in Uw Wezen, die door de mond van kleintjes en kinderen Uw glorie heeft gevestigd, wij moeten U belijden, aanbidden en verheerlijken in alle seizoenen en tijden, Heer van alles...}

\marmita{3}

\PSALMtitle{8}{Hoe groot is de mens!}

\psalmsubtitle{a) Wat zijt Gij groot, O God, in de schepping!}

\begin{halfparskip}
  Heer, onze Heer, hoe wonderbaar is Uw Naam over heel de aarde;~\sep\ boven de hemelen hebt Gij Uw Majesteit verheven.

  \qanona{O Zoon, die door de kinderen in Jeruzalem werd geprezen met hun hosanna's, wij vragen U: red degenen die U aanbidden.}

  Ten spijt van Uw weerstrevers hebt Gij U lof bereid uit de mond van kind en zuigeling,~\sep\ om te beteugelen Uw vijand en hater.

  Als ik Uw hemelen zie, het werk van Uw vingeren,~\sep\ maan en sterren, die Gij hebt gegrondvest;

  Want is dan de mens, dat Gij hem gedenkt,~\sep\ of een mensenkind, dat Gij zorg voor hem draagt?
\end{halfparskip}

\psalmsubtitle{b) Wat hebt Gij de mens hoog verheven!}

\begin{halfparskip}
  Toch hebt Gij hem weinig minder dan de engelen gemaakt,~\sep\ met glorie en eer hem gekroond.

  Gij schonkt hem macht over de werken van Uw handen,~\sep\ alles hebt Gij onder zijn voeten gelegd:

  Alle schapen en runderen,~\sep\ ook de dieren in het wild,

  De vogels in de lucht en de vissen in de zee:~\sep\ al wat de paden der zeeën doorwandelt.

  Heer, onze Heer,~\sep\ hoe wonderbaar is Uw Naam over heel de aarde!
\end{halfparskip}

\PSALMtitle{9}{Gods rechtvaardig wereldbestuur}

\psalmsubtitle{a) Dank, Heer, voor de overwinning!}

\begin{halfparskip}
  Prijzen wil ik U, Heer, uit heel mijn hart,~\sep\ verhalen al Uw wonderwerken.

  \qanona{Wij danken U, want in Uw erbarmen hebt U ons gekeerd naar Uw wijsheid, laat onze tegenstanders worden beschaamd.}

  Om U wil ik juichen en jubelen,~\sep\ Uw Naam, Allerhoogste, bezingen.

  Want teruggeweken zijn mijn vijanden,~\sep\ ze zijn gevallen en kwamen om voor Uw aanschijn.

  Want Gij hebt mijn recht en mijn rechtszaak behartigd,~\sep\ Gij waart gezeten op Uw troon, als rechtvaardige Rechter.

  Gij hebt de heidenen getuchtigd, de goddeloze doen omkomen,~\sep\ hun naam uitgewist voor eeuwig.

  De vijanden kwamen om, voor immer ten onder gebracht;~\sep\ de steden hebt Gij verwoest: de gedachtenis aan hen is vergaan.

  Maar de Heer troont in eeuwigheid,~\sep\ en heeft Zijn rechterstoel gevestigd ten oordeel.

  En Hij zelf zal de wereld oordelen volgens recht,~\sep\ de volken richten volgens billijkheid.

  De Heer zal voor de verdrukte een toevlucht zijn,~\sep\ een veilige toevlucht in bange tijden.

  En die Uw Naam kennen, zullen hopen op U,~\sep\ want die U zoeken, Heer, verlaat Gij niet.
\end{halfparskip}

\psalmsubtitle{b) God, help mij in de strijd!}

\begin{halfparskip}
  Bezingt de Heer, die woont in Sion,~\sep\ verkondigt aan de volken Zijn daden.

  Want Hij, die bloedschuld wreekt, was hun indachtig,~\sep\ de kreten der armen vergeet Hij niet.

  Wees mij genadig, Heer, zie de ellende die ik lijd vanwege mijn vijanden,~\sep\ Gij die mij terugvoert van de poorten van de dood;

  Opdat ik al Uw lof verkondige bij de poorten der dochter van Sion,~\sep\ en juiche over Uw hulp.

  Bedolven zijn de volken in de kuil, die zij groeven;~\sep\ hun voet zit verward in de strik, die zij heimelijk hebben gelegd.

  De Heer heeft Zich geopenbaard, Hij heeft recht gesproken;~\sep\ in de werken van zijn handen ligt de zondaar verstrikt.

  Dat de bozen neerdalen in het dodenrijk,~\sep\ alle volken, die God zijn vergeten.

  Neen, niet voor immer zal de arme aan de vergetelheid worden prijsgegeven,~\sep\ niet voor immer de hoop der ellendigen worden beschaamd.

  Heer, sta op, laat de mens niet overmachtig worden;~\sep\ laten de heidenen voor Uw aanschijn worden gericht.

  Sla hen met ontzetting, Heer;~\sep\ dat de heidenen inzien, dat zij maar mensen zijn.
\end{halfparskip}

\PSALMtitle{10}{Bede voor hulp tegen vijanden}

\psalmsubtitle{a) De verdrukker minacht Gods gerechtigheid}

\begin{halfparskip}
  Waarom, Heer, blijft Gij veraf,~\sep\ verbergt Gij U in bange tijden.

  \qanona{Omdat de goddelozen hinderlagen hebben gelegd voor (\translationoptionNl{verraderlijk hebben gehandeld jegens}) de rechtvaardigen en Uw Naam hebben ontheiligd, vernietig, Heer, hun plannen!}

  Terwijl de goddeloze zich trots verheft, de arme verdrukt wordt~\sep\ en verstrikt ligt in de listen, die hij heeft uitgedacht.

  Ziet, de boze gaat groot op zijn driften,~\sep\ en de woekeraar lastert en minacht de Heer.

  De goddeloze spreekt in trots gemoed: ``Hij zal niet straffen;~\sep\ er bestaat geen God'': dit is heel zijn gedachtengang.

  Zijn wandel is immer voorspoedig,~\sep\ hij bekommert zich niet om Uw gerichten, al zijn weerstrevers veracht hij.

  Hij zegt bij zichzelf: ``Ik zal niet wankelen,~\sep\ van geslacht tot geslacht zal geen onheil mij treffen.''

  Zijn mond is vol verwensingen, vol list en bedrog;~\sep\ smart en kwelling kleeft aan zijn tong.

  Hij ligt bij de dorpen in hinderlaag, doodt de onschuldige in het geheim;~\sep\ zijn ogen bespieden de ongelukkige.

  Hij ligt op de loer in zijn schuilplaats als een leeuw in zijn hol; hij bespiedt de ongelukkige om hem te grijpen:~\sep\ hij sleept de rampzalige weg en trekt hem in zijn net.

  Hij bukt, werpt zich op de grond,~\sep\ en onder zijn geweld bezwijken de armen.

  Hij zegt bij zichzelf: ``God denkt er niet aan,~\sep\ Hij wendt Zijn aangezicht af, ziet nooit naar hem om.''
\end{halfparskip}

\psalmsubtitle{b) Heer, wreek de verdrukte!}

\begin{halfparskip}
  Rijs op, Heer, God, hef Uw hand omhoog;~\sep\ vergeet de armen niet!

  Waarom blijft de boze God tergen,~\sep\ en spreekt hij bij zichzelf: ``Neen, Hij zal toch niet straffen?''

  Maar Gij ziet toe: lijden en smart staan U voor ogen,~\sep\ om het in Uw handen te nemen.

  Op U verlaat zich de ongelukkige,~\sep\ Gij zijt de Helper der wezen.

  Verbrijzel de arm van zondaar en boze;~\sep\ zijn boosheid zult Gij straffen, en geen spoor blijve er van over.

  De Heer is Koning in eeuwigheid,~\sep\ in Zijn land zijn de heidenen vernietigd.

  Het verlangen der ellendigen hebt Gij gehoord, Heer;~\sep\ Gij hebt hun hart versterkt, Uw oor naar hen geneigd,

  Om het recht te beschermen van wees en verdrukte,~\sep\ zodat geen mens ter wereld hen nog vrees aanjaagt.
\end{halfparskip}

% % % % % % % % % % % % % % % % % % % % % % % % % % % % % % % % % % % % % % % %

\PSALMtitle{}{}

\psalmsubtitle{}

\begin{halfparskip}
  \qanona{}
\end{halfparskip}

\psalmsubtitle{}

\begin{halfparskip}
\end{halfparskip}

% % % % % % % % % % % % % % % % % % % % % % % % % % % % % % % % % % % % % % % %

\end{document}