\documentclass[12pt,twoside,a5paper]{article}

\usepackage[main=dutch]{babel}
\usepackage{divine-office}

\setlength{\parskip}{0.6\origparskip}

% % % % % % % % % % % % % % % % % % % % % % % % % % % % % % % % % % % % % % % %

\begin{document}

% Translation based on R.A. Taylor's translation, revised by Fr. R. Matheus with the help of Dr. S. Brock.

\title{Dawidaya --- Psalterium + ondertitels}
\author{}
\date{}
\maketitle

% % % % % % % % % % % % % % % % % % % % % % % % % % % % % % % % % % % % % % % %

\hulala{1}

\Slota{Maak ons waardig, onze Heer en God, dat we mogen handelen met (of: geleid worden in) een deugdzaam gedrag dat Uw Majesteit behaagt, en dat onze wil volgens Uw wet mag zijn, en dat we er dag en nacht over mogen mediteren, Heer van alles, Vader, Zoon, en Heilige Geest in eeuwigheid. --- Amen.}

\marmita{1}

\PSALMtitle{1}{Verschillend levenslot}

\psalmsubtitle{a) De vromen ontvangen Gods zegen}

\begin{halfparskip}
  Zalig de man die de raad van goddelozen niet volgt,~\sep\ die de weg van zondaars niet inslaat, noch neerzit in de kring van spotters.

  \qanona{Heer, gezegend is hij, die Uw juk draagt, en dag en nacht Uw wet overweegt.}

  Maar die zijn vreugde vindt in de Wet van de Heer,~\sep\ en Zijn Wet overweegt bij dag en bij nacht.

  Hij is als een boom,~\sep\ geplant aan waterstromen.

  Die vrucht geeft op zijn tijd, en wiens lover niet verwelkt;~\sep\ ja, al wat hij doet, gedijt.
\end{halfparskip}

\psalmsubtitle{b) De goddelozen zullen vergaan}

\begin{halfparskip}
  Niet zo de goddelozen, niet zo;~\sep\ ze zijn als kaf, dat de wind verstrooit.

  Daarom zullen de goddelozen niet standhouden in het oordeel,~\sep\ noch de zondaars in de kring der rechtvaardigen.

  Want de Heer draagt zorg voor de weg der rechtvaardigen,~\sep\ maar de weg der goddelozen loopt uit op verderf.
\end{halfparskip}

\PSALMtitle{2}{DE Messias, opperste koning}

\psalmsubtitle{a) Tevergeefs staan de heidenen op tegen Christus}

\begin{halfparskip}
  Waarom woelen de heidenen,~\sep\ en smeden de naties ijdele plannen?

  \qanona{Als paarden zonder onderscheidingsvermogen raasden de arroganten en kruisigden de Messias.}

  De koningen der aarde rijzen samen op,~\sep\ en de vorsten spannen tezamen tegen de Heer en Zijn Gezalfde:

  ``Laten wij Hun boeien verbreken,~\sep\ en werpen wij Hun kluisters van ons af''.
\end{halfparskip}

\psalmsubtitle{b) De Heer spot met hen}

\begin{halfparskip}
  Die in de hemelen woont, Hij lacht hen uit,~\sep\ de Heer drijft de spot met hen.

  Dan spreekt Hij hen toe in Zijn toorn,~\sep\ in Zijn gramschap doet Hij hen sidderen:

  ``Maar Ik, Ik heb Mijn Koning aangesteld,~\sep\ op de Sion, Mijn heilige berg!''
\end{halfparskip}

\psalmsubtitle{c) Opperheerschappij van Christus}

\begin{halfparskip}
  Ik wil het besluit van de Heer bekendmaken: de Heer sprak tot mij:~\sep\ ``Mijn Zoon zijt Gij, Ik heb U heden voortgebracht.

  Vraag Mij, en Ik geef U de volken tot erfdeel,~\sep\ en tot Uw bezit de grenzen der aarde.

  Gij zult hen regeren met ijzeren scepter,~\sep\ hen in stukken slaan als het vat van een pottenbakker''.
\end{halfparskip}

\psalmsubtitle{d) Koningen, dient de Heer}

\begin{halfparskip}
  Nu dan, koningen, komt tot inzicht,~\sep\ laat u gezeggen, die de wereld bestuurt.

  Dient de Heer in vreze en juicht Hem toe;~\sep\ met huivering Hem uw hulde gebracht!

  Opdat Hij niet toorne en gij van de weg af vergaat als weldra Zijn toorn zal zijn ontbrandt;~\sep\ zalig allen die vluchten tot Hem.
\end{halfparskip}

\PSALMtitle{3}{God is mijn beschermer}

\psalmsubtitle{a) Talrijk zijn mijn vijanden; Gij, Heer, helpt mij}

\begin{halfparskip}
  Heer, hoe talrijk zijn zij, die mij kwellen!~\sep\ Velen staan tegen mij op!

  \qanona{Terwijl ik sprak over Uw waarheid, O Heer, stonden de goddelozen tegen mij op. Red mij van hun geweld.}

  Velen zijn er, die van mij zeggen:~\sep\ ``Voor hem is er geen redding bij God!''

  Maar Gij, Heer, zijt mijn schild,~\sep\ Gij mijn roem, die mijn hoofd opbeurt.
\end{halfparskip}

\psalmsubtitle{b) Op U, O God, vertrouw ik}

\begin{halfparskip}
  Ik riep tot de Heer met luide stem,~\sep\ en Hij verhoorde mij van Zijn heilige berg.

  Ik legde mij neer en ik sliep in;~\sep\ dan stond ik op, want de Heer is mijn steun.

  Neen, nu vrees ik de drommen van duizenden niet,~\sep\ die zich opstellen rondom mij heen.

  Verhef U, Heer!~\sep\ red mij, mijn God!

  Want al mijn weerstrevers hebt Gij op de kaken geslagen,~\sep\ de tanden der bozen hebt Gij verbrijzeld.

Bij de Heer is redding:~\sep\ Uw zegen zij over Uw volk!
\end{halfparskip}

\PSALMtitle{4}{Vertrouwvol avondgebed}

\begin{halfparskip}
  Verhoor mij, als ik U aanroep, mijn rechtvaardige God,

  \qanona{Er is niemand zoals de Heer, in wie ik heb vertrouwd. Hij redt mij uit de strikken en listen der bozen.}

  die mij in kwelling verlichting bracht;~\sep\ wees mij genadig en verhoor mijn gebed.
\end{halfparskip}

\psalmsubtitle{a) Vermaning tot bekering}

\begin{halfparskip}
  Mannen, hoelang nog blijft gij verstokt van hart;~\sep\ waarom ijdelheid bemind en leugen gezocht?

  Weet dat de Heer jegens Zijn heiligen wonderbaar handelt;~\sep\ de Heer zal mij verhoren, als ik Hem aanroep.

  Siddert en wilt niet zondigen,~\sep\ denkt na bij u zelf, op uw sponde, en zwijgt!

  Brengt gerechte offers,~\sep\ en hoopt op de Heer.
\end{halfparskip}

\psalmsubtitle{b) Gij helpt de vromen, O God}

\begin{halfparskip}
  Velen zeggen: ``Wie zal ons voorspoed doen zien?''~\sep\ Doe opgaan over ons, Heer, het licht van Uw gelaat!

  Gij hebt mij een vreugde in het hart gestort,~\sep\ groter dan bij overvloed van tarwe en wijn.

  Zodra ik mij neerleg, slaap ik in vrede,~\sep\ want Gij alleen, Heer, stelt mij in veiligheid.
\end{halfparskip}

\Slota{Hoor de woorden van onze gebeden, Heer, neig Uw oor naar de klank van ons geween; en wend U niet af van het geluid van onze smeekbeden, O Goede, in wie wij altijd ons vertrouwen stellen, Heer van alles...}

\marmita{2}

\PSALMtitle{5}{Morgenbede in Gods tempel}

\psalmsubtitle{a) Heer, schenk mij Uw hulp!}

\begin{halfparskip}
  Luister, Heer, naar mijn woorden, geef acht op mijn zuchten.~\sep\ Let op mijn bede, mijn Koning en God!

  \qanona{U hebt mij terechtgewezen, Heer, om mij wijs te maken; wijs mijn verzoek niet af.}

  Tot U toch richt ik mijn bede, Heer; in de morgen hoort Gij mijn stem,~\sep\ in de morgen leg ik mijn bede voor U neer en wacht.
\end{halfparskip}

\psalmsubtitle{b) Heer, Gij vergeldt naar werken}

\begin{halfparskip}
  Neen, geen God zijt Gij, aan wie de boosheid behaagt; geen boze mag blijven bij U,~\sep\ noch houden goddelozen voor Uw aanschijn stand.

  Gij haat allen, die ongerechtigheid plegen,~\sep\ en stort alle leugenaars in het verderf;

  De bloeddorstige en de bedrieger~\sep\ zijn een afschuw voor de Heer.

  Maar dank aan Uw vele genaden,~\sep\ zal ik binnentreden in Uw huis,

  En mij neerwerpen voor Uw heilige tempel,~\sep\ in ontzag voor U, Heer.

  Geleid mij in Uw gerechtigheid omwille van mijn vijanden;~\sep\ baan Uw weg voor mij uit.

  Want geen oprechtheid is in hun mond,~\sep\ hun hart zint op bedrog;

  Een open graf is hun keel,~\sep\ vleitaal spreekt hun tong.

  Kastijd hen, o God;~\sep\ dat zij falen in hun plannen;

  Verdrijf hen om hun vele misdaden,~\sep\ want ze zijn weerspannig tegen U.

  Mogen allen zich verblijden, die vluchten tot U,~\sep\ en juichen voor immer!

  Wil hen beschermen, en dat in U zich verblijden,~\sep\ die Uw Naam beminnen!

  Want Gij, o Heer, zult de gerechtige zegenen;~\sep\ met welwillendheid hem omgeven als met een schild.
\end{halfparskip}

\PSALMtitle{6}{Boetgebed}

\psalmsubtitle{a) O God, wees mij genadig!}

\begin{halfparskip}
  Heer, straf mij niet in Uw toorn,~\sep\ en in Uw gramschap kastijd mij niet.

  \qanona{Heb medelijden met mijn zwakheid, mijn Schepper, en kastijd mij in Uw liefde.}

  Wees mij genadig, o Heer, omdat ik zwak ben;~\sep\ genees mij, Heer, want ontwricht is mijn gebeente.

  En mijn ziel is diep geschokt;~\sep\ maar Gij, Heer, hoelang nog?

  Keer terug, Heer, en bevrijd mij;~\sep\ red mij om Uw barmhartigheid.

  Want in de dood denkt niemand aan U;~\sep\ wie looft U in het dodenrijk?

  Door mijn zuchten ben ik afgetobd, ik besproei mijn sponde iedere nacht door mijn geween,~\sep\ en drenk met mijn tranen mijn rustbed.

  Mijn ogen staan dof van verdriet,~\sep\ en flets om allen, die mij haten.
\end{halfparskip}

\psalmsubtitle{b) Laat mijn vijanden afdeinzen, O Heer!}

\begin{halfparskip}
  Gaat weg van mij, gij allen, die onrecht pleegt,~\sep\ want de Heer heeft mijn schreien gehoord.

  De Heer heeft naar mijn smeken geluisterd,~\sep\ de Heer heeft mijn bede aanvaard.

  Dat al mijn vijanden zich schamen en hevig ontstellen,~\sep\ haastig vluchten, met schande overdekt!
\end{halfparskip}

\PSALMtitle{7}{Beroep op God}

\psalmsubtitle{a) Red mij, Heer, om mijn onschuld}

\begin{halfparskip}
  Heer, mijn God, naar U vlucht ik heen;~\sep\ verlos en bevrijd mij van al mijn vervolgers,

  \qanona{Geloofd zij God, die Zijn dienaren corrigeert en troost.}

  Opdat er geen als een leeuw mij het leven ontrove,~\sep\ mij verscheure, en niemand mij redt.

  Heer, mijn God, als ik dat heb gedaan,~\sep\ als er onrecht kleeft aan mijn handen,

  Als ik mijn vriend soms kwaad heb berokkend,~\sep\ terwijl ik toch spaarde die mij onrechtmatig bestreden:

  Dan mag de vijand mij achtervolgen en grijpen, op de grond mij vertreden,~\sep\ en mijn eer vergooien in het stof.
\end{halfparskip}

\psalmsubtitle{b) Verschaf mij recht, O Heer}

\begin{halfparskip}
  Rijs op, Heer, in Uw toorn, verhef U tegen de woede van mijn verdrukkers,~\sep\ en treed voor mij op in het gericht, door U bepaald.

  De vergaderde volken mogen U omringen;~\sep\ zetel boven hen uit in de hoge.

  De Heer is de Rechter der volken: doe mij recht, O Heer, naar mijn gerechtigheid,~\sep\ en naar de onschuld van mijn hart.

  De snoodheid der bozen neme een einde, maar geef de rechtvaardige kracht,~\sep\ Gij, rechtvaardige God, die harten en nieren doorgrondt.
\end{halfparskip}

\psalmsubtitle{c) God straft de onboetvaardige}

\begin{halfparskip}
  God is mij tot schild;~\sep\ Hij redt de oprechten van hart.

  God is een rechtvaardige Rechter,~\sep\ een God, die voortdurend bedreigt.

  Bekeert men zich niet, dan scherpt Hij Zijn zwaard,~\sep\ dan spant en richt Hij Zijn boog,

  Moordende schichten bereidt Hij voor hen,~\sep\ en gloeiend maakt Hij Zijn pijlen.

  Zie, door ongerechtigheid werd hij bevrucht, hij gaat zwanger van boosheid;~\sep\ en wat hij baart, is bedrog.

  Hij groef een kuil, en diepte hem uit,~\sep\ maar viel zelf in de groeve, die hij had gedolven.

  Op eigen hoofd zal zijn boosheid wederkeren,~\sep\ en neerdalen op eigen schedel zijn wreedheid.

  Maar ik zal de Heer om Zijn gerechtigheid prijzen,~\sep\ en de Naam van de allerhoogste Heer bezingen onder citerspel.
\end{halfparskip}

\slota{Heer, onze Heer, verborgen in Uw Wezen, die door de mond van kleintjes en kinderen Uw glorie heeft gevestigd, wij moeten U belijden, aanbidden en verheerlijken in alle seizoenen en tijden, Heer van alles...}

\marmita{3}

\PSALMtitle{8}{Hoe groot is de mens!}

\psalmsubtitle{a) Wat zijt Gij groot, O God, in de schepping!}

\begin{halfparskip}
  Heer, onze Heer, hoe wonderbaar is Uw Naam over heel de aarde;~\sep\ boven de hemelen hebt Gij Uw Majesteit verheven.

  \qanona{O Zoon, die door de kinderen in Jeruzalem werd geprezen met hun hosanna's, wij vragen U: red degenen die U aanbidden.}

  Ten spijt van Uw weerstrevers hebt Gij U lof bereid uit de mond van kind en zuigeling,~\sep\ om te beteugelen Uw vijand en hater.

  Als ik Uw hemelen zie, het werk van Uw vingeren,~\sep\ maan en sterren, die Gij hebt gegrondvest;

  Want is dan de mens, dat Gij hem gedenkt,~\sep\ of een mensenkind, dat Gij zorg voor hem draagt?
\end{halfparskip}

\psalmsubtitle{b) Wat hebt Gij de mens hoog verheven!}

\begin{halfparskip}
  Toch hebt Gij hem weinig minder dan de engelen gemaakt,~\sep\ met glorie en eer hem gekroond.

  Gij schonkt hem macht over de werken van Uw handen,~\sep\ alles hebt Gij onder zijn voeten gelegd:

  Alle schapen en runderen,~\sep\ ook de dieren in het wild,

  De vogels in de lucht en de vissen in de zee:~\sep\ al wat de paden der zeeën doorwandelt.

  Heer, onze Heer,~\sep\ hoe wonderbaar is Uw Naam over heel de aarde!
\end{halfparskip}

\PSALMtitle{9}{Gods rechtvaardig wereldbestuur}

\psalmsubtitle{a) Dank, Heer, voor de overwinning!}

\begin{halfparskip}
  Prijzen wil ik U, Heer, uit heel mijn hart,~\sep\ verhalen al Uw wonderwerken.

  \qanona{Wij danken U, want in Uw erbarmen hebt U ons gekeerd naar Uw wijsheid, laat onze tegenstanders worden beschaamd.}

  Om U wil ik juichen en jubelen,~\sep\ Uw Naam, Allerhoogste, bezingen.

  Want teruggeweken zijn mijn vijanden,~\sep\ ze zijn gevallen en kwamen om voor Uw aanschijn.

  Want Gij hebt mijn recht en mijn rechtszaak behartigd,~\sep\ Gij waart gezeten op Uw troon, als rechtvaardige Rechter.

  Gij hebt de heidenen getuchtigd, de goddeloze doen omkomen,~\sep\ hun naam uitgewist voor eeuwig.

  De vijanden kwamen om, voor immer ten onder gebracht;~\sep\ de steden hebt Gij verwoest: de gedachtenis aan hen is vergaan.

  Maar de Heer troont in eeuwigheid,~\sep\ en heeft Zijn rechterstoel gevestigd ten oordeel.

  En Hij zelf zal de wereld oordelen volgens recht,~\sep\ de volken richten volgens billijkheid.

  De Heer zal voor de verdrukte een toevlucht zijn,~\sep\ een veilige toevlucht in bange tijden.

  En die Uw Naam kennen, zullen hopen op U,~\sep\ want die U zoeken, Heer, verlaat Gij niet.
\end{halfparskip}

\psalmsubtitle{b) God, help mij in de strijd!}

\begin{halfparskip}
  Bezingt de Heer, die woont in Sion,~\sep\ verkondigt aan de volken Zijn daden.

  Want Hij, die bloedschuld wreekt, was hun indachtig,~\sep\ de kreten der armen vergeet Hij niet.

  Wees mij genadig, Heer, zie de ellende die ik lijd vanwege mijn vijanden,~\sep\ Gij die mij terugvoert van de poorten van de dood;

  Opdat ik al Uw lof verkondige bij de poorten der dochter van Sion,~\sep\ en juiche over Uw hulp.

  Bedolven zijn de volken in de kuil, die zij groeven;~\sep\ hun voet zit verward in de strik, die zij heimelijk hebben gelegd.

  De Heer heeft Zich geopenbaard, Hij heeft recht gesproken;~\sep\ in de werken van zijn handen ligt de zondaar verstrikt.

  Dat de bozen neerdalen in het dodenrijk,~\sep\ alle volken, die God zijn vergeten.

  Neen, niet voor immer zal de arme aan de vergetelheid worden prijsgegeven,~\sep\ niet voor immer de hoop der ellendigen worden beschaamd.

  Heer, sta op, laat de mens niet overmachtig worden;~\sep\ laten de heidenen voor Uw aanschijn worden gericht.

  Sla hen met ontzetting, Heer;~\sep\ dat de heidenen inzien, dat zij maar mensen zijn.
\end{halfparskip}

\PSALMtitle{10}{Bede voor hulp tegen vijanden}

\psalmsubtitle{a) De verdrukker minacht Gods gerechtigheid}

\begin{halfparskip}
  Waarom, Heer, blijft Gij veraf,~\sep\ verbergt Gij U in bange tijden.

  \qanona{Omdat de goddelozen hinderlagen hebben gelegd voor (\translationoptionNl{verraderlijk hebben gehandeld jegens}) de rechtvaardigen en Uw Naam hebben ontheiligd, vernietig, Heer, hun plannen!}

  Terwijl de goddeloze zich trots verheft, de arme verdrukt wordt~\sep\ en verstrikt ligt in de listen, die hij heeft uitgedacht.

  Ziet, de boze gaat groot op zijn driften,~\sep\ en de woekeraar lastert en minacht de Heer.

  De goddeloze spreekt in trots gemoed: ``Hij zal niet straffen;~\sep\ er bestaat geen God'': dit is heel zijn gedachtengang.

  Zijn wandel is immer voorspoedig,~\sep\ hij bekommert zich niet om Uw gerichten, al zijn weerstrevers veracht hij.

  Hij zegt bij zichzelf: ``Ik zal niet wankelen,~\sep\ van geslacht tot geslacht zal geen onheil mij treffen.''

  Zijn mond is vol verwensingen, vol list en bedrog;~\sep\ smart en kwelling kleeft aan zijn tong.

  Hij ligt bij de dorpen in hinderlaag, doodt de onschuldige in het geheim;~\sep\ zijn ogen bespieden de ongelukkige.

  Hij ligt op de loer in zijn schuilplaats als een leeuw in zijn hol; hij bespiedt de ongelukkige om hem te grijpen:~\sep\ hij sleept de rampzalige weg en trekt hem in zijn net.

  Hij bukt, werpt zich op de grond,~\sep\ en onder zijn geweld bezwijken de armen.

  Hij zegt bij zichzelf: ``God denkt er niet aan,~\sep\ Hij wendt Zijn aangezicht af, ziet nooit naar hem om.''
\end{halfparskip}

\psalmsubtitle{b) Heer, wreek de verdrukte!}

\begin{halfparskip}
  Rijs op, Heer, God, hef Uw hand omhoog;~\sep\ vergeet de armen niet!

  Waarom blijft de boze God tergen,~\sep\ en spreekt hij bij zichzelf: ``Neen, Hij zal toch niet straffen?''

  Maar Gij ziet toe: lijden en smart staan U voor ogen,~\sep\ om het in Uw handen te nemen.

  Op U verlaat zich de ongelukkige,~\sep\ Gij zijt de Helper der wezen.

  Verbrijzel de arm van zondaar en boze;~\sep\ zijn boosheid zult Gij straffen, en geen spoor blijve er van over.

  De Heer is Koning in eeuwigheid,~\sep\ in Zijn land zijn de heidenen vernietigd.

  Het verlangen der ellendigen hebt Gij gehoord, Heer;~\sep\ Gij hebt hun hart versterkt, Uw oor naar hen geneigd,

  Om het recht te beschermen van wees en verdrukte,~\sep\ zodat geen mens ter wereld hen nog vrees aanjaagt.
\end{halfparskip}

% % % % % % % % % % % % % % % % % % % % % % % % % % % % % % % % % % % % % % % %

\hulala{2}

\Slota{Bevestig, O Heer, Uw vertrouwen in ons, en vul onze ziel met Uw hulp; moge Uw genade onze zonden vergeven, en mogen de eeuwige barmhartigheid van Uw glorieuze Drie-eenheid Uw aanbidders die U aanroepen en U smeken, te hulp komen in alle seizoenen en tijden, Heer van alles...}

\marmita{4}

\PSALMtitle{11}{Wankelt niet}

\psalmsubtitle{a) Het gevaar dringt tot vluchten}

\begin{halfparskip}
  Ik vlucht tot de Heer; hoe kunt gij mij zeggen:~\sep\ ``Vlieg weg als een vogel naar het gebergte.

  \qanona{De zondaars hebben mij bedrogen; Ik heb op U vertrouwd, Heer!}

  Want zie, de bozen spannen de boog, ze zetten de pijl op de pees,~\sep\ om de oprechten van hart in het duister te treffen.

  Als zelfs de grondvesten worden gesloopt,~\sep\ wat zal de gerechtige dan nog vermogen?''
\end{halfparskip}

\psalmsubtitle{b) Gij, Heer, zijt de Beschermer van het recht}

\begin{halfparskip}
  De Heer woont in Zijn heilige tempel,~\sep\ de Heer heeft in de hemel Zijn troon.

  Zijn ogen zien rond,~\sep\ Zijn wimpers doorvorsen de kinderen der mensen.

  De Heer doorvorst de gerechte en de boze;~\sep\ die het onrecht liefheeft, is Hem een gruwel.

  Hij zal op de zondaars gloeiende kolen en zwavel doen regenen;~\sep\ een verzengende wind is de dronk van hun beker.

  Want de Heer is rechtvaardig en heeft de gerechtigheid lief;~\sep\ de goeden zullen Zijn aanschijn aanschouwen.
\end{halfparskip}

\PSALMtitle{12}{Troost in ontrouw}

\psalmsubtitle{a) De wereld is trouweloos}

\begin{halfparskip}
  Heer, schenk redding, want er zijn geen vromen meer;~\sep\ verdwenen is de trouw onder de kinderen der mensen.

  \qanona{Het bedrog is toegenomen en de liefde afgenomen; keer u niet van ons af, O Messias!}

  Allen liegen ze elkander voor,~\sep\ ze spreken met bedrieglijke lippen en vals gemoed.

  De Heer rukke al die bedrieglijke lippen uit,~\sep\ die grootsprekende tong,

  Hen, die zeggen: ``Sterk zijn wij door onze tong;~\sep\ wij hebben onze lippen met ons: wie kan ons overmeesteren?''
\end{halfparskip}

\psalmsubtitle{b) Gij, Heer, zijt getrouw}

\begin{halfparskip}
  ``Om de nood der verdrukten en het gejammer der armen zal Ik nu opstaan,'' zegt de Heer:~\sep\ ``redding zal Ik brengen aan wie er naar smacht.''

  De woorden van de Heer zijn oprechte woorden,~\sep\ zuiver zilver, van stof ontdaan, tot zevenmaal gelouterd.

  Gij, Heer, zult ons behouden,~\sep\ ons eeuwig beschermen tegen dit geslacht.

  De bozen zwermen om ons heen,~\sep\ terwijl de heffe van het volk oprijst.
\end{halfparskip}

\PSALMtitle{13}{Vertrouwvolle noodkreet}

\psalmsubtitle{a) Groot is mijn ellende}

\begin{halfparskip}
  Hoelang nog, Heer, zult Gij mij geheel vergeten,~\sep\ hoelang nog voor mij Uw aanschijn verbergen?

  \qanona{Verzoen u met mij, Heer, en red mij, zodat ik U kan prijzen.}

  Hoelang nog zal ik de smart overdenken in mijn ziel,~\sep\ en het wee in mijn hart van dag tot dag?

  Hoelang nog zal mijn vijand zich boven mij verheffen?~\sep\ Zie neer en verhoor mij, O Heer, mijn God!
\end{halfparskip}

\psalmsubtitle{b) Red mij, Heer!}

\begin{halfparskip}
  Stort licht in mijn ogen, opdat ik de doodsslaap niet inga,~\sep\ en mijn vijand niet zegge: ``Ik heb hem overwonnen'';

  En mijn weerstrevers niet juichen over mijn val,~\sep\ daar ik op Uw erbarming vertrouwde.

  Nu juiche mijn hart om Uw hulp!~\sep\ De Heer wil ik bezingen, die mij heeft welgedaan.
\end{halfparskip}

\PSALMtitle{14}{Het bederf is groot}

\psalmsubtitle{a) Boos zijn de mensen}

\begin{halfparskip}
  De dwaas zegt bij zichzelf:~\sep\ ``Er is geen God.''~\sep\ Ze zijn bedorven, gruwelen hebben ze bedreven;~\sep

  \qanona{Red Uw Kerk van de bozen, Heer van alle schepselen!}

  daar is er niet één, die deugdzaam handelt.

  De Heer blikt uit de hemel neer op de kinderen der mensen,~\sep\ om te zien of er wel één is met verstand, wel één die God zoekt.

  Allen zonder uitzondering zijn ze afgedwaald, allen diep bedorven;~\sep\ daar is er niet één, die deugdzaam handelt, niet één.
\end{halfparskip}

\psalmsubtitle{b) Gij, Heer, straft de bozen}

\begin{halfparskip}
  Zullen al die bozen dan nimmer tot inzicht komen,~\sep\ zij, die Mijn volk verslinden, als aten zij brood?

  Zij riepen de Heer niet aan. Eens zullen zij sidderen van angst,~\sep\ want God is met het geslacht der rechtvaardigen.

  Het beleid van de verdrukte wilt gij beschamen,~\sep\ maar de Heer is zijn toevlucht.

  O, mocht er uit Sion toch heil voor Israël dagen! Als de Heer het lot van Zijn volk ten goede keert,~\sep\ zal er gejubel zijn in Jacob, vreugde in Israël.
\end{halfparskip}

\slota{Maak ons waardig, onze Heer en onze God, met een geweten gezuiverd en geheiligd door Uw waarheid, in Uw heilig tabernakel te wonen en onberispelijk Uw weg te bewandelen alle dagen van ons leven, Heer van alles…}

\marmita{5}

\PSALMtitle{15}{'s Heren gast}

\begin{halfparskip}
  Heer, wie mag in Uw tent verblijven,~\sep\ wie wonen op Uw heilige berg?

  \qanona{Laat mij staan met zuivere gedachten voor Uw heilig altaar, O Heer.}

  Die vlekkeloos wandelt en deugdzaam leeft, en in zijn hart wat goed is denkt,~\sep\ en niet lastert met zijn tong;

  Die zijn evenmens geen kwaad berokkent,~\sep\ en zijn nabuur geen smaad aandoet;

  Die de boze voor verachtelijk houdt,~\sep\ maar eert die vrezen de Heer;

  Die aan een schadelijke eed niet tornt, zijn geld niet uitleent met woeker,~\sep\ en onschuldigen ten koste geen steekpenning aanvaardt.

  Wie zo handelt,~\sep\ zal niet wankelen in eeuwigheid.
\end{halfparskip}

\PSALMtitle{16}{God is mijn hoop voor hier en hiernamaals}

\psalmsubtitle{a) Gij, God, zijt mijn enig goed!}

\begin{halfparskip}
  Bewaar mij, God, want ik vlucht tot U;~\sep\ ik zeg tot de Heer: ``Mijn Heer zijt Gij; geen geluk voor mij zonder U''.

  \qanona{Glorierijk is het vertrouwen in U, O onze Schepper, want in haar verheugt zich mijn zwakheid.}

  Hoe schonk Hij mij voor de heiligen in Zijn land,~\sep\ een wondergrote liefde!

  Zij vermeerderen hun smarten,~\sep\ die achter vreemde goden lopen.

  Ik zal niet delen in hun plengoffers van bloed,~\sep\ noch hun namen op mijn lippen nemen.

  De Heer is mijn erfdeel, de dronk van mijn beker;~\sep\ Gij zijt het, die mijn lot in handen houdt.

  Voor mij viel het meetsnoer op heerlijke velden;~\sep\ ja, mijn erfdeel behaagt mij ten volle.
\end{halfparskip}

\psalmsubtitle{b) Mijn erfdeel, O Heer, van eeuwig leven}

\begin{halfparskip}
  Ik prijs de Heer, daar Hij mij inzicht gaf,~\sep\ en zelfs 's nachts mijn hart vermaant.

  De Heer houd ik immer voor ogen;~\sep\ omdat Hij staat aan mijn rechterzijde, zal ik niet wankelen.

  Daarom verheugt zich mijn hart en jubelt mijn ziel,~\sep\ zelfs mijn vlees zal in veiligheid rusten.

  Want mijn ziel zult Gij niet in het dodenrijk laten,~\sep\ Uw heilige het bederf niet doen zien.

  Gij zult de weg naar het leven mij tonen, overvloedige vreugden bij U,~\sep\ en geneugten voor eeuwig aan Uw rechterhand.
\end{halfparskip}

\PSALMtitle{17}{Vertrouwvol gebed in uiterste nood}

\psalmsubtitle{a) Heer, wreek mijn onschuld}

\begin{halfparskip}
  Luister, Heer, naar een rechtvaardige zaak, geef acht op mijn geroep,~\sep\ hoor de bede van argeloze lippen.

  \qanona{Mijn Heer en mijn God, heb medelijden met mij; want ik word onrechtvaardig vervolgd!}

  Van Uw aanschijn ga over mij het oordeel uit:~\sep\ Uw ogen zien wat recht is.

  Peil mijn hart, doorvors het 's nachts, beproef mij met vuur,~\sep\ geen onrecht zult Gij in mij vinden.

  Mijn mond misdeed niet zoals mensen gewoon zijn;~\sep\ naar de woorden van Uw lippen heb ik de wegen der Wet gevolgd.

  Vast drukten mijn schreden Uw paden,~\sep\ mijn voeten struikelden niet.
\end{halfparskip}

\psalmsubtitle{b) God, red mij van de bozen!}

\begin{halfparskip}
  Ik roep U aan, o God, want Gij zult mij verhoren;~\sep\ neig Uw oor naar mij en luister naar mijn bede!

  Toon U wonderbaar in Uw erbarmen,~\sep\ Gij die redt van weerstrevers al wie aan Uw zijde zijn toevlucht zoekt.

  Behoed me als de appel van het oog, verberg me in de schaduw van Uw vleugels,~\sep\ voor de zondaars die me geweld aandoen.

  Mijn vijanden omringen mij woedend; zij sluiten hun zinnelijk hart,~\sep\ hun mond spreekt trotse woorden.

  Hun schreden omringen mij thans;~\sep\ zij loeren om mij ter aarde te werpen.

  Ze zijn als de leeuw, die de muil spert naar prooi,~\sep\ als een leeuwenwelp, die in hinderlaag ligt.
\end{halfparskip}

\psalmsubtitle{c) Wees, Heer, een God van vergelding}

\begin{halfparskip}
  Rijs op, Heer, hem tegemoet en vel hem terneer, red mij door Uw zwaard van de boze,~\sep\ door Uw hand van mensen, O Heer,

  Van mensen, wier deel dit leven is,~\sep\ en wier schoot Gij vult met Uw schatten;

  Wier zonen zich verzadigen,~\sep\ en wat hun overblijft aan hun kinderen achterlaten.

  Ik echter zal door gerechtigheid Uw aanschijn aanschouwen,~\sep\ en mij bij het ontwaken met Uw aanblik verzadigen.
\end{halfparskip}

\slota{Wij moeten U belijden, aanbidden en verheerlijken, O glorieuze Kracht van Uw dienaren, sterke Hoop van Uw aanbidders, en machtige Toevlucht van hen die U vrezen, Helper die de hoorn van onze verlossing verheft, in alle seizoenen en tijden, Heer van alles...}

\marmita{6}

\PSALMtitle{18}{Davids triomfantelijk lied}

\psalmsubtitle{a) God, mijn Redder, ik bemin U}

\begin{halfparskip}
  Ik heb U lief, o Heer, mijn Sterkte,~\sep\ Heer, mijn Rots, mijn Burcht, mijn Bevrijder;

  \qanona{Hemel en aarde en alles wat in hen is, de hemelse en aardse wezens, knielen en aanbidden God, hun Schepper.}

  Mijn God, mijn Rotswand, waarheen ik vlucht,~\sep\ mijn Schild, de Hoorn van mijn heil, mijn Toeverlaat.

  Aanroepen zal ik de Heer, de Lofwaardige,~\sep\ en van mijn vijanden worden verlost.

  Mij omspoelden de golven van de dood,~\sep\ en vernietigende stromen ontstelden mij.

  De strikken van het dodenrijk omknelden mij,~\sep\ de boeien van de dood vielen op mij neer.

  In mijn nood riep ik tot de Heer,~\sep\ en mijn geschrei steeg op tot mijn God;

  En Hij hoorde mijn stem vanuit Zijn tempel,~\sep\ en mijn hulpgeroep drong door tot Zijn oren.
\end{halfparskip}

\psalmsubtitle{b) Gods ingrijpen onder het beeld van een storm}

\begin{halfparskip}
  Daar schudde de aarde en beefde; de grondvesten der bergen werden geschokt,~\sep\ en zij dreunden want Hij brandde van toorn.

  Rook steeg uit Zijn neusgaten op, verslindend vuur uit Zijn mond,~\sep\ gloeiende kolen sprongen van Hem uit.

  Hij haalde de wolkenhemel neer en daalde af,~\sep\ en zwarte wolken hingen onder Zijn voeten.

  Hij voer op de cherub en vloog,~\sep\ op de wieken van de wind werd Hij gedragen.

  Hij omhulde zich met duisternis als met een kleed,~\sep\ met donkere nevels en dichte wolken als met een mantel.

  Door de gloed vóór Hem uit,~\sep\ ontbrandden gloeiende kolen.

  En de Heer deed de donder rollen uit de hemel,~\sep\ en weergalmen deed de Allerhoogste Zijn stem.

  En Hij schoot Zijn pijlen af en dreef hen uiteen,~\sep\ talloze flitsen, en Hij velde hen neer.

  En de bodem der zee kwam te voorschijn,~\sep\ en het fundament der aarde lag bloot.

  Door het dreigen van de Heer,~\sep\ door de ademtocht van Zijn toorn.
\end{halfparskip}

\psalmsubtitle{c) Redding uit diepste nood}

\begin{halfparskip}
  Hij strekte Zijn hand uit de hoge, Hij greep mij aan,~\sep\ en trok mij op uit de watervloed.

  Hij bevrijdde mij van mijn geweldige vijand,~\sep\ en van hen, die mij haatten, die machtiger waren dan ik.

  Zij overvielen mij op de dag van mijn rampspoed,~\sep\ maar tot bescherming was mij de Heer.

  En Hij leidde mij uit in het vrije veld;~\sep\ Hij heeft mij gered, omdat Hij mij liefheeft.
\end{halfparskip}

\psalmsubtitle{d) Mijn God, ik was U trouw}

\begin{halfparskip}
  Zo loonde mij de Heer naar mijn gerechtigheid;~\sep\ naar de reinheid van mijn handen vergold Hij mij.

  Want de wegen van de Heer heb ik gevolgd,~\sep\ door geen zonde ben ik afgeweken van mijn God.

  Ja, al Zijn geboden hield ik voor ogen,~\sep\ en Zijn wetten wierp ik niet van mij af.

  Maar voor Zijn aanschijn was ik rein,~\sep\ en ik heb mij behoed voor de zonde.

  Zo vergold mij de Heer naar mijn gerechtigheid,~\sep\ naar de reinheid van mijn handen voor Zijn ogen.
\end{halfparskip}

\psalmsubtitle{e) Gij, God, vergeldt naar werken}

\begin{halfparskip}
  Met de vrome handelt Gij liefdevol,~\sep\ met de rechtschapene rechtschapen;

  Voor de reine toont Gij U rein,~\sep\ met de sluwe handelt Gij slim.

  Want Gij redt het nederige volk,~\sep\ maar trotse blikken slaat Gij neer.

  Ja, Gij doet mijn lamp schijnen, O Heer;~\sep\ mijn God, mijn duisternis maakt Gij tot licht.

  Ja, met U storm ik los op de drommen der vijanden,~\sep\ en met mijn God bespring ik de wallen.

  Gods wegen zijn volmaakt, het woord van de Heer is door het vuur gelouterd;~\sep\ Hij is een schild voor allen, die vluchten tot Hem.
\end{halfparskip}

\psalmsubtitle{f) Gij, God, hebt mij vaardig gemaakt}

\begin{halfparskip}
  Wie is God buiten de Heer,~\sep\ of wie een rots buiten onze God?

  God, die mij met kracht heeft omgord,~\sep\ en mij een veilige weg heeft gebaand;

  Die aan mijn voeten de snelheid der hinden gaf,~\sep\ en mij plaatste op de hoogten,

  Die mijn handen oefende tot de strijd,~\sep\ en tot het spannen van de koperen boog mijn armen.
\end{halfparskip}

\psalmsubtitle{g) Van U, Heer, kwam de zege}

\begin{halfparskip}
  Gij schonkt mij Uw schild, dat redding brengt, en Uw rechterhand heeft mij staande gehouden,~\sep\ en Uw zorgzame liefde maakte mij groot.

  Gij hebt de weg voor mijn schreden verbreed,~\sep\ en mijn voeten wankelden niet.

  Ik zette mijn vijanden na, en greep ze aan,~\sep\ en ik keerde niet terug, eer ik ze had vernietigd.

  Ik heb ze verpletterd en opstaan konden ze niet,~\sep\ ze bleven liggen onder mijn voeten.

  Ja, Gij hebt mij met kracht omgord tot de strijd;~\sep\ en die mij weerstaan, hebt Gij voor mij doen bukken.

  Gij hebt mijn vijanden op de vlucht gedreven,~\sep\ en die mij haten, hebt Gij verdelgd.

  Zij schreeuwden het uit - maar niemand schonk redding -~\sep\ tot de Heer, maar Hij verhoorde hen niet.

  En ik heb ze vergruisd als stof voor de wind,~\sep\ vertrapt als slijk in de straten.

  Gij hebt mij ontrukt aan het muitende volk,~\sep\ mij gesteld aan het hoofd van de naties.

  Een volk, dat mij vreemd was, werd mij dienstbaar;~\sep\ nauwelijks hoorde het van mij, of het was mij onderdanig.

  Vreemden brachten mij vleiend hulde,~\sep\ vreemden, geslagen met schrik, kropen sidderend uit hun burchten.
\end{halfparskip}

\psalmsubtitle{h) U, Heer, zij lof!}

\begin{halfparskip}
  Leve de Heer, mijn Rots zij gezegend;~\sep\ hooggeprezen zij God, mijn Redder!

  God, die mij de wraak in handen gaf,~\sep\ en mij de volkeren onderwierp,

  Gij, die mij van mijn vijanden hebt bevrijd, en mij verheven hebt boven mijn weerstrevers,~\sep\ mij hebt ontrukt aan de geweldenaar.

  Daarom zal ik U prijzen onder de volken, O Heer,~\sep\ en verheerlijken Uw Naam.

  Gij hebt Uw koning een schitterende zege verleend,~\sep\ en barmhartigheid bewezen aan Uw gezalfde, aan David en zijn geslacht voor eeuwig.
\end{halfparskip}

\slota{Wij moeten U belijden, aanbidden en verheerlijken, U die voor allen in Uw Wezen verborgen zijt maar Uzelf hebt geopenbaard in de wonderbaarlijke daden van Uw heilsbestel. Hemel en aarde verkondigen Uw kracht in alle seizoenen en tijden, Heer van alles...}

\marmita{7}

\PSALMtitle{19}{Gods wet, de zon van onze ziel}

\psalmsubtitle{a) Heerlijkheid der natuur}

\begin{halfparskip}
  De hemelen verhalen de glorie van God,~\sep\ en het uitspansel roemt het werk van Zijn handen.

  \qanona{Aanbiddenswaardig is de eeuwige God, die de rationele wezens geschapen heeft om Zijn andere werken te begrijpen en Hem te prijzen!}

  De dag galmt het uit aan de dag,~\sep\ en de nacht geeft het door aan de nacht.

  Dat is geen taal, dat zijn geen woorden,~\sep\ waarvan de klank niet wordt vernomen.

  Over heel de wereld golft hun sein,~\sep\ en tot de grenzen der aarde hun uitspraak.

  Daar sloeg Hij Zijn tent op voor de zon, die als een bruidegom uit zijn bruidskamer treedt,~\sep\ en als een juichende reus zijn baan doorloopt.

  Aan het einde van de hemel is zijn opgang, en zijn kringloop reikt tot het einde van de hemel,~\sep\ aan zijn gloed kan niets zich onttrekken.
\end{halfparskip}

\psalmsubtitle{b) Voortreffelijk, Heer, is Uw Wet}

\begin{halfparskip}
  De Wet van de Heer is volmaakt: zij schenkt aan de ziel nieuw leven;~\sep\ het gebod van de Heer staat vast, het onderricht de eenvoudige.

  De voorschriften van de Heer zijn rechtmatig: een vreugde voor het hart;~\sep\ het bevel van de Heer is rein: een licht voor de ogen.

  De vrees van de Heer is zuiver: zij blijft eeuwig bestaan;~\sep\ de oordelen van de Heer zijn waarachtig: alle even rechtvaardig,

  Te verkiezen boven goud en schatten van het edelst metaal,~\sep\ en zoeter dan honing en druipend honingzeem.

  Al wijdt er Uw dienaar zijn aandacht aan,~\sep\ en onderhoudt hij ze vol ijver,

  Wie kent er zijn fouten?~\sep\ Reinig mij van die mij verborgen zijn!

  Behoed ook Uw dienaar voor hoogmoed;~\sep\ dat hij mij niet beheerse!

  Dan zal ik zuiver zijn,~\sep\ en rein van zware misdaad.

  Mogen de woorden van mijn mond en de overweging van mijn hart welgevallig zijn,~\sep\ voor Uw aanschijn, O Heer, mijn Rots en mijn Redder!
\end{halfparskip}

\PSALMtitle{20}{Gebed voor de koning, die optrekt ten strijd}

\psalmsubtitle{a) Heer, help hem in de strijd!}

\begin{halfparskip}
  Dat de Heer u verhore in tijden van nood,~\sep\ u bescherme de Naam van Jacobs God!

  \qanona{Laten we ons vertrouwen in God plaatsen, want Hij redt de nederigen.}

  Hij zende u hulp uit het heiligdom,~\sep\ en uit Sion steune Hij u.

  Hij gedenke al uw offergaven,~\sep\ en uw brandoffer behage Hem.

  Hij schenke u wat uw hart begeert,~\sep\ en doe al uw plannen slagen.

  Mogen we over uw zegepraal juichen, in de Naam van onze God de banieren verheffen;~\sep\ dat de Heer al uw beden vervulle!
\end{halfparskip}

\psalmsubtitle{b) Gij, God, schonkt reeds verhoring}

\begin{halfparskip}
  Reeds weet ik dat de Heer Zijn gezalfde de zegepraal schonk,~\sep\ hem heeft verhoord vanuit Zijn heilige hemel door de kracht van Zijn verwinnende rechterhand.

  Dat anderen vertrouwen op strijdwagens, anderen op rossen,~\sep\ wij echter zijn sterk door de Naam van de Heer, onze God.

  Zij zijn gevallen en liggen terneer,~\sep\ maar wij houden onwankelbaar stand.

  O Heer, schenk de koning de zege,~\sep\ en verhoor ons op de dag van ons smeken.
\end{halfparskip}

\PSALMtitle{21}{Dank voor de overwinning}

\psalmsubtitle{a) Dank, Heer, voor de zege van de koning!}

\begin{halfparskip}
  Over Uw macht, O Heer, verheugt zich de koning,~\sep\ hoe uitbundig jubelt hij over Uw hulp!

  \qanona{De Heer neemt het leed weg van Zijn dienaren, en verblijdt hen door Zijn macht.}

  Zijn hartewens hebt Gij verhoord,~\sep\ en de bede van zijn lippen niet afgewezen.

  Ja, Gij hebt hem voorkomen met rijke zegen,~\sep\ een kroon van zuiver goud hem op het hoofd gedrukt.

  Leven vroeg hij U; Gij hebt hem gegeven,~\sep\ lengte van dagen voor immer.

  Groot is zijn roem, dank aan Uw hulp;~\sep\ met majesteit en luister hebt Gij hem getooid,

  Ja, Gij hebt hem voor eeuwig overladen met zegening,~\sep\ hem met vreugde overstelpt voor Uw aanschijn.

  Want de koning vertrouwt op de Heer,~\sep\ en door de gunst van de Allerhoogste zal hij niet wankelen.
\end{halfparskip}

\psalmsubtitle{b) Sla de vijand neer, O God!}

\begin{halfparskip}
  Moge Uw hand al Uw vijanden treffen,~\sep\ Uw rechter aangrijpen al die U haten.

  Maak ze als tot een gloeiende oven,~\sep\ wanneer Gij Uw aanschijn zult tonen.

  Dat de Heer hen in Zijn toorn vertere,~\sep\ en het vuur hen verslinde.

  Verdelg hun kroost op aarde,~\sep\ en hun zaad onder de kinderen der mensen.

  Als zij U kwaad willen doen, listige plannen beramen,~\sep\ zullen zij niets vermogen;

  Want Gij zult ze doen vluchten,~\sep\ Uw boog op hun aangezicht richten.

  Rijs op, o Heer, in Uw kracht!~\sep\ Wij zullen Uw macht bezingen en prijzen.
\end{halfparskip}

% % % % % % % % % % % % % % % % % % % % % % % % % % % % % % % % % % % % % % % %

\hulala{3}

\Slota{We moeten Uw glorieuze Godheid, vol barmhartigheid, mededogen, hoop, leven en redding voor alle schepselen, belijden, aanbidden en verheerlijken in alle seizoenen en tijden, Heer van alles...}

\marmita{8}

\PSALMtitle{22}{Van God verlaten}

\psalmsubtitle{a) In diepe verlatenheid}

\begin{halfparskip}
  Mijn God, mijn God, waarom hebt Gij mij verlaten?~\sep\ Ver houdt Gij U af van mijn bede, van mijn noodgeschrei.

  \qanona{Mijn God, mijn God, verwerp mij niet samen met de mensen, die U niet kennen.}

  Bij dag roep ik U aan, mijn God, en Gij verhoort mij niet;~\sep\ bij nacht, en Gij slaat geen acht op mij.

  Toch woont Gij in het heiligdom,~\sep\ Gij, de roem van Israël.

  Onze vaderen hoopten op U,~\sep\ zij hoopten op U, en Gij hebt hen bevrijd;

  Zij riepen U aan, en werden gered;~\sep\ zij hoopten op U, en zijn niet beschaamd.
\end{halfparskip}

\psalmsubtitle{b) Veracht en bespot}

\begin{halfparskip}
  Maar ik ben een worm en geen mens,~\sep\ de smaad der mensen en de verachting van het volk.

  Allen, die mij zien, spotten met mij,~\sep\ vertrekken de lippen en schudden het hoofd:

  ``Hij vertrouwt op de Heer; laat Die hem bevrijden,~\sep\ laat Die hem verlossen, zo Hij hem bemint.''

  Ja, Gij hebt mij geleid van de moederschoot af,~\sep\ mij veilig gelegd aan de borst van mijn moeder.

  U werd ik toevertrouwd vanaf mijn geboorte,~\sep\ vanaf de schoot van mijn moeder zijt Gij mijn God.

  Blijf toch niet ver van mij, want ik word gekweld;~\sep\ wees mij nabij, want er is geen helper.
\end{halfparskip}

\psalmsubtitle{c) In uiterste doodsnood}

\begin{halfparskip}
  Jonge stieren stuwen in menigte om mij heen,~\sep\ stieren van Basan omsingelen mij.

  Zij sperren hun muil tegen mij open,~\sep\ als een roofzuchtige en brullende leeuw.

  Als water ben ik uitgestort,~\sep\ en al mijn beenderen zijn ontwricht.

  Mijn hart is geworden als was,~\sep\ het smelt in mijn binnenste weg.

  Mijn keel is droog als een potscherf, en mijn tong kleeft vast aan mijn gehemelte;~\sep\ Gij hebt mij gebracht tot het stof van de dood.

  Want vele honden staan om mij heen,~\sep\ een bende boosdoeners houdt mij omsingeld.

  Zij hebben mijn handen en voeten doorboord,~\sep\ ik kan al mijn beenderen tellen.

  Zij slaan mij gade, en bij die aanblik verheugen zij zich; zij verdelen mijn klederen onder elkander,~\sep\ en werpen het lot over mijn gewaad.
\end{halfparskip}

\psalmsubtitle{d) Red mij van de dood!}

\begin{halfparskip}
  Gij nu, Heer, blijf niet van verre staan;~\sep\ mijn Bijstand, snel mij te hulp!

  Ontruk mijn ziel aan het zwaard,~\sep\ aan de greep van de hond mijn leven.

  Red mij uit de muil van de leeuw,~\sep\ mij, ongelukkige, van de hoornen der buffels.
\end{halfparskip}

\psalmsubtitle{e) Eeuwige dank aan God}

\begin{halfparskip}
  Ik zal mijn broeders Uw Naam verkondigen,~\sep\ in volle vergadering U prijzen:

  ``Looft de Heer, gij, die Hem vreest, heel Jacobs geslacht, verheerlijk Hem:~\sep\ vreest Hem, alle kinderen van Israël!

  Want Hij heeft niet versmaad, noch geminacht de ellende van de verdrukte, en Hij hield zijn aanschijn voor hem niet verborgen;~\sep\ Hij heeft hem aanhoord, toen hij riep tot Hem.''

  Van U komt mijn lof in de volle vergadering;~\sep\ ten aanschouwen van Zijn vereerders zal ik mijn geloften volbrengen.

  De armen zullen eten en zich verzadigen; die de Heer zoeken, zullen Hem loven:~\sep\ ``dat uw harten leven in eeuwigheid!''
\end{halfparskip}

\psalmsubtitle{f) Alle volken zullen Hem eren}

\begin{halfparskip}
  Dit indachtig, zullen tot de Heer zich bekeren,~\sep\ alle grenzen der aarde;

  En voor Zijn aanschijn zullen neervallen,~\sep\ alle stammen der heidenen,

  Want aan de Heer behoort het koningschap,~\sep\ Hij is het, die over de volkeren heerst;

  Hem alleen zullen allen aanbidden die onder de aarde rusten,~\sep\ voor Hem zullen allen zich buigen, die neerdalen in het stof.

  En mijn ziel zal leven voor Hem,~\sep\ mijn nageslacht Hem dienen;

  Het zal van de Heer verhalen aan het geslacht, dat komen zal,~\sep\ en Zijn gerechtigheid zal men vermelden aan het volk, dat wordt geboren: ``Dit heeft de Heer gedaan.''
\end{halfparskip}

\PSALMtitle{23}{God is mijn herder}

\psalmsubtitle{a) Gij zijt een goede Herder, O Heer}

\begin{halfparskip}
  De Heer is mijn Herder: het ontbreekt mij aan niets;~\sep\ in groenende beemden laat Hij mij sluimeren.

  \qanona{Laten we onze zorgen toevertrouwen aan de Heer, de Zorgdrager voor Zijn gezin.}

  Hij voert mij naar wateren, waar ik kan rusten;~\sep\ Hij verkwikt mijn ziel.

  Hij leidt mij langs rechte wegen,~\sep\ omwille van Zijn Naam.

  Al schrijd ik dan voort in een donker dal,~\sep\ geen kwaad zal ik vrezen, omdat Gij met mij zijt.

  Uw roede en Uw herdersstaf,~\sep\ zijn mij tot troost.
\end{halfparskip}

\psalmsubtitle{b) Mijn God, Gij zijt een milde Gastheer}

\begin{halfparskip}
  Gij richt voor mij een maaltijd aan,~\sep\ ten aanschouwen van mijn weerstrevers.

  Met olie zalft Gij mijn hoofd;~\sep\ mijn beker is overvol.

  Goedertierenheid en genade zullen mij volgen,~\sep\ al de dagen van mijn leven;

  En wonen zal ik in het huis van de Heer,~\sep\ tot in de verre toekomst.
\end{halfparskip}

\PSALMtitle{24}{Feestlied}

\psalmsubtitle{a) Uw huis, O Heer, vraagt heiligheid}

\begin{halfparskip}
  De Heer behoort de aarde met al wat zij bevat,~\sep\ de wereld en die er op wonen.

  \qanona{Laat ons zorgvuldig zijn in onze plicht, want de Almachtige heeft ons gered.}

  Want Hij heeft haar op de zeeën gegrondvest,~\sep\ en legde haar vast op de stromen.

  Wie mag de berg van de Heer bestijgen,~\sep\ of wie verwijlen in Zijn heilige plaats?

  Die rein is van handen en zuiver van hart, zijn geest niet richt op ijdele dingen,~\sep\ en tegen zijn naaste geen meineed zweert.

  Die zal zegen ontvangen van de Heer,~\sep\ en loon van God, zijn Redder.

  Dit is het geslacht van die naar Hem zoeken,~\sep\ van die zoeken het aanschijn van Jacobs God.
\end{halfparskip}

\psalmsubtitle{b) De intrede van de Heer in Zijn heiligdom}

\begin{halfparskip}
  Poorten, uw bogen omhoog, omhoog, gij, aloude poorten,~\sep\ opdat de Koning der glorie Zijn intrede doe!

  ``Wie is die Koning der glorie?''~\sep\ ``De Heer, de Sterke en de Machtige, de Heer, de Held in de strijd.''

  Poorten, uw bogen omhoog, omhoog, gij, aloude poorten,~\sep\ opdat de Koning der glorie Zijn intrede doe!

  ``Wie is die Koning der glorie?''~\sep\ De Heer der heerscharen, Hij is de Koning der glorie.''
\end{halfparskip}

\slota{Naar U, onze Heer en onze God, zijn de ogen van onze zielen opgeheven; in U is onze hoop en ons vertrouwen; en van U vragen wij vergeving voor onze overtredingen. Schenk ons dit altijd in Uw liefderijke goedheid en barmhartigheid, zoals U gewend bent, Heer van alles, Vader...}

\marmita{9}

\PSALMtitle{25}{Gebed om bescherming en vergeving}

\psalmsubtitle{a) Vergeef mij, Heer, mijn zonden!}

\begin{halfparskip}
  Tot U verhef ik mijn ziel, O Heer, mijn God.~\sep\ Op U vertrouw ik; laat mij niet te schande worden;~\sep

  \qanona{Mijn Heer, ik hef mijn ogen op tot U, want Gij zijt mijn ware Hoop!}

  dat mijn vijanden niet over mij juichen!

  Want van wie op U hopen, wordt niemand beschaamd,~\sep\ maar wel worden te schande, die hun woord vermetel breken.

  Toon mij Uw wegen, O Heer,~\sep\ en leer mij Uw paden kennen.

  Leid mij in Uw waarheid en geef mij onderricht, omdat Gij, God, mijn Redder zijt,~\sep\ en immer hoop ik op U.

  Gedenk Uw ontferming, o Heer,~\sep\ en Uw barmhartigheid, die van oudsher zijn.

  De zonden van mijn jeugd en mijn misslagen, gedenk ze niet; wees mij naar Uw erbarming indachtig,~\sep\ vanwege Uw goedheid, Heer.
\end{halfparskip}

\psalmsubtitle{b) God wijst de nederige de rechte weg}

\begin{halfparskip}
  Goed en rechtvaardig is de Heer;~\sep\ daarom wijst Hij de zondaars de weg.

  De nederigen leidt Hij in gerechtigheid,~\sep\ de nederigen toont Hij Zijn weg.

  Alle wegen van de Heer zijn goedheid en trouw,~\sep\ voor wie Zijn Verbond en Zijn wetten bewaren.

  Omwille van Uw Naam, Heer,~\sep\ vergeef mij mijn zonde, want zij is groot.

  Wie is de man, die de Heer vreest?~\sep\ Hij wijst hem de weg, die hij moet kiezen.

  Hij zelf zal in voorspoed leven,~\sep\ en zijn geslacht het land bezitten.

  De Heer is een vriend voor hen, die Hem vrezen:~\sep\ Zijn Verbond doet Hij hun kennen.

  Mijn ogen zijn immer gericht op de Heer,~\sep\ want uit de strik zal Hij mijn voeten bevrijden.
\end{halfparskip}

\psalmsubtitle{c) Bevrijd mij, Heer, van kwellingen!}

\begin{halfparskip}
  Zie op mij neer en wees mij genadig,~\sep\ want eenzaam ben ik en ellendig.

  Verlicht de druk van mijn hart,~\sep\ en bevrijd mij van mijn angsten.

  Zie mijn ellende en mijn kwelling;~\sep\ en vergeef mij al mijn zonden.

  Let op mijn vijanden, want ze zijn talrijk,~\sep\ en haten mij met felle haat.

  Bescherm mijn leven en red mij;~\sep\ het zij mij niet tot schande, dat ik bij U mijn toevlucht zocht.

  Dat mijn onschuld en deugd mij beschermen,~\sep\ daar ik hoop op U, O Heer.

  Verlos Israël, o God,~\sep\ uit al zijn kommernissen!
\end{halfparskip}

\PSALMtitle{26}{Vertrouwen van een goed geweten}

\psalmsubtitle{a) Schaf mij recht, O God, om mijn onschuld!}

\begin{halfparskip}
  Heer, schaf mij recht, want ik leefde in onschuld;~\sep\ vertrouwend op de Heer, heb ik niet gewankeld.

  \qanona{O Rechter, hoogste der rechters, druk mijn hoofd niet terneer in Uw oordeel!}

  Onderzoek mij, Heer, en stel mij op de proef;~\sep\ doorgrond mijn nieren en mijn hart.

  Want Uw welwillendheid staat mij voor ogen,~\sep\ en ik wandel naar Uw waarheid.

  Met ongerechtigen zit ik niet neer,~\sep\ en met bedriegers kom ik niet samen.

  Ik haat het gezelschap van bozen,~\sep\ en met goddelozen zit ik niet samen.

  In onschuld was ik mijn handen,~\sep\ en ga rond Uw altaar, O Heer.

  Om openlijk Uw lof te verkondigen,~\sep\ en al Uw wonderen te verhalen.

  Heer, ik heb lief het verblijf van Uw huis,~\sep\ en de woontent van Uw heerlijkheid.
\end{halfparskip}

\psalmsubtitle{b) Verderf mij niet, Heer, met de bozen!}

\begin{halfparskip}
  Ruk mijn ziel niet weg met de zondaars,~\sep\ noch mijn leven met bloeddorstige mannen,

  Aan wier handen de misdaad kleeft,~\sep\ en wier rechterhand met geschenken gevuld is.

  Ik echter wandel in onschuld:~\sep\ red mij, en wees mij genadig.

  Mijn voet staat op effen baan;~\sep\ ik zal de Heer in de vergadering loven.
\end{halfparskip}

\PSALMtitle{27}{Vertrouwen op God}

\psalmsubtitle{a) Met God vrees ik geen mens}

\begin{halfparskip}
  De Heer is mijn Licht en mijn Heil: wie zou Ik vrezen?~\sep\ De Heer is de Schuts van mijn leven: voor wie zou ik sidderen?

  \qanona{Verjaag mij niet van voor Uw aangezicht, Gij die de geheime dingen doorgrondt!}

  Als de bozen mij bestormen om mijn vlees te verslinden,~\sep\ mijn vijanden en haters, zij struikelen en vallen.

  Al stond er een krijgsmacht tegenover mij, mijn hart zou niet vrezen;~\sep\ al brak er een oorlog tegen mij uit, dan nog zou ik vertrouwen.
\end{halfparskip}

\psalmsubtitle{b) Maar één hartewens: het huis van de Heer}

\begin{halfparskip}
  Dit alleen vraag ik de Heer, dit alleen streef ik na:~\sep\ te wonen in het huis van de Heer alle dagen van mijn leven,

  Te genieten de zoetheid van de Heer,~\sep\ en Zijn tempel te aanschouwen.

  Want in Zijn woontent zal Hij mij bergen in tijden van nood,~\sep\ Hij zal mij doen schuilen diep in Zijn tent, mij plaatsen boven op de rots.

  Nu verheft zich mijn hoofd,~\sep\ boven de vijanden, die mij omringen;

  Jubeloffers zal ik brengen in Zijn tent,~\sep\ zingen voor de Heer en spelen op de citer.
\end{halfparskip}

\psalmsubtitle{c) Uw aanschijn zoek ik, O Heer}

\begin{halfparskip}
  Heer, luister naar mijn stem, waarmee ik luid roep;~\sep\ ontferm U over mij, en schenk mij verhoring!

  Tot U spreekt mijn hart, U zoeken mijn ogen;~\sep\ ik zoek Uw aanschijn, o Heer.

  Verberg mij Uw aanschijn niet,~\sep\ stoot Uw dienaar niet af in Uw toorn!

  Gij zijt mijn hulp; verwerp mij niet!~\sep\ Verlaat mij niet, O God, mijn Redder!

  Zou mijn vader en moeder mij ook verlaten,~\sep\ dan nog neemt de Heer mij op.

  Wijs mij Uw weg, o Heer,~\sep\ en leid mij op effen baan omwille van mijn weerstrevers.

  Geef mij niet prijs aan de moedwil van mijn vijanden,~\sep\ want valse getuigen en geweldenaars stonden tegen mij op.

  Ik ben er zeker van de weldaden van de Heer te zien,~\sep\ in het land der levenden.

  Zie uit naar de Heer, wees onversaagd;~\sep\ sterk zij Uw hart, zie uit naar de Heer.
\end{halfparskip}

\slota{Tot U, Heer, roepen wij; bij U zoeken wij onze toevlucht, en aan U vragen wij vergeving van onze overtredingen en zonden; schenk ons dit in Uw genade en barmhartigheid zoals U gewend bent, te allen tijde, Heer van alles...}

\marmita{10}

\PSALMtitle{28}{Smeek- en dankgebed}

\psalmsubtitle{a) Heer, verwerp mij niet met de bozen!}

\begin{halfparskip}
  Tot U roep ik, O Heer;~\sep\ mijn Rots, wees niet doof voor mij.

  Opdat ik niet, als Gij niet hoort naar mij,~\sep\ gelijk worde aan hen, die in de grafkuil dalen.

  \qanona{Onze zielen roepen tot U: kom ons te hulp en red ons!}

  Hoor de stem van mijn smeken, nu ik roep tot U,~\sep\ nu ik mijn handen ophef naar Uw heilige tempel.

  Ruk mij niet weg met de zondaars,~\sep\ met hen, die kwaad bedrijven,

  Die vriendelijk spreken met hun naaste,~\sep\ maar in hun hart kwade bedoelingen koesteren.

  Handel met hen naar hun daden,~\sep\ en naar de boosheid van hun werken.

  Zet hun het werk van hun handen betaald,~\sep\ vergeld ze hun daden.

  Want ze slaan geen acht op de daden van de Heer en het werk van Zijn handen;~\sep\ Hij richte hen te gronde en heffe hen niet op.
\end{halfparskip}

\psalmsubtitle{b) Ja, Hij heeft mij verhoord}

\begin{halfparskip}
  Gezegend de Heer, want Hij hoorde mijn dringende bede;~\sep\ de Heer, mijn kracht en mijn schild,

  Op Hem vertrouwde mijn hart, en ik ben geholpen;~\sep\ daarom jubelt mijn hart en prijs ik Hem met mijn zang.

  De Heer is een kracht voor Zijn volk,~\sep\ en voor Zijn Gezalfde een heilzame schutse.

  Red Uw volk en zegen Uw erfdeel;~\sep\ weid hen en draag hen voor eeuwig.
\end{halfparskip}

\PSALMtitle{29}{Gods majesteit in het onweer}

\begin{halfparskip}
  Kent toe aan de Heer, zonen van God,~\sep\ kent toe aan de Heer glorie en macht!

  \qanona{Gij, Goede, Barmhartige, lof komt U toe.}

  Kent toe aan de Heer de roem van Zijn Naam,~\sep\ aanbidt de Heer in heilige feesttooi.
\end{halfparskip}

\psalmsubtitle{a) Gods Majesteit in het onweer}

\begin{halfparskip}
  De stem van de Heer over de wateren! De God van Majesteit doet de donder rollen:~\sep\ de Heer over de wijde wateren!

  De stem van de Heer vol kracht,~\sep\ de stem van de Heer vol majesteit!

  De stem van de Heer verbrijzelt de ceders,~\sep\ de Heer verbrijzelt de ceders van de Libanon.

  Hij doet de Libanon opspringen als een kalf,~\sep\ en de Sarion als het jong van een buffel.

  De stem van de Heer schiet vlammende schichten, de stem van de Heer doet de wildernis beven,~\sep\ de Heer doet Cades' wildernis beven.

  De stem van de Heer buigt eiken krom en ontschorst de bomen der wouden:~\sep\ en in Zijn tempel roepen allen: Glorie!

  De Heer troonde boven de watervloed,~\sep\ en de Heer zal tronen als Koning voor eeuwig.

  De Heer zal sterkte schenken aan Zijn volk,~\sep\ de Heer zal Zijn volk met vrede zegenen.
\end{halfparskip}

\PSALMtitle{30}{Dank na herstel van ziekte}

\psalmsubtitle{a) Dank, Heer, voor mijn genezing!}

\begin{halfparskip}
  Ik wil U roemen, o Heer, daar Gij mij gered hebt,~\sep\ en niet mijn vijanden over mij liet juichen.

  \qanona{Wij zullen Uw Naam prijzen, want U hebt ons gered; en door Uw kracht hebt U onze vijanden verpletterd.}

  Heer, mijn God,~\sep\ ik riep tot U, en Gij hebt mij genezen.

  Heer, uit het dodenrijk hebt Gij mij weggevoerd,~\sep\ mij gered uit hen, die ten grave dalen.

  Speelt op de citer voor de Heer, gij, Zijn heiligen;~\sep\ en dankt Zijn heilige Naam.

  Want Zijn toorn duurt slechts een ogenblik,~\sep\ maar Zijn welwillendheid het hele leven door.

  's Avonds komt er geween te gast,~\sep\ maar 's morgens is er gejubel.
\end{halfparskip}

\psalmsubtitle{b) Gij, Heer, waart mij genadig}

\begin{halfparskip}
  In overmoed nu heb ik gezegd:~\sep\ ``In eeuwigheid zal ik niet wankelen.''

  Het was Uw gunst, o Heer, die mij ere schonk en macht;~\sep\ maar toen Gij Uw aanschijn verborgen hieldt, werd ik ontsteld.

  Ik roep tot U, O Heer,~\sep\ en smeek bij mijn God om erbarming:

  ``Wat kan mijn bloed U baten,~\sep\ of mijn neerdalen in het graf?

  Zal het stof U soms prijzen,~\sep\ of roemen Uw trouw?''

  Luister, O Heer, en wees mij genadig;~\sep\ O Heer, wees toch mijn Helper !

  Gij hebt mijn rouw in een reidans veranderd,~\sep\ mijn rouwkleed verscheurd, mij met vreugde omgord,

  Opdat mijn ziel U zou prijzen en nimmermeer zwijgen.~\sep\ Heer, mijn God, ik zal U loven voor eeuwig!
\end{halfparskip}

% % % % % % % % % % % % % % % % % % % % % % % % % % % % % % % % % % % % % % % %

\PSALMtitle{}{}

\psalmsubtitle{}

\begin{halfparskip}
  \qanona{}
\end{halfparskip}

\psalmsubtitle{}

\begin{halfparskip}
\end{halfparskip}

% % % % % % % % % % % % % % % % % % % % % % % % % % % % % % % % % % % % % % % %

\end{document}