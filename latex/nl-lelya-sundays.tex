\documentclass[12pt,twoside,a5paper]{article}

\usepackage{multicol}

\usepackage[main=dutch]{babel}
\usepackage{divine-office}

% % % % % % % % % % % % % % % % % % % % % % % % % % % % % % % % % % % % % % % %

% Version: 2024-12-08
\begin{document}

\title{Lelya~--- zondagen}
\author{}
\date{}
\maketitle

% The following prevents footnotes and paracol from interacting in bad ways.
% Not really an idea why...
% See: https://stackoverflow.com/questions/61779911/paracol-and-footnote-placing-in-latex
\footnotelayout{\ }

% % % % % % % % % % % % % % % % % % % % % % % % % % % % % % % % % % % % % % % %

\begin{halfparskip}
  \cc~Eer aan God in den hoge \liturgicalhint{(3x)}. En op aarde vrede en goede hoop aan de mensen, altijd en in eeuwigheid.

  [Amen.]~--- \rr~Zegen, Heer.~--- \liturgicalhint{[Vredekus.]}

  \cc~Onze Vader die in de hemelen zijt,
  \rr~Geheiligd zij Uw Naam. Uw rijk kome, heilig, heilig, heilig zijt Gij. Onze Vader die in de hemelen zijt, de hemel en de aarde zijn gevuld met Uw onmetelijke glorie; de engelen en de mensen roepen U toe: heilig, heilig, heilig zijt Gij.~--- Onze Vader die in de hemelen zijt, geheiligd zij Uw Naam. Uw rijk kome, Uw wil geschiede op aarde zoals in de hemel. Geef ons heden het brood dat we nodig hebben en vergeef ons onze schulden en zonden zoals wij ook vergeven hebben aan onze schuldenaren. En leid ons niet in bekoring, maar verlos ons van de Kwade. Want van U is het koninkrijk en de kracht en de heerlijkheid in eeuwigheid, amen.

  \cc~Eer aan de Vader, de Zoon, en de Heilige Geest.

  \rr~Vanaf het begin en in alle eeuwigheid, amen en amen. Onze Vader die in de hemelen zijt, geheiligd zij Uw naam, Uw rijk kome, heilig, heilig, heilig zijt Gij. Onze Vader die in de hemelen zijt, de hemel en de aarde zijn gevuld met Uw onmetelijke glorie; de engelen en de mensen roepen U toe: heilig, heilig, heilig zijt Gij.

  \dd~Laat ons opstaan om te bidden, vrede zij met ons.

  \cc~Laat ons opstaan, o Heer, in Uw kracht, en bevestigd worden in Uw hoop, mogen we opgetild en gesterkt worden door de hoge arm van Uw macht; en mogen we waardig zijn, met de hulp van Uw goedertierenheid, om te allen tijde lof, eer, belijdenis en aanbidding tot U te verheffen, Heer van alles, Vader...
\end{halfparskip}

% % % % % % % % % % % % % % % % % % % % % % % % % % % % % % % % % % % % % % % %

\markedsection{PSALMEN}

\liturgicalhint{Op Zondagen worden 3 hulale gebeden met alleluia (respectievelijk hulale 5-7; of 9-11 + Ps 81; of 12-14 of 16-18 + Ps 129)\footnote{In dit boek vervangen we elke hulala door 1 marmita, de marmiyata gebruikt in ramsa uitsluitend. We volgen het oude systeem waarin de laatste hulala (hier marmita) beëindigd wordt met de psalm die de qalta begeleidt: \Pss{81 \& 129}.}, gevolgd op Zondagen ``voor'' met Ps. 81, op Zondagen ``na'' met Ps. 129, met alleluia na elk vers.}

\markedsubsectionrubricwithhint{1. Zondagen ``voor'':}{Marmita 29 (\Pss{75--77})}

\PSALMtitle{75}{Gods oordeel komt}

\begin{halfparskip}
  Wij loven U, Heer, wij loven U,~\sep\ en prijzen Uw Naam,

  \liturgicalhint{Alleluia, alleluia, alleluia.}~--- \liturgicalhint{Eerste vers:} Wij loven U, Heer, wij loven U,~\sep\ en prijzen Uw Naam,

  verhalen Uw wonderen.~\sep

  ``Op de tijd, die Ik zal bepalen,~\sep\ zal Ik oordelen volgens recht.

  Al wankelt de aarde met al haar bewoners,~\sep\ Ik heb haar zuilen bevestigd.

  De trotsen roep Ik toe: ``Legt Uw hoogmoed af'';~\sep\ en de goddeloze: ``Steekt Uw hoorn niet op'';

  Steekt Uw hoorn niet op tegen de Allerhoogste,~\sep\ uit tegen God geen schaamteloze taal.

  Want noch van het oosten, noch van het westen;~\sep\ noch uit de woestijn, noch van de bergen:

  Maar God is de Rechter:~\sep\ de een drukt Hij neer, de ander verheft Hij.

  Want in de hand van de Heer is een beker,~\sep\ vol kruiden, die schuimt van wijn,

  Hij geeft er uit te drinken; tot de droesem zal men hem ledigen,~\sep\ alle bozen der aarde zullen ervan drinken.''

  Maar ik zal in eeuwigheid juichen,~\sep\ voor de God van Jacob de citer bespelen.

  En alle hoornen der bozen zal ik verbreken,~\sep\ maar de hoornen der gerechtigen worden verheven.
\end{halfparskip}

\PSALMtitle{76}{Overwinningslied}

\begin{halfparskip}
  \psalmsubtitle{a) Gij hebt de vijand verdelgd, O Heer!}

  God is in Juda bekend,~\sep\ groot is Zijn Naam in Israël.

  Zijn tent staat in Salem,~\sep\ en Zijn woning in Sion.

  Daar brak Hij stuk de schichten van de boog,~\sep\ schild en zwaard en wapentuig.

  Gij, Machtige, schitterend van licht, zijt gekomen,~\sep\ van de eeuwige bergen.

  Ontwapend zijn de stoutmoedigen, zij slapen hun doodsslaap;~\sep\ en de handen van al die helden vielen slap.

  Door Uw dreigen, God van Jacob,~\sep\ werden wagens en paarden verlamd.
\end{halfparskip}

\begin{halfparskip}
  \psalmsubtitle{b) Gij zijt ontzagwekkend, O Heer}

  Schrikwekkend zijt Gij, en wie zal U weerstaan,~\sep\ bij het geweld van Uw toorn?

  Vanuit de hemel hebt Gij Uw vonnis doen horen:~\sep\ de aarde ontstelde en zweeg,

  Toen God oprees ten oordeel,~\sep\ om alle verdrukten van het land te redden.

  Want de woede van Edom zal U tot glorie strekken,~\sep\ en die in Emath overbleven, zullen feesten om U.

  Doet geloften aan de Heer Uw God, en komt ze na,~\sep\ dat allen rondom Hem heen aan de Ontzagwekkende een offer brengen.

  Aan Hem, die de trots der vorsten fnuikt,~\sep\ die schrikwekkend is voor de koningen der aarde.
\end{halfparskip}

\PSALMtitle{77}{Gebed in nood}

\begin{halfparskip}
  \psalmsubtitle{a) Groot is mijn droefheid}
  Luid verheft zich mijn stem tot God, mijn stem tot God, opdat Hij mij hore;~\sep

  op de dag van mijn kwelling zoek ik de Heer.

  Onvermoeid strekken bij nacht mijn handen zich uit;~\sep\ mijn ziel is ontroostbaar.

  Denk ik aan God, dan moet ik zuchten,~\sep\ peins ik na, dan verlies ik de moed.

  Gij houdt mijn ogen geopend;~\sep\ ik ben ontsteld en kan niet meer spreken.

  Ik overpeins de vroegere dagen,~\sep\ aan vervlogen jaren denk ik terug.

  Ik overweeg 's nachts in mijn hart,~\sep\ ik peins na, en mijn geest tracht uit te vorsen:
\end{halfparskip}

\begin{halfparskip}
  \psalmsubtitle{b) Heeft God Zijn volk verlaten?}

  ``Zou God dan voor eeuwig verwerpen,~\sep\ en nooit meer genadig zijn?

  Zou Zijn liefde voorgoed zijn verdwenen,~\sep\ Zijn belofte verijdeld voor alle geslachten?

  Heeft God soms vergeten Zich te erbarmen,~\sep\ of in Zijn toorn Zijn ontferming bedwongen?''

  Dan zeg ik: ``Dit is mijn smart,~\sep\ dat de rechterhand van de Allerhoogste is veranderd''.

  Ik denk terug aan de werken van de Heer,~\sep\ ja, ik denk terug aan Uw aloude wonderen.

  Ik overweeg al Uw werken,~\sep\ en overpeins Uw daden.
\end{halfparskip}

\begin{halfparskip}
  \psalmsubtitle{c) Gij hebt ons steeds gered, O God}

  O God, Uw weg is heilig:~\sep\ welke god is groot als onze God?

  Gij zijt de God, die wonderen doet,~\sep\ hebt Uw macht aan de volken doen kennen.

  Door Uw arm hebt Gij Uw volk verlost:~\sep\ de zonen van Jacob en Jozef.

  De wateren zagen U, O God, de wateren zagen U: zij beefden,~\sep\ en de golven werden onstuimig.

  Het zwerk stortte zijn stromen uit, de wolken verhieven hun stem,~\sep\ en Uw flitsen doorkliefden de lucht.

  Uw donder ratelde in de wervelwind, Uw bliksems verlichtten het aardrijk:~\sep\ de aarde sidderde en beefde.

  Uw weg werd gebaand door de zee en Uw pad door de machtige wateren,~\sep\ maar Uw sporen bleven onzichtbaar.

  Als een kudde hebt Gij Uw volk geleid,~\sep\ door de hand van Moses en Aäron.
\end{halfparskip}

\markedsubsectionrubricwithhint{1. Zondagen ``na'':}{Marmita 49 (\Ps{118b})}

\begin{halfparskip}
  \acrosticletter{Lamed} Eeuwig, O Heer, blijft Uw woord,~\sep\ het staat vast als de hemel.

  \liturgicalhint{Alleluia, alleluia, alleluia.}~--- \liturgicalhint{Eerste vers.}

  Van geslacht tot geslacht blijft Uw trouw;~\sep\ Gij hebt de aarde gegrondvest en zij houdt stand.

  Volgens Uw besluiten blijven zij immer bestaan,~\sep\ omdat alles U dienstbaar is.

  Als niet Uw Wet mijn vreugde was,~\sep\ reeds was ik in mijn ellende vergaan.

  Nimmer zal ik Uw bevelen vergeten,~\sep\ want daardoor deedt Gij mij leven.

  Ik ben de uwe: wees mij tot redding,~\sep\ omdat ik uitzag naar Uw bevelen.

  Zondaars wachten mij op om mij te verderven;~\sep\ ik geef op Uw voorschriften acht.

  Begrensd zag ik alle volmaaktheid,~\sep\ maar onbeperkt strekt Uw gebod zich uit.
\end{halfparskip}

\begin{halfparskip}
  \acrosticletter{Men} Hoe lief heb ik Uw Wet, O Heer;~\sep\ de gehele dag overweeg ik haar.

  Uw gebod maakte mij wijzer dan mijn vijanden,~\sep\ want het staat mij eeuwig ter zijde.

  Verstandiger ben ik dan al mijn leraars,~\sep\ omdat ik Uw voorschriften overweeg.

  Ik ben scherper van inzicht dan grijsaards,~\sep\ omdat ik Uw bevelen onderhoud.

  Van alle verkeerde wegen houd ik mijn schreden af,~\sep\ om Uw woorden na te leven.

  Ik wijk niet af van Uw besluiten,~\sep\ want Gij hebt mij onderwezen.

  Hoe zoet voor mijn gehemelte zijn Uw uitspraken,~\sep\ zoeter dan honing voor mijn mond!

  Door Uw bevelen krijg ik inzicht,~\sep\ daarom haat ik iedere weg van ongerechtigheid.
\end{halfparskip}

\begin{halfparskip}
  \acrosticletter{Nun} Uw woord is een lamp voor mijn voeten,~\sep\ en een licht op mijn pad.

  Ik zweer en neem mij voor,~\sep\ Uw rechtvaardige besluiten na te leven.

  Ik ben in de diepste ellende, O Heer;~\sep\ spaar mijn leven naar Uw woord.

  Aanvaard, O Heer, de offers van mijn mond,~\sep\ en leer mij Uw besluiten.

  In voortdurend gevaar is mijn leven,~\sep\ maar Uw Wet vergeet ik niet.

  De bozen hebben mij een strik gelegd,~\sep\ maar van Uw bevelen week ik niet af.

  Uw voorschriften zijn mijn erfdeel voor eeuwig,~\sep\ want ze zijn de vreugde van mijn hart.

  Ik heb er mijn hart op gezet Uw verordeningen na te komen,~\sep\ voortdurend en stipt.
\end{halfparskip}

\begin{halfparskip}
  \acrosticletter{Samech} Ik haat de wankelmoedigen,~\sep\ maar Uw Wet heb ik lief.

  Gij zijt mijn Beschermer en mijn schild,~\sep\ ik vertrouw op Uw woord.

  Gij, bozen, gaat van mij heen:~\sep\ en ik zal de geboden van mijn God onderhouden.

  Sterk mij naar Uw belofte, opdat ik leef;~\sep\ stel mijn hoop niet teleur.

  Sta mij bij, en ik zal behouden zijn,~\sep\ en op Uw verordeningen zal ik acht slaan altijd.

  Die Uw verordeningen verlaten, verwerpt Gij,~\sep\ want hun gedachten zijn bedrieglijk.

  Als afval beschouwt Gij alle bozen op aarde,~\sep\ daarom heb ik Uw voorschriften lief.

  Van vreze beeft mijn vlees voor U,~\sep\ en ik heb ontzag voor Uw besluiten.
\end{halfparskip}

\begin{halfparskip}
  \acrosticletter{Ain} Recht en gerechtigheid heb ik beoefend;~\sep\ lever mij niet over aan mijn verdrukkers.

  Sta borg voor het welzijn van Uw dienaar,~\sep\ opdat de trotsen mij niet verdrukken.

  Mijn ogen kwijnen van verlangen naar Uw hulp,~\sep\ en naar Uw rechtvaardige uitspraak.

  Handel met Uw dienaar naar Uw goedheid,~\sep\ en leer mij Uw verordeningen.

  Ik ben Uw dienstknecht, onderricht mij,~\sep\ opdat ik Uw voorschriften kenne.

  Voor de Heer is het tijd om te handelen:~\sep\ zij hebben Uw Wet verkracht.

  Daarom heb ik Uw geboden lief,~\sep\ meer dan goud en het edelst metaal.

  Daarom koos ik al Uw bevelen tot mijn deel;~\sep\ van iedere dwaalweg heb ik een afschuw.
\end{halfparskip}

\begin{halfparskip}
  \acrosticletter{Phe} Wonderbaar zijn Uw voorschriften,~\sep\ daarom onderhoudt ze mijn ziel.

  De openbaring van Uw woorden geeft licht,~\sep\ de onbedrevenen onderricht zij.

  Ik open smachtend mijn mond,~\sep\ want naar Uw geboden verlang ik.

  Wend U tot mij en wees mij genadig,~\sep\ zoals Gij gewoon zijt voor wie Uw Naam beminnen.

  Richt mijn schreden naar Uw uitspraak,~\sep\ en laat mij geen onrecht beheersen.

  Verlos mij van de verdrukking der mensen;~\sep\ en Uw bevelen zal ik onderhouden.

  Toon Uw dienaar Uw vredig gelaat,~\sep\ en leer mij Uw verordeningen.

  Stromen van tranen ontwelden mijn ogen,~\sep\ omdat men Uw Wet niet onderhield.
\end{halfparskip}

\begin{halfparskip}
  \acrosticletter{Sade} Rechtvaardig zijt Gij, O Heer,~\sep\ en Uw oordeel is billijk.

  Uw voorschriften gaaft Gij in gerechtigheid,~\sep\ en met grote kracht.

  Mijn ijver verteert mij,~\sep\ omdat mijn weerstrevers Uw woorden vergeten.

  Terdege beproefd is Uw uitspraak,~\sep\ en Uw dienaar heeft ze lief.

  Al ben ik dan klein en veracht,~\sep\ Uw bevelen vergeet ik niet.

  Uw gerechtigheid is gerechtigheid voor eeuwig,~\sep\ en onveranderlijk is Uw Wet.

  Al troffen mij kommer en kwelling,~\sep\ Uw geboden zijn mijn geneugte.

  De rechtvaardigheid van Uw voorschriften is eeuwig,~\sep\ onderricht mij, opdat ik mag leven.
\end{halfparskip}

\begin{halfparskip}
  \acrosticletter{Coph} Ik roep uit heel mijn hart: verhoor mij, Heer;~\sep\ Uw verordeningen leef ik na.

  Ik roep tot U; behoud mij,~\sep\ en ik zal Uw voorschriften onderhouden.

  Ik kom bij de dageraad en roep om Uw hulp,~\sep\ ik vertrouw op Uw woorden.

  Vóór de nachtwaken zijn mijn ogen geopend,~\sep\ om Uw uitspraak te overwegen.

  Hoor mijn smeken, Heer, naar Uw barmhartigheid,~\sep\ en schenk mij leven naar Uw besluit.

  Die mij boosaardig vervolgen, naderen mij,~\sep\ ver zijn zij verwijderd van Uw Wet.

  Gij zijt nabij, O Heer,~\sep\ en waarachtig zijn al Uw geboden.

  Reeds vroeger heb ik uit Uw voorschriften begrepen,~\sep\ dat Gij ze gegeven hebt voor eeuwig.
\end{halfparskip}

\begin{halfparskip}
  \acrosticletter{Res} Zie mijn ellende en bevrijd mij,~\sep\ want Uw Wet heb ik niet vergeten.

  Verdedig mijn zaak, en verlos mij;~\sep\ naar Uw uitspraak schenk mij het leven.

  Ver blijft het heil van de zondaars,~\sep\ want zij storen zich niet aan Uw verordeningen.

  Groot is Uw erbarming, O Heer;~\sep\ schenk mij het leven naar Uw besluiten.

  Velen vervolgen en kwellen mij:~\sep\ van Uw voorschriften wijk ik niet af.

  Ik zag overtreders en het walgde mij,~\sep\ want Uw uitspraak volgden zij niet.

  Zie, Heer, ik heb Uw bevelen lief,~\sep\ spaar mijn leven naar Uw barmhartigheid.

  Geheel Uw woord ligt vervat in standvastigheid;~\sep\ en ieder besluit van Uw gerechtigheid is eeuwig.
\end{halfparskip}

\begin{halfparskip}
  \acrosticletter{Sin} Vorsten vervolgen mij zonder reden,~\sep\ maar mijn hart eerbiedigt Uw woorden.

  Ik verheug mij over Uw uitspraken,~\sep\ als iemand, die rijke buit heeft gemaakt.

  Ongerechtigheid haat en verfoei ik,~\sep\ Uw Wet heb ik lief.

  Zevenmaal daags breng ik U lof,~\sep\ om Uw rechtvaardige oordelen.

  Veel vrede is weggelegd voor die Uw Wet beminnen:~\sep\ geen struikelblok ligt ooit op hun weg.

  Van U, O Heer, verwacht ik hulp,~\sep\ en ik onderhoud Uw geboden.

  Ik leef Uw voorschriften na,~\sep\ en heb ze van harte lief.

  Ik onderhoud Uw bevelen en geboden,~\sep\ want heel mijn weg ligt open voor U.
\end{halfparskip}

\begin{halfparskip}
  \acrosticletter{Tau} Mijn geroep kome tot U, O Heer,~\sep\ geef mij inzicht naar Uw woord.

  Mijn bede dringe door tot U;~\sep\ red mij naar Uw uitspraak.

  Van mijn lippen moge een lofzang vloeien,~\sep\ als Gij mij Uw verordeningen  zult hebben geleerd.

  Mijn tong bezinge Uw uitspraak,~\sep\ want rechtvaardig zijn al Uw geboden.

  Uw hand zij gereed mij te helpen,~\sep\ want Uw bevelen heb ik verkoren.

  Van U verwacht ik redding, O Heer,~\sep\ en Uw Wet is mijn geneugte.

  Leve mijn ziel om U te prijzen,~\sep\ en dat Uw besluiten mij helpen!

  Als een verloren schaap dool ik rond; zoek toch Uw dienaar op,~\sep\ want Uw geboden heb ik niet vergeten.
\end{halfparskip}

\begin{halfparskip}
  \dd~Eer aan...~--- Alleluia, alleluia; Eer aan U, God, alleluia; Eer aan U, God, alleluia; Heer, ontferm U over ons. Laat ons bidden; vrede zij met ons.

  \cc~Versterk, onze Heer en onze God, in Uw mededogen onze zwakheid; bemoedig (\translationoptionNl{troost}) en help in Uw genade de armzaligheid van onze ziel; wek de slaperigheid van onze geest; verlicht (\translationoptionNl{neem weg}) de last van onze ledematen; was en reinig het vuil van onze schulden en zonden; verlicht de duisternis van ons intellect; strek een helpende hand uit en geef ons kracht, zodat we daardoor mogen opstaan om U onophoudelijk te belijden en te verheerlijken, alle dagen van ons leven, Heer van alles...
\end{halfparskip}

\PSALMtitle{}{}

\begin{halfparskip}
  \psalmsubtitle{}
\end{halfparskip}

% % % % % % % % % % % % % % % % % % % % % % % % % % % % % % % % % % % % % % % %

\end{document}