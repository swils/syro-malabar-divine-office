\documentclass[12pt,twoside,a5paper]{article}

\usepackage{multicol}

\usepackage[main=dutch]{babel}
\usepackage{divine-office}

% % % % % % % % % % % % % % % % % % % % % % % % % % % % % % % % % % % % % % % %

% Version: 2024-12-19
\begin{document}

\title{Lelya~--- feesten en gedachtenissen}
\author{}
\date{}
\maketitle

% The following prevents footnotes and paracol from interacting in bad ways.
% Not really an idea why...
% See: https://stackoverflow.com/questions/61779911/paracol-and-footnote-placing-in-latex
\footnotelayout{\ }

% % % % % % % % % % % % % % % % % % % % % % % % % % % % % % % % % % % % % % % %

\begin{halfparskip}
  \cc~Eer aan God in den hoge \liturgicalhint{(3x)}. En op aarde vrede en goede hoop aan de mensen, altijd en in eeuwigheid.

  [Amen].~--- \rr~Zegen, Heer.~--- \liturgicalhint{[vredekus].}

  \cc~Onze Vader die in de hemelen zijt,

  \rr~Geheiligd zij Uw Naam. Uw rijk kome, heilig, heilig, heilig zijt Gij. Onze Vader die in de hemelen zijt, de hemel en de aarde zijn gevuld met Uw onmetelijke glorie; de engelen en de mensen roepen U toe: heilig, heilig, heilig zijt Gij.~--- Onze Vader die in de hemelen zijt, geheiligd zij Uw Naam. Uw rijk kome, Uw wil geschiede op aarde zoals in de hemel. Geef ons heden het brood dat we nodig hebben en vergeef ons onze schulden en zonden zoals wij ook vergeven hebben aan onze schuldenaren. En leid ons niet in bekoring, maar verlos ons van de Kwade. Want van U is het koninkrijk en de kracht en de heerlijkheid in eeuwigheid, amen.

  \cc~Eer aan de Vader, de Zoon, en de Heilige Geest.

  \rr~Vanaf het begin en in alle eeuwigheid, amen en amen. Onze Vader die in de hemelen zijt, geheiligd zij Uw naam, Uw rijk kome, heilig, heilig, heilig zijt Gij. Onze Vader die in de hemelen zijt, de hemel en de aarde zijn gevuld met Uw onmetelijke glorie; de engelen en de mensen roepen U toe: heilig, heilig, heilig zijt Gij.

  \dd~Laat ons opstaan om te bidden, vrede zij met ons.

  \liturgicalhint{Feesten:} Laat ons opstaan, o Heer, in de verborgen kracht van Uw Godheid, mogen wij bevestigd (of: versterkt) worden in de wonderbaarlijke hoop van Uw Majesteit, mogen wij verheven en versterkt worden door de hoge arm van Uw macht; en mogen wij door de hulp van Uw genade waardig zijn om altijd U te loven, te eren, te belijden en te aanbidden, Heer van alles, Vader...

  \liturgicalhint{Gedachtenissen:} Laat ons opstaan, o Heer, in Uw kracht, en bevestigd worden in Uw hoop, verheven en versterkt worden door de hoge arm van Uw macht; en mogen wij waardig zijn, door de hulp van Uw genade, om altijd U te loven, te eren, te belijden en te aanbidden, Heer van alles, Vader...

  \liturgicalhint{Feesten en gedachtenissen: hulale 12--14 geïnterpoleerd met hun qanone en gebeden uit het psalter. Eer aan... na elke marmita. Op Kerstmis en Epifanie: hulale 12--15.}
\end{halfparskip}

% % % % % % % % % % % % % % % % % % % % % % % % % % % % % % % % % % % % % % % %

\hulala{12}

\Slota{Wij moeten U belijden, aanbidden en verheerlijken, O Koning die de koningen aanbidden. Koren van tienduizenden engelen en aartsengelen, staan met grote angst en vrees, en dienen en vieren de aanbiddelijke eer van Uw Majesteit, in alle seizoenen en tijden, Heer van alles...}

\marmita{32}

\PSALMtitle{82}{Tegen onrechtvaardige rechters}

\begin{halfparskip}
  God rijst op in de goddelijke raad, Hij houdt gericht te midden der goden.

  \qanona{Rechters, oordeelt rechtvaardig en blijft ver van (\translationoptionNl{vermijdt}) ongerechtigheid.}

  \liturgicalhint{Feesten \& gedachtenissen.} \emph{De waakzame Intelligentie riep mij als uit de slaap en wekte mij: Sta op, jij die in slaap bent verzonken en werp het gewicht van je luiheid af.}

  ``Hoe lang nog zult gij onrechtvaardig oordelen,~\sep\ en de zaak der bozen begunstigen?

  Verdedigt verdrukten en wezen,~\sep\ geeft aan ellendigen en armen hun recht,

  Bevrijdt verdrukten en behoeftigen:~\sep\ ontrukt ze aan de hand van de bozen.''

  Zij hebben inzicht noch verstand, ze wandelen in duisternis;~\sep\ alle grondslagen der aarde worden geschokt.

  Ik heb gezegd: ``Goden zijt gij,~\sep\ en zonen van de Allerhoogste, gij allen.

  Toch zult gij sterven als mensen,~\sep\ neervallen als welke machthebber ook''.

  Rijs op, O God, en richt de aarde,~\sep\ want rechtens horen alle volken U toe.
\end{halfparskip}

\PSALMtitle{83}{De vijand dringt op!}

\begin{halfparskip}
  \psalmsubtitle{a) De vijanden, Heer, spannen tegen ons samen}

  Wil niet zwijgen, O Heer,~\sep\ wil niet zwijgen, O God, noch rusten!

  \qanona{Er is niemand zoals U onder de dingen die gemaakt zijn, O almachtige God.}

  Want zie, Uw vijanden woelen,~\sep\ en die U haten, steken het hoofd omhoog.

  Tegen Uw volk smeden zij plannen,~\sep\ en tegen Uw beschermelingen spannen zij samen.

  ``Komt'', zo spreken zij, ``vagen wij hen weg uit de rij van de volken,~\sep\ en dat men de naam van Israël niet meer gedenke.''

  Waarlijk, zij overleggen eensgezind,~\sep\ en sluiten een verbond tegen U:

  De tenten van Edom en de Ismaëlieten,~\sep\ Moab en de zonen van Hagar,

  Gebal, Ammon en Amalec,~\sep\ de Filistijnen en de bewoners van Tyrus;

  Ook de Assyriërs verbonden zich met hen,~\sep\ en leenden hun arm aan de zonen van Lot.
\end{halfparskip}

\begin{halfparskip}
  \psalmsubtitle{b) Verdelg de vijand, O God, gelijk weleer!}

  Behandel hen als Madian,~\sep\ als Sisara en Jabin bij de beek Cison,

  Die bij Endor werden verdelgd,~\sep\ en tot mest op het veld zijn gemaakt.

  Maak hun vorsten als Oreb en Zeb,~\sep\ als Zebëe en Salmana, al hun leiders,

  Die zeiden:~\sep\ ``Laten wij het gebied van God gaan bezetten.''

  Maak ze, mijn God, als bladeren die dwarrelen in de storm,~\sep\ als een strohalm, heen en weer gezwiept door de wind,

  Als vuur, dat bossen verteert,~\sep\ en als een vlam, die bergen verschroeit,

  Zo moogt Gij hen vervolgen met Uw stormwind,~\sep\ door Uw orkaan hen verwarren.

  Overdek hun gelaat met schande,~\sep\ opdat zij Uw Naam zoeken, O Heer;

  Laat ze beschaamd en verbijsterd staan voor eeuwig,~\sep\ laat ze te schande worden en vergaan.

  Dat zij erkennen dat Gij, wiens naam de Heer is,~\sep\ de enig Verhevene zijt over heel de aarde.
\end{halfparskip}

\PSALMtitle{84}{Verlangen naar het huis van God}

\begin{halfparskip}
  \psalmsubtitle{a) Mijn God, ik verlang naar Uw heiligdom}

  Hoe liefelijk is Uw woonstede, Heer der legerscharen;~\sep\ mijn ziel verlangt, ziet smachtend uit naar de voorhoven van de Heer.

  \qanona{Hoe heerlijk en prachtig is Uw heiligdom, O God, die alles heiligt.}

  Mijn hart en mijn lichaam,~\sep\ juichen voor de levende God.

  Ook de mus vindt een woning,~\sep\ en de zwaluw een nest, waar ze haar jongen in neerlegt.

  Uw altaren, O Heer der legerscharen,~\sep\ mijn Koning en mijn God!
\end{halfparskip}

\begin{halfparskip}
  \psalmsubtitle{b) Gelukkig hij, die naar het heiligdom gaat}

  Gelukkig zij, die wonen in Uw huis, O Heer;~\sep\ eeuwig loven zij U.

  Gelukkig de man, die hulp krijgt van U,~\sep\ als hij het plan heeft op bedevaart te gaan:

  Trekken zij door een dorre vallei, dan maken zij haar tot bron,~\sep\ en de vroege regen bekleedt haar met zegeningen.

  Al gaande zal hun kracht vermeerderen:~\sep\ de God der goden zullen zij in Sion zien.
\end{halfparskip}

\begin{halfparskip}
  \psalmsubtitle{c) Leid mij, Heer, naar Uw heiligdom!}

  Heer der legerscharen, hoor naar mijn bede,~\sep\ ach, luister toch, O God van Jacob.

  Zie toe, O God, ons schild,~\sep\ en zie op het gelaat van Uw gezalfde.

  Waarlijk, één dag in Uw voorhoven is beter,~\sep\ dan duizend andere.

  Liever blijf ik staan op de drempel van het huis van mijn God,~\sep\ dan te toeven in de tenten der bozen.

  Want een zon en een schild is God de Heer;~\sep\ de Heer schenkt genade en glorie.

  Hij weigert het goede niet,~\sep\ aan die in onschuld wandelen.

  Heer der legerscharen,~\sep\ gelukkig de mens, die op U vertrouwt.
\end{halfparskip}

\Slota{Schep genoegen, Heer, in het gebed van Uw dienaren; neem behagen in de dienst van Uw aanbidders; vergeef de schulden van hen die U verheerlijken; en verwijder (\translationoptionNl{laat verdwijnen}) Uw woede van hen van Uw huishouden, U die goed bent en voor onze levens zorgt, in alle seizoenen en tijden, Heer van alles...}

\marmita{33}

\PSALMtitle{85}{Verlangen naar het Messiaanse Rijk}

\begin{halfparskip}
  \psalmsubtitle{a) Dank, Heer, voor de bevrijding!}

  Gij zijt Uw land genadig geweest, O Heer,~\sep\ hebt het lot van Jacob ten goede gekeerd.

  \qanona{Zend, O onze Heer, hulp en redding aan Uw gelovigen door de grote kracht van Uw Kruis.}

  \liturgicallbracket\liturgicaloption{Of:} \emph{Wie de zon wil afbeelden, vergist zich enorm, want hij heeft niet het verstand om haar heerlijke dingen te vergelijken (\translationoptionNl{te bevatten}). Hij die een erfgenaam wil zijn van het hemels koninkrijk, moet zijn ziel bevrijden van de slavernij van deze wereld.}\liturgicalrbracket

  Vergeven hebt Gij de schuld van Uw volk,~\sep\ en al zijn zonden bedekt.

  Uw gramschap hebt Gij geheel bedwongen,~\sep\ de gloed van Uw toorn gestild.
\end{halfparskip}

\begin{halfparskip}
  \psalmsubtitle{b) Voltooi, Heer, de Verlossing!}

  Herstel ons, o God, onze Redder,~\sep\ en leg Uw wrevel tegen ons af.

  Zult Gij dan eeuwig tegen ons toornen,~\sep\ of verbolgen blijven van geslacht tot geslacht?

  Zult Gij ons dan niet opnieuw doen leven,~\sep\ opdat Uw volk zich verblijde in U?

  Toon ons, Heer, Uw barmhartigheid,~\sep\ en schenk ons Uw heil!
\end{halfparskip}

\begin{halfparskip}
  \psalmsubtitle{c) Gij zult ons zegenen, Heer}

  Ik wil horen naar wat de Heer God spreekt:~\sep\ vrede voorzeker kondigt Hij aan.

  Voor Zijn volk en Zijn heiligen,~\sep\ en voor hen, die zich van harte keren tot Hem.

  Ja waarlijk, Zijn heil is nabij voor wie Hem vrezen,~\sep\ en zo zal er glorie wonen in ons land:

  Barmhartigheid en trouw zullen elkander ontmoeten,~\sep\ gerechtigheid en vrede elkander de kus geven.

  Trouw zal aan de aarde ontspruiten,~\sep\ en gerechtigheid neerzien vanuit de hemel.

  De Heer zelf zal zegen schenken,~\sep\ en ons land zijn vruchten geven.

  Gerechtigheid zal vóór Hem uitgaan,~\sep\ en heil zijn schreden volgen.
\end{halfparskip}

\PSALMtitle{86}{Bede om hulp}

\begin{halfparskip}
  \psalmsubtitle{a) Mijn God, help mij!}

  Neig Uw oor, O Heer; verhoor mij,~\sep\ want ik ben ellendig en arm.

  \qanona{Christus, de Vriend van de boeteling, open de deur voor ons gebed en aanvaard onze bede.}

  Bescherm mij, want ik ben U toegewijd;~\sep\ red Uw dienaar, die op U hoopt.

  Mijn God zijt Gij; wees mij genadig, O Heer,~\sep\ want almaar door roep ik tot U.

  Verblijd de ziel van Uw dienaar,~\sep\ want tot U, O Heer, verhef ik mijn ziel.

  Want Gij, O Heer, zijt goed en genadig,~\sep\ vol erbarming voor al wie U aanroept.

  Luister, Heer, naar mijn bede,~\sep\ en geef acht op de stem van mijn smeken.

  Op de dag van mijn kwelling riep ik tot U,~\sep\ omdat Gij mij verhoren zult.
\end{halfparskip}

\begin{halfparskip}
  \psalmsubtitle{b) Gij zult mij helpen, O Heer}

  Onder de goden, O Heer, is er geen als Gij,~\sep\ en geen werk is gelijk aan het Uwe.

  Alle volken, door U geschapen, zullen komen, en U aanbidden, O Heer,~\sep\ en verheerlijken Uw Naam.

  Want Gij zijt groot en Gij doet wonderwerken:~\sep\ Gij zijt God, en Gij alleen.
\end{halfparskip}

\begin{halfparskip}
  \psalmsubtitle{c) Gij wilt mij helpen, O Heer}

  Toon mij Uw weg, O Heer, opdat ik wandele in Uw waarheid,~\sep\ richt mijn hart op de vrees voor Uw Naam.

  Ik zal U prijzen, Heer, mijn God, uit heel mijn hart:~\sep\ en eeuwig Uw Naam verheerlijken.

  Want Uw erbarming voor mij was groot,~\sep\ en uit de diepten van het dodenrijk hebt Gij mij opgehaald.

  Trotsen, O God, zijn tegen mij opgestaan, een bende geweldenaars staat mij naar het leven,~\sep\ zij houden U niet voor ogen.

  Maar Gij, O Heer, zijt een barmhartige en liefdevolle God,~\sep\ lankmoedig, rijk aan ontferming en trouw.

  Blik op mij neer en wees mij genadig;~\sep\ schenk aan Uw dienaar Uw kracht, en red de zoon van Uw dienstmaagd.

  Geef mij een teken van Uw gunst, opdat die mij haten, Heer, vol schaamte zien,~\sep\ dat Gij, O Heer, mij hulp en troost hebt geschonken.
\end{halfparskip}

\Slota{Bevestig, mijn Heer, de fundamenten van Uw Kerk in Uw mededogen, versterk haar balken in Uw liefderijke goedheid, en laat Uw heerlijkheid wonen in de tempel gereserveerd voor Uw dienst, alle dagen van de wereld, Heer van alles...}

\marmita{34}

\PSALMtitle{87}{Ode aan Jeruzalem}

\begin{halfparskip}
  Zijn stichting op de heilige bergen bemint de Heer:~\sep\ de poorten van Sion~---

  \qanona{Aanbiddenswaardig is God de Schepper, die zorgt voor alle generaties.}

  \liturgicallbracket\liturgicaloption{Of:} \emph{Wij zijn arm en zwak (geworden) door onze daden, er zijn onder ons geen rechtvaardigen en deugdzamen om U gunstig te stemmen. We weten ook niet hoe te bidden of hoe (U) te verheerlijken, en we vrezen woorden uit te spreken die U niet waardig zijn, daarom bouwde de Heer voor Hem een huis op aarde, zodat wie de Heer wil zien, zou komen naar Zijn huis.}\liturgicalrbracket

  boven alle tenten van Jacob.

  Roemrijke dingen verhaalt men van u,~\sep\ O stad van God!

  Rahab en Babel zal Ik tot Mijn vereerders rekenen:~\sep\ zie, Filistea en Tyrus en het volk der Ethiopiërs: daar zijn ze geboren!

  Over Sion zal men zeggen: ``Allen, man voor man, zijn in haar geboren,~\sep\ en de Allerhoogste zelf heeft haar bevestigd.''

  De Heer zal schrijven in het boek der volken:~\sep\ ``Daar zijn ze geboren.''

  En in reidans zullen zij zingen:~\sep\ ``Al mijn bronnen zijn in u.''
\end{halfparskip}

\PSALMtitle{87}{De vertwijfeling nabij!}

\begin{halfparskip}
  \psalmsubtitle{a) Groot, Heer, is mijn ellende!}

  Heer, mijn God, ik roep overdag,~\sep\ en ik jammer 's nachts voor Uw aanschijn.

  \qanona{Gij, die ons gemaakt hebt, zijt barmhartig; in Uw goedertierenheid heb medelijden met ons.}

  Dringe mijn bede toch door tot U,~\sep\ neig Uw oor naar mijn klagen!

  Want mijn ziel is verzadigd met rampen,~\sep\ mijn leven is het dodenrijk nabij.

  Ik word gerekend onder hen, die ten grave dalen,~\sep\ ik ben als een man zonder kracht.

  Onder de doden is mijn legerstede,~\sep\ als van verslagenen, die liggen in het graf,

  Aan wie Gij niet meer denkt,~\sep\ die aan Uw zorgen zijn onttrokken.

  In een diepe groeve hebt Gij mij neergelegd,~\sep\ in duisternis, in een diep ravijn.

  Uw verontwaardiging drukt zwaar op mij,~\sep\ met al Uw golven slaat Gij mij neer.

  Gij hebt mijn vrienden van mij vervreemd, mij tot afschuw voor hen gemaakt;~\sep\ ik zit gevangen, en kan niet ontkomen.
\end{halfparskip}

\begin{halfparskip}
  \psalmsubtitle{b) Heer, kom mij te hulp!}

  Van ellende verkwijnen mijn ogen; iedere dag roep ik tot U, O Heer,~\sep\ naar U strek ik mijn handen uit.

  Of doet Gij voor doden nog wonderen,~\sep\ of zullen gestorvenen, herrijzend, U loven?

  Of wordt Uw goedheid in het graf verkondigd,~\sep\ Uw trouw in het dodenrijk?

  Openbaart men in het duister Uw wonderen,~\sep\ in het land der vergetelheid Uw genade?
\end{halfparskip}

\begin{halfparskip}
  \psalmsubtitle{c) Help mij, Heer, in mijn ongeluk!}

  Ik echter roep tot U, O Heer,~\sep\ mijn bede stijgt tot U op in de morgen.

  Waarom toch, O Heer, verstoot Gij mij,~\sep\ verbergt Gij voor mij Uw gelaat?

  Van jongs af ben ik ellendig en stervend,~\sep\ ik torste Uw verschrikkingen en kwijnde.

  Uw toorn is over mij heengegaan,~\sep\ Uw verschrikkingen sloegen mij neer.

  Zij omgeven mij immer als water,~\sep\ omringen mij alle tezamen.

  Vriend en makker hebt Gij van mij vervreemd,~\sep\ mijn vertrouweling is de duisternis.
\end{halfparskip}

% % % % % % % % % % % % % % % % % % % % % % % % % % % % % % % % % % % % % % % %

\hulala{13}

\Slota{Stort Uw genaden over ons uit, O Heer, vermenigvuldig Uw hulp voor ons en sterk ons, zoals U gewend bent, om U te behagen volgens Uw wil, om te wandelen volgens Uw geboden en Uw Godheid gunstig te stemmen met goede daden van gerechtigheid, alle dagen van ons leven, Heer van alles...}

\marmita{35}

\PSALMtitle{}{}

\begin{halfparskip}
  \psalmsubtitle{}
\end{halfparskip}

% % % % % % % % % % % % % % % % % % % % % % % % % % % % % % % % % % % % % % % %

\end{document}