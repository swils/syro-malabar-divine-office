\documentclass[12pt,twoside,a5paper]{article}

\usepackage{multicol}

\usepackage[main=dutch]{babel}
\usepackage{divine-office}

% % % % % % % % % % % % % % % % % % % % % % % % % % % % % % % % % % % % % % % %

% Version: 2024-06-11

\begin{document}

\title{Ramsa --- weekdagen}
\author{}
\date{}
\maketitle

% % % % % % % % % % % % % % % % % % % % % % % % % % % % % % % % % % % % % % % %

\begin{halfparskip}
  \cc~Eer aan God in den hoge \liturgicalhint{(3x)}. En op aarde vrede en goede hoop aan de mensen, altijd en in eeuwigheid. [Amen\footnote{De Hudra en de Syro-Malabar liturgie voegen hier ``Amen'' aan toe, in tegenstelling tot Breviarium.}.]~--- \rr~Zegen, Heer.~--- \liturgicalhint{Vredekus.}

  \cc~Onze Vader die in de hemelen zijt,

  \rr~Geheiligd zij Uw Naam. Uw rijk kome, heilig, heilig, heilig zijt Gij. Onze Vader die in de hemelen zijt, de hemel en de aarde zijn gevuld met Uw onmetelijke glorie; de engelen en de mensen roepen U toe: heilig, heilig, heilig zijt Gij. --- Onze Vader die in de hemelen zijt, geheiligd zij Uw Naam. Uw rijk kome, Uw wil geschiede op aarde zoals in de hemel. Geef ons heden het brood dat we nodig hebben en vergeef ons onze schulden en zonden zoals wij ook vergeven hebben aan onze schuldenaren. En leid ons niet in bekoring, maar verlos ons van de Kwade. Want van U is het koninkrijk en de kracht en de heerlijkheid in eeuwigheid, amen.

  \cc~Eer aan de Vader, de Zoon, en de Heilige Geest.

  \rr~Vanaf het begin en in alle eeuwigheid, amen en amen. Onze Vader die in de hemelen zijt, geheiligd zij Uw naam, Uw rijk kome, heilig, heilig, heilig zijt Gij. Onze Vader die in de hemelen zijt, de hemel en de aarde zijn gevuld met Uw onmetelijke glorie; de engelen en de mensen roepen U toe: heilig, heilig, heilig zijt Gij.

  \dd~Laat ons bidden, vrede zij met ons.

  \cc~We willen, Heer, Uw Godheid prijzen \liturgicalhint{(herhaal)} en Uw Majesteit aanbidden, en aan Uw glorierijke Drievuldigheid eeuwige, altijddurende lof brengen, Heer van alles, Vader... in alle eeuwigheid. --- \rr~Amen.
\end{halfparskip}

% % % % % % % % % % % % % % % % % % % % % % % % % % % % % % % % % % % % % % % %

\markedsection{Marmita}

\liturgicalhint{Na de eerste zin van de eerste psalm van elke marmita, zeg 3x alleluja, en herhaal het eerste vers.}

\begin{halfparskip}
  \liturgicalOption{Maandagen ``voor'':} \liturgicalhint{Marmita 4 (\Pss{11--14}).}
\end{halfparskip}

\begin{halfparskip}
  \psalm{\Ps{11}} Ik vlucht tot de Heer; hoe kunt gij mij zeggen:~\sep\ ``Vlieg weg als een vogel naar het gebergte.

  \qanona{Alleluia, alleluia, alleluia.} --- \liturgicalhint{Eerste vers.}

  Want zie, de bozen spannen de boog; ze zetten de pijl op de pees,~\sep\ om de oprechten van hart in het duister te treffen.

  Als zelfs de grondvesten worden gesloopt,~\sep\ wat zal de gerechtige dan nog vermogen?''

  De Heer woont in Zijn heilige tempel,~\sep\ de Heer heeft in de hemel Zijn troon.

  Zijn ogen zien rond,~\sep\ Zijn wimpers doorvorsen de kinderen der mensen.

  De Heer doorvorst de gerechte en de boze;~\sep\ die het onrecht liefheeft, is Hem een gruwel.

  Hij zal op de zondaars gloeiende kolen en zwavel doen regenen;~\sep\ een verzengende wind is de dronk van hun beker.

  Want de Heer is rechtvaardig en heeft de gerechtigheid lief;~\sep\ de goeden zullen Zijn aanschijn aanschouwen.
\end{halfparskip}

\begin{halfparskip}
  \psalm{\Ps{12}} Heer, schenk redding, want er zijn geen vromen meer;~\sep\ verdwenen is de trouw onder de kinderen der mensen.

  Allen liegen ze elkander voor,~\sep\ ze spreken met bedrieglijke lippen en vals gemoed.

  De Heer rukke al die bedrieglijke lippen uit,~\sep\ die grootsprekende tong,

  Hen die zeggen: ``Sterk zijn wij door onze tong;~\sep\ wij hebben onze lippen met ons, wie kan ons overmeesteren?''

  ``Om de nood der verdrukten en het gejammer der armen zal Ik nu opstaan,'' zegt de Heer:~\sep\ ``redding zal Ik brengen aan wie er naar smacht.''

  De woorden van de Heer zijn oprechte woorden,~\sep\ zuiver zilver, van stof ontdaan, tot zevenmaal gelouterd.

  Gij, Heer, zult ons behouden,~\sep\ ons eeuwig beschermen tegen dit geslacht.

  De bozen zwermen om ons heen,~\sep\ terwijl de heffe van het volk oprijst.
\end{halfparskip}

\begin{halfparskip}
  \psalm{\Ps{13}} Hoelang nog, Heer, zult Gij mij geheel vergeten,~\sep\ hoelang nog voor mij Uw aanschijn verbergen?

  Hoelang nog zal ik de smart overdenken in mijn ziel,~\sep\ en het wee in mijn hart van dag tot dag?

  Hoelang nog zal mijn vijand zich boven mij verheffen?~\sep\ Zie neer en verhoor mij, o Heer, mijn God!

  Stort licht in mijn ogen opdat ik de doodsslaap niet inga~\sep\ en mijn vijand niet zegge: ``Ik heb hem overwonnen'',

  En mijn weerstrevers niet juichen over mijn val,~\sep\ daar ik op Uw erbarming vertrouwde.

  Nu juiche mijn hart om Uw hulp!~\sep\ De Heer wil ik bezingen, die mij heeft welgedaan.
\end{halfparskip}

\begin{halfparskip}
  \psalm{\Ps{14}} De dwaas zegt bij zichzelf:~\sep\ ``Er is geen God.''

  Ze zijn bedorven, gruwelen hebben ze bedreven;~\sep\ daar is er niet één, die deugdzaam handelt.

  De Heer blikt uit de hemel neer op de kinderen der mensen,~\sep\ om te zien of er wel één is met verstand, wel één die God zoekt.

  Allen zonder uitzondering zijn ze afgedwaald, allen diep bedorven;~\sep\ daar is er niet één die deugdzaam handelt, niet één.

  Zullen al die bozen dan nimmer tot inzicht komen,~\sep\ zij, die mijn volk verslinden, als aten zij brood?

  Zij riepen de Heer niet aan. Eens zullen zij sidderen van angst,~\sep\ want God is met het geslacht der rechtvaardigen.

  Het beleid van de verdrukte wilt gij beschamen,~\sep\ maar de Heer is zijn toevlucht.

  O, mocht er uit Sion toch heil voor Israël dagen! Als de Heer het lot van Zijn volk ten goede keert,~\sep\ zal er gejubel zijn in Jacob, vreugde in Israël.
\end{halfparskip}

\begin{halfparskip}
  \liturgicalhint{Eer} --- \liturgicalhint{3x Alleluia.} --- \liturgicalhint{Eerste vers:} Ik vlucht tot de Heer; hoe kunt gij mij zeggen:~\sep\ ``Vlieg weg als een vogel naar het gebergte.
\end{halfparskip}

\begin{halfparskip}
  \liturgicalOption{Dinsdagen ``voor'':} \liturgicalhint{Marmita 9 (\Pss{25--27}).}
\end{halfparskip}

\begin{halfparskip}
  \psalm{\Ps{25}} Tot U verhef ik mijn ziel,~\sep\ o Heer, mijn God.~\sep\ Op U vertrouw ik; laat mij niet te schande worden;

  \qanona{Alleluia, alleluia, alleluia.} --- \liturgicalhint{Eerste vers.}

  dat mijn vijanden niet over mij juichen!

  Want van wie op U hopen, wordt niemand beschaamd,~\sep\ maar wel worden te schande, die hun woord vermetel breken.

  Toon mij Uw wegen, o Heer,~\sep\ en leer mij Uw paden kennen.

  Leid mij in Uw waarheid, en geef mij onderricht, omdat Gij, God, mijn Redder zijt,~\sep\ en immer hoop ik op U.

  Gedenk Uw ontferming, o Heer,~\sep\ en Uw barmhartigheid, die van oudsher zijn.

  De zonden van mijn jeugd en mijn misslagen, gedenk ze niet; wees mij naar Uw erbarming indachtig, *
  vanwege Uw goedheid, Heer.

  Goed en rechtvaardig is de Heer;~\sep\ daarom wijst Hij de zondaars de weg.

  De nederigen leidt Hij in gerechtigheid,~\sep\ de nederigen toont Hij Zijn weg.

  Alle wegen van de Heer zijn goedheid en trouw,~\sep\ voor wie Zijn Verbond en Zijn wetten bewaren.

  Omwille van Uw Naam, o Heer,~\sep\ vergeef mij mijn zonde, want zij is groot.

  Wie is de man, die de Heer vreest?~\sep\ Hij wijst hem de weg, die hij moet kiezen.

  Hij zelf zal in voorspoed leven,~\sep\ en zijn geslacht het land bezitten.

  De Heer is een vriend voor hen, die Hem vrezen:~\sep\ Zijn Verbond doet Hij hun kennen.

  Mijn ogen zijn immer gericht op de Heer,~\sep\ want uit de strik zal Hij mijn voeten bevrijden.

  Zie op mij neer en wees mij genadig,~\sep\ want eenzaam ben ik en ellendig.

  Verlicht de druk van mijn hart,~\sep\ en bevrijd mij van mijn angsten.

  Zie mijn ellende en mijn kwelling;~\sep\ en vergeef mij al mijn zonden.

  Let op mijn vijanden, want ze zijn talrijk,~\sep\ en haten mij met felle haat.

  Bescherm mijn leven en red mij;~\sep\ het zij mij niet tot schande, dat ik bij U mijn toevlucht zocht.

  Dat mijn onschuld en deugd mij beschermen,~\sep\ daar ik hoop op U, o Heer.

  Verlos Israël, o God,~\sep\ uit al zijn kommernissen!
\end{halfparskip}

\begin{halfparskip}
  \psalm{\Ps{26}} Heer, schaf mij recht, want ik leefde in onschuld;~\sep\ vertrouwend op de Heer, heb ik niet gewankeld.

  Onderzoek mij, Heer, en stel mij op de proef;~\sep\ doorgrond mijn nieren en mijn hart.

  Want Uw welwillendheid staat mij voor ogen,~\sep\ en ik wandel naar Uw waarheid.

  Met ongerechtigen zit ik niet neer,~\sep\ en met bedriegers kom ik niet samen.

  Ik haat het gezelschap van bozen,~\sep\ en met goddelozen zit ik niet samen.

  In onschuld was ik mijn handen,~\sep\ en ga rond Uw altaar, o Heer.

  Om openlijk Uw lof te verkondigen,~\sep\ en al Uw wonderen te verhalen.

  Heer, ik heb lief het verblijf van Uw huis,~\sep\ en de woontent van Uw heerlijkheid.

  Ruk mijn ziel niet weg met de zondaars,~\sep\ noch mijn leven met bloeddorstige mannen,

  Aan wier handen de misdaad kleeft,~\sep\ en wier rechterhand met geschenken gevuld is.

  Ik echter wandel in onschuld:~\sep\ red mij, en wees mij genadig.

  Mijn voet staat op effen baan;~\sep\ ik zal de Heer in de vergadering loven.
\end{halfparskip}

\begin{halfparskip}
  \psalm{\Ps{27}} De Heer is mijn licht en mijn heil: wie zou Ik vrezen?~\sep\ De Heer is de schuts van mijn leven: voor wie zou ik sidderen?

  Als de bozen mij bestormen om mijn vlees te verslinden,~\sep\ mijn vijanden en haters, zij struikelen en vallen.

  Al stond er een krijgsmacht tegenover mij, mijn hart zou niet vrezen;~\sep\ al brak er een oorlog tegen mij uit, dan nog zou ik vertrouwen.

  Dit alleen vraag ik de Heer, dit alleen streef ik na:~\sep\ te wonen in het huis van de Heer alle dagen van mijn leven,

  Te genieten de zoetheid van de Heer,~\sep\ en Zijn tempel te aanschouwen.

  Want in Zijn woontent zal Hij mij bergen in tijden van nood,~\sep\ Hij zal mij doen schuilen diep in Zijn tent, mij plaatsen boven op de rots.

  Nu verheft zich mijn hoofd,~\sep\ boven de vijanden, die mij omringen;

  Jubeloffers zal ik brengen in Zijn tent,~\sep\ zingen voor de Heer en spelen op de citer.

  Heer, luister naar mijn stem, waarmee ik luid roep,~\sep\ ontferm u over mij, en schenk mij verhoring!

  Tot U spreekt mijn hart, U zoeken mijn ogen;~\sep\ ik zoek Uw aanschijn, o Heer.

  Verberg mij Uw aanschijn niet,~\sep\ stoot Uw dienaar niet af in Uw toorn!

  Gij zijt mijn hulp; verwerp mij niet!~\sep\ Verlaat mij niet, o God, mijn Redder!

  Zou mijn vader en moeder mij ook verlaten,~\sep\ dan nog neemt de Heer mij op.

  Wijs mij Uw weg, o Heer,~\sep\ en leid mij op effen baan omwille van mijn weerstrevers.

  Geef mij niet prijs aan de moedwil van mijn vijanden,~\sep\ want valse getuigen en geweldenaars stonden tegen mij op.

  Ik ben er zeker van de weldaden van de Heer te zien,~\sep\ in het land der levenden.

  Zie uit naar de Heer, wees onversaagd;~\sep\ sterk zij Uw hart, zie uit naar de Heer.
\end{halfparskip}

\begin{halfparskip}
  \liturgicalhint{Eer} --- \liturgicalhint{3x Alleluia.} --- \liturgicalhint{Eerste vers:} Tot U verhef ik mijn ziel,~\sep\ o Heer, mijn God.~\sep\ Op U vertrouw ik; laat mij niet te schande worden.
\end{halfparskip}

\begin{halfparskip}
  \liturgicalOption{Woensdagen ``voor'':} \liturgicalhint{Marmita 23 (\Pss{62--64}).}
\end{halfparskip}

\begin{halfparskip}
  \psalm{\Ps{62}} In God alleen rust mijn ziel,~\sep\ van Hem komt mijn heil.

  \qanona{Alleluia, alleluia, alleluia.} --- \liturgicalhint{Eerste vers.}

  Hij alleen is mijn rots en mijn heil,~\sep\ mijn bescherming: neen, ik zal niet wankelen.

  Hoe lang stormt gij op één man los, werpt gij hem met u allen omver,~\sep\ als een hellende wand, als een
  wankelende muur?

  Ja waarlijk, zij beramen hoe zij mij van mijn verheven plaats zullen stoten;~\sep\ zij verlustigen zich in de leugens;

  Ze zegenen met hun mond,~\sep\ maar met hun hart vervloeken zij.

  Rust in God alleen, mijn ziel,~\sep\ want van Hem komt wat ik hoop.

  Hij alleen is mijn rots en mijn heil,~\sep\ mijn bescherming: neen, ik zal niet wankelen.

  Bij God is mijn heil en mijn roem,~\sep\ mijn sterke rots: in God is mijn schuilplaats.

  Hoop op Hem, o volk, ten allen tijde: stort voor Hem uw harten uit:~\sep\ God is ons tot toevlucht.

  Een ademtocht slechts zijn de kinderen der mensen,~\sep\ bedrieglijk de zonen van mannen.

  Zij rijzen op de weegschaal omhoog:~\sep\ lichter dan een zucht zijn zij allen tezamen.

  Verwacht niets van verdrukking, en roem niet ijdel op roof;~\sep\ als uw vermogen toeneemt, wilt er uw hart niet aan hechten.

  Eén zaak heeft God gezegd; deze twee dingen vernam ik: "Bij God berust de macht, en bij U de goedheid, o Heer,~\sep\ want een ieder zult Gij naar werk vergelden.
\end{halfparskip}

\begin{halfparskip}
  \psalm{\Ps{63}} God, mijn God zijt Gij:~\sep\ met aandrang zoek ik U.

  Naar U dorst mijn ziel, naar U smacht mijn lichaam,~\sep\ als een dor en dorstig, waterloos land.

  Zo blijf ik U beschouwen in Uw heiligdom,~\sep\ om Uw macht te zien en Uw glorie.

  Daar Uw genade meer dan het leven geldt,~\sep\ zullen mijn lippen U loven.

  Zo wil ik U prijzen mijn leven lang,~\sep\ in Uw Naam mijn handen verheffen.

  Als met merg en vet zal ik verzadigd worden,~\sep\ en met jubelende lippen zal mijn mond U loven.

  Als ik aan U denk op mijn legerstede,~\sep\ blijf ik in de nachtwaken peinzen over U.

  Want Gij zijt mijn Helper geworden,~\sep\ en ik juich in de schaduw van Uw vleugelen.

  Mijn ziel hecht zich aan U,~\sep\ Uw rechterhand is mij tot steun.

  Maar zij, die mij naar het leven staan,~\sep\ zullen verzinken in de diepten der aarde.

  Ze zullen vallen in de macht van het zwaard,~\sep\ en de prooi van vossen worden.

  Maar de koning zal zich verblijden in God; een ieder, die bij Hem zweert, zal roemen,~\sep\ omdat de mond van de lasteraars zal worden gestopt.
\end{halfparskip}

\begin{halfparskip}
  \psalm{\Ps{64}} Luister, o God, als ik klaag, naar mijn stem;~\sep\ behoed mijn leven voor de schrik van de vijand.

  Bescherm mij tegen de samenscholing van bozen,~\sep\ tegen het woelen van hen, die onrecht bedrijven.

  Die hun tongen scherpen als een zwaard,~\sep\ als pijlen hun giftige woorden richten,

  Om vanuit een schuilplaats de onschuldige te treffen,~\sep\ hem onverhoeds te treffen, zonder iets te duchten.

  Vastberaden smeden zij boze plannen, en werken samen om heimelijk strikken te spannen;~\sep\ zij zeggen: ``Wie slaat er acht op ons?''

  Zij denken misdaden uit, verbergen hun weloverwogen gedachten;~\sep\ een afgrond is hun geest en hart.

  Maar God treft hen met Zijn pijlen:~\sep\ onverhoeds worden zij met wonden geslagen,

  En hun eigen tong brengt hun verderf;~\sep\ allen, die hen zien, schudden het hoofd.

  En allen zijn vol ontzag en prijzen Gods werk,~\sep\ en overwegen Zijn daden.

  De rechtvaardige juicht in de Heer en vlucht tot Hem,~\sep\ en allen roemen, die oprecht van harte zijn.
\end{halfparskip}

\begin{halfparskip}
  \liturgicalhint{Eer} --- \liturgicalhint{3x Alleluia.} --- \liturgicalhint{Eerste vers:} In God alleen rust mijn ziel,~\sep\ van Hem komt mijn heil.
\end{halfparskip}

\begin{halfparskip}
  \liturgicalOption{Donderdagen ``voor'':} \liturgicalhint{Marmita 38 (\Pss{96--98}).}
\end{halfparskip}

\begin{halfparskip}
  \psalm{\Ps{96}} Zingt een nieuw lied voor de Heer,~\sep\ zingt voor de Heer alle landen!

  \qanona{Alleluia, alleluia, alleluia.} --- \liturgicalhint{Eerste vers.}

  Zingt voor de Heer, zegent Zijn Naam,~\sep\ verkondigt Zijn heil van dag tot dag!

  Maakt onder de heidenen Zijn glorie bekend,~\sep\ onder alle volken Zijn wonderen.

  Want groot is de Heer en hoog te prijzen,~\sep\ meer te duchten dan alle goden.

  Want de goden der heidenen zijn allen verzinsels,~\sep\ maar de Heer heeft de hemel geschapen.

  Majesteit en pracht gaan vóór Hem uit;~\sep\ macht en luister zijn in Zijn heilige woning.

  Kent toe aan de Heer, geslachten der volken, kent toe aan de Heer glorie en macht;~\sep\ kent toe aan de Heer de roem van Zijn Naam!

  Treedt met Uw offer Zijn voorhoven binnen;~\sep\ aanbidt de Heer in heilige feesttooi!

  Beef voor Zijn aanschijn, geheel de aarde;~\sep\ roept tot de volken: De Heer is Koning!

  Hij maakte de aarde onwankelbaar vast,~\sep\ heerst over de volken met billijkheid.

  Dat de hemelen juichen en de aarde jubele, laat bruisen de zee met wat ze bevat,~\sep\ laat jubelen het veld met wat er op groeit.

  Dan zullen juichen alle bomen van het woud voor het aanschijn van de Heer, want Hij komt,~\sep\ want Hij komt de aarde regeren.

  Hij zal met rechtvaardigheid de wereld regeren,~\sep\ en de volkeren volgens Zijn trouw.
\end{halfparskip}

\begin{halfparskip}
  \psalm{\Ps{97}} De Heer is Koning, dat de aarde jubele,~\sep\ dat de vele eilanden juichen!

  Wolken en duisternis omgeven Hem,~\sep\ gerechtigheid en recht zijn de steun van Zijn troon.

  Vuur gaat vóór Hem uit,~\sep\ en verbrandt Zijn vijanden om Hem heen.

  Zijn bliksems verlichten het aardrijk;~\sep\ de aarde ziet het en beeft.

  Bergen versmelten als was voor de Heer,~\sep\ voor de Beheerser van heel de aarde.

  De hemelen verkondigen Zijn gerechtigheid,~\sep\ en alle volken aanschouwen Zijn glorie.

  Beschaamd staan allen die beelden vereren, die roemen op valse goden:~\sep\ voor Hem werpen alle goden zich neer.

  Sion hoort het vol vreugde, en de steden van Juda juichen,~\sep\ om Uw oordelen, Heer.

  Want Gij, Heer, zijt verheven boven heel de aarde,~\sep\ hoogverheven onder alle goden.

  De Heer heeft lief die het kwade haten, Hij behoedt het leven van Zijn heiligen,~\sep\ en redt hen uit de hand der bozen.

  Een licht rijst op voor de rechtvaardige,~\sep\ en blijdschap voor de oprechten van hart.

  Verheugt u, rechtschapenen, in de Heer,~\sep\ en verheerlijkt Zijn heilige Naam!
\end{halfparskip}

\begin{halfparskip}
  \psalm{\Ps{98}} Zingt een nieuw lied voor de Heer,~\sep\ want wonderen heeft Hij gewrocht;

  Zege bracht Hem Zijn rechterhand,~\sep\ Zijn heilige arm.

  De Heer heeft Zijn heil doen kennen,~\sep\ voor het oog van de volken Zijn gerechtigheid getoond.

  Zijn liefde en trouw was Hij indachtig,~\sep\ ten gunste van Israëls huis.

  Alle grenzen der aarde hebben aanschouwd,~\sep\ het heil van onze God.

  Juicht voor de Heer, alle landen,~\sep\ weest blij, verheugt u en tokkelt de snaren.

  Zingt voor de Heer bij citerspel,~\sep\ en de klank van de harp,

  Met trompetten en bazuingeschal,~\sep\ jubelt voor het aanschijn van de Koning en Heer!

  De zee, met wat ze bevat, verheffe haar stem,~\sep\ de aarde en die haar bewonen;

  Dat de stromen in de handen klappen,~\sep\ en tegelijk de bergen juichen.

  Voor het aanschijn van de Heer, want Hij komt,~\sep\ want Hij komt de aarde regeren.

  Hij zal met rechtvaardigheid de wereld regeren,~\sep\ en de volkeren met billijkheid.
\end{halfparskip}

\begin{halfparskip}
  \liturgicalhint{Eer} --- \liturgicalhint{3x Alleluia.} --- \liturgicalhint{Eerste vers:} Zingt een nieuw lied voor de Heer,~\sep\ zingt voor de Heer alle landen!
\end{halfparskip}

\begin{halfparskip}
  \liturgicalOption{Vrijdagen ``voor'':} \liturgicalhint{Marmita 33 (\Pss{85--86}).}
\end{halfparskip}

\begin{halfparskip}
  \psalm{\Ps{85}} Gij zijt Uw land genadig geweest, o Heer,~\sep\ hebt het lot van Jacob ten goede gekeerd.

  \qanona{Alleluia, alleluia, alleluia.} --- \liturgicalhint{Eerste vers.}

  Vergeven hebt Gij de schuld van Uw volk,~\sep\ en al zijn zonden bedekt.

  Uw gramschap hebt Gij geheel bedwongen,~\sep\ de gloed van Uw toorn gestild.

  Herstel ons, o God, onze Redder,~\sep\ en leg Uw wrevel tegen ons af.

  Zult Gij dan eeuwig tegen ons toornen,~\sep\ of verbolgen blijven van geslacht tot geslacht?

  Zult Gij ons dan niet opnieuw doen leven,~\sep\ opdat Uw volk zich verblijde in U?

  Toon ons, Heer, Uw barmhartigheid,~\sep\ en schenk ons Uw heil!

  Ik wil horen naar wat de Heer God spreekt:~\sep\ vrede voorzeker kondigt Hij aan.

  Voor Zijn volk en Zijn heiligen,~\sep\ en voor hen, die zich van harte keren tot Hem.

  Ja waarlijk, Zijn heil is nabij voor wie Hem vrezen,~\sep\ en zo zal er glorie wonen in ons land:

  Barmhartigheid en trouw zullen elkander ontmoeten,~\sep\ gerechtigheid en vrede elkander de kus geven.

  Trouw zal aan de aarde ontspruiten,~\sep\ en gerechtigheid neerzien vanuit de hemel.

  De Heer zelf zal zegen schenken,~\sep\ en ons land zijn vruchten geven.

  Gerechtigheid zal vóór Hem uitgaan,~\sep\ en heil zijn schreden volgen.
\end{halfparskip}

\begin{halfparskip}
  \psalm{\Ps{86}} Neig Uw oor, o Heer; verhoor mij,~\sep\ want ik ben ellendig en arm.

  Bescherm mij, want ik ben U toegewijd;~\sep\ red Uw dienaar, die op U hoopt.

  Mijn God zijt Gij; wees mij genadig, o Heer,~\sep\ want almaar door roep ik tot U.

  Verblijd de ziel van Uw dienaar,~\sep\ want tot U, o Heer, verhef ik mijn ziel.

  Want Gij, o Heer, zijt goed en genadig,~\sep\ vol erbarming voor al wie U aanroept.

  Luister, Heer, naar mijn bede,~\sep\ en geef acht op de stem van mijn smeken.

  Op de dag van mijn kwelling riep ik tot U,~\sep\ omdat Gij mij verhoren zult.

  Onder de goden, o Heer, is er geen als Gij,~\sep\ en geen werk is gelijk aan het Uwe.

  Alle volken, door U geschapen, zullen komen, en U aanbidden, O Heer,~\sep\ en verheerlijken Uw Naam.

  Want Gij zijt groot en Gij doet wonderwerken:~\sep\ Gij zijt God, en Gij alleen.

  Toon mij Uw weg, o Heer, opdat ik wandele in Uw waarheid,~\sep\ richt mijn hart op de vrees voor Uw Naam.

  Ik zal U prijzen, Heer, mijn God, uit heel mijn hart:~\sep\ en eeuwig Uw Naam verheerlijken.

  Want Uw erbarming voor mij was groot,~\sep\ en uit de diepten van het dodenrijk hebt Gij mij opgehaald.

  Trotsen, o God, zijn tegen mij opgestaan, een bende geweldenaars staat mij naar het leven,~\sep\ zij houden U niet voor ogen.

  Maar Gij, o Heer, zijt een barmhartige en liefdevolle God,~\sep\ lankmoedig, rijk aan ontferming en trouw.

  Blik op mij neer en wees mij genadig;~\sep\ schenk aan Uw dienaar Uw kracht, en red de zoon van Uw
  dienstmaagd.

  Geef mij een teken van Uw gunst, opdat die mij haten, Heer, vol schaamte zien,~\sep\ dat Gij, o Heer, mij hulp en troost hebt geschonken.
\end{halfparskip}

\begin{halfparskip}
  \liturgicalhint{Eer} --- \liturgicalhint{3x Alleluia.} --- \liturgicalhint{Eerste vers:} Gij zijt Uw land genadig geweest, o Heer,~\sep\ hebt het lot van Jacob ten goede gekeerd.
\end{halfparskip}

\begin{halfparskip}
  \liturgicalOption{Vrijdagen ``midden'':} \liturgicalhint{Marmita 33 of 34 (\Pss{85--86} of 87--88).}
\end{halfparskip}

\begin{halfparskip}
  \liturgicalOption{Zaterdagen ``voor'':} \liturgicalhint{Marmita 56 (\Pss{144--146}).}
\end{halfparskip}

\begin{halfparskip}
  \psalm{\Ps{144}} Ik wil U roemen, mijn God, de Koning,~\sep\ en prijzen Uw Naam in eeuwen der eeuwen.

  \qanona{Alleluia, alleluia, alleluia.} --- \liturgicalhint{Eerste vers.}

  Dag aan dag wil ik U prijzen,~\sep\ en loven Uw Naam in de eeuwen der eeuwen.

  Groot is de Heer en hoogst lofwaardig,~\sep\ en Zijn grootheid is niet te meten.

  Het ene geslacht roemt bij het andere Uw werken,~\sep\ en verkondigt Uw macht.

  Zij spreken over de heerlijke luister van Uw majesteit,~\sep\ en maken Uw wonderwerken bekend.

  Zij bezingen de kracht van Uw ontzagwekkende daden,~\sep\ en verhalen Uw grootheid.

  Zij verkondigen de lof van Uw grote goedheid,~\sep\ en juichen om Uw gerechtigheid.

  Zachtmoedig en barmhartig is de Heer,~\sep\ lankmoedig en genaderijk.

  Goed is de Heer voor allen,~\sep\ en barmhartig voor al Zijn werken.

  Dat al Uw werken U roemen, Heer,~\sep\ en dat Uw heiligen U prijzen,

  Dat zij de glorie van Uw rijk bezingen,~\sep\ en Uw macht verkondigen,

  Om aan de zonen der mensen Uw macht te doen kennen,~\sep\ en de roem van Uw luistervol rijk.

  Uw rijk is een rijk van alle eeuwen,~\sep\ en Uw heerschappij duurt voort door alle geslachten heen.

  Trouw is de Heer in al Zijn woorden,~\sep\ en heilig in al Zijn werken.

  De Heer steunt allen, die dreigen te vallen,~\sep\ en alle terneergedrukten beurt Hij op.

  De ogen van allen zien vertrouwvol naar U,~\sep\ en Gij schenkt hun spijs te bekwamen tijde,

  Gij opent Uw hand,~\sep\ en verzadigt vol goedheid al wat leeft.

  Rechtvaardig is de Heer in al Zijn wegen,~\sep\ en heilig in al Zijn werken.

  De Heer is allen, die Hem aanroepen, nabij,~\sep\ allen, die Hem aanroepen in oprechtheid.

  Hij zal de wensen vervullen van hen, die Hem vrezen,~\sep\ hun smeken aanhoren en hen redden.

  De Heer behoedt allen, die Hem beminnen,~\sep\ maar alle bozen zal Hij verdelgen.

  Laat mijn mond verkondigen de lof van de Heer,~\sep\ en alle vlees prijzen zijn heilige Naam in alle eeuwen der eeuwen.
\end{halfparskip}

\begin{halfparskip}
  \psalm{\Ps{145}} Loof, mijn ziel, de Heer! Loven zal ik de Heer mijn leven lang,~\sep\ mijn God bezingen zolang ik besta.

  Vertrouwt niet op vorsten,~\sep\ niet op een mens, door wie geen redding komt.

  Als zijn geest is geweken, keert hij terug tot zijn stof;~\sep\ dan vallen al zijn plannen in duigen.

  Gelukkig hij, die Jacobs God tot helper heeft,~\sep\ wiens hoop is gevestigd op de Heer, zijn God,

  Die hemel en aarde gemaakt heeft,~\sep\ en de zee met al wat zij bevatten,

  Die trouw blijft voor eeuwig, recht schaft aan de verdrukten,~\sep\ en brood aan de hongerigen schenkt.

  De Heer bevrijdt de gevangenen,~\sep\ de Heer opent blinden de ogen;

  De Heer richt gebukten op,~\sep\ de Heer heeft rechtvaardigen lief.

  De Heer behoedt de vreemden, is voor wees en weduwe een steun;~\sep\ maar de weg der bozen verstoort Hij.

  De Heer zal heersen in eeuwigheid,~\sep\ uw God, o Sion, van geslacht tot geslacht.
\end{halfparskip}

\begin{halfparskip}
  \psalm{\Ps{146}} Looft de Heer want Hij is goed, bezingt onze God want Hij is liefelijk:~\sep\ Hem past lofbetuiging.

  De Heer bouwt Jeruzalem op,~\sep\ verzamelt de verstrooiden van Israël.

  Hij geneest de gebrokenen van harte,~\sep\ en verbindt hun wonden.

  Hij bepaalt het getal der sterren,~\sep\ Hij noemt ze elk bij hun naam.

  Groot is onze Heer en machtig in kracht;~\sep\ Zijn wijsheid kent geen grenzen.

  De Heer heft de nederigen op,~\sep\ drukt de bozen neer in het stof.

  Zingt de Heer een danklied toe,~\sep\ speelt voor onze God op de citer,

  Die de hemel bedekt met wolken,~\sep\ voor de aarde de regen bereidt,

  Die op de bergen het gras doet ontspruiten,~\sep\ en kruid ten dienste der mensen,

  Die voedsel schenkt aan het vee,~\sep\ aan de jongen der raven, die roepen tot Hem.

  Niet in de kracht van het ros heeft Hij behagen,~\sep\ noch in de schenkels van de man Zijn welgevallen.

  Behagen heeft de Heer in hen, die Hem vrezen,~\sep\ en op Zijn goedheid vertrouwen.
\end{halfparskip}

\begin{halfparskip}
  \liturgicalhint{Eer} --- \liturgicalhint{3x Alleluia.} --- \liturgicalhint{Eerste vers:} Ik wil U roemen, mijn God, de Koning,~\sep\ en prijzen Uw Naam in eeuwen der eeuwen.
\end{halfparskip}

% % % % % % % % % % % % % % % % % % % % % % % % % % % % % % % % % % % % % % % %

\end{document}