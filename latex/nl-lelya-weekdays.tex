\documentclass[12pt,twoside,a5paper]{article}

\usepackage{multicol}

\usepackage[main=dutch]{babel}
\usepackage{divine-office}

% % % % % % % % % % % % % % % % % % % % % % % % % % % % % % % % % % % % % % % %

% Version: 2024-07-13
\begin{document}

\title{Lelya~--- weekdagen}
\author{}
\date{}
\maketitle

% The following prevents footnotes and paracol from interacting in bad ways.
% Not really an idea why...
% See: https://stackoverflow.com/questions/61779911/paracol-and-footnote-placing-in-latex
\footnotelayout{\ }

% % % % % % % % % % % % % % % % % % % % % % % % % % % % % % % % % % % % % % % %

\begin{halfparskip}
  \cc~Eer aan God in den hoge \liturgicalhint{(3x)}. En op aarde vrede en goede hoop aan de mensen, altijd en in eeuwigheid.

  [Amen].~--- \rr~Zegen, Heer.~--- \liturgicalhint{[vredekus].}

  \cc~Onze Vader die in de hemelen zijt,

  \rr~Geheiligd zij Uw Naam. Uw rijk kome, heilig, heilig, heilig zijt Gij. Onze Vader die in de hemelen zijt, de hemel en de aarde zijn gevuld met Uw onmetelijke glorie; de engelen en de mensen roepen U toe: heilig, heilig, heilig zijt Gij.~--- Onze Vader die in de hemelen zijt, geheiligd zij Uw Naam. Uw rijk kome, Uw wil geschiede op aarde zoals in de hemel. Geef ons heden het brood dat we nodig hebben en vergeef ons onze schulden en zonden zoals wij ook vergeven hebben aan onze schuldenaren. En leid ons niet in bekoring, maar verlos ons van de Kwade. Want van U is het koninkrijk en de kracht en de heerlijkheid in eeuwigheid, amen.

  \cc~Eer aan de Vader, de Zoon, en de Heilige Geest.

  \rr~Vanaf het begin en in alle eeuwigheid, amen en amen. Onze Vader die in de hemelen zijt, geheiligd zij Uw naam, Uw rijk kome, heilig, heilig, heilig zijt Gij. Onze Vader die in de hemelen zijt, de hemel en de aarde zijn gevuld met Uw onmetelijke glorie; de engelen en de mensen roepen U toe: heilig, heilig, heilig zijt Gij.

  \dd~Laat ons opstaan om te bidden, vrede zij met ons.\footnote{In deze verkorte editie hebben we elke hulala door een marmita vervangen.}

  \cc~Laat ons opstaan, o Heer, in Uw kracht, en bevestigd worden in Uw hoop, mogen we opgetild en gesterkt worden door de hoge arm van Uw macht; en mogen we waardig zijn, met de hulp van Uw goedertierenheid, om te allen tijde lof, eer, belijdenis en aanbidding tot U te verheffen, Heer van alles, Vader...
\end{halfparskip}

% % % % % % % % % % % % % % % % % % % % % % % % % % % % % % % % % % % % % % % %

\markedsection{PSALMEN}

\markedsection{Eerste marmita\footnote{Breviarium: alle Maandagen: hulala 1--3; Dinsdagen: 4--6; Woensdagen: 7--9; Donderdagen: 10--11 en 15; Vrijdagen: 16--18; Zaterdagen: 19--21; Hulale 12--14 zijn gereserveerd voor feesten en gedachtenissen.}\markedsectionhint{(origineel: 3 hullale)}}

\liturgicalhint{Eer aan... na elke marmita.}

\markedsubsectionrubricwithhint{Maandagen ``voor'': Marmita 1*}{(origineel: hulala 1)}

\begin{halfparskip}
  \psalm{\Ps{1}} Zalig de man die de raad van goddelozen niet volgt,~\sep\ die de weg van zondaars niet inslaat,

  \liturgicalhint{Alleluia, Alleluia, Alleluia.~--- Eerste vers.}

  noch neerzit in de kring van spotters.

  Maar die zijn vreugde vindt in de Wet van de Heer,~\sep\ en Zijn Wet overweegt bij dag en bij nacht.

  Hij is als een boom,~\sep\ geplant aan waterstromen.

  Die vrucht geeft op zijn tijd, en wiens lover niet verwelkt;~\sep\ ja, al wat hij doet, gedijt.

  Niet zo de goddelozen, niet zo;~\sep\ ze zijn als kaf, dat de wind verstrooit.

  Daarom zullen de goddelozen niet standhouden in het oordeel,~\sep\ noch de zondaars in de kring der
  rechtvaardigen.

  Want de Heer draagt zorg voor de weg der rechtvaardigen,~\sep\ maar de weg der goddelozen loopt uit op verderf.

  \psalm{\Ps{2}} Waarom woelen de heidenen,~\sep\ en smeden de naties ijdele plannen?

  De koningen der aarde rijzen samen op,~\sep\ en de vorsten spannen tezamen tegen de Heer en Zijn Gezalfde:

  ``Laten wij Hun boeien verbreken,~\sep\ en werpen wij Hun kluisters van ons af''.

  Die in de hemelen woont, Hij lacht hen uit,~\sep\ de Heer drijft de spot met hen.

  Dan spreekt Hij hen toe in Zijn toorn,~\sep\ in Zijn gramschap doet Hij hen sidderen:

  ``Maar Ik, Ik heb Mijn Koning aangesteld,~\sep\ op de Sion, Mijn heilige berg!''

  Ik wil het besluit van de Heer bekendmaken: de Heer sprak tot mij:~\sep\ ``Mijn Zoon zijt Gij, Ik heb U heden voortgebracht.

  Vraag Mij, en Ik geef U de volken tot erfdeel,~\sep\ en tot Uw bezit de grenzen der aarde.

  Gij zult hen regeren met ijzeren scepter,~\sep\ hen in stukken slaan als het vat van een pottenbakker''.

  Nu dan, koningen, komt tot inzicht,~\sep\ laat u gezeggen, die de wereld bestuurt.

  Dient de Heer in vreze en juicht Hem toe;~\sep\ met huivering Hem uw hulde gebracht!

  Opdat Hij niet toorne en gij van de weg af vergaat als weldra Zijn toorn zal zijn ontbrand;~\sep\ zalig allen die vluchten tot Hem.

  \psalm{\Ps{3}} Heer, hoe talrijk zijn zij, die mij kwellen!~\sep\ Velen staan tegen mij op!

  Velen zijn er, die van mij zeggen:~\sep\ ``Voor hem is er geen redding bij God!''

  Maar Gij, Heer, zijt mijn schild,~\sep\ Gij mijn roem, die mijn hoofd opbeurt.

  Ik riep tot de Heer met luide stem,~\sep\ en Hij verhoorde mij van Zijn heilige berg.

  Ik legde mij neer en ik sliep in;~\sep\ dan stond ik op, want de Heer is mijn steun.

  Neen, nu vrees ik de drommen van duizenden niet,~\sep\ die zich opstellen rondom mij heen.

  Verhef U, Heer!~\sep\ red mij, mijn God!

  Want al mijn weerstrevers hebt Gij op de kaken geslagen,~\sep\ de tanden der bozen hebt Gij verbrijzeld.

  Bij de Heer is redding:~\sep\ Uw zegen zij over Uw volk!

  \psalm{\Ps{4}} Verhoor mij, als ik U aanroep, mijn rechtvaardige God,

  die mij in kwelling verlichting bracht;~\sep\ wees mij genadig en verhoor mijn gebed.

  Mannen, hoelang nog blijft gij verstokt van hart;~\sep\ waarom ijdelheid bemind en leugen gezocht?

  Weet dat de Heer jegens Zijn heiligen wonderbaar handelt;~\sep\ de Heer zal mij verhoren, als ik Hem aanroep.

  Siddert en wilt niet zondigen,~\sep\ denkt na bij u zelf, op uw sponde, en zwijgt!

  Brengt gerechte offers,~\sep\ en hoopt op de Heer.

  Velen zeggen: ``Wie zal ons voorspoed doen zien?''~\sep\ Doe opgaan over ons, Heer, het licht van Uw gelaat!

  Gij hebt mij een vreugde in het hart gestort,~\sep\ groter dan bij overvloed van tarwe en wijn.

  Zodra ik mij neerleg, slaap ik in vrede,~\sep\ want Gij alleen, Heer, stelt mij in veiligheid.
\end{halfparskip}

\markedsubsectionrubricwithhint{Dinsdagen ``voor'': Marmita 11*}{(origineel: hulala 4)}

\begin{halfparskip}
  \psalm{\Ps{31}} Ik vlucht tot U, O Heer: dat ik nimmer te schande worde;~\sep\ bevrijd mij toch in Uw gerechtigheid!

  \liturgicalhint{Alleluia, Alleluia, Alleluia.~--- Eerste vers.}

  Neig Uw oor naar mij;~\sep\ haast U mij te redden!

  Wees mij een rots, waar ik vluchten kan,~\sep\ een versterkte burcht tot mijn behoud.

  Want Gij zijt mijn Rots en mijn burcht,~\sep\ en omwille van Uw Naam zult Gij mijn Leider zijn en Gids.

  Gij zult mij trekken uit het net, dat zij heimelijk mij spanden,~\sep\ want Gij zijt mijn toevlucht.

  In Uw handen beveel ik mijn geest:~\sep\ Gij zult mij bevrijden, O Heer, getrouwe God.

  Gij haat, die nietige afgoden dienen,~\sep\ maar ik vertrouw op de Heer.

  Vol vreugde zal ik juichen om Uw ontferming, omdat Gij op mijn ellende hebt neergezien:~\sep\ Gij waart mijn hulp in tijden van nood.

  Gij gaaft mij niet prijs aan de macht van een vijand,~\sep\ maar plaatste mijn voeten op ruime baan.

  O Heer, wees mij genadig, want ik ben in nood;~\sep\ van droefheid kwijnt mijn oog, mijn ziel en mijn lichaam.

  Ja, mijn leven teert weg in smart,~\sep\ en mijn jaren in geween.

  Van verdriet is mijn kracht gebroken,~\sep\ en mijn gebeente verdord.

  Voor al mijn vijanden ben ik tot smaad geworden, voor mijn buren tot spot, en tot afschrik voor mijn bekenden;~\sep\ die mij buiten zien, vluchten van mij heen.

  Ik ben als een dode door vergetelheid uit het hart gewist,~\sep\ en ik werd als een vat in scherven.

  Ja, ik hoorde het fluiten van velen - verschrikking van alle zijden!~\sep\ Zij schoolden tegen mij samen en zonnen op moord.

  Maar ik, O Heer, vertrouw op U,~\sep\ en zeg: Gij zijt mijn God.

  Mijn levenslot ligt in Uw hand,~\sep\ ontruk mij aan de hand van mijn vijanden en vervolgers.

  Toon aan Uw dienstknecht Uw vredig gelaat,~\sep\ red mij in Uw barmhartigheid.

  Heer, laat mij niet beschaamd worden, want U riep ik aan;~\sep\ maar dat de bozen zich schamen en zwijgen, voortgedreven naar het dodenrijk.

  Dat de leugenlippen verstommen,~\sep\ die vermetel tegen de rechtvaardige spreken, in trots en verachting.

  Hoe groot, Heer, is Uw goedheid,~\sep\ die Gij hebt weggelegd voor hen, die U vrezen,

  Die Gij bewijst aan hen, die vluchten tot U,~\sep\ ten aanschouwen der mensen.

  Gij beschermt hen onder de schutse van Uw aanschijn,~\sep\ tegen het samenzweren der mannen,

  Gij verbergt hen in Uw tent,~\sep\ tegen het schelden der tongen.

  Gezegend de Heer, want Hij bewees mij~\sep\ Zijn wondere barmhartigheid in de versterkte stad.

  Wel sprak ik in mijn onrust:~\sep\ ``Ik ben van Uw aanschijn verstoten.''

  Maar Gij hebt de stem van mijn smeken gehoord,~\sep\ daar ik tot U riep.

  Bemint de Heer, gij, al Zijn heiligen:~\sep\ de getrouwen behoedt de Heer,

  Maar overvloedig vergeldt Hij,~\sep\ die handelen in trots.

  Houdt moed, en weest sterk van hart,~\sep\ gij allen, die hoopt op de Heer.

  \psalm{\Ps{32}} Gelukkig hij, wiens misdaad vergeven,~\sep\ wiens zonde bedekt is.

  Gelukkig de mens, wie de Heer zijn schuld niet toerekent,~\sep\ en in wiens geest geen bedrog is.

  Zolang ik bleef zwijgen, werd mijn gebeente verteerd,~\sep\ onder mijn aanhoudend gezucht.

  Want dag en nacht drukte Uw hand op mij;~\sep\ mijn kracht teerde weg als bij zomerse hitte.

  Mijn zonde heb ik beleden voor U,~\sep\ en mijn schuld hield ik niet verborgen;

  Ik sprak: ``Ik belijd mijn boosheid voor de Heer,''~\sep\ en Gij hebt de schuld van mijn zonde vergeven.

  Daarom moet iedere vrome bidden tot U,~\sep\ in tijden van nood.

  En al breekt dan de watervloed los,~\sep\ hem zal hij niet genaken.

  Gij zijt mijn Toeverlaat, Gij zult voor nood mij behoeden,~\sep\ en mij omgeven met vreugde over mijn redding.

  Ik zal u leren, en de weg wijzen, die gij moet gaan,~\sep\ u onderrichten en vast Mijn ogen richten op u.

  Weest niet als paard en muilezel zonder verstand, wier onstuimigheid men bedwingt met toom en gebit,~\sep\ anders komen ze niet naar u toe.

  Veel smart valt de boze ten deel,~\sep\ maar erbarming omgeeft wie vertrouwt op de Heer.

  Verheugt u in de Heer, weest blijde, gij, rechtvaardigen,~\sep\ en jubelt, gij allen, die oprecht zijt van harte.
\end{halfparskip}

\markedsubsectionrubricwithhint{Woensdagen ``voor'': Marmita 17*}{(origineel: hulala 7)}

\begin{halfparskip}
  \psalm{\Ps{44}} Met eigen oren, O God, hebben wij het gehoord,~\sep\ onze vaderen hebben het ons verhaald:

  \liturgicalhint{Alleluia, Alleluia, Alleluia.~--- Eerste vers.}

  Het werk, dat Gij gewrocht hebt in hun dagen,~\sep\ in overoude tijden.

  Met eigen hand hebt Gij de heidenen verdreven, maar hen geplant;~\sep\ om hen te doen bloeien hebt Gij volken verslagen.

  Neen, hun zwaard was het niet, dat het land heeft veroverd,~\sep\ noch hun arm, die hun redding bracht.

  Maar Uw rechter en Uw arm,~\sep\ en het licht van Uw gelaat, want Gij hadt hen lief.

  Gij zijt mijn Koning, mijn God,~\sep\ die de zege verleende aan Jacob.

  Door U hebben wij onze tegenstanders verdreven,~\sep\ en in Uw Naam vertrapten wij die tegen ons waren opgestaan.

  Neen, ik heb niet vertrouwd op mijn boog,~\sep\ noch was het mijn zwaard, dat mij redding bracht.

  Maar Gij hebt ons verlost van onze weerstrevers,~\sep\ en die ons haten, hebt Gij beschaamd.

  In God roemden wij ten allen tijde,~\sep\ en Uw Naam prezen wij immer.

  Maar nu hebt Gij ons verstoten en te schande gemaakt,~\sep\ en Gij trekt niet meer op, God, met onze legerscharen.

  Gij hebt ons doen wijken voor onze tegenstanders;~\sep\ die ons haten, hebben ons uitgeplunderd.

  Als slachtschapen hebt Gij ons overgeleverd,~\sep\ en onder de volken verstrooid.

  Voor een spotprijs hebt Gij Uw volk verkocht,~\sep\ en weinig winst heeft U de verkoop gebracht.

  Gij hebt ons te schande gemaakt voor onze buren,~\sep\ tot spot en hoon voor hen, die ons omringen.

  Gij hebt ons tot spreekwoord gemaakt onder de heidenen;~\sep\ de volken schudden het hoofd over ons.

  Mijn schande staat mij immer voor ogen,~\sep\ en schaamte bedekt mijn gelaat,

  Om de praatjes van schimper en spotter,~\sep\ om vijand en tegenstander.

  Dit alles kwam over ons; en toch: wij zijn U niet vergeten;~\sep\ en hebben Uw verbond niet geschonden;

  Ook viel ons hart niet van U af,~\sep\ en onze schreden weken niet af van Uw wegen,

  Toen Gij ons gebroken hebt in het oord der kwelling,~\sep\ en ons met duisternis hebt omhuld.

  Hadden wij de Naam van onze God vergeten,~\sep\ en naar een vreemde god onze handen uitgestrekt,

  Zou God dat niet hebben ontdekt?~\sep\ Hij toch doorschouwt de geheimen van het hart.

  Ja, om Uwentwil blijft men ons doden,~\sep\ worden wij als slachtschapen beschouwd.

  Ontwaak dan, Heer, wat slaapt Gij?~\sep\ Waak op, blijf ons niet eeuwig verstoten!

  Waarom verbergt Gij Uw aanschijn,~\sep\ vergeet Gij onze ellende en onze verdrukking?

  Want onze ziel ligt neer in het stof,~\sep\ ons lichaam kleeft aan de aarde.

  Sta op om ons te helpen,~\sep\ en bevrijd ons om Uw barmhartigheid.

  \psalm{\Ps{45}} Een heerlijk lied welt op uit mijn hart: de Koning wijd ik mijn zang;~\sep\ mijn tong is de stift van een vaardige schrijver.

  Gij zijt de schoonste onder de kinderen der mensen; bevalligheid ligt op uw lippen:~\sep\ daarom heeft God u voor eeuwig gezegend.

  Gord uw zwaard om de heup, gij, machtige held,~\sep\ uw sieraad en luister.

  Ruk zegerijk uit voor waarheid en recht;~\sep\ uw rechterhand lere u roemrijke daden.

  Uw pijlen zijn scherp; volkeren worden aan u onderworpen;~\sep\ aan de vijanden van de Koning ontzinkt de moed.

  In de eeuwen der eeuwen staat Uw troon, O God,~\sep\ een scepter van recht is de scepter van Uw rijk.

  Gij hebt de gerechtigheid lief en haat de boosheid; daarom heeft God, uw God, u gezalfd,~\sep\ met de olie der vreugde boven uw genoten.

  Van mirre en aloë en cassia geuren uw gewaden;~\sep\ uit ivoren paleizen klinkt u blij het harpgeluid tegen.

  Koningsdochters treden u tegemoet,~\sep\ de Koningin staat aan uw rechterhand, met goud uit Ofir getooid.

  Hoor, dochter, en zie, en neig uw oor,~\sep\ en vergeet uw volk en het huis van uw vader.

  Dan zal aan de Koning uw schoonheid behagen;~\sep\ Hij is uw Heer, breng Hem uw hulde.

  Dan komt met geschenken het volk van Tyrus,~\sep\ de voornamen onder het volk dingen om uw gunst.

  In volle luister treedt de dochter van de Koning binnen;~\sep\ met goud doorweven is haar gewaad.

  In een kleurige mantel wordt zij voor de Koning geleid,~\sep\ in haar gevolg worden maagden, haar
  gezellinnen, tot u gevoerd;

  Zij worden voorgeleid in blijde jubel,~\sep\ en treden het paleis van de Koning binnen.

  In de plaats van uw vaderen komen uw zonen:~\sep\ gij zult hen aanstellen tot vorsten over heel de wereld.

  Ik zal uw naam doen gedenken bij alle geslachten;~\sep\ daarom zullen de volken u prijzen in de eeuwen der eeuwen.

  \psalm{\Ps{46}} God is ons een Toevlucht en Kracht;~\sep\ een machtige Helper toonde Hij zich in de nood.

  Daarom vrezen wij niet, al wordt ook de aarde geschokt,~\sep\ al storten de bergen midden in zee.

  Laat bruisen en koken haar wateren,~\sep\ laat schudden de bergen door haar geweld:

  De Heer der heerscharen is met ons,~\sep\ de God van Jacob is onze bescherming.

  De armen van de stroom verblijden de stad van God,~\sep\ de heilige woontent van de Allerhoogste.

  God woont daarbinnen, zij zal niet wankelen;~\sep\ God staat haar bij van de vroege dageraad af.

  Volkeren woedden, en koninkrijken werden geschokt:~\sep\ daar galmde Zijn donderstem, en weg vloeide de aarde.

  De Heer der heerscharen is met ons,~\sep\ de God van Jacob is onze bescherming.

  Komt en aanschouwt de werken van de Heer,~\sep\ de wonderen, die Hij op aarde gewrocht heeft.

  Hij bedwingt de oorlogen tot aan het einde der aarde,~\sep\ Hij verbrijzelt de bogen, breekt lansen stuk, in het vuur verbrandt Hij de schilden.

  Houdt op, en erkent Mij als God,~\sep\ verheven onder de volken, verheven op aarde.

  De Heer der heerscharen is met ons,~\sep\ de God van Jacob is onze bescherming.
\end{halfparskip}

\markedsubsectionrubricwithhint{Donderdagen ``voor'': Marmita 25*}{(origineel: hulala 10)}

\begin{halfparskip}
  \psalm{\Ps{68}} God rijst op: Zijn vijanden stuiven uiteen,~\sep\ en die Hem haten, vluchten weg voor Zijn aanschijn.

  \liturgicalhint{Alleluia, Alleluia, Alleluia.~--- Eerste vers.}

  Zij verdwijnen, zoals rook verdwijnt;~\sep\ zoals was wegsmelt bij vuur, zo vergaan de zondaars voor Gods
  aanschijn.

  Maar de rechtvaardigen juichen, springen op voor het aanschijn van God,~\sep\ zijn opgetogen van vreugde.

  Zingt God toe, tokkelt het psalter voor Zijn Naam;~\sep\ baant een weg voor Hem, die voorttrekt door de woestijn,

  Wiens naam is: de Heer,~\sep\ en juicht voor Zijn aanschijn!

  Vader der wezen en Beschermer der weduwen,~\sep\ is God in Zijn heilige woonstede.

  Voor verlatenen bereidt God een woning, gevangenen brengt Hij tot voorspoed;~\sep\ slechts de weerspannigen blijven achter in het verdorde land.

  Bij Uw uittocht, O God, aan het hoofd van Uw volk,~\sep\ bij Uw optrekken door de woestijn,

  Beefde de aarde en dropen de hemelen voor het aanschijn van God;~\sep\ de Sinaï sidderde voor God, de God van Israël.

  Milde regen hebt Gij, O God, over Uw erfdeel uitgestort;~\sep\ en als het was uitgeput, hebt Gij het verkwikt.

  Uw kudde heeft er gewoond;~\sep\ in Uw goedheid, O God, hebt Gij het de arme bereid.

  De Heer doet een uitspraak:~\sep\ en groot is de menigte, die blijde dingen meldt:

  ``De vorsten der legerscharen vluchten, vluchten:~\sep\ en de huisgenoten verdelen de buit.

  Terwijl gij rustte in de stallen der kudden, schitterden van zilver de vleugels der duif,~\sep\ en haar pennen in goudgele glans.

  Terwijl er de Almachtige de vorsten verstrooide,~\sep\ viel er sneeuw op de Salmon.''

  Hoge bergen zijn de bergen van Basan,~\sep\ steile bergen zijn de bergen van Basan.

  Waarom ziet gij afgunstig, gij, steile bergen, naar de berg, waar het God behaagd heeft te wonen,~\sep\ ja, waar de Heer zelfs voor immer zal wonen?

  De strijdwagens van God zijn myriaden in aantal, duizendmaal duizend;~\sep\ van de Sinaï af trekt de Heer naar Zijn heiligdom.

  Gij hebt de hoogte bestegen, gevangenen meegevoerd, mensen als gaven ontvangen,~\sep\ zelfs hen, die weigeren bij God de Heer te wonen.

  Geprezen zij de Heer, dag aan dag;~\sep\ onze lasten draagt God, ons heil!

  Onze God is een God, die redding brengt,~\sep\ en de Heer God schenkt uitkomst in doodsgevaar.

  Waarlijk, God verbrijzelt de hoofden van Zijn vijanden,~\sep\ de ruige schedel van hem, die de weg der misdaad bewandelt.

  De Heer heeft gesproken: ``Ik haal ze terug uit Basan,~\sep\ ik haal ze op uit de diepte der zee,

  Opdat ge uw voet moogt dopen in bloed,~\sep\ en de tong van uw honden haar deel aan de vijanden heeft!''

  Zij zien Uw intrede, O God,~\sep\ de intrede van mijn God, mijn Koning, in het heiligdom:

  Voorop gaan de zangers, de harpspelers aan het einde,~\sep\ in het midden slaan de meisjes de pauken.

  ``Looft God in uw feestelijke bijeenkomsten,~\sep\ de Heer, gij, die uit Israël geboren zijt!''

  Daar treedt Benjamin aan, de jongste; hij gaat voor hen uit; dan volgen de vorsten van Juda met hun scharen,~\sep\ de vorsten van Zabulon, de vorsten van Neftali.

  Ontplooi Uw macht, O God,~\sep\ Uw macht, O God, Gij, die werkt voor ons!

  Omwille van Uw tempel, die in Jeruzalem staat,~\sep\ mogen de vorsten U gaven brengen!

  Bedwing het wilde beest in het riet,~\sep\ de troep stieren met de runderen der volken;

  Dat zij zich ter aarde werpen met platen van zilver:~\sep\ verstrooi de volken, die zich verlustigen in oorlog.

  Laat de machtigen uit Egypte zich aandienen,~\sep\ dat Ethiopië de handen uitstrekke naar God.

  Koninkrijken der aarde, zingt voor God, bespeelt het psalter voor de Heer,~\sep\ die door de hemelen, de aloude hemelen rijdt!

  Zie, Zijn stem laat Hij horen, Zijn machtige stem:~\sep\ ``Erkent de macht van God'';

  Zijn Majesteit strekt zich uit over Israël,~\sep\ en in de wolken straalt Zijn macht.

  Ontzagwekkend is God in Zijn heiligdom, Hij, de God van Israël. Hij zelf schenkt macht en kracht aan Zijn volk.~\sep\ Geprezen zij God!
\end{halfparskip}

\markedsubsectionrubricwithhint{Vrijdagen ``voor'': Marmita 41*}{(origineel: hulala 16)}

\begin{halfparskip}
  \psalm{\Ps{104}} Loof, mijn ziel, de Heer!~\sep\ Heer, mijn God, Gij zijt ontzaglijk groot;

  \liturgicalhint{Alleluia, Alleluia, Alleluia.~--- Eerste vers.}

  Met majesteit en luister zijt Gij omkleed,~\sep\ met licht omhuld als met een mantel.

  Als een tentdoek hebt Gij de hemelen uitgespreid,~\sep\ op de wateren Uw opperzalen gebouwd.

  Gij maakt de wolken tot Uw wagen,~\sep\ en zweeft op de wieken der winden.

  Tot Uw boden maakt Gij de winden,~\sep\ tot Uw knechten het verzengende vuur.

  Gij hebt de aarde op haar pijlers gegrondvest;~\sep\ in de eeuwen der eeuwen zal zij niet wankelen.

  Met de oceaan als kleed hebt Gij haar bedekt;~\sep\ boven de bergen stonden de wateren.

  Bij Uw dreigen vluchtten zij heen,~\sep\ ze sidderden voor de stem van Uw donder.

  Bergen rezen en dalen zonken,~\sep\ op de plaats, die Gij hun hadt bestemd.

  Gij hebt een grens gesteld, die zij niet overschrijden mogen,~\sep\ opdat ze niet opnieuw de aarde bedekken.

  Gij beveelt de bronnen in de beken te vloeien,~\sep\ die tussen de bergen stromen;

  Zij laven alle dieren van het veld;~\sep\ de woudezels lessen hun dorst.

  Daar nestelen de vogels van de hemel,~\sep\ en doen er hun lied in de takken weerklinken.

  Vanuit Uw zalen besproeit Gij de bergen,~\sep\ met de vrucht van Uw werken wordt de aarde verzadigd.

  Gij laat voor het vee het gras ontspruiten,~\sep\ en het kruid tot nut van de mens,

  Opdat hij uit de aarde brood zal verwekken,~\sep\ en wijn, die het hart van de mens verblijdt;

  Opdat hij met olie het gelaat zou doen glanzen,~\sep\ en met brood zou verkwikken het hart van de mens.

  De bomen van de Heer worden verzadigd,~\sep\ de ceders van de Libanon, die door Hem zijn geplant.

  Daar bouwen de vogels hun nest;~\sep\ tot verblijf van de ooievaar dienen de dennen.

  De hoogste bergen zijn voor de gemzen,~\sep\ de rotsen bieden de egels een schuilplaats.

  Gij hebt de maan geschapen om de tijd te bepalen,~\sep\ de zon kent het uur van haar ondergang.

  \emph{Wij aanbidden U,~\sep\ Vader, Zoon en Heilige Geest.}

  Maakt Gij het donker en wordt het nacht,~\sep\ dan zwerven erin alle dieren van het woud.

  De leeuwenwelpen brullen om prooi,~\sep\ en vragen aan God om voedsel.

  En rijst dan de zon, dan sluipen ze weg,~\sep\ en leggen zich neer in hun holen.

  De mens gaat uit naar zijn werk,~\sep\ naar zijn arbeid tot de avond toe.
  Hoe talrijk zijn Uw werken, O Heer! Gij hebt alles met wijsheid geschapen,~\sep\ van Uw schepselen is de aarde vervuld.

  Zie de zee, groot en wijd naar elke einder:~\sep\ daar wemelt het van talloze vissen, van dieren, klein en groot.

  Daar trekken de schepen doorheen,~\sep\ Leviathan, door U geschapen, om er te spelen.

  Allen zien naar U uit,~\sep\ dat Gij op tijd hun voedsel schenkt.

  Geeft Gij het hun: ze verzamelen,~\sep\ opent Gij Uw hand: ze worden met het goede verzadigd.

  Verbergt Gij Uw aanschijn: ze worden ontsteld;~\sep\ ontneemt Gij hun de adem: ze sterven en keren terug tot hun stof.

  Zendt Gij Uw geest uit: zij worden geschapen,~\sep\ en Gij vernieuwt het aanschijn der aarde.

  Eeuwig dure de glorie van de Heer;~\sep\ de Heer verheuge zich over Zijn werken.

  Hij blikt naar de aarde: zij beeft;~\sep\ Hij raakt de bergen aan: zij roken.

  Ik wil zingen voor de Heer, mijn leven lang,~\sep\ voor mijn God de snaren tokkelen, zolang ik besta.

  Moge mijn zang Hem behagen:~\sep\ ik zal in de Heer mij verblijden.

  Dat de zondaars van de aarde verdwijnen, en de goddelozen niet langer bestaan!~\sep\ Zegen, mijn ziel, de Heer!
\end{halfparskip}

\markedsubsectionrubricwithhint{Zaterdagen ``voor'': Marmita 47*}{(origineel: hulala 19)}

\begin{halfparskip}
  \psalm{\Ps{115}} De Heer heb ik lief, omdat Hij hoorde,~\sep\ naar de stem van mijn smeken,

  Omdat Hij Zijn oor neigde tot mij,~\sep\ op welke dag ik ook tot Hem riep.

  \liturgicalhint{Alleluia, Alleluia, Alleluia.~--- Eerste vers.}

  Mij omstrengelden strikken van de dood en boeien van het dodenrijk omknelden mij;~\sep\ kommer en smart werden mijn deel.

  Maar de Naam van de Heer riep ik aan:~\sep\ ``Ach, Heer, red toch mijn leven !''

  De Heer is goed en rechtvaardig,~\sep\ ja, meedogend is onze God.

  De Heer behoedt de eenvoudigen:~\sep\ ik was ellendig en Hij heeft mij gered.

  Keer terug, mijn ziel, tot Uw rust,~\sep\ want de Heer heeft u welgedaan.

  Want Hij heeft mij ontrukt aan de dood,~\sep\ mijn ogen behoed voor tranen, mijn voeten voor de val.

  Ik zal wandelen voor de Heer,~\sep\ in het land der levenden.

  Ik bleef vertrouwen, ook toen ik sprak:~\sep\ ``Ik ben diep bedroefd.''

  In mijn verslagenheid heb ik gezegd:~\sep\ ``Er is geen mens te vertrouwen.''

  Wat zal ik de Heer weergeven,~\sep\ voor alles, wat Hij mij schonk?

  De kelk van het heil zal ik heffen,~\sep\ en aanroepen de Naam van de Heer.

  Ik zal de Heer mijn geloften volbrengen,~\sep\ in het bijzijn van geheel Zijn volk.

  Kostbaar in de ogen van de Heer,~\sep\ is de dood van Zijn heiligen.

  Ik ben Uw dienstknecht, Heer, Uw dienstknecht, een zoon van Uw dienstmaagd,~\sep\ Gij hebt mijn boeien verbroken.

  Een dankoffer zal ik U brengen,~\sep\ en aanroepen de Naam van de Heer.

  Ik zal de Heer mijn geloften volbrengen,~\sep\ in het bijzijn van geheel Zijn volk,

  In de voorhoven van het huis van de Heer,~\sep\ in uw midden, Jeruzalem.

  \psalm{\Ps{116}} Looft de Heer, alle volkeren,~\sep\ alle naties, verheerlijkt Hem.

  Want Zijn erbarming blijft ons verzekerd,~\sep\ en de trouw van de Heer duurt eeuwig.

  \psalm{\Ps{117}} Dankt de Heer, want Hij is goed,~\sep\ want eeuwig duurt Zijn barmhartigheid.

  Het huis van Israël zegge:~\sep\ ``Eeuwig duurt Zijn barmhartigheid.''

  Het huis van Aäron zegge:~\sep\ ``Eeuwig duurt Zijn barmhartigheid?''

  Die de Heer vrezen, zeggen:~\sep\ ``Eeuwig duurt Zijn barmhartigheid.''

  In kwelling riep ik tot de Heer;~\sep\ de Heer heeft mij verhoord en bevrijd.

  De Heer is met mij: ik ben niet bevreesd;~\sep\ wat kan een mens mij aandoen?

  De Heer is met mij, Hij is mijn Helper;~\sep\ ik zal mijn vijanden beschaamd zien staan.

  Het is beter tot de Heer te vluchten,~\sep\ dan op een mens te vertrouwen;

  Het is beter tot de Heer te vluchten,~\sep\ dan te vertrouwen op vorsten.

  Alle volken omsingelden mij;~\sep\ in de Naam van de Heer sloeg ik ze neer.

  Van alle zijden omringden zij mij;~\sep\ in de Naam van de Heer sloeg ik ze neer.

  Ze zwermden als bijen om mij heen, zij brandden mij weg als het vuur de doornen;~\sep\ in de Naam van de Heer sloeg ik ze neer.

  Zij stieten mij hevig, opdat ik zou vallen,~\sep\ maar de Heer kwam mij te hulp.

  Mijn kracht en mijn sterkte is de Heer,~\sep\ Hij is mij tot Redder geworden.

  Er klinkt een juich- en zegekreet~\sep\ in de tenten der rechtvaardigen.

  De rechterhand van de Heer ontplooide haar kracht, de rechterhand van de Heer hief mij op,~\sep\ de rechterhand van de Heer ontplooide haar kracht.

  Niet sterven zal ik, maar leven,~\sep\ en de werken van de Heer verkondigen.

  Wel heeft de Heer mij streng gekastijd,~\sep\ maar Hij gaf mij niet prijs aan de dood.

  Ontsluit mij de poorten der gerechtigheid:~\sep\ door hen wil ik ingaan om de Heer te gaan danken.

  Dit is de poort van de Heer,~\sep\ de rechtvaardigen zullen er door binnentreden.

  Ik zal U danken, daar Gij mij verhoord hebt,~\sep\ en mij tot Redder geworden zijt.

  De steen, die de bouwlieden hebben verworpen,~\sep\ is tot een hoeksteen geworden.

  Dit heeft de Heer gedaan,~\sep\ het is wonderbaar in onze ogen.

  Dit is de dag, die de Heer heeft gemaakt:~\sep\ laat ons daarover juichen en blijde zijn.

  Schenk redding, O Heer;~\sep\ O Heer, geef voorspoed!

  Gezegend die komt in de Naam van de Heer, wij zegenen u uit het huis van de Heer.~\sep\ God is de Heer, Hij heeft ons verlicht.

  Spreidt praal ten toon met weelderig lover,~\sep\ tot bij de hoornen van het altaar.

  Mijn God zijt Gij, U breng ik dank;~\sep\ mijn God, hoog wil ik U prijzen.

  Dankt de Heer, want Hij is goed,~\sep\ eeuwig duurt Zijn barmhartigheid.
\end{halfparskip}

\markedsubsectionrubricwithhint{Maandagen ``na'': Marmita 6*}{(origineel: hulala 1)}

\begin{halfparskip}
  \psalm{\Ps{18}} Ik heb U lief, o Heer, mijn Sterkte,~\sep\ Heer, mijn Rots, mijn Burcht, mijn Bevrijder;

  \liturgicalhint{Alleluia, Alleluia, Alleluia.~--- Eerste vers.}

  Mijn God, mijn Rotswand, waarheen ik vlucht,~\sep\ mijn Schild, de Hoorn van mijn heil, mijn Toeverlaat.

  Aanroepen zal ik de Heer, de Lofwaardige,~\sep\ en van mijn vijanden worden verlost.

  Mij omspoelden de golven van de dood,~\sep\ en vernietigende stromen ontstelden mij.

  De strikken van het dodenrijk omknelden mij,~\sep\ de boeien van de dood vielen op mij neer.

  In mijn nood riep ik tot de Heer,~\sep\ en mijn geschrei steeg op tot mijn God;

  En Hij hoorde mijn stem vanuit Zijn tempel,~\sep\ en mijn hulpgeroep drong door tot Zijn oren.

  Daar schudde de aarde en beefde; de grondvesten der bergen werden geschokt,~\sep\ en zij dreunden want Hij brandde van toorn.

  Rook steeg uit Zijn neusgaten op, verslindend vuur uit Zijn mond,~\sep\ gloeiende kolen sprongen van Hem uit.

  Hij haalde de wolkenhemel neer en daalde af,~\sep\ en zwarte wolken hingen onder Zijn voeten.

  Hij voer op de cherub en vloog,~\sep\ op de wieken van de wind werd Hij gedragen.

  Hij omhulde zich met duisternis als met een kleed,~\sep\ met donkere nevels en dichte wolken als met een mantel.

  Door de gloed vóór Hem uit,~\sep\ ontbrandden gloeiende kolen.

  En de Heer deed de donder rollen uit de hemel,~\sep\ en weergalmen deed de Allerhoogste Zijn stem.

  En Hij schoot Zijn pijlen af en dreef hen uiteen,~\sep\ talloze flitsen, en Hij velde hen neer.

  En de bodem der zee kwam te voorschijn,~\sep\ en het fundament der aarde lag bloot.

  Door het dreigen van de Heer,~\sep\ door de ademtocht van Zijn toorn.

  Hij strekte Zijn hand uit de hoge, Hij greep mij aan,~\sep\ en trok mij op uit de watervloed.

  Hij bevrijdde mij van mijn geweldige vijand,~\sep\ en van hen, die mij haatten, die machtiger waren dan ik.

  Zij overvielen mij op de dag van mijn rampspoed,~\sep\ maar tot bescherming was mij de Heer.

  En Hij leidde mij uit in het vrije veld;~\sep\ Hij heeft mij gered, omdat Hij mij liefheeft.

  Zo loonde mij de Heer naar mijn gerechtigheid;~\sep\ naar de reinheid van mijn handen vergold Hij mij.

  Want de wegen van de Heer heb ik gevolgd,~\sep\ door geen zonde ben ik afgeweken van mijn God.

  Ja, al Zijn geboden hield ik voor ogen,~\sep\ en Zijn wetten wierp ik niet van mij af.

  Maar voor Zijn aanschijn was ik rein,~\sep\ en ik heb mij behoed voor de zonde.

  Zo vergold mij de Heer naar mijn gerechtigheid,~\sep\ naar de reinheid van mijn handen voor Zijn ogen.

  Met de vrome handelt Gij liefdevol,~\sep\ met de rechtschapene rechtschapen;

  Voor de reine toont Gij U rein,~\sep\ met de sluwe handelt Gij slim.

  Want Gij redt het nederige volk,~\sep\ maar trotse blikken slaat Gij neer.

  Ja, Gij doet mijn lamp schijnen, O Heer;~\sep\ mijn God, mijn duisternis maakt Gij tot licht.

  Ja, met U storm ik los op de drommen der vijanden,~\sep\ en met mijn God bespring ik de wallen.

  Gods wegen zijn volmaakt, het woord van de Heer is door het vuur gelouterd;~\sep\ Hij is een schild voor allen, die vluchten tot Hem.

  Wie is God buiten de Heer,~\sep\ of wie een rots buiten onze God?

  God, die mij met kracht heeft omgord,~\sep\ en mij een veilige weg heeft gebaand;

  Die aan mijn voeten de snelheid der hinden gaf,~\sep\ en mij plaatste op de hoogten,

  Die mijn handen oefende tot de strijd,~\sep\ en tot het spannen van de koperen boog mijn armen.

  Gij schonkt mij Uw schild, dat redding brengt, en Uw rechterhand heeft mij staande gehouden,~\sep\ en Uw zorgzame liefde maakte mij groot.

  Gij hebt de weg voor mijn schreden verbreed,~\sep\ en mijn voeten wankelden niet.

  Ik zette mijn vijanden na, en greep ze aan,~\sep\ en ik keerde niet terug, eer ik ze had vernietigd.

  Ik heb ze verpletterd en opstaan konden ze niet,~\sep\ ze bleven liggen onder mijn voeten.

  Ja, Gij hebt mij met kracht omgord tot de strijd;~\sep\ en die mij weerstaan, hebt Gij voor mij doen bukken.

  Gij hebt mijn vijanden op de vlucht gedreven,~\sep\ en die mij haten, hebt Gij verdelgd.

  Zij schreeuwden het uit - maar niemand schonk redding -~\sep\ tot de Heer, maar Hij verhoorde hen niet.

  En ik heb ze vergruisd als stof voor de wind,~\sep\ vertrapt als slijk in de straten.

  Gij hebt mij ontrukt aan het muitende volk,~\sep\ mij gesteld aan het hoofd van de naties.

  Een volk, dat mij vreemd was, werd mij dienstbaar;~\sep\ nauwelijks hoorde het van mij, of het was mij onderdanig.

  Vreemden brachten mij vleiend hulde,~\sep\ vreemden, geslagen met schrik, kropen sidderend uit hun burchten.

  Leve de Heer, mijn Rots zij gezegend;~\sep\ hooggeprezen zij God, mijn Redder!

  God, die mij de wraak in handen gaf,~\sep\ en mij de volkeren onderwierp,

  Gij, die mij van mijn vijanden hebt bevrijd, en mij verheven hebt boven mijn weerstrevers,~\sep\ mij hebt ontrukt aan de geweldenaar.

  Daarom zal ik U prijzen onder de volken, O Heer,~\sep\ en verheerlijken Uw Naam.

  Gij hebt Uw koning een schitterende zege verleend,~\sep\ en barmhartigheid bewezen aan Uw gezalfde, aan David en zijn geslacht voor eeuwig.
\end{halfparskip}

\markedsubsectionrubricwithhint{Dinsdagen ``na'': Marmita 14*}{(origineel: hulala 4)}

\begin{halfparskip}
  \psalm{\Ps{37}} Ontbrand niet in toorn vanwege de zondaars,~\sep\ en benijd de boosdoeners niet.

  \liturgicalhint{Alleluia, Alleluia, Alleluia.~--- Eerste vers.}

  Spoedig toch vallen zij neer als hooi,~\sep\ en verwelken als het groene gras.

  Vertrouw op de Heer, en doe het goede,~\sep\ om het Land te bewonen en een veilig bestaan te genieten.

  Stel uw vreugde in de Heer,~\sep\ en Hij zal u schenken wat uw hart maar begeert.

  Vertrouw de Heer uw levensweg toe,~\sep\ en hoop op Hem, Hij zal wel zorgen.

  Hij zal als het licht uw gerechtigheid doen opgaan,~\sep\ en als de middagzon uw recht.

  Verlaat u op de Heer,~\sep\ en stel uw hoop op Hem.

  Vertoorn u niet op hem, die het wel gaat in het leven,~\sep\ op de mens, die het kwade beraamt.

  Leg uw verbolgenheid af en laat varen uw gramschap,~\sep\ ontbrand niet in toorn om geen kwaad te bedrijven.

  Immers: de boosdoeners worden te gronde gericht,~\sep\ maar die hopen op de Heer, zullen het land bezitten.

  Nog een weinig tijds, en weg is de boze;~\sep\ en zoekt ge zijn plaats, hij is er niet meer.

  Maar de zachtmoedigen zullen het Land bezitten,~\sep\ en een overvloedige vrede genieten.

  De goddeloze zint op onheil tegen de rechtvaardige,~\sep\ en knarst tegen hem met de tanden;

  De Heer spot met hem,~\sep\ omdat Hij zijn dag ziet naderen.

  De bozen trekken het zwaard en spannen de boog, om de ellendige en arme neer te vellen,~\sep\ te doden die gaan langs de rechte weg.

  Maar hun zwaard zal hun eigen hart doorboren,~\sep\ en hun bogen zullen worden gebroken.

  Beter het schamele, dat de gerechte bezit,~\sep\ dan de grote rijkdom der bozen.

  Want de armen der bozen zullen worden gebroken,~\sep\ maar de Heer is een steun voor de rechtvaardigen.

  De Heer draagt zorg voor het leven der vromen,~\sep\ en hun erfdeel blijft eeuwig bestaan.

  Zij zullen bij rampspoed niet worden beschaamd,~\sep\ maar verzadigd worden bij hongersnood.

  Maar de goddelozen zullen vergaan, en de vijanden van de Heer als de tooi der weiden verwelken:~\sep\ ze zullen vervliegen als rook.

  De goddeloze leent en geeft niet terug,~\sep\ maar de gerechte is genadig en geeft.

  Want die Hij zegent, zullen het Land bezitten,~\sep\ maar die Hij vloekt, zullen vergaan.

  Door de Heer worden de schreden van de mens ondersteund;~\sep\ en in zijn wandel schept Hij behagen.

  Mocht hij vallen, hij valt niet languit,~\sep\ want de Heer houdt hem vast bij de hand,

  Eens was ik een kind en nu ben ik een grijsaard, maar nooit heb ik een rechtvaardige verlaten gezien,~\sep\ noch zag ik zijn kinderen bedelen om brood.

  Steeds is hij meedogend en geeft hij te leen,~\sep\ en zijn nakroost zal gezegend worden.

  Houd u af van het kwaad en doe het goede,~\sep\ opdat gij voor eeuwig moogt leven.

  Want de Heer heeft de gerechtigheid lief,~\sep\ en Hij verlaat Zijn heiligen niet.

  De bozen worden vernietigd,~\sep\ het geslacht der goddelozen verdelgd.

  De rechtvaardigen zullen het Land bezitten,~\sep\ en zullen daar wonen voor immer.

  De mond van de rechtvaardige spreekt wijsheid,~\sep\ en wat recht is, verkondigt zijn tong.

  Hij draagt de Wet van zijn God in zijn hart,~\sep\ en zijn schreden wankelen niet.

  De goddeloze bespiedt de rechtvaardige,~\sep\ en zoekt hem te doden.

  De Heer laat hem niet in zijn macht,~\sep\ en veroordeelt hem niet in het gericht.

  Vertrouw op de Heer,~\sep\ en bewandel Zijn weg;

  Dan helpt Hij u voort om het Land te bezitten,~\sep\ en gij zult vol vreugde de verdelging der bozen aanschouwen.

  Ik heb de goddeloze gezien in zijn trots:~\sep\ hij breidde zich uit als een bladerrijke ceder.

  En ik ging voorbij, doch zie, hij was er niet meer,~\sep\ ik zocht naar hem, maar hij was niet te vinden.

  Let op de vrome en beschouw de rechtvaardige,~\sep\ want een vreedzaam man heeft een nageslacht.

  Maar de zondaars gaan allen te gronde,~\sep\ het nakroost der goddelozen zal worden verdelgd.

  Het heil der rechtvaardigen komt van de Heer;~\sep\ hun Toevlucht is Hij ten tijde van rampspoed.

  De Heer is hun Helper, Hij schenkt hun bevrijding,~\sep\ Hij bevrijdt hen van bozen en brengt hun de redding, daar zij hun toevlucht nemen tot Hem.
\end{halfparskip}

\markedsubsectionrubricwithhint{Woensdagen ``na'': Marmita 20*}{(origineel: hulala 7)}

\begin{halfparskip}
  \psalm{\Ps{53}} De dwaas zegt bij zich zelf:~\sep\ ``Er is geen God.''~\sep\ Ze zijn bedorven, gruwelen hebben ze bedreven;~\sep

  \liturgicalhint{Alleluia, Alleluia, Alleluia.~--- Eerste vers.}

  daar is er niet één, die deugdzaam handelt.

  God blikt uit de hemel neer op de kinderen der mensen,~\sep\ om te zien of er wel één is met verstand, wel één, die God zoekt.

  Maar allen zonder uitzondering zijn ze afgedwaald, allen diep bedorven;~\sep\ er is er niet één die deugdzaam handelt, niet één.

  Zullen die bozen dan nimmer tot inzicht komen, zij die Mijn volk verslinden als aten ze brood;~\sep\ roepen zij dan God niet aan?

  Zij sidderden van angst,~\sep\ waar niets viel te vrezen;

  Want God heeft de beenderen verstrooid van die u bestookten;~\sep\ ze werden ontsteld, want God heeft hen verworpen.

  O, mocht er uit Sion toch heil voor Israël dagen! Als God het lot van Zijn volk ten goede keert,~\sep\ zal er gejubel zijn in Jacob en vreugde in Israël!

  \psalm{\Ps{54}} Red mij, o God, door Uw Naam,~\sep\ en treed in Uw kracht voor mijn rechtszaak op!

  Luister naar mijn bede, O God,~\sep\ hoor naar de woorden van mijn mond!

  Want trotsaards zijn tegen mij opgestaan, en geweldenaars stonden mij naar het leven;~\sep\ zij hielden God niet voor ogen.

  Zie, God komt mij te hulp,~\sep\ de Heer behoudt mijn leven.

  Wend op mijn vijanden de rampen af,~\sep\ vernietig hen omwille van Uw trouw.

  Van harte wil ik U offers brengen;~\sep\ Uw Naam zal ik prijzen, O Heer, want Hij is goed.

  Want Hij heeft mij verlost uit alle nood,~\sep\ en mijn oog zag mijn vijanden beschaamd staan

  \psalm{\Ps{55}} Luister, O God, naar mijn bede, en wend U niet af van mijn smeken;~\sep\ geef acht op mij en schenk mij verhoring.

  Ik sidder van angst en word ontsteld~\sep\ bij het razen van de vijand, en het geschreeuw van de zondaar.

  Want zij storten rampen over mij uit,~\sep\ en vallen mij grimmig aan.

  Mijn hart is ontsteld in mijn binnenste,~\sep\ en de angst van de dood overvalt mij.

  Vrees en siddering storten op mij neer,~\sep\ en ontzetting grijpt mij aan.

  En ik zeg: Had ik maar vleugelen als de duif,~\sep\ dan vloog ik heen en vond ik rust.

  Ja, ver, ver zou ik heenvluchten,~\sep\ verwijlen in de woestijn.

  Spoedig zou Ik mij een schuilplaats zoeken,~\sep\ tegen stormwind en orkaan.

  Verstrooi ze, Heer, verwar hun spraak,~\sep\ want in de stad zie ik geweld en twist.

  Dag en nacht doen zij de ronde op haar wallen,~\sep\ en daarbinnen heerst boosheid en verdrukking.

  Hinderlagen legt men er,~\sep\ en onrecht en bedrog wijken niet van haar straten.

  Och, was het slechts mijn vijand, die mij had gehoond,~\sep\ voorzeker, ik had het verdragen;

  Of was, die mij haatte, tegen mij opgestaan,~\sep\ ik had mij voor hem verborgen.

  Maar gij waart het, mijn disgenoot,~\sep\ mijn vriend en vertrouweling,

  Met wie ik zo gemoedelijk omging;~\sep\ in het huis van God trokken we samen met de feeststoet op.

  Dat de dood op hen aanstorme, dat zij levend neerdalen in het dodenrijk,~\sep\ want boosheid woont in hun huizen en in hun binnenste!

  Ik echter zal roepen tot God,~\sep\ en de Heer zal mij redden.

  's Avonds en 's morgens en 's middags zal ik weeklagen en jammeren,~\sep\ en mijn stem zal Hij horen.

  Mijn leven zal Hij in veiligheid brengen tegen hen, die mij bestoken:~\sep\ want velen staan tegen mij op.

  God zal het horen, en Die van eeuwigheid regeert, hen neerdrukken;~\sep\ want onverbeterlijk zijn ze en ze vrezen God niet.

  Ieder heft zijn hand op tegen zijn vrienden,~\sep\ en schendt zijn verbond.

  Gladder dan boter is hun gelaat,~\sep\ maar hun hart wil strijd;

  Zachter dan olie zijn hun woorden,~\sep\ maar het zijn getrokken zwaarden.

  Werp Uw zorg op de Heer, en Hij zal Uw steun zijn:~\sep\ nooit zal Hij dulden dat de rechtvaardige wankelt.

  Maar hen, O God, zult Gij neerstorten,~\sep\ in de afgrond van verderf.

  Die bloed vergieten en bedriegen, zullen de helft van hun dagen niet halen;~\sep\ ik echter hoop op U, O Heer.
\end{halfparskip}

\markedsubsectionrubricwithhint{Donderdagen ``na'': Marmita 28*}{(origineel: hulala 10)}

\begin{halfparskip}
  \psalm{\Ps{73}} Hoe goed is Israëls God voor de rechtvaardigen,~\sep\ de Heer voor die rein zijn van hart!~\sep\ Toch wankelden bijna mijn voeten,~\sep

  \liturgicalhint{Alleluia, Alleluia, Alleluia.~--- Eerste vers.}

  haast gleden mijn schreden uit.

  Daar ik de goddelozen benijdde,~\sep\ toen ik de voorspoed der zondaren zag.

  Want kwellingen kennen zij niet,~\sep\ gezond en gezet is hun lichaam.

  De zorgen der stervelingen delen zij niet,~\sep\ en zij ontkomen de gesels der mensen.

  Daarom omsluit hen de trots als een halssnoer,~\sep\ en bedekt hen geweld als een kleed.

  De misdaad puilt uit hun zinnelijk hart,~\sep\ de verzinsels van hun geest dringen door naar buiten.

  Zij spotten en lasteren,~\sep\ zij dreigen op hoge toon met geweld,

  Zij zetten een mond op tegen de hemel,~\sep\ en hun tongen striemen de aarde.

  Daarom loopt mijn volk achter hen aan,~\sep\ en slurpen zij water in overvloed.

  En zij zeggen ``Hoe zou God het weten,~\sep\ en zou de Allerhoogste er kennis van dragen?''

  Zie, zo zijn de zondaars,~\sep\ en, steeds ongestoord, vermeerderen zij hun macht.

  Heb ik dan vergeefs mijn hart in reinheid bewaard,~\sep\ en mijn handen in onschuld gewassen?

  Want almaar door word ik gegeseld,~\sep\ en iedere dag gekastijd.

  Had ik gedacht: Laat mij spreken als zij,~\sep\ dan had ik de aard van Uw kinderen verloochend.

  Ik dacht dus na om het te vatten,~\sep\ maar het leek mij een moeilijke zaak,

  Totdat ik binnentrad in Gods heiligdom,~\sep\ en op hun einde ging letten.

  Waarlijk, Gij zet hen op een glibberig pad,~\sep\ stort hen neer in het verderf.

  Hoe zijn ze in een oogwenk ineengestort,~\sep\ verdwenen, in schrikkelijke angst vergaan!

  Als een droombeeld, O Heer, bij hem, die ontwaakt,~\sep\ zo zult Gij hun beeld, als Gij oprijst, versmaden.

  Toen mijn geest verbitterd was,~\sep\ en mijn hart werd geprikkeld,

  Was ik een dwaas zonder enig begrip,~\sep\ als een stuk vee voor Uw aanschijn.

  Maar ik zal immer bij U zijn:~\sep\ Gij houdt mij vast aan mijn rechterhand.

  Met Uw raad zult Gij mij leiden,~\sep\ en mij opnemen, eens, in de glorie.

  Wie bezit ik in de hemel buiten U;~\sep\ en ben ik bij U, dan geeft mij de aarde geen vreugde.

  Mijn lichaam bezwijkt en mijn hart,~\sep\ de Rots van mijn hart en mijn aandeel voor eeuwig is God.

  Want zie, die U verlaten, zullen vergaan;~\sep\ die U afvallig worden, verdelgt Gij allen.

  Maar mij is het goed bij God te zijn,~\sep\ mijn toevlucht te nemen bij God, de Heer.

  Al Uw werken zal ik verhalen,~\sep\ in de poorten der dochter Sion.

  \psalm{\Ps{74}} Waarom, God, hebt Gij ons voor eeuwig verstoten,~\sep\ ontbrandt Uw toorn tegen de schapen van Uw weide?

  Gedenk Uw volksgemeenschap, die Gij in oude tijden gesticht hebt, de stam, die Gij tot Uw bezit hebt vrijgekocht,~\sep\ de Sionsberg, waar Gij Uw zetel hebt gevestigd.

  Richt Uw schreden naar de eeuwige puinen:~\sep\ alles heeft de vijand in het heiligdom verwoest.

  Uw tegenstanders raasden op de plaats van Uw vergadering,~\sep\ en richtten er hun banieren als zegetekens op.

  Ze zijn als zij die met de bijl in het kreupelhout zwaaien;~\sep\ en zie, met houweel en hamer verbrijzelen zij tezamen zijn deuren.

  Aan het vuur hebben zij Uw heiligdom prijsgegeven,~\sep\ de woontent van Uw Naam tot de grond toe ontwijd.

  Zij spraken bij zichzelf: ``Laten wij hen allen tezamen verdelgen;~\sep\ verbrandt alle heiligdommen Gods in het land.''

  Onze tekenen zien wij reeds niet meer, er is geen profeet;~\sep\ en niemand onder ons weet hoe lang nog.

  Hoe lang nog, O God, zal de vijand smaden,~\sep\ de tegenstander Uw Naam maar immer lasteren?

  Waarom wendt Gij Uw hand van ons af,~\sep\ en houdt Gij Uw rechter terug in Uw schoot?

  God toch is van oudsher mijn Koning,~\sep\ die midden op de aarde redding brengt.

  Gij hebt door Uw macht de zee gescheiden,~\sep\ in de wateren de koppen der draken verpletterd.

  Gij hebt de koppen van Leviathan verbrijzeld,~\sep\ hem tot voedsel gegeven aan de monsters der zee.

  Gij liet bronnen en beken ontspringen,~\sep\ Gij hebt waterrijke stromen drooggelegd.

  Van U is de dag en van U is de nacht;~\sep\ maan en zon hebt Gij hun vaste plaats gegeven.

  Alle grenzen der aarde hebt Gij bepaald;~\sep\ Gij hebt zomer en winter geschapen.

  Herinner U dit: de vijand heeft U gehoond, O Heer,~\sep\ en een waanzinnig volk heeft Uw Naam gelasterd.

  Geef het leven van Uw tortel niet prijs aan de gier,~\sep\ vergeet het leven van Uw armen niet voor immer.

  Denk aan Uw verbond,~\sep\ want geweld heerst in de schuilhoeken van land en veld.

  Dat geen verdrukte vol schaamte heenga:~\sep\ dat de arme en behoeftige prijzen Uw Naam.

  Rijs op, O God, verdedig Uw zaak,~\sep\ gedenk de smaad, die de dwaze U aandoet dag aan dag.

  Vergeet het geraas van Uw vijanden niet;~\sep\ het geschreeuw van die opstaan tegen U stijgt immer omhoog.
\end{halfparskip}

\markedsubsectionrubricwithhint{Vrijdagen ``na'': Marmita 44*}{(origineel: hulala 16)}

\begin{halfparskip}
  \psalm{\Ps{107}} Looft de Heer, want Hij is goed, want eeuwig duurt Zijn barmhartigheid.~\sep

  Zo moeten nu spreken, die de Heer heeft verlost,~\sep

  \liturgicalhint{Alleluia, Alleluia, Alleluia.~--- Eerste vers.}

  die Hij redde uit de hand van de vijand,

  Die Hij bijeenbracht uit de landen,~\sep\ van oost en west, van noord en zuid.

  Daar doolden er in woestijn en wildernis rond,~\sep\ en vonden geen weg naar een bewoonbare stad.

  Zij werden gekweld door honger en dorst,~\sep\ hun leven verkwijnde in hen.

  En zij riepen tot de Heer in hun nood,~\sep\ en uit hun ellende verloste Hij hen.

  Hij leidde hen langs een rechte weg,~\sep\ om in een bewoonbare stad te komen.

  Dat zij de Heer om Zijn barmhartigheid danken,~\sep\ om Zijn wonderwerken voor de kinderen der mensen,

  Want wie was uitgehongerd, heeft Hij verzadigd,~\sep\ en de hongerige met het goede vervuld.

  Daar zaten er in donker en duister,~\sep\ in ellende, en in boeien geslagen,

  Want zij hadden Gods woorden weerstreefd,~\sep\ en het raadsbesluit van de Allerhoogste veracht.

  Toen heeft Hij hun hart door rampspoed gebroken;~\sep\ zij wankelden, maar niemand die hielp.

  En zij riepen tot de Heer in hun nood,~\sep\ en uit hun ellende bevrijdde Hij hen.

  Hij voerde hen weg uit donker en duister,~\sep\ en sloeg hun boeien aan stukken.

  Dat zij de Heer om Zijn barmhartigheid danken,~\sep\ om Zijn wonderwerken voor de kinderen der mensen.

  Daar Hij bronzen poorten stuk heeft geslagen,~\sep\ en ijzeren grendels verbrijzeld.

  Daar waren er ziek om hun zonden,~\sep\ in lijden om hun wangedrag;

  Alle voedsel was hun een walg,~\sep\ en zij naderden de poorten van de dood.

  En zij riepen tot de Heer in hun nood,~\sep\ en uit hun ellende bevrijdde Hij hen.

  Zijn woord zond Hij uit om hen te genezen,~\sep\ hen aan de dood te ontrukken.

  Dat zij de Heer om Zijn barmhartigheid danken,~\sep\ om Zijn wonderwerken voor de kinderen der mensen.

  Dat zij Hem dankoffers brengen,~\sep\ en jubelend Zijn werken verkondigen.

  Daar staken er op schepen in zee,~\sep\ om handel te drijven op de wijde wateren;

  Dezen hebben de werken van de Heer aanschouwd,~\sep\ Zijn wonderen op de hoge zee.

  Hij sprak: en Hij joeg de stormwind op,~\sep\ die zweepte haar golven omhoog.

  Tot de hemel sloegen ze op, in de diepten ploften ze neer;~\sep\ zij vergingen van angst in die rampen.

  Zij waggelden en knikten als waren zij dronken;~\sep\ en al hun kunde verdween.

  En zij riepen tot de Heer in hun nood,~\sep\ en uit hun ellende verloste Hij hen.

  Hij bedaarde de stormwind tot een bries,~\sep\ en de golven der zee legden zich neer.

  Zij waren blij dat het stil was geworden,~\sep\ en Hij hen voerde naar de verlangde haven.

  Dat zij de Heer om Zijn barmhartigheid danken,~\sep\ om Zijn wonderwerken voor de kinderen der mensen.

  Dat zij Hem roemen in de vergadering van het volk,~\sep\ in de raad van de oudsten Hem prijzen.

  Hij maakte rivieren tot een woestijn,~\sep\ en waterbronnen tot dorstige aarde,

  Tot zilte grond het vruchtbare land,~\sep\ om de boosheid van die er wonen.

  Hij herschiep de woestijn in een watervlakte,~\sep\ tot waterbronnen het dorre land.

  Daar gaf Hij een plaats aan de hongerigen,~\sep\ en zij stichtten een bewoonbare stad.

  Zij bezaaiden de akkers, legden wijngaarden aan,~\sep\ en verkregen een opbrengst aan vruchten.

  Hij zegende hen en zij groeiden sterk aan,~\sep\ en Hij schonk hun een talrijk vee.

  Toen slonk hun getal en ze werden verachtelijk,~\sep\ onder druk van rampen en kwelling.

  Maar Hij, die vorsten met smaad overstelpt,~\sep\ hen laat dwalen door ongebaande woestijnen,

  Hij richtte uit de ellende de behoeftige op,~\sep\ en maakte de gezinnen zo talrijk als kudden.

  De goeden zien het vol blijdschap,~\sep\ en al wat boos is, sluit zijn mond.

  Wie is er zo wijs en neemt dit ter harte,~\sep\ en overweegt naar behoren de barmhartigheden van de Heer?

  \psalm{\Ps{108}} Gesterkt is mijn hart, O God, gesterkt is mijn hart;~\sep\ ik zal zingen en het psalter bespelen.

  Ontwaak toch, mijn ziel, gij, psalter en citer, ontwaakt;~\sep\ ik wil de dageraad wekken.

  Onder de volken wil ik U prijzen, O Heer,~\sep\ en U bezingen onder de naties.

  Want hoog tot de hemel reikt Uw barmhartigheid,~\sep\ en tot de wolken Uw trouw.

  Verschijn hoog boven de hemelen, O God;~\sep\ dat Uw glorie strale over heel de aarde.

  Opdat Uw geliefden worden bevrijd,~\sep\ help ons door Uw rechterhand, en verhoor ons.

  God heeft gesproken in Zijn heiligdom:~\sep\ ``Ik zal juichen en Sichem verdelen, en het dal van Succoth meten.

  Van Mij is het land Galaäd, van Mij het land Manasse,~\sep\ Efraïm is de helm van Mijn hoofd, Juda Mijn scepter.

  Moab is Mijn wasbekken, op Edom werp Ik Mijn schoeisel,~\sep\ over Filistea zal Ik zegevieren.''

  Wie zal mij binnenvoeren in de versterkte stad,~\sep\ wie mij naar Edom geleiden?

  Zijt Gij het niet, O God, die ons hebt verstoten,~\sep\ en niet meer uittrekt, O God, met onze legerscharen!

  Schenk ons Uw hulp tegen de vijand,~\sep\ want ijdel is de hulp van mensen.

  Met God zullen wij dapper strijden,~\sep\ en Hij zelf zal onze vijanden vertreden.
\end{halfparskip}

\markedsubsectionrubricwithhint{Zaterdagen ``na'': Marmita 53*}{(origineel: hulala 19)}

\begin{halfparskip}
  \psalm{\Ps{153}} Looft de Heer, want Hij is goed:~\sep\ want eeuwig duurt Zijn barmhartigheid!

  \liturgicalhint{Alleluia, Alleluia, Alleluia.~--- Eerste vers.}

  Looft de Heer der heren:~\sep\ want eeuwig duurt Zijn barmhartigheid!

  Die grote wonderen deed, Hij alleen:~\sep\ want eeuwig duurt Zijn barmhartigheid!

  Die de hemelen met wijsheid schiep:~\sep\ want eeuwig duurt Zijn barmhartigheid!

  Die op de wateren de aarde spreidde:~\sep\ want eeuwig duurt Zijn barmhartigheid!

  Die de grote lichten schiep:~\sep\ want eeuwig duurt Zijn barmhartigheid!

  De zon, om over de dag te heersen:~\sep\ want eeuwig duurt Zijn barmhartigheid!

  Maan en sterren, om over de nacht te heersen,~\sep\ want eeuwig duurt Zijn barmhartigheid!

  Die Egypte sloeg in zijn eerstgeborenen:~\sep\ want eeuwig duurt Zijn barmhartigheid!

  En Israël uit hun midden voerde:~\sep\ want eeuwig duurt Zijn barmhartigheid!

  Met sterke hand en uitgestrekte arm,~\sep\ want eeuwig duurt Zijn barmhartigheid!

  Die de Rode Zee in tweeën deelde:~\sep\ want eeuwig duurt Zijn barmhartigheid!

  En Israël er middendoor deed gaan:~\sep\ want eeuwig duurt Zijn barmhartigheid.

  En Farao met zijn legermacht neerwierp in de Rode Zee,~\sep\ want eeuwig duurt Zijn barmhartigheid.

  Die Zijn volk door de woestijn heeft geleid:~\sep\ want eeuwig duurt Zijn barmhartigheid!

  Die machtige vorsten versloeg:~\sep\ want eeuwig duurt Zijn barmhartigheid!

  En sterke koningen doodde:~\sep\ want eeuwig duurt Zijn barmhartigheid!

  Sehon, de vorst der Amorieten,~\sep\ want eeuwig duurt Zijn barmhartigheid!

  En Og, de vorst van Basan:~\sep\ want eeuwig duurt Zijn barmhartigheid.

  En hun land tot eigendom gaf:~\sep\ want eeuwig duurt Zijn barmhartigheid!

  Tot eigendom aan Israël, Zijn dienstknecht:~\sep\ want eeuwig duurt Zijn barmhartigheid!

  Die in onze verdrukking aan ons dacht:~\sep\ want eeuwig duurt Zijn barmhartigheid!

  En ons verloste van onze vijanden,~\sep\ want eeuwig duurt Zijn barmhartigheid!

  Die voedsel geeft aan alle vlees:~\sep\ want eeuwig duurt Zijn barmhartigheid!

  Looft de God van de hemel:~\sep\ want eeuwig duurt Zijn barmhartigheid!

  \psalm{\Ps{136}} Aan de stromen van Babylon, daar zaten wij en weenden,~\sep\ als wij aan Sion dachten.

  Aan de wilgen van dat land,~\sep\ hingen wij onze citers op.

  Want die ons hadden weggevoerd, vroegen ons daarginds om liederen, en die ons verdrukten, om een jubelzang:~\sep\ ``Zingt ons uit Sions liederen!''

  Hoe zouden wij een lied van de Heer zingen,~\sep\ op vreemde bodem!

  Jeruzalem, indien ik u zou vergeten,~\sep\ dan worde mijn rechterhand aan de vergetelheid prijsgegeven.

  Dat mijn tong aan mijn gehemelte kleve,~\sep\ zo ik u niet gedenk,

  Zo ik Jeruzalem niet stel,~\sep\ boven al mijn geneugten.

  Heer reken de zonen van Edom,~\sep\ de dag van Jeruzalem aan.

  Zij riepen: ``Verdelgt haar, verdelgt haar,~\sep\ tot in haar grondvesten toe!''

  O gij, dochter van Babylon, verdelgster;~\sep\ gelukkig hij, die u vergeldt de rampen, die gij bracht over ons!

  Gelukkig hij, die uw kleinen grijpt,~\sep\ en te pletter slaat tegen de rots!

  \psalm{\Ps{137}} Ik wil U prijzen, Heer, uit heel mijn hart,~\sep\ daar Gij hebt geluisterd naar de woorden van mijn mond.

  Ik wil U bezingen voor het aanschijn der engelen,~\sep\ mij neerwerpen, naar Uw heilige tempel gericht,

  En Uw Naam zal ik prijzen,~\sep\ om Uw goedheid en trouw,

  Daar Gij boven alles verheerlijkt hebt,~\sep\ Uw Naam en Uw belofte.

  Wanneer ik U aanriep, hebt Gij mij verhoord,~\sep\ en mijn zielskracht vermeerderd.

  Alle koningen der aarde zullen U prijzen, O Heer,~\sep\ als zij horen de woorden van Uw mond.

  En zij zullen de wegen van de Heer bezingen:~\sep\ ``Waarlijk, groot is de glorie van de Heer!''

  Waarlijk, verheven is de Heer, op de geringe rust Zijn oog,~\sep\ maar op de trotse ziet Hij neer van verre.

  Al wandel ik ook te midden van kwelling, Gij houdt mij in leven, en strekt Uw hand naar mijn woedende vijanden uit;~\sep\ Uw rechterhand is mijn redding.

  Wat de Heer begon, voltooit Hij voor mij; Heer, Uw goedheid blijft eeuwig;~\sep\ trek u niet terug van het werk van Uw handen.
\end{halfparskip}

\begin{halfparskip}
  \liturgicalOption{Alle dagen:}~\sep\ \dd~Alleluia, alleluia; Eer aan U, God, alleluia; Eer aan U, God, alleluia; Heer, ontferm U over ons. Laat ons bidden; vrede zij met ons.

  \cc~Versterk, onze Heer en onze God, onze zwakheid in Uw mededogen; in Uw goedertierenheid bemoedig
  (\translationoptionNl{troost}) en help de broosheid van onze ziel; maak de slaperigheid van onze gedachten wakker, verlicht (\translationoptionNl{neem weg}) de last van onze ledematen; was en reinig het vuil van onze schulden en zonden; verlicht de duisternis van ons intellect; strek uit en leg de machtige hand van bescherming op ons, zodat wij daardoor kunnen opstaan en U onophoudelijk belijden en verheerlijken alle dagen van ons leven, Heer van alles...
\end{halfparskip}

% % % % % % % % % % % % % % % % % % % % % % % % % % % % % % % % % % % % % % % %

\end{document}