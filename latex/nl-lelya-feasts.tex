\documentclass[12pt,twoside,a5paper]{article}

\usepackage{multicol}

\usepackage[main=dutch]{babel}
\usepackage{divine-office}

% % % % % % % % % % % % % % % % % % % % % % % % % % % % % % % % % % % % % % % %

% Version: 2024-12-19
\begin{document}

\title{Lelya~--- feesten en gedachtenissen}
\author{}
\date{}
\maketitle

% The following prevents footnotes and paracol from interacting in bad ways.
% Not really an idea why...
% See: https://stackoverflow.com/questions/61779911/paracol-and-footnote-placing-in-latex
\footnotelayout{\ }

% % % % % % % % % % % % % % % % % % % % % % % % % % % % % % % % % % % % % % % %

\begin{halfparskip}
  \cc~Eer aan God in den hoge \liturgicalhint{(3x)}. En op aarde vrede en goede hoop aan de mensen, altijd en in eeuwigheid.

  [Amen.]~--- \rr~Zegen, Heer.~--- \liturgicalhint{[Vredekus.]}

  \cc~Onze Vader die in de hemelen zijt,

  \rr~Geheiligd zij Uw Naam. Uw rijk kome, heilig, heilig, heilig zijt Gij. Onze Vader die in de hemelen zijt, de hemel en de aarde zijn gevuld met Uw onmetelijke glorie; de engelen en de mensen roepen U toe: heilig, heilig, heilig zijt Gij.~--- Onze Vader die in de hemelen zijt, geheiligd zij Uw Naam. Uw rijk kome, Uw wil geschiede op aarde zoals in de hemel. Geef ons heden het brood dat we nodig hebben en vergeef ons onze schulden en zonden zoals wij ook vergeven hebben aan onze schuldenaren. En leid ons niet in bekoring, maar verlos ons van de Kwade. Want van U is het koninkrijk en de kracht en de heerlijkheid in eeuwigheid, amen.

  \cc~Eer aan de Vader, de Zoon, en de Heilige Geest.

  \rr~Vanaf het begin en in alle eeuwigheid, amen en amen. Onze Vader die in de hemelen zijt, geheiligd zij Uw naam, Uw rijk kome, heilig, heilig, heilig zijt Gij. Onze Vader die in de hemelen zijt, de hemel en de aarde zijn gevuld met Uw onmetelijke glorie; de engelen en de mensen roepen U toe: heilig, heilig, heilig zijt Gij.

  \dd~Laat ons opstaan om te bidden, vrede zij met ons.

  \liturgicalhint{Feesten:} Laat ons opstaan, o Heer, in de verborgen kracht van Uw Godheid, mogen wij bevestigd (\translationoptionNl{versterkt}) worden in de wonderbaarlijke hoop van Uw Majesteit, mogen wij verheven en versterkt worden door de hoge arm van Uw macht; en mogen wij door de hulp van Uw genade waardig zijn om altijd U te loven, te eren, te belijden en te aanbidden, Heer van alles, Vader...

  \liturgicalhint{Gedachtenissen:} Laat ons opstaan, o Heer, in Uw kracht, en bevestigd worden in Uw hoop, verheven en versterkt worden door de hoge arm van Uw macht; en mogen wij waardig zijn, door de hulp van Uw genade, om altijd U te loven, te eren, te belijden en te aanbidden, Heer van alles, Vader...

  \liturgicalhint{Feesten en gedachtenissen: hulale 12--14 geïnterpoleerd met hun qanone en gebeden uit het psalter. Eer aan... na elke marmita. Op Kerstmis en Epifanie: hulale 12--15.}
\end{halfparskip}

% % % % % % % % % % % % % % % % % % % % % % % % % % % % % % % % % % % % % % % %

\hulala{12}

\Slota{Wij moeten U belijden, aanbidden en verheerlijken, O Koning die de koningen aanbidden. Koren van tienduizenden engelen en aartsengelen, staan met grote angst en vrees, en dienen en vieren de aanbiddelijke eer van Uw Majesteit, in alle seizoenen en tijden, Heer van alles...}

\marmita{32}

\PSALMtitle{82}{Tegen onrechtvaardige rechters}

\begin{halfparskip}
  God rijst op in de goddelijke raad, Hij houdt gericht te midden der goden.

  \qanona{Rechters, oordeelt rechtvaardig en blijft ver van (\translationoptionNl{vermijdt}) ongerechtigheid.}

  \liturgicalhint{Feesten \& gedachtenissen.} \emph{De waakzame Intelligentie riep mij als uit de slaap en wekte mij: Sta op, jij die in slaap bent verzonken en werp het gewicht van je luiheid af.}

  ``Hoe lang nog zult gij onrechtvaardig oordelen,~\sep\ en de zaak der bozen begunstigen?

  Verdedigt verdrukten en wezen,~\sep\ geeft aan ellendigen en armen hun recht,

  Bevrijdt verdrukten en behoeftigen:~\sep\ ontrukt ze aan de hand van de bozen.''

  Zij hebben inzicht noch verstand, ze wandelen in duisternis;~\sep\ alle grondslagen der aarde worden geschokt.

  Ik heb gezegd: ``Goden zijt gij,~\sep\ en zonen van de Allerhoogste, gij allen.

  Toch zult gij sterven als mensen,~\sep\ neervallen als welke machthebber ook''.

  Rijs op, O God, en richt de aarde,~\sep\ want rechtens horen alle volken U toe.
\end{halfparskip}

\PSALMtitle{83}{De vijand dringt op!}

\begin{halfparskip}
  \psalmsubtitle{a) De vijanden, Heer, spannen tegen ons samen}

  Wil niet zwijgen, O Heer,~\sep\ wil niet zwijgen, O God, noch rusten!

  \qanona{Er is niemand zoals U onder de dingen die gemaakt zijn, O almachtige God.}

  Want zie, Uw vijanden woelen,~\sep\ en die U haten, steken het hoofd omhoog.

  Tegen Uw volk smeden zij plannen,~\sep\ en tegen Uw beschermelingen spannen zij samen.

  ``Komt'', zo spreken zij, ``vagen wij hen weg uit de rij van de volken,~\sep\ en dat men de naam van Israël niet meer gedenke.''

  Waarlijk, zij overleggen eensgezind,~\sep\ en sluiten een verbond tegen U:

  De tenten van Edom en de Ismaëlieten,~\sep\ Moab en de zonen van Hagar,

  Gebal, Ammon en Amalec,~\sep\ de Filistijnen en de bewoners van Tyrus;

  Ook de Assyriërs verbonden zich met hen,~\sep\ en leenden hun arm aan de zonen van Lot.
\end{halfparskip}

\begin{halfparskip}
  \psalmsubtitle{b) Verdelg de vijand, O God, gelijk weleer!}

  Behandel hen als Madian,~\sep\ als Sisara en Jabin bij de beek Cison,

  Die bij Endor werden verdelgd,~\sep\ en tot mest op het veld zijn gemaakt.

  Maak hun vorsten als Oreb en Zeb,~\sep\ als Zebëe en Salmana, al hun leiders,

  Die zeiden:~\sep\ ``Laten wij het gebied van God gaan bezetten.''

  Maak ze, mijn God, als bladeren die dwarrelen in de storm,~\sep\ als een strohalm, heen en weer gezwiept door de wind,

  Als vuur, dat bossen verteert,~\sep\ en als een vlam, die bergen verschroeit,

  Zo moogt Gij hen vervolgen met Uw stormwind,~\sep\ door Uw orkaan hen verwarren.

  Overdek hun gelaat met schande,~\sep\ opdat zij Uw Naam zoeken, O Heer;

  Laat ze beschaamd en verbijsterd staan voor eeuwig,~\sep\ laat ze te schande worden en vergaan.

  Dat zij erkennen dat Gij, wiens naam de Heer is,~\sep\ de enig Verhevene zijt over heel de aarde.
\end{halfparskip}

\PSALMtitle{84}{Verlangen naar het huis van God}

\begin{halfparskip}
  \psalmsubtitle{a) Mijn God, ik verlang naar Uw heiligdom}

  Hoe liefelijk is Uw woonstede, Heer der legerscharen;~\sep\ mijn ziel verlangt, ziet smachtend uit naar de voorhoven van de Heer.

  \qanona{Hoe heerlijk en prachtig is Uw heiligdom, O God, die alles heiligt.}

  Mijn hart en mijn lichaam,~\sep\ juichen voor de levende God.

  Ook de mus vindt een woning,~\sep\ en de zwaluw een nest, waar ze haar jongen in neerlegt.

  Uw altaren, O Heer der legerscharen,~\sep\ mijn Koning en mijn God!
\end{halfparskip}

\begin{halfparskip}
  \psalmsubtitle{b) Gelukkig hij, die naar het heiligdom gaat}

  Gelukkig zij, die wonen in Uw huis, O Heer;~\sep\ eeuwig loven zij U.

  Gelukkig de man, die hulp krijgt van U,~\sep\ als hij het plan heeft op bedevaart te gaan:

  Trekken zij door een dorre vallei, dan maken zij haar tot bron,~\sep\ en de vroege regen bekleedt haar met zegeningen.

  Al gaande zal hun kracht vermeerderen:~\sep\ de God der goden zullen zij in Sion zien.
\end{halfparskip}

\begin{halfparskip}
  \psalmsubtitle{c) Leid mij, Heer, naar Uw heiligdom!}

  Heer der legerscharen, hoor naar mijn bede,~\sep\ ach, luister toch, O God van Jacob.

  Zie toe, O God, ons schild,~\sep\ en zie op het gelaat van Uw gezalfde.

  Waarlijk, één dag in Uw voorhoven is beter,~\sep\ dan duizend andere.

  Liever blijf ik staan op de drempel van het huis van mijn God,~\sep\ dan te toeven in de tenten der bozen.

  Want een zon en een schild is God de Heer;~\sep\ de Heer schenkt genade en glorie.

  Hij weigert het goede niet,~\sep\ aan die in onschuld wandelen.

  Heer der legerscharen,~\sep\ gelukkig de mens, die op U vertrouwt.
\end{halfparskip}

\Slota{Schep genoegen, Heer, in het gebed van Uw dienaren; neem behagen in de dienst van Uw aanbidders; vergeef de schulden van hen die U verheerlijken; en verwijder (\translationoptionNl{laat verdwijnen}) Uw woede van hen van Uw huishouden, U die goed bent en voor onze levens zorgt, in alle seizoenen en tijden, Heer van alles...}

\marmita{33}

\PSALMtitle{85}{Verlangen naar het Messiaanse Rijk}

\begin{halfparskip}
  \psalmsubtitle{a) Dank, Heer, voor de bevrijding!}

  Gij zijt Uw land genadig geweest, O Heer,~\sep\ hebt het lot van Jacob ten goede gekeerd.

  \qanona{Zend, O onze Heer, hulp en redding aan Uw gelovigen door de grote kracht van Uw Kruis.}

  \liturgicallbracket\liturgicaloption{Of:} \emph{Wie de zon wil afbeelden, vergist zich enorm, want hij heeft niet het verstand om haar heerlijke dingen te vergelijken (\translationoptionNl{te bevatten}). Hij die een erfgenaam wil zijn van het hemels koninkrijk, moet zijn ziel bevrijden van de slavernij van deze wereld.}\liturgicalrbracket

  Vergeven hebt Gij de schuld van Uw volk,~\sep\ en al zijn zonden bedekt.

  Uw gramschap hebt Gij geheel bedwongen,~\sep\ de gloed van Uw toorn gestild.
\end{halfparskip}

\begin{halfparskip}
  \psalmsubtitle{b) Voltooi, Heer, de Verlossing!}

  Herstel ons, o God, onze Redder,~\sep\ en leg Uw wrevel tegen ons af.

  Zult Gij dan eeuwig tegen ons toornen,~\sep\ of verbolgen blijven van geslacht tot geslacht?

  Zult Gij ons dan niet opnieuw doen leven,~\sep\ opdat Uw volk zich verblijde in U?

  Toon ons, Heer, Uw barmhartigheid,~\sep\ en schenk ons Uw heil!
\end{halfparskip}

\begin{halfparskip}
  \psalmsubtitle{c) Gij zult ons zegenen, Heer}

  Ik wil horen naar wat de Heer God spreekt:~\sep\ vrede voorzeker kondigt Hij aan.

  Voor Zijn volk en Zijn heiligen,~\sep\ en voor hen, die zich van harte keren tot Hem.

  Ja waarlijk, Zijn heil is nabij voor wie Hem vrezen,~\sep\ en zo zal er glorie wonen in ons land:

  Barmhartigheid en trouw zullen elkander ontmoeten,~\sep\ gerechtigheid en vrede elkander de kus geven.

  Trouw zal aan de aarde ontspruiten,~\sep\ en gerechtigheid neerzien vanuit de hemel.

  De Heer zelf zal zegen schenken,~\sep\ en ons land zijn vruchten geven.

  Gerechtigheid zal vóór Hem uitgaan,~\sep\ en heil zijn schreden volgen.
\end{halfparskip}

\PSALMtitle{86}{Bede om hulp}

\begin{halfparskip}
  \psalmsubtitle{a) Mijn God, help mij!}

  Neig Uw oor, O Heer; verhoor mij,~\sep\ want ik ben ellendig en arm.

  \qanona{Christus, de Vriend van de boeteling, open de deur voor ons gebed en aanvaard onze bede.}

  Bescherm mij, want ik ben U toegewijd;~\sep\ red Uw dienaar, die op U hoopt.

  Mijn God zijt Gij; wees mij genadig, O Heer,~\sep\ want almaar door roep ik tot U.

  Verblijd de ziel van Uw dienaar,~\sep\ want tot U, O Heer, verhef ik mijn ziel.

  Want Gij, O Heer, zijt goed en genadig,~\sep\ vol erbarming voor al wie U aanroept.

  Luister, Heer, naar mijn bede,~\sep\ en geef acht op de stem van mijn smeken.

  Op de dag van mijn kwelling riep ik tot U,~\sep\ omdat Gij mij verhoren zult.
\end{halfparskip}

\begin{halfparskip}
  \psalmsubtitle{b) Gij zult mij helpen, O Heer}

  Onder de goden, O Heer, is er geen als Gij,~\sep\ en geen werk is gelijk aan het Uwe.

  Alle volken, door U geschapen, zullen komen, en U aanbidden, O Heer,~\sep\ en verheerlijken Uw Naam.

  Want Gij zijt groot en Gij doet wonderwerken:~\sep\ Gij zijt God, en Gij alleen.
\end{halfparskip}

\begin{halfparskip}
  \psalmsubtitle{c) Gij wilt mij helpen, O Heer}

  Toon mij Uw weg, O Heer, opdat ik wandele in Uw waarheid,~\sep\ richt mijn hart op de vrees voor Uw Naam.

  Ik zal U prijzen, Heer, mijn God, uit heel mijn hart:~\sep\ en eeuwig Uw Naam verheerlijken.

  Want Uw erbarming voor mij was groot,~\sep\ en uit de diepten van het dodenrijk hebt Gij mij opgehaald.

  Trotsen, O God, zijn tegen mij opgestaan, een bende geweldenaars staat mij naar het leven,~\sep\ zij houden U niet voor ogen.

  Maar Gij, O Heer, zijt een barmhartige en liefdevolle God,~\sep\ lankmoedig, rijk aan ontferming en trouw.

  Blik op mij neer en wees mij genadig;~\sep\ schenk aan Uw dienaar Uw kracht, en red de zoon van Uw dienstmaagd.

  Geef mij een teken van Uw gunst, opdat die mij haten, Heer, vol schaamte zien,~\sep\ dat Gij, O Heer, mij hulp en troost hebt geschonken.
\end{halfparskip}

\Slota{Bevestig, mijn Heer, de fundamenten van Uw Kerk in Uw mededogen, versterk haar balken in Uw liefderijke goedheid, en laat Uw heerlijkheid wonen in de tempel gereserveerd voor Uw dienst, alle dagen van de wereld, Heer van alles...}

\marmita{34}

\PSALMtitle{87}{Ode aan Jeruzalem}

\begin{halfparskip}
  Zijn stichting op de heilige bergen bemint de Heer:~\sep\ de poorten van Sion~---

  \qanona{Aanbiddenswaardig is God de Schepper, die zorgt voor alle generaties.}

  \liturgicallbracket\liturgicaloption{Of:} \emph{Wij zijn arm en zwak (geworden) door onze daden, er zijn onder ons geen rechtvaardigen en deugdzamen om U gunstig te stemmen. We weten ook niet hoe te bidden of hoe (U) te verheerlijken, en we vrezen woorden uit te spreken die U niet waardig zijn, daarom bouwde de Heer voor Hem een huis op aarde, zodat wie de Heer wil zien, zou komen naar Zijn huis.}\liturgicalrbracket

  boven alle tenten van Jacob.

  Roemrijke dingen verhaalt men van u,~\sep\ O stad van God!

  Rahab en Babel zal Ik tot Mijn vereerders rekenen:~\sep\ zie, Filistea en Tyrus en het volk der Ethiopiërs: daar zijn ze geboren!

  Over Sion zal men zeggen: ``Allen, man voor man, zijn in haar geboren,~\sep\ en de Allerhoogste zelf heeft haar bevestigd.''

  De Heer zal schrijven in het boek der volken:~\sep\ ``Daar zijn ze geboren.''

  En in reidans zullen zij zingen:~\sep\ ``Al mijn bronnen zijn in u.''
\end{halfparskip}

\PSALMtitle{87}{De vertwijfeling nabij!}

\begin{halfparskip}
  \psalmsubtitle{a) Groot, Heer, is mijn ellende!}

  Heer, mijn God, ik roep overdag,~\sep\ en ik jammer 's nachts voor Uw aanschijn.

  \qanona{Gij, die ons gemaakt hebt, zijt barmhartig; in Uw goedertierenheid heb medelijden met ons.}

  Dringe mijn bede toch door tot U,~\sep\ neig Uw oor naar mijn klagen!

  Want mijn ziel is verzadigd met rampen,~\sep\ mijn leven is het dodenrijk nabij.

  Ik word gerekend onder hen, die ten grave dalen,~\sep\ ik ben als een man zonder kracht.

  Onder de doden is mijn legerstede,~\sep\ als van verslagenen, die liggen in het graf,

  Aan wie Gij niet meer denkt,~\sep\ die aan Uw zorgen zijn onttrokken.

  In een diepe groeve hebt Gij mij neergelegd,~\sep\ in duisternis, in een diep ravijn.

  Uw verontwaardiging drukt zwaar op mij,~\sep\ met al Uw golven slaat Gij mij neer.

  Gij hebt mijn vrienden van mij vervreemd, mij tot afschuw voor hen gemaakt;~\sep\ ik zit gevangen, en kan niet ontkomen.
\end{halfparskip}

\begin{halfparskip}
  \psalmsubtitle{b) Heer, kom mij te hulp!}

  Van ellende verkwijnen mijn ogen; iedere dag roep ik tot U, O Heer,~\sep\ naar U strek ik mijn handen uit.

  Of doet Gij voor doden nog wonderen,~\sep\ of zullen gestorvenen, herrijzend, U loven?

  Of wordt Uw goedheid in het graf verkondigd,~\sep\ Uw trouw in het dodenrijk?

  Openbaart men in het duister Uw wonderen,~\sep\ in het land der vergetelheid Uw genade?
\end{halfparskip}

\begin{halfparskip}
  \psalmsubtitle{c) Help mij, Heer, in mijn ongeluk!}

  Ik echter roep tot U, O Heer,~\sep\ mijn bede stijgt tot U op in de morgen.

  Waarom toch, O Heer, verstoot Gij mij,~\sep\ verbergt Gij voor mij Uw gelaat?

  Van jongs af ben ik ellendig en stervend,~\sep\ ik torste Uw verschrikkingen en kwijnde.

  Uw toorn is over mij heengegaan,~\sep\ Uw verschrikkingen sloegen mij neer.

  Zij omgeven mij immer als water,~\sep\ omringen mij alle tezamen.

  Vriend en makker hebt Gij van mij vervreemd,~\sep\ mijn vertrouweling is de duisternis.
\end{halfparskip}

% % % % % % % % % % % % % % % % % % % % % % % % % % % % % % % % % % % % % % % %

\hulala{13}

\Slota{Stort Uw genaden over ons uit, O Heer, vermenigvuldig Uw hulp voor ons en sterk ons, zoals U gewend bent, om U te behagen volgens Uw wil, om te wandelen volgens Uw geboden en Uw Godheid gunstig te stemmen met goede daden van gerechtigheid, alle dagen van ons leven, Heer van alles...}

\marmita{35}

\PSALMtitle{89}{Jubel over Gods belofte aan David}

\begin{halfparskip}
  \psalmsubtitle{a) Heer, Gij hebt een belofte aan David gedaan}

  De gunsten van de Heer wil ik eeuwig bezingen,~\sep\ door alle geslachten heen zal mijn mond Uw trouw verkondigen,

  \qanona{De goede dingen die God beloofde aan Abram en David, vervulde Hij in onze dagen door de werken in Christus. Eer aan Hem.}

  \liturgicallbracket\liturgicaloption{Of:} \emph{Mogen die goedertierenheid en die gave, die neerdaalde en verbleef op de leerlingen, Uw aanbidders overschaduwen, mijn Heer, en voor altijd op hen blijven.}\liturgicalrbracket

  Want Gij hebt gezegd; ``De genade staat eeuwig vast'';~\sep\ in de hemel hebt Gij Uw trouw gegrondvest.

  ``Een verbond ging Ik aan met Mijn uitverkorene;~\sep\ aan David, Mijn dienaar, zwoer Ik een eed:

  Ik zal uw nazaat voor eeuwig bevestigen,~\sep\ en uw troon in stand houden door alle geslachten.''
\end{halfparskip}

\begin{halfparskip}
  \psalmsubtitle{b) Gij zijt almachtig, Heer, en wilt Uw belofte vervullen}

  De hemelen loven Uw wonderen, O Heer,~\sep\ en Uw trouw in de kring der heiligen.

  Want wie in de wolken zal de Heer evenaren,~\sep\ wie onder Gods zonen is gelijk aan de Heer?

  Ontzagwekkend is God in de gemeenschap der heiligen,~\sep\ groot en vreeswekkend boven allen om Hem heen.

  O Heer, God der legerscharen, wie is U gelijk?~\sep\ Machtig zijt Gij, O Heer, en van Uw trouw omgeven,

  Gij beheerst de trotse zee,~\sep\ Gij breekt haar onstuimige golven.

  Gij hebt Rahab doorstoken en vertreden,~\sep\ met Uw machtige arm Uw vijanden verstrooid.

  Van U zijn de hemelen, van U is de aarde;~\sep\ Gij grondvestte de wereld met wat ze bevat.

  Noord en zuid hebt Gij geschapen:~\sep\ Thabor en Hermon juichen om Uw Naam.

  Gij hebt een krachtige arm,~\sep\ sterk is Uw hand, Uw rechter opgeheven.

  Gerechtigheid en recht zijn de grondslag van Uw troon;~\sep\ genade en trouw gaan voor U uit.

  Gelukkig het volk, dat weet te jubelen;~\sep\ het wandelt, o Heer, in het licht van Uw aanschijn,

  Het verheugt zich voor immer om Uw Naam,~\sep\ en roemt in Uw gerechtigheid,

  Want Gij zijt de glans van hun kracht,~\sep\ door Uw gunst verheft zich onze hoorn.

  Ja, van de Heer komt ons schild,~\sep\ van Israëls Heilige komt onze koning.
\end{halfparskip}

\begin{halfparskip}
  \psalmsubtitle{c) Dit hebt Gij David beloofd:}

  Eens hebt Gij in een visioen tot Uw heiligen het woord gesproken:~\sep\ ``Ik heb een held de kroon opgezet, een uit het volk verkoren en verheven.

  Ik heb David gevonden, Mijn dienaar,~\sep\ hem gezalfd met Mijn heilige olie,

  Opdat Mijn hand voor immer met hem zij,~\sep\ en Mijn arm hem sterke.

  Geen vijand zal hem misleiden,~\sep\ geen booswicht hem verdrukken;

  Maar zijn weerstrevers zal Ik voor zijn aanschijn verpletteren,~\sep\ en die hem haten, doorsteken.

  Mijn trouw en Mijn genade zullen met hem zijn,~\sep\ en in Mijn Naam zal zijn hoorn zich verheffen.

  Zijn hand zal Ik uitstrekken over de zee,~\sep\ en over de stromen zijn rechter.

  Hij zal tot Mij roepen: ``Mijn Vader zijt Gij,~\sep\ mijn God en de Rots van mijn heil.''

  En Ik zal hem maken tot eerstgeborene,~\sep\ tot de hoogste onder de koningen der aarde.

  Ik zal hem eeuwig Mijn goedgunstigheid bewaren,~\sep\ en Mijn verbond met hem zal duurzaam zijn.

  Eeuwigdurend maak Ik zijn geslacht,~\sep\ en zijn troon als de dagen van de hemel.
\end{halfparskip}

\begin{halfparskip}
  \psalmsubtitle{d) Uw trouw, Heer, is onwankelbaar}

  Mochten zijn zonen Mijn wet verlaten,~\sep\ en niet wandelen naar Mijn geboden;

  Mochten zij Mijn voorschriften schenden,~\sep\ Mijn geboden niet bewaren,

  Dan zal Ik met de roede hun misdaad bestraffen,~\sep\ en met gesels hun schuld;

  Maar Mijn genade zal Ik hem niet onttrekken,~\sep\ en Ik zal niet breken Mijn trouw.

  Mijn verbond zal Ik niet schenden,~\sep\ noch de uitspraak van Mijn lippen veranderen.

  Eens en voor immer heb Ik bij Mijn heiligheid gezworen:~\sep\ Nooit breek Ik David Mijn woord,

  Zijn nakroost zal blijven in eeuwigheid,~\sep\ en zijn troon zal voor Mijn aanschijn zijn als de zon,

  Als de maan, die eeuwig blijft,~\sep\ een trouwe getuige aan de hemel.''
\end{halfparskip}

\begin{halfparskip}
  \psalmsubtitle{e) Gij schijnt ontrouw, O Heer}

  Toch hebt Gij Uw gezalfde verstoten en verworpen,~\sep\ hevig tegen hem getoornd;

  Gij hebt het verbond met Uw dienaar versmaad,~\sep\ zijn kroon ontwijd in het stof,

  Gij hebt al zijn wallen geslecht,~\sep\ zijn vestingwerken in puin gelegd.

  Iedere voorbijganger heeft hem beroofd,~\sep\ hij is de spot van zijn buren geworden.

  Gij hebt de rechterhand van zijn vijanden verheven,~\sep\ al zijn weerstrevers met vreugde vervuld.

  De snee van zijn zwaard hebt Gij afgestompt,~\sep\ hem niet gesteund in de strijd.

  Zijn luister hebt Gij doen tanen,~\sep\ en zijn troon ter aarde geworpen.

  De dagen van zijn jeugd hebt Gij verkort,~\sep\ Gij hebt hem met schande bedekt.
\end{halfparskip}

\begin{halfparskip}
  \psalmsubtitle{f) Vervul, Heer, Uw belofte!}

  Hoe lang nog, Heer; zult Gij U dan immer verbergen!~\sep\ Moet Uw misnoegen als een vuur blijven branden?

  Bedenk hoe kort mijn leven is,~\sep\ hoe zwak Gij alle mensen geschapen hebt.

  Wie is er, die leeft, en de dood niet ziet,~\sep\ die zich aan de macht van de onderwereld onttrekt?

  Waar is Uw aloude goedgunstigheid, Heer,~\sep\ die Gij David bij Uw trouw hebt gezworen?

  Gedenk toch, Heer, de smaad van Uw dienaars;~\sep\ ik draag in mijn boezem al de vijandschap van de volkeren,

  Waarmee Uw weerstrevers U honen, O Heer,~\sep\ en bij iedere schrede Uw gezalfde verguizen.

  Gezegend de Heer in eeuwigheid;~\sep\ zo zij het, zo zij het!
\end{halfparskip}

\Slota{Wij belijden, aanbidden en verheerlijken U, die alles in handen heeft door de macht van Uw woord, en de werelden en schepselen regeert volgens het milde doel van Uw wil, grote Koning der heerlijkheid, Wezen dat van eeuwigheid is, in alle seizoenen en tijden, Heer van alles...}

\marmita{36}

\PSALMtitle{90}{Ons leven is kort en vergankelijk}

\begin{halfparskip}
  \psalmsubtitle{a) God van eeuwigheid!}

  Heer, Gij waart ons tot toevlucht,~\sep\ van geslacht tot geslacht.~\sep\ Eer de bergen werden verwekt,

  \qanona{Gij die zorgt voor ons, Almachtige, heb medelijden met onze zondigheid.}

  \liturgicallbracket\liturgicaloption{Of:} \emph{O barmhartige Goedertierenheid, O Nederigheid, die de zwakheid der aardse wezens hebt verheven.}\liturgicalrbracket

  eer aarde en wereld geboren werden:~\sep\ van eeuwigheid tot eeuwigheid zijt Gij, O God.

  De stervelingen beveelt Gij terug te keren tot stof,~\sep\ Gij spreekt: ``Keert terug, gij kinderen der mensen.''

  Want duizend jaren zijn in Uw ogen als de dag van gister, die verstreek,~\sep\ en als een wake in de nacht.

  Gij rukt ze weg: zij worden als een droom in de morgen,~\sep\ als het groenende gras.

  `s~Morgens bloeit het en groeit het,~\sep\ `s~avonds wordt het gemaaid en verdort.
\end{halfparskip}

\begin{halfparskip}
  \psalmsubtitle{b) Om onze zonden, O God, zijn wij in ellende}

  Waarlijk, door Uw toorn zijn wij verteerd,~\sep\ door Uw verbolgenheid ontsteld;

  Gij hebt onze schuld U voor ogen gesteld,~\sep\ onze geheime zonden in het licht van Uw aanschijn.

  Ja, door Uw toorn vloden al onze dagen voorbij,~\sep\ vergingen als een zucht onze jaren.

  Zeventig jaren duurt hoogstens ons leven,~\sep\ of tachtig, als we krachtig zijn;

  De meeste daarvan zijn kommer en schijn,~\sep\ want snel gaan zij heen en zij vlieden voorbij.

  Wie kan de kracht van Uw gramschap meten,~\sep\ en Uw toorn naar de vreze, verschuldigd aan U?
\end{halfparskip}

\begin{halfparskip}
  \psalmsubtitle{c) Geef blijdschap, Heer, om het doorgestane leed!}

  Leer ons onze dagen tellen,~\sep\ opdat wij de wijsheid van het hart verwerven.

  Keer terug, O Heer; hoelang nog?~\sep\ en wees Uw dienaars genadig.

  Verzadig ons spoedig met Uw ontferming,~\sep\ opdat wij juichen en blij zijn al onze dagen.

  Geef ons vreugde voor de dagen toen Gij ons hebt gekastijd,~\sep\ voor de jaren, waarin wij ellende doorstonden.

  Laat Uw werk voor Uw dienaren stralen,~\sep\ en voor hun kinderen Uw glorie!

  Dale op ons neer de goedheid van de Heer, onze God; doe het werk van onze handen gedijen voor ons,~\sep\ doe het werk van onze handen gedijen!
\end{halfparskip}

\PSALMtitle{91}{Onder Gods hoede}

\begin{halfparskip}
  \psalmsubtitle{a) De Allerhoogste, uw Behoeder in alle nood}

  Gij, die onder de hoede van de Allerhoogste leeft,~\sep\ in de schaduw van de Almachtige woont,

  \qanona{Gij zijt mijn hoop, Christus, laat mij nooit beschaamd staan.}

  Zeg tot de Heer: ``Mijn toevlucht en burcht,~\sep\ mijn God, op wie ik vertrouw.''

  Want Hij zal u bevrijden uit de strik van de jagers,~\sep\ van de verderfelijke pest.

  Hij zal u met Zijn vleugels beschermen, en onder Zijn wieken zult gij vluchten;~\sep\ een beukelaar en schild is Zijn trouw.

  Geen verschrikking bij nacht zult gij vrezen,~\sep\ geen pijl, die voortsnort bij dag.

  Geen pest, die rondwaart in het duister,~\sep\ geen verwoestend verderf op de middag.

  Al vallen er duizend aan uw zijde, en tienduizend aan uw rechterhand,~\sep\ het zal tot u niet genaken.

  Ja, met eigen ogen zult gij het zien,~\sep\ en de vergelding der bozen aanschouwen.

  Want de Heer is uw toevlucht,~\sep\ de Allerhoogste hebt gij gemaakt tot uw schutse.

  Geen ramp zal u genaken,~\sep\ en geen plaag zal naderen tot uw tent.

  Want Hij gaf over u een bevel aan Zijn engelen,~\sep\ u te behoeden op al uw wegen.

  Zij zullen u op de handen dragen,~\sep\ opdat gij aan geen steen uw voet zoudt stoten.

  Op slang en adder zult gij trappen,~\sep\ leeuw en draak zult gij vertreden,
\end{halfparskip}

\begin{halfparskip}
  \psalmsubtitle{b) Ik zal hem bevrijden en verheerlijken}

  Omdat hij Mij aanhangt, zal Ik hem redden;~\sep\ omdat hij Mijn Naam kent, hem beschermen.

  Hij zal Mij aanroepen, en Ik zal hem verhoren, in de nood zal Ik met hem zijn,~\sep\ Ik zal hem redden en hem eren.

  Ik zal hem verzadigen met lengte van dagen,~\sep\ en hem tonen Mijn heil.
\end{halfparskip}

\PSALMtitle{92}{Vreugde over Gods rechtvaardigheid}

\begin{halfparskip}
  \psalmsubtitle{a) Ik wil U loven, O Heer!}

  Heerlijk is het de Heer te loven,~\sep\ Uw Naam te bezingen, O Allerhoogste;

  \qanona{Machtige en Alvermogende, bewaar Uw aanbidders.}

  vroeg in de morgen Uw erbarming te verkondigen,~\sep\ en Uw trouw gedurende de nacht,

  Op de tiensnarige harp en de lier,~\sep\ met zang bij citerspel.

  Want door Uw daden, Heer, verheugt Gij mij,~\sep\ ik juich om het werk van Uw handen.
\end{halfparskip}

\begin{halfparskip}
  \psalmsubtitle{b) De boze miskent Uw Voorzienigheid en vergaat}

  Hoe luisterrijk, Heer, zijn Uw werken,~\sep\ hoe diepzinnig zijn Uw gedachten!

  De onverstandige beseft het niet,~\sep\ en de dwaze ziet het niet in.

  Al bloeien de goddelozen als gras,~\sep\ al schitteren allen, die kwaad bedrijven,

  Tot eeuwige ondergang zijn zij gedoemd;~\sep\ maar Gij, O Heer, zijt eeuwig verheven.

  Want zie, Uw vijanden, Heer, Uw vijanden zullen vergaan:~\sep\ alle bozen zullen worden verstrooid.

  Als de hoorn van een buffel hebt Gij mijn hoorn verheven,~\sep\ mij gezalfd met de zuiverste olie.

  Mijn oog zag op mijn vijanden neer,~\sep\ en over de bozen, die tegen mij opstonden, vernamen mijn oren verblijdende dingen.
\end{halfparskip}

\begin{halfparskip}
  \psalmsubtitle{c) Maar de vrome bloeit als een palmboom}

  De rechtvaardige zal als een palmboom bloeien,~\sep\ als een ceder van de Libanon gedijen.

  Die staan geplant in het huis van de Heer,~\sep\ zullen bloeien in de voorhoven van onze God.

  Tot in hun ouderdom dragen zij vrucht,~\sep\ blijven zij sappig en fris,

  Om te verkondigen hoe rechtvaardig de Heer is,~\sep\ mijn Rots, en dat in Hem geen onrecht is.
\end{halfparskip}

\marmita{37}

\PSALMtitle{93}{God is Koning voor eeuwig}

\begin{halfparskip}
  De Heer is Koning, met majesteit bekleed;~\sep\ bekleed is de Heer met macht, Hij heeft Zich omgord;

  \qanona{We aanbidden Uw Wezen, dat zonder begin is, Gij die glorierijk in den hoge zijt, bewaar Uw Kerk en red haar.}

  Hij heeft het aardrijk bevestigd,~\sep\ dat niet zal wankelen.

  Hecht staat Uw troon van ouds;~\sep\ Gij zijt van eeuwigheid.

  De stromen verheffen, O Heer, de stromen verheffen hun stem,~\sep\ de stromen verheffen hun bruisen.

  Maar boven de stem der wijde wateren, boven de branding der zee~\sep\ is machtig de Heer in de hoge.

  Betrouwbaar bovenmate zijn Uw getuigenissen;~\sep\ Uw huis, Heer, past heiligheid in lengte van dagen.
\end{halfparskip}

\PSALMtitle{94}{Beroep op Gods rechtvaardigheid}

\begin{halfparskip}
  \psalmsubtitle{a) Straf, Heer, onze goddeloze verdrukkers!}

  Wrekende God, O Heer,~\sep\ wrekende God, verschijn!

  \qanona{U die alles weet en almachtig zijt, die allen oordeelt en onze Heer zijt, red Uw dienaren die U aanroepen.}

  Rijs op, Gij, Rechter der aarde,~\sep\ vergeld naar verdienste de trotsen!

  Hoelang nog zullen de bozen, O Heer,~\sep\ hoelang nog zullen de bozen roemen,

  Zullen zij snoeven, onbeschaamd spreken,~\sep\ pochen, die het kwade bedrijven?

  Uw volk, Heer, vertrappen zij,~\sep\ Uw erfdeel drukken zij neer;

  Zij doden weduwe en vreemdeling,~\sep\ en wezen brengen zij om.

  En dan zeggen ze nog: ``De Heer ziet het niet,~\sep\ de God van Jacob merkt het niet op.''
\end{halfparskip}

\begin{halfparskip}
  \psalmsubtitle{b) Gij, Heer, ziet en bestraft de bozen}

  Komt toch tot inzicht, gij dwazen onder het volk,~\sep\ onverstandigen, wanneer wordt gij wijs?

  Zou Hij, die het oor heeft geplant, niet horen,~\sep\ of die het oog heeft gevormd, niet zien?

  Zou Hij, die de volken opvoedt, niet straffen,~\sep\ Hij, die de mensen inzicht geeft?

  De Heer kent de gedachten der mensen:~\sep\ want ja, ze zijn ijdel.
\end{halfparskip}

\begin{halfparskip}
  \psalmsubtitle{c) Gij, de trouwe Behoeder van het recht}

  Gelukkig de man, die Gij onderricht, O Heer,~\sep\ en die Gij onderwijst door Uw Wet,

  Om hem rust te schenken in tijden van nood,~\sep\ tot voor de boze het graf is gedolven.

  Want nooit zal de Heer Zijn volk verstoten,~\sep\ noch Zijn erfdeel verlaten;

  Maar weer zal er recht in de rechtspraak zijn,~\sep\ de rechtschapenen van hart zullen allen het volgen.

  Wie treedt voor mij tegen de boosdoeners op,~\sep\ wie staat mij tegen de booswichten bij?

  Stond de Heer mij niet bij,~\sep\ dan woonde ik spoedig in het oord van stilte.

  Als ik denk: ``Nu wankelt mijn voet'',~\sep\ dan steunt mij Uw goedheid, Heer.

  Als kommer steeds meer mijn hart benauwt,~\sep\ dan verkwikt Uw vertroosting mijn ziel.

  Heeft met U iets gemeen een partijdige rechtbank,~\sep\ die kwellingen wekt onder schijn van wet?

  Men mag dan de gerechte het leven belagen,~\sep\ en onschuldig bloed voor schuldig verklaren,

  De Heer zal mij zeker tot beschutting zijn,~\sep\ en mijn God mijn beschermende rots.

  Maar hun zal hij hun onrecht vergelden, hen door eigen boosheid te gronde doen gaan;~\sep\ de Heer onze God zal hen verderven.
\end{halfparskip}

\PSALMtitle{95}{Eert God de Schepper!}

\begin{halfparskip}
  \psalmsubtitle{a) Looft God!}

  Komt, laten wij jubelen voor de Heer,~\sep\ laten wij juichen voor de Rots van ons heil.

  \qanona{Onze Heer heeft ons in Zijn mededogen gered van dwaling, overtredingen en dood; laten wij Hem aanbidden en verheerlijken.}

  Treden wij voor Zijn aanschijn met lofzangen,~\sep\ juichen wij Hem met liederen toe!

  Want de Heer is een grote God,~\sep\ en een grote Koning boven alle goden.

  Hij houdt in Zijn hand de diepten der aarde,~\sep\ en de toppen der bergen behoren Hem toe.

  Van Hem is de zee; Hij heeft ze geschapen,~\sep\ en het vaste land, door Zijn handen gevormd.

  Komt, laten wij aanbidden en ons neerwerpen,~\sep\ de knieën buigen voor de Heer, die ons heeft gemaakt.

  Want Hij is onze God;~\sep\ wij het volk van Zijn weide en de schapen van Zijn hand.
\end{halfparskip}

\begin{halfparskip}
  \psalmsubtitle{b) Gehoorzaamt God!}

  Mocht gij toch heden Zijn stem vernemen; ``Wilt niet Uw harten verstokken als bij Meriba, als op de dag van Massa in de woestijn,~\sep\ waar Uw vaderen Mij tergden, Mij beproefden, hoewel zij Mijn werken zagen.

  Veertig jaar was dat geslacht Mij een walg en Ik zei:~\sep\ Het is een volk dat dwaalt in zijn hart en Mijn wegen kennen zij niet.

  Daarom heb Ik in Mijn gramschap gezworen:~\sep\ Neen, zij zullen Mijn rust niet binnengaan.''
\end{halfparskip}

\Slota{Wij moeten altijd een nieuwe lofprijs brengen, O Heer, een verheven belijdenis, een smekende aanbidding en een voortdurende dankzegging aan Uw glorieuze Drie-eenheid, Heer van alles...}

\marmita{38}

\PSALMtitle{96}{Looft de Heer!}

\begin{halfparskip}
  \psalmsubtitle{a) Dat alle volken U prijzen, O Heer!}

  Zingt een nieuw lied voor de Heer,~\sep\ zingt voor de Heer, alle landen!

  \qanona{Gezegend is Uw komst, Christus, Verlosser van allen, want U hebt ons waardig gemaakt U met de geestelijke wezens te verheerlijken.}

  \liturgicaloption{Kerstmis (Denha):} \emph{Voor deze redding, die voor ons op deze dag is gekomen, door Jezus, de Zoon van ons ras, die geboren (gedoopt) is op deze dag.}

  Zingt voor de Heer, zegent Zijn Naam,~\sep\ verkondigt Zijn heil van dag tot dag!

  Maakt onder de heidenen Zijn glorie bekend,~\sep\ onder alle volken Zijn wonderen.

  Want groot is de Heer en hoog te prijzen,~\sep\ meer te duchten dan alle goden.

  Want de goden der heidenen zijn allen verzinsels,~\sep\ maar de Heer heeft de hemel geschapen.

  Majesteit en pracht gaan vóór Hem uit;~\sep\ macht en luister zijn in Zijn heilige woning.
\end{halfparskip}

\begin{halfparskip}
  \psalmsubtitle{b) Alle volken, O Heer, zien uit naar Uw komst!}

  Kent toe aan de Heer, geslachten der volken, kent toe aan de Heer glorie en macht;~\sep\ kent toe aan de Heer de roem van Zijn Naam!

  Treedt met Uw offer Zijn voorhoven binnen;~\sep\ aanbidt de Heer in heilige feesttooi!

  Beef voor Zijn aanschijn, geheel de aarde;~\sep\ roept tot de volken: De Heer is Koning!

  Hij maakte de aarde onwankelbaar vast,~\sep\ heerst over de volken met billijkheid.


  Dat de hemelen juichen en de aarde jubele, laat bruisen de zee met wat ze bevat,~\sep\ laat jubelen het veld met wat er op groeit.

  Dan zullen juichen alle bomen van het woud voor het aanschijn van de Heer, want Hij komt,~\sep\ want Hij komt de aarde regeren.

  Hij zal met rechtvaardigheid de wereld regeren,~\sep\ en de volkeren volgens Zijn trouw.
\end{halfparskip}

\PSALMtitle{97}{Gods rechtvaardig bestuur}

\begin{halfparskip}
  \psalmsubtitle{a) De Heer komt ten oordeel}

  De Heer is Koning, dat de aarde jubele,~\sep\ dat de vele eilanden juichen!

  \qanona{O Kerk, zing glorie tot de Heer, die u heeft vernieuwd en uw zwakheid heeft verheven door Zijn hemelvaart en allen heeft verblijd.}

  \liturgicallbracket\liturgicaloption{Kerstmis (Denha):} \emph{Jubel, volk, dat gered zijt; geef glorie en zwijg niet bij de geboorte (doopsel) van de Verlosser, die de hoogten en de diepten [hemel en aarde] heeft verblijd.}\liturgicalrbracket

  Wolken en duisternis omgeven Hem,~\sep\ gerechtigheid en recht zijn de steun van Zijn troon.

  Vuur gaat vóór Hem uit,~\sep\ en verbrandt Zijn vijanden om Hem heen.

  Zijn bliksems verlichten het aardrijk;~\sep\ de aarde ziet het en beeft.

  Bergen versmelten als was voor de Heer,~\sep\ voor de Beheerser van heel de aarde.

  De hemelen verkondigen Zijn gerechtigheid,~\sep\ en alle volken aanschouwen Zijn glorie.
\end{halfparskip}

\begin{halfparskip}
  \psalmsubtitle{b) Uw komst, Heer, is de vreugde van de vromen}

  Beschaamd staan allen die beelden vereren, die roemen op valse goden:~\sep\ voor Hem werpen alle goden zich neer.

  Sion hoort het vol vreugde, en de steden van Juda juichen,~\sep\ om Uw oordelen, Heer.

  Want Gij, Heer, zijt verheven boven heel de aarde,~\sep\ hoogverheven onder alle goden.

  De Heer heeft lief die het kwade haten, Hij behoedt het leven van Zijn heiligen,~\sep\ en redt hen uit de hand van de bozen.

  Een licht rijst op voor de rechtvaardige,~\sep\ en blijdschap voor de oprechten van hart.

  Verheugt u, rechtschapenen, in de Heer,~\sep\ en verheerlijkt Zijn heilige Naam!
\end{halfparskip}

\PSALMtitle{98}{Zegelied}

\begin{halfparskip}
  \psalmsubtitle{a) Dank, Heer, voor de redding van Uw volk}

  Zingt een nieuw lied voor de Heer,~\sep\ want wonderen heeft Hij gewrocht;

  \qanona{Gezegend is Hij die Zich neerboog en gedoopt werd door Johannes in de Jordaan en door Zijn doopsel aan allen vergeving schonk.}

  \liturgicallbracket\liturgicaloption{Kerstmis (Denha):} \emph{Geeft glorie, zonen van de Kerk, zonen van de getrouwe Kerk, aan Hem die uit de Maagd geboren is (gedoopt werd door de zoon van de onvruchtbare vrouw) voor de redding van de wereld.}\liturgicalrbracket


  Zege bracht Hem Zijn rechterhand,~\sep\ Zijn heilige arm.

  De Heer heeft Zijn heil doen kennen,~\sep\ voor het oog van de volken Zijn gerechtigheid getoond.

  Zijn liefde en trouw was Hij indachtig,~\sep\ ten gunste van Israëls huis.

  Alle grenzen der aarde hebben aanschouwd,~\sep\ het heil van onze God.
\end{halfparskip}

\begin{halfparskip}
  \psalmsubtitle{b) Dat heel de aarde U prijze, O Heer!}

  Juicht voor de Heer, alle landen,~\sep\ weest blijde, verheugt u en tokkelt de snaren.

  Zingt voor de Heer bij citerspel,~\sep\ en de klank van de harp,

  Met trompetten en bazuingeschal!~\sep\ Jubelt voor het aanschijn van de Koning en Heer!

  De zee, met wat ze bevat, verheffe haar stem,~\sep\ de aarde en die haar bewonen;

  Dat de stromen in de handen klappen,~\sep\ en tegelijk de bergen juichen.

  Voor het aanschijn van de Heer, want Hij komt,~\sep\ want Hij komt de aarde regeren.

  Hij zal met rechtvaardigheid de wereld regeren,~\sep\ en de volkeren met billijkheid.

  \liturgicaloption{Kerstmis:} \emph{Wat is dit dat de landen hun goden hebben verlaten en zweren in de naam van een Man die niet is zoals u zegt? Wat is dit dat de landen hun beelden hebben verworpen, en de Vader, de Zoon en de H. Geest hebben geloofd en beleden?}
\end{halfparskip}

\marmita{39}

\PSALMtitle{99}{Gods koninklijke oppermacht}

\begin{halfparskip}
  \psalmsubtitle{a) Gij zijt heilig, O Heer!}

  De Heer is Koning: de volken sidderen;~\sep\ Hij troont op de cherubs: de aarde beeft.

  \qanona{Er is geen macht zoals die van U, O Heiland, die de heilige apostelen heeft vervuld met de Heilige Geest.}

  \liturgicallbracket\liturgicaloption{Kerstmis (Denha):} \emph{Glorie, alleluja en aanbidding bij de geboorte (doop) van Christus.}\liturgicalrbracket

  Groot is de Heer op de Sion,~\sep\ en boven alle volken verheven.

  Dat zij prijzen Uw grote en ontzagwekkende Naam:~\sep\ Hij toch is heilig.

  De Machtige heerst, die rechtvaardigheid bemint: wat recht is, hebt Gij gegrondvest,~\sep\ Gij handhaaft in Jacob gerechtigheid en recht.

  Prijst de Heer, onze God, en werpt u neer voor Zijn voetbank:~\sep\ Hij toch is heilig.
\end{halfparskip}

\begin{halfparskip}
  \psalmsubtitle{b) Gij zijt Uw dienaars genadig}

  Mozes en Aäron zijn onder Zijn priesters, en Samuël onder hen die Zijn Naam aanriepen;~\sep\ zij riepen tot de Heer, en Hij schonk hun verhoring.

  In een wolkkolom sprak Hij tot hen:~\sep\ zij luisterden naar Zijn geboden, en naar de wet, die Hij hun gaf.

  Heer, onze God, Gij hebt hen verhoord;~\sep\ Gij waart hun genadig, O God, maar hun fouten hebt Gij gestraft.

  Prijst de Heer, onze God, en werpt u neer voor Zijn heilige berg,~\sep\ want heilig is de Heer, onze God.
\end{halfparskip}

\PSALMtitle{100}{Looft de Heer!}

\begin{halfparskip}
  Juicht voor de Heer, alle landen,~\sep\ dient de Heer met vreugde;

  \qanona{De rechtvaardigen kleden zich met glorie en vliegen boven in de wolken om onze Heer te ontmoeten wanneer Hij komt.}

  \liturgicallbracket\liturgicaloption{Kerstmis (Denha):} \emph{Laat ons belijden en verheerlijken het Kind dat geboren (gedoopt) was voor ons en de Zoon, die ons is gegeven.}\liturgicalrbracket

  Treedt voor Zijn aanschijn~\sep\ met gejubel!

  Weet het wel: de Heer is God; Hij heeft ons gemaakt, Hem behoren we toe,~\sep\ Zijn volk zijn wij en de schapen van Zijn weide.

  Treedt Zijn poorten met lofzang binnen, Zijn voorhoven met jubelzang;~\sep\ brengt Hem hulde en zegent Zijn Naam.

  Want goed is de Heer: eeuwig duurt Zijn barmhartigheid,~\sep\ en Zijn trouw van geslacht tot geslacht.
\end{halfparskip}

\PSALMtitle{101}{Ideale voornemens}

\begin{halfparskip}
  Van liefde en recht wil ik zingen,~\sep\ voor U de citer bespelen, O Heer.

  \qanona{De Heer, op wie ik vertrouwd heb, redt mij van de vijanden die mij haatten.}

  \liturgicallbracket\liturgicaloption{Kerstmis:} \emph{Laat ons loven en verheerlijken het heldere Licht van rechtvaardigheid, dat straalde vanuit het huis van David.}\liturgicalrbracket

  Op de weg der onschuld zal ik wandelen;~\sep\ wanneer zult Gij tot mij komen?

  Rein van hart wil ik leven,~\sep\ binnen mijn huis.

  Mijn ogen zal ik niet vestigen,~\sep\ op ongerechtigheid;

  Wie onrecht pleegt, is mij een gruwel:~\sep\ hij zal met mij geen omgang hebben.

  Een bedorven hart blijft verre van mij;~\sep\ van kwaad wil ik niets weten.

  Wie heimelijk zijn naaste belastert,~\sep\ die zal ik te gronde richten.

  De trotse blik en het hovaardig hart,~\sep\ zal ik niet dulden.

  Mijn ogen zien uit naar de getrouwen in het land,~\sep\ opdat ze bij mij wonen.

  Wie wandelt langs de goede weg,~\sep\ die zal mijn dienaar zijn.

  In mijn huis zal niet verblijven,~\sep\ die aan bedrog zich schuldig maakt.

  Wie leugens spreekt,~\sep\ houdt het niet uit onder mijn ogen.

  Dag aan dag zal ik verdelgen,~\sep\ alle boosdoeners in het land,

  Verbannen uit de stad van de Heer,~\sep\ allen die kwaad bedrijven.

  \liturgicaloption{Gedachtenissen:} \emph{Komt, mijn geliefden, laat ons lof zingen op de dag van de herdenking van de beroemde mar~\NN\ (of feest \NN ); en zingen en roepen we het alleluja der engelen, die neergedaald zijn in koren en hem (\translationoptionNl{haar / het}) met processies gevierd hebben.}
\end{halfparskip}

\begin{halfparskip}
  \dd~Laat ons bidden; vrede zij met ons.
\end{halfparskip}

\begin{halfparskip}
  \liturgicalhint{\textbf{Slota voor de mawtba.}} Wij smeken U, Schat van hulp en Bron van alle weldaden, overstromende Zee van mededogen en barmhartigheid, grote afgrond van vergeving en medelijden, keer U naar ons, Heer, en heb medelijden en genade met ons, zoals U gewend bent, te allen tijde. Heer van alles...
\end{halfparskip}

% % % % % % % % % % % % % % % % % % % % % % % % % % % % % % % % % % % % % % % %

\markedsection{Eerste Mawtba \markedsectionhint{(Onyata d-mawtba + Qala d-udrane\footnote{Als de Hudra meerdere qale (``slawata'') geeft en er onvoldoende tijd is, zeg de eerste 2 strofen van beiden + de laatste strofen van Maria, Kruis, Heiligen, Patroonheilige en de laatste der overledenen (volledig), en dan Eer aan...}. Er is geen qalta op feesten en gedachtenissen.)}}

\begin{halfparskip}
  \dd~Laat ons bidden, vrede zij met ons.

  \cc~Voor Uw wonderbaarlijke en onuitsprekelijke heilsbestel, Heer, dat in barmhartigheid en mededogen werd vervolmaakt, voltooid en vervuld voor de vernieuwing en redding van onze zwakke natuur, in de Eersteling die van ons was, brengen wij lof, eer, belijdenis en aanbidding, te allen tijde, Heer van alles...
\end{halfparskip}

% % % % % % % % % % % % % % % % % % % % % % % % % % % % % % % % % % % % % % % %

\markedsection{Qanona \markedsectionhint{(Psalm: eigen tekst.)\footnote{Zeg de eerste 2 petgame, dan het eerste deel van de qanona (refrein); zeg de hele psalm; herhaal het eerste deel van de qanona; Eer aan...; zeg het tweede deel; herhaal de eerste 2 petgame; eindig met het derde deel van de qanona.}}}

% % % % % % % % % % % % % % % % % % % % % % % % % % % % % % % % % % % % % % % %

\markedsection{Tesbohta \markedsectionhint{(Eigen tekst.)}}

% % % % % % % % % % % % % % % % % % % % % % % % % % % % % % % % % % % % % % % %

\markedsection{Karozuta \markedsectionhint{(Eigen tekst, of:)}}

\begin{halfparskip}
  \dd~Laat ons allen ordelijk staan met vreugde en vrolijkheid; laat ons bidden en zeggen: Heer, ontferm U over ons.~--- \rr~Heer, ontferm U over ons. \liturgicalhint{(Wordt herhaald na elke aanroeping.)}

  Machtige Heer, eeuwig Wezen, die op de hoogste hoogten woont, wij bidden U.

  U die, in Uw grote liefde waarmee U ons liefhad, de vorming van ons ras naar het beeld van Uw glorie hebt geëerd, wij bidden U.

  U die aan de trouwe Abraham goede dingen beloofde aan hen die U liefhebben, en die door de openbaring van Christus aan Uw Kerk bekend werden gemaakt, wij bidden U.

  U die niet wilt dat onze natuur ten onder gaat, maar dat zij zich bekeert van de dwaling van de duisternis naar de kennis van de waarheid, wij bidden U.

  U die de enige Maker en Vormer der geschapen dingen bent en in het voortreffelijke licht verblijft, wij bidden U,

  Voor de gezondheid van onze heilige vaders, Paus \NN , hoofd van de hele Kerk van Christus, van Patriarch \NN ,

  van onze Catholicos N, van onze Metropoliet \NN , van onze Bisschop \NN , en van al hun helpers, wij bidden U,

  Barmhartige God, die met Uw liefde alles bestuurt, wij bidden U,

  Gij die in de hemel wordt geprezen en op aarde wordt aanbeden, wij bidden U,

  Geef ons de overwinning, Christus onze Heer, bij Uw komst, en geef vrede aan Uw Kerk, gered door Uw kostbaar bloed, en ontferm U over ons.

  \cc~Van U, die vol genade en meedogen bent, van de grote rijkdom van Uw liefdevolle goedheid en de overvloedige schat van Uw meedogen, vragen wij hulp, kracht, verlossing, behoud en genezing voor de pijnen van ons lichamen en zielen. Schenk ons dit in Uw genade en barmhartigheid, zoals U gewend bent, te allen tijde, Heer van alles...

  \cc~Gezegend, aanbiddelijk, hoog, verheven en onbegrijpelijk is de eeuwige genade van Uw glorieuze Drie-eenheid, die medelijden heeft met de zondaars, O onze goede Hoop en Toevlucht vol barmhartigheid, die overtredingen en zonden vergeeft, Heer van alles...
\end{halfparskip}

% % % % % % % % % % % % % % % % % % % % % % % % % % % % % % % % % % % % % % % %

\markedsection{Madrasa \markedsectionhint{(Eigen tekst, zie Hudra of Gaza, geen madrasa van Pasen tot Elia.)}}

\begin{halfparskip}
  \dd~Laat ons bidden, vrede zij met ons.

  \cc~Mogen de klanken van onze alleluias en de melodieën van onze liederen, U behagen, O onze Heer en onze God; en aanvaard van ons in Uw liefderijke goedheid de redelijke vruchten van onze lippen die wij met lof aan Uw glorieuze Drie-eenheid aanbieden, dag en nacht, Heer van alles...
\end{halfparskip}

% % % % % % % % % % % % % % % % % % % % % % % % % % % % % % % % % % % % % % % %

\newpage
\title{Qale d'sahra}
\inlinemaketitle

\markedsection{Psalmen}

\liturgicalhint{Drie eigen Psalmen, gegeven in de Hudra of Gaza.\footnote{Zeg het eerste vers met alleluia; reciteer de psalm; dan Eer... en Vanaf het begin...; alleluja; eerste vers; alleluja. Soms wordt een Hpakta eraan toegevoegd.}}

\begin{halfparskip}
  \dd~Laat ons bidden, vrede zij met ons.

  \cc~Moge de Naam van Uw Godheid en van Uw Majesteit, Heer, worden aanbeden, verheerlijkt, geëerd, verheven, beleden en gezegend in hemel en op aarde door de redelijke monden die U hebt geschapen, door de verheerlijkende tongen die U hebt gevormd, en door alle gezelschappen van boven en beneden, Heer van alles...
\end{halfparskip}

% % % % % % % % % % % % % % % % % % % % % % % % % % % % % % % % % % % % % % % %

\markedsection{Onita d-lelya \markedsectionhint{(Eigen, zie Hudra of Gaza.)}}

\begin{halfparskip}
  \dd~Laat ons bidden, vrede zij met ons.

  \liturgicaloption{Christus:} Voor Uw wonderbaarlijk en onuitsprekelijk heilsbestel, Heer, dat in barmhartigheid en meedogen werd vervolmaakt, voltooid en vervuld voor de vernieuwing en redding van onze zwakke natuur, in de Eersteling die van ons was, brengen wij lof, eer, belijdenis en aanbidding, te allen tijde, Heer van alles...

  \liturgicaloption{Kruis:} Laat Uw vrede wonen in alle gebieden, verhef Uw Kerk door Uw Kruis, bewaar haar kinderen in Uw goedheid, zodat zij in haar U altijd lof, eer, belijdenis en aanbidding kunnen brengen, Heer van alles...

  \liturgicaloption{Maria:} Moge het gebed, Heer, van de heilige maagd, de petitie van de gezegende moeder, het verzoeken en smeken van haar die vol genade is, de gezegende mart Maria, bij ons zijn, te allen tijde en in alle seizoenen, Heer van alles...

  \liturgicaloption{Apostelen:} Heer, moge het gebed der heilige apostelen, de bede der ware predikers, het bidden en smeken der beroemde atleten, de verkondigers van gerechtigheid en zaaiers van vrede in de schepping, altijd bij ons zijn, in alle seizoenen en tijden, Heer van alles...

  \liturgicaloption{Patroonheilige:} Moge het gebed, verzoek, smeken en vragen van onze beroemde (\translationoptionNl{zegevierende}) en heilige vader, de glorieuze mar \NN , en van al zijn metgezellen, voortdurend bij ons zijn, te allen tijde en in alle seizoenen, Heer van alles...

  \liturgicaloption{Martelaren en belijders:} Moge het gebed, Heer, van Uw martelaren, de petitie van Uw belijders, het verzoeken en smeken van de atleten die Uw wil vervulden, voor ons bidden tot Uw Godheid, dat U ons Uw vrede en voorspoed mag schenken alle dagen van de wereld, Heer van alles...

  \liturgicaloption{Kerkleraren:} Moge het gebed, Heer, der heilige priesters, het verzoek der illustere leraren, het smeken en bidden der atleten, de vervullers van Uw wil, voortdurend bij ons zijn, altijd en in eeuwigheid, Heer van alles...

  \liturgicaloption{Johannes de Doper:} Moge het gebed van de beproefde en geteste Doper, de petitie van de goede heraut, het verzoeken en smeken van de ware prediker, de glorieuze, heilige en beroemde martelaar mar Johannes, voortdurend bij ons zijn, in alle tijden en seizoenen, Heer van alles...
\end{halfparskip}

% % % % % % % % % % % % % % % % % % % % % % % % % % % % % % % % % % % % % % % %

\markedsection{Qanona \markedsectionhint{(Eigen tekst, vaak \Pss{148; 150; 116}.)} + Hpakta \markedsectionhint{(Vervolg, zie Hudra of Gaza.)}}

\vspace{\parskip}

% % % % % % % % % % % % % % % % % % % % % % % % % % % % % % % % % % % % % % % %

\markedsection{I. Feesten: Tesbohta \markedsectionhint{(van Mar Narsai)~--- (Gedachtenissen: zie volgende bladzijden.)}}

\begin{halfparskip}
  Lof aan de Goede die ons ras heeft bevrijd van de slavernij van de boze en van de dood.

  En die vrede heeft gesloten tussen ons en de koren van boven, die boos waren vanwege onze ongerechtigheid.

  Gezegend is de Medelevende, die, ook al zochten wij Hem niet, naar voren kwam om ons te zoeken en Zich in ons leven verheugde.

  Hij schilderde een gelijkenis van onze afdwaling en onze terugkeer in het verloren schaap.

  Hij noemde onze natuur ``erfgenaam'' en ``zoon'', die afdwaalde en terugkeerde, die stierf en weer tot leven kwam.

  Hij heeft de geestelijke koren blij gemaakt door ons berouw en onze verrijzenis.

  Onuitsprekelijk is de grote liefde die de Vriend van ons ras jegens ons heeft getoond,

  Die van ons ras een Middelaar nam, en de wereld verzoende met Zijn Majesteit.

  Ver hoog boven ons en alle schepselen staat dit nieuwe ding dat Hij voor onze mensheid heeft volbracht:

  dat Hij van ons lichaam een heilige tempel heeft gemaakt, zodat Hij daarin de aanbidding van allen zou kunnen vervolmaken.

  Kom, aardse en hemelse wezens, verwondert en staat versteld van de grootsheid van de stap (\translationoptionNl{waardigheid}), want ons ras heeft de grote hoogten van de onbereikbare Godheid bereikt.

  Mogen hemel en aarde, en alles wat daarin is, samen met ons Hem belijden die ons ras heeft verheven.

  Hij heeft ons beeld vernieuwd en onze ongerechtigheid uitgewist; Hij riep ons bij Zijn Naam en maakte alle dingen aan ons onderworpen.

  Hij is waardig om door alle monden geprezen te worden, Hij heeft ons boven allen en alles verheven.

  Laten we allen Hem loven, voor altijd en eeuwig, amen en amen.
\end{halfparskip}

% % % % % % % % % % % % % % % % % % % % % % % % % % % % % % % % % % % % % % % %

\markedsection{Karozuta \markedsectionhint{(Eigen of:)}}

\begin{halfparskip}
  \dd~Laat ons allen ordelijk staan met vreugde en vrolijkheid; laat ons bidden en zeggen: Heer, ontferm U over ons.~--- \rr~Heer, ontferm U over ons. \liturgicalhint{(Wordt herhaald na elke aanroeping.)}

  U die ons geleerd hebt te bidden en traagheid te vermijden, wij bidden U.

  U die de nacht hebt doorgebracht in gebed tot God voor de redding van ons ras, wij bidden U.

  U die ons door onze aardse vaders een bewijs van Uw barmhartigheid heeft gegeven, wij bidden U.

  U die ons van de machtige dood heeft gered, en op wie wij vertrouwen om ons te redden, wij bidden U.

  U die ons gered heeft van de macht van de duisternis en ons naar het koninkrijk van Uw geliefde Zoon hebt gebracht, wij bidden U.

  U die zei: ``Vraag en u zal gegeven worden, zoek en u zult vinden, klop en de schat van barmhartigheid zal voor u opengaan'', wij bidden U.

  Voor de gezondheid van onze heilige vaders, Paus \NN , hoofd van de hele Kerk van Christus, van Patriarch \NN , van onze Catholicos \NN , van onze Metropoliet \NN , van onze Bisschop \NN , en van al hun helpers, wij bidden U, Barmhartige God, die met Uw liefde alles bestuurt, wij bidden U,

  Gij die in de hemel wordt geprezen en op aarde wordt aanbeden, wij bidden U,

  Geef ons de overwinning, Christus onze Heer, door Uw komst, en geef vrede aan Uw Kerk, gered door Uw kostbare bloed, en ontferm U over ons.
\end{halfparskip}

% % % % % % % % % % % % % % % % % % % % % % % % % % % % % % % % % % % % % % % %

\markedsection{II. Gedachtenissen: Tesbohta}

\begin{halfparskip}
  Eer aan de Goede die in Zijn liefde Zijn glorie openbaarde aan de mensen.

  Hij schiep een stomme natuur uit stof; en versierde het met een ziel begiftigd met schatten.

  Hij plaatste Zijn lof in een nederig (\translationliteralNl{verachtelijk}) lichaam, zodat de hele schepping Zijn glorie zou zingen.

  Komt, jullie, begiftigd met spraak, zingt glorie voor Hem, voordat we de slaap des doods slapen.

  Laten we in de lange nacht de dood gedenken, die onze mond sluit en ons tot stilte brengt.

  De rechtvaardigen die Hem `s~nachts verheerlijkten, zelfs als ze dood zijn, leven.

  En de goddelozen die Zijn grote glorie hebben ontkend, zelfs terwijl ze nog leefden, zijn dood.

  Laten we ons lichaam wakker maken door gebeden; en door alleluias van verborgen macht.

  Dat wij metgezellen mogen worden van de wijze maagden die onze Heer prees.

  En dat wij in de nacht, wanneer Hij de werelden doet beven, waakzaam mogen zijn en de Zoon mogen zien.

  Dat we niet verzinken in (kwade) verlangens, maar dat we Zijn glorie mogen zien op de dag waarop Hij verschijnt;

  en dat wij voor Hem waakzame dienaren mogen zijn in het uur waarop Hij de zonen van Zijn bruidskamer leidt,

  en de goddelozen blijven gemarteld; en plotseling is de deur van genade gesloten.

  Laten we tijdens ons leven ons een beetje inspannen, want na de dood is de dag van de vergelding.

  Het lichaam dat zich vermoeit met gebeden vliegt op de dag van de opstanding door de lucht,

  en ziet onze Heer zonder schaamte; en betreedt met Hem het huis van het koninkrijk.

  De engelen en rechtvaardigen hebben Hem lief, zij die waakzaam zijn geweest en hebben gebeden.

  Gezegend is Hij die van ons vaten van Zijn glorie heeft gemaakt en Zijn lof heeft geplaatst in een stoffelijke mond.

  Eer zij aan Zijn genade die de aardse en geestelijke wezens verenigde, zodat zij elke nacht en te allen tijde ``Heilig'' mogen zingen voor Zijn Naam.

  En laten wij allen Hem voor eeuwig en altijd loven, amen en amen.
\end{halfparskip}

% % % % % % % % % % % % % % % % % % % % % % % % % % % % % % % % % % % % % % % %

\markedsection{Karozuta \markedsectionhint{(Eigen of:)}}

\begin{halfparskip}
  \dd~Laat ons allen ordelijk staan met vreugde en vrolijkheid; laat ons bidden en zeggen: Heer, ontferm U over ons.~--- \rr~Heer, ontferm U over ons. \liturgicalhint{(Wordt herhaald na elke aanroeping.)}

  Machtige Heer, eeuwig Wezen, die op de hoogste hoogten woont, wij bidden U.

  U die, in Uw grote liefde waarmee U ons liefhad, de vorming van ons ras naar het beeld van Uw glorie hebt geëerd, wij bidden U.

  U die aan de trouwe Abraham goede dingen beloofde aan hen die U liefhebben, en die door de openbaring van Christus aan Uw Kerk bekend werden gemaakt, wij bidden U.

  U die niet wilt dat onze natuur ten onder gaat, maar dat zij zich bekeert van de dwaling van de duisternis naar de kennis van de waarheid, wij bidden U.

  U die alleen de Maker en Vormer der geschapen dingen bent en in het voortreffelijke licht verblijft, wij bidden U,

  Voor de gezondheid van onze heilige vaders, Paus \NN , hoofd van de hele Kerk van Christus, van Patriarch \NN , van onze Catholicos \NN , van onze Metropoliet \NN , van onze Bisschop \NN , en van al hun helpers, wij bidden U,

  Barmhartige God, die met Uw liefde alles bestuurt, wij bidden U,

  Gij die in de hemel wordt geprezen en op aarde wordt aanbeden, wij bidden U,

  Geef ons de overwinning, Christus onze Heer, bij Uw komst, en geef vrede aan Uw Kerk, gered door Uw kostbaar bloed, en ontferm U over ons.
\end{halfparskip}

\begin{halfparskip}
  \liturgicalhint{\textbf{Slota} (eigen tekst ofwel de volgende:)} \dd~Laat ons bidden; vrede zij met ons.

  \cc~Maak ons waardig, onze Heer en onze God, om U te dienen volgens de wil van Uw Godheid en Uw glorieuze Majesteit, puur en nobel, waakzaam en ernstig, rechtvaardig en oprecht, heilig en onberispelijk. En moge onze dienst, mijn Heer, U behagen, ons gebed en onze wake U overtuigen, ons verzoek U gunstig stemmen, ons smeken U eren, onze smeekbede U verzoenen; en moge de barmhartigheid en het mededogen van Uw Godheid de overtredingen van Uw volk vergeven en de zonden kwijtschelden van alle schapen van Uw weide, die U voor Uzelf hebt uitgekozen in Uw genade en barmhartigheid, Gij goede Vriend der mensen, Heer van alles...
\end{halfparskip}

% % % % % % % % % % % % % % % % % % % % % % % % % % % % % % % % % % % % % % % %

\end{document}